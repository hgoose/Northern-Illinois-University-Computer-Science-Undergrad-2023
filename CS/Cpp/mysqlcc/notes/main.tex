\documentclass{report}

\input{~/dev/latex/template/preamble.tex}
\input{~/dev/latex/template/macros.tex}

\title{\Huge{}}
\author{\huge{Nathan Warner}}
\date{\huge{}}
\fancyhf{}
\rhead{}
\fancyhead[R]{\itshape Warner} % Left header: Section name
\fancyhead[L]{\itshape\leftmark}  % Right header: Page number
\cfoot{\thepage}
\renewcommand{\headrulewidth}{0pt} % Optional: Removes the header line
%\pagestyle{fancy}
%\fancyhf{}
%\lhead{Warner \thepage}
%\rhead{}
% \lhead{\leftmark}
%\cfoot{\thepage}
%\setborder
% \usepackage[default]{sourcecodepro}
% \usepackage[T1]{fontenc}

% Change the title
\hypersetup{
    pdftitle={MYSQL in c++}
}

\begin{document}
    % \maketitle
        \begin{titlepage}
       \begin{center}
           \vspace*{1cm}
    
           \textbf{MYSQL in c++}
    
           \vspace{0.5cm}
            
                
           \vspace{1.5cm}
    
           \textbf{Nathan Warner}
    
           \vfill
                
                
           \vspace{0.8cm}
         
           \includegraphics[width=0.4\textwidth]{~/niu/seal.png}
                
           Computer Science \\
           Northern Illinois University\\
           United States\\
           
                
       \end{center}
    \end{titlepage}
    \tableofcontents
    \pagebreak 
    \unsect{Setup and compilation}
    \bigbreak \noindent 
    We begin by including the required library
    \bigbreak \noindent 
    \begin{cppcode}
    #include <mysql/mysql.h>
    \end{cppcode}
    \bigbreak \noindent 
    We can then compile  with
    \begin{cppcode}
    g++ -o program -I/usr/include/mariadb -lmariadb program.cc
    \end{cppcode}


    \pagebreak 
    \unsect{Initialization and deinitialization}
    \bigbreak \noindent 
    We start with the initialization
    \bigbreak \noindent 
    \begin{cppcode}
    mysql_library_init(0, NULL, NULL);
    MYSQL* connection = mysql_init(NULL);

    const char* host, *user, *password, *db;
    host = "...";
    user = "...";
    password = "...";
    db = "...";

    connection = mysql_real_connect(connection, host, user, password, db, 0, NULL, 0);


    // At the end of the program
    mysql_close(connection);
    mysql_library_end();
    \end{cppcode}

    \pagebreak 
    \unsect{Running a query, getting number of affected rows, and getting the number of columns}
    \bigbreak \noindent 
    \begin{cppcode}
    const char* query = "SELECT * FROM S";
    int err = mysql_query(connection, query);

    // Returns non-zero on failure, zero on success
    if (err) {
        const char* e = mysql_error(connection);
        unsigned int en = mysql_errno(connection);
        cout << "Error numeric code: " << en << "\t Message: \t" << e << endl;
        exit(1);
    }
    my_ulonglong affected = mysql_affected_rows(connection);
    unsigned int fc = mysql_field_count(connection);

    cout << "Affected: " << affected << endl;
    cout << "Field count: " << fc << endl << endl;

    \end{cppcode}
    \bigbreak \noindent 

    \pagebreak 
    \unsect{Errors}
    \bigbreak \noindent 
    Many of the functions return error codes, we also have two functions regarding errors.
    \bigbreak \noindent 
    \begin{cppcode}
    unsigned int mysql_errno(MYSQL *mysql);
    const char *mysql_error(MYSQL *mysql);
    \end{cppcode}


    \pagebreak 
    \unsect{Fetching and processing the result set}
    \bigbreak \noindent 
    \begin{cppcode}
        MYSQL_RES* res = mysql_store_result(connection);
        ull returned = mysql_num_rows(res);

        cout << "Returned: " << returned << endl;

        MYSQL_ROW row;
        unsigned int num_fields = mysql_num_fields(res);
        while ( (row = mysql_fetch_row(res)) ) {
            for (unsigned int i=0; i< num_fields; ++i) {
                cout << (row[i] ? row[i] : "NULL") << '\t';
            }
            cout << endl;
        }
        mysql_free_result(res);
    \end{cppcode}




\end{document}
