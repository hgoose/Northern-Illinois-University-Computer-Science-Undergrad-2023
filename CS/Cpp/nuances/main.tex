\documentclass{report}

\input{~/dev/latex/template/preamble.tex}
\input{~/dev/latex/template/macros.tex}

\title{\Huge{}}
\author{\huge{Nathan Warner}}
\date{\huge{}}
\fancyhf{}
\rhead{}
\fancyhead[R]{\itshape Warner} % Left header: Section name
\fancyhead[L]{\itshape\leftmark}  % Right header: Page number
\cfoot{\thepage}
\renewcommand{\headrulewidth}{0pt} % Optional: Removes the header line
%\pagestyle{fancy}
%\fancyhf{}
%\lhead{Warner \thepage}
%\rhead{}
% \lhead{\leftmark}
%\cfoot{\thepage}
%\setborder
% \usepackage[default]{sourcecodepro}
% \usepackage[T1]{fontenc}

% Change the title
\hypersetup{
    pdftitle={Cpp Nuances}
}

\begin{document}
    % \maketitle
        \begin{titlepage}
       \begin{center}
           \vspace*{1cm}
    
           \textbf{Cpp Nuances}
    
           \vspace{0.5cm}
            
                
           \vspace{1.5cm}
    
           \textbf{Nathan Warner}
    
           \vfill
                
                
           \vspace{0.8cm}
         
           \includegraphics[width=0.4\textwidth]{~/niu/seal.png}
                
           Computer Science \\
           Northern Illinois University\\
           United States\\
           
                
       \end{center}
    \end{titlepage}
    \tableofcontents
    \pagebreak 
    \unsect{Converting char to std::string}
    \bigbreak \noindent 
    Suppose we have a char variable, and we need to "convert" it to a string. To do this we use the string class constructor, which has two parameters, the size of the string to create, and the character to use as the fill.
    \bigbreak \noindent 
    \subsection{Constructor signature}
    \bigbreak \noindent 
    \begin{cppcode}
    string(size_t n, char x) 
    \end{cppcode}
    \bigbreak \noindent 
    \subsection{Example}
    \bigbreak \noindent 
    \begin{cppcode}
    char c = 'a';
    string s(1,c);
    \end{cppcode}

    \pagebreak 
    \unsect{std::string::npos}
    \bigbreak \noindent 
    \begin{concept}
        In C++, std::string::npos is a static member constant value with the greatest possible value for an element of type size\_t. This value, when used as the length in string operations, typically represents "until the end of the string." It is often used in string manipulation functions to specify that the operation should proceed from the starting position to the end of the string, or until no more characters are found.
    \end{concept}
    \bigbreak \noindent 
    \subsection{Example}
    \bigbreak \noindent 
    \begin{cppcode}
    string infix = buffer.substr(index + 2, string::npos);
    \end{cppcode}
    \bigbreak \noindent 
    std::string::npos is defined as the maximum value representable by the type size\_t. This value is typically used to signify an error condition or a not-found condition when working with strings and other sequence types. However, when used as a length argument in methods like std::string::substr, it effectively becomes a directive to process characters until the end of the string. This is because any attempt to access beyond the end of the string would exceed the string's length, and the methods are designed to stop processing at that point.
    

    
\end{document}
