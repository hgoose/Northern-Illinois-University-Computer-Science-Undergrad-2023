\documentclass{report}

\input{~/dev/latex/template/preamble.tex}
\input{~/dev/latex/template/macros.tex}

\title{\Huge{}}
\author{\huge{Nathan Warner}}
\date{\huge{}}
\fancyhf{}
\rhead{}
\fancyhead[R]{\itshape Warner} % Left header: Section name
\fancyhead[L]{\itshape\leftmark}  % Right header: Page number
\cfoot{\thepage}
\renewcommand{\headrulewidth}{0pt} % Optional: Removes the header line
%\pagestyle{fancy}
%\fancyhf{}
%\lhead{Warner \thepage}
%\rhead{}
% \lhead{\leftmark}
%\cfoot{\thepage}
%\setborder
% \usepackage[default]{sourcecodepro}
% \usepackage[T1]{fontenc}

% Change the title
\hypersetup{
    pdftitle={Systems Programming in C++}
}

\begin{document}
    % \maketitle
        \begin{titlepage}
       \begin{center}
           \vspace*{1cm}
    
           \textbf{Systems Programming in C++} \\
           CS330
    
           \vspace{0.5cm}
            
                
           \vspace{1.5cm}
    
           \textbf{Nathan Warner}
    
           \vfill
                
                
           \vspace{0.8cm}
         
           \includegraphics[width=0.4\textwidth]{~/niu/seal.png}
                
           Computer Science \\
           Northern Illinois University\\
           March 29, 2024 \\
           United States\\
           
                
       \end{center}
    \end{titlepage}
    \tableofcontents
    \pagebreak 
    \unsect{C Library Functions}
    \bigbreak \noindent 
    \subsection{cstdlib utils}}
    \bigbreak \noindent 
    \subsection{getenv}
    \bigbreak \noindent 
    \begin{concept}
        The \textbf{getenv} function is a standard library function that provides access to the environment variables of the process. Environment variables are dynamic-named values that affect the processes running on a computer. They can be used to configure system settings, pass configuration data to applications, and enable communication between different parts of an operating system or between different programs.
        \bigbreak \noindent 
        \textbf{Signature}
        \begin{cppcode}
            char* getenv(const char* name);
        \end{cppcode}
        \bigbreak \noindent 
        Where name is a a C-string (const char*) representing the name of the environment variable whose value is being requested.
        \bigbreak \noindent 
        \textbf{Return Value}
        \bigbreak \noindent 
        If the environment variable is found, getenv returns a pointer to a C-string containing the value of the variable.
        \bigbreak \noindent 
        If the environment variable is not found, it returns a null pointer (NULL in C, nullptr in C++).
        \bigbreak \noindent 
        \textbf{Example} 
        \bigbreak \noindent 
        \begin{cppcode}
            #include <cstdlib>
            #include <iostream>

            int main() {
                // Attempt to retrieve the PATH environment variable
                const char* path = getenv("PATH");
                if (path != nullptr) {
                    std::cout << "PATH: " << path << std::endl;
                } else {
                    std::cout << "PATH environment variable not found." << std::endl;
                }
                return 0;
            }
        \end{cppcode}
    \end{concept}

    \pagebreak 
    \subsubsection{exit}
    \bigbreak \noindent 
    \textbf{Signature}
    \bigbreak \noindent 
    \begin{cppcode}
    void exit(int status)
    \end{cppcode}
    \bigbreak \noindent 
    The exit function is quite simple, it terminates the calling process
    \bigbreak \noindent 
    Zero for a successful termination, anything else is unsuccessful termination.
    \bigbreak \noindent 

    \bigbreak \noindent 
    \subsubsection{system}
    \bigbreak \noindent 
    \textbf{Signature}
    \bigbreak \noindent 
    \begin{cppcode}
    int system(const char* command)
    \end{cppcode}
    \bigbreak \noindent 
    The system command allows us to run shell commands. It invokes the command processor to execute a comamnd. The function returns the exit status of the command.
    \bigbreak \noindent 
    \textbf{Note:} If command is a nullptr, the function only checks if a command processor is available. 
    \bigbreak \noindent 
    \textbf{Example}
    \bigbreak \noindent 
    \begin{cppcode}
        int main(int argc, const char* argv[]) {
            int rs;
            if (!system(NULL)) {
                exit(EXIT_FAILURE);
            }
            cout << system("ls -la");

            return EXIT_SUCCSESS;
        }
    \end{cppcode}

    \pagebreak 
    \unsect{Regular expressions library <regex.h>}
    \bigbreak \noindent 
    \subsection{}

    

    
\end{document}
