\documentclass{report}

\input{~/dev/latex/template/preamble.tex}
\input{~/dev/latex/template/macros.tex}

\title{\Huge{}}
\author{\huge{Nathan Warner}}
\date{\huge{}}
\fancyhf{}
\rhead{}
\fancyhead[R]{\itshape Warner} % Left header: Section name
\fancyhead[L]{\itshape\leftmark}  % Right header: Page number
\cfoot{\thepage}
\renewcommand{\headrulewidth}{0pt} % Optional: Removes the header line
%\pagestyle{fancy}
%\fancyhf{}
%\lhead{Warner \thepage}
%\rhead{}
% \lhead{\leftmark}
%\cfoot{\thepage}
%\setborder
% \usepackage[default]{sourcecodepro}
% \usepackage[T1]{fontenc}

% Change the title
\hypersetup{
    pdftitle={Final exam review}
}

\begin{document}
    % \maketitle
        \begin{titlepage}
       \begin{center}
           \vspace*{1cm}
    
           \textbf{Final exam review}
    
           \vspace{0.5cm}
            
                
           \vspace{1.5cm}
    
           \textbf{Nathan Warner}
    
           \vfill
                
                
           \vspace{0.8cm}
         
           \includegraphics[width=0.4\textwidth]{~/niu/seal.png}
                
           Computer Science \\
           Northern Illinois University\\
           United States\\
           
                
       \end{center}
    \end{titlepage}
    \tableofcontents
    \pagebreak 
    \unsect{Vocabulary}
    \bigbreak \noindent 
    You’re responsible for knowing the terms used in the lectures. There will be some questions specifically on vocabulary. Other questions will use the terms as part of a larger question – they will not be redefined during the test
    \begin{itemize}
        \item \textbf{Database:}  A database is a collection of stored operational data used by the application systems of some particular enterprise, better yet a collection of related data
        \item \textbf{Enterprise}: : a generic term for any reasonably large-scale commercial, scientific, technical, or other application. 
        \item \textbf{Operational data}: Data maintained about the operation of an enterprise. Note that this DOES NOT include input/output data
        \item \textbf{dbms}: A Database Management System (DBMS) is a collection of programs that enables users to create and maintain a database
        \item \textbf{Transactions}: Transaction management is a feature that provides correct, concurrent access to the database, possibly by many users at the same time, ability to simultaneously manage large numbers of transactions
        \item \textbf{Access control}: Access control is the ability to limit access to data by unauthorized users along with the capability to check the validity of the data. This is to protect against loss when database crashes and prevent unauthorized access to portions of the data
        \item \textbf{Resiliency}: : Resiliency is the ability to recover from system failures without losing data
        \item \textbf{External level:} a view or sub-schema, a portion of the logical database, may be in a higher level language
        \item \textbf{Logical Level:} abstraction of the real world as it pertains to the users of the database. DBMS provides a data definition language (DDL) to describe the logical schema in terms of a specific data model such as relational, hierarchical, network, inverted list, etc.
        \item \textbf{Physical Level:} The collection of files and indices, the collection of files and indices, this is the actual data
        \item \textbf{Instance}: An instance of the database is the actual contents of the data
    \item \textbf{Schema}: The schema of a database is the data about what the data represents}
    \item \textbf{Data Independence:} Data Independence is a property of an appropriately designed database system, it has to do with the mapping of logical level to physical level, and logical to externa}
    \item \textbf{Data vs information}: 
        \begin{itemize}
            \item \textbf{Data:} Data refers to raw, unprocessed facts, figures, and details. It represents basic elements that have not been interpreted or given any meaning.
            \item \textbf{Information:} Information is processed, organized, or structured data that is meaningful and useful. It is data that has been interpreted or analyzed to provide context, relevance, and purpose.
        \end{itemize}
    \item \textbf{Data models}: A means of describing the structure of data, we typically have A set of operations that manipulate the data (for data models that are implemented)
    \end{itemize}




\end{document}
