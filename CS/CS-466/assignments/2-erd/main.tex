\documentclass{report}

\input{~/dev/latex/template/preamble.tex}
\input{~/dev/latex/template/macros.tex}

\title{\Huge{}}
\author{\huge{Nathan Warner}}
\date{\huge{}}
\fancyhf{}
\rhead{}
\fancyhead[R]{\itshape Warner} % Left header: Section name
\fancyhead[L]{\itshape\leftmark}  % Right header: Page number
\cfoot{\thepage}
\renewcommand{\headrulewidth}{0pt} % Optional: Removes the header line
%\pagestyle{fancy}
%\fancyhf{}
%\lhead{Warner \thepage}
%\rhead{}
% \lhead{\leftmark}
%\cfoot{\thepage}
%\setborder
% \usepackage[default]{sourcecodepro}
% \usepackage[T1]{fontenc}

% Change the title
\hypersetup{
    pdftitle={}
}

\geometry{
  left=.3in,
  right=.3in,
  top=1in,
  bottom=1in
}


\begin{document}
    % \maketitle
    %     \begin{titlepage}
    %    \begin{center}
    %        \vspace*{1cm}
    % 
    %        \textbf{}
    % 
    %        \vspace{0.5cm}
    %         
    %             
    %        \vspace{1.5cm}
    % 
    %        \textbf{Nathan Warner}
    % 
    %        \vfill
    %             
    %             
    %        \vspace{0.8cm}
    %      
    %        \includegraphics[width=0.4\textwidth]{~/niu/seal.png}
    %             
    %        Computer Science \\
    %        Northern Illinois University\\
    %        United States\\
    %        
    %             
    %    \end{center}
    % \end{titlepage}
    % \tableofcontents
    \pagebreak \bigbreak \noindent
    Nate Warner \ \quad \quad \quad \quad \quad \quad \quad \quad \quad \quad \quad \quad  CS 466 \quad  \quad \quad \quad \quad \quad \quad \quad \quad \ \ \quad \quad Fall 2024
    \begin{center}
        \textbf{Pset 2 - Due: Friday, September 13}
    \end{center}
    \bigbreak \noindent 
    \begin{mdframed}
        1. You are being employed by a company that offers a fitness tracking service. They are working on a phone app that will allow the user to track what they
        eat, as well as when/how they work out. Another employee will be designing the user interface, but you are responsible for designing the database. Design
        an ER diagram to fulfill this goal, making sure to meet all of the requirements. All entities must have an appropriate identifier specified. If a surrogate key
        is used, explain why a natural key was not appropriate. In the interests of saving space, attributes that are not part of an identifier may be omitted from the
        diagram, but they should be included and explained in that portion of your submission. 
        \begin{itemize}
            \item Every user will have an account, and all of their meals and workouts will be linked to this account. An account is for a single user.
            \item The user will update their weight periodically. This data must be retained so that progress can be tracked over time.
            \item The serving size will be some number of some unit (grams, ounces, Tbsp, cups, lbs, etc.). There will be information stored for conversion between different unit types.
            \item There will need to be a database of foods/beverages. Each of these will have information on serving size, calories per serving, and grams per serving of each of the macronutrients (protein, fat, carbohydrates).
            \item It should be possible to store information on the quantities of micronutrients or chemicals (i.e. vitamin D, caffeine) that are present in a given food or beverage in significant amounts. Recommended daily values for any of these should be stored, when applicable.
            \item Each time the user eats, a record of who ate how many servings of what and when they ate it is stored.
            \item The app needs to allow a user to track their workouts. This includes the type of the workout, its intensity, and its duration.
            \item When a user tracks their workout, a record is created of who did what type of workout, when, and for how long.
        \end{itemize}
    \end{mdframed}
    \bigbreak \noindent 
    \nt{If you are not sure what needs to be included for foods/drinks/etc., a good place to look for example data would be the nutritional facts on food packaging.}
    \bigbreak \noindent 
\begin{figure}[ht]
    \centering
    \incfig{1}
    \label{fig:1}
\end{figure}



    
\end{document}
