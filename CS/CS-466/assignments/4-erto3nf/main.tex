\documentclass{report}

\input{~/dev/latex/template/preamble.tex}
\input{~/dev/latex/template/macros.tex}

\title{\Huge{}}
\author{\huge{Nathan Warner}}
\date{\huge{}}
\fancyhf{}
\rhead{}
\fancyhead[R]{\itshape Warner} % Left header: Section name
\fancyhead[L]{\itshape\leftmark}  % Right header: Page number
\cfoot{\thepage}
\renewcommand{\headrulewidth}{0pt} % Optional: Removes the header line
%\pagestyle{fancy}
%\fancyhf{}
%\lhead{Warner \thepage}
%\rhead{}
% \lhead{\leftmark}
%\cfoot{\thepage}
%\setborder
% \usepackage[default]{sourcecodepro}
% \usepackage[T1]{fontenc}

% Change the title
\hypersetup{
    pdftitle={}
}

\geometry{
  left=1in,
  right=1in,
  top=1in,
  bottom=1in
}

\begin{document}
    % \maketitle
    %     \begin{titlepage}
    %    \begin{center}
    %        \vspace*{1cm}
    % 
    %        \textbf{}
    % 
    %        \vspace{0.5cm}
    %         
    %             
    %        \vspace{1.5cm}
    % 
    %        \textbf{Nathan Warner}
    % 
    %        \vfill
    %             
    %             
    %        \vspace{0.8cm}
    %      
    %        \includegraphics[width=0.4\textwidth]{~/niu/seal.png}
    %             
    %        Computer Science \\
    %        Northern Illinois University\\
    %        United States\\
    %        
    %             
    %    \end{center}
    % \end{titlepage}
    % \tableofcontents
    \pagebreak \bigbreak \noindent
    Nate Warner \ \quad \quad \quad \quad \quad \quad \quad \quad \quad \quad \quad \quad  CS 466 \quad  \quad \quad \quad \quad \quad \quad \quad \quad \ \ \quad \quad Fall 2024
    \begin{center}
        \textbf{Pset 4 - Due: Wednesday, October 2}
    \end{center}
    \bigbreak \noindent 
    \begin{mdframed}
        1. Convert the following ER diagram, which models the operational data for a department store, to a set of relations that conform to Third Normal Form (3NF)
    \end{mdframed}
    \bigbreak \noindent 
    \fig{.75}{./figures/1.png}
    \fc{Entity relationship diagram}

    \pagebreak \bigbreak \noindent 
    1.) First, we convert the strong, non-subtype entities to relations. We have
    \begin{align*}
        &\text{Warehouse(\underline{ID}, Address, Manager)} \\
        &\text{City(\underline{State}, \underline{City}, Headquarters)} \\
        &\text{Product(\underline{ItemID}, Color, Description, Weight, Dimensions)} \\
        &\text{Order(\underline{OrderNum}, Date)} \\
        &\text{Customer(\underline{Name}, Address, Email)}
    .\end{align*}
    \bigbreak \noindent 
    Since there are no subtype or weak entities, we move to the relationships. There are 5 relationships, two binary one-to-many, and 3 binary many-to-many. We start with the simpler.
    \bigbreak \noindent 
    To handle the binary one-to-many relationships, we insert a foreign key in the many entity to the one entity. Thus,
    \begin{align*}
         &\text{Warehouse(\underline{ID}, Address, Manager, City$^{\dag}$)} \\
        &\text{City(\underline{State}, \underline{City}, Headquarters)} \\
        % &\text{Product(\underline{ItemID}, Color, Description, Weight, Dimensions)} \\
        &\text{Order(\underline{OrderNum}, Date, Name$^{\dag}$)} \\
        &\text{Customer(\underline{Name}, Address, Email)}
    .\end{align*}
    \bigbreak \noindent 
    Where city in the warehouse relation is a foreign key to the city relation, and name in the order relation is a foreign key to the customer relation. The implied functional dependencies are
    \begin{align*}
        &\text{ID} \to \text{City} \\
        &\text{OrderNum} \to \text{Name}
    .\end{align*}
    \bigbreak \noindent 
    Next, we need to handle the remaining relationships, which are the three binary many-to-many. For these, we create a new relation. The primary key of the new relations will be the concatenation of the primary keys of the entity relations, and the intersection data as non-prime attributes
    \begin{align*}
         &\text{Warehouse(\underline{ID}, Address, Manager, City$^{\dag}$)} \\
        &\text{City(\underline{State}, \underline{City}, Headquarters)} \\
        &\text{Product(\underline{ItemID}, Color, Description, Weight, Dimensions)} \\
        &\text{Order(\underline{OrderNum}, Date, Name$^{\dag}$)} \\
        &\text{Customer(\underline{Name}, Address, Email)}\\
        &\text{ProductHolding(\underline{ID}$^{\dag}$, \underline{ItemID}$^{\dag}$, quantity)} \\
        &\text{ProductStoredIn(\underline{City}$^{\dag}$, \underline{ItemID}$^{\dag}$, Quantity)} \\
        &\text{ProductOrderedIn(\underline{ItemID}$^{\dag}$, \underline{OrderNum}$^{\dag}$, Quantity)}
    .\end{align*}
    \bigbreak \noindent 
    The ProductHolding relation has a foreign key \textit{ID} to the Warehouse relation, and foreign key \textit{ItemID} to the Product relation.
    \bigbreak \noindent 
    The ProductStoredIn relation has a foreign key \textit{City} to the City relation, and a foreign key \textit{ItemID} to the Product relation.
    \bigbreak \noindent 
    The ProductOrderedIn relation has a foreign key \textit{ItemID} to the Product relation, and a foreign key \textit{OrderNum} too the Order relation. 
    \pagebreak \bigbreak \noindent 
    Thus, the final schema of the database is the following eight relations
    \bigbreak \noindent 
    \begin{align*}
         &\text{Warehouse(\underline{ID}, Address, Manager, City$^{\dag}$)} \\
         &\text{City(\underline{State}, \underline{City}, Headquarters)} \\
         &\text{Product(\underline{ItemID}, Color, Description, Weight, Dimensions)} \\
         &\text{Order(\underline{OrderNum}, Date, Name$^{\dag}$)} \\
         &\text{Customer(\underline{Name}, Address, Email)}\\
         &\text{ProductHolding(\underline{ID}$^{\dag}$, \underline{ItemID}$^{\dag}$, quantity)} \\
         &\text{ProductStoredIn(\underline{City}$^{\dag}$, \underline{ItemID}$^{\dag}$, Quantity)} \\
         &\text{ProductOrderedIn(\underline{ItemID}$^{\dag}$, \underline{OrderNum}$^{\dag}$, Quantity)}
     .\end{align*}



    
\end{document}
