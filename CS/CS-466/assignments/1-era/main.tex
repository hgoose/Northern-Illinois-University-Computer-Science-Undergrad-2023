\documentclass{report}

\input{~/dev/latex/template/preamble.tex}
\input{~/dev/latex/template/macros.tex}

\title{\Huge{}}
\author{\huge{Nathan Warner}}
\date{\huge{}}
\fancyhf{}
\rhead{}
\fancyhead[R]{\itshape Warner} % Left header: Section name
\fancyhead[L]{\itshape\leftmark}  % Right header: Page number
\cfoot{\thepage}
\renewcommand{\headrulewidth}{0pt} % Optional: Removes the header line
%\pagestyle{fancy}
%\fancyhf{}
%\lhead{Warner \thepage}
%\rhead{}
% \lhead{\leftmark}
%\cfoot{\thepage}
%\setborder
% \usepackage[default]{sourcecodepro}
% \usepackage[T1]{fontenc}

% Change the title
\hypersetup{
    pdftitle={}
}

\begin{document}
    % \maketitle
    %     \begin{titlepage}
    %    \begin{center}
    %        \vspace*{1cm}
    % 
    %        \textbf{}
    % 
    %        \vspace{0.5cm}
    %         
    %             
    %        \vspace{1.5cm}
    % 
    %        \textbf{Nathan Warner}
    % 
    %        \vfill
    %             
    %             
    %        \vspace{0.8cm}
    %      
    %        \includegraphics[width=0.4\textwidth]{~/niu/seal.png}
    %             
    %        Computer Science \\
    %        Northern Illinois University\\
    %        United States\\
    %        
    %             
    %    \end{center}
    % \end{titlepage}
    % \tableofcontents
    \pagebreak \bigbreak \noindent
    Nate Warner \ \quad \quad \quad \quad \quad \quad \quad \quad \quad \quad \quad \quad  CS 466 \quad  \quad \quad \quad \quad \quad \quad \quad \quad \ \ \quad \quad Fall 2024
    \begin{center}
        \textbf{Pset 1 - Due: Friday, September 6}
    \end{center}
    \bigbreak \noindent 
    \begin{mdframed}
        1. Dog is an entity. It has two attributes, “Name” and “Tag Number”. A given dog can be identified by its Tag Number. Person is an entity. It has
two attributes, “Name” and “ID”. ID can be used to identify a person. There is a relationship, “owns”, between Person and Dog. A person can
own many dogs, but a dog may only be owned by a single person.
    \end{mdframed}
    \bigbreak \noindent 
\begin{figure}[ht]
    \centering
    \incfig{1}
    \label{fig:1}
\end{figure}

\bigbreak \noindent 
\begin{mdframed}
    2. Dogs, Cats, and Fish are types of Pet. A Pet can exist without being one of those types, but no pet exists that is more than one of these types. All
pets have a name. Dogs alone have an attribute for their favorite type of bone
\end{mdframed}
\bigbreak \noindent 
\begin{figure}[ht]
    \centering
    \incfig{2}
    \label{fig:2}
\end{figure}

\pagebreak \bigbreak \noindent 
\begin{mdframed}
    3. You’re working on a simple social network and want to store information on how people are related. This will center around a (recursive) “is friends
with” relationship between two People, and a “likes” relationship between a Person and some Thing (an entity). The “is friends with” relationship
is between a single person on one side and many people on the other side. The “likes” relationship is many-to-many. Each Person will have a Name
and a UserID. The UserID can uniquely identify a given Person. Note: You may only use one Person entity.
\end{mdframed}
\begin{figure}[ht]
    \centering
    \incfig{3}
    \label{fig:3}
\end{figure}

\bigbreak \noindent 
\begin{mdframed}
    4. A is an entity. It has an attribute, “a\_id” as its identifier. B is a weak entity that depends on A. It has “b\_id” as its discriminator. Many instances of
B may belong to the same instance of A
\end{mdframed}
\bigbreak \noindent 
\begin{figure}[ht]
    \centering
    \incfig{5}
    \label{fig:5}
\end{figure}

\pagebreak 
\bigbreak \noindent 
\begin{mdframed}
    5. G, H, and J are entities. There is a relationship between all three, called K. For this K relationship, for each G and H there is one J. For each H and
J there can be many G’s. For each G and J, there is one H. Every time the relationship, K, comes about, a single value, “X”, will be stored.
\end{mdframed}
\begin{figure}[ht]
    \centering
    \incfig{6}
    \label{fig:6}
\end{figure}

    
\end{document}
