\documentclass{report}

\input{~/dev/latex/template/preamble.tex}
\input{~/dev/latex/template/macros.tex}

\title{\Huge{}}
\author{\huge{Nathan Warner}}
\date{\huge{}}
\fancyhf{}
\rhead{}
\fancyhead[R]{\itshape Warner} % Left header: Section name
\fancyhead[L]{\itshape\leftmark}  % Right header: Page number
\cfoot{\thepage}
\renewcommand{\headrulewidth}{0pt} % Optional: Removes the header line
%\pagestyle{fancy}
%\fancyhf{}
%\lhead{Warner \thepage}
%\rhead{}
% \lhead{\leftmark}
%\cfoot{\thepage}
%\setborder
% \usepackage[default]{sourcecodepro}
% \usepackage[T1]{fontenc}

% Change the title
\hypersetup{
    pdftitle={Neovim}
}

\begin{document}
    % \maketitle
        \begin{titlepage}
       \begin{center}
           \vspace*{1cm}
    
           \textbf{Neovim}
    
           \vspace{0.5cm}
            
                
           \vspace{1.5cm}
    
           \textbf{Nathan Warner}
    
           \vfill
                
                
           \vspace{0.8cm}
         
           \includegraphics[width=0.4\textwidth]{~/niu/seal.png}
                
           Computer Science \\
           Northern Illinois University\\
           United States\\
           
                
       \end{center}
    \end{titlepage}
    \tableofcontents
    \pagebreak 
    \unsect{Marks}
    \bigbreak \noindent 
    \begin{itemize}
        \item \textbf{Number of marks possible}: In Neovim (and Vim), you can set 52 normal marks in total:
            \begin{center}
                \begin{tabular}{llll}
                    Type	&Count	&Range	&Description \\
                    \hline \\[0.01cm]
                    Lowercase marks	&26	&'a to 'z	&Local to the current buffer (each buffer has its own set). \\[2ex]
                    Uppercase marks	&26	&'A to 'Z	&Global across all buffers (refer to specific files).\\[2ex]
                \end{tabular}
            \end{center}
        \item \textbf{Special marks}:
            \begin{center}
                \begin{tabular}{ll}
                    Mark	&Meaning \\
                    \hline \\[0.01cm]
                    '' 	&Last jump position\\[2ex]
                    '.	&Last change\\[2ex]
                    '[ / ']	&Start / end of last change or yank\\[2ex]
                    '< / '>	&Start / end of last visual selection\\[2ex]
                    '0 – '9	&Last file positions (older jump list entries)\\[2ex]
                    '""	&Last exit position in a file\\[2ex]
                    '<' and '>'`	&Visual selection start/end
                \end{tabular}
            \end{center}
    \end{itemize}
    

























    
\end{document}
