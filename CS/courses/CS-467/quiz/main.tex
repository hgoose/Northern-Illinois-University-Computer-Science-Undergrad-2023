\documentclass{report}

\input{~/dev/latex/template/preamble.tex}
\input{~/dev/latex/template/macros.tex}

\title{\Huge{}}
\author{\huge{Nathan Warner}}
\date{\huge{}}
\fancyhf{}
\rhead{}
\fancyhead[R]{\itshape Warner} % Left header: Section name
\fancyhead[L]{\itshape\leftmark}  % Right header: Page number
\cfoot{\thepage}
\renewcommand{\headrulewidth}{0pt} % Optional: Removes the header line
%\pagestyle{fancy}
%\fancyhf{}
%\lhead{Warner \thepage}
%\rhead{}
% \lhead{\leftmark}
%\cfoot{\thepage}
%\setborder
% \usepackage[default]{sourcecodepro}
% \usepackage[T1]{fontenc}

% Change the title
\hypersetup{
    pdftitle={467 Quiz Answers}
}

\begin{document}
    % \maketitle
        \begin{titlepage}
       \begin{center}
           \vspace*{1cm}
    
           \textbf{467 Quiz Answers}
    
           \vspace{0.5cm}
                
                
           \vspace{1.5cm}
    
           \textbf{Nathan Warner}
    
           \vfill
                
                
           \vspace{0.8cm}
         
           \includegraphics[width=0.4\textwidth]{~/niu/seal.png}
                
           Computer Science \\
           Northern Illinois University\\
           United States\\
           
                
       \end{center}
    \end{titlepage}
    \tableofcontents
    \pagebreak 
    \unsect{Quiz 1}
    \begin{itemize}
        \item \textbf{One of the ethical principles of the ACM/IEEE code of ethics is that software engineers shall be competitive, there is no need not be fair and supportive of colleagues.}: False
        \item \textbf{Software engineering is an engineering discipline that is concerned with all aspects of software production from the early stages of system specification through to maintaining the system after it has gone into use}: True
        \item \textbf{What are the fundamental types of software product?}: Generic and customized
        \item \textbf{computer science and software engineering are the same discipline}: False
        \item \textbf{Which of the following is not an issue of professional responsibility ?}: Salary
        \item \textbf{Which of the following is not a fundamental activity in software processes ?}: software estimation
        \item \textbf{Which of the following is not a general issue that affect many different types of software ?}: Value
        \item \textbf{Systems should be developed using a managed and understood development process: one process fits all.}: False
        \item \textbf{Which of the following is not an essential attribute of good software ?}: Affordability
        \item \textbf{Even when following all ethical principles, ethical dilemmas can occur.}: True
    \end{itemize}

    \pagebreak 
    \unsect{Quiz 2}
    \begin{itemize}
        \item \textbf{Requirements engineering is the process of developing a software specification.} : True
        \item \textbf{One approach to reducing the costs of rework is "change anticipation", where the process is designed so that changes can be accommodated at relatively low cost.}: False
        \item \textbf{Programming is an individual activity with no standard process.}: True
        \item \textbf{It is increasingly irrelevant to distinguish between software development and evolution.}: true
        \item \textbf{Process descriptions may include pre- and post-conditions, which are statements that are true before and after a process activity has been enacted or a product produced.}: True
        \item \textbf{An advantage of the waterfall model is the ease of accommodating change after the process is underway.}: False
        \item \textbf{Agile processes are processes where all of the process activities are planned in advance and progress is measured against this plan.}: False
        \item \textbf{Component adaptation and integration is a key stage of which software process model ?}: Integration and configuration
        \item \textbf{Which of the following is not a stage of testing?}: Value testing
        \item \textbf{In the SEI Capability Maturity Model which level denotes that process improvement strategies have been defined and are being used ?}: Optimized
    \end{itemize}

    \pagebreak 
    \unsect{Quiz 3}
    \begin{itemize}
        \item \textbf{Which of the following is not part of the Agile Manifesto ?}: comprehensive documentation over working software 
        \item \textbf{In Srum,which term matches this definition: "An individual (or possibly a small group) whose job is to identify product features or requirements, prioritize these for development and continuously review the product backlog to ensure that the project continues to meet critical business needs"}: Product owner
        \item \textbf{Extreme Programming (XP) takes an ‘extreme’ approach to iterative development.}: True
        \item \textbf{In Extreme Programming, user requirements are expressed as user stories or scenarios. They are not documented but broken into tasks. These tasks are the basis of schedule and cost estimates.}: False i guess
        \item \textbf{Agile development is a plan driven approach. }: False
        \item \textbf{Which of the following is not a principle of Agile methods ? }: Process focus
        \item \textbf{Which of the following is not an example of re-factoring ?}: the coding of a low-level device driver in assembly language
        \item \textbf{In Pair Programming 2 programmers are paired up for the duration of the complete project.}: False
        \item \textbf{Scaling up’ is concerned with how agile methods can be introduced across a large organization with many years of software development experience.}: False
        \item \textbf{Consider this description of programming practices: "Large amounts of overtime are not considered acceptable as the net effect is often to reduce code quality and medium term productivity" Which term is used in Extreme programming for this practice ?}: Sustainable pace
    \end{itemize}

    \pagebreak 
    \unsect{Quiz 4}
    \begin{itemize}
        \item \textbf{The requirements engineering process can be viewed as a spiral.}: True
        \item \textbf{Which of the following is not a generic activity in the requirements engineering process ?}: Modeling
        \item \textbf{The system design model is generated during the requirements engineering process.}: False
        \item \textbf{Many agile methods argue that producing detailed system requirements is a waste of time. }: True
        \item \textbf{The software requirements document should set out WHAT the system should do rather than HOW it should do it. }: True
        \item \textbf{The requirements engineering process is an iterative process that includes requirements elicitation, specification and validation.}: True
        \item \textbf{Which of the following is not a problem in requirements elicitation ?}: Stakeholders know what they really want.
        \item \textbf{System managers cannot be system stakeholders.}: False
        \item \textbf{Which of the following is not a metric for specifying nonfunctional requirements ?}: Luck
        \item \textbf{Which of the following is not a type of requirement ?}: Environmental concerns
    \end{itemize}
    






\end{document}
