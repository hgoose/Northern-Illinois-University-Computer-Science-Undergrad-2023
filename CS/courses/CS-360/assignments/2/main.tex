 \documentclass{report}
 
 \input{~/dev/latex/template/preamble.tex}
 \input{~/dev/latex/template/macros.tex}
 
 \title{\Huge{}}
 \author{\huge{Nathan Warner}}
 \date{\huge{}}
 \fancyhf{}
 \rhead{}
 \fancyhead[R]{\itshape Warner} % Left header: Section name
 \fancyhead[L]{\itshape\leftmark}  % Right header: Page number
 \cfoot{\thepage}
 \renewcommand{\headrulewidth}{0pt} % Optional: Removes the header line
 %\pagestyle{fancy}
 %\fancyhf{}
 %\lhead{Warner \thepage}
 %\rhead{}
 % \lhead{\leftmark}
 %\cfoot{\thepage}
 %\setborder
 % \usepackage[default]{sourcecodepro}
 % \usepackage[T1]{fontenc}
 
 % Change the title
 \hypersetup{
     pdftitle={}
 }

 \geometry{
  left=1.5in,
  right=1.5in,
  top=1in,
  bottom=1in
}
 
 \begin{document}
     % \maketitle
     %     \begin{titlepage}
     %    \begin{center}
     %        \vspace*{1cm}
     % 
     %        \textbf{}
     % 
     %        \vspace{0.5cm}
     %         
     %             
     %        \vspace{1.5cm}
     % 
     %        \textbf{Nathan Warner}
     % 
     %        \vfill
     %             
     %             
     %        \vspace{0.8cm}
     %      
     %        \includegraphics[width=0.4\textwidth]{~/niu/seal.png}
     %             
     %        Computer Science \\
     %        Northern Illinois University\\
     %        United States\\
     %        
     %             
     %    \end{center}
     % \end{titlepage}
     % \tableofcontents
    \pagebreak \bigbreak \noindent
    Nate Warner \ \quad \quad \quad \quad \quad \quad \quad \quad \quad \quad \quad \quad  CS 360 \quad  \quad \quad \quad \quad \quad \quad \quad \quad \ \ \quad \quad Spring 2025
    \begin{center}
        \textbf{Assignment 2 - Due: Fri, Jan 31}
    \end{center}
    \bigbreak \noindent 
    \begin{mdframed}
        1. Convert the following binary numbers to their decimal representations:
        \begin{enumerate}[label=\alph*.]
            \item 11
            \item 1101
            \item 111011
            \item 0101
            \item 1101011
        \end{enumerate}
    \end{mdframed}
    \bigbreak \noindent 
    We have
    \begin{enumerate}[label=(\alph*)]
        \item $11_{2} = 1 \cdot 2^{0} + 1\cdot 2^{1} = 3_{10} $
        \item $1101_{2} = 1\cdot 2^{0} + 0 \cdot 2^{1} + 1\cdot 2^{2} + 1\cdot 2^{3}  = 13_{10}$
        \item $111011_{2} =  1 \cdot 2^{0} + 1\cdot 2^{1} + 0 + 1\cdot 2^{3} + 1\cdot 2^{4} + 1\cdot 2^{5} = 59_{10}$
        \item $0101_{2} = 1 \cdot 2^{0} + 0 \cdot 2^{1} + 1 \cdot 2 ^{2} + 0 \cdot 2^{3} = 5_{10}$
        \item $1101011_{2} = 1 \cdot 2^{0} + 1\cdot 2^{1} + 0 + 1\cdot 2^{3} + 0 + 1\cdot 2^{5} + 1 \cdot 2^{6} = 107_{10}  $
    \end{enumerate}



    \bigbreak \noindent 
    \begin{mdframed}
        2. Convert the following hexadecimal numbers to their decimal representations
        \begin{enumerate}[label=\alph*.]
            \item 11 
            \item $A1$
            \item $CEF$
            \item $BA9$
            \item $C89$
        \end{enumerate}
    \end{mdframed}
    \bigbreak \noindent 
    We have
    \begin{enumerate}[label=(\alph*)]
        \item $11_{16} = 1 \cdot 16^{0} + 1 \cdot 16^{1} = 17_{10} $
        \item $A1_{16} = 1 \cdot 16^{0} + 10 \cdot 16^{1}= 161_{10} $
        \item $CEF_{16} = 15 \cdot 16^{0} + 14 \cdot 16^{1} + 12 \cdot 16^{2} = 3311_{10} $
        \item $BA9_{16} = 9 \cdot 16^{0} + 10 \cdot 16^{1} + 11 \cdot 16^{2} = 2985_{10} $
        \item $C89_{16} = 9 \cdot 16^{0} + 8 \cdot 16^{1} + 12 \cdot 16^{2} = 3209 $
    \end{enumerate}

    \pagebreak \bigbreak \noindent 
    \begin{mdframed}
        3. Convert the following decimal numbers to both their hexadecimal and binary representations
        \begin{enumerate}[label=\alph*.]
            \item 11 
            \item 4000
            \item 42
            \item 4095
        \end{enumerate}
    \end{mdframed}
    \bigbreak \noindent 
    a.) We first convert $11_{10}$ to its base two representation using the division algorithm. If $n$ is an integer in its decimal representation, we divide $n$ by two to get its quotient and remainder, we then express the remainder in base two representation and set $n=q$, where $q$ is the quotient. We stop this procedure once we hit $q=0$. We form the binary representation by working down the expressions, adding each remainder to the left of the existing representation.
    \begin{align*}
        11 &= 2(5) + 1:\ 1_{10} = 1_{2} \\
        5 &= 2(2) + 1:\ 2_{10} = 1_{2} \\
        2 &= 2(1) + 0:\ 0_{10} = 0_{2} \\
        1 &= 2(0) + 1:\ 1_{10} = 1_{2} 
    \end{align*}
    Thus, $11_{10} = 1011_{2}$. A similar algorithm converts $11_{10}$ to its hexadecimal representation
    \begin{align*}
        11 &= 16(0) + 11:\ 11_{10} = B_{16}
    \end{align*}
    Thus, $11_{10} =  B_{16}$
    \bigbreak \noindent 
    b.) In binary, we have 
    \begin{align*}
        4000 &= 2(2000) + 0:\ 0_{10} = 0_{2} \\
        2000 &= 2(1000) + 0:\ 0_{10} = 0_{2} \\
        1000 &= 2(500) + 0:\ 0_{10} = 0_{2} \\
        500 &= 2(250) + 0:\ 0_{10} = 0_{2} \\
        250 &= 2(125) + 0:\ 0_{10} = 0_{2} \\
        125 &= 2(62) + 1:\ 1_{10} = 1_{2} \\
        62 &= 2(31) + 0:\ 0_{10} = 0_{2} \\
        31 &= 2(15) + 1:\ 1_{10} = 1_{2} \\
        15 &= 2(7) +1:\ 1_{10} = 1_{2} \\
        7 &= 2(3) +1:\ 1_{10} = 1_{2} \\
        3 &=2(1) + 1:\ 1_{10} = 1_{2} \\
        1 &= 2(0) + 1:\ 1_{10} = 1_{2}
    \end{align*}
    Thus, $4000_{10} = 111110100000_{2}$. For further conversions, we will omit part of the remainder conversion and simply state its representation. For example, $62 = 2(31) + 0:\ 0_{10} = 0_{2}$ should simply be stated as $62 = 2(31) + 0: 0_{2}$.
    \bigbreak \noindent 
    In hex, we have
    \begin{align*}
        4000 &= 16(250) + 0:\ 0_{16} \\
        250 &= 16(15) + 10:\ A_{16} \\
        15 &= 16(0) + 15:\ F_{16}
    \end{align*}
    Thus, $4000_{10} = FA0_{16} $
    \bigbreak \noindent 
    c.)  Binary:
    \begin{align*}
        42 &= 2(21) + 0:\ 0_{2}  \\
        21 &= 2(10) +1:\ 1_{2} \\
        10 &= 2(5) +0 :\ 0_{2} \\
        5 &= 2(2) + 1:\ 1_{2} \\
        2 &= 2(1) + 0:\ 0_{2} \\
        1 &= 2(0) + 1:\ 1_{2}
    \end{align*}
    Thus, $42_{10} =101010_{10} $. For hex,
    \begin{align*}
        42 &= 16(2) + 10:\ A_{16} \\
        2 &= 16(0) + 2:\ 2_{16}
    \end{align*}
    Thus, $42_{10} = 2A_{16} $
    \bigbreak \noindent 
    d.) Binary:
    \begin{align*}
        4095 &= 2(2047) + 1:\ 1_{2} \\
        2047 &= 2(1023) + 1:\ 1_{2} \\
        1023 &= 2(511) + 1:\ 1_{2} \\
        511 &= 2(255) + 1:\ 1_{2} \\
        255 &= 2(127) + 1:\ 1_{2} \\
        127 &= 2(63) +1:\ 1_{2} \\
        63 &= 2(31) + 1:\ 1_{2} \\
        31 &= 2(15) + 1:\ 1_{2} \\
        15 &= 2(7) + 1:\ 1_{2} \\
        7 &= 2(3) + 1:\ 1_{2} \\
        3 &=2(1) + 1:\ 1_{2} \\
        1 &= 2(0) +1:\ 1_{2}
    \end{align*}
    Thus, $4095_{10} = 111111111111_{2} $. For hex,
    \begin{align*}
        4095 &= 16(255) + 15:\ F_{16} \\
        255 &= 16(15) + 15:\ F_{16} \\
        15 &= 16(0) + 15:\ F_{16}
    \end{align*}
    Thus, $4095_{10} =FFF_{16} $




    \pagebreak \bigbreak \noindent 
    \begin{mdframed}
        4. Do the following binary arithmetic giving the answer in binary
        \begin{enumerate}[label=\alph*.]
            \item $10110 + 01101$
            \item $11001 + 00101$
            \item $10110 - 01101$
        \end{enumerate}
    \end{mdframed}
    \bigbreak \noindent 
    a.)
    \begin{align*}
        \begin{array}{cccccc}
            1 & 1 & 1 &  &  &   \\
              &1&0&1&1&0 \\
              +& 0 & 1 & 1 & 0 & 1 \\
              \hline
              1 & 0 & 0 & 0 & 1 & 1
        \end{array}
    \end{align*}
    b.)
    \begin{align*}
        \begin{array}{cccccc}
            &&&&1& \\
            &1&1&0&0&1 \\
            +&0&0&1&0&1 \\
            \hline 
             &1&1&1&1&0
        \end{array}
    \end{align*}
    c.) 
    \begin{align*}
        \begin{array}{cccccc}
            &0&2&&0&2 \\
             &\diagstrike{1} & \diagstrike{0} & 1 & \diagstrike{1} & \diagstrike{0} \\
            -&0&1&1&0&1\\
            \hline 
             &0&1&0&0&1
        \end{array}
    \end{align*}


    \bigbreak \noindent 
    \begin{mdframed}
        5. Do the following hexadecimal arithmetic giving the answer in hexadecimal
        \begin{enumerate}[label=(\alph*)]
            \item $82CD + 1982$
            \item $E2C + A31$
            \item $FB28 - 3254$
            \item $E2C - A31$
        \end{enumerate}
    \end{mdframed}
    \bigbreak \noindent 
    a.)
    \begin{align*}
        \begin{array}{ccccc}
           &&1&& \\
           &8&2&C&D \\
            +&1&9&8&2 \\
            \hline
             &9&C&4&F
        \end{array}
    \end{align*}
    b.)
    \begin{align*}
        \begin{array}{cccc}
            1&&& \\
             &E&2&C \\
            +&A&3&1 \\
            \hline
            1&8&5&D
        \end{array}
    \end{align*}
    \bigbreak \noindent 
    c.)
    \begin{align*}
        \begin{array}{ccccc}
           &&A&18&  \\
           &F&\diagstrike{B}&\diagstrike{2}&8 \\
            -&3&2&5&4 \\
            \hline
             &C&8&D&4
        \end{array}
    \end{align*}
    d.)
    \begin{align*}
        \begin{array}{cccc}
           &D&18&C \\
           &\diagstrike{E}&\diagstrike{2}&C \\
            -&A&3&1 \\
            \hline
             &3&F&B
        \end{array}
    \end{align*}

    

    \bigbreak \noindent 
    \begin{mdframed}
        6. Do the following arithmetic as if these were five-bit signed representations and indicate if overflow occurs and, if so, why
        \begin{enumerate}[label=(\alph*)]
            \item $10110 + 01101$
            \item $11001 + 00101$
            \item $10110 - 01101$
            \item $11111 - 01011$
        \end{enumerate}
    \end{mdframed}
    \bigbreak \noindent 
    a.) 
    \begin{align*}
        \begin{array}{cccccc}
            \boxed{1} & \boxed{1} & 1 & &  & \\
                     &1&0&1&1&0 \\
            +&0&1&1&0&1 \\
            \hline 
             &0&0&0&1&1
        \end{array}
    \end{align*}
    Since the carry into the sign bit and the carry out of the sign bit (the boxed numbers) match, there is no overflow and the result is valid.
    \bigbreak \noindent 
    b.)
    \begin{align*}
        \begin{array}{cccccc}
            \boxed{0} & \boxed{0} & & & &  \\
                      &1&1&0&0&1 \\
            +&0&0&1&0&1 \\
            \hline 
             &1&1&1&1&0
        \end{array}
    \end{align*}
    Since the boxed carries match, no overflow.
    \bigbreak \noindent 
    c.)  We convert the subtrahend to its two's complement and add. $01101$ has two's complement $10011$. Thus,
    \begin{align*}
        \begin{array}{cccccc}
            \boxed{1} & \boxed{0} & 1&1&& \\
                      &1&0&1&1&0 \\
            +&1&0&0&1&1 \\
            \hline
             &0&1&0&0&1
        \end{array}
    \end{align*}
    Since the boxed carries match, we have overflow.
    \bigbreak \noindent 
    d.) We first convert the subtrahend to its two's complement, then add. $01011$ has two's complement $10101$. Thus,
    \begin{align*}
        \begin{array}{cccccc}
            \boxed{1}&\boxed{1}&1&1&1& \\
             &1&1&1&1&1 \\
            +&1&0&1&0&1 \\
            \hline
             &1&0&1&0&0
        \end{array}
    \end{align*}
    \bigbreak \noindent 
    Since the boxed carries match, no overflow.

    \pagebreak \bigbreak \noindent 
    \begin{mdframed}
        7. Assume that
        \begin{align*}
            &\text{Register 0 contains $0007F144$} \\
            &\text{Register 1 contains $00000028$} \\
            &\text{Register 7 contains $EC088840$}
        \end{align*}
        If they are valid, calculate the absolute $D(X,B)$ addresses for the representations below.  If they are not valid, explain why
        \begin{enumerate}[label=(\alph*)]
            \item $56(,1)$
            \item $0(0,1,7)$
            \item $6(7,0)$
            \item $11(1,7)$
        \end{enumerate}
        Note: Remember that addresses are 24 bits long, NOT 32.)
    \end{mdframed}
    \bigbreak \noindent 
    a.) First, we convert $56_{10}$ to its hexadecimal representation.
    \begin{align*}
        56 &= 16(3) + 8:\ 8_{16} \\
        3 &= 16(0) + 3:\ 3_{16}
    \end{align*}
    Thus, $56_{10} = 38_{16}$. Next, we add the contents of $R0$ and $R1$, because $D(,B)$ is shorthand for $D(0,B)$. Thus, we have $0007F144 + 00000028 = 0007F16C$
    \bigbreak \noindent 
    Last, we add $38_{16}$ to the contents of $R1 + R0$. That is, $0007F16C + 38 = 07F1A4$
    \bigbreak \noindent 
    Thus, the absolute address of $56(,1)$ is $07F1A4$
    \bigbreak \noindent 
    b.) Not valid, notation has no meaning
    \bigbreak \noindent 
    c.) First we add the contents of $R7$ and $R0$. We have $EC088840 + 0007F144 = EC107984$. Next, we add $6_{10} = 6_{16}$. Thus, $EC107984 + 6 = EC10798A$
    \bigbreak \noindent 
    Therefore, the absolute address $6(7,0)$ is $10798A$
    \bigbreak \noindent 
    d.) First we add the contents of $R1$ and $R7$. We have $EC088840 + 00000028 = EC088868$. Then, we add $11_{10} = B_{16}$. Thus, $EC088868 + B = EC088873$
    \bigbreak \noindent 
    Therefore, the absolute address of $11(1,7)$ is $088873$















 \end{document} % (:
