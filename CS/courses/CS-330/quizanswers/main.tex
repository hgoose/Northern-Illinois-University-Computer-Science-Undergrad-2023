\documentclass{report}

\input{~/dev/latex/template/preamble.tex}
\input{~/dev/latex/template/macros.tex}

\title{\Huge{}}
\author{\huge{Nathan Warner}}
\date{\huge{}}
\fancyhf{}
\rhead{}
\fancyhead[R]{\itshape Warner} % Left header: Section name
\fancyhead[L]{\itshape\leftmark}  % Right header: Page number
\cfoot{\thepage}
\renewcommand{\headrulewidth}{0pt} % Optional: Removes the header line
%\pagestyle{fancy}
%\fancyhf{}
%\lhead{Warner \thepage}
%\rhead{}
% \lhead{\leftmark}
%\cfoot{\thepage}
%\setborder
% \usepackage[default]{sourcecodepro}
% \usepackage[T1]{fontenc}

% Change the title
\hypersetup{
    pdftitle={Quiz Answers}
}

\begin{document}
    % \maketitle
        \begin{titlepage}
       \begin{center}
           \vspace*{1cm}
    
           \textbf{Quiz Answers} \\
           CS-330 Unix and Network Programming
    
           \vspace{0.5cm}
            
                
           \vspace{1.5cm}
    
           \textbf{Nathan Warner}
    
           \vfill
                
                
           \vspace{0.8cm}
         
           \includegraphics[width=0.4\textwidth]{~/niu/seal.png}
                
           Computer Science \\
           Northern Illinois University\\
           March 1, 2024 \\
           United States\\
           
                
       \end{center}
    \end{titlepage}
    \tableofcontents
    \pagebreak 
    \unsect{Quiz 6: Awk}
    \begin{itemize}
        \item Which function in awk is used to divide a string into pieces separated by the field seperator and store the pieces in an array ?
            \begin{itemize}
                \item split
            \end{itemize}
        \item What is the meaning of the \$0 variable in awk ?
            \begin{itemize}
                \item it holds the entire record 
            \end{itemize}
        \item awk reads input lines automatically
            \begin{itemize}
                \item true
            \end{itemize}
        \item Variables in awk are initialized how ?
            \begin{itemize}
                \item to 0 or "" depending on context when first used
            \end{itemize}
        \item Every awk program must use the BEGIN and END patterns.
            \begin{itemize}
                \item false
            \end{itemize}
        \item Which command line option allows to specifiy an awk program in a file ?  
            \begin{itemize}
                \item \textbf{-f}
            \end{itemize}
        \item awk allows strings as array index.
            \begin{itemize}
                \item true
            \end{itemize}
        \item awk can use which character as field seperator ?
            \begin{itemize}
                \item any character
            \end{itemize}
        \item In awk, strings printed with the "\%20s" printf directive will always be left justified.
            \begin{itemize}
                \item false
            \end{itemize}
        \item What is the string concatenation operator in awk ?
            \begin{itemize}
                \item the space character
            \end{itemize}
    \end{itemize}

    \pagebreak 
    \unsect{Quiz 7: Sed}
    \begin{itemize}
        \item In sed, the "i" command adds lines before the address
            \begin{itemize}
                \item True
            \end{itemize}
        \item In sed, when using a range address, the lines specified need not be consecutive.
            \begin{itemize}
                \item False
            \end{itemize}
        \item In sed, the dollar sign (\$) can be used as a single line address. What is its meaning?
            \begin{itemize}
                \item the last line of input
            \end{itemize}
        \item If the exclamation mark "!" is used after an sed address, then it will only search for obsolete lines.
            \begin{itemize}
                \item False
            \end{itemize}
        \item Which character is used to delimit the search and replacement components of the sed "s" command?
            \begin{itemize}
                \item any, as long as all 3 are the same
            \end{itemize}
        \item If sed is invoked as "sed -n" then it sorts its input numerically.
            \begin{itemize}
                \item False
            \end{itemize}
        \item The sed editor can be called from a shell script
            \begin{itemize}
                \item True
            \end{itemize}
        \item Which of the following utilities can be used to systematically process files?
            \begin{itemize}
                \item all of the above
            \end{itemize}
        \item Which of the following is NOT a valid address type for sed?
            \begin{itemize}
                \item sublet address
            \end{itemize}
        \item If no address is specified for the sed command then the command is applied to every input line.
            \begin{itemize}
                \item true
            \end{itemize}
    \end{itemize}

    \pagebreak 
    \unsect{Quiz 9: Systems Programming}
    \begin{itemize}
        \item The C library function perror translate an error code into an understandable error message.
            \begin{itemize}
                \item True
            \end{itemize}
        \item In C++, C strings are handled the same way as instances of the standard string class.
            \begin{itemize}
                \item False
            \end{itemize}
        \item C++ can call functions from the standard C library. 
            \begin{itemize}
                \item true
            \end{itemize}
        \item In C++ what is the correct include to use the strlen or strcpy library functions ?
            \begin{itemize}
                \item #include <cstring>
            \end{itemize}
        \item The C regular expression library uses 2 functions: regcomp and regexec. Which statement is true about these functions?
            \begin{itemize}
                \item regexec runs the search that was prepared by regcomp
            \end{itemize}
        \item For the strcpy C library function, the size of the destination array must be long enough to contain the source string including the terminating null character.
            \begin{itemize}
                \item true
            \end{itemize}
        \item A C++ program cannot access environment variables.
            \begin{itemize}
                \item false
            \end{itemize}
        \item C library function exit terminates a process and allows to set the return status. A status of 0 indicates failure.
            \begin{itemize}
                \item false
            \end{itemize}
        \item The C structure dirent does not contain the file name.
            \begin{itemize}
                \item false
            \end{itemize}
        \item Which C library functions enable directory I/O ?
            \begin{itemize}
                \item opendir, readdir 
            \end{itemize}
    \end{itemize}

    When it lands on the spring it has the same energy as above $\frac{1}{2}mv_{i}^{2}  = 79 J$. That will all go into spring potential energy
    \begin{align*}
        79J = \frac{1}{2}kl^{2} \\
        \implies x = 0.4m
    .\end{align*}

    \pagebreak 
    \unsect{Quiz 10}
    \begin{itemize}
        \item A system call is linked to an executable and becomes logically part of the executable.
            \begin{itemize}
                \item False
            \end{itemize}
        \item C++ uses which of the following to identify a  file in low-level IO using system calls ?
            \begin{itemize}
                \item File descriptor number
            \end{itemize}
        \item The read and write system calls use C++ strings to handle its data.
            \begin{itemize}
                \item False
            \end{itemize}
        \item open, close, create, read and write are system calls to achieve low-level I/O.
            \begin{itemize}
                \item False
            \end{itemize}
        \item Which of the following is not a valid flag to for the open system call ?
            \begin{itemize}
                \item O\_READ
            \end{itemize}
        \item Which of to following is a system call that can be used to create a file ?
            \begin{itemize}
                \item Open
            \end{itemize}
        \item A system call uses a special C++ syntax for invocation.
            \begin{itemize}
                \item False
            \end{itemize}
        \item Which system call is used to remove a file ?
            \begin{itemize}
                \item unlink
            \end{itemize}
        \item The dup system call is used to claim standard I/O from inside a program.
            \begin{itemize}
                \item True
            \end{itemize}
        \item System calls typically set the errno variable and return -1 if they encounter and error.
            \begin{itemize}
                \item True
            \end{itemize}
    \end{itemize}

    \pagebreak 
    \unsect{Quiz 11}
\begin{enumerate}
    \item Match the following exec functions to their description:
    \begin{itemize}
        \item execl, execlp -- specify arguments as list. \textbf{Answer: Option 1}
        \item execv, execvp -- specify arguments as array of strings. \textbf{Answer: Option 2}
        \item execlp, execvp -- look for new executable via PATH. \textbf{Answer: Option 0}
    \end{itemize}
    
    \item Unix processes can communicate via Unix pipes.
    \begin{itemize}
        \item True
    \end{itemize}
    
    \item Which of the following is not a potential return value from the fork system call?
    \begin{itemize}
        \item 1, to indicate that 1 child process was created. \textbf{Answer: This is not a potential return value}
    \end{itemize}
    
    \item The pipe system call creates a bidirectional data channel.
    \begin{itemize}
        \item False
    \end{itemize}
    
    \item A process is a program in execution.
    \begin{itemize}
        \item True
    \end{itemize}
    
    \item Which system call creates a new process?
    \begin{itemize}
        \item fork
    \end{itemize}
    
    \item How long does the wait system call wait?
    \begin{itemize}
        \item until a child process terminates
    \end{itemize}
    
    \item System calls pipe, dup, and fork are used together to implement the "|" pipe character used by the command interpreter on the command line.
    \begin{itemize}
        \item True
    \end{itemize}
    
    \item The fork system call duplicates the current process. The 2 processes (old and newly created) are exactly the same.
    \begin{itemize}
        \item False
    \end{itemize}
    
    \item A Unix pipe is represented in C++ as an array of \_\_\_\_\_ file descriptors.
    \begin{itemize}
        \item 2
    \end{itemize}
\end{enumerate}

    \pagebreak 
    \unsect{Quiz 12: Networking}
    \begin{itemize}
        \item OSI in the OSI reference model stands for:
            \begin{itemize}
                \item Open systems interconnection
            \end{itemize}
        \item OSI in the OSI reference model stands for:
            \begin{itemize}
                \item dns
            \end{itemize}
        \item getaddrinfo is a Unix system call ?
            \begin{itemize}
                \item False
            \end{itemize}
        \item Both IPv4 and IPv6 addresses are 32 bit long.
            \begin{itemize}
                \item False 
            \end{itemize}
        \item Which of the following is not a technology used in the physical layer of the OSI model ?
            \begin{itemize}
                Tcp
            \end{itemize}
            \item Match the port number to the service that is usually provided on it. 
                \begin{itemize}
                    \item 22: SSH
                    \item 25: SMTP
                    \item 80: HTTP
                    \item 53: DNS
                \end{itemize}
        \item In the OSI reference model complexities of communication are organized into successive layers of protocols:  lower-level layers are more specific to medium, higher-level layers are more specific to application. Match the layer name to its functionality:
            \begin{itemize}
                \item Transport layer: Provides functions to gaurantee reliable network link
                \item Network link: Establishes, maintains, and terminates network connections
                \item Data link: Ensures the reliability of link
                \item Physical layer: Controls transmission of the raw bit stream over the medium
            \end{itemize}
        \item A protocol describes the syntax, semantics, and synchronization of communication
            \begin{itemize}
                \item true
            \end{itemize}
        \item Which layer in the OSI model provides addressing via an IP address ?
            \begin{itemize}
                \item Network layer                
            \end{itemize}
        \item A ___________ describes the syntax, semantics, and synchronization of communication
            \begin{itemize}
                \item Protocol
            \end{itemize}

    \end{itemize}
    \end{itemize}

    \pagebreak 
    \unsect{Quiz 13: UDP and TCP}
    \begin{itemize}
        \item A socket is end-point of a communication link, it is identified by 
            \begin{itemize}
                \item ip address 
                \item port number
            \end{itemize}
        \item  The TCP protocol uses a datagram as message.
            \begin{itemize}
                \item False
            \end{itemize}
        \item Which of the following protocols is not part of the OSI model transport layer ?
            \begin{itemize}
                \item HTTP
            \end{itemize}
        \item The socket programming concept was first introduced as part of Windows 2000.
            \begin{itemize}
                \item False
            \end{itemize}
        \item The socket programming concept was first introduced as part of Windows 2000.
            \begin{itemize}
                \item SYN 
                \item SYN,ACK(SYN) 
                \item ACK(SYN)
            \end{itemize}
        \item Match the system calls to their functionality.
            \begin{itemize}
                \item \textbf{Socket}: Create a new communication endpoint
                \item \textbf{bind}: attach a local address to a socket
                \item \textbf{sendto}: send (write) some data over the connection
                \item \textbf{recvfrom}: recieve (read) some data over the connection
            \end{itemize}
        \item TCP is the transmission control protocol that is connection oriented and guarantees delivery.
            \begin{itemize}
                \item True
            \end{itemize}
    \item The listen system call allows to specify the length of the incoming queue.
        \begin{itemize}
            \item True
        \end{itemize}
    \item What is the purpose of the htons library function ?
        \begin{itemize}
            \item  convert a host integer into a network 2 byte integer
        \end{itemize}
    \item The accept system call is used in UDP server programming.
        \begin{itemize}
            \item False
        \end{itemize}

    \end{itemize}

    \pagebreak 
    \unsect{Quiz 14: TCP Server and Job control}
    \begin{itemize}
        \item A program can be running in the foreground and background at the same time.
            \begin{itemize}
                \item false
            \end{itemize}
        \item The dup system call can be used in a forking TCP server to simplify sending responses to the client.
            \begin{itemize}
                \item true 
            \end{itemize}
        \item Upon success, the opendir function returns a file descriptor for the directory.
            \begin{itemize}
                \item false 
            \end{itemize}
        \item Which command shows the jobs for a user:
            \begin{itemize}
                \item jobs 
            \end{itemize}
        \item Which number represents the strongest signal that can be sent to a process ?
            \begin{itemize}
                \item 9  
            \end{itemize}
        \item A typical TCP server uses what to reduce client waiting time
            \begin{itemize}
                \item  fork to create a child process for the client request 
            \end{itemize}
        \item What is the command to send a signal to a process ?
            \begin{itemize}
                \item kill 
            \end{itemize}
        \item The readdir function returns a structure which contains the name of the directory entry.
            \begin{itemize}
                \item true 
            \end{itemize}
        \item Only the superuser can create periodically running jobs via crontab.
            \begin{itemize}
                \item false 
            \end{itemize}
        \item For the read system call, what can we program to allow an unlimited amount of data being read ?
            \begin{itemize}
                \item use a loop to read fixed chunks of data 
            \end{itemize}
    \end{itemize}
    
    \pagebreak 
    \unsect{Quiz 15: Sys admin}
    \begin{itemize}
        \item The only way to create a new user on Ubuntu is to use the GUI dialog "Settings".
            \begin{itemize}
                \item false
            \end{itemize}
        \item Only users or groups that are listed in /etc/sudoers can execute commands as super user.
            \begin{itemize}
                \item true
            \end{itemize}
        \item Which of the following is not a usual way to install new software on Linux ?
            \begin{itemize}
                \item update-manager
            \end{itemize}
        \item What does this command do:  mount /dev/sdb1 /mnt/extra
            \begin{itemize}
                \item it makes the first partition on disk sdb available as logical directory /mnt/extra
            \end{itemize}
        \item Which of the following files is not involved in storing user information ?
            \begin{itemize}
                \item /etc/root 
            \end{itemize}
        \item Which of the following is not a file format used to package Linux software ?
            \begin{itemize}
                \item mp4
            \end{itemize}
        \item In Unix there is one logical file system that is constructed from one or more physical file systems.
            \begin{itemize}
                \item true
            \end{itemize}
        \item Which UNIX command is used to change the login password ?
            \begin{itemize}
                \item passwd
            \end{itemize}
        \item a user can belong to only one group.
            \begin{itemize}
                \item false
            \end{itemize}
        \item What is the command to create a file system on a physical device ?
            \begin{itemize}
                \item mkfs
            \end{itemize}
    \end{itemize}
   
\end{document}
