\documentclass{report}

\input{~/dev/latex/template/preamble.tex}
\input{~/dev/latex/template/macros.tex}

\title{\Huge{}}
\author{\huge{Nathan Warner}}
\date{\huge{}}
\fancyhf{}
\rhead{}
\fancyhead[R]{\itshape Warner} % Left header: Section name
\fancyhead[L]{\itshape\leftmark}  % Right header: Page number
\cfoot{\thepage}
\renewcommand{\headrulewidth}{0pt} % Optional: Removes the header line
%\pagestyle{fancy}
%\fancyhf{}
%\lhead{Warner \thepage}
%\rhead{}
% \lhead{\leftmark}
%\cfoot{\thepage}
%\setborder
% \usepackage[default]{sourcecodepro}
% \usepackage[T1]{fontenc}

% Change the title
\hypersetup{
    pdftitle={Linux Sysadmin }
}

\begin{document}
    % \maketitle
        \begin{titlepage}
       \begin{center}
           \vspace*{1cm}
    
           \textbf{Linux Systems Administrator}
    
           \vspace{0.5cm}
            
                
           \vspace{1.5cm}
    
           \textbf{Nathan Warner}
    
           \vfill
                
                
           \vspace{0.8cm}
         
           \includegraphics[width=0.4\textwidth]{~/niu/seal.png}
                
           Computer Science \\
           Northern Illinois University\\
           United States\\
           
                
       \end{center}
    \end{titlepage}
    \tableofcontents
    
    \pagebreak 
    \unsect{Shell Job control}
    \bigbreak \noindent 
    \subsubsection{Terminology}
    \begin{itemize}
        \item Process is a program in execution
            \begin{itemize}
                \item process is reated every time you run a command
                \item each process has a unique process id
                \item processes are removed from the system when the command finishes its execution
            \end{itemize}
        \item Job is a unit of work
            \begin{itemize}
                \item Consists of the commands specified in a single command line
                \item A single job may involve several processes, each consisitng of an executable program
            \end{itemize}
    \end{itemize}
    \bigbreak \noindent 
    \subsubsection{Job contral Terminology}
    \begin{itemize}
        \item \textbf{Foreground job:}
            \begin{itemize}
                \item A job that has our immediate attention
                \item user has to wait for job to complete
            \end{itemize}
        \item \textbf{Background job:}
            \begin{itemize}
                \item a job that the user does not wait for 
                \item it runs independently of user interaction
            \end{itemize}
        \item \textbf{Unix shells allow users to:}
            \begin{itemize}
                \item Make jobs execute in the background
                \item move jobs from foreground to background
                \item determine their status, and terminate them
            \end{itemize}
    \end{itemize}

    \bigbreak \noindent 
    \subsubsection{Background jobs}
    \begin{itemize}
        \item How do we deiced which jobs to place in the background?
            \begin{itemize}
                \item jobs that are run non-interactively 
                \item jobs that do not require user input
            \end{itemize}
    \end{itemize}
    \bigbreak \noindent 
    To execute a command in the background, we put \& after it

    \bigbreak \noindent 
    \subsubsection{Managing jobs}
    \begin{itemize}
        \item \textbf{Display jobs}
            \begin{itemize}
                \item Command "jobs" lists your active jobs
                \item each job has job number
                \item job number with "\%" is used to refer to job
            \end{itemize}
        \item \textbf{Send job to background:} bg
        \item \textbf{Move job to foreground:} fg
    \end{itemize}

    \bigbreak \noindent 
    \subsubsection{Signaling jobs}
    \begin{itemize}
        \item \textbf{Command to send signal to job:} kill
            \begin{itemize}
                \item Example: kill -HUP 12324
                \item Example: kill -INT  \%1
            \end{itemize}
    \end{itemize}

    \bigbreak \noindent 
    \subsubsection{Ending jobs}
    \begin{itemize}
        \item \textbf{To stop a job:}
            \begin{itemize}
                \item kill -STOP
                \item resume via "bg" or "fg" command
            \end{itemize}
        \item \textbf{To terminate a job:}
            \begin{itemize}
                \item kill
                \item kill -INT
                \item kill -9
            \end{itemize}
        \item Once a job finishes it will display exit status
    \end{itemize}

    \bigbreak \noindent 
    \subsubsection{Scheduling utils}
    \begin{itemize}
        \item \textbf{crontab}
            \begin{itemize}
                \item Run a job based on a schedule
                \item Job is executed on a periodic basis
            \end{itemize}
        \item \textbf{at}
            \begin{itemize}
                \item run a job some time in the future
            \end{itemize}
        \item \textbf{batch}
            \begin{itemize}
                \item run a job when system load is low
            \end{itemize}
    \end{itemize}

    \bigbreak \noindent 
    \subsubsection{Crontab}
    \begin{itemize}
        \item Crontab is based on control file
        \item crontab file has 6 columns
            \begin{center}
                \begin{verbatim}
                    minute    hour    day    month    weekday    command
                \end{verbatim}
            \end{center}
        \item \textbf{Meaning:}
            \begin{enumerate}
                \item Minute 0-59
                \item Hour 0-23
                \item Day 1-31
                \item Month 1-12
                \item Weekday 1-7 (1=mon, 2=tue,..., 7=sun)
                \item Command Any unix command
            \end{enumerate}
    \end{itemize}
    \bigbreak \noindent 
    \nt{"*" means any value}
    \bigbreak \noindent 
    \subsubsection{Example crotab file}
    \bigbreak \noindent 
    \fig{.6}{./figures/cron.png}

    \bigbreak \noindent 
    \subsubsection{Crontab command}
    \bigbreak \noindent 
    \textbf{Options:}
    \begin{itemize}
        \item -e to edit the control file
        \item -l to list the control file
        \item -r to remove the control file
    \end{itemize}
    \bigbreak \noindent 
    \textbf{For superuser:}
    \begin{itemize}
        \item -u to edit another user's control file 
    \end{itemize}

    \bigbreak \noindent 
    \subsubsection{One time execution: at}
    \begin{itemize}
        \item Utility to run command(s) at a later time
            \begin{itemize}
                \item Must specify on the command the time and date on which your command to be executed
                \item No need to be logged in when the commands are scheduled to run
                \item Any output from command is sent via email
            \end{itemize}
    \end{itemize}
    \bigbreak \noindent 
    \subsubsection{At command syntax}
    \bigbreak \noindent 
    \begin{bashcode}
    at timeDate
        > command
        > <EOT>
    \end{bashcode}

    \bigbreak \noindent 
    \subsubsection{at examples}
    \bigbreak \noindent 
    \fig{.6}{./figures/at.png}

    \bigbreak \noindent 
    \subsubsection{at utils}
    \begin{itemize}
        \item atq lists users's scheduled jobs
        \ime atrm removes specified job from at queue
    \end{itemize}

    \bigbreak \noindent 
    \subsubsection{Batch command}
    \begin{itemize}
        \item \textbf{batch:} Schedules job to be performed while system load is \textit{low}
    \end{itemize}
    \bigbreak \noindent 
    \subsubsection{Batch command syntax}
    \bigbreak \noindent 
    \begin{bashcode}
    batch command
    \end{bashcode}



    \pagebreak 
    \unsect{Linux System Administration}
    \begin{itemize}
        \item User management
            \begin{itemize}
                \item adduser, sudo
            \end{itemize}
        \item software management
            \begin{itemize}
                \item apt-get, synaptic
            \end{itemize}
        \item file system management
            \begin{itemize}
                \item fdisk, mkfs, mount, fsck
            \end{itemize}
    \end{itemize}
    \bigbreak \noindent 
    \subsection{User Configuration}
    \begin{itemize}
        \item User info is stored in file /etc/passwd
            \begin{itemize}
                \item userid, user name, group, home directory, shell
                \item passwords are stored in separate file: /etc/shadow
            \end{itemize}
        \item Group info is stored in file /etc/group
            \begin{itemize}
                \item group id, group name
                \item additional group members
            \end{itemize}
        \item To find out group info, use: groups user-id
    \end{itemize}

    \bigbreak \noindent 
    \subsection{Steps to create a new user}
    \begin{enumerate}
        \item add info to /etc/passwd
        \item add info to /etc/shadow
        \item add info to /etc/group
        \item create home directory
        \item add default content to home directory
        \item set password
    \end{enumerate}
    \bigbreak \noindent 
    \subsubsection{Common debian utils}
    \begin{itemize}
        \item adduser, deluser
        \item addgroup, delgroup
    \end{itemize}

    \bigbreak \noindent 
    \subsection{Sudo}
    \begin{itemize}
        \item Execute commands as super user "root"
            \begin{itemize}
                \item Will be prompted for password  
            \end{itemize}
        \item /etc/sudoers
            \begin{itemize}
                \item list designated users/groups
                    \begin{itemize}
                        \item group "sudo"
                    \end{itemize}
            \end{itemize}
        \item lists allowed commands
            \begin{itemize}
                \item root can do anything
            \end{itemize}
    \end{itemize}

    \bigbreak \noindent 
    \subsubsection{sudo -s}
    \bigbreak \noindent 
    Runs new shell with super user privileges

    \bigbreak \noindent 
    \subsection{Software management}
    \begin{itemize}
        \item applications are bundled into package file
            \begin{itemize}
                \item tar
                    \begin{itemize}
                        \item original (tape) archive format
                    \end{itemize}
                \item rpm
                    \begin{itemize}
                        \item Redhat package manager format, download and intall via: \textbf{yum}
                    \end{itemize}
                \item deb
                    \begin{itemize}
                        \item Debian package format, download and install via: apt-get 
                    \end{itemize}
            \end{itemize}
    \end{itemize}

    \bigbreak \noindent 
    \subsubsection{Deb pacakge management}
    \begin{itemize}
        \item Basic utils
            \begin{itemize}
                \item dpkg - package manager
                \item apt-get - package handling util
            \end{itemize}
        \item User friendly interfaces
            \begin{itemize}
                \item aptitude - command line frontend
                \item synaptic - GUI frontend
            \end{itemize}
        \item Software manager
            \begin{itemize}
                \item Unifed web-based application store
            \end{itemize}
    \end{itemize}

    \bigbreak \noindent 
    \subsubsection{apt-get config}
    \bigbreak \noindent 
    \begin{itemize}
        \item /etc/apt/sources.list - contains locations of packages
    \end{itemize}

    \bigbreak \noindent 
    \subsubsection{apt-get sub-commands}
    \begin{itemize}
        \item update
            \begin{itemize}
                \item re-sync pacakge listing
            \end{itemize}
        \item install
            \begin{itemize}
                \item Install packages 
            \end{itemize}
        \item upgrade
            \begin{itemize}
                \item Update packages
            \end{itemize}
        \item remove, purge
            \begin{itemize}
                \item remove packages (delete config files)
            \end{itemize}
        \item dist-upgrade
            \begin{itemize}
                \item Update system
            \end{itemize}
        \item clean
            \begin{itemize}
                \item Empty local cache of downloaded packages
            \end{itemize}
    \end{itemize}

    \bigbreak \noindent 
    \subsection{File system commands}
    \begin{itemize}
        \item df - displays make up of logical file system
        \item fdisk - prepare partitions on physical medium
        \item mkfs
            \begin{itemize}
                \item Create file system on physical device 
                \item select file system type, ex: ext4
            \end{itemize}
        \item mount
            \begin{itemize}
                \item add additional physical into logical file system
                \item undone via: umount
                \item made permanent with entry into /etc/fstab
            \end{itemize}
    \end{itemize}

    \bigbreak \noindent 
    \subsection{Steps to enable new hard drive}
    \begin{itemize}
        \item Find device name: \textbf{fdisk -l}
        \item Edit partition table: \textbf{fdisk /dev/sdb}
            \begin{itemize}
                \item Create partition /dev/sdb1
            \end{itemize}
        \item create file system: \textbf{mkfs -t ext4 /dev/sdb1}
        \item mount file system
            \begin{itemize}
                \item mkdir /mnt/extra
                \item mount /dev/sdb1 /mnt/extra
            \end{itemize}
        \item see file systems: \textbf{df}
    \end{itemize}
    
\end{document}
