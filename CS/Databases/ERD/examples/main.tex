\documentclass{report}

\input{~/dev/latex/template/preamble.tex}
\input{~/dev/latex/template/macros.tex}

\title{\Huge{}}
\author{\huge{Nathan Warner}}
\date{\huge{}}
\fancyhf{}
\rhead{}
\fancyhead[R]{\itshape Warner} % Left header: Section name
\fancyhead[L]{\itshape\leftmark}  % Right header: Page number
\cfoot{\thepage}
\renewcommand{\headrulewidth}{0pt} % Optional: Removes the header line
%\pagestyle{fancy}
%\fancyhf{}
%\lhead{Warner \thepage}
%\rhead{}
% \lhead{\leftmark}
%\cfoot{\thepage}
%\setborder
% \usepackage[default]{sourcecodepro}
% \usepackage[T1]{fontenc}

% Change the title
\hypersetup{
    pdftitle={ERD Examples}
}

\begin{document}
    % \maketitle
        \begin{titlepage}
       \begin{center}
           \vspace*{1cm}
    
           \textbf{ERD Examples}
    
           \vspace{0.5cm}
            
                
           \vspace{1.5cm}
    
           \textbf{Nathan Warner}
    
           \vfill
                
                
           \vspace{0.8cm}
         
           \includegraphics[width=0.4\textwidth]{~/niu/seal.png}
                
           Computer Science \\
           Northern Illinois University\\
           United States\\
           
                
       \end{center}
    \end{titlepage}
    \tableofcontents
    \pagebreak 
    \unsect{Example 1: Space doctors}
    \bigbreak \noindent 
    \begin{mdframed}
        1. The purpose of this exercise is to design an ER diagram for a database that will be able to
        handle the storage of data that is necessary/useful to those working in the medical bay of
        a starship. The goal is to keep things simple – there may be many more things stored in a
        similar database if ever implemented in real life – but we will try to focus on the essentials
        based on the provided requirements, which have been moved to the next page so they can
        all be shown at once. This document will be available on Blackboard, in case you would
        like to download it and follow along 
        \bigbreak \noindent 
        This database will be used by the medical personnel of the starship’s sick bay, as well as
        by epidemiologists that are interested in tracking where and how diseases are contracted.
        To serve this purpose, we will be tracking which diseases were contracted by which crew
        members and when. To track where diseases are spreading, we will also want to keep a
        record of which planets each crew member visited, and when they were there.
        Some types of data that need to be available from this database (requirements) are listed
        below
        \begin{itemize}
            \item Every time a ship’s physician treats a crew member, a record must be stored of what illness was treated, and what treatment was used.
            \item Crew members may have allergies to certain treatments that need to be avoided. The doctor needs to be able to find this information in our database.
            \item Whenever a crew member visits a planet, information must be stored on the period of time they spent there. Information on planets that the starship has or may visit should be stored. This information will include a the planet’s name and location. If there are diseases known to be endemic to the planet, that information should be available.
            \item For each disease that is known, we need to be able to store a description of the symptoms and a list of effective treatments.
            \item For each type of treatment, a list of possible side effects caused by that treatment should be available.
            \item The captain of the starship should be able to list the names of all of the crew members that a given physician has treated.
            \item It needs to be possible to generate a list of the names of all of the planets that a given crew member has visited, along with the diseases common on those planets and recommended treatments for those disease.
            \item For a given disease, it should be possible to list the treatments for that disease, with a list of all of the crew members that have allergies to each of those treatments.
            \item The system should be able to produce the following statement for each crew member:
            \item Crew Member has received Treatment for Disease using Treatment by Physician at Time on Date
        \end{itemize}
    \end{mdframed}
\begin{figure}[ht]
    \centering
    \incfig{1}
    \label{fig:1}
\end{figure}

    
\end{document}
