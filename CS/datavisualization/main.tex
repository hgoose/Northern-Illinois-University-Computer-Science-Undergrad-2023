\documentclass{report}

\input{~/dev/latex/template/preamble.tex}
\input{~/dev/latex/template/macros.tex}

\title{\Huge{}}
\author{\huge{Nathan Warner}}
\date{\huge{}}
\fancyhf{}
\rhead{}
\fancyhead[R]{\itshape Warner} % Left header: Section name
\fancyhead[L]{\itshape\leftmark}  % Right header: Page number
\cfoot{\thepage}
\renewcommand{\headrulewidth}{0pt} % Optional: Removes the header line
%\pagestyle{fancy}
%\fancyhf{}
%\lhead{Warner \thepage}
%\rhead{}
% \lhead{\leftmark}
%\cfoot{\thepage}
%\setborder
% \usepackage[default]{sourcecodepro}
% \usepackage[T1]{fontenc}

% Change the title
\hypersetup{
    pdftitle={Data Visualization}
}

\begin{document}
    % \maketitle
        \begin{titlepage}
       \begin{center}
           \vspace*{1cm}
    
           \textbf{Data Visualization}
    
           \vspace{0.5cm}
            
                
           \vspace{1.5cm}
    
           \textbf{Nathan Warner}
    
           \vfill
                
                
           \vspace{0.8cm}
         
           \includegraphics[width=0.4\textwidth]{~/niu/seal.png}
                
           Computer Science \\
           Northern Illinois University\\
           United States\\
           
                
       \end{center}
    \end{titlepage}
    \tableofcontents
    \pagebreak 
    \unsect{Web programming: Html, css, and JS}
    \subsection{HTML and CSS}
    \begin{itemize}
        \item \textbf{SVGS with html}: An SVG (Scalable Vector Graphics) file is an XML-based image format used to display vector graphics. Unlike PNG or JPG images, SVGs scale infinitely without losing quality.        
            \bigbreak \noindent 
            Key properties:
            \begin{itemize}
                \item Resolution-independent
                \item Small file size for simple graphics
                \item Fully stylable with CSS
                \item Scriptable with JavaScript
                \item Ideal for icons, diagrams, charts, and UI graphics
            \end{itemize}
        \item \textbf{Embedding SVG directly into HTML (inline SVG)}: This is the most powerful and flexible method.
            \bigbreak \noindent 
            \begin{htmlcode}
                <svg width="200" height="100" viewBox="0 0 200 100">
                    <rect x="10" y="10" width="180" height="80" fill="steelblue" />
                    <circle cx="100" cy="50" r="30" fill="orange" />
                </svg>
            \end{htmlcode}
            \bigbreak \noindent 
            \begin{itemize}
                \item <svg> defines the canvas
                \item width / height define display size
                \item viewBox defines the internal coordinate system
                \item Shapes (rect, circle, line, path) are drawn inside
            \end{itemize}
            \begin{itemize}
                \item Fully stylable with CSS
                \item Can be animated
                \item JavaScript access to elements
                \item Best for interactive graphics
            \end{itemize}
        \item \textbf{SVG elements}: 
            \begin{center}
                \begin{tabularx}{\textwidth}{@{}XX@{}}
                    \toprule
                    \textbf{Element} &	\textbf{Purpose} \\
                    \midrule
                    <rect>	& Rectangle\\[2ex]
                    <circle>&	Circle\\[2ex]
                    <ellipse>	&Ellipse\\[2ex]
                    <line>	&Line\\[2ex]
                    <polyline>	&Connected lines\\[2ex]
                    <polygon>	&Closed shape\\[2ex]
                    <path>	&Complex shapes\\[2ex]
                    <text>	&Text\\[2ex]
                    \bottomrule
                \end{tabularx}
            \end{center}
        \item \textbf{SVG viewbox}: 
            \bigbreak \noindent 
            \begin{htmlcode}
                <svg viewBox="0 0 200 100">
            \end{htmlcode}
            \bigbreak \noindent 
            Means:
            \begin{itemize}
                \item Coordinate system starts at (0, 0)
                \item Width = 200 units
                \item Height = 100 units
            \end{itemize}
            This allows scaling without distortion.
        \item \textbf{CSS for <svg>}: 
            \begin{itemize}
                \item fill
                \item fill-opacity
                \item fill-rule
                \item stroke
                \item stroke-width
                \item stroke-opacity
                \item stroke-linecap
                \item stroke-linejoin
                \item stroke-dasharray
                \item stroke-dashoffset
                \item stroke-miterlimit
                \item color
                \item opacity
                \item x
                \item y
                \item cx
                \item cy
                \item r
                \item rx
                \item ry
                \item width
                \item height
                \item transform
                \item transform-origin
                \item transform-box
                \item font-family
                \item font-size
                \item font-style
                \item font-weight
                \item letter-spacing
                \item word-spacing
                \item text-anchor
                \item dominant-baseline
                \item alignment-baseline
                \item direction
                \item writing-mode
                \item display
                \item visibility
                \item overflow
                \item clip-path
                \item mask
                \item filter
                \item cursor
                \item pointer-events
                \item animation
                \item animation-name
                \item animation-duration
                \item animation-delay
                \item animation-iteration-count
                \item animation-timing-function
                \item transition
                \item transition-property
                \item transition-duration
            \end{itemize}
        \item \textbf{Paths}: The <path> element is the most powerful and flexible shape in SVG. Unlike <rect> or <circle>, a path can describe:
            \begin{itemize}
                \item Straight lines
                \item Curves
                \item Arcs
                \item Complex shapes
                \item Icons, symbols, letters
                \item Entire illustrations
            \end{itemize}
            It works by following a series of drawing commands stored in the $d$ attribute. The $d$ string is a mini drawing language, it reads left to right.
            \begin{itemize}
                \item \textbf{Move to (M)}: Moves the “pen” without drawing.
                    \bigbreak \noindent 
                    \begin{center}
                        M x y
                    \end{center}
                    Moves to $(x,y)$
                \item \textbf{Line to (L)}: Draws a straight line.
                    \bigbreak \noindent 
                    \begin{center}
                        L x y
                    \end{center}
                    Draws a line from the current point to $(x,y)$
                \item \textbf{Close path (Z): Closes the shape by connecting back to the start.}
                    \bigbreak \noindent 
                    \begin{center}
                        Z
                    \end{center}
                \item \textbf{Quadratic Curve (Q)}: (cx, cy) = control point (x, y) = end point
                    \bigbreak \noindent 
                    \begin{center}
                        Q cx cy x y
                    \end{center}
                    The control point:
                    \begin{itemize}
                        \item pulls the curve
                        \item bends its direction
                        \item determines how steep or shallow the curve is
                    \end{itemize}
                    The curve only passes through:
                    \begin{itemize}
                        \item The start point
                        \item The end point
                    \end{itemize}

                \item \textbf{Cubic Bézier (C)}: 
                    \bigbreak \noindent 
                    \begin{center}
                        C x1 y1, x2  y2, x y
                    \end{center}
                    Two control points, more control.
                \item \textbf{Arc Command (Rounded Shapes) (A)}: 
                    \bigbreak \noindent 
                    \begin{center}
                        A rx ry x-axis-rotation large-arc sweep x y
                    \end{center}
            \end{itemize}
            Note the difference between lowercase and uppercase control characters
            \begin{itemize}
                \item \textbf{Uppercase}:	Absolute position
                \item \textbf{Lowercase}:	Relative movement
            \end{itemize}
        \item \textbf{Triangle with path}: 
            \bigbreak \noindent 
            \begin{htmlcode}
                <path d="M 50 10 L 30 80 L 70 80 Z" />
            \end{htmlcode}
        \item \textbf{Fill and stroke}: 
            \bigbreak \noindent 
            \begin{htmlcode}
                <path d="M 50 10 L 30 80 L 70 80 Z" 
                stroke="black"
                stroke-width="6"
                fill="red"
                />
            \end{htmlcode}
        \item \textbf{Grouping}: The <g> element is used to group multiple SVG elements together so that transformations, styles, or attributes can be applied to them collectively.
            \bigbreak \noindent 
            \begin{htmlcode}
                <svg width="200" height="200">
                    <g>
                        <!-- SVG elements go here -->
                    </g>
                </svg>
            \end{htmlcode}
            \bigbreak \noindent 
            Instead of transforming each element individually, you apply the transformation once to the group.
            \bigbreak \noindent 
            \begin{htmlcode}
                <g transform="translate(50, 50)">
                    <circle cx="0" cy="0" r="20" />
                    <rect x="30" y="-10" width="40" height="20" />
                </g>
            \end{htmlcode}
            \bigbreak \noindent 
            Both shapes move together. You can apply styles such as fill, stroke, opacity, etc., to all child elements.
            \bigbreak \noindent 
            \begin{htmlcode}
                <g fill="blue" stroke="black" stroke-width="2">
                    <circle cx="50" cy="50" r="20" />
                    <rect x="90" y="30" width="40" height="40" />
                </g>
            \end{htmlcode}
            \bigbreak \noindent 
            Events applied to a <g> affect all child elements.
            \bigbreak \noindent 
            \begin{htmlcode}
                <g onclick="alert('Clicked group!')">
                    <circle cx="50" cy="50" r="20" />
                    <rect x="80" y="40" width="30" height="30" />
                </g>
            \end{htmlcode}
    \end{itemize}

    \pagebreak 
    \subsection{JS}
    \begin{itemize}
        \item \textbf{Var, let, and const}:  
            \begin{itemize}
                \item \textbf{Var}: Function-scoped, not block-scoped. Ignores \{\} blocks such as if, for, and while. Hoisted to the top of the function. Initialized as \textbf{undefined}. Can be reassigned ,can be redeclared
                \item \textbf{Let}: Block-scoped, exists only inside \{\} where it is defined. Hoisted, but not initialized. Exists in the Temporal Dead Zone (TDZ) until declared. Can be reassigned ,cannot be redeclared in the same scope
                    \bigbreak \noindent 
                    \begin{jscode}
                        let x = 3;
                        x = 4;     // OK
                        let x = 5; // Error
                    \end{jscode}
                \item \textbf{Const}: Block-scoped, same as let. Hoisted but in the TDZ. Must be initialized at declaration
                    \bigbreak \noindent 
                    \begin{jscode}
                    const z  = 10; // Good
                    const y;       // Error
                    \end{jscode}
                    \bigbreak \noindent 
                    Cannot be reassigned. Const prevents reassignment, not mutation.
                \item \textbf{Immutable types}: These cannot be changed after creation. Any “modification” creates a new value. Primitive Types are all immutable 
                    \begin{itemize}
                        \item number	
                        \item string
                        \item boolean
                        \item null	
                        \item undefined	
                        \item symbol
                        \item bigint
                    \end{itemize}
                    \bigbreak \noindent 
                    \begin{jscode}
                        let s = "hello";
                        s[0] = "H";   // No effect
                        console.log(s); // "hello"
                    \end{jscode}
                \item \textbf{Mutable types}:  These are objects and collections, whose contents can change without changing the reference.
                    \begin{itemize}
                        \item Object	
                        \item Array	
                        \item Function	
                        \item Date	
                        \item Map / Set
                    \end{itemize}
                \item \textbf{Pass by value and pass by reference}: JavaScript does not technically have pass-by-reference.
                    \begin{itemize}
                        \item Primitive values are passed by value
                        \item Objects are passed by value of their reference
                    \end{itemize}
                    All primitives are passed by value. 
                    \bigbreak \noindent 
                    Objects are somewhat passed by reference, technically by value of reference. The reference (memory address) is copied, not the object itself.
                    \bigbreak \noindent 
                    \begin{jscode}
                        function modify(obj) {
                            obj.x = 10;
                        }

                        const data = { x: 1 };
                        modify(data);

                        console.log(data.x); // 10
                    \end{jscode}
                    \begin{itemize}
                        \item data holds a reference to the object
                        \item That reference is copied into obj
                        \item Both point to the same object
                        \item Mutating the object affects both
                    \end{itemize}
                \item \textbf{Objects}: Key value pairs
                    \bigbreak \noindent 
                    \begin{jscode}
                        var obj = {x: 2, y: 4}; obj.x = 3; obj.y = 5;
                    \end{jscode}
                    \bigbreak \noindent 
                    Prototypes for instance functions. We can access properties via dot notation or [] notation. Objects may also contain functions
                    \bigbreak \noindent 
                    \begin{jscode}
                        var student = {firstName: "John",
                            lastName: "Smith",
                            fullName: function() { return this.firstName + " " + this.lastName; }};
                        student.fullName()
                    \end{jscode}
                    \bigbreak \noindent 
                    \textbf{Note:} Dot-notation only works with certain identifiers, bracket notation works with more identifiers, like if the key was a string.
                \item \textbf{JSON}: Data interchange format, subset of JS. Uses nested objects and arrays. Data only, no functions.
                \item \textbf{Functional programming in JS}




            \end{itemize}
    \end{itemize}






















    
\end{document}
