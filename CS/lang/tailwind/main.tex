\documentclass{report}

\input{~/dev/latex/template/preamble.tex}
\input{~/dev/latex/template/macros.tex}

\title{\Huge{}}
\author{\huge{Nathan Warner}}
\date{\huge{}}
\fancyhf{}
\rhead{}
\fancyhead[R]{\itshape Warner} % Left header: Section name
\fancyhead[L]{\itshape\leftmark}  % Right header: Page number
\cfoot{\thepage}
\renewcommand{\headrulewidth}{0pt} % Optional: Removes the header line
%\pagestyle{fancy}
%\fancyhf{}
%\lhead{Warner \thepage}
%\rhead{}
% \lhead{\leftmark}
%\cfoot{\thepage}
%\setborder
% \usepackage[default]{sourcecodepro}
% \usepackage[T1]{fontenc}

% Change the title
\hypersetup{
    pdftitle={Tailwind CSS Framework}
}

\begin{document}
    % \maketitle
        \begin{titlepage}
       \begin{center}
           \vspace*{1cm}
    
           \textbf{Tailwind CSS Framework}
    
           \vspace{0.5cm}
            
                
           \vspace{1.5cm}
    
           \textbf{Nathan Warner}
    
           \vfill
                
                
           \vspace{0.8cm}
         
           \includegraphics[width=0.4\textwidth]{~/niu/seal.png}
                
           Computer Science \\
           Northern Illinois University\\
           United States\\
           
                
       \end{center}
    \end{titlepage}
    \tableofcontents
    \pagebreak 
    \unsect{Getting started}
    \bigbreak \noindent 
    \subsection{Using the CDN (Quick Start)}
    \bigbreak \noindent 
    If you’re just experimenting or making a small demo, create an HTML file and add the following 
    \bigbreak \noindent 
    \begin{htmlcode}
        <script src="https://cdn.tailwindcss.com"></script>
    \end{htmlcode}
    \bigbreak \noindent 
    To <head>... The full html setup might look something like
    \bigbreak \noindent 
    \begin{htmlcode}
    <!DOCTYPE html>
    <html lang="en">
      <head>
        <meta charset="UTF-8" />
        <meta name="viewport" content="width=device-width, initial-scale=1.0" />
        <title>My Tailwind Project</title>
        <!-- Tailwind via CDN (for prototyping only) -->
        <script src="https://cdn.tailwindcss.com"></script>
      </head>
      <body>
        ...
      </body>
    </html>
    \end{htmlcode}
    \bigbreak \noindent 
    This approach quickly gets you started but is limited in customization and performance optimizations.


    \bigbreak \noindent 
    \subsection{Setting Up a Full Project with Tailwind (npm-Based Installation)}
    \bigbreak \noindent 
    For a production-ready project, follow these steps
    \begin{enumerate}
        \item \textbf{Install Node.js and npm}: Make sure you have Node.js installed since npm is bundled with it. You can check installation by running:
            \bigbreak \noindent 
            \begin{bashcode}
            node -v
            npm -v
            \end{bashcode}
        \item \textbf{Initialize Your Project}: In your project directory, initialize a new Node.js project
            \bigbreak \noindent 
            \begin{bashcode}
            npm init -y 
            \end{bashcode}
        \item \textbf{Install Tailwind}:
            \bigbreak \noindent 
            \begin{bashcode}
            npm install tailwindcss @tailwindcss/cli
            \end{bashcode}
        \item \textbf{Import tailwind in your CSS}: Add to your main CSS file
            \bigbreak \noindent 
            \begin{cppcode}
            @import "tailwindcss";
            \end{cppcode}
        \item \textbf{Start the Tailwind CLI build process}:
            \bigbreak \noindent 
            \begin{bashcode}
            npx @tailwindcss/cli -i ./src/input.css -o ./src/output.css --watch
            \end{bashcode}
        \item \textbf{Start using Tailwind in your HTML}
            \bigbreak \noindent 
            \begin{htmlcode}
              <link href="./output.css" rel="stylesheet">
            \end{htmlcode}
    \end{enumerate}

    \bigbreak \noindent 
    \subsection{What is Tailwind?}
    \bigbreak \noindent 
    Tailwind CSS is fundamentally built around a comprehensive collection of CSS utility classes.
    \bigbreak \noindent 
    Utility classes are self-descriptive, single-purpose CSS classes
    \bigbreak \noindent 
    \begin{cppcode}
        .flex {
            display: flex;
        }
    \end{cppcode}
    \bigbreak \noindent 
    Developers use these functional classes to build without writing additional CSS because if the style is in the library, you can use it over and over and over
    \bigbreak \noindent 
    \begin{htmlcode}
    <div class="flex"> </div>
    \end{htmlcode}
    \bigbreak \noindent 
    A more involved example may look something like
    \bigbreak \noindent 
    \begin{csscode}
        * {
            margin: 0;
        }
        .fs {
            height: 100vh;
            width: 100vw;
            margin: 0;
        }

        .bg-pink {
            background-color: pink;
        }

        .flex {
            display: flex;
        }

        .justify-center {
            justify-content: center;
        }

        .align-center {
            align-items: center;
        }

        .box-red {
            width: 20vw;
            height: 20vh;
            background-color: red;
        }
    \end{csscode}
    \bigbreak \noindent 
    Then, in the html document, we can put them to use
    \bigbreak \noindent 
    \begin{htmlcode}
    <!DOCTYPE html>
    <html lang="en">

    <head>
        <meta charset="utf-8">
        <meta http-equiv="X-UA-Compatible" content="IE=edge">
        <meta name="viewport" content="width=device-width, initial-scale=1.0">
        <title>test2</title>
        <link rel="stylesheet" href="styles.css">
        <link href='https://unpkg.com/boxicons@2.1.4/css/boxicons.min.css' rel='stylesheet'>
        <!-- <meta http-equiv="refresh" content="1"> -->
    </head>

    <body>
        <div class="fs flex justify-center align-center bg-pink"> 
            <div class="box-red"> </div>
        </div>
    </body>

    </html>
    \end{htmlcode}
    \bigbreak \noindent 
    \fig{.5}{./figures/1.png}

    \pagebreak 
    \unsect{Variants and plugins}
    \bigbreak \noindent 
    Variants are modifiers that allow your utility classes to respond to different states or conditions. Essentially, they generate alternative versions of your utilities that apply under specific circumstances, such as:
    \begin{itemize}
        \item \textbf{State-based conditions:} e.g., hover, focus, active, disabled
        \item \textbf{Responsive breakpoints:} e.g., sm:, md:, lg:, xl:
        \item \textbf{Contextual conditions:} e.g., group-hover, first, last, odd, even
        \item \textbf{Dark mode:} e.g., dark:
    \end{itemize}
    \bigbreak \noindent 
    For example, writing hover:bg-blue-500 tells Tailwind to apply the blue background color only when the user hovers over the element. This system eliminates the need for writing custom CSS pseudo-class selectors by hand.

    \pagebreak 
    \unsect{Plugins}
    \bigbreak \noindent 
    Plugins are modules that extend the core functionality of Tailwind CSS. They let you add new utility classes, components, or even entirely new variant generators without modifying the Tailwind core. Plugins are especially powerful when you need to enforce a design pattern or add a set of repeated, custom styles across projects.

    \pagebreak 
    \unsect{Media queries (Responsive design)}
    \bigbreak \noindent 
    There are five breakpoints by default, inspired by common device resolutions:
    \bigbreak \noindent 
    \begin{center}
        \begin{tabular}{c|c|c}
            Breakpoint prefix	&Minimum width	&CSS \\
            \hline
            sm	&640px	&@media (min-width: 640px) { ... } \\
            md	&768px	&@media (min-width: 768px) { ... } \\
            lg	&1024px	&@media (min-width: 1024px) { ... }\\
            xl	&1280px	&@media (min-width: 1280px) { ... }\\
            2xl	&1536px	&@media (min-width: 1536px) { ... }
        \end{tabular}
    \end{center}
    \bigbreak \noindent 
    To add a utility but only have it take effect at a certain breakpoint, all you need to do is prefix the utility with the breakpoint name, followed by the : character:
    \bigbreak \noindent 
    \begin{htmlcode}
        <!-- Width of 16 by default, 32 on medium screens, and 48 on large screens -->
        <img class="w-16 md:w-32 lg:w-48" src="...">
    \end{htmlcode}
    \bigbreak \noindent 
    This works for every utility class in the framework, which means you can change literally anything at a given breakpoint — even things like letter spacing or cursor styles.
    \bigbreak \noindent 
    \subsection{Mobile first}
    \bigbreak \noindent 
    By default, Tailwind uses a mobile first breakpoint system, similar to what you might be used to in other frameworks like Bootstrap.
    \bigbreak \noindent 
    Where this approach surprises people most often is that to style something for mobile, you need to use the unprefixed version of a utility, not the sm: prefixed version. Don’t think of sm: as meaning “on small screens”, think of it as “at the small breakpoint“.
    \bigbreak \noindent 
    For this reason, it’s often a good idea to implement the mobile layout for a design first, then layer on any changes that make sense for sm screens, followed by md screens, etc.
    \bigbreak \noindent 
    Tailwind’s breakpoints only include a min-width and don’t include a max-width, which means any utilities you add at a smaller breakpoint will also be applied at larger breakpoints.
    \bigbreak \noindent 
    If you’d like to apply a utility at one breakpoint only, the solution is to undo that utility at larger sizes by adding another utility that counteracts it.
    \bigbreak \noindent 
    Here is an example where the background color is red at the md breakpoint, but green at every other breakpoint
    \bigbreak \noindent 
    \begin{htmlcode}
        <div class="bg-green-500 md:bg-red-500 lg:bg-green-500">
          <!-- ... -->
        </div>
    \end{htmlcode}
    \bigbreak \noindent 
    You can completely customize your breakpoints in your tailwind.config.js file:
    \bigbreak \noindent 
    \begin{cppcode}
 // tailwind.config.js
module.exports = {
  theme: {
    screens: {
      'tablet': '640px',
      // => @media (min-width: 640px) { ... }

      'laptop': '1024px',
      // => @media (min-width: 1024px) { ... }

      'desktop': '1280px',
      // => @media (min-width: 1280px) { ... }
    },
  }
}   
    \end{cppcode}

    \pagebreak 
    \unsect{Hover, Focus, and Other States}
    \bigbreak \noindent 
    Not all state variants are enabled for all utilities by default due to file-size considerations, but we’ve tried our best to enable the most commonly used combinations out of the box.
    \bigbreak \noindent 
    \subsection{Hover}
    \bigbreak \noindent 
    Add the hover: prefix to only apply a utility on hover.
    \bigbreak \noindent 
    \begin{htmlcode}
        <button class="bg-red-500 hover:bg-red-700 ...">
            Hover me
        </button>
    \end{htmlcode}
    \bigbreak \noindent 
    By default, the hover variant is enabled for the following core plugins:
    \begin{itemize}
        \item backgroundColor
        \item backgroundOpacity
        \item borderColor
        \item borderOpacity
        \item boxShadow
        \item gradientColorStops
        \item opacity
        \item rotate
        \item scale
        \item skew
        \item textColor
        \item textDecoration
        \item textOpacity
        \item translate
    \end{itemize}
    \bigbreak \noindent 
    You can control whether hover variants are enabled for a plugin in the variants section of your tailwind.config.js file:
    \bigbreak \noindent 
    \begin{cppcode}
        // tailwind.config.js
        module.exports = {
            // ...
            variants: {
                extend: {
                    padding: ['hover'],
                }
            },
        }
    \end{cppcode}

    \bigbreak \noindent 
    \subsection{Focus}
    \bigbreak \noindent 
    Add the focus: prefix to only apply a utility on focus.
    \bigbreak \noindent 
    \begin{htmlcode}
    <input class="focus:ring-2 focus:ring-blue-600 ...">
    \end{htmlcode}
    \bigbreak \noindent 
    By default, the focus variant is enabled for the following core plugins:
    \begin{itemize}
        \item accessibility
        \item backgroundColor
        \item backgroundOpacity
        \item borderColor
        \item borderOpacity
        \item boxShadow
        \item gradientColorStops
        \item opacity
        \item outline
        \item placeholderColor
        \item placeholderOpacity
        \item ringColor
        \item ringOffsetColor
        \item ringOffsetWidth
        \item ringOpacity
        \item ringWidth
        \item rotate
        \item scale
        \item skew
        \item textColor
        \item textDecoration
        \item textOpacity
        \item translate
        \item zIndex
    \end{itemize}
    \bigbreak \noindent 
    You can control whether focus variants are enabled for a plugin in the variants section of your tailwind.config.js file:
    \bigbreak \noindent 
    \begin{cppcode}
        // tailwind.config.js
        module.exports = {
            // ...
            variants: {
                extend: {
                    maxHeight: ['focus'],
                }
            },
        }
    \end{cppcode}
    \bigbreak \noindent 
    \subsection{Active}
    Add the active: prefix to only apply a utility when an element is active.
    \bigbreak \noindent 
    \begin{htmlcode}
        <button class="bg-green-500 active:bg-green-700 ...">
            Click me
        </button>
    \end{htmlcode}
    \bigbreak \noindent 
    By default, the active variant is not enabled for any core plugins.
    \bigbreak \noindent 
    You can control whether active variants are enabled for a plugin in the variants section of your tailwind.config.js file:
    \bigbreak \noindent 
    \subsection{Visited}
    \bigbreak \noindent 
    Add the visited: prefix to only apply a utility when a link has been visited.
    \bigbreak \noindent 
    \begin{htmlcode}
    <a href="#" class="text-blue-600 visited:text-purple-600 ...">Link</a>
    \end{htmlcode}
    \bigbreak \noindent 
    By default, the visited variant is not enabled for any core plugins.
    \bigbreak \noindent 
    You can control whether visited variants are enabled for a plugin in the variants section of your tailwind.config.js file:
    \bigbreak \noindent 
    \subsection{Disabled}
    \bigbreak \noindent 
    Add the disabled: prefix to only apply a utility when an element is disabled.
    \bigbreak \noindent 
    \begin{htmlcode}
        <button class="disabled:opacity-50 ...">
            Submit
        </button>
        <button class="disabled:opacity-50 ..." disabled>
            Submit
        </button>
    \end{htmlcode}
    \bigbreak \noindent 
    By default, the disabled variant is not enabled for any core plugins.
    \bigbreak \noindent 
    You can control whether disabled variants are enabled for a plugin in the variants section of your tailwind.config.js file:
    \bigbreak \noindent 
    \subsection{Checked}
    \bigbreak \noindent 
    Add the checked: prefix to only apply a utility when a radio or checkbox is currently checked.
    \bigbreak \noindent 
    \begin{htmlcode}
        <input type="checkbox" class="appearance-none checked:bg-blue-600 checked:border-transparent ...">
    \end{htmlcode}
    \bigbreak \noindent 
    By default, the checked variant is not enabled for any core plugins.
    \bigbreak \noindent 
    You can control whether checked variants are enabled for a plugin in the variants section of your tailwind.config.js file:

    \bigbreak \noindent 
    \subsection{Dark Mode}
    \bigbreak \noindent 
    Now that dark mode is a first-class feature of many operating systems, it’s becoming more and more common to design a dark version of your website to go along with the default design.
    \bigbreak \noindent 
    To make this as easy as possible, Tailwind includes a dark variant that lets you style your site differently when dark mode is enabled:

    \bigbreak \noindent 
    \begin{htmlcode}
<div class="bg-white dark:bg-gray-800">
  <h1 class="text-gray-900 dark:text-white">Dark mode is here!</h1>
  <p class="text-gray-600 dark:text-gray-300">
    Lorem ipsum...
  </p>
</div>
    \end{htmlcode}
    \bigbreak \noindent 
    It’s important to note that because of file size considerations, the dark mode variant is not enabled in Tailwind by default.
    \bigbreak \noindent 
    To enable it, set the darkMode option in your tailwind.config.js file to media:
    \bigbreak \noindent 
    \begin{cppcode}
        // tailwind.config.js
        module.exports = {
            darkMode: 'media',
            // ...
        }
    \end{cppcode}
    \bigbreak \noindent 
    By default, when darkMode is enabled dark variants are only generated for color-related classes, which includes text color, background color, border color, gradients, and placeholder color.

    \pagebreak 
    \unsect{Container}
    \bigbreak \noindent 
    The container class sets the max-width of an element to match the min-width of the current breakpoint. This is useful if you’d prefer to design for a fixed set of screen sizes instead of trying to accommodate a fully fluid viewport.
    \bigbreak \noindent 
    To center a container, use the mx-auto utility:
    \bigbreak \noindent 
    \begin{htmlcode}
        <div class="container mx-auto">
            <!-- ... -->
        </div>
    \end{htmlcode}
    \bigbreak \noindent 
    To add horizontal padding, use the px-\{size\} utilities:
    \bigbreak \noindent 
    \begin{htmlcode}
    <div class="container mx-auto px-4">
      <!-- ... -->
    </div>
    \end{htmlcode}

    \pagebreak 
    \unsect{Display}
    \bigbreak \noindent 
    \begin{center}
        \begin{tabular}{c|c}
            Class &Properties \\
            block	&display: block;\\
            inline-block	&display: inline-block;\\
            inline	&display: inline;\\
            flex	&display: flex;\\
            inline-flex	&display: inline-flex;\\
            table	&display: table;\\
            inline-table	&display: inline-table;\\
            table-caption	&display: table-caption;\\
            table-cell	&display: table-cell;\\
            table-column	&display: table-column;\\
            table-column-group	&display: table-column-group;\\
            table-footer-group	&display: table-footer-group;\\
            table-header-group	&display: table-header-group;\\
            table-row-group	&display: table-row-group;\\
            table-row	&display: table-row;\\
            flow-root	&display: flow-root;\\
            grid	&display: grid;\\
            inline-grid	&display: inline-grid;\\
            contents	&display: contents;\\
            list-item	&display: list-item;\\
            hidden	&display: none;
        \end{tabular}
    \end{center}

    \pagebreak 
    \unsect{Floats}
    \bigbreak \noindent 
    Utilities for controlling the wrapping of content around an element.
    \bigbreak \noindent 
    \begin{itemize}
        \item Use float-right to float an element to the right of its container.
        \item Use float-left to float an element to the left of its container.
        \item Use float-none to reset any floats that are applied to an element. This is the default value for the float property.
    \end{itemize}

    \pagebreak 
    \unsect{Object Fit}
    \bigbreak \noindent 
    \begin{center}
        \begin{tabular}{c|c}
            Class &Properties \\
            \hline
            object-contain	&object-fit: contain;\\
            object-cover	&object-fit: cover;\\
            object-fill	&object-fit: fill;\\
            object-none	&object-fit: none;\\
            object-scale-down	&object-fit: scale-down;
        \end{tabular}
    \end{center}


    \pagebreak 
    \unsect{Object Position}
    \begin{center}
        \begin{tabular}{c|c}
            Class &Properties\\
            \hline
            object-bottom	&object-position: bottom;\\
            object-center	&object-position: center;\\
            object-left	&object-position: left;\\
            object-left-bottom	&object-position: left bottom;\\
            object-left-top	&object-position: left top;\\
            object-right	&object-position: right;\\
            object-right-bottom	&object-position: right bottom;\\
            object-right-top	&object-position: right top;\\
            object-top	&object-position: top;
        \end{tabular}
    \end{center}

    \pagebreak 
    \unsect{Overflow}
    \begin{center}
        \begin{tabular}{c|c}
            Class &Properties\\
            \hline
            overflow-auto	&overflow: auto;\\
            overflow-hidden	&overflow: hidden;\\
            overflow-visible	&overflow: visible;\\
            overflow-scroll	&overflow: scroll;\\
            overflow-x-auto	&overflow-x: auto;\\
            overflow-y-auto	&overflow-y: auto;\\
            overflow-x-hidden	&overflow-x: hidden;\\
            overflow-y-hidden	&overflow-y: hidden;\\
            overflow-x-visible	&overflow-x: visible;\\
            overflow-y-visible	&overflow-y: visible;\\
            overflow-x-scroll	&overflow-x: scroll;\\
            overflow-y-scroll	&overflow-y: scroll;
        \end{tabular}
    \end{center}

    \pagebreak 
    \unsect{Position}
    \begin{center}
        \begin{tabular}{c|c}
            Class &Properties \\
            \hline
            static	&position: static; \\
            fixed	&position: fixed; \\
            absolute	&position: absolute; \\
            relative	&position: relative; \\
            sticky	&position: sticky;
        \end{tabular}
    \end{center}

    
    \pagebreak 
    \unsect{Visibility}
    \begin{center}
        \begin{tabular}{c|c}
            Class & Properties\\
            \hline
            visible	&visibility: visible; \\
            invisible	&visibility: hidden;
        \end{tabular}
    \end{center}

    \pagebreak 
    \unsect{Z-Index}
    \begin{center}
        \begin{tabular}{c|c}
            Class &Properties \\ 
            \hline
            z-0	&z-index: 0;\\
            z-10	&z-index: 10;\\
            z-20	&z-index: 20;\\
            z-30	&z-index: 30;\\
            z-40	&z-index: 40;\\
            z-50	&z-index: 50;\\
            z-auto	&z-index: auto;
        \end{tabular}
    \end{center}


















\end{document}
