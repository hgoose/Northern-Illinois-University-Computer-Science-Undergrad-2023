\documentclass{report}

\input{~/dev/latex/template/preamble.tex}
\input{~/dev/latex/template/macros.tex}

\title{\Huge{}}
\author{\huge{Nathan Warner}}
\date{\huge{}}
\fancyhf{}
\rhead{}
\fancyhead[R]{\itshape Warner} % Left header: Section name
\fancyhead[L]{\itshape\leftmark}  % Right header: Page number
\cfoot{\thepage}
\renewcommand{\headrulewidth}{0pt} % Optional: Removes the header line
%\pagestyle{fancy}
%\fancyhf{}
%\lhead{Warner \thepage}
%\rhead{}
% \lhead{\leftmark}
%\cfoot{\thepage}
%\setborder
% \usepackage[default]{sourcecodepro}
% \usepackage[T1]{fontenc}

% Change the title
\hypersetup{
    pdftitle={LaTeX}
}

\begin{document}
    % \maketitle
        \begin{titlepage}
       \begin{center}
           \vspace*{1cm}
    
           \textbf{LaTeX}
    
           \vspace{0.5cm}
            
                
           \vspace{1.5cm}
    
           \textbf{Nathan Warner}
    
           \vfill
                
                
           \vspace{0.8cm}
         
           \includegraphics[width=0.4\textwidth]{~/niu/seal.png}
                
           Computer Science \\
           Northern Illinois University\\
           United States\\
           
                
       \end{center}
    \end{titlepage}
    \tableofcontents
    \pagebreak 
    \unsect{Tabularx}
    \begin{itemize}
        \item \textbf{Standard column specifiers (from tabular)}: These work in any tabular or tabularx environment:
            \begin{center}
                \begin{tabularx}{\textwidth}{@{}lllX@{}}
                    \toprule
                    Specifier &Alignment &Text wrapping &Notes \\
                    \midrule
                    l &	Left-aligned	 &No	&Each cell fits content width; text doesn’t wrap. \\[2ex]
                    c&	Centered	 &No	&Centered horizontally; fixed to content width.\\[2ex]
                    r	&Right-aligned	 &No	&Good for numbers.\\[2ex]
                    p\{width\}	&Paragraph column&	 Yes	&Fixed width; text wraps automatically.\\[2ex]
                    m\{width\}	&Paragraph column (vertically centered)	& Yes	&Requires array package.\\[2ex]
                    b\{width\}	&Paragraph column (bottom-aligned)	 &Yes	&Requires array package. \\
                    \bottomrule
                \end{tabularx}
            \end{center}
        \item \textbf{tabularx-specific specifiers}: The tabularx package adds the magic column type:
            \begin{center}
                \begin{tabularx}{\textwidth}{@{}lXlX@{}}
                    \toprule
                    Specifier	&Alignment	&Text wrapping	&Description \\
                    \midrule
                    X	&Justified (like p\{\})	& Yes	&Automatically expands or shrinks to make the table fit the specified total width. \\
                    \bottomrule
                \end{tabularx}
            \end{center}
            You can think of X as a “flexible” version of p{}. The table width (e.g., \textbackslash textwidth) is distributed among all X columns proportionally.
            \bigbreak \noindent 
            \begin{texcode}
                \begin{tabularx}{\textwidth}{l X l}
            \end{texcode}
        \item \textbf{Modifiers}:
            \begin{center}
                \begin{tabularx}{\textwidth}{@{}XXX@{}}
                    \toprule
                    \textbf{Syntax} &	\textbf{Meaning} &	\textbf{Example} \\
                    \midrule
                    @\{text\}	&Inserts custom text or spacing between columns	& @\{\textbackslash hspace\{1em\}\} or @\{--\}\\[2ex]
                    @\{\}	&Removes default inter-column space	&@\{\}lXl@\{\}\\[2ex]
                    >\{declaration\}	&Applies a LaTeX command at the start of each cell in that column	&>\{\textbackslash centering\textbackslash arraybackslash\}p\{3cm\}\\[2ex]
                    <\{declaration\}	&Applies a command at the end of each cell	&<\{\textbackslash hfill\} \\
                    \bottomrule
                \end{tabularx}
            \end{center}
            
    \end{itemize}

    \pagebreak 
    \unsect{Booktabs}
    \begin{itemize}
        \item \textbf{Booktabs}: booktabs is a LaTeX package that improves table appearance by providing better horizontal rules (lines) and proper spacing — making tables look like those in professional books and journals (hence the name booktabs).
            \bigbreak \noindent 
            It replaces the default \textbackslash hline-based style with commands that give you clean, typographically correct tables.
            \begin{center}
                \begin{tabularx}{\textwidth}{@{}lXl@{}}
                    \toprule
                    \textbf{Command} & \textbf{Purpose} & \textbf{Typical Use} \\
                    \midrule
                    \textbackslash toprule	&Thick top line	&Start of table \\[2ex]
                    \textbackslash midrule	&Medium line	&Between header and body\\[2ex]
                    \textbackslash bottomrule	& Thick bottom line	& End of table\\[2ex]
                    \bottomrule
                \end{tabularx}
            \end{center}
            Adds vertical whitespace around rules so text doesn’t feel cramped. he booktabs manual explicitly advises not to use | — clean design relies on spacing, not lines.
        \item \textbf{Optional extras}
            \begin{itemize}
                \item \textbackslash haddlinespace - adds extra vertical space between rows.
                \item \textbackslash cmidrule(lr){1-2} - draws a partial horizontal line (useful for subheaders).
            \end{itemize}
             
    \end{itemize}



\end{document}
