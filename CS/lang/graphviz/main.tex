\documentclass{report}

\input{~/dev/latex/template/preamble.tex}
\input{~/dev/latex/template/macros.tex}

\title{\Huge{}}
\author{\huge{Nathan Warner}}
\date{\huge{}}
\fancyhf{}
\rhead{}
\fancyhead[R]{\itshape Warner} % Left header: Section name
\fancyhead[L]{\itshape\leftmark}  % Right header: Page number
\cfoot{\thepage}
\renewcommand{\headrulewidth}{0pt} % Optional: Removes the header line
%\pagestyle{fancy}
%\fancyhf{}
%\lhead{Warner \thepage}
%\rhead{}
% \lhead{\leftmark}
%\cfoot{\thepage}
%\setborder
% \usepackage[default]{sourcecodepro}
% \usepackage[T1]{fontenc}

% Change the title
\hypersetup{
    pdftitle={Graphviz}
}

\begin{document}
    % \maketitle
        \begin{titlepage}
       \begin{center}
           \vspace*{1cm}
    
           \textbf{Graphviz}
    
           \vspace{0.5cm}
            
                
           \vspace{1.5cm}
    
           \textbf{Nathan Warner}
    
           \vfill
                
                
           \vspace{0.8cm}
         
           \includegraphics[width=0.4\textwidth]{~/niu/seal.png}
                
           Computer Science \\
           Northern Illinois University\\
           United States\\
           
                
       \end{center}
    \end{titlepage}
    \tableofcontents
    \pagebreak 
    \unsect{Getting started}
    \bigbreak \noindent 
    Install graphiz with your package manager
    \bigbreak \noindent 
    \begin{bashcode}
    sudo pacman -Sy graphviz
    \end{bashcode}
    \bigbreak \noindent 
    Create a graphviz file with the .dot extension, generate pngs from your graphs with
    \bigbreak \noindent 
    \begin{bashcode}
    dot -Tpng mygraph.dot -o outputname.png
    \end{bashcode}

    \pagebreak 
    \unsect{Directed and undirected graphs}
    \bigbreak \noindent 
    \subsection{Directed graphs}
    \bigbreak \noindent 
    Create directed graphs with the following syntax
    \bigbreak \noindent 
    \begin{cppcode}
        graph G {
            ...
        }
    \end{cppcode}
    \bigbreak \noindent 
    Where $G$ is the name of the graph
    \bigbreak \noindent 
    Create an edge connecting two nodes with
    \bigbreak \noindent 
    \begin{cppcode}
        graph G {
            A -- B
        }
    \end{cppcode}
    \bigbreak \noindent 
    Connect a vertex to multiple vertices with
    \bigbreak \noindent 
    \begin{cppcode}
        graph G {
            A -- {B1, B2, B3}
        }
    \end{cppcode}

    \pagebreak 
    \subsection{Digraphs}
    \bigbreak \noindent 
    Create digraphs with
    \bigbreak \noindent 
    \begin{cppcode}
        digraph G {
            ...
        }
    \end{cppcode}
    \bigbreak \noindent 
    Connect vertices with
    \bigbreak \noindent 
    \begin{cppcode}
        digraph G {
            A -> B
            B -> {C,D}
        }
    \end{cppcode}

    \pagebreak 
    \unsect{Styles}
    \bigbreak \noindent 
    \subsection{Node styles}
    \bigbreak \noindent 
    Define node styles with 
    \bigbreak \noindent 
    \begin{cppcode}
        graph G {
            node [ key=value, ... ]
        }
    \end{cppcode}
    \bigbreak \noindent 
    For example, 
    \bigbreak \noindent 
    \begin{cppcode}
        graph G {
            node [ style=filled, color=red ]
        }
    \end{cppcode}

    \bigbreak \noindent 
    \subsubsection{Shape}
    \bigbreak \noindent 
    Change the shape of the nodes with
    \bigbreak \noindent 
    \begin{cppcode}
        graph G {
            node [ shape= ]
        }
    \end{cppcode}
    \bigbreak \noindent 
    Options for shape include
    \begin{itemize}
        \item Ellipse
        \item Circle
        \item Box
        \item Diamond
    \end{itemize}

    \bigbreak \noindent 
    \subsubsection{Label}
    \bigbreak \noindent 
    Specifies the text displayed within the node.
    \bigbreak \noindent 
    \begin{cppcode}
        graph G {
            node [ label="..." ]
        }
    \end{cppcode}

    \bigbreak \noindent 
    \subsubsection{Color}
    \bigbreak \noindent 
    Sets the color of the node's border.
    \bigbreak \noindent 
    \begin{cppcode}
    graph G {
        node [ color=yellow ]
    } 
    \end{cppcode}
    \bigbreak \noindent 
    \subsubsection{fillcolor}
    \bigbreak \noindent 
    Determines the interior fill color (effective when combined with a style such as filled).
    \bigbreak \noindent 
    \begin{cppcode}
        graph G {
            node [ style=filled, fillcolor=red ]
        }
    \end{cppcode}

    \bigbreak \noindent 
    \subsubsection{Style}
    \bigbreak \noindent 
    Alters the appearance (e.g., filled, dashed, rounded).
    \bigbreak \noindent 
    \begin{cppcode}
        graph G {
            node [ style=dashed ]
        }
    \end{cppcode}

    \bigbreak \noindent 
    \subsubsection{fontname, fontsize, fontcolor} 
    \bigbreak \noindent 
    Customize the font type, size, and color for the label.}

    \bigbreak \noindent 
    \subsubsection{width and height}
    \bigbreak \noindent 
    Control the dimensions of the node.}


    \bigbreak \noindent 
    \subsubsection{Penwidth}
    \bigbreak \noindent 
    Adjusts the thickness of the node border.

    \bigbreak \noindent 
    \subsubsection{margin}
    \bigbreak \noindent 
     Provides spacing between the label and the node border.

     \bigbreak \noindent 
     \subsection{Styling specific nodes}
     \bigbreak \noindent 
     In Graphviz, you can override the default node attributes by specifying attributes directly for that node. For example, if you want most nodes to have one style but a specific node to have a different style, you can define the default attributes at the top and then override them for the specific node.
     \bigbreak \noindent 
     \begin{cppcode}
         digraph Example {
             // Global node attributes
             node [shape=box, style=filled, fillcolor=lightblue];

             // Default nodes
             A;
             B;
             C;

             // Override node B's style
             B [fillcolor=yellow, style=rounded];

             // Create some connections
             A -> B;
             B -> C;
         }
     \end{cppcode}

     \bigbreak \noindent 
     \subsection{Edges}
     \bigbreak \noindent 
     \begin{itemize}
         \item \textbf{label:} Sets a text label on the edge.
         \item \textbf{color:} Specifies the color of the edge line.
         \item \textbf{style:} Defines the style of the edge (e.g., solid, dashed, dotted, bold, invis).
         \item \textbf{penwidth:} Determines the thickness of the edge line.
         \item \textbf{arrowhead:} Sets the style of the arrow at the head end (e.g., normal, vee, dot, none).
         \item \textbf{arrowtail:} Sets the style of the arrow at the tail end.
         \item \textbf{dir:} Controls the direction of the arrow; can be forward, back, both, or none.
         \item \textbf{fontsize:} Specifies the size of the font used for the label.
         \item \textbf{fontname:} Sets the font used for the label text.
         \item \textbf{fontcolor:} Specifies the color of the label text.
         \item \textbf{weight:} Provides a relative weight that influences the layout; heavier edges tend to pull nodes closer.
         \item \textbf{constraint:} A boolean that, when true, makes the edge influence the layout positioning.
         \item \textbf{headlabel / taillabel:} Allows you to add labels near the head or tail of the edge.
         \item \textbf{headport / tailport:} Specify which part of the node boundary the edge should connect to.
     \end{itemize}

     \bigbreak \noindent 
     \subsection{Layouts}
     \bigbreak \noindent 
     Graphviz offers several layout engines, each suited for different types of graphs and visual styles
     \begin{itemize}
         \item \textbf{dot (default):} Uses a hierarchical (layered) approach, making it ideal for directed graphs and flowcharts.
         \item \textbf{neato:} Implements a spring model (force-directed) layout suitable for undirected graphs.
         \item \textbf{fdp:} Similar to neato but uses a different force-directed algorithm; good for undirected graphs.
         \item \textbf{sfdp:} A scalable force-directed layout engine designed for larger graphs.
         \item \textbf{twopi:} Produces radial layouts by positioning nodes in concentric circles around a central point.
         \item \textbf{circo:} Generates circular layouts, particularly effective for cyclic graphs.
     \end{itemize}

     \bigbreak \noindent 
     \begin{cppcode}
         graph G {
             layout=neato
         }
     \end{cppcode}






\end{document}
