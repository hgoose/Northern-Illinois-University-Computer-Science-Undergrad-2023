\documentclass{report}

\input{~/dev/latex/template/preamble.tex}
\input{~/dev/latex/template/macros.tex}

\title{\Huge{}}
\author{\huge{Nathan Warner}}
\date{\huge{}}
\fancyhf{}
\rhead{}
\fancyhead[R]{\itshape Warner} % Left header: Section name
\fancyhead[L]{\itshape\leftmark}  % Right header: Page number
\cfoot{\thepage}
\renewcommand{\headrulewidth}{0pt} % Optional: Removes the header line
%\pagestyle{fancy}
%\fancyhf{}
%\lhead{Warner \thepage}
%\rhead{}
% \lhead{\leftmark}
%\cfoot{\thepage}
%\setborder
% \usepackage[default]{sourcecodepro}
% \usepackage[T1]{fontenc}

% Change the title
\hypersetup{
    pdftitle={Java programming}
}

\begin{document}
    % \maketitle
        \begin{titlepage}
       \begin{center}
           \vspace*{1cm}
    
           \textbf{Java programming}
    
           \vspace{0.5cm}
            
                
           \vspace{1.5cm}
    
           \textbf{Nathan Warner}
    
           \vfill
                
                
           \vspace{0.8cm}
         
           \includegraphics[width=0.4\textwidth]{~/niu/seal.png}
                
           Computer Science \\
           Northern Illinois University\\
           United States\\
           
                
       \end{center}
    \end{titlepage}
    \tableofcontents
    \pagebreak 
    \unsect{User Input (scanner)}
    \bigbreak \noindent 
    The Scanner class is used to get user input, and it is found in the \textbf{java.util} package.
    \bigbreak \noindent 
    \begin{javacode}
        import java.util.Scanner;  // Import the Scanner class

        class Main {
            public static void main(String[] args) {
                Scanner myObj = new Scanner(System.in);  // Create a Scanner object
                System.out.println("Enter username");

                String userName = myObj.nextLine();  // Read user input
                System.out.println("Username is: " + userName);  // Output user input
            }
        }
    \end{javacode}

    \bigbreak \noindent 
    \subsection{Input Types}
    \begin{itemize}
        \item \textbf{nextBoolean()}	Reads a boolean value from the user
        \item \textbf{nextByte()}	Reads a byte value from the user
        \item \textbf{nextDouble()}	Reads a double value from the user
        \item \textbf{nextFloat()}	Reads a float value from the user
        \item \textbf{nextInt()}	Reads a int value from the user
        \item \textbf{nextLine()}	Reads a String value from the user
        \item \textbf{nextLong()}	Reads a long value from the user
        \item \textbf{nextShort()}	Reads a short value from the user
    \end{itemize}

    \bigbreak \noindent 
    \subsection{Checks}
    \begin{itemize}
        \item \textbf{hasNextBoolean()}	
        \item \textbf{hasNextByte()}
        \item \textbf{hasNextDouble()}
        \item \textbf{hasNextFloat()}	
        \item \textbf{hasNextInt()}	
        \item \textbf{hasNextLine()}	
        \item \textbf{hasNextLong()}
        \item \textbf{hasNextShort()}	
    \end{itemize}

    \pagebreak 
    \unsect{Arrays}
    \bigbreak \noindent 
    \subsection{Important methods}
    \bigbreak \noindent 
    These static methods are found in \texttt{java.util.Arrays}
    \bigbreak \noindent 
    \begin{itemize}
        \item \textbf{Arrays.fill()}: Fills all elements of the specified array with the specified value.
        \item \textbf{Arrays.equals()}: Returns a Boolean true value if both arrays are of the same type and all of the elements within the arrays are equal to each other. 
        \item \textbf{Arrays.copyOf()}: Copies the specified array, truncating or padding with default values if necessary so the copy has the specified length.
        \item \textbf{Arrays.copyOfRange()}: Copies the specified range from the index1 element up to, but not including, the index2 element of the specified array into a new array
        \item \textbf{Arrays.sort()}
        \item \textbf{Arrays.binarySearch}
    \end{itemize}

    \bigbreak \noindent 
    \subsection{Sorting}
    \bigbreak \noindent 
    \subsection{The Comparable Interface}
    \bigbreak \noindent 
    In Java, the \texttt{Comparable<T>} interface (in java.lang) lets a class define its natural ordering by implementing a single method:
    \bigbreak \noindent 
    \begin{javacode}
    public interface Comparable<T> {
        int compareTo(T other);
    }
    \end{javacode}
    \bigbreak \noindent 
    Enables objects to be sorted (e.g. by Collections.sort() or Arrays.sort()), or used in sorted collections (e.g. TreeSet, TreeMap).
    \bigbreak \noindent 
    \textbf{Contract}:
    \begin{itemize}
        \item \textbf{this.compareTo(other) < 0} means this precedes other
        \item \textbf{== 0} means they’re considered equal in ordering
        \item \textbf{> 0} means this follows other
    \end{itemize}

    \pagebreak \bigbreak \noindent 
    \begin{javacode}
        import java.util.Scanner;
        import java.util.Collections;
        import java.util.ArrayList;
        import java.util.List;

        public class t1 implements Comparable<t1> {
            public int x,y;

            public t1(int x, int y) { this.x = x; this.y = y; }

            @Override
            // Ascending
            public int compareTo(t1 other) {
                if (this.x == other.x) return 0; 
                else if (this.x > other.x) return 1;
                else return -1;
            }

            @Override
            // Descending
            public int compareTo(t1 other) {
                if (this.x == other.x) return 0; 
                else if (this.x > other.x) return -1;
                else return 1;
            }

            public static void main(String[] args) {
                ArrayList<t1> arr = new ArrayList<>(List.of(new t1(4,2), new t1(2,6), new t1(1,8), new t1(9,18), new t1(5,0)));

                Collections.sort(arr);

                for (t1 item : arr) {
                    System.out.println("(" + item.x + "," + item.y + ")");
                }
            }
        }
    \end{javacode}

    \pagebreak 
    \subsection{Comparator}
    \bigbreak \noindent 
    The Comparator<T> interface (in java.util) defines a custom ordering for objects—even if the class itself doesn’t implement Comparable. It has one primary method
    \bigbreak \noindent 
    \begin{javacode}
        public interface Comparator<T> {
            /**
            * Compares its two arguments for order.
            *
            * @param o1 the first object to be compared.
            * @param o2 the second object to be compared.
            * @return a negative integer if o1 <  o2,
            *         zero                if o1 == o2,
            *         a positive integer if o1 >  o2.
            */
            int compare(T o1, T o2);

        }
    \end{javacode}
    \bigbreak \noindent 
    We can use it to define an ordering for objects without implementing the Comparable interface
    \bigbreak \noindent 
    \begin{javacode}
        import java.util.Collections;
        import java.util.Comparator;
        import java.util.ArrayList;
        import java.util.List;

        public class t1 {
            public int x, y;

            public t1(int x, int y) {
                this.x = x;
                this.y = y;
            }

            public static void main(String[] args) {
                List<t1> arr = new ArrayList<>(List.of( new t1(4,2), new t1(2,6), new t1(1,8), new t1(9,18), new t1(5,0)));

                // 1) Create a Comparator that compares by x:
                Comparator<t1> byX = new Comparator<>() {
                    @Override
                    public int compare(t1 a, t1 b) {
                        // Integer.compare handles a.x < b.x, ==, >
                        return Integer.compare(a.x, b.x);
                    }
                };

                // 2) Sort using that Comparator:
                Collections.sort(arr, byX);

                // 3) Print out:
                for (t1 item : arr) {
                    System.out.println("(" + item.x + "," + item.y + ")");
                }
            }
        }
    \end{javacode}




\end{document}
