\documentclass{report}

\input{~/dev/latex/template/preamble.tex}
\input{~/dev/latex/template/macros.tex}

\title{\Huge{}}
\author{\huge{Nathan Warner}}
\date{\huge{}}
\fancyhf{}
\rhead{}
\fancyhead[R]{\itshape Warner} % Left header: Section name
\fancyhead[L]{\itshape\leftmark}  % Right header: Page number
\cfoot{\thepage}
\renewcommand{\headrulewidth}{0pt} % Optional: Removes the header line
%\pagestyle{fancy}
%\fancyhf{}
%\lhead{Warner \thepage}
%\rhead{}
% \lhead{\leftmark}
%\cfoot{\thepage}
%\setborder
% \usepackage[default]{sourcecodepro}
% \usepackage[T1]{fontenc}

% Change the title
\hypersetup{
    pdftitle={Javascript notes}
}

\begin{document}
    % \maketitle
        \begin{titlepage}
       \begin{center}
           \vspace*{1cm}
    
           \textbf{Javascript Notes}
    
           \vspace{0.5cm}
            
                
           \vspace{1.5cm}
    
           \textbf{Nathan Warner}
    
           \vfill
                
                
           \vspace{0.8cm}
         
           \includegraphics[width=0.4\textwidth]{~/niu/seal.png}
                
           Computer Science \\
           Northern Illinois University\\
           United States\\
           
                
       \end{center}
    \end{titlepage}
    \tableofcontents
    \pagebreak 

    \unsect{Getting started}
    \bigbreak \noindent 
    \subsection{Javascript directly in HTML}
    \bigbreak \noindent 
    Here is an example of how to write a very simple web page that will give a pop-up box saying Hi there!:
    \bigbreak \noindent 
    \begin{htmlcode}
        <html>
            <script type="text/javascript"> alert("Hi there!");
            </script>
        </html>
    \end{htmlcode}
    \bigbreak \noindent 


    \bigbreak \noindent 
    \subsection{External file}
    \bigbreak \noindent 
    First, we are going to create a separate JavaScript file. These files have the postfix .js. I'm going to call it ch1\_alert.js. This will be the content of our file:
    \bigbreak \noindent 
    \begin{jscode}
        alert("Saying hi from a different file!");
    \end{jscode}
    \bigbreak \noindent
    Then, in an html file
    \bigbreak \noindent 
    \begin{htmlcode}
        <html>
            <script type="text/javascript" src="ch1_alert.js"></script>
        </html>
    \end{htmlcode}

    \bigbreak \noindent 
    \subsection{Running javascript like a script with node}
    \bigbreak \noindent 
    You can run JavaScript code in the terminal using Node.js.
    \bigbreak \noindent 
    \begin{jscode}
    console.log("Hello world")
    \end{jscode}
    \bigbreak \noindent 
    \begin{bashcode}
    node script.js
    \end{bashcode}



    \bigbreak \noindent 
    \subsection{Semicolons}
    \bigbreak \noindent 
    After every statement, you should insert a semicolon. JavaScript is very forgiving and will understand many situations in which you have forgotten one
    \bigbreak \noindent 
    \subsection{Comments}
    \bigbreak \noindent 
    Js has two types of comments, single line // and multi line /* */

    \bigbreak \noindent 
    \subsection{Prompts}
    \bigbreak \noindent 
    Another thing we would like to show you here is also a command prompt. It works very much like an alert, but instead, it takes input from the user
    \bigbreak \noindent 
    \begin{jscode}
    prompt("Enter something")
    \end{jscode}

    \bigbreak \noindent 
    \subsection{Random}
    \bigbreak \noindent 
    We can generate random reals between 0 and 1 with
    \bigbreak \noindent 
    \begin{jscode}
    Math.random()
    \end{jscode}
    \bigbreak \noindent 
    Multiply by 100 to get reals between 0 and 100
    \bigbreak \noindent 
    \begin{jscode}
    Math.random() * 100
    \end{jscode}
    \bigbreak \noindent 
    Use the floor function to truncate to integers
    \bigbreak \noindent 
    \begin{jscode}
    Math.floor(Math.random() * 100)
    \end{jscode}

    \pagebreak 
    \unsect{Essentials}
    \bigbreak \noindent 
    \subsection{Variables}
    \bigbreak \noindent 
    \subsubsection{let, var, and const}
    \bigbreak \noindent 
    A variable definition consists of three parts: a variable-defining keyword (let, var, or const), a name, and a value. Let's start with the difference between let, var, or const. Here you can see some examples of variables using the different keywords:
    \bigbreak \noindent 
    \begin{jscode}
        let nr1 = 12;
        var nr2 = 8;
        const PI = 3.14159;
    \end{jscode}
    \bigbreak \noindent 
    let and var are both used for variables that might have a new value assigned to them somewhere in the program. The difference between let and var is complex. It is related to scope.
    \bigbreak \noindent 
    var has global scope and let has block scope. var's global scope means that you can use the variables defined with var in the entire script. On the other hand, let's block scope means you can only use variables defined with let in the specific block of code in which they were defined
    \bigbreak \noindent 
    On the other hand, const is used for variables that only get a value assigned once— for example, the value of pi, which will not change. If you try reassigning a value declared with const, you will get an error:

    \bigbreak \noindent 
    \subsection{Types}
    \bigbreak \noindent 
    \subsubsection{Primitive types}
    \bigbreak \noindent 
























    
\end{document}
