\documentclass{report}

\input{~/dev/latex/template/preamble.tex}
\input{~/dev/latex/template/macros.tex}

\title{\Huge{}}
\author{\huge{Nathan Warner}}
\date{\huge{}}
\fancyhf{}
\rhead{}
\fancyhead[R]{\itshape Warner} % Left header: Section name
\fancyhead[L]{\itshape\leftmark}  % Right header: Page number
\cfoot{\thepage}
\renewcommand{\headrulewidth}{0pt} % Optional: Removes the header line
%\pagestyle{fancy}
%\fancyhf{}
%\lhead{Warner \thepage}
%\rhead{}
% \lhead{\leftmark}
%\cfoot{\thepage}
%\setborder
% \usepackage[default]{sourcecodepro}
% \usepackage[T1]{fontenc}

% Change the title
\hypersetup{
    pdftitle={CS Complete}
}

\begin{document}
    % \maketitle
        \begin{titlepage}
       \begin{center}
           \vspace*{1cm}
    
           \textbf{Comprehensive CS}
    
           \vspace{0.5cm}
            
                
           \vspace{1.5cm}
    
           \textbf{Nathan Warner}
    
           \vfill
                
                
           \vspace{0.8cm}
         
           \includegraphics[width=0.4\textwidth]{~/niu/seal.png}
                
           Computer Science \\
           Northern Illinois University\\
           United States\\
           
                
       \end{center}
    \end{titlepage}
    \tableofcontents
    \pagebreak 
    \unsect{Theory of Computation}
    \bigbreak \noindent 
    \subsection{Natural Languages, Formal languages: Definitions and theorems}
    \begin{itemize}
        \item \textbf{G\"odel's incompleteness theorem}: 
        G\"odel's Incompleteness Theorems are two fundamental results in mathematical logic that state: 
        \begin{itemize}
            \item  Proved that for some axiomatic
                systems that there is no algorithm
                that will generate all true
                statements from those axioms.
            \item No such system can prove its own consistency.
        \end{itemize}
        This was the first indication that there are inherent limts on algorithms
        \item \textbf{Turing}: Alan Turing later provided formalism to the concepts of an "algorithm" and "computation", he Invented definition for an abstract machine called the "universal algorithm machine", he Provided means to formally (i.e., with mathematical rigor) explore the boundaries of what algorithms could, and could not, accomplish. Turing's model for a universal abstract machine was the basis for the first computer - in fact, Turing was involved in the construction of the first computer.
        \item \textbf{Natural languages}: We communicate via a \textit{natural language}, Although we don't often think about it, our language is
            guided by rules; spelling, grammar, punctuation
        \item \textbf{Formal language}:
            Formal languages, which are not intended for human-to-
            human communication, are similar to natural languages in
            that they too have rules that define "correct" words and
            statements, but they are also different than natural languages
            in two key ways;
            \begin{itemize}
                \item The rules that define a formal language are strictly enforced. There is no tolerance for misspellings, bad grammar, etc.
                \item For the purpose of determining if a word or statement is
                    acceptable in a formal language, meaning is ignored. Determining
                    if something is (or is not) part of a language is determined by the
                    language's defining rules which do not attach meaning (i.e., no
                    definitions of words like in natural languages)
            \end{itemize}
            In short, formal languages is a game of symbols, not meaning
        \item \textbf{Formal Languge terminology}:
            \begin{itemize}
                \item \textbf{Symbol}: it is an abstract entity that is not formally defined - like a point or a line in geometry - but think of it as a single character like a letter, numerical digit, punctuation mark, or emoticon
                \item \textbf{String (or Word)}: A finite sequence (i.e., order matters) of zero or more symbols
                \item \textbf{Length}: The length of a string $w$ is denoted by $length(w)$ or $\abs{w}$ and is the number of symbols composing the string. Because strings, by definition, are finite then a string's lengths is always defined (sometimes zero).
                \item \textbf{Prefix, suffix}: Any number of leading/trailing symbols of the string.
                \item \textbf{Concatenation}: The concatenation of two strings $w$ and $x$ is formed by writing the first string $w$ then the second string $x$
                    \bigbreak \noindent 
                    \textbf{Note:} For any string $w$, $\Lambda w = w\Lambda = w$  
                \item \textbf{Alphabet}: A finite set of symbols, typically denoted by the
                    Greek capital letter sigma $\Sigma$, for example
                    \begin{align*}
                        \Sigma = \{a,b,c\} \quad \Sigma = \{0,1\} \quad \Sigma = \varnothing \quad \text{(special case)}
                    .\end{align*}
            \end{itemize}
        \item \textbf{The empty string}:  A string with zero symbols is called the empty string
            and is denoted by the capital Greek letter lambda $\Lambda$, or sometimes
            lower case Greek letter epsilon $\epsilon$, where $\Lambda$ and $\epsilon$ are \textbf{not} symbols
            \bigbreak \noindent 
            Thus,
            \begin{align*}
                \abs{\Lambda} = 0
            .\end{align*}
        \item \textbf{Formal language definition}: A formal language is a et of strings from some \textbf{one} alphabet. Given an alphabet we generally define a formal language over that alphabet by
            specifying rules that either;
            \begin{enumerate}
                \item Tell us how to test a candidate word, or
                \item Tell us how to construct all words.
            \end{enumerate}
            For example, Given $\Sigma_{1}= \{x\} $, we can define languages
            \begin{align*}
                L_{1} &= \text{ any non empty string } = \{x, xx, xxx,...\} \\
                L_{2} &= \{X^{n}:\ x = 2k+1,\ k\in \mathbb{Z} \} = \{x,xxx,xxxxx,xxxxxxx,...\} \
                L_{3} &= \{x,xxxxxxxx\}
            .\end{align*}
        \item \textbf{The empty language}: The empty language $L = \varnothing$ is typically  denoted with the capital  greek letter phi $\Phi$. Thus, $L = \varnothing = \Phi $
        \item \textbf{Notes on formal languages}:
            \begin{itemize}
                \item All languages are defined over some alphabet; cannot define a language without an alphabet.
                \item Some languages are finite, some languages are infinite (remember, alphabets are always finite).
                \item Some languages include the empty string \(\Lambda\), some do not.
                \item Some languages are defined by rules, some are simply written completely (e.g., \(\Sigma_1 = \{x\}\), \(L_3 = \{x, \text{xxxxxxxxxx}\}\)).
                \item No matter what the alphabet \(\Sigma\) (even \(\Sigma = \emptyset\)), you can always define at least two languages; \(L_1 = \{\Lambda\}\) and \(L_2 = \emptyset\).
            \end{itemize}
        \item \textbf{Closure of an alphabet (closure of $\Sigma$) (Kleene closure)}:
            The language defined by the set of all strings (including the empty string $\Lambda$) over a fixed alphabet $\Sigma$.
            \begin{itemize}
                \item \textbf{Examples:}
                    \begin{align*}
                        \Sigma &= \{a\} & \Sigma^* &= \{\Lambda, a, aa, aaa, aaaa, \dots\} \\
                        \Sigma &= \{0, 1\} & \Sigma^* &= \{\Lambda, 0, 1, 00, 01, 10, 11, 000, \dots\} \\
                        \Sigma &= \emptyset & \Sigma^* &= \{\Lambda\}
                    \end{align*}
                    \bigbreak \noindent 
                    \textbf{Note:} If $\Sigma = \emptyset$ then $\Sigma^*$ is finite and $\Sigma^* = \{\Lambda\}$, otherwise $\Sigma^*$ is infinite.
            \end{itemize}
        \item \textbf{Positive closure}: $\Sigma^{+} = \Sigma^{*} - \{\Lambda\}$, you just take the empty string out of the kleene closure 
        \item \textbf{Recall: Power set}: The power set of any set $S$, written $\mathcal{P}(S)$ is the set of all subsets of $S$, including the empty set and the set $S$ itself.
            \bigbreak \noindent 
            In other words, given a set $S$, then its power set $\mathcal{P}(S)$ is a set of sets
            \begin{itemize}
                \item \textbf{Note:}
                    \begin{itemize}
                        \item If $S = \emptyset$, then then $\mathcal{P}(S) = \mathcal{P}(\emptyset) = \{\emptyset\} = \{\emptyset\}$ = a set with one element $=\emptyset$.
                        \item If $S$ is non-empty and finite with $n$ elements, then $\mathcal{P}(S)$ will be finite with $2^n$ elements.
                        \item If $S$ is infinite, then $\mathcal{P}(S)$ will be infinite.
                    \end{itemize}

                \item \textbf{Example:}

                    If $S = \{x, y, z\}$, then $\mathcal{P}(S)$ will have the following $2^3 = 8$ elements (each a set):
                    \[
                        \mathcal{P}(S) = \{\emptyset, \{x\}, \{y\}, \{z\}, \{x, y\}, \{x, z\}, \{y, z\}, \{x, y, z\}\}
                    \]
            \end{itemize}
        \item \textbf{Power set of the kleene closure $\mathcal{P}(\Sigma^{\star}) $}: Given some alphabet $\Sigma$ we can construct the set of all possible languages from $\Sigma$ as follows (assume non-empty $\Sigma$):
            \bigbreak \noindent 
            \fig{.6}{./figures/6.png}
        \item \textbf{From formal languages to computers}:
            \begin{itemize}
                \item Given an alphabet $\Sigma$ we can define many formal languages - the range of which is captured by $\mathcal{P}(\Sigma^*)$.

                \item We can define many formal languages verbally, but is there a way to define/express every language in any $\mathcal{P}(\Sigma^*)$ with some formal system or abstract machine?

                \item We search for a formal system or abstract machine with enough "power" to define any language in any $\mathcal{P}(\Sigma^*)$.

                \item \textbf{KEY POINT} \\
                    The abstract machines we discover along our search to cover $\mathcal{P}(\Sigma^*)$ turn out to be \textit{the theoretical basis for all computing}.

                \item In other words, by understanding the power (and limitations) of abstract machines that cover $\mathcal{P}(\Sigma^*)$, we are simultaneously discovering the same limits about all computing.
            \end{itemize}
    \end{itemize}

    \pagebreak 
    \subsection{Regular languages}
    \bigbreak \noindent 
    \textbf{Preface.} The first few subsubsections will be in the world of regular languages. In the context of computation theory, regular languages are a class of formal languages that can be recognized by finite automata. These languages are important because they are the simplest class of languages that can be described by a computational model. The characteristics of regular languages are as follows,
    \begin{itemize}
        \item \textbf{Finite Automata:} Regular languages can be recognized by deterministic or nondeterministic finite automata (DFA or NFA).
        \item \textbf{Regular Expressions:} Regular languages can be described using regular expressions.
        \item \textbf{Closure Properties:} Regular languages are closed under several operations, including:
            \begin{itemize}
                \item \textbf{Union:} The union of two regular languages is also regular.
                \item \textbf{Concatenation:} The concatenation of two regular languages is also regular.
                \item \textbf{Kleene Star:} The Kleene star operation, which involves repeating a regular language any number of times (including zero), results in a regular language.
                \item \textbf{Intersection and Difference:} Regular languages are also closed under intersection and difference.
            \end{itemize}
        \item \textbf{Decision Problems:} Certain decision problems are decidable for regular languages. For example, it is possible to determine whether a given string belongs to a regular language (membership problem), whether two regular languages are equivalent, or whether a regular language is empty.
    \end{itemize}
    \bigbreak \noindent 
    \subsubsection{Finite Automata}
    \begin{itemize}
        \item \textbf{Informal definition}: Described informally, a finite automaton (FA) is always associated with some alphabet $\Sigma$ and is an abstract machine which has 
            \begin{enumerate}
                \item A non-empty finite number of states, exactly one of which is designated as the "start state" and some number (possibly zero) of which are designated as "accepting states".
                \item A transition table that shows how to move from one state to another based on symbols in the alphabet $\Sigma$
            \end{enumerate}
        \item \textbf{A simple example of a FA}:
            \bigbreak \noindent 
            \fig{.5}{./figures/23.png}
            \begin{itemize}
                \item Defined over alphabet $\Sigma = \{0, 1\}$.
                \item States are circles; transitions are directed edges (i.e., arrows) between states.
                \item Has exactly three states; \textbf{A}, \textbf{B}, and \textbf{C}.
                \item Every FA must have exactly one start state. In this example, the start state is \textbf{A} and denoted as the only state that has an edge coming to it from no other state.
                \item There is only one accepting state, \textbf{C}, and it is denoted by its \textit{double circle}. (We could have more than one but in this case we only have one)
                \item \textbf{Very important:}
                    \begin{itemize}
                        \item Each symbol in the alphabet has exactly one associated edge leaving every state.
                        \item In other words, every state must have exactly one edge leaving it for each symbol in the alphabet.
                    \end{itemize}
            \end{itemize}
        \item \textbf{How to use an FA}: The purpose of a FA is to define a language over its alphabet $\Sigma$. The FA provides the means by which to test a candidate string from $\Sigma$ and determine whether or not the string is in the language. It does this by "writing" the candidate string on an fictitious input tape and proceeding as follows:
        \begin{enumerate}
            \item Set the FA to the start state.
            \item If end-of-string then halt.
            \item Read next symbol on tape.
            \item Update the state according to the current state and the last symbol read.
            \item Goto step 2.
        \end{enumerate}
        When the process halts check which state the FA is in. If it is in any accepting state, then the string is in the language defined by the FA, otherwise the string is not in the language
    \item \textbf{Using the previous FA}: Let's now try to use our FA to test whether or not the string 1001 is in the language
        \bigbreak \noindent 
        We start by writing the string on an input tape, placing the read head at the beginning of the tape, and placing the FA in its initial state, $A$
        \bigbreak \noindent 
        \fig{.8}{./figures/24.png}
        \bigbreak \noindent 
        Since the tape head is not at the end of the tape we
        \begin{enumerate}
            \item Read the next symbol from the tape.
            \item Follow the edge from the state we are currently in that corresponds to the symbol we just read to transition to the next state.
            \item Move the tape head
        \end{enumerate}
        \bigbreak \noindent 
        \fig{.8}{./figures/25.png}
        \bigbreak \noindent 
        In this case, we started in state $A$, read symbol 1, and followed the edge labeled 1 from $A$ which brought us back to $A$
        \bigbreak \noindent 
        We proceed in this way, read, change state, move tape head until we reach the end of the tape
        \bigbreak \noindent 
        Once the tape head reaches the end of the tape we simply look to see whether or not the FA ended in an accepting state.
        \bigbreak \noindent 
        In this case it ended in state $C$, which is an accepting state, which means that string 1001 is in the language.
        \bigbreak \noindent 
        \fig{.8}{./figures/26.png}
        \bigbreak \noindent 
        We deduce that the language has only strings with two consecutive zeroes somewhere.
        \pagebreak \bigbreak \noindent
    \item \textbf{FA Example Two: The set of all strings that do not contain a one ($\Sigma = \{0,1\})$}:
        \bigbreak \noindent 
        \fig{.5}{./figures/27.png}
        \bigbreak \noindent 
        This one is pretty simple. If we have a zero, stay in the accepting state, if we see a one, toss it to the other non-accepting state, its not coming back.
    \item \textbf{FA Example Three: The set of all strings that end in one ($\Sigma = \{0,1\})$}:
        \bigbreak \noindent 
        \fig{.5}{./figures/28.png}
    \item \textbf{FA Example Four: The set of all strings with an odd number of zeros ($\Sigma = \{0,1\})$}:
        \bigbreak \noindent 
        \fig{.5}{./figures/29.png}
    \item \textbf{FA Example Five: The set of all strings where the second to last symbol is one ($\Sigma = \{0,1\})$}:
        \bigbreak \noindent 
        \fig{.5}{./figures/30.png}
    \item \textbf{States are "memory"}: Consider the four FA we just created, in each instance the solution required us to design an FA that remembered at least part of what it had already read from the input tape. The type of memory that an FA has is very different than the RAM we find in
        contemporary computers, but the FA does have memory. Each time the FA enters a different state it is, in effect, redefining the memory of the
        entire FA. The FA can only be in a finite number of states, and that number can be arbitrarily
        large, but (as we will see) that difference in memory has a profound limiting effect in
        what FAs can compute.
        \bigbreak \noindent 
    \item \textbf{Limits of a FA}:
        \bigbreak \noindent 
        \textbf{Limited Memory:}
        \begin{itemize}
            \item \textbf{Finite State:} A finite automaton has a finite number of states. This means it can only "remember" a limited amount of information about the input it has processed. Once a finite automaton transitions to a new state, it forgets all previous information except for the current state.
            \item \textbf{No Stack or Tape:} Unlike more powerful models such as pushdown automata (which have a stack) or Turing machines (which have an infinite tape), finite automata cannot use any form of auxiliary memory to keep track of an unbounded number of items or to perform operations that require more complex memory management.
        \end{itemize}
        \bigbreak \noindent 
        \textbf{Inability to Count Unboundedly:}
        \begin{itemize}
            \item \textbf{No Arbitrary Counting}: Finite automata cannot count occurrences of symbols beyond the number of states they have. For example, a DFA with $n$ states can only count up to $n-1$ occurrences of a symbol reliably. Thus, they cannot recognize languages that require matching counts of different symbols if those counts are unbounded, such as $\{a^n b^n \mid n \geq 1\}$, where the number of 'a's must match the number of 'b's.
        \end{itemize}
    \item \textbf{FA Formal Definition}:
     We formally denote a \textit{finite automaton} by a 5-tuple $(Q, \Sigma, q_0, T, \delta)$, where
    \begin{itemize}
        \item $Q$ is a finite set of \textit{states}.
        \item $\Sigma$ is an alphabet of \textit{input symbols}.
        \item $q_0 \in Q$, is the \textit{start state}.
        \item $T \subseteq Q$, is the set of \textit{accepting states}.
        \item $\delta$ is the \textit{transition function} that maps a state in $Q$ and a symbol in $\Sigma$ to some state in $Q$. In mathematical notation, we say that $\delta: Q \times \Sigma \rightarrow Q$.
            With:
            \begin{itemize}
                \item $Q \times \Sigma$: The Cartesian product of the set of states $Q$ and the alphabet $\Sigma$. This represents all possible pairs of a state and an input symbol.
                \item $\rightarrow Q$: Indicates that the transition function maps each pair $(q, \sigma)$ (where $q \in Q$ and $\sigma \in \Sigma$) to a single state in $Q$.
            \end{itemize}
    \end{itemize}
    \item \textbf{Formally Specifying Our First FA}:
        \bigbreak \noindent 
        \fig{.5}{./figures/23.png}
        \bigbreak \noindent 
        Recall our first FA that accepts any string with two consecutive zeros somewhere.
        \bigbreak \noindent 
        We drew it as a Finite State diagram, but to formally define this FA we must specify each of the elements from the 5-tuple $(Q, \Sigma, q_0, T, \delta)$.
        \begin{itemize}
            \item $Q$ is a finite set of \textit{states}: \hspace{0.2cm} $Q = \{A, B, C\}$
            \item $\Sigma$ is an alphabet of \textit{input symbols}: \hspace{0.2cm} $\Sigma = \{0, 1\}$
            \item $q_0 \in Q$, is the \textit{start state}: \hspace{0.2cm} $q_0 = A$
            \item $T \subseteq Q$, is the set of \textit{accepting states}: \hspace{0.2cm} $T = \{C\}$
            \item $\delta$ is the \textit{transition function} $\delta: Q \times \Sigma \rightarrow Q$
        \end{itemize}
        \[
            \begin{array}{c|cc}
                \delta & \text{0} & \text{1} \\
                \hline
                A & B & C \\
                B & C & A \\
                C & C & C \\
            \end{array}
        \]
    \item \textbf{Unary}: consisting of or involving a single component or element.
    \item \textbf{Unary language}: One where the alphabet has only one symbol.
    \item \textbf{Binary}: Relating to, composed of, or involving two things.
    \item \textbf{Ternary}: Composed of three parts.
    \item \textbf{Dead state (trap state)}: This is a state that once entered, can never be left.
    \item \textbf{Deterministic finite automaton (DFA)}: The FA's we have looked at thus far have been DFA's. A DFA is a finite automaton where, for each state and each input symbol, there is exactly one transition to a new state. This means that given a current state and an input symbol, the next state is uniquely determined. In the future we will look at nondeterministic finite automaton (NFA). An NFA is a finite automaton where, for each state and input symbol, there can be multiple possible transitions to different states. Additionally, an NFA can have transitions that do not consume any input symbol ($\epsilon$-transitions).
    








    \end{itemize}

    \pagebreak 
    \subsubsection{Finite Automata: More examples}
    \begin{itemize}
        \item \textbf{$\Sigma = \{0,1\}$, all strings that start with 00}
        \item \textbf{$\Sigma = \{0,1\}$, all strings that end with 00}
            \bigbreak \noindent 
            \fig{.5}{./figures/31.png}
            \bigbreak \noindent 
            With:
            \begin{itemize}
                \item $Q = \{A,B,C\}$
                \item $\Sigma = \{0,1\}$
                \item $q_{0} = A$
                \item $T = C$
                \item $\delta:\ Q \times \Sigma \to  Q$ defined by                
                    \begin{align*}
                        \begin{array}{c|cc}
                            \delta & 0 & 1 \\
                            \hline
                            A & B & A\\
                            B & C & A\\
                            C & C & A
                        \end{array}
                    .\end{align*}
            \end{itemize}
                
    \end{itemize}


    \pagebreak 
    \subsubsection{nondeterministic Finite automata (NFA)}
    \begin{itemize}
        \item \textbf{NFA definition}:
            \begin{itemize}
                \item If an automaton gets to a state where there is more than one possible transition corresponding to the symbol read from the tape, the automaton may  choose any of those paths. (nondeterminism) We say it \textbf{branches}
                \item if an automaton gets to a state where there is no transition for the symbol read from the tape, then that path of the automaton halts and rejects the string. We say it \textbf{dies}
                \item the automaton accepts the input string if and only if there exists a choice of transitions that ends in an accept state.
            \end{itemize}
            \bigbreak \noindent 
            \textbf{Example}: Consider this nondeterministic FA (NFA) over $\Sigma = \{0, 1\}$
            \bigbreak \noindent 
            \fig{.5}{./figures/32.png}
        \item \textbf{DFA or NFA?}:
            Consider the language $L$ over $\Sigma = \{a, b\}$ which is defined by
            \begin{align*}
                L = (a^{*}) + (ab)^{*}
            .\end{align*}
            \bigbreak \noindent 
            \fig{.5}{./figures/33.png}
        \item \textbf{NFA Formal definition}: We define an NFA $M(Q, \Sigma, q_{0}, T, \delta) $
            \begin{itemize}
                \item $Q$ is a finite set of states
                \item $\Sigma$ is an alphabet of input symbols
                \item $q_0 \in Q$ is the start state
                \item $T \subseteq Q$ is the set of accepting states
                \item $\delta$ is the transition function $\delta: Q \times \Sigma \to P(Q)$
            \end{itemize}
        \item \textbf{Transition function, DFA vs NFA}:
            \bigbreak \noindent 
            \fig{.5}{./figures/34.png}
        \item \textbf{NFA with $\epsilon$-transitions}: $\epsilon$-transitions allow the automaton to change state without
            consuming an input symbol
            \bigbreak \noindent 
            Changing states without consuming input symbols can go on arbitrarily long as there are $\epsilon$-transitions to traverse.
        \item \textbf{DFA or NFA with $\epsilon$-moves?}: Consider the language L over $\Sigma = \{a, b\}$ which is
            \begin{align*}
                L = (b^{*}a) + (a^{*}b)
            .\end{align*}
            \bigbreak \noindent 
            \fig{.5}{./figures/35.png}
        \item \textbf{NFA with $\epsilon$-transitions formal definition}: Everything is the same except for the transition function, we now have
            \begin{align*}
                \delta:\ Q \times (\Sigma \cup \{\epsilon\}) \to \mathcal{P}(Q)
            .\end{align*}
            \pagebreak \bigbreak \noindent 
        \item \textbf{$\delta$ - DFA, NFA, and NFA with $\epsilon$-moves}:
            \bigbreak \noindent 
            \fig{.35}{./figures/36.png}
        \item \textbf{DFA, NFA, or NFA with $\epsilon$ moves, who can define the most languages?}: We begin by noting, by definition, every DFA is an NFA. This means that any language you can define with a DFA can also be defined by an NFA. Thus,
            \begin{align*}
                \text{Languages defined by DFA} \subseteq \text{ Languages defined by NFA}
            .\end{align*}
            Also, by definition, every DFA is an NFA with $\epsilon$-moves, an NFA is an NFA with $\epsilon$ moves, even if it doesnt have any. Thus,
            \begin{align*}
                \text{Languages defined by DFA} \subseteq \text{ Languages defined by NFA with $\epsilon$-moves}
            .\end{align*}
            \bigbreak \noindent 
            But, by definition, every NFA is an NFA with $\epsilon$-moves. Thus,
            \begin{align*}
                \text{Languages defined by NFA} \subseteq \text{ Languages defined by NFA with $\epsilon$-moves}
            .\end{align*}
            \bigbreak \noindent 
            This tells us that
            \begin{itemize}
                
                \item NFAs are at least as powerful in defining languages as DFAs
                \item NFAs with $\epsilon$-moves are at least as powerful in defining languages as DFAs and NFAs.
            \end{itemize}
            \bigbreak \noindent 
            It turns out that these three are \textbf{equally} as powerful. We assert
            \begin{align*}
                &\text{Languages defined by DFA's} \\
                &=\text{Languages defined by NFA's} \\
                &=\text{Languages defined by NFA's with $\epsilon$-moves}
            .\end{align*}
            We prove this by showing an algorithm that converts any NFA with $\epsilon$-moves (or any NFA) to a DFA that accepts the exact same language
            \bigbreak \noindent 
            This means that there does not exist a language that can be defined by an NFA with $\epsilon$-moves (or NFA) that cannot also be defined by a DFA.
        \item \textbf{$\epsilon$-closure}: Before we can look at the algorithm we must first define the $\epsilon$-closure of a set of states 
            \bigbreak \noindent 
            Given:
            \begin{itemize}
                \item an NFA with $\epsilon$-moves $M(Q, \Sigma, q_{0}, T, \delta) $
                \item Some set of states $S \subseteq Q$
            \end{itemize}
            \bigbreak \noindent 
            \text{We define the } \varepsilon\text{-closure}(S) \text{ as the set of states that are reachable from the set of states } S \text{ using only zero or more } \varepsilon\text{-moves in } \delta.
            \bigbreak \noindent 
            \text{Note: it is always the case that } S \subseteq \varepsilon\text{-closure}(S)
            \bigbreak \noindent 
            The formal definition is
            \begin{align*}
                \epsilon-\text{closure}(q) = \{q\} \cup \{p:\ q \xrightarrow{\epsilon} p\}
            .\end{align*}
        \item \textbf{$\epsilon$-closure alternate notation}. 
            \begin{align*}
                \epsilon\text{-closure}(\{A\}) = \epsilon(\{A\}) = E(\{A\})
            .\end{align*}
        \item \textbf{$\epsilon$-closure of the empty set $\varnothing$}: The epsilon closure of the empty set is $\epsilon(\varnothing)  = \varnothing$
        \item \textbf{Algorithm: Converting NFA with $\epsilon$-moves to DFA}: The algorithm constructs a new DFA $M^{\prime}(Q^{\prime}, \Sigma, q_{0}^{\prime}, T^{\prime}, \delta^{\prime}) $ From an NFA with $\epsilon$-moves $M(Q, \Sigma, q_{0}, T, \delta) $. $\Sigma$ will remain the same
            \bigbreak \noindent 
            Things to note about the conversion:
            \begin{itemize}
                \item Same alphabet $\Sigma $
                \item Lose column $\epsilon$
                \item Lose all nondeterminism
                \item Lose all empty sets
                \item Cell values change from sets of states to states
            \end{itemize}
            \bigbreak \noindent 
            \textbf{Example}:
            \textbf{Consider the following NFA with $\varepsilon$-moves M(Q, $\Sigma$, $q_0$, T, $\delta$) over $\Sigma = \{0, 1\}$ and its associated transition table $\delta$: Q $\times$ ($\Sigma \cup \{\varepsilon\}$) $\rightarrow$ P(Q)}
            \bigbreak \noindent 
            \[
                \begin{array}{|c|c|c|c|}
                    \hline
                     & 0 & 1 & \varepsilon \\
                    \hline
                    X & \{Y\} & \{Y\} & \emptyset \\
                    \hline
                    Y & \{X, Z\} & \{Z\} & \{Z\} \\
                    \hline
                    Z & \emptyset & \{Y\} & \emptyset \\
                    \hline
                \end{array}
            \]
            \bigbreak \noindent 
            Start by computing the $\epsilon$-closure of the start state in $\delta$.
            \bigbreak \noindent 
            \fig{.4}{./figures/37.png}
            \bigbreak \noindent 
            There is a subtle - but very important - point to be made here ...
            \bigbreak \noindent 
            we cannot simply take the $\epsilon$-closure (a set) and use it to create a row in $\delta^{\prime}$ (which needs to be a state). What we do is create a label for the new state in $\delta^{\prime}$ that represents the set of states from $\delta$ and then add that new state to $\delta^{\prime}$
            \bigbreak \noindent 
            In this instance we represented the set of states $\{X\}$ by a single state whose label is $X^{\prime}$
            \bigbreak \noindent 
            We continue by filling the columns of the start state for each symbol $\Sigma = \{0, 1\}$
            \bigbreak \noindent 
            Processing $\delta'$ state $X'$ which represents the set of states $\{X\}$ in $M$:
            \begin{itemize}
                \item Processing input symbol 0 (process each state in $\{X\}$ using $\delta$):
                    \begin{itemize}
                        \item Process $X$
                            \[
                                \delta(X, 0) = \{Y\}
                            \]
                            \[
                                \varepsilon\text{-closure}(\{Y\}) = \{Y, Z\}
                            \]
                    \end{itemize}
            \end{itemize}
    Since there are no more states in $\{X\}$ to process, we have finished processing the symbol 0 and have produced the set of states $\{Y, Z\}$.
    \bigbreak \noindent 
    We create a new state with label $Y'Z'$ (or $Z'Y'$, order does not matter) for $\delta'$ that represents $\{Y, Z\}$ in $M$ and define:
    \[
    \delta'(X', 0) = Y'Z'
    \]
    We note that $Y'Z'$ is a new state in $\delta'$ and so we create a new row for it in $\delta'$.
    \bigbreak \noindent 
    We continue this until we reach 
    \bigbreak \noindent 
    \fig{.8}{./figures/38.png}
    \bigbreak \noindent 
    Processing $\delta'$ state $Y'Z'$ which represents the set of states $\{Y, Z\}$ in $M$:
    \begin{itemize}
        \item Processing 0:
            \begin{itemize}
                \item Process $Y$
                    \[
                        \delta(Y, 0) = \{X, Z\}, \quad \varepsilon\text{-closure}(\{X, Z\}) = \{X, Z\}
                    \]
                \item Process $Z$
                    \[
                        \delta(Z, 0) = \emptyset, \quad \varepsilon\text{-closure}(\emptyset) = \emptyset
                    \]
            \end{itemize}
    \end{itemize}
    Here is our first instance of processing a state and symbol where the state in $\delta'$ represents multiple states in NFA $M$. When this happens, the set of states in NFA $M$ is computed by \textit{taking the union of the $\varepsilon$-closures}: $\{X, Z\} \cup \emptyset = \{X, Z\}$.
    \bigbreak \noindent 
    This produces a new label $X'Z'$ which we use to define:
    \[
        \delta'(X'Y', 0) = X'Z'
    \]
    and since $X'Z'$ is a new state, we add it to $\delta'$.
    \bigbreak \noindent 
    We continue this until we reach 
    \bigbreak \noindent 
    \fig{.8}{./figures/39.png}
    \bigbreak \noindent 
    A state in $M^{\prime}$ is an accepting state iff at least one of the states that it represents in $M$ is an accepting state ... in this case $T^{\prime}= \{Y^{\prime}Z^{\prime}\}$.
    \bigbreak \noindent 
    We can now draw the new DFA 
    \bigbreak \noindent 
    \fig{.7}{./figures/40.png}
    \bigbreak \noindent 
    \textbf{Note:} If the closure or union of closures is the empty set, we do this
    \bigbreak \noindent 
    \fig{.7}{./figures/41.png}
    \bigbreak \noindent 
    This "emtpy" is a state and represents a garbage state, what goes does not leave.
\item \textbf{Kleene's theorem revisited}: The following are equivalent for a language $L$
    \begin{enumerate}
        \item There is a DFA for $L$
        \item There is an NFA for $L$
        \item There is an RE for $L$
    \end{enumerate}
    \item \textbf{Union of two DFA's (cartesian product construction)}:
        The process of finding the union of two deterministic finite automata (DFAs) involves creating a new DFA that accepts the union of the languages accepted by the original DFAs. This is done using a product construction (also called the Cartesian product construction), where you combine the states of both DFAs in a systematic way to ensure the resulting DFA accepts strings from either of the original DFAs.
        \bigbreak \noindent 
        Let's say we have two DFAs:
        \[
            D_1 = (Q_1, \Sigma, \delta_1, q_1^{\text{start}}, F_1)
        \]
        that recognizes language \( L_1 \).
        \[
            D_2 = (Q_2, \Sigma, \delta_2, q_2^{\text{start}}, F_2)
        \]
        that recognizes language \( L_2 \).
        \bigbreak \noindent 
        \textbf{Create a New DFA State Set}:
        \bigbreak \noindent 
        \begin{itemize}
            \item The states of the new DFA are pairs of states, one from each of the original DFAs. The new state set will be the Cartesian product \(Q_1 \times Q_2\), meaning every possible combination of a state from \(D_1\) and a state from \(D_2\).
            \item If \(D_1\) has \(n\) states and \(D_2\) has \(m\) states, the new DFA will have \(n \times m\) states.
        \end{itemize}
        \bigbreak \noindent 
        \textbf{Define the New Start State}:
        \begin{itemize}
            \item The new start state is \((q_1^{\text{start}}, q_2^{\text{start}})\), where \(q_1^{\text{start}}\) is the start state of \(D_1\) and \(q_2^{\text{start}}\) is the start state of \(D_2\).
        \end{itemize}
        \textbf{Define the New Transition Function:}
        \begin{itemize}
            \item The transition function \(\delta\) for the new DFA operates by taking an input symbol and applying the transition functions of both original DFAs in parallel.
            \item For each input symbol \(a \in \Sigma\), the new DFA transitions from state \((q_1, q_2)\) to state \((\delta_1(q_1, a), \delta_2(q_2, a))\).
            \item In other words, if \(q_1\) moves to \(q_1'\) on input \(a\) in \(D_1\), and \(q_2\) moves to \(q_2'\) on input \(a\) in \(D_2\), the new DFA will move from \((q_1, q_2)\) to \((q_1', q_2')\).
        \end{itemize}
        \bigbreak \noindent 
        \textbf{Define the New Set of Accepting (Final) States}:
        The new DFA will accept a string if either of the original DFAs would accept it. Therefore, the set of final states \( F \) in the new DFA is defined as:
        \[
            F = \{ (q_1, q_2) \mid q_1 \in F_1 \ \text{or} \ q_2 \in F_2 \}
        \]
        This means that if either \( q_1 \) is a final state in \( D_1 \), or \( q_2 \) is a final state in \( D_2 \), the pair \( (q_1, q_2) \) is a final state in the new DFA.
        \bigbreak \noindent 
        \textbf{Note:} It is possible in the new DFA (constructed as the union of two DFAs) to have states that are unreachable-meaning there are states in the DFA that cannot be reached from the start state. This typically happens because, in the product construction, we generate all possible pairs of states from the two original DFAs, but not all of these pairs are necessarily reachable.
        \bigbreak \noindent 
        The union of two finite automata (FAs) is useful for constructing a new automaton that recognizes any string accepted by either of the two original automata. This has several practical applications in theoretical computer science and programming:
    \item \textbf{Finding the intersection of two DFA's}: The process is basically the same as finding the union, but it differs in how we define the accepting states in the new machine, the accepting states will be
        \begin{align*}
            T = \{(q_{1},q_{2}):\ q_{1} \in T_{1} \text{ and } q_{2} \in T_{2}\}
        .\end{align*}
        \bigbreak \noindent 
        \textbf{Note:} The intersection of two DFAs is useful in various practical applications where you need to accept only the strings that satisfy the conditions or rules of both automata
    \item \textbf{Concatenation of two DFA's}: The process is simple
        \bigbreak \noindent 
        For two machines $M_{1}(Q_{1}, \Sigma, q_{0_{1}}, T_{1}, \delta_{1})$, and $M_{2}(Q_{2}, \Sigma, q_{0_{2}}, T_{2}, \delta_{2})$ 
        \begin{enumerate}
            \item Connect the final states of the first machine to the start state of the second machine (With $\epsilon$-transitions)
            \item Clear $T_{1}$, There are no more final states in the first machine
            \item Convert $\epsilon$-NFA to DFA
        \end{enumerate}
        \textbf{Note:} The concatenation of two DFAs has practical uses in many scenarios where the language of interest is the concatenation of two sublanguages. Concatenating two DFAs allows you to recognize strings that can be divided into two parts, where the first part is recognized by one DFA and the second part is recognized by the other.
    \item \textbf{Finding the union of two NFA's}:
        taking the union of two nondeterministic finite automata (NFAs) involves constructing a new NFA that accepts any string that is accepted by either of the original NFAs. This process can be done by creating a new NFA that combines the two original NFAs. 
        \bigbreak \noindent 
        Given $M_{1}(Q_{1}, \Sigma, q_{0_{1}}, T_{1}, \delta_{1}) $, and $M_{2}(Q_{2}, \Sigma, q_{0_{2}}, T_{2}, \delta_{2}) $
        \bigbreak \noindent 
        \begin{enumerate}
            \item \textbf{New start state}: Start by defining a new start state $q^{\prime}_{0}$, this state will have $\epsilon$ transitions to the start states of both machines.
            \item \textbf{Define $Q^{\prime}$, the new set of states}: The new set of states will be the set of all states in $M_1$, and it will include all the states in $M_2$, along with the new start state. Thus,
                \begin{align*}
                    Q^{\prime} = Q_{1} \cup Q_{2} \cup \{q^{\prime}_{0}\}
                .\end{align*}
            \item \textbf{Define the transition function}: The transition function $\delta^{\prime}$ of the new NFA will include:
                \begin{itemize}
                    \item All the transitions of $M_{1}$ and $M_{2}$
                    \item Two $\epsilon$ transitions from the new start state to the start states of the two original machines $q_{0_{1}}$ and $q_{0_{2}}$. Thus,
                        \begin{align*}
                            \delta^{\prime}(q_{0}^{\prime}, \epsilon) = \{q_{0_{1}}, q_{0_{2}}\}
                        .\end{align*}
                \end{itemize}
            \item \textbf{Define the set of accepting states}: The set of accepting states will be
                \begin{align*}
                    T^{\prime} = T_{1} \cup T_{2}
                .\end{align*}
        \end{enumerate}
        \pagebreak \bigbreak \noindent 
        \textbf{Example:}
        \begin{figure}[ht]
            \centering
            \incfig{machine10}
            \label{fig:machine10}
        \end{figure}
        \bigbreak \noindent 
        $M_1 \cup M_2$ is then
        \begin{figure}[ht]
            \centering
            \incfig{machine11}
            \label{fig:machine11}
        \end{figure}
        \pagebreak \bigbreak \noindent 
    \item \textbf{Finding the intersection of two NFA's}: 
        For NFAs, intersection is more complex because NFAs are nondeterministic and don't handle intersection naturally. Typically, you convert the NFAs to DFAs and then apply the DFA product construction
    \item \textbf{Concatenation of two NFA's}: The process is the same as with two DFA's (see above), but you don't need to convert to a DFA at the end.
    % \item \textbf{Convert DFA into RE}: To convert a DFA (Deterministic Finite Automaton) into a Regular Expression (RE), you can use the state elimination method or generalized transition automaton method. This process works by gradually reducing the DFA's states and transitions until only a regular expression representing the entire language remains.
    %     \bigbreak \noindent 
    %     Given a DFA $M(Q, \Sigma, \delta, q_{0}, F)$, the goal is to find a regular expression that represents the language recognized by this DFA.
    %     \bigbreak \noindent 
    %     \textbf{Process:}
    %     \begin{enumerate}
    %         \item \textbf{Add a new Start and accept state}:
    %             \begin{itemize}
    %                 \item Add a new start state \( q_s \) with an \(\epsilon\)-transition (empty string) to the original start state \( q_0 \). 
    %                     \bigbreak \noindent 
    %                     \textbf{Note:} Once you add the new start state \( q_s \) with an \(\epsilon\)-transition to the original start state \( q_0 \), \( q_0 \) is no longer considered the start state. Instead, \( q_0 \) becomes just another intermediate state in the automaton. The new start state is \( q_s \), and it immediately transitions to \( q_0 \) without consuming any input (via the \(\epsilon\)-transition).
    %                 \item Add a new accept state \( q_f \) and add \(\epsilon\)-transitions from each of the original accept states to this new accept state \( q_f \).
    %                     \bigbreak \noindent 
    %                     \textbf{Note:} Similarly, when you add the new accept state \( q_f \) and connect it via \(\epsilon\)-transitions from the original final states in \( F \), the original final states are no longer considered final states in the sense of marking the end of a string's acceptance. Now, the new final state \( q_f \) serves as the sole final state, and the automaton reaches \( q_f \) via \(\epsilon\)-transitions from the original final states.
    %             \end{itemize}
    %             These new states simplify the process because now there's exactly one start state and one accept state.
    %         \item \textbf{Eliminate States One by One}:
    %             \begin{itemize}
    %                 \item The idea is to progressively eliminate states from the DFA while updating the transitions between the remaining states with regular expressions.
    %                 \item Every time you eliminate a state $r$, you need to update the regular expressions on the transitions between the remaining states to account for the paths that go through $r$.
    %             \end{itemize}
    %             For any three states \( p \), \( r \), and \( q \), if there is a path from \( p \) to \( q \) that goes through \( r \), the new transition after eliminating \( r \) will include the regular expression:
    %             \[
    %                 R(p \rightarrow q) = R(p \rightarrow q) + R(p \rightarrow r) R(r \rightarrow r)^* R(r \rightarrow q)
    %             \]
    %             \pagebreak \bigbreak \noindent 
    %             Where:
    %             \begin{itemize}
    %                 \item \( R(p \rightarrow q) \) is the regular expression for the direct transition from \( p \) to \( q \).
    %                 \item \( R(p \rightarrow r) \) is the regular expression for the transition from \( p \) to \( r \).
    %                 \item \( R(r \rightarrow r) \) is the regular expression for the loop on state \( r \).
    %                 \item \( R(r \rightarrow q) \) is the regular expression for the transition from \( r \) to \( q \).
    %                 \item \( + \) represents union, and \( * \) represents the Kleene star (zero or more repetitions).
    %             \end{itemize}
    %             \bigbreak \noindent 
    %             After updating the transitions, remove the state \( r \).
    %             \bigbreak \noindent 
    %         \item \textbf{Repeat the Elimination Until Only Two States Remain}:
    %             Continue eliminating states and updating the transitions until only two states remain: the start state \( q_s \) and the new accept state \( q_f \).
    %             \bigbreak \noindent 
    %             At this point, the regular expression on the transition from \( q_s \) to \( q_f \) represents the language of the DFA.
    %     \end{enumerate}
    %     \bigbreak \noindent 
    %     \textbf{Example}: For the alphabet $\Sigma = \{0,1\}$, let's take the machine that accepts the strings with any number of ones, but the total number of zero's must be odd, and convert it to a RE.
    %     \begin{figure}[ht]
    %         \centering
    %         \incfig{machine1}
    %         \label{fig:machine1}
    %     \end{figure}
    %     \bigbreak \noindent 
    %     Let's start by making the new start and end states
    %     \pagebreak \bigbreak \noindent 
    %     \begin{figure}[ht]
    %         \centering
    %         \incfig{machine4}
    %         \label{fig:machine4}
    %     \end{figure}
    %     \bigbreak \noindent 
    %     Now we start eliminating states, note that it does not matter in which order we eliminate the states, but for this example we will begin by eliminating state $A$. To get from the start state $q_{0}$ to state $B$, we need to pass through $A$, to get from $A$ to $B$, we can have any number of $1's$ followed by a zero which takes us to be. Thus, the transition from $q_{0}$ to $B$ is the regular expression $1^{*}0$
    %     \bigbreak \noindent 
    %     We also have to consider the original transition from $B$ to $A$, and then back to $B$, for this we have the RE $01^{*}0$. Thus the machine becomes
    %     \bigbreak \noindent 
    %     \begin{figure}[ht]
    %         \centering
    %         \incfig{machine9}
    %         \label{fig:machine9}
    %     \end{figure}
    %     \bigbreak \noindent 
    %     To eliminate $B$, we need to consider the transitions through $B$ ie from $q_{0}$ to $q_{f}$. We know to get from $q_{0}$ to $B$ we have the RE $1^{*}0$, then from $B$ to $q_{f}$ we have $(1 + 01^{*}0)^{*} $. Thus, the transition for $q_{0}$ to $q_{f}$ is $1^{*}0(1 + 1^{*}01^{*}0)^{*} $. And the final machine with only one regular expression is 
    %     \begin{figure}[ht]
    %         \centering
    %         \incfig{machine8}
    %         \label{fig:machine8}
    %     \end{figure}
    %     \pagebreak 
%     \item \textbf{Convert RE to NFA}: Before we begin, recall order of operations (From highest to lowest )
%         \begin{enumerate}
%             \item Parenthesis  
%             \item Kleene star
%             \item Concatenation
%             \item Union (+ or \|)
%         \end{enumerate}
%         Let's consider the regular expression $aa(a+b)^{*}bb$
%         \begin{enumerate}
%             \item We start by defining simple NFA's for each symbol ($a$ and $b$)
%         \end{enumerate}
%         \begin{figure}[ht]
%             \centering
%             \incfig{machine12}
%             \label{fig:machine12}
%         \end{figure}
%         \bigbreak \noindent 
%         Then, by precedence, we design an NFA for inside the parenthesis, and then for the kleene star of the parenthesis. To make the NFA for $a + b$, we follow the rules for the union of two NFAs
%         \bigbreak \noindent 
%     \begin{figure}[ht]
%         \centering
%         \incfig{machine13}
%         \label{fig:machine13}
%     \end{figure}
%     \bigbreak \noindent 
%     In order to take the kleen star of this machine, we need to allow for zero or more repetitions. Thus,
%     \bigbreak \noindent 
% \begin{figure}[ht]
%     \centering
%     \incfig{machine14}
%     \label{fig:machine14}
% \end{figure}
% \bigbreak \noindent 
% Now we create two more NFA's, one for $aa$, and one for $bb$
% \bigbreak \noindent 
% \begin{figure}[ht]
%     \centering
%     \incfig{machine15}
%     \label{fig:machine15}
% \end{figure}
% \pagebreak \bigbreak \noindent 
% Now, we combine them all using the logic of concatenation. The final product is then
% \bigbreak \noindent 
% \begin{figure}[ht]
%     \centering
%     \incfig{machine17}
%     \label{fig:machine17}
% \end{figure}
% \bigbreak \noindent 
% Notice we added the extra epsilon transition (pink), this is to bypass the kleene star if the choice of zero occurences is executed.
%


    \pagebreak 
    \item \textbf{Properties of union, intersect, and concatenation for two FA's}: the properties of union, intersection, and concatenation for finite automata (FAs) are directly tied to the properties of regular languages.
        \bigbreak \noindent 
        \textbf{Union of two FA}
        \begin{itemize}
            \item \textbf{Closure}: The class of regular languages (those recognized by FA) is closed under union. This means the union of two regular languages is also regular, and there exists an FA that recognizes the union of the languages.
            \item \textbf{Commutative:} Union is commutative for FA, meaning the order of combining automata does not matter.
            \item \textbf{Associative:} Union is associative, so it doesn't matter how automata are grouped when performing multiple unions.
            \item \textbf{Distributive over Intersection} Union distributes over intersection for regular languages, just as with sets.
        \end{itemize}
        \textbf{Intersection of two FA}
        \begin{itemize}
            \item \textbf{Closure:} The class of regular languages is also closed under intersection, meaning there is always an FA (typically constructed as a DFA) that recognizes the intersection of two regular languages.
            \item \textbf{Commutative:} Intersection is commutative, meaning the order of combining automata doesn't matter.
            \item \textbf{Associative:} Intersection is associative, so the grouping doesn't matter.
            \item \textbf{Distributive over Union:} Intersection distributes over union for regular languages, just as with sets.
        \end{itemize}
        \bigbreak \noindent 
        \textbf{Concatenation}
        \begin{itemize}
            \item \textbf{Closure:} Regular languages are closed under concatenation.
            \item \textbf{Associativity:} Concatenation is associative. This means that the way in which you group the automata when performing concatenation doesn't matter. 
            \item \textbf{Identity Element:} The identity element for concatenation is the language that contains only the empty string,
                \begin{align*}
                    L(A)  \cdot \{\epsilon\} = L(A)
                .\end{align*}
            \item \textbf{Distributivity Over Union:} Concatenation distributes over union. This means:
                \begin{align*}
                    L(A) \cdot (L(B) \cup (L(C)) = L(A) \cdot L(B)) \cup (L(A) \cdot L(C))
                .\end{align*}
            \item \textbf{Concatenation with the Empty Set:} Concatenating any language with the empty set results in the empty set. This is because there are no strings to concatenate if one of the languages is empty:
        \end{itemize}
        \bigbreak \noindent 
        \textbf{Not commutative}






    \end{itemize}

        \pagebreak 
    \subsubsection{Regular expressions}
    \begin{itemize}
        \item \textbf{RE}: A RE corresponds to a set of strings; that is, a RE describes a language
        \item \textbf{RE three operations}:
            \begin{enumerate}
                \item Union (+)
                \item concatenation (xy)
                \item star (zero or more copies)
            \end{enumerate}
        \item \textbf{RE special symbols}
            \begin{align*}
                + \quad * \quad (\ \ )
            .\end{align*}
        \item \textbf{Grouping}: The parenthesis are used for grouping, 
        \item \textbf{Union}: the plus sign means \textbf{union}. Thus, writing
            \begin{align*}
                0 + 1
            .\end{align*}
            Means zero or one, we refer to + as "or"
        \item \textbf{Concatenation}: We concatenate simply by writing one expression after the other, with no spaces
            \begin{align*}
                (0+1)0
            .\end{align*}
            Is the pair of strings 00 and 10
        \item \textbf{Empty string}: We can also use the empty string $\epsilon$
            \begin{align*}
                (0 + 1)(0 + \epsilon)
            .\end{align*}
            corresponds to 00, 0, 10, and 1
        \item \textbf{Zero or more copies (star)}: Using the start indicates zero or more copies, thus
            \begin{align*}
                a*
            .\end{align*}
            corresponds to any string of a's: $\{\epsilon, a,aa,aaa,...\} $
        \item \textbf{More on union}:
            If you form an RE by the or of two REs, call them $R$ and $S$, then the resulting language is the union of the languages of $R$ and $S$.
            \bigbreak \noindent 
            Suppose $R = (0+1) = \{0, 1\}$, and $S=\{01(0+1)\}  = \{ 010,011\}$, then $R+S = (0+1) + (01(0+1))  = \{0,1,010, 011\}$
        \item \textbf{More on concatenation}: If you form an RE by the or of two REs, call them $R$ and $S$, then the resulting language consists of all strings that can be formed by taking one string from the language of $R$ and one string from the language of $S$ and concatenating them.
            \bigbreak \noindent 
            Suppose $R = (0+1) = \{0, 1\}$, and $S=\{01(0+1)\}  = \{ 010,011\}$, then $RS = (0+1)01(0+1) = \{0010,0011,1010,1011\}$
        \item \textbf{More on star}: If you for man RE by taking the star of an RE $R$, then the resulting language consists of all strings that can be formed by taking any number of strings from the language of $R$ (they need not be the same and they need not be different), and concatenating them.
            \bigbreak \noindent 
            Suppose $R = 01(0+1) = \{010, 011\}$, then $R^{*} = 01(0+1)* \{010, 010010, ..., 011,011011,... 010011, ...\} $
    \item \textbf{Precedence of the operations}
        \begin{enumerate}
            \item Star (*)
            \item Concatenation
            \item Union (+)
        \end{enumerate}
        \item \textbf{Recursive definition of the kleene star (closure) ($L^{*}$)}:
            \begin{enumerate}
                \item $\epsilon \in L^{*} $
                \item If $x \in L^{*}$ and $y\in L$, then $xy \in L^{*}$
            \end{enumerate}
            \bigbreak \noindent 
            \textbf{Base case:} The first rule provides a starting point by ensuring that the empty string \( \epsilon \) is in \( L^* \).
            \bigbreak \noindent 
            \textbf{Recursive step:} The second rule allows you to take any string \( x \) already in \( L^* \) and concatenate it with a string \( y \in L \) to produce a new string \( xy \in L^* \).
            \bigbreak \noindent 
            After using the second rule once to generate a new string \( xy \in L^* \), you can apply the rule again by concatenating this new string with another string from \( L \). This recursive process can continue indefinitely, generating all possible strings that can be formed by concatenating zero or more strings from \( L \).
        \item \textbf{Recursive definition of the kleene star (other)}
            \begin{enumerate}
                \item $L^{0} = \{\epsilon\}$ $\quad$ (Start with the empty string, always in the closure)
                \item $L^{i}=LL^{i-1}$ for $i>0$ $\quad$ (Start recursively building strings)
                \item $L^{*} = \bigcup_{i=0}^{\infty} L^{i}$ $\quad$ (the whole thing)
            \end{enumerate}
            \bigbreak \noindent 
            \textbf{Note:} We also define the positive closure of \( L \), denoted \( L^+ \), as \( L^* - \{\epsilon\} \) or
            \[
                L^+ = \bigcup_{i=1}^{\infty} L^i.
            \]
        \item \textbf{Closure of the empty language}: $\Phi^{*} = \{\epsilon\} $
        \item \textbf{Regular expression for the empty language}: $\Phi  = \varnothing$  is the regular expression for the empty language (empty set)
        \item \textbf{More on language composition operators}:
            The language composition operators were defined over any language and, in turn, generate new languages. As such, composition operators take any one or two languages from $P(\Sigma^{*})$ and can produce any language in $P(\Sigma^{*})$.
        \item \textbf{Regular languages (regular sets), regular expression limits}: Although regular expressions are based on language composition operators, their recursive definition (i.e., only regular expressions, therefore only languages defined by regular expressions) limits the languages that they can define.
            \bigbreak \noindent 
            \textbf{Note:} Regular expressions cannot produce all languages in $P(\Sigma^{*})$.
            \bigbreak \noindent 
            In fact, the set of languages that regular expressions can define have a special name - they are called regular languages (or sometimes regular sets).

        \item \textbf{Kleene's theorem}:
            There is an FA for a language if and only if there is an RE for the language
        \item \textbf{Regular expressions order of operations}: From highest to lowest precedence
            \begin{enumerate}
                \item Parenthesis  
                \item Kleene star
                \item Concatenation
                \item Union ($+$ or $\mid$)
            \end{enumerate}
    \item \textbf{Properties of regular expressions}:
        \bigbreak \noindent 
        \textbf{Note:} Intersection is a operation not defined for regular expressions
        \bigbreak \noindent 
        \textbf{Union}
        \begin{itemize}
            \item \textbf{Commutative}:
                \[
                    R_1 \cup R_2 = R_2 \cup R_1
                \]

            \item \textbf{Associative}:
                \[
                    (R_1 \cup R_2) \cup R_3 = R_1 \cup (R_2 \cup R_3)
                \]

            \item \textbf{Identity Element}:
                \[
                    R_1 \cup \emptyset = R_1
                \]

            \item \textbf{Idempotent}:
                \[
                    R_1 \cup R_1 = R_1
                \]
        \end{itemize}
        \textbf{2. Concatenation (\(\cdot\))}
        \begin{itemize}
            \item \textbf{Non-commutative}:
                \[
                    R_1 \cdot R_2 \neq R_2 \cdot R_1
                \]

            \item \textbf{Associative}:
                \[
                    (R_1 \cdot R_2) \cdot R_3 = R_1 \cdot (R_2 \cdot R_3)
                \]

            \item \textbf{Identity Element}:
                \[
                    R_1 \cdot \epsilon = \epsilon \cdot R_1 = R_1
                \]

            \item \textbf{Concatenation with \(\emptyset\)}:
                \[
                    R_1 \cdot \emptyset = \emptyset \cdot R_1 = \emptyset
                \]
        \end{itemize}
        \bigbreak \noindent 
        \textbf{Kleene Star (\(*\))}
        \begin{itemize}
            \item \textbf{Kleene Star of \(\epsilon\)}:
                \[
                    \epsilon^* = \{\epsilon\}
                \]

            \item \textbf{Kleene Star of \(\emptyset\)}:
                \[
                    \emptyset^* = \{\epsilon\}
                \]

            \item \textbf{Idempotent}:
                \[
                    (R^*)^* = R^*
                \]
        \end{itemize}
        \bigbreak \noindent 
        \textbf{Distributive Properties}
        \begin{itemize}
            \item \textbf{Union over Concatenation}:
                \[
                    R_1 \cdot (R_2 \cup R_3) = (R_1 \cdot R_2) \cup (R_1 \cdot R_3)
                \]

            \item \textbf{Concatenation over Union}:
                \[
                    (R_1 \cup R_2) \cdot R_3 = (R_1 \cdot R_3) \cup (R_2 \cdot R_3)
                \]
        \end{itemize}
    \item \textbf{Language of a RE notation}: $L(RE)$ is the language defined by the regular expression $RE$, If we have an RE $R$, then the language $L(R)$ is the language defined by the RE $R$
    \item \textbf{When a regular expression is the empty set $\varnothing$}: When a regular expression (RE) represents the empty set it means that the RE matches no strings at all, not even the empty string.
        \bigbreak \noindent 
        The language is then
        \begin{align*}
            L(\varnothing) = \Phi        
        .\end{align*}
        Where $\Phi$ denotes the empty language
    \item \textbf{One or more occurences $RR^{*}$}: We denote this by plus instead of star, ie $RR^{*} = R^{+}$, but you also must redefine union as $\mid$ instead of $+$
    \item \textbf{Simplifying regular expressions (Some can also be found above in properties)}:
        \begin{itemize}
            \item \textbf{Concatenation of stars}: $(R^{*})^{*}  = R^{*}$
            \item \textbf{Concatenation of Repeated Expressions}: $R^{*}R^{*} = R^{*} $
            \item \textbf{Idempotence of Union}: $R\mid R = R$
            \item \textbf{Empty Set in Union and Concatenation:} $R \mid \varnothing  = R$, $R\varnothing = \varnothing $
            \item \textbf{Empty string in concatenation}: $\epsilon R = R\epsilon = R $
            \item \textbf{Union with the kleene star}: $R^{*}\mid R = R^{*}$
            \item \textbf{Distributive Property:} $R_{1}(R_{2} \mid R_{3}) = R_{1}R_{2} \mid R_{1}R_{3} $
            \item \textbf{Absorption:} $R\mid (RR^{*})  = RR^{*} = R^{+}$
        \end{itemize}
    \item \textbf{The RE operators with the empty language $\Phi$}:
        \begin{enumerate}
            \item $\varnothing r  = r\varnothing = \varnothing\varnothing = \varnothing$ for any regular expression $r$
            \item $r + \varnothing = \varnothing + r = r $
            \item $\varnothing + \varnothing = \varnothing $
            \item $\varnothing^{*} = \{\epsilon\} $
        \end{enumerate}
        These cases can also be represented with language notation
        \begin{enumerate}
            \item $\Phi L  = L\Phi = \Phi\Phi = \Phi$\ $\forall L$
            \item $L + \Phi = \Phi + L = L $
            \item $\Phi + \Phi = \Phi $
            \item $\Phi^{*} = \{\epsilon\} $
        \end{enumerate}
    \item \textbf{Convert RE to NFA-$\epsilon$}: The conversion algorithm starts by defining an NFA with $\epsilon$-moves for each of the three base cases from the recursive definition of a regular expressions over an alphabet $\Sigma$
        \begin{enumerate}
            \item $\varnothing$ is a regular expression and denotes the empty set (i.e., the empty language $\Phi$)
            \item $\epsilon$ is a regular expression and denotes the set $\{\epsilon\}$
            \item For each symbol $x \in \Sigma$, $x$ is a regular expression and denotes the set $\{x\} $.
        \end{enumerate}
        \bigbreak \noindent 
        Conditions on the NFAs with $\epsilon$-moves for This Algorithm
        \begin{enumerate}
            \item  There must be exactly one accepting state.
            \item No transitions (not even $\epsilon$-moves) may leave the one accepting state.
        \end{enumerate}
        \bigbreak \noindent 
        \textbf{Note:} If faced with an NFA with $\epsilon$-moves that has more than one accepting state and/or accepting states with transitions leaving it then simply modify the NFA with $\epsilon$-moves by
        \begin{enumerate}
            \item Adding a new accepting state.
            \item Add an $\epsilon$-move from each of the original accepting states to the newly added accepting state.
            \item Convert all of the original accepting states to non-accepting states.
        \end{enumerate}
        \bigbreak \noindent 
        The three base cases have the following nfa that satisfy the above criteria
        \bigbreak \noindent 
\begin{figure}[ht]
    \centering
    \incfig{crit2}
    \label{fig:crit2}
\end{figure}
\bigbreak \noindent 
We use those NFAs as the basic building blocks to iteratively build more complex NFA's with $\epsilon$-moves (all the while honoring the accepting state conditions for this algorithm) as we apply the recursive part of the regular expression definition. Recall:
\bigbreak \noindent 
If $r$ and $s$ are regular expressions denoting the sets $R$ and $S$, respectively, then
\begin{enumerate}
    \item $r+s$ is a regular expression denoting the set $R + S$, (i.e,. union of languages),
    \item $rs$ is a regular expression denoting the set $RS$ (i.e., concatenating languages), and
\item $r*$ is a regular expression denoting the set $R^{*}$ (i.e., Kleene closure of a language).
\end{enumerate}
\bigbreak \noindent 
Once we define an NFA with $\epsilon$-moves for each of the base cases (which we have done) then when we address each recursive part of the definition (e.g., union above)
\begin{enumerate}
    \item We may assume that there already exists NFAs with $\epsilon$-moves for each of the regular expressions $r$ and $s$ (and that each also satisfies the acceptance state conditions of this algorithm) and 
    \item Then our job is to use those NFAs with $\epsilon$-moves to create a new NFA with $\epsilon$-moves that accepts $r+s$ and that also satisfies the acceptance state conditions of this algorithm.
\end{enumerate}
\bigbreak \noindent 
\textbf{The algorithm:}
\begin{itemize}
    \item \textbf{Handling union:} We start by assuming there already exists NFAs with $\epsilon$-moves $M_{1}$ and $M_{2}$ that accept regular expressions $r$ and $s$, respectively, and that both $M_{1}$ and $M_{2}$ satisfy the acceptance state conditions (i.e., one accepting state, no exit) of this algorithm.
        \bigbreak \noindent 
        \begin{figure}[ht]
            \centering
            \incfig{crit3}
            \label{fig:crit3}
        \end{figure}
        \bigbreak \noindent 
        \textbf{Note:} The details of the machine arn't important here, all we know is the machine has a start, does whatever else it needs to (repesented by the elipsis), and then accepts strings represented by $r$ in the top machine and $s$ in the bottom
        \bigbreak \noindent 
        We then use these machines $M_{1}$ and $M_{2}$ to create a new machine $M$ that accepts $r+s$
        \bigbreak \noindent 
        \begin{figure}[ht]
            \centering
            \incfig{crit4}
            \label{fig:crit4}
        \end{figure}
        \bigbreak \noindent 
        So what did we do here
        \begin{enumerate}
            \item Create new start state and add $\epsilon$-moves to the original start states
            \item Create new accepting state and add emoves from all the original accepting states.
            \item Change the original accepting states to non-accepting states.
        \end{enumerate}
    \item \textbf{Handle Concatenation}: We again start by assuming there already exists NFAs with $\epsilon$-moves $M_{1}$ and $M_{2}$ that accept regular expressions $r$ and $s$, respectively, and that both $M_{1}$ and $M_{2}$ satisfy the acceptance state conditions (i.e., one accepting state, no exit) of this algorithm.
        \bigbreak \noindent 
        \begin{figure}[ht]
            \centering
            \incfig{crit5}
            \label{fig:crit5}
        \end{figure}
        \bigbreak \noindent 
        We use $M_{1}$ and $M_{2}$ to construct new NFA with $\epsilon$-move $M$ that accepts $rs$.
        \bigbreak \noindent 
        \begin{figure}[ht]
            \centering
            \incfig{crit6}
            \label{fig:crit6}
        \end{figure}
        \bigbreak \noindent 
        \begin{enumerate}
            \item Add an $\epsilon$-move from $M_{1}$'s accepting state to $M_{2}$'s start state.
            \item Change $M_{1}$'s accepting state to a nonaccepting state.
        \end{enumerate}
        \pagebreak 
    \item \textbf{Handle Kleene closure}: We again start by assuming there already exists an NFA with $\epsilon$-moves $M_{1}$ that accepts regular expressions $r$ and that satisfies the acceptance state conditions (i.e., one accepting state, no exit) of this algorithm.
        \bigbreak \noindent 
        \begin{figure}[ht]
            \centering
            \incfig{crit7}
            \label{fig:crit7}
        \end{figure}
\end{itemize}
\bigbreak \noindent 
We use $M_{1}$ to construct new NFA with $\epsilon$-move $M$ that accepts $r^{*}$
\begin{figure}[ht]
    \centering
    \incfig{crit9}
    \label{fig:crit9}
\end{figure}
\bigbreak \noindent 
\begin{enumerate}
    \item Create new start and accepting states.
    \item Add $\epsilon$-move from new start to $M_{1}$ start, $M_{1}$ accepting to new accepting, and new start to new accepting.
    \item Add $\epsilon$-move from $M_{1}$ accepting to $M_{1}$ start. 
    \item Change $M_{1}$'s accepting state to a non accepting state.
\end{enumerate}

\item \textbf{RE to NFA conversion: Special cases}: Recall the special case with regular expressions, the empty language $\Phi$ - and how it behaved with the three regular expression operators;
    \begin{enumerate}
        \item $\Phi L  = L\Phi = \Phi\Phi = \Phi$\ $\forall L$
        \item $L + \Phi = \Phi + L = L $
        \item $\Phi + \Phi = \Phi $
        \item $\Phi^{*} = \{\epsilon\} $
    \end{enumerate}
    \bigbreak \noindent 
    \pagebreak \bigbreak \noindent 
    We can now check these operations using the algorithm with $\Phi $. First, we define the base case NFA's
    \bigbreak \noindent 
    \begin{figure}[ht]
        \centering
        \incfig{base}
        \label{fig:base}
    \end{figure}
    \bigbreak \noindent 
    \begin{enumerate}
        \item Confirming $L+\Phi = \Phi+L= L$ and $\Phi+\Phi = \Phi$:
            \bigbreak \noindent 
            \begin{figure}[ht]
                \centering
                \incfig{base2}
                \label{fig:base2}
            \end{figure}
            \pagebreak 
        \item Confirming $\Phi L = L\Phi = \Phi\Phi = \Phi$
            \bigbreak \noindent 
            \begin{figure}[ht]
                \centering
                \incfig{base6}
                \label{fig:base6}
            \end{figure}
        \item Confirming $\Phi^{*} = \{\epsilon\}$
            \bigbreak \noindent 
            \begin{figure}[ht]
                \centering
                \incfig{base7}
                \label{fig:base7}
            \end{figure}
    \end{enumerate}

    \item \textbf{Convert NFA-$\epsilon$ to RE}:
        If necessary, first modify the NFA with $\epsilon$-moves to satisfy these two conditions (i.e., conditions of this algorithm, not requirements of all NFA's with $\epsilon$-moves);
        \begin{enumerate}[label=\alph*)]
            \item No transition may enter the start state - not even a loop. 
            \item  If there exists even one accepting state, then there can be only one accepting state and no transition may leave that accepting state.
        \end{enumerate}
        \bigbreak \noindent 
        \textbf{Note:} If there was no accepting state then do not create one, stop the algorithm, and output the regular expression $\varnothing$ to denote the empty language $\Phi$.
        \bigbreak \noindent 
        Then we start the algorithm. We need to convert the label on each transition to a regular expression until there is only two states, a start state and an accepting state, and all transitions between these two states are regular expressions. The final regular expression will be the union of all transitions.
        \begin{enumerate}
            \item  While there are more "middle" states (i.e., states that are neither the start state or accepting state)
                \begin{enumerate}[label=(\roman*)]
                    \item Select one of the remaining middle states.
                    \item Bypass the middle state creating new transitions as necessary annotating each new transition with a regular expression.
                    \item Remove the bypassed middle state.
                \end{enumerate}
            \item If there are any transitions between the start and accepting state, then the regular expression that accepts the same language as the original FA is the the union (i.e., "+") of the regular expressions of all the transitions. 
                \bigbreak \noindent 
                If there are no transitions between the start and accepting state, then output the regular expression $\varnothing$ to denote the empty language $\Phi$.
        \end{enumerate}
        \bigbreak \noindent 
        Recall the following from the definition of regular expressions;
        \begin{enumerate}[label=(\roman*)]
            \item \relax [base case]: $L$ is a regular expression and denotes the set $\{L\}$
            \item  \relax [base case]: For each symbol $x\in\Sigma$, $x$ is a regular expression and denotes the set $\{x\}$
            \item  \relax [recursive case]: $r+s$ is a regular expression denoting the set $R + S$, (i.e,. union of languages).                       
        \end{enumerate}
        \bigbreak \noindent 
        We use those to covert every $\epsilon$ to a $\Lambda$, every symbol $x\in\Sigma$ to a regular expression of the same symbol, and every case comma-separated transition label to a regular expression with "+".
        \bigbreak \noindent 
        After ensuring the FA abides by the start and end state conditions, and we convert every transition to the simple regular expressions, we begin eliminating states.
        \bigbreak \noindent 
        \textbf{Some notes:}
        \begin{enumerate}[label=(\alph*)]
            \item Regarding the middle states (states that are neither the start nor accepting state), it doesn't matter in which order we choose to eliminate them.
            \item For each state we are eliminating, we count the number of incoming and outgoing transitions (loops don't add to the count but we still need to take care of them with the regular expressions), there will be a new regular expression transition for all combinations of outgoing and incoming transitions. Ie pick a state to eliminate, then
                \begin{align*}
                    \text{New RE transitions} = N(\text{outgoing}) \times N(\text{incoming}) 
                .\end{align*}
                Not including the loops
        \end{enumerate}
        \pagebreak \bigbreak \noindent 
        \textbf{Example:} Consider the NFA-$\epsilon$
        \bigbreak \noindent 
        \begin{figure}[ht]
            \centering
            \incfig{re1}
            \label{fig:re1}
        \end{figure}
        \bigbreak \noindent 
        Before we begin eliminating states, we see that this FA does not obey the two constraints described above. So, we create a new accepting state such that there is only one accepting state. Each old accepting state has $\epsilon$ transitions to this new accepting state. This FA has no incoming transitions to the start state so nothing to fix there.
        \bigbreak \noindent 
        \begin{figure}[ht]
            \centering
            \incfig{re2}
            \label{fig:re2}
        \end{figure}
        \bigbreak \noindent 
        Now, we start eliminating states one at a time. We recall that it does not matter the order in which we eliminate them.
        \bigbreak \noindent 
        We start by eliminating state $X$. There is one incoming transition and one outgoing transition. Thus, there is $1\times 1 = 1$ new transition. To get from the start state, through $X$, to the accepting state, the regular expression is $aa^{*}\Lambda = aa^{*}$. Thus,
        \bigbreak \noindent 
\begin{figure}[ht]
    \centering
    \incfig{re3}
    \label{fig:re3}
\end{figure}
\bigbreak \noindent 
Next, we choose to eliminate state $Z$. We have one incoming and two outgoing transitionss. Thus, we have $1\times 2  = 2$ new regular expression transitions. To get from $Y$ through $Z$ to the accepting state, the RE is $b\epsilon =  b$. To go from $Y$ through $Z$ back to $Y$, we have $ba$. Thus
\bigbreak \noindent 
\begin{figure}[ht]
    \centering
    \incfig{re4}
    \label{fig:re4}
\end{figure}
\bigbreak \noindent 
Finally, we eliminate $Y$. To get from the start state, through $Y$, to the accepting state, the regular expression is $a(ba)^{*}b$. Thus, 
\bigbreak \noindent 
\begin{figure}[ht]
    \centering
    \incfig{re5}
    \label{fig:re5}
\end{figure}
\bigbreak \noindent 
The final regular expression is then 
\begin{align*}
    aa^{*} + \epsilon + a(ba)^{*}b
.\end{align*}
\blacksquare










    \end{itemize}

    \pagebreak 
    \subsubsection{Properties of regular languages}
    \begin{itemize}
        \item \textbf{Recall: Regular language}: Recall that we call a language a regular language if, and only if, the language is accepted by some regular expression. 
        \item \textbf{Recall: Recursive definition of regular expressions}: Recall also our recursive definition of regular expressions over some alphabet $\Sigma $
            \bigbreak \noindent 
            Let $\Sigma$ be an alphabet. The regular expressions over $\Sigma$ and the sets (i.e., languages) that they denote are defined recursively as follows:
            \bigbreak \noindent 
            \textbf{Base cases:}
            \begin{enumerate}
                \item $\varnothing$ is a regular expression and denotes the empty set (i.e., the empty language $\Phi$).
                \item $L$ is a regular expression and denotes the set $\{L\}$.
                \item For each symbol $x\in \Sigma$, $x$ is a regular expression and denotes the set $\{x\}$.
            \end{enumerate}
            \bigbreak \noindent 
            \textbf{Recursion:} If $r$ and $s$ are regular expressions denoting the sets $R$ and $S$, respectively, then
            \begin{enumerate}
                \item $r+s$ is a regular expression denoting the set $R + S$, (i.e,. union of languages),
                \item $rs$ is a regular expression denoting the set $RS$ (i.e., concatenating languages), and
                \item $r^{*}$ is a regular expression denoting the set $R^{*}$ (i.e., Kleene closure of a language).
            \end{enumerate}
        \item \textbf{Relationship between regular languages and the set of all possible languages $\mathcal{P}(\Sigma^{*})$}:
            We know that the set of regular languages must be a subset of $\mathcal{P}(\Sigma^{*})$ (it may be equal to $\mathcal{P}(\Sigma^{*})$), and so, we can start by creating that subset.
            \bigbreak \noindent 
            We can then start noting the languages that we know are regular based on the base cases from the definition of regular expressions and for some given alphabet, say $\Sigma = \{a, b\}$.
            \bigbreak \noindent 
            The recursion from the definition tells us that we can take any language, or pair of languages, from the already existing set of regular languages, and use it/them to create a new language this is also a regular language. And the recursion may be applied over and over (i.e., without limit), always taking only regular languages that have been previously created (original base case languages or languages subsequently derived), to create new languages. (i.e., the set of regular languages is infinite).
            \bigbreak \noindent 
        \item \textbf{Closure of regular languages and their operations}: Expanding on the item above, formally, we say that the set of regular languages is \textbf{closed} under the language composition operations union, concatentation, and Kleene star
        \item \textbf{Closure of complement and intersection}: In addition to the three operations that come from the definition of regular expressions (i.e., union, concatenation, and Kleene star), the set of regular languages is also closed under;
            \begin{itemize}
                \item Complement
                \item Intersection
            \end{itemize}
        \item \textbf{Complement of a Language}: Recall that every language is a set of strings - empty, finite, or infinite - that is always a subset of $\Sigma^{*}$
            \bigbreak \noindent 
            That is,
            \begin{itemize}
                \item For any alphabet $\Sigma$ we get $\Sigma^{*} $
                \item We define some language $L$ over that alphabet $\Sigma$
                \item Then $L$ is a subset of $\Sigma^{*}$; $L \subseteq \Sigma^{*}$
            \end{itemize}
            \bigbreak \noindent 
            We define a new language, the complement of $L$, denoted $L^{\prime}$ as the set of strings that are not in the language $L$ (i.e., $L^{\prime} = \Sigma^{*}  - L$).
        \item \textbf{Proof: Regular languages are closed under complement}: We assert that if you take the complement of a regular language, the resulting language is then regular
            \bigbreak \noindent 
            \textbf{Proof:}
            \begin{enumerate}
                \item A regular language is one that is accepted by some regular expression.
               \item By Kleene's Theorem we know that any regular expression can be converted to a NFA with $\epsilon$-moves that accepts the same language, and vice versa.
                \item  We can convert any NFA with $\epsilon$-moves to a DFA that accepts the same language and every DFA is, by definition, an NFA with $\epsilon$-moves. 
            \end{enumerate}
            So $L$ is a regular language iff it is accepted by some DFA...
            \bigbreak \noindent 
            Consider some DFA $M(Q, \Sigma, q_{0}, T, \delta)$ that accepts regular language $L$.
            \bigbreak \noindent 
            We construct a new DFA $M^{\prime}(Q, \Sigma, q_{0}, T^{\prime}, \delta)$ from M by defining $T^{\prime} = Q - T$, that is, every accepting state in $M$ becomes a non- accepting state in $M^{\prime}$, and vice versa.
            \bigbreak \noindent 
            Since $M^{\prime}$ accepts $L^{\prime}$ and $M^{\prime}$ is a DFA, then $L^{\prime}$ is a regular language.
            \bigbreak \noindent 
            Since M was chosen arbitrarily, the complement of any regular language is also a regular language
            \bigbreak \noindent 
            \blacksquare
            \bigbreak \noindent 
            \textbf{Note:} Note: The proof must be based on DFA's ... would not have worked for non-deterministic FA's.
        \item \textbf{Intersection of Languages}:
            Given any two languages, $L_{1}$ and $L_{2}$, over some alphabet $\Sigma$ we can create a new language $L$ that is the intersection of the two sets $L_{1}$ and $L_{2}$.
            \bigbreak \noindent 
            That is, 
            \begin{align*}
                L = L_{1} \cap L_{2} = \{x:\ x \in L_{1} \land x\in L_{2} \}
            .\end{align*}
        % \item \textbf{Proof: Regular languages are closed under intersection}: This means that when you take the intersection of any two regular languages the language you produce is always regular. 
        \item \textbf{Proof: Regular languages are closed under intersection}: This means that when you take the intersection of any two regular languages, the resulting language is always regular.
            \bigbreak \noindent 
            Assume you have two regular languages \(L_1\) and \(L_2\).
            \bigbreak \noindent 
            We define a new language \(L\), which is the intersection of \(L_1\) and \(L_2\), namely
            \[
                L = \{x \mid x \in L_1 \land x \in L_2\},
            \]
            \begin{center}
                where ``\(\land\)'' means "logical and."
            \end{center}
            \bigbreak \noindent 
            Demorgan's law:
            \begin{align*}
                (a \land b) = \sim(\sim a \lor \sim b)
            .\end{align*}
            \bigbreak \noindent 
            To prove that \(L = L_1 \cap L_2\) is regular, we use the fact that regular languages are closed under complement and union. This leads us to apply De Morgan's Law:
            \[
                L_1 \cap L_2 = \sim (\sim L_1 \cup \sim L_2).
            \]
            \bigbreak \noindent 
            Here, \(\sim L_1\) and \(\sim L_2\) represent the complements of \(L_1\) and \(L_2\), respectively. Since \(L_1\) and \(L_2\) are regular, their complements \(\sim L_1\) and \(\sim L_2\) are also regular (because regular languages are closed under complement).
            \bigbreak \noindent 
            Next, since regular languages are closed under union, the language \(\sim L_1 \cup \sim L_2\) is also regular.
            \bigbreak \noindent 
            Finally, the complement of \(\sim L_1 \cup \sim L_2\), i.e., \((\sim L_1 \cup \sim L_2)'\), is also regular because regular languages are closed under complement. But \((\sim L_1 \cup \sim L_2)'\) is precisely \(L_1 \cap L_2\).
            \bigbreak \noindent 
            Thus, \(L = L_1 \cap L_2\) is regular, as required.
            \[
                L_1 \cap L_2 = \sim (\sim L_1 \cup \sim L_2) = \{x \mid x \in L_1 \land x \in L_2\}.
            \]
            \bigbreak \noindent 
            \(\blacksquare\)
        \item {Addings to regular expression recursive}: Thus, we add complement and intersection to recursive cases when building regular languages defined in $\mathcal{P}(\Sigma^{*})$
            \bigbreak \noindent 
            \textbf{Recursion:} If $r$ and $s$ are regular expressions denoting the sets $R$ and $S$, respectively, then
            \begin{enumerate}
                \item $r+s$ is a regular expression denoting the set $R + S$, (i.e,. union of languages),
                \item $rs$ is a regular expression denoting the set $RS$ (i.e., concatenating languages), and
                \item $r^{*}$ is a regular expression denoting the set $R^{*}$ (i.e., Kleene closure of a language).
            \end{enumerate}
            \bigbreak \noindent 
            or by taking the complement or intersection of one or two previously created regular languages.


            % \bigbreak \noindent 
            % Assume you have two regular languages \(L_1\) and \(L_2\).
            % \bigbreak \noindent 
            % We define a new language \(L\) that is the intersection of \(L_1\) and \(L_2\), namely \(L = \{x \mid x \in L_1 \land x \in L_2\}\),
            % \begin{center}
            %     where ``\(\land\)'' means ``logical and''.
            % \end{center}
            % \bigbreak \noindent 
            % By DeMorgan's Law (logic) we know that \((a \land b) \equiv (\sim a \lor \sim b)\),
            % \begin{center}
            %     where ``\(\sim\)'' means ``logical negation, or not'' and ``\(\lor\)'' means ``logical or''.
            % \end{center}
            % \bigbreak \noindent 
            % We can use DeMorgan's Law to re-write \(L\) as follows:
            % \[
            %     L = \{x \mid x \in L_1 \land x \in L_2\} = \sim \{x \mid x \notin L_1 \lor x \notin L_2\} = \sim(L_1' \cup L_2')'
            % \]
            % \bigbreak \noindent 
            % Since \(L_1\) and \(L_2\) are regular languages, then we know that their complements \(L_1'\) and \(L_2'\) are also regular languages.
            % \bigbreak \noindent 
            % Since \(L_1'\) and \(L_2'\) are regular languages, then we know that their union \(L_1' \cup L_2'\) is also a regular language.
            % \bigbreak \noindent 
            % Since \(L_1' \cup L_2'\) is a regular language, then we know its complement \((L_1' \cup L_2')' = L\) is also a regular language.
            % \bigbreak \noindent 
            % Since \(L_1\) and \(L_2\) were two arbitrarily chosen regular languages, then the intersection of any two regular languages is also a regular language.
            % \bigbreak \noindent 
            % It then follows from $L = \{x \mid x \in L_1 \land x \in L_2\} = \sim \{x \mid x \notin L_1 \lor x \notin L_2\} = \sim(L_1' \cup L_2')'$
            % \begin{align*}
            %     \sim(L^{\prime}_{1} \cup L^{\prime}_{2})     &= L_{1} \cap L_{2}
            % .\end{align*}
            %
            % \blacksquare
    \end{itemize}

    \pagebreak 
    \subsubsection{Applications of finite automata}
    \begin{itemize}
        \item \textbf{FA with output}:
            Thus far we have only considered finite automata as language acceptors (i.e., defining some regular language).
            \bigbreak \noindent 
            Finite automata can also serve another purpose. They can be used to process an input string to produce some output.
            \bigbreak \noindent 
            When used in this way, the finite automata do not define a language. In fact, they do not have any accepting states.
            \bigbreak \noindent 
            Their sole purpose is to process input to generate output
        \item \textbf{Moore machine}: A Moore machine is a deterministic finite automaton and is
            defined by a 6-tuple $(Q, \Sigma, q_{0}, \delta, \Gamma, O)$, where
            \begin{itemize}
                \item $\Gamma$ is an alphabet of output symbols.
                \item $O$ is the output function: $O:\ Q \to \Gamma$
            \end{itemize}
            \bigbreak \noindent 
            Each state is annotated with an output symbol. Output "printed" upon entering state
            \bigbreak \noindent 
            \textbf{Note:} start state's output symbol always "printed", even on empty string $\epsilon$
        \item \textbf{Mealy Machine}: A Mealy machine is a deterministic finite automaton and is defined by a 5-tuple $(Q, \Sigma, q_{0}, \delta, \Gamma)$, where 
            \begin{itemize}
                \item $\Gamma$ is an alphabet of output symbols
            \end{itemize}
            \bigbreak \noindent 
            Each transition is annotated with an output symbol Output "printed" when traversing edge
            \bigbreak \noindent 
            \textbf{Note:} The number of input and output symbols are always identical.
            \bigbreak \noindent 

    \end{itemize}

    \pagebreak 
    \subsection{Nonregular languages}
    \begin{itemize}
    \item \textbf{Intro to nonregular languages}: Suppose we have some alphabet $\Sigma$, we can then find $\Sigma^{*}$, which is the language consisting of all possible strings (including the empty string $\epsilon$) using the symbols from $\Sigma$. The we can take the power set of $\Sigma^{*}$ to get the set of all possible subsets of $\Sigma^{*}$. The alphabet $\Sigma$ is always finite, if $\Sigma$ is nonempty, then $\Sigma^{*}$ is always infinite. (If $\Sigma = \varnothing$, $\Sigma^{*} = \{\epsilon\} $). Since $\Sigma^{*}$ is infinite, $\mathcal{P}(\Sigma^{*})$ is also infinite. If $\Sigma = \varnothing \implies \Sigma^{*} = \{\epsilon\} \rightarrow \mathcal{P}(\Sigma^{*}) = \{\{\epsilon\}, \varnothing\}$. Note that for any set $S$, $\varnothing \subset S$.
        \bigbreak \noindent 
        What types of languages are contained in $\mathcal{P}(\Sigma^{*})$? 
        \begin{enumerate}
            \item If a language is finite, it is always regular. This is a consequence of Kleene's theorem, which asserts that if a regular expression expresses a language, the language is regular. If a language is finite, we can \textbf{always} make a regular expression for it. Simply take all strings in the language, and take the union. For example, if $L = \{a,ab,abc\}$, then a regular expression for the language is simply $a + ab + abc$ and the language is therefore regular. $\quad \blacksquare$ 
            \item We also know that there are infinitely many regular expressions than can express infinitely many languages from $\mathcal{P}(\Sigma^{*})$. This is a consequence of the recursive definition of regular expressions. Therefore there are infinitely many regular expressions to describe infinite languages.
        \end{enumerate}
        So, $\mathcal{P}(\Sigma^{*})$ is the infinite set of all possible languages, finite or infinite, that can be generated from a given language $\Sigma$. All Finite languages from this set are regular, and there are also infinitely many infinite languages from this set that are regular. But, are there any languages that are \textit{not} regular? Infinite languages that cannot be expressed as a regular expressions? 
        \bigbreak \noindent 
        It may seem unlikely that nonregular languages exist at all. To claim that a language is nonregular one must prove that no regular expression or FA that accepts the language exists
        \bigbreak \noindent 
        \begin{figure}[ht]
            \centering
            \incfig{p2}
            \label{fig:p2}
        \end{figure}
        \pagebreak 
    \item \textbf{The Pumping Lemma}: Before we continue our conversation about nonregular languages we first look at a lemma about regular languages
        \bigbreak \noindent 
        \textbf{Lemma.} Let $L$ be an infinite regular language, then
        \begin{align*}
            \exists\ x,y,z \in \Sigma^{*} \mid y\ne \epsilon \land xy^{n}z \in L\ \forall\ n> 0
        .\end{align*}
        In other words, all regular languages have the property that, there exists some strings $x,y,z$, where $y\ne \epsilon$ such that all the strings of the form $xy^{n}z \in L$ for all $n>0 $
        \bigbreak \noindent 
        This means that for some $x,y,z$, we can infintely pump in more copies of $y$, and the string remains in the language. For example,
        \begin{align*}
            xyz \quad xyyz \quad xyyyz \quad ...
        .\end{align*}
    \item \textbf{Pumping Lemma Proof}: Assume you have some regular language $L$ with infinitely many strings. Because $L$ is regular, there must exists some DFA $M(Q, \Sigma, q_{0}, T, \delta) $ that accepts $L$. Since FA's are required to have finite states, let's say it has $n$ states.
        \bigbreak \noindent 
        Because $L$ is infinite, we can always find strings that have length greater than $n$. Ie $\abs{w} \ge n$, where $w\in L$. Because $w$ has at least as many characters as there are states in $M$, as we process $w$ with $M$ we know that one or more of the states in $M$ must be revisited. We know that as we process $w$, it must traverse what we call the \textit{circuit}. which is a sequence of one or more edges that contains at least one state that is visited more than once.
        \bigbreak \noindent 
        Given that circuit and because we also selected $w \in L$, we note that we can modify $w$ to create a new word $w^{\prime}$ pumping into $w$ as many symbols as are needed and in just the right location in $w$ that would cause $M $ to traverse the circuit one more time than it did when processing $w$. We note that the new word $w^{\prime} \in L$.
        \bigbreak \noindent 
        In fact, given DFA $M$ with $n$ states and string $w \in L$, $|w| \geq n$, we can generate an infinite supply of new words by simply pumping into $w$, and $t$ the right location, more and more copies of the string that causes $M$ to traverse the circuit. We note that the new words created in this way are all in $L$.
        \bigbreak \noindent 
        This gives us the existence of the $x,y,z$ strings the Pumping Lemma needs as follows:
            \begin{itemize}
                \item $x$ is the prefix of $w$ that is consumed by $M$ as the DFA wanders up to the circuit (x may be $\Lambda$ and this sequence of states may be empty).
                \item $y$ is the substring of $w$ that is consumed by $M$ as the DFA traverses the circuit (since the circuit must visit at least one state more than once, it must consume some symbols, and so $y$ cannot be $\Lambda$).
                \item $z$ is the suffix of $w$ that is consumed by $M$ as the DFA leaves the circuit and goes to an accepting state (z may be $\Lambda$ and this sequence of states may be empty).
            \end{itemize}
            \bigbreak \noindent 
            Therefore, $L$ must contain all the strings of the form $xy^{n}z$ for all $n>0$.
            \bigbreak \noindent 
            \pagebreak \bigbreak \noindent 
\begin{figure}[ht]
    \centering
    \incfig{p3}
    \label{fig:p3}
\end{figure}
\bigbreak \noindent 
\item \textbf{The value of the pumping lemma}: The Pumping Lemma tells us something that is true of all regular infinite languages. 
    \bigbreak \noindent 
    The real value of The Pumping Lemma is to prove that some infinite language is nonregular, that is a language \textbf{cannot} be regular if it does not satisfy the claim in the pumping lemma. Thus, we can prove that a language is non-regular by contradiction.
    \bigbreak \noindent 
    \begin{enumerate}
        \item Assume the language is regular
        \item You show that it is not possible to find strings $x,y,z$ that satisfy the claim in the pumping lemma.
        \item We conclude that our assumption that the original language is a regular language must be false, and therefore, it must be nonregular.
    \end{enumerate}
\item \textbf{Note on the pumping lemma}: The pumping lemma gaurantees that for a large enough $w\in L$ ($\abs{w} \geq n$), where $n$ is the number of states in the assumed machine) we can find an $x,y,z$ such that $y\ne \epsilon$ and $xy^{n}z \in L\ \forall\ n > 0$. If the word you select is greater than or equal to $n$, and no such $y$ holds, then the language must be nonregular
    \bigbreak \noindent 
    The condition is $ \geq n$, because, for a machine with $n$ states, there are $n-1$ edges that must be traversed to reach the end. If a machine has four states, then there are $4-1 = 3$ edges to reach the end.
\item \textbf{The pumping language with length: Background}: In the case that you find an $x,y,z$ such that $y \ne \epsilon$ and $xy^{n}z \in L\ \forall\ n> 0$. This does \textbf{not} mean the language must be regular. We know that all regular languages do have this property, but that does not mean that simply possessing this property makes the language regular. In logic theory
    \begin{align*}
        a \rightarrow b 
        \not\implies b\rightarrow a
    .\end{align*}
    In words, if $a$ implies $b$, $b$ does not imply $a$
    \bigbreak \noindent 
    Thus, the pumping lemma described above is sometimes not enough, and we may need something more powerful.

\item \textbf{The pumping lemma with length}:
    Let $L$ be an infinite language accepted by a FA with $n$ states. Then, for all $w\in L$, where $\abs{w} \geq n$, there exists some three strings $x,y,z$ such that $w=xyz$, $y\ne \epsilon$, $\abs{xy} \leq n$, and all the strings of the form $xy^{i}z \in L$ for all $i>0$
    \bigbreak \noindent 
    The Pumping Lemma With Length adds that you must always be able to find $x,y,z$ in all sufficiently long words $w\in L$
    This means:
    \begin{itemize}
        \item For each word in $L$ that has length greater than $n$, where $n$ is the number of states in the assumed machine, there must be a composition $xyz$, where $x,y,z \in \Sigma^{*}$, $y\ne \epsilon$, and $\abs{xy} \leq n$. The $x,y,z$ need not be the same for every  word in $L$ with length greater than $n$, but a pair needs to exist for each word.
        \item Furthermore, for each word $w$, where $\abs{w} \geq n$, that has $x,y,z$ such that $w = xyz$. That same $x,y,z$ needs to have the property $xy^{i}z \in L \ \forall\ i>0$.
        \item Thus, to show that a language is not regular, we only need to show that such an $x,y,z$ does not exist for a single word. If we choose a word and $x,y,z$ exists, we need to keep looking.
    \end{itemize}
    If words keep leading to valid $x,y,z$ at some point we need to stop looking and start trying to prove that the language is actually regular. Whether by creating an RE or an FA. Keep in mind there will be infinite words to check.
    \end{itemize}

    \pagebreak 
    \subsubsection{Pumping lemma examples}
    \begin{itemize}
 \item \textbf{Pumping lemma example}: Use the pumping lemma to show that the language $L = a^{k}b^{k}\ \forall\ k > 0$ over $\Sigma = \{a,b\}$ is nonregular
    \bigbreak \noindent
    Suppose that $L$ is regular, then we should be able to find some $x,y,z$, where $y \ne \epsilon$ such that $xy^{n}z \in L\ \forall\ n > 0$. For simplicity, let's first examine the possible choices for $y$
    \begin{enumerate}
        \item $y = a^{\ell}$ or $y=b^{\ell}$ for some $\ell > 0$
        \item $y = a^{\ell}b^{\lambda}$ for some $\ell, \lambda > 0$
    \end{enumerate}
    In case one, pumping would lead to an imbalance in the number of $a$'s or $b$'s. In case two, pumping would lead to more than one $ab$ boundary.
    \bigbreak \noindent 
    Thus, no such $x,y,z$ exists and the language is nonregular $\quad \blacksquare $
    \bigbreak \noindent 
\item \textbf{Pumping lemma example:} Show that the language $L =  \{a^{t}:\ t \in \mathbb{P}\} $ over $\Sigma = \{a\}$ is nonregular.
    \bigbreak \noindent 
    Suppose $L$ is regular. Then we will find strings $x,y,z$ such that $y\ne \epsilon$ and $xy^{n}z \in L\ \forall\ n>0$.
    \bigbreak \noindent 
     The only choice of $y$ in this case is $y=a^{r},\ r>0$. 
     \bigbreak \noindent 
    First, define $w = a^{t} = xyz,\ t \in \mathbb{P}$. That is, since $w$ is a member of $L$, we can partition it into the form $a^{t} = xyz$. Furthermore, $xy^{n}z \in L\ \forall\ n>0$. Since this needs to hold \textbf{for all $n>0$}, showing that it doesn't work for a single $n>0$ breaks the argument. Let $n = t+1$. This leads us to some algebreic manipulations
    \begin{align*}
        xy^{n}z &= xy^{t+1}z = xyy^{t}z \\
                &=xyzy^{t} \quad \text{(Everything is $a$, we can commute)} \\
                &=a^{t}y^{t} \\
                &=a^{t}(a^{r})^{t} \\
                &= a^{t + rt} \\
                &= a^{t(1+r)}
    .\end{align*}
    This next assertion is one of number theory. We assert that if $t\in \mathbb{P}$, $t(r+1) \not\in \mathbb{P}$ (Since $r>0$, $r+1 > 0 $). 
    \bigbreak \noindent 
    Since we only have one choice of $y$ in this case, and we showed that it does not hold when $n=t+1$, the language must be irregular. $\quad \blacksquare $
 
\item \textbf{Using the Pumping Lemma With Length: Palindrome example}: Prove that the language \textit{Palindrome} over $\Sigma = \{a,b\} $ is a nonregular language.
    \bigbreak \noindent 
    Assume there exists an FA with $n$ states that accepts palindrome. Consider $w = a^{n+1}ba^{n+1} \in $ Palindrome
    \bigbreak \noindent 
    Since we assume $L$ is regular, then for $w=a^{n+1}ba^{n+1}$, which is clearly a palindrome with length $\geq n$, must have the property $w=xyz$, for some $x,y,z  \in \Sigma^{*}$, where $y\ne \epsilon$, and $\abs{xy} \leq n$. Thus, $y$ must be contained within $a^{n+1}$, which implies $xy$ must be contained within $a^{n+1}$. This leads to the conclusion that pumping more copies of $y$ will lead to an $a$ imbalance to the left of the $b$, which is \textbf{not} a palindrome. Thus, the language is nonregular. $\quad \blacksquare$.
\item \textbf{Pumping lemma with length example}: 
    Show with the pumping lemma that $L = a^{n}b^{m}c^{n+m},\ \forall\ m,n > 0$ is a nonregular language
    \bigbreak \noindent 
    The pumping lemma states that for an infinite regular language that has an FA with $k$ states, then
    \begin{align*}
        \forall w \in L,\ \abs{w} \ge n,\ \exists\ x,y,z\in \Sigma^{*} \mid w =xyz,\ \abs{xy} \le n \land xy^{i}z \in L \ \forall\ i > 0
    .\end{align*}
    Let $m,n = k$, then $w = a^{k}b^{k}c^{2k}$. Since $\abs{xy} \le n$, $y$ must be $a^{r},\ 0 < r \le k$. This implies $w = a^{r}a^{k-r}b^{k}c^{2k}$. Furthermore, $w^{\prime} = a^{ir}a^{k-r}b^{k}c^{2k} \in L\ \forall\ i > 0$. If $i = 2$, $w^{\prime} = a^{2r}a^{k-r}b^{k}c^{2k} = a^{k+r}b^{k}c^{2k}$. For $w^{\prime}$ to be in $L$, $2k$ must equal $k+r + k$. Since $2k \ne 2k + r$, we have a contradiction. Thus, pumping more copies of $y$ not satisfy $a^{n}b^{m} c^{n+m}$, as the number of $c$'s would not equal the number of $a$'s plus the number of $b$'s $\quad \blacksquare$
\item \textbf{Pumping lemma with length example}: 
    Over the alphabet $\Sigma = \{a,b,c\}$, show that the language that houses all strings that are not palindromes is nonregular.
    \bigbreak \noindent 
    Assume the language has an FA with $k$ states, then $\forall\ w \in L$, $\abs{w} \leq k,\ \exists \ x,y,z\in \Sigma^{*} \mid w = xyz, \abs{xy} \le k \ \land \ xy^{i}z \in L \ \forall \ i>0$.
    \bigbreak \noindent 
    An easy way to prove this is by showing that $L = \text{Palindrome}$ is nonregular, which is much simpler. Since regular languages are closed under complement, the complement of a regular language must be regular. Thus, the complement of a nonregular language must be nonregular.
    \bigbreak \noindent 
    Assume $L = \text{palindrome}$ is infinite and regular defined by an FA with $k$ states.
    \bigbreak \noindent 
    Let $w=a^{k}b^{k}a^{k}$, then $y=a^{r}, \ 0 < r \leq k$. Then $w = a^{r}a^{k-r}b^{k}a^{k}$ and $w^{\prime} = a^{ir}a^{k-r}b^{k}a^{k} \in L \ \forall \ i>0$. If $i=2$, $w^{\prime} = a^{2r}a^{k-r}b^{k}a^{k} = a^{k-r+2r}b^{k}a^{k} = a^{k + r}b^{k}a^{k}$. Since $r >0,\ k  + r > k \implies a^{k+r} > a^{k}$ and $w^{\prime}$ is not a palindrome. Thus, we have a contradiction  and $L$ must be nonregular. Since $L$ is nonregular, $L^{\prime}$ must be nonregular. Therefore, the language of all strings that are not palindromes is nonregular. $\quad \blacksquare$
\item \textbf{Pumping lemma with length example}: Show that the language $a^{t},\ t \in \mathbb{P}$ over the alphabet $\Sigma = \{a\}$ is nonregular.
    \bigbreak \noindent 
    Assume the language is regular defined by a FA with $n$ states. Choose $w=a^{\ell},\ \ell \in \mathbb{P},\ \ell \geq n$. Then, $y$ must be $a^{r},\ r>0$. From this, $w = xyz = a^{r}a^{\ell - r}$, and $w^{\prime} = xy^{i}z \in L \ \forall \ i > 0$. Thus, $w^{\prime} = a^{ir}a^{\ell - r} \in L \ \forall \ i > 0$. If we can show that some selection of $i$ yields a $w^{\prime} \not\in L$, then we have a contradiction and the language must be nonregular.
    \bigbreak \noindent 
    Choose $i = \ell  + 1$, which implies $w^{\prime} = a^{(\ell + 1)r}a^{\ell -r} = a^{\ell -r  + \ell r + r }  = a^{\ell + \ell r} = a^{\ell (1 + r)}$. Since $\ell \in \mathbb{P}$, and $r>0$, $\ell (1 + r)$ cannot be prime. Thus, we have a contradiction and the assumption does not hold for $L = a^{t},\ t\in \mathbb{P}$.
    \end{itemize}

    \pagebreak 
    \subsection{Context free grammers}
    \begin{itemize}

        \item A \textbf{Context Free Grammar} (CFG) is a 4-tuple \((V, \Sigma, S, P)\), where
            \begin{itemize}
                \item \(V\) is a non-empty finite set of \textit{variables}
                \item \(\Sigma\) is a finite alphabet of \textit{terminals}\\
                    (we assume \(V\) and \(\Sigma\) are disjoint)
                \item \(S \in V\), is the \textit{start variable}
                \item \(P\) is a finite set of \textit{productions} of the form \(A \rightarrow \alpha\) where
                    \begin{itemize}
                        \item \(A\) is a variable (i.e., \(A \in V\)) and
                        \item \(\alpha\) is a string of symbols from \((V \cup \Sigma)^*\) (i.e., \(\alpha \in (V \cup \Sigma)^*\)).
                    \end{itemize}
            \end{itemize}   
        \item \textbf{CFG notational convention}: 
            Variables are upper case letters with S always being the start variable.
            \bigbreak \noindent 
            Terminals are lower case letters, symbols, or constants, including $\epsilon$ to denote the empty symbol.
        \item \textbf{A simple CFG}: Consider the following CFG that has two productions
            \begin{align*}
                &S \to 0S1 \\
                &S \to \epsilon
            .\end{align*}
        \item \textbf{Derivations}:
            We say that a finite string $w$, consisting only of terminals, is generated by a CFG if, starting with the start variable $S$, you can apply a sequence of productions that result in $w$.
            \bigbreak \noindent 
            The sequence is called a derivation of $w$.
        \item \textbf{Derivation examples}: All derivations must start with the start variable $S$.
            \bigbreak \noindent 
            As long as there is at least one variable in our string we must continue the derivation by
            \begin{enumerate}
                \item Selecting a variable from our string,
                \item Selecting a production whose left side matches the variable we selected from our string, and
                \item Replacing the variable in our string with the right side of the production we selected.
            \end{enumerate}
            \bigbreak \noindent 
            With the two productions
            \begin{align*}
                &S \to 0S1 \\
                &S \to \epsilon
            .\end{align*}
            We select the first production and apply it 
            \begin{align*}
                S \Rightarrow 0S1 \Rightarrow 00S11 \Rightarrow 00\epsilon11 = 0011
            .\end{align*}
            This yields the string $w=0011$. Thus, we have derived this string from the grammer. And we also note that there are some other strings for which there is no derivation using our CFG
            \bigbreak \noindent 
            In short, we observe that a CFG defines a language over its alphabet of terminals $\Sigma$.
        \item \textbf{Context Free Languages}: A language is a context free language if it is generated by some context free grammar.
        \item \textbf{Is palindrome context free?}: Recall that the language Palindrome (i.e., the set of all strings that read the same forward as they do backward) is a non-regular language.
            \bigbreak \noindent 
            We can create a grammer for this language
            \begin{align*}
                &S \to aSa \\
                &S \to bSb \\
                &S \to a \\
                &S \to b \\
                &S \to \epsilon 
            .\end{align*}
            \bigbreak \noindent 
            The CFG builds the string from the middle always pushing outward by inserting the same pair of characters each time.
        \item \textbf{Notational convenience}: We can express the grammer above simply as
            \begin{align*}
                &S \to aSa \mid bSb \mid a \mid b \mid \epsilon
            .\end{align*}
            Where the pipe deliminates each production. Note that they are still separate productions, but since they share the same variable $S$, we can put them on the same line separated by a pipe.
        \item \textbf{Connection between regular and context free languages}: We assert that every regular language is generated by a context free grammar.
            \bigbreak \noindent 
            \textbf{Proof:} There is a recursive algorithm that converts any regular expression to a context free grammar that accepts the same language.
            \begin{itemize}
                \item If the overall language is the union of two or more pieces, then you can start with $S \to A_{1} | A_{2} | ... | A_{n}$
                \item If the overall language is the concatenation of two or more pieces, then you can start with $S \to A_{1}A_{2}\cdots A$
                \item If the overall language is the Kleene star of a piece, say generated by $E$, then you can start with $S \to ES \ \mid \ \epsilon$
                \item Recursively and similarly generate productions for variables until you are left with only terminals.
            \end{itemize}
            \bigbreak \noindent 
            \textbf{Example:} 
            Consider the regular expression \((11 + 00)^* 11\)
            \bigbreak \noindent 
            At its top level it is a concatenation of \((11 + 00)^*\) and \(11\), so we start with the following production:
            \[
                S \rightarrow AB
            \]
            We continue by generating productions for each new variable we introduced \(A\) and \(B\).
            \bigbreak \noindent 
            \(A\) is a Kleene Star and \(B\) is straightforward so we add the following productions:
            \[
                A \rightarrow CA \mid \varepsilon
            \]
            \[
                B \rightarrow 11
            \]
            All that remains is production for the variable \(C\) which is a union, so we add this final production.
            \[
                C \rightarrow 11 \mid 00
            \]
            \bigbreak \noindent 
            Thus, all regular languages are context free, but not all context free languages are regular, we know that there exist some nonregular languages that are context free, like palindrome.
        \item \textbf{Left and rightmost derivations}:
            \begin{itemize}
                \item \textbf{Leftmost derivation:} At each stage one replaces the leftmost variable.
                \item \textbf{Rightmost derivation:} At each stage one replaces the rightmost variable.
            \end{itemize}
        \item \textbf{Derivation (parse) trees}: It is sometimes useful to display derivations as trees called derivation trees or parse trees. Where a parse tree satisfies
            \begin{itemize}
                \item The root of the tree is always the CFG start symbol $S$
                \item All of the internal nodes (i.e., non-leaf nodes) are variables.
                \item All of the leaf nodes are terminals.
                \item The children of a node are ordered left to right and appear in the same order as they do in the right-hand-side of a production whose left-hand-side matches variable in the parent node.
                \item The word generated by the derivation tree is the sequence of terminals read from the leaf nodes in left-to-right order.
            \end{itemize}
        \item \textbf{Ambiguous parsing}: If for some words in a language there are more than one parse trees, we say it has ambiguous parsing
        \item \textbf{Ambiguous grammer}: A context free grammar (CFG) defining some language $L$ is said to be an ambiguous grammar iff there exists some word $w\in L$ that has two different derivation trees.
            \bigbreak \noindent 
            Because each derivation tree represents unique leftmost and rightmost derivations;
            \begin{itemize}
                \item a CFG is ambiguous iff there exists some word $w\in L$ for which there are two different leftmost derivations or two different rightmost derivations.
            \end{itemize}
        A CFG that is not ambiguous is said to be an unambiguous grammar
    \end{itemize}

    \pagebreak 
    \subsubsection{Pushdown automata (PDA)}
    \begin{itemize}
        \item \textbf{Intro to PDA}: PDA's are an abstract machine that is used as an acceptor for CFG's, similar to how FAs were used as acceptors for regular languages.
        \item \textbf{Comparison to NFA}: PDAs are closest to NFA's with $\epsilon$-moves in that they both:
            \begin{itemize}
                \item are nondeterministic (need only one set of choices to accept a string),
                \item may crash (i.e., not every symbol must leave each state), and
                \item allow for $\epsilon$-moves
            \end{itemize}
            Recall that NFA with $\epsilon$-moves also:
            \begin{itemize}
                \item                 Have finitely many states, some of which are accepting states.
                \item                 Read string from an input tape, must read entire string to accept.
                \item                 Reject by reaching the end of string while not in an accepting state.
            \end{itemize}
            All the same is true for pushdown automata with the following minor revisions and one important addition:
            \begin{itemize}
                \item \relax [revision] There is only one accepting state, once entered accept string (whether the entire input string has been read or not).
                \item \relax [revision] There is no notion of reaching the "end of the string".
                \item \relax [addition] There is a stack of infinite capacity.
            \end{itemize}
        \item \textbf{The Tape and Stack}: Pushdown automata have a stack of infinite capacity, initially empty.
            \bigbreak \noindent 
            Because pushdown automata have both an input tape and stack they must distinguish which they are manipulating in each of their states; read, push, or pop.
            \pagebreak 
        \item \textbf{State Symbols}:
            \begin{itemize}
                \item \textbf{Start state}: No incoming edges, one outgoing edge
                \item \textbf{Accepting}: state No outgoing edges
                \item \textbf{Read}: (nondeterministic)
                \item \textbf{Pop}: (nondeterministic)
                \item \textbf{Push}: One outgoing edge
            \end{itemize}
            \bigbreak \noindent 
            \begin{figure}[ht]
                \centering
                \incfig{pda1}
                \label{fig:pda1}
            \end{figure}
\item \textbf{Initialization}: Like finite automata, we start pushdown automata by writing the string onto the input tape. Recall that the input tape has a beginning and an infinite capacity.
    \bigbreak \noindent 
    In pushdown automata, after we write the string onto the input tape we append an infinite sequence of a special symbol $\Delta$ to denote the end of the string (i.e., assumed that $\Delta$ is not part of the input alphabet).
    \bigbreak \noindent 
    We initialize the pushdown automaton's stack with an infinite supply of the special symbol $\Delta$.
    \bigbreak \noindent 
    We place the pushdown automaton in its START state
    \pagebreak 
\item \textbf{PDA Example}: Consider the PDA that accepts the language $0^{n}1^{n} \ \forall \ n >0 $
    \bigbreak \noindent 
    \begin{figure}[ht]
        \centering
        \incfig{pda2}
        \label{fig:pda2}
    \end{figure}
    \bigbreak \noindent 
    It starts by reading all of the leading 0's and using the STACK to ``count'' how many 0's have been read (i.e., by pushing an $x$ for each 0 that was read).
    \bigbreak \noindent 
    Once we encounter our first 1, we enter a different part of the PDA in which we confirm there is an $x$ on the stack for every 1 we read from the tape.
    \bigbreak \noindent 
    If we READ a 1 and then pop something other than a $x$, then the PDA crashes in the POP state and rejects the string. This happens when there are more 1's than 0's.
    \bigbreak \noindent 
    Once we encounter our first $\Delta$ then we have reached the end of the input string and we are assured that there are at least as many 1's as there are 0's.
    \bigbreak \noindent 
    We still must POP the stack to confirm that there are no $x$'s. If there was an $x$ on the STACK then there were more 0's than 1's.

    \end{itemize}

    \pagebreak 
    \subsubsection{Chomsky Normal Form}
    \begin{itemize}
        \item \textbf{Intro}: Noam Chomsky showed that it is possible to convert any CFG (i.e., all productions of the form $A\to v$ to another CFG that accepts the same language where all the productions are either
            \begin{align*}
                &A\to \text{ exactly two variables} \\
                &\text{or} \\
                &A\to \text{ one terminal (the terminal cannot be $\epsilon$)} 
            .\end{align*}
            Which converts the CFG into Chomsky Normal Form (CNF)
            \bigbreak \noindent 
            The conversion is done in four steps which must be in this sequence:
            \begin{enumerate}
                \item Eliminate null productions
                \item Eliminate unit productions
                \item Almost CNF
                \item CNF
            \end{enumerate}
        \item \textbf{Eliminate Null Productions}: A null production is of the form $A\to \epsilon$.
            \bigbreak \noindent 
            We start our conversion to Chomksy Normal Form (CNF) by first eliminating all null productions and replacing each eliminated null production with other new production(s).
            \bigbreak \noindent 
            The basic idea is that the new production(s) we are adding to replace a null production $N\to \epsilon$ come from other productions where N appears on the right.
            \bigbreak \noindent 
            If such a production exists, then we add new productions that have all possible subsets of \(N\) deleted, for example:
            \[
                X \rightarrow aNb \quad \text{adds one production:} \quad X \rightarrow ab
            \]
            \[
                X \rightarrow aNbNc \quad \text{adds three productions:} \quad X \rightarrow abNc \ \mid\ aNbc \ \mid\ abc
            \]
            If \(N\) appears \(k\) times in a production then add \(2^k - 1\) new productions (i.e., every possible combination to remove \(N\)).
            \bigbreak \noindent 
            How do we remove all null productions from a grammar? One possible strategy is to remove each one at a time ... but that can be problematic. 
        \item \textbf{Nullable variable}: The solution is to eliminate all the null productions at the same time, but that requires a new definition.
            \bigbreak \noindent 
            Given a context free grammar and a variable $N$ in that grammar,
            we call $N$ a nullable variable iff:
            \begin{enumerate}
                \item There is a production $N\to \epsilon$ or
                \item There is a derivation that starts with $N$ and leads to $\epsilon$
            \end{enumerate}
        \item \textbf{Identify all nullable variables}: How do we identify all of the nullable variables in a context free grammar? By essentially taking a transitive closure.
            \begin{enumerate}
                \item For every null production, "paint" the variable that appears on the left side of the production \textcolor{red}{"red"}. These are all nullable variables.
                \item "Paint" every occurrence of every nullable variable \textcolor{red}{"red"} throughout the entire grammar, including occurrences that appear on the right side of productions.
                \item "Paint" any variable that appears on the left side of a production \textcolor{red}{"red"} if the right side of the production is all \textcolor{red}{"red"}. These are nullable variables.
                \item Repeat steps 2 and 3 until you have painted nothing else \textcolor{red}{"red"}.
            \end{enumerate}
        \item \textbf{Eliminate the null productions}: How do we eliminate all null productions from a grammar?
            \begin{enumerate}
                \item Identify all the nullable variables.
                \item Remove all the null productions.
                \item For every production that has a nullable variable in its right side, add new productions all with the same left side and new right sides with every possible subset of the nullable removed, but do not allow a new null production to be produced, even if the only symbol on the right side of the production is a nullable variable (i.e., do not add $Y\to \epsilon$ even if there is production $Y\to A$ and $A$ was found to be nullable).
            \end{enumerate}
        \item \textbf{Eliminating null productions example}: Consider the following grammar
            \begin{align*}
                &S \to XY \\
                &X \to Zb \\
                &Y \to bW \\
                &W \to Z \\
                &Z \to AB \\
                &A \to aA | bA | \epsilon \\
                &B \to Ba | Bb | \epsilon
            .\end{align*}
            The nullables are $W,Z,A,B$ and the two null productions are $A\to \epsilon$ and $B\to \epsilon$
            \bigbreak \noindent 
            We have identified the nullables. Now, we remove the null products $A\to \epsilon$, $B\to \epsilon$
            \begin{align*}
                &S \to XY \\
                &X \to Zb \\
                &Y \to bW \\
                &W \to Z \\
                &Z \to AB \\
                &A \to aA | bA \\
                &B \to Ba | Bb 
            .\end{align*}
            \bigbreak \noindent 
            Finally, for every production that has a nullable variable in its right side, add new productions to the left side with every possible subset of the nullable removed.
            \begin{align*}
                 &S \to XY \\
                &X \to Zb \ \mid \ b \\
                &Y \to bW \\
                &W \to Z \\
                &Z \to AB \ \mid \ A \ \ \mid \ B\\
                &A \to aA \ \mid \ bA \ \mid \ a \ \mid \ b \\
                &B \to Ba \ \mid \ Bb \ \mid \ a \ \mid b
            .\end{align*}
        \item \textbf{Eliminate unit productions}: A production of the form \textit{one variable} $\to$ \textit{one variable} is called a unit production.
            \bigbreak \noindent 
            Given a context free grammar with no null productions, it is possible to create a new context free grammar that accepts the same language and that has no unit productions.
            \bigbreak \noindent 
            For every pair of variables \(A\) and \(B\), if the CFG has a unit production \(A \rightarrow B\) or if there is a derivation from \(A\) to \(B\),
            \[
                A \Rightarrow X_1 \Rightarrow \dots \Rightarrow X_n \Rightarrow B
            \]
            where each \(X_i\) is a single variable, then we introduce the following productions according to the following rule:
            \[
                \text{If all the non-unit productions from } B \text{ are}
            \]
            \[
                B \rightarrow S_1 \mid S_2 \mid \dots \mid S_m
            \]
            then add the productions
            \[
                A \rightarrow S_1 \mid S_2 \mid \dots \mid S_m
            \]
            Conclude by removing all unit productions from the CFG.
            \bigbreak \noindent 
            Consider the following grammar
            \begin{align*}
                &S\to A
                &A\to a \ \mid \ B \ \mid \ b \ \mid \ C
                &B\to b \ \mid \ C \ \mid \ AAa
                &C\to A
            .\end{align*}
            Split the grammar into unit productions and non-unit productions as follows:
            \begin{align*}
                &S \to A \\
                &A\to B \ \mid \ C \quad \quad A\to a \ \mid \ b \\
                &B\to C \quad \quad \quad \quad  B\to b \ \mid \ AAa \\
                &C\to A
            .\end{align*}
            \bigbreak \noindent 
            We must consider each unit production, one at a time, and make longer and longer derivation chains that end in a single variable without introducing any cycles (e.g., do not allow $A \Rightarrow B \Rightarrow... \Rightarrow B)$
            \bigbreak \noindent 
            New grammar with unit productions removed and new productions added
            \begin{align*}
                &S \to a | b | AAa \\
                &A \to a | b | AAa \\
                &B \to a | b | AAa \\
                &C \to a | b | AAa
            .\end{align*}
        \item \textbf{Almost CNF}: If $L$ is a language generated by some CFG then there is another CFG that generates all the non null words in $L$ all of whose productions are of one of two basic forms:
            \begin{center}
                \textit{Variable} $\to$ \textit{string of only variables} or \textit{Variable} $\to$ \textit{one terminal}
            \end{center}
            Start with a CFG having variables \(S, X_1, \dots, X_n\).
            \bigbreak \noindent 
            For each terminal (e.g., \(a, b, c\)) add a new production by introducing a new variable:
            \[
                A' \rightarrow a, \quad B' \rightarrow b, \quad C' \rightarrow c
            \]
            In all of the \textit{original} productions (i.e., only those that started with \(S, X_1, \dots, X_n\)) replace the terminals \textit{(not necessarily every terminal)} with the newly created associated variable.
            \bigbreak \noindent 
            We only rewrite productions with these new productions if the terminal in the production is not alone. For example $A \to Bc $ turns to $A \to BC^{\prime}$, but $A \to c$ stays the same.
        \item \textbf{Chomsky Normal Form}: Going from Almost CNF to CNF is easy
            \bigbreak \noindent 
            For every production of the form \(A \rightarrow X_1 X_2 \dots X_n\) where \(n > 2\):
            \begin{itemize}
                \item Introduce a new variable \(R\)
                \item Add new production \(R \rightarrow X_2 \dots X_n\)
                    \begin{itemize}
                        \item The new production has \(n-1\) variables on the right side.
                        \item Because the original production had \(n \geq 2\) variables, this new production will have \(n \geq 2\) variables (i.e., cannot introduce unit productions) on the right side.
                    \end{itemize}
                \item Re-write the original production \(A \rightarrow X_1 X_2 \dots X_n\) as \(A \rightarrow X_1 R\)
            \end{itemize}
            Repeat Step 1 as necessary, including applying it to any new productions that were generated by Step 1.
            \bigbreak \noindent 
            \textbf{Example}: 
            \begin{align*}
                A \to BCDE 
            .\end{align*}
            Becomes
            \begin{align*}
                &A \to BR_{1} \\
                &R_{1} \to CR_{2} \\
                &R_{2}\to DE
            .\end{align*}
        \item \textbf{Note about CNF}: A CFG converted to CNF will accept the exact same language \textit{except} the empty string $\epsilon$. If the original grammer accepted the empty string, the CNF grammer will not. If the original grammer did not accept the empty string, then the grammer in CNF will accept the exact same language.



    \end{itemize}

    \pagebreak 
    \subsection{Equivalence of Context-Free Grammars and Pushdown Automata}
    \begin{itemize}
        \item \textbf{Recall}: A language is generated by a context free grammar if and only if it is accepted by a pushdown automaton.
            \bigbreak \noindent 
        \item \textbf{Equivalence theorem}: The set of languages that can be defined by context free grammars - what we call the
            set of context free languages - is exactly the same set of languages that can be defined by pushdown
            automata.
            \bigbreak \noindent 
            \textbf{Claim:} There is no language that can be defined by a CFG for which there is no PDA that accepts the same language.
            \bigbreak \noindent 
            There is no language that can be defined by a PDA for which there is no CFG that accepts the same language.
        \item \textbf{Equivalence theorem proof}: The proof is in two parts and is similar in structure to the way
            we proved Kleene's Theorem - by constructive algorithm.
            \begin{enumerate}
                \item Given any CFG that accepts some language $L$, then there is an algorithm to convert the CFG to a PDA that accepts the same language $L$.
                \item Given any PDA that accepts some language $L$, then there is an algorithm to convert the PDA to a CFG that accepts the same language $L$.
            \end{enumerate}
        \item \textbf{Converting CFG to PDA}: We can convert any CFG to another CFG that is in Chomsky Normal Form (CNF) that accepts the same language, most of the time; there is one noted exception (the empty string $\epsilon$).
            \bigbreak \noindent 
            Recall that CNF requires that every production be in one of following two forms: \textit{Variable} $\to$ \textit{exactly 2 variables} or \textit{Variable} $\to$ \textit{one terminal}
            \bigbreak \noindent 
            Consider a CFG that defines language \( L_o \) and its conversion to another CFG in CNF that defines language \( L_c \),
            \[
                \text{If } \epsilon \in L_o \text{ then } L_c = L_o - \{\epsilon\}, \text{ otherwise } L_c = L_o.
            \]
            It is a simple matter to determine if the empty string \( \Lambda \) is in the original language;
            \[
                \epsilon \in L_o \text{ if and only if the start symbol } S \text{ in the original CFG is nullable}.
            \]
            \bigbreak \noindent 
            To convert any CFG to a PDA that accepts the same language we start by converting the CFG to CNF and taking note if we lost the empty string $\epsilon$ in the conversion.
            \pagebreak \bigbreak \noindent 
            We then start building our PDA, which always starts with the same five states (i.e., no matter what the CFG).
            \bigbreak \noindent 
            \begin{figure}[ht]
                \centering
                \incfig{starter}
                \label{fig:starter}
            \end{figure}
            \bigbreak \noindent 
            For every production in the CNF CFG of the form $X_{m}\to a_{n}$ we add a transition from the POP state and a READ state.
            \bigbreak \noindent 
            For every production in the CNF CFG of the form
            $X_{i} \to X_{j} X_{k}$ we add a transition from the POP state and
            two PUSH states, first PUSH $X_{k}$ and then PUSH $X_{j}$
            \bigbreak \noindent 
            If the language defined by the original CFG accepted the empty string $L$, then we add one more transition labeled $S$ from the POP state back to the POP state.
            \bigbreak \noindent 
            \begin{figure}[ht]
                \centering
                \incfig{starter2}
                \label{fig:starter2}
            \end{figure}
        \item \textbf{Example: Convert CNF CFG to PDA}: Consider the following CFG in CNF that accepts $a^{n}b^{m}$ for $n,m>0$. 
            \begin{align*}
                &S \to AB \\
                &A\to AA \ \mid \ a \\
                &B\to BB  \ \mid \ b
            .\end{align*}
            \bigbreak \noindent 
            Which converts to the PDA
            \bigbreak \noindent 
            \begin{figure}[ht]
                \centering
                \incfig{pda5}
                \label{fig:pda5}
            \end{figure}
            \bigbreak \noindent 
            We start building our PDA with the required five states.
            \bigbreak \noindent 
            We add the READ states for the two productions that have terminals on their right side.
            \bigbreak \noindent 
            We add the PUSH states for the three productions that have two variables on their right side.
            \bigbreak \noindent 
            Since the empty string $\epsilon$ is not in this language we do not have to add one more transition labeled $S$ from the POP state back to the POP state, so we are done.
        \item \textbf{Conversion between PDA and CFG}: Given an arbitrary PDA there exists an algorithm to convert it to a CFG that accepts the same language.
            \bigbreak \noindent 
            The algorithm is complex, and so, we will take it on faith that it exists and that it is correct.
        \item \textbf{Equivalence theorem}: We accept the theorem
            \bigbreak \noindent 
            A language is generated by a context free grammar if and only if it is accepted by a pushdown automaton.
        \item \textbf{Current state of  $\mathcal{P}(\Sigma^{*})$}:
            Regular languages are accepted by regular expressions and finite automata.
            \bigbreak \noindent 
            Context free languages are accepted by context free grammars and pushdown automata.
            \bigbreak \noindent 
            Don't forget ... CFGs and PDAs can also accept all regular languages.
            \begin{figure}[ht]
                \centering
                \incfig{rc1}
                \label{fig:rc1}
            \end{figure}
            \bigbreak \noindent 
            Where the regular languages are accepted by regular expressions and finite automata, and the context free languages are accepted by context free grammers and pushdown automata





    \end{itemize}

    \pagebreak 
    \subsection{Non context free languages}
    \begin{itemize}
        \item \textbf{Intro}:
            We know that set of regular languages has all the finite languages, $\Sigma^{*}$, $\Phi$, and infinitely many infinite languages.
            \bigbreak \noindent 
            We know that the set of context-free languages contains all the regular languages, some other languages that are not regular, and that all the context-free languages that are not regular languages are infinite languages.
            \bigbreak \noindent 
            But are there any non-context free languages?  if there are any, we know they must all be infinite languages
            \bigbreak \noindent 
            But what would that mean? 
            \bigbreak \noindent 
            A language is context free if and only if there exists a context-free grammar that accepts it, and
            \bigbreak \noindent 
                a language is generated by a context-free grammar if and only if it is accepted by a pushdown automaton.
                \bigbreak \noindent 
                If someone comes to us with a description of a language and asks us to develop a CFG or PDA that accepts the language, we work on it for a long time but fail to come with anything, what does that mean?
                \bigbreak \noindent 
                It may simply mean that the problem is too hard or that we are not clever enough ... it does not mean that no CFG or PDA exists.
                \bigbreak \noindent 
                To claim at a language is non-context free one must prove that no CFG or PDA exists (very different than trying for a time and failing).
        \item \textbf{Derivations From CNF CFGs}: Before we continue our conversation about the prospect of noncontext free languages we first look at derivations from context free grammars that are in Chomsky Normal Form (CNF).
            \bigbreak \noindent 
            Consider the derivations that come from a CFG in CNF
            \bigbreak \noindent 
            Every derivation starts with the grammar's start symbol (a variable) $S$ and applies one production at each stage generating a working string until the final stage when the word $w \in L$ defined by the CFG is produced.
            \bigbreak \noindent 
            Every application of a live production has the net effect of adding a variable to the working string.
            \bigbreak \noindent 
            Every application of a dead production has the net effect of replacing a variable (i.e., removing a variable) with a terminal in the working string
            \bigbreak \noindent 
            Any derivation of a non-empty word \( w \in L \) defined by the CFG in CNF whose length \( |w| = n > 0 \) must therefore always have exactly \( 2n - 1 \) steps, or applications of productions, as follows:
            \begin{itemize}
                \item \textcolor{green}{\( n - 1 \) applications of \textit{live productions} to convert the single variable (i.e., start symbol) \( S \) to the \( n \) variables needed for the subsequent \textit{dead productions}} and
                \item \textcolor{red}{\( n \) applications of \textit{dead productions}; one to convert each variable to a single terminal.}
            \end{itemize}
            \bigbreak \noindent 
            We now turn our attention to derivation trees
            \bigbreak \noindent 
            Consider a finite language with exactly one word $L=\{12345\}$, the following two CNF CFGs each accepting $L$, and the derivation trees using each grammar.
            \bigbreak \noindent 
            \begin{figure}[ht]
                \centering
                \incfig{helloworld2}
                \label{fig:helloworld2}
            \end{figure}
            \bigbreak \noindent 
            Using 
            \begin{align*}
                &S \to AB \\
                &A \to CD \\
                &B \to EF \\
                &C \to GH \\
                &D \to 3 \\
                &E \to 4 \\
                &F \to 5 \\
                &G \to 1 \\
                &H \to 2
            .\end{align*}
            \bigbreak \noindent 
            And 
            \begin{figure}[ht]
                \centering
                \incfig{inspace2}
                \label{fig:inspace2}
            \end{figure}
            \bigbreak \noindent 
            Using
            \begin{align*}
                &S \to AB \\
                &A \to CD \\
                &B \to 5 \\
                &C \to EF \\ 
                &D \to 4 \\ 
                &E \to GH \\
                &F \to 3 \\
                &G \to 1 \\
                &H \to 2
            .\end{align*}
            \bigbreak \noindent 
            We first note that both derivation trees have
            \begin{itemize}
                \item The expected number of live productions $(n-1 = 5-1 = 4)$ and
                \item The expected number of dead productions $(n = 5)$
            \end{itemize}
            \bigbreak \noindent 
            But we also note
            \begin{itemize}
                \item The shapes of the trees are different
                \item The depths of the trees are different
            \end{itemize}
            \bigbreak \noindent 
            But what is the shallowest depth a derivation tree can be to produce a word with length $n$ from a CFG in CNF?
            \bigbreak \noindent 
            In a derivation tree from a CFG in CNF, 
            \begin{itemize}
                \item \textbf{Live productions}: The \textit{branching factor}$ = 2$, children are internal nodes
                \item \textbf{Dead productions}: The \textit{branching factor}$ = 1$, children are leaf nodes
            \end{itemize}
            \bigbreak \noindent 
            The shallowest tree is one that first (i.e., from the root down) takes the fullest advantage of the live production's higher branching factor by pushing all the live productions to the top of the tree and pushing all the dead productions to the bottom.
            \bigbreak \noindent 
            So what is the shallowest depth a derivation tree can be to produce a word with length $k$ from a CFG in CNF? The answer is 
            \begin{align*}
                \lceil\log_{2}{(k)} + 2\rceil &= d \\
                \implies 2^{d-2} &= k
            .\end{align*}
            Where $d$ is the shallowest depth, with at least one path from root to leaf with $(\lceil \log_{2}{(k)}\rceil+2) -1 = \lceil\log_{2}{(k)}\rceil+1$ variables.
        \item \textbf{The Pumping Lemma for Context Free Languages}: Let \( L \) be any context-free language. Then there is a constant \( n \) that depends on \( L \) such that if \( z \in L \) and \( |z| \geq n \), then we may re-write \( z = uvwxy \) such that
            \begin{enumerate}
                \item \( |vwx| \leq n \),
                \item \( |vx| > 0 \), and
                \item for all \( i \geq 0 \), \( uv^i wx^i y \in L \).
            \end{enumerate}
        \item \textbf{The pumping lemma for context free languages proof}:
            Let \( G \) be any context-free grammar that accepts a context-free language \( L \).
            \bigbreak \noindent 
            Let \( G' \) be the CNF CFG that accepts \( L - \{\Lambda\} \) having \( m \) variables.
            \bigbreak \noindent 
            If \( L \) is an infinite language, we can always find a word \( z \in L \) that is longer than some arbitrary length, say the Pumping Lemma's constant \( n \).
            \bigbreak \noindent
            \textit{Set} \( n = 2^m \) \textit{and then select a word} \( z \in L \) \textit{such that} \( |z| \geq n \) \textit{(i.e.,} \( |z| \geq 2^m \)).
            \bigbreak \noindent 
            Consider the derivation tree of $z$ using the CNF CFG $G^{\prime}$.
            \bigbreak \noindent 
            Because \( |z| \geq 2^m \), we know that there is at least one path from root to leaf with at least \( \lfloor \log_2 2^m \rfloor + 1 \), or at least \( m + 1 \), variables.
            \bigbreak \noindent 
                Because the CNF CFG \( G' \) has only \( m \) variables, we know that same path must repeat at least one variable at least once.
                \bigbreak \noindent 
                We traverse that path from the leaf node up to the root noting the lowest and second lowest occurrence of the first repeated variable.
                \bigbreak \noindent 
                Derivation trees for words \( w \in L \) from a CNF CFG have a minimum depth that is based on \( |w| \), and from that, we also know how many variables there must be in at least one path from the root to a leaf (i.e., terminal).
                \bigbreak \noindent 
                If you pick a \( w \in L \) with \( |w| \geq 2^m \), where \( m \) is the number of variables in the CNF CFG, then you are guaranteed that the derivation tree will have a path from root to terminal in which a variable is repeated ... \textit{the repeated variable is our "cycle"}, and
                \bigbreak \noindent 
                We can use that cycle to generate an infinite number of words that all must also be in the language \( L \).
            \item \textbf{The Real Value of the Pumping Lemma for Context Free Languages}:
                The real value of the Pumping Lemma for CFLs is to prove that some infinite language is
                not context free ... here is the basic strategy - it is a proof by contradiction;
                \bigbreak \noindent 
                You start with some infinite language that you suspect may be non context free.
                \bigbreak \noindent 
                You assume that the language is a context free language.
                \bigbreak \noindent 
                This is a crucial step in a proof by contradiction. You assume something, run the proof from that
                assumption, come to a contradiction (something that cannot possibly be true), and then that proves that
                your assumption must be false.
                \bigbreak \noindent 
                You show that it is not possible to decompose a string into uvwxy such that ... and so on (i.e.,
                what The Pumping Lemma for CFLs assures us is always possible to do with infinite CFLs).
                \bigbreak \noindent 
                This gives us our contradiction; we assume the language is a CFL, which means the Pumping Lemma for
                CFLs is in effect, which means we can always decompose a string into uvwxy such that ... etc., but we
                find we cannot decompose the string into uvwxy , so contradiction.
                \bigbreak \noindent 
                We conclude that our assumption that the original language is a CFL must be false, and
                therefore, it must be a non context free language.
                \pagebreak 
            \item \textbf{Pumping lemma example}: Suppose $L = a^{k}b^{k}c^{k} \ \forall \ n \geq 0$
                \bigbreak \noindent 
                \textbf{Proof}. Suppose $L$ is context free, then $\exists \ n \in \mathbb{R}\  \mid \ \ \forall \ z \in L,\ \abs{z} \geq n \ z = uvwxy$ for $u,v,w,x,y \in \Sigma^{*}$, with $\abs{vx} > 0,\ \abs{vwy} \leq n$ and $z^{\prime} = uv^{i}wx^{i}y \in L \ \forall \ i \geq 0$.
                \bigbreak \noindent 
                Let $k=n+1$, then $z= a^{n+1}b^{n+1}c^{n+1}$. Since the middle portion of the decomposed string $z=uvwxy$, $vwx$ must be weakly less than $n$ in length, we have a couple cases for $v,x$. The first case is when $v$ and $x$ are all a's, all b's, or all c,s. In this case, pumping copies of $v$ and $x$ leads to an inbalance in one of the symbols. The second case is that $v$ and $x$ are homogenous in different symbols, for example $v$ is all a's, and $x$ is all b's. In this case, pumping would lead to an inbalance in two of the symbols. The final case is that $v$ or $x$  span some boundary. For example, $v = a^{r}b^{\ell}$. In this case, pumping would lead to more than one boundary, which leads strings not in $L$. Thus, there are no choices for $v,x$ such that $z^{\prime} = uv^{i}wx^{i}y \in L \ \forall \ i \geq 0$. Since we have a contradiction, $L$ must not be context free.
                \bigbreak \noindent 
            \item \textbf{PL example two}: Show that the language $L = a^{k}b^{k+1}a^{k} \ \forall \ n>0$ is not context free.
                \bigbreak \noindent 
                \textbf{Proof}: Assume $L$ is context free, then $\exists \ n \in \mathbb{Z}^+\  \mid \ \ \forall \ z \in L,\ \abs{z} \geq n \ z = uvwxy$ for $u,v,w,x,y \in \Sigma^{*}$, with $\abs{vx} > 0,\ \abs{vwy} \leq n$ and $z^{\prime} = uv^{i}wx^{i}y \in L \ \forall \ i \geq 0$.
                \bigbreak \noindent 
                Let $z = a^{n+1}b^{2(n+1)}a^{n+1}$, for $k=n+1$. If $v,x$ are homogenous substrings, for example $v=a^{r}$, $x=b^{\ell}$ for $ 0 \leq r < n +1, \ 0 < \ell < n+1 - r$, for all combinations of pairs of $a,b,a$. In this case, pumping would lead to one of the symbol sequences unchanged. For example, suppose $v=a^{r}$, $x=b^{\ell}$. Then, $v = a^{n+1-r}a^{r}b^{\ell}b^{2(n+1)}a^{n+1}$. Then  $z^{\prime} = a^{n+1-r+ir}b^{2(n+1)-\ell + i\ell} a^{n+1} \in L \ \forall \ i \geq 0 $. Let $i=2$, then $z^{\prime} = a^{n+1+r}b^{2(n+1)+\ell}a^{n+1}$, we see that the $a$ and $b$ at the beginning grow, while the $a$ at the end remains unchanged.
                \bigbreak \noindent 
                The other case is that either $v$ or $x$ is non-homogenous, and span across some boundary. For example, $v=a^{r}b^{\ell}$, and $x$ is some number of $b$'s. In this case, pumping would lead to additional $ab$ boundarys, which leads to strings not in the language. Since there are no choices for $v,x$ that satisifies the property of context free languages, we have a contradiction and the language must not be context free. $\blacksquare$
            \item \textbf{PL example three}: Consider $L = \{ww:\ w \in (0+1)^{*}\} $. In other words, the left half of the string must be the same as the right half. For example, $z = 00110011 \in L$.
                \bigbreak \noindent 
                \textbf{Proof}: Assume L is context free, then $\exists \ n \in \mathbb{Z}^+\  \mid \ \ \forall \ z \in L,\ \abs{z} \geq n \ z = uvwxy$ for $u,v,w,x,y \in \Sigma^{*}$, with $\abs{vx} > 0,\ \abs{vwy} \leq n$ and $z^{\prime} = uv^{i}wx^{i}y \in L \ \forall \ i \geq 0$.
                \bigbreak \noindent 
                Consider $z = 0^{n}1^{n}0^{n}1^{n}$. Since $\abs{vwx} \leq n$, the only hope of being able to satisfy the criterion is when $v,x$ span the middle of the string. If $v,x$ were contained entirely in the left or right half, then its clear that pumping would lead to a string not in $L$. Suppose then that $v,x$ are contained within the middle portion such that $v$ is $1^{r}$, and $x = 0^{\ell}$, and their lengths satisfy $\abs{uwx} \leq n$. Then, $z = 0^{n}1^{n-r}1^{r}0^{\ell}0^{n-\ell}1^{n}$, and $z^{\prime} = 0^{n}1^{n-r+ir}0^{n-\ell+i\ell}1^{n}$, let $i=0$, then $z^{\prime}  = 0^{n}1^{n-r}0^{n-\ell}1^{n} \not\in L$. We note that for any $i \geq 0$, $z^{\prime} \not\in L$. To find a $v,x$ that satisfies the pumping lemma, we would need to be able to select a symbol in the first half as $v$, and then reach the same symbol in the second half for $x$, which is not possible given the condition $\abs{vwx} \leq n$, since the lengths of each symbol are $n$ characters long. The best we could do is reach the opposite (first) symbol in the second half for $x$, with $v$ being the second symbol in the first half. 
                \bigbreak \noindent 
                Thus, we have a contradiction and the language must not be context free.
            \item \textbf{PL example four}: Suppose $L = \{a^{p}:\ p \in \mathbb{P}\}$. Choose $z=a^{\ell}$ for $\ell \geq n$. Then $v=a^{r},\ x = a^{s}$ for $0 < r+s \leq  n$, which implies $z = a^{r}a^{s}a^{\ell-r-s}$, and $z^{\prime} = a^{ir}a^{is}a^{\ell-r-s}$. Let $i = \ell + 1$, then $z^{\prime} = a^{\ell - r - s  + r(\ell+1) + s (\ell+1)}  = a^{\ell + r\ell + s\ell} = a^{\ell(1+r+s)}$. Since $\ell \in \mathbb{P}$, and $r+s > 0$, we know $p(1+r+s) \not\in \mathbb{P}$. Thus, we have a contradiction and $L$ must not be context free.
                



    \end{itemize}

    \pagebreak 
    \subsection{Turing machines}
    \begin{itemize}
        \item \textbf{Intro}: In 1930's Alan Turing presented an abstract machine called the Turing Machine which serves to this day as the model for all computation (i.e., algorithms) and computing (i.e., computers)
        \item \textbf{TM General Description}: Turing Machines (TM) are similar to deterministic finite automata (DFA) in that a TM has:
            \begin{itemize}
                \item a tape that is infinite in one direction and a tape head
                \item a finite number of states with exactly one that is designated as the start state, and
                \item A set of transitions (i.e., directed edges) that deterministically take the TM from one state to the next based on what the tape head reads from the tap
                \item Zero or more HALT state(s) that stops the TM and accepts the input string,
                \item no requirement that every state must have an outgoing edge for every symbol (i.e., requirement of DFAs), with the same understanding that if the TM is in a state from which there is no edge for the symbol pointed at by the tape head, then the TM crashes and rejects the input string, and
                \item no requirement that the entire input string must be read before accepting or rejecting the string.
            \end{itemize}
            \bigbreak \noindent 
            Turing Machines (TM) differ from FA and PDA in that a TM does the following ordered sequence on each transition:
            \begin{enumerate}
                \item Read the symbol from the tape (same as FA and PDA).
                \item Write a symbol to the tape at the same location (may be the same symbol that was read).
                \item Move the tape head one slot either left or right with the understanding that an attempt to move left from the leftmost part of the tape crashes the TM (i.e., halts execution and rejects the input string).
            \end{enumerate}
            \bigbreak \noindent 
            Accordingly, each transition in a TM is annotated with the three above
        \item \textbf{Turing machine example}:
            \bigbreak \noindent 
            \begin{figure}[ht]
                \centering
                \incfig{t1}
                \label{fig:t1}
            \end{figure}
            \bigbreak \noindent 
            In this example the TM
            \begin{itemize}
                \item always wrote the same symbol that it read
                \item always moved the tape head to the right
            \end{itemize}
            That is not always the the case with a TM.
        \item \textbf{Another TM example}: Consider the language $L = a^{n}b^{n} \ \forall \ n> 0$. We know that $L$ is not a regular language and that it is a context free language, so there exists a pushdown automaton (PDA) that accepts $L$.
            \bigbreak \noindent 
            A TM has no stack, but it has a tape head that can write and can move in either direction. What, then, is the strategy in designing a TM that accepts $L$?
            \bigbreak \noindent 
            As you read each $a$
            \begin{itemize}
                \item Cross the a off as an "already read" a (e.g., replace it with an X).
                \item Move the tape head to the right until we encounter our first $b $.
                \item Cross that $b$ off as an "already read" $b$ (e.g., replace it with a $Y$).
                \item Move the tape head back to the left until we encounter the $X$ we
                \item had just written, then move the tape head one space to the right
            \end{itemize}
            At this point the tape head should point either to the next $a$ or to the first $Y$ that we wrote.
            \begin{itemize}
                \item If the tape head points to the next $a$, then simply process the steps above again.
                \item If the tape head points to the first $Y$, then move the tape head to the right making sure that it passes only over $Y$'s until it reaches a blank $\Delta$ (to assure there were not more $b$'s than $a$'s).
            \end{itemize}
            \bigbreak \noindent 
            \begin{figure}[ht]
                \centering
                \incfig{t2}
                \label{fig:t2}
            \end{figure}
            \pagebreak 
        \item \textbf{Palindrome}: We know that Palindrome is not a regular language and that it is a context free language, so there is a PDA that accepts it. The basic strategy in designing a PDA that accepts Palindrome relies on the PDA's non- determinism.
            \begin{itemize}
                \item In the first portion of the PDA; read and push the first half of the string onto the stack.
                \item Non-deterministically transition to the second portion of the PDA by either;
                    \begin{itemize}
                        \item reading a single symbol off the tape without pushing it onto the stack (for odd-length palindromes) or
                        \item make a $\epsilon$-move (for even-length palindromes).
                    \end{itemize}
                \item In the second portion of the PDA; read the second half of the string and pop the stack for each symbol read to make sure that the two symbols (read and popped) match.
                \item When you reach the end of the tape make sure that there are no more symbols in the stack.
            \end{itemize}
            \bigbreak \noindent 
            A TM has no stack and it is deterministic. What, then, is your strategy in designing a TM that accepts Palindrome ?
            \begin{itemize}
                \item Process the string by reading and erasing characters from the ends taking care to make sure that the characters at each end of the string always match.
                \item Take care to account for both even- and odd-length palindromes.
            \end{itemize}
            \bigbreak \noindent 
            \fig{.5}{./figures/51.png}
        \item \textbf{Turing machine example}: Consider the language $L = a^{n}b^{n}a^{n} \ \forall \ n \geq 0$. We know $L$ is not context free
            \begin{itemize}
                \item Start by marking the first a as read with a *
                \item Move tape head to $ba$ transition and backup to $b$
                \item Replace that $b$ with an $a$ as read with a *
                \item  Move tape head to end of input and backup to $a$
                \item Replace last two $a$'s with blanks as read with $a$ * 
                \item Back tape head to * and advance to $a$
                \item Repeat
            \end{itemize}
            \bigbreak \noindent 
            \fig{.5}{./figures/52.png}
        \item \textbf{Formal Definition TM}: Formally, a Turing Machine (TM) is denoted
            \begin{align*}
                M = (Q, \Sigma, \Gamma, \delta, q_{0}, \Delta, F)
            .\end{align*}
            Where
            \begin{itemize}
                \item $Q$ is a finite set of states
                \item $\Gamma$ is a finite set of tape symbols
                \item $\Delta \in \Gamma$ is the blank
                \item $\Sigma \subset \Gamma$ such that $\delta \not\in \Sigma$, is a set of input symbols
                \item $q_{0} \in Q$ is the start state
                \item $\delta$ is the transition function that maps 
                    \begin{itemize}
                        \item a state in $Q$ and a (read) tape symbol in $\Gamma$ to
                        \item some (written) tape symbol in $\Gamma$, and a tape head move direction (L or R), and some state in $Q$
                        \item $\delta: Q \times \Gamma\to \Gamma \times \{L,R\} \times Q $
                    \end{itemize}
                    \bigbreak \noindent 
                    \textbf{Note:} $\delta$ is a partial function in that it may be undefined for some values in $Q \times \Gamma$
                \item $F \subseteq Q$ is the (possibly empty) set of HALT state(s)

            \end{itemize}
        \item \textbf{Prepend \#}: With a tape that is infinite in only one direction it is sometimes helpful to be able to shift the entire input string to the right one cell and then place in the leftmost cell of the tape a special symbol (e.g., \#) with the understanding that the special symbol will never appear anywhere else on the tape
            \bigbreak \noindent 
            In this way the rest of the TM can safely move the tape head left always taking care to not move left from \#.
            \bigbreak \noindent 
            There is a TM that can be used at the beginning of any TM - it shifts the entire input string, prepends \#, and positions the tape head at the first symbol of the original input string.
            \bigbreak \noindent 
            This TM is written for the input alphabet $\Sigma = \{a,b\}$ but it can be easily modified to work with any input alphabet.
            \bigbreak \noindent 
            \fig{.5}{./figures/53.png}
        \item \textbf{Turing machines as transducers}: 
            Thus far we have considered the use of Turing Machines as language acceptors.
            \bigbreak \noindent 
            Because Turing Machines can write to the tape, we can also use them to perform calculations that transform value(s) that are initially written on the tape (i.e., input) to some other value (i.e., output) on the tape (e.g., the TM implements some function f(x)).
            \bigbreak \noindent 
        \item We write the input variables onto the tape, and the TM must reach a HALT state (so the input string is in the language defined by the TM), but what is left on the tape is the output of the function.
            \bigbreak \noindent 
            When we use TMs in this way they act as a language acceptor (i.e., the set of valid inputs) but because the output is also valuable we say the TM is a transducer
        \item \textbf{TM transducer example}: Consider the language $L = 1^{m}01^{n} \ \forall \ m,n > 0 $
            \bigbreak \noindent 
            We can create a TM that accepts $L$, but can we also create a TM that leaves on the tape (i.e., as output) a string of the form $1^{m+n}$?
            \bigbreak \noindent 
            In other words, can we create a TM that computes
            \begin{align*}
                f(m,n) = m + n
            .\end{align*}
            \pagebreak \bigbreak \noindent 
            \begin{figure}[ht]
                \centering
                \incfig{t5}
                \label{fig:t5}
            \end{figure}
        \item \textbf{Another TM Transducer }: Consider the regular language $L = 1^{n}$ for $n > 0$
            \bigbreak \noindent 
            Can we create a TM that accepts $L$ and leaves on the tape a string of the form $1^{2n} $
            \bigbreak \noindent 
            In other words, can we create a TM that computes $f(n) = 2n$?
            \begin{itemize}
                \item Start by marking the first 1 as read with a $A$
                \item Move tape head to first $D$ , replace with $A$
                \item Back tape head to first $A$, then move $R$ one
                \item Now enter loop replacing each original 1 with $B$ and appending a 1 for each, always backing up tape to last $B$ and then moving $R$ one
                \item Eventually we will have processed the last original 1 which will position the tape head on the second $A$
                \item All that remains is to move the tape left replacing all the $A $'s and $B$'s with 1's taking care to stop and move $R$ when we reach the first $A$.
            \end{itemize}
            \bigbreak \noindent 
            \fig{.5}{./figures/54.png}
        \item \textbf{Final TM Transducer}: $f(m,n) = mn$
            \bigbreak \noindent 
            In the previous two examples the output was written left-justified on the tape
            \bigbreak \noindent 
            What if we temporarily relaxed that restriction by saying that the output may appear anywhere on the tape as long as it is contiguous (i.e., output may be preceded on tape by an arbitrary number of blank cells).
            \bigbreak \noindent 
            Consider the regular language $L = 1^{m}\#1^{n}\$$ for $m,n > 0$
            \bigbreak \noindent 
            Can we create a TM that accepts $L$ and leaves somewhere on the tape a string of the form $1^{mn}$ to compute $f(m,n) = mn$
            \begin{enumerate}
                \item Start by marking the first \(1\) as "read" by erasing it and advancing the tape head to the first \(1\) after the \#.
                \item Erase the first \(1\), append a \(\Delta\), return the tape head to the \(\Delta\) you just wrote, and then move right.
                \item Repeat until you have erased and appended a copy of the entire second block of \(1\)'s; this will leave the tape head at the \(\$ \).
                \item Return the tape head to the first \(\Delta\) we wrote and then move right, taking care to restore the second block of \(1\)'s.
                \item Repeat that process until you have appended a copy of the second block for the last \(1\) in the first block; this will leave tape at \#.
                \item Finish by advancing the tape head to \(\$\) and then one more cell right, erasing \#, the second block of \(1\)'s, and \(\$\).
            \end{enumerate}
            \bigbreak \noindent 
            \fig{.5}{./figures/55.png}

    \end{itemize}

    \pagebreak 
    \subsubsection{Decidability and Languages Accepted by Turing Machines}
    \begin{itemize}
        \item \textbf{Membership in Languages}: Given a language and an acceptor for that language (e.g., FA or PDA) one could ask the question... "Is a given string $w\in \Sigma^{*} $ in (or not in) the language?"
            \bigbreak \noindent 
            Or to put the question in different ways
            \begin{enumerate}
                \item For any $w \in \Sigma^{*}$ will the acceptor stop and determine whether or not w is in the language defined by the acceptor? 
                \item Does the acceptor always stop and partition $\Sigma^{*} $ into strings that are in the language and strings that are not in the language? For regular languages the answer is yes. We can see this from the definition of deterministic finite automata (DFA). 
                    \bigbreak \noindent 
                    In this way for any string $w\in \Sigma^{*}$ the DFA always stops and reports whether (or not) the $w$ is in the language.
                    \bigbreak \noindent 
                    In this way for any string $w\in \Sigma^{*}$ the DFA always stops and reports whether (or not) the w is in the language.
                    \bigbreak \noindent 
                    For context free languages the answer is also yes. We can see this from the definition of context free grammars (CFG) that are in Chomsky Normal Form (CNF).
                    \bigbreak \noindent 
                    Recall that any derivation of a non-empty word $w \in L$ defined by a CFG in CNF whose length $|w| = n > 0$ must always have exactly $2n-1$ steps ($n-1$ live productions plus $n$ dead productions).
                    Since any CFG has only finitely many productions, then for any string $w \in \Sigma^*$ with length $|w| = n > 0$, there are only finitely many derivations having exactly $2n-1$ steps $\ldots$ simply check them all to determine if any one derivation produces $w$.
                    \bigbreak \noindent 
                    In this way the CNF CFG always stops and reports whether (or not) the $w$ is in the language.
                    \bigbreak \noindent 
                    In other words, the CNF CFG partitions $\Sigma^*$ into strings that are in the language and strings that are not in the language.
            \end{enumerate}
        \item \textbf{Decidability}: So, for regular and context-free languages we can take any string $w \in \Sigma^{*}$ and determine whether or not $w$ is in the language.
            \bigbreak \noindent 
            In formal terms, we say that the question of membership for regular and context free languages is \textit{decidable.}
            \bigbreak \noindent 
            A question whose answer is boolean (i.e., yes or no) is decidable if there exists an effective method (or effective procedure or algorithm) that can determine the answer.
            \bigbreak \noindent 
            An effective method is one that always halts and produces the correct answer
            \bigbreak \noindent 
            A question whose answer is boolean and for which there is no algorithm (i.e., stops on all input and correctly answers "yes" or "no") is \textit{undecidable}.
            \bigbreak \noindent 
            We can, therefore, say that the question of membership for regular and context free languages is decidable because we have an effective procedure that answers the question for both classes of languages.
            \begin{itemize}
                \item \textbf{Regular languages:} Use the DFA which always halts and gives correct answer.
                \item \textbf{Context free languages:} Use the CNF CFG which has only finitely many derivations that could possible produce a given string, simply check them all to see if one does.
            \end{itemize}
            \bigbreak \noindent 
            It is sometimes convenient to represent a decidable question by depicting its decision procedure (or algorithm) as a "black box".
            \bigbreak \noindent 
            \begin{figure}[ht]
                \centering
                \incfig{chekc1}
                \label{fig:chekc1}
            \end{figure}
            \bigbreak \noindent 
            Note that in this representation the "black box" is not the language acceptor automaton (i.e., it is not a DFA or PDA)
            \bigbreak \noindent 
            For the decidable question, "Is $w\in L $?"
            \begin{itemize}
                \item \textbf{For Regular languages:} The "black box" represents the algorithm that uses the DFA which always halts and gives correct answer.
                \item \textbf{For Context free languages:} The "black box" represents the algorithm that uses the CNF CFG by checking all of the finitely many derivations that have exactly $2n-1$ productions.
            \end{itemize}
            \bigbreak \noindent 
            As we can see by the definitions of decidable/undecidable, the question of whether or not algorithm exists (i.e., a decision procedure that always stops and correctly answers "yes" or "no") is not limited to questions about formal languages (e.g., is membership in regular languages or CFLs decidable).
            \bigbreak \noindent 
            The question of decidability (i.e., whether an algorithm exists) can be, and is, applied to many domains.
            \bigbreak \noindent 
            Questions of decidability are at the core of all computability (i.e., does an algorithm exist?).
        \item \textbf{Euclid's conjecture}: Sometimes we answer a "yes/no" question by providing a proof, which answers the question entirely, and therefore, we do not need an algorithm
            \bigbreak \noindent 
            For example, there is Euclid's Conjecture which states:
            \begin{center}
                \textit{There are infinitely many prime numbers.}
            \end{center}
            \bigbreak \noindent 
            \textbf{Proof (by contradiction).} Assume there is a finite list of all the prime numbers $p_{1}, p_{2}, ... p_{r} $
            \bigbreak \noindent 
            Consider $m= p_{1}\cdot p_{1} \cdot ... \cdot p_{r} + 1$
            \bigbreak \noindent 
            We observe that $m$ cannot be any of the numbers in our original list $p_{1},p_{2},...,p_{r}$ but that list was supposed to be all the primes, so contradiction.
            \bigbreak \noindent 
            If $m$ is not prime, then it must be evenly divisible by some prime, call it $p$
            \bigbreak \noindent 
            We observe that p cannot be our original list $p_{1},p_{2},...,p_{r}$, but that list was supposed to be all the primes, so contradiction.
        \item \textbf{Is a Number Prime?}: Other times we answer a "yes/no" question by providing an algorithm.
            \begin{center}
                \textit{Given some integer $n>1$, is n prime?}
            \end{center}
            \bigbreak \noindent 
            Algorithm:
            \begin{enumerate}
                \item Divide $n$ by every integer $i$ in the range $2 \leq i \leq \sqrt{n}$
                \item If any $i$ in that range evenly divides $n$ then $n$ is not prime; otherwise $n$ is prime
            \end{enumerate}
        \item \textbf{Goldbach's Conjecture}: Sometimes we encounter a question for which there is no proof and there is no algorithm (i.e., always halts with "yes/no" answer).
            \bigbreak \noindent 
            One of the oldest and best-known unsolved problems in number theory was posed on June 7, 1742 by the German mathematician Christian Goldbach in his letter to Leonhard Euler. It states:
            \bigbreak \noindent 
            \begin{center}
                \textit{Every even integer greater than two can be expressed as the sum of two primes.}
            \end{center}
            \bigbreak \noindent 
            1690-1764 One approach (not an algorithm):
            \begin{enumerate}
                \item Test every even number that is greater than two.
                \item For each number try to find two primes whose sum is equal to that number
                \item If we find two such primes, then proceed to the next even number; otherwise halt and declare Goldbach's Conjecture false. 
            \end{enumerate}
            \bigbreak \noindent 
            If Goldbach's Conjecture is false, then the proposed approach will eventually HALT and provide the answer "no"; otherwise it will run forever
            \bigbreak \noindent 
            We say that Goldbach's Conjecture is undecidable.
        \item \textbf{Semi-decidable}: While it is true that Goldbach's Conjecture is undecidable (i.e., because there is no algorithm that always halts with "yes/no" answer), the fact that we have a method that always halts and gives one of the "yes/no" answers gives rise to a new definition.
            \bigbreak \noindent 
            A question whose answer is boolean (i.e., yes or no) is \textit{semi-decidable} if there exists a method (or algorithm) that always halts when the answer is yes or no, but may run forever for the other answer
            \bigbreak \noindent 
            \begin{figure}[ht]
                \centering
                \incfig{sd1}
                \label{fig:sd1}
            \end{figure}
        \item \textbf{Summary}: Combining our new definition semi-decidable with our previous definition of decidable/undecidable we have the following:
            \begin{itemize}
                \item \textbf{Decidable:} There is an algorithm that always halts with correct "yes/no" answer.
                \item \textbf{Undecidable:} Any yes/no question that is not decidable.
                    \begin{itemize}
                        \item \textbf{Semi-decidable:} an undecidable question for which there exists an algorithm that always halts with one of the answers "yes" or "no".
                    \end{itemize}
            \end{itemize}
        \item \textbf{What About TMs?}: We know that Turing Machines (TMs) can be used as language acceptors.
            \bigbreak \noindent 
            In fact, TMs can accept all regular and context free languages plus other languages that are not context free
            \bigbreak \noindent 
            But do TM's always partition $\Sigma^{*}$ into strings that are in the language and strings that are not in the language?
            \bigbreak \noindent 
            In other words, is the question of membership in languages accepted by Turing Machines decidable?
            \bigbreak \noindent 
            The answer is not always
            \pagebreak \bigbreak \noindent 
            Consider the following TM that accepts all strings with a double a, (i.e., aa):
            \bigbreak \noindent 
            \begin{figure}[ht]
                \centering
                \incfig{tm1}
                \label{fig:tm1}
            \end{figure}
            \bigbreak \noindent 
            For all $w\in \Sigma^{*}$ this TM:
            \begin{itemize}
                \item For all $w \in L$, HALT and accept
                \item For all $w \not\in L$, crash and reject
            \end{itemize}
            \bigbreak \noindent 
            Consider this slightly modified TM that also accepts all strings with a double a, (i.e., aa):
            \bigbreak \noindent 
            \begin{figure}[ht]
                \centering
                \incfig{tm2}
                \label{fig:tm2}
            \end{figure}
            \bigbreak \noindent 
            For all $w\in \Sigma^{*}$ this TM:
            \begin{itemize}
                \item For all $w \in L$, HALT and accept
                \item For all $w \not\in L$,
                    \begin{itemize}
                        \item sometimes crash and reject (e.g., no double a but ends in a)
                        \item sometimes loop forever (e.g., no double a, ends in b)
                    \end{itemize}
            \end{itemize}
            \bigbreak \noindent 
            You may ask... "If there is an algorithm that always stops, then why not simply use that?".
            \bigbreak \noindent 
            Because there are some problems for which the only algorithm(s) cannot eliminate the possibility of looping
            \bigbreak \noindent 
            Every Turing Machine (TM) partitions $\Sigma^{*}$ into three classes;
            \begin{enumerate}
                \item HALT and accept
                \item crash (i.e., stop) and reject
                \item loop forever 
            \end{enumerate}
            \bigbreak \noindent 
            For any specific TM one or more of those three classes may be empty, meaning, for any specific TM
            \begin{itemize}
                \item Might always either HALT or crash; might always crash; might always accept.
                \item \relax [general case]: Sometimes HALT and accept, sometimes crash and reject, or sometimes loop forever
            \end{itemize}
        \item \textbf{A Closer Look at a General TM }: In general, a TM will process strings by;
            \begin{itemize}
                \item $w \in L$, always halt and accept any string that is in the language.
                \item $w \not\in L$, sometimes halt by crashing, other times loop forever. 
            \end{itemize}
            \bigbreak \noindent 
            In the black box depiciction, we might make the no line dashed
            \bigbreak \noindent 
            Recall, a decidable problem is one for which there exists an algorithm that always halts with the correct "yes/no" answer.
            \bigbreak \noindent 
            If there is a possibility that the algorithm may sometimes not halt, then that possibility alone renders the problem undecidable
            \bigbreak \noindent 
            This gives us the definition of two new languages accepted by Turing Machines
        \item \textbf{Languages defined by turing machines}: A language that is accepted by a Turing Machine is said to be a \textit{recursively enumerable language.}
            \bigbreak \noindent 
            It is convenient to single out a subset of the recursively enumerable languages as follows:
            \bigbreak \noindent 
            A language that is accepted by at least one Turing Machine that halts on all inputs is said to be a recursive language.
            \bigbreak \noindent 
            Membership in recursive languages is decidable because, by the definition of recursive languages, there is a TM that always halts.
            \bigbreak \noindent 
            Membership in recursively enumerable languages is not decidable; it is semi-decidable. For recursively enumerable languages (that are not recursive) there are only Turing Machines that
            \begin{itemize}
                \item For any $w \in L$ always HALT and accept
                \item But for a $w \not\in L$, sometimes crash and reject, other times loop forever
            \end{itemize}
            \bigbreak \noindent 
            The existence of recursively enumerable languages raises the unsettling point ...
            \bigbreak \noindent 
            If a TM is running for a long time, how do we know if the machine will eventually stop (HALT+accept or crash+reject) or loop forever?
            \bigbreak \noindent 
            We don't know... Worst yet, we can't know
            \bigbreak \noindent 
            \fig{.5}{./figures/56.png}

    \end{itemize}

    \pagebreak 
    \subsubsection{TM Variations}
    \begin{itemize}
        \item \textbf{Intro}: As we have seen, different classes of automata have different power in their ability to define languages
            \begin{itemize}
                \item \textbf{Finite automata}: Regular languages
                \item \textbf{Pushdown automata}: Context free languages
                \item  \textbf{Turing Machines}: Recursively enumerable languages
            \end{itemize}
            \bigbreak \noindent 
            We have also experimented (and we can do other experiments) with different automata to see how such changes might impact the languages each accepts
            \bigbreak \noindent 
            For Finite automata
            \begin{itemize}
                \item Adding $\epsilon$-moves and/or non-determinism didn't make a difference
                \item Adding a stack makes a difference; equivalent to PDA
            \end{itemize}
            \bigbreak \noindent 
            For Pushdown automata
            \begin{itemize}
                \item Removing non-determinism (i.e., forcing determinism) makes a difference; reduces the languages accepted
                \item Adding one or more stacks (i.e., >1 stack) makes a difference; equivalent to Turing Machine
            \end{itemize}
        \item \textbf{Can we do better?} In the cases that we have studied for which modifications make no difference, we demonstrate equivalence by presenting algorithmic conversions between the automaton with and without the modifications
            \bigbreak \noindent 
            In such cases we never concern ourselves with efficiency; How much time or how many steps would one automaton take compared to another?
            \bigbreak \noindent 
            We only concerned ourselves with whether or not both automata always produce the same result; Does the modification render the automaton more restrictive, less restrictive, or keep it exactly the same.
            \bigbreak \noindent 
            We now consider variations on the Turing Machine with the same question in mind; Will the variation change the set of languages the TM can accept?
            \bigbreak \noindent 
            We ignore questions of efficiency
        \item \textbf{Stay Option}: What if we allowed the TM to not move the tape head on a transition, instead of just L or R also allow S, which means "stay"?
            \bigbreak \noindent 
            On the surface this may sound trivial, but it adds the ability to change state without disturbing the tape or tape head
            \bigbreak \noindent 
            While it is obvious that adding the stay option does not reduce the languages a Turing Machine can accept one might ask ... Does the stay option give Turing Machines any extra real power? The answer is no it does not
            \bigbreak \noindent 
            Although the stay option can be useful in reducing the number of states a TM may require, it adds no new power to TMs.
            \bigbreak \noindent 
            \textit{Theorem}: For any Turing Machine with the stay option there is some Turing Machine that acts the same way on all inputs; looping, crashing, or accepting, while leaving the same data on the tape, and vice versa
            \bigbreak \noindent 
            First, $TM \subseteq TM$ with stay option. Every TM is a TM with a stay option that simply does not have any stay option transitions
            \bigbreak \noindent 
            Part II... TM with stay option $\to$ equivalent TM
            \bigbreak \noindent 
            Construct a new TM from the TM with the stay option by first copying the machine and then converting every transition with a stay option as follows
            \bigbreak \noindent 
            \fig{.5}{./figures/57.png}
            \bigbreak \noindent 
            $\therefore$ Turing machine with stay option = Turing Machine
        \item \textbf{Move Multiple}: What if we allowed the TM to move the tape head multiple cells on a transition, instead of just L or R (one cell)?
            \bigbreak \noindent 
            \textit{Theorem}: For any Turing Machine with the move multiple option there is some Turing Machine that acts the same way on all inputs; looping, crashing, or accepting, while leaving the same data on the tape, and vice versa.
            \bigbreak \noindent 
            \textit{Proof.} Part I: TM $\subseteq$ TM with move multiple option. Every TM is a TM with a move multiple option where $n=1$ on all move transitions
            \bigbreak \noindent 
            Part II. TM with move multiple option $\to$ equivalent TM
            \bigbreak \noindent 
            Construct a new TM from the TM with the move multiple option by first copying the machine and then converting every transition where $n>1$ as in the the following $3L$ example (do likewise for $nR$):
            \bigbreak \noindent 
            \fig{.5}{./figures/58.png}
            \bigbreak \noindent 
            $\therefore$ Turing machine with move multiple option = Turing Machine
        \item \textbf{Move-in-State}: Consider the following variation to Turing Machines that specifies tape head movement in states rather than on transitions:
            \bigbreak \noindent 
            \fig{.5}{./figures/59.png}
            \bigbreak \noindent 
            Transitions labeled $p,q$ meaning read $p$ and replace with $q$ on tape
            \bigbreak \noindent 
            States (except HALT) labeled $q_{i} /\{L \text{ or } R\}$ meaning move tape head either L or R upon entering state.
            \bigbreak \noindent 
            Do not move tape head at beginning in $q_{0}$ but if you re-enter $q_{0}$ then move the tape head as indicated.
            \bigbreak \noindent 
            In original TMs when we entered a state like $q_{j}$ above we moved the tape head, sometimes L other times R, before entering the state. How do we handle this with a move-in-state TM?
            \bigbreak \noindent 
            Does this make the TM or the move-in-state TM more powerful than the other? ... No, they are equally powerful.
            \bigbreak \noindent 
            \textit{Theorem}. For any Turing Machine with the move-in-state option there is some Turing Machine that acts the same way on all inputs; looping, crashing, or accepting, while leaving the same data on the tape, and vice versa.
            \bigbreak \noindent 
            \textit{Proof}. Part I: TM with move-in-state $\to$ TM




    \end{itemize}

    \pagebreak 
    \subsubsection{Encoding TMs}
    \begin{itemize}
        \item \textbf{Recall}: Recall that a Turing Machine is formally defined as a 7-tuple
            \begin{align*}
                M = (Q, \Sigma, \Gamma, \delta, q_{0}, \Delta, F)
            .\end{align*}
            \bigbreak \noindent 
            And that we conveniently depict it as a state diagram. 
        \item \textbf{Encoding}: Rather than using the 7-tuple to draw a picture we could, instead, represent the TM as a string of characters (i.e., encode the TM)
            \begin{align*}
                M = (Q, \Sigma, \Gamma, \delta, q_{0}, \Delta, F) \to \text{ some string $w$ that represents $M$}
            .\end{align*}
            \bigbreak \noindent 
            There are many ways to encode a Turing Machine, we present one.
            \bigbreak \noindent 
            Recall that $Q$ is a finite set of states where
            \begin{itemize}
                \item $q_{0} \in Q$ is the start state
                \item $F \subseteq Q$ is the (possibly empty) set of HALT state(s)
            \end{itemize}
            \bigbreak \noindent 
            Without loss of generality we can modify $M$ by "collapsing" any and all HALT states to a single HALT state.
            \bigbreak \noindent 
            In other words, if $F$ was originally non-empty, then $F$ will be left with a single state (and $\delta$ modified accordingly), otherwise $F$ will remain empty and $\delta$ will be left unchanged.
            \bigbreak \noindent 
            We continue by assigning numbers to each of the states in $Q$ as follows:
            \begin{itemize}
                \item The start state $q_{0}$ is assigned 1.
                \item The (now single) HALT state, if one exists, is assigned 2.
                \item The remaining states are assigned arbitrary numbers other than 1 or 2 and such that each state is assigned a unique number.
            \end{itemize}
            \bigbreak \noindent 
            Use the numbered states and the transition function $\delta$ to create a new table that has a row for each transition (i.e., cell in $\delta$).
            \bigbreak \noindent 
            \fig{.5}{./figures/60.png}
            \bigbreak \noindent 
            We next, use input symbols in $\Sigma \subset \gamma $ to create:
            \bigbreak \noindent 
            \begin{itemize}
                \item A fixed-length encoding for each tape symbol in $\Gamma$ and
                \item An encoding for each direction (L or R)
            \end{itemize}
            \bigbreak \noindent 
            Note, in order to do this $\Sigma$ must have at least two symbols
            \bigbreak \noindent 
            \fig{.5}{./figures/61.png}
            \bigbreak \noindent 
            We now encode each row of the new table as follows;
            \begin{itemize}
                \item \textbf{Transition from state}: $i \to j:\ a^{i}ba^{j}b$
                \item \textbf{Read/Write:} use fixed-width encoding
                \item \textbf{Direction}: use fixed-width encoding
            \end{itemize}
            \bigbreak \noindent 
            \fig{.5}{./figures/62.png}
            \bigbreak \noindent 
            The encoding of the TM concludes by concatenating the encoded rows (in any order) to create a single string. (with the understanding that start state = 1 and halt state = 2.)
        \item \textbf{Notes and the code word language (CWL):} Every TM with at least two symbols in $\Sigma$ can be encoded to a string
            \bigbreak \noindent 
            Not every string
            \begin{itemize}
                \item Can be decoded to a TM (e.g., strings that begin with 'b' do not represent a TM)
                \item \textbf{Decodes to a valid TM}: Might decode to a TM that is non-deterministic, has transitions from the HALT state, have READ, WRITE, or Move encodings that are invalid, etc.

            \end{itemize}
            \bigbreak \noindent 
            In fact, using our last example where each symbol in $\Gamma$ had an encoding length=3, from $\Sigma^{*} $ we see that anything outside $(a^{+}ba^{+}b(a+b)^{7})^{*} $ is definitely not a valid TM, and even some of those strings are not a valid TM.
            \bigbreak \noindent 
            Nonetheless, that regular expression comes pretty close to defining the set of valid TM's so we call it the Code Word Language (CWL)
            \begin{align*}
                \text{CWL } = (a^{+}ba^{+}b(a+b)^{7})^{*}
            .\end{align*}
            \bigbreak \noindent 
            And we note
            \begin{align*}
                \text{valid TM encodings } \subset   \text{ CWL } \subset \Sigma^{*}
            .\end{align*}
        

    \end{itemize}

    \pagebreak 
    \subsubsection{Revisiting Recursively Enumerable Languages}
    \begin{itemize}
        \item \textbf{Recall}:  The definitions of recursively enumerable and recursive languages: A language that is accepted by a Turing Machine is said to be a recursively enumerable language
            \bigbreak \noindent 
            A language that is accepted by at least one Turing Machine that halts on all inputs is said to be a recursive language.
            \bigbreak \noindent 
            \fig{.5}{./figures/63.png}
            \bigbreak \noindent 
            We've seen an example of a recursive language that is not a context free language: $a^{n}b^{n}c^{n}$
            \bigbreak \noindent 
            But is there a language that is recursively enumerable that is not recursive? Is there a language that is not even recursively enumerable?
            \bigbreak \noindent 
            Yes to both 
        \item \textbf{Turing Machines and Their Encodings}: Recall that we can take a Turing Machine (having at least two input symbols in $\Sigma$) and encode it using its own input alphabet $\Sigma$
            \bigbreak \noindent 
            That means we can use a TM to process its own encoding.
            \bigbreak \noindent 
            Thinking about this same prospect in a different way, consider the strings in CWL = $(a^{+}ba^{+}b(a+b)^{7})^{*}$, which we can partition into the following three sets:
            \begin{itemize}
                \item Invalid TMs
                \item Valid TMs that accept their own encoding
                \item Valid TMs that do not accept their own encoding

            \end{itemize}
            \bigbreak \noindent 
            And given that partitioning we can define two new languages, Acc,NotAcc $\subseteq$ CWL, as follows:
            \begin{align*}
               Acc = 
               \begin{cases}
                   w \in \text{ CWL such that: } \\
                   w \text{ is a valid TM that is } \\
                   \text{accepted by the TM it encodes}
               \end{cases} \\
               NotAcc = 
               \begin{cases}
                   w \in \text{ CWL such that: } \\
                   \text{1. w is not a valid TM or} \\
                   \text{2. w is a valid TM that is not accepted by the TM it encodes }
               \end{cases}
            .\end{align*}
            What kind of language is NotAcc? Regular? Context-free? Recursive? Recursively enumerable? none of the above, meaning, there does not exist a TM that accepts NotAcc.
            \bigbreak \noindent 
            \textit{Theorem}. There exists a language that is not recursively enumerable.
\bigbreak \noindent 
\textit{Proof (by contradiction)}. Assume that the language NotAcc is recursively enumerable.:
\bigbreak \noindent 
\fig{.5}{./figures/64.png}
\bigbreak \noindent 
This means that there exists some Turing Machine, $T$, that accepts NotAcc.
\bigbreak \noindent If we have a Turing Machine, $T$, then we can create a string $w$ from $T$'s input alphabet $\Sigma$ that encodes $T$. Because $w$ was built from $T$'s input alphabet $\Sigma$, we can process $w$ using $T$ (i.e., have a TM process its own encoding).
\bigbreak \noindent There are only two possibilities; either $T$ accepts $w$, or it does not.
\bigbreak \noindent 
Case I - $T$ accepts $w$
\bigbreak \noindent 
Because $w$ is a valid encoding of $T$, and (in this case) $T$ accepts $w$, we see that $w \in$ "Valid TM's, Accepts own Encoding",
which means $w \not\in $ NotAcc. However, (in this case) $T$ accepts $w$, which means $w \in$ NotAcc. (contradiction)
\bigbreak \noindent 
Case II - $T$ does not accept $w$
\bigbreak \noindent 
Because w is a valid encoding of T and (in this case) T does not accept w, we see
that $w \in$ "Valid TM's, Does not accept own Encoding", which means $w \in$ NotAcc.
\bigbreak \noindent 
However, T (in this case) T did not accept w,
which means $w \not\in$ NotAcc (contradiction)
\bigbreak \noindent 
Since the only two possible cases each led to a contradiction, we
conclude that our assumption that NotAcc is a recursively
enumerable language must be false.
\bigbreak \noindent 
$\therefore$ NotAcc is not a recursively enumerable language.



    \end{itemize}

    \pagebreak 
    \subsubsection{Universal turing machine}
    \begin{itemize}
        \item \textbf{Turing Machines and Their Encodings}: We have seen that NotAcc is not a recursively enumerable language, meaning one cannot create a TM that accepts NotAcc.
            \bigbreak \noindent 
            What about the remaining partition?
            \bigbreak \noindent 
            What kind of language is Acc? Regular? Context-free? Recursive? Recursively enumerable? ... before we can answer that we need to create a special TM.
        \item \textbf{UTM}: We will create a special TM and call it a UTM. We will describe our new UTM in general terms rather than drawing it explicitly.
            \bigbreak \noindent 
            First, the UTM takes as input some $w$ which is an encoding of some TM.
            \bigbreak \noindent 
            The basic function of the UTM is to simulate the TM that is encoded as its input, that is, the UTM should;
            \begin{itemize}
                \item accept strings the encoded TM would accept
                \item Reject (i.e., crash) strings the encoded TM would reject
                \item Loop on strings on which the encoded TM would loop
            \end{itemize}
            \bigbreak \noindent 
            The first thing the UTM does is
            \begin{enumerate}
                \item Shift the input string to the right one cell inserting a \# in the leftmost cell
                \item Append a \$ to the end of the input string
                \item Append a copy of the original input string just after the \$
            \end{enumerate}
            \bigbreak \noindent 
            The general idea is that the UTM will use
            \begin{enumerate}
                \item $w_{1}$ to simulate the encoded TM (never changing it)
                \item $w_{2}$ as the input to the encoded TM (changing often)
            \end{enumerate}
            \bigbreak \noindent 
            Recall that an encoded TM is a concatenation of rows from a table (i.e., one row per transition in the encoded TM).
            \bigbreak \noindent 
            The UTM next inserts a blank ($\Delta$)
            \begin{enumerate}
                \item Before each row in $w_{1}$
                \item Before each symbol in $w_{2}$
            \end{enumerate}
            \bigbreak \noindent 
            \fig{.5}{./figures/65.png}
            \bigbreak \noindent 
            To complete its initialization phase, the UTM Inserts a "dummy row" at the beginning of the tape whose only value is its "To" state which must indicate the start state = 1
            \bigbreak \noindent 
            Places an * before
            \begin{itemize}
                \item the newly inserted "dummy row" (to indicate that was the last transition processed)
                \item The first character in $w_{2}$ to indicate the simulated tape head location
            \end{itemize}
            \bigbreak \noindent 
            This ends the setup phase for the UTM. It is now ready to enter its next phase in which it simulates the TM encoded by $w$.
            \bigbreak \noindent 
            The UTM runs right down the tape to read the next character from $w_{2}$
            \bigbreak \noindent 
            It next runs left back up the tape down a different branch in the UTM depending on the character that it read.
            \bigbreak \noindent 
            \fig{.5}{./figures/66.png}
            \bigbreak \noindent 
            The UTM runs right down the tape to find the * in $w_{1}$ to discover what state the simulated TM is in (i.e., the To state from the row $r_{i}$ last processed).
            \bigbreak \noindent 
            Then, Find the row $r_{j}$ in $w_{1}$ whose
            \begin{itemize}
                \item From state matches the simulated TM's current state
                \item Read character matches the last character read by the simulated TM
            \end{itemize}
            \bigbreak \noindent 
            If no such row $r_{j}$ in $w_{1}$ can be found, then the simulated TM crashes, so crash the UTM. Otherwise
            \begin{itemize}
                \item Move the * in $w_{1}$ to precede the matching row $r_{j}$
                \item Update the character in $w_{2}$ (i.e., WRITE)
                \item Move the * in $w_{2}$ (L or R).
            \end{itemize}
            \bigbreak \noindent 
            If the To state in matching row $r_{j}$ is 2 (i.e., HALT), then the simulated TM HALTs (accepts) its input, so the UTM should also HALT (accept).
            \bigbreak \noindent 
            In this way the UTM accurately simulates the TM that is encoded as its input in that the UTM
            \begin{enumerate}
                \item accepts strings the encoded TM would accept
                \item Rejects (i.e., crash) strings the encoded TM would reject
                \item Loops on strings on which the encoded TM would loop

            \end{enumerate}
            \bigbreak \noindent 
            In other words, the UTM is a TM that accepts the language Acc. Therefore, Acc is a recursively enumerable language.
            \bigbreak \noindent 
            Is Acc a recursive language?
            \bigbreak \noindent 
            Before we can answer that question, we must first take a closer look at recursive languages by revisiting regular languages.
            \bigbreak \noindent 
            Recall that if $L$ is a regular language, then its compliment $L^{\prime}$ is also a regular language.
            \bigbreak \noindent 
            We accomplish this by taking a DFA that accepts $L$ and changing all accepting states to non-accepting, and vice versa.
            \bigbreak \noindent 
            We can do something with recursive languages to show that if $L$ is a recursive language then its compliment $L^{\prime}$ is also a recursive language.
            \bigbreak \noindent 
            We accomplish this by constructing a new TM that swaps "yes/no" decisions.
            \bigbreak \noindent 
            \fig{.5}{./figures/67.png}
            \bigbreak \noindent 
            Similarly, the union of two recursive languages is also a recursive language. We accomplish this by constructing a new TM from the two recursive TMs.
            \bigbreak \noindent 
            \fig{.5}{./figures/68.png} 
            \bigbreak \noindent 
            Returning to Acc ... we know
            \begin{itemize}
               \item CWL is a regular language
                \item CWL$^{\prime}$ is a regular language
                \item CWL$^{\prime}$ is a recursive language
            \end{itemize}
            \bigbreak \noindent 
            \begin{itemize}
                \item (CWL $^{\prime} \cup $  Acc) is a recursive language
                \item (CWL $^{\prime} \cup $ Acc)$^{\prime}$ is a recursive language 
            \end{itemize}
            \bigbreak \noindent 
            \fig{.5}{./figures/69.png}
            \bigbreak \noindent 
            However, (CWL$^{\prime} \cup $ Acc)$^{\prime}$ = NotAcc, which we know is not recursively enumerable (much less recursive) ... contradiction.
            \bigbreak \noindent 
            $\therefore$ Acc is a recursively enumerable language that is not recursive.
            \bigbreak \noindent 
            \fig{.5}{./figures/70.png}
        \item \textbf{More on UTM}: Consider the UTM we just described to accept Acc. UTM took as input an encoding of some TM and started by making a copy of that encoding;
            \begin{itemize}
                \item $w_{1}$ - encoding of TM that UTM did not change
                \item $w_{2}$ - copy of encoded TM (w1) that served as input to simulated TM - UTM frequently changed
            \end{itemize}
            \bigbreak \noindent 
            After the initialization phase (i.e., making a copy of $w_{1}$) and the UTM entered the simulation phase, it didn't matter that that $w_{2}$ was a copy of $w_{1}$.
            \bigbreak \noindent 
        In other words, we could have placed any input for $w_{2}$ and the UTM would have simulated the TM encoded by $w_{1}$ processing the arbitrary input we placed as $w_{2}$
        \bigbreak \noindent 
        Now the UTM becomes our \textit{first stored-program computer}
        \bigbreak \noindent 
        The UTM takes as input (a) an encoded TM and (b) input for that encoded TM, and simulates the TM (a) as it processes input (b)
        \bigbreak \noindent 
        The UTM is a TM that can simulate any TM processing any data, and hence its name, The Universal Turing Machine (Turing, 1936).
        \bigbreak \noindent 
        The Universal Turing Machine is the foundation of all computing theory and was the conceptual archetype of the early computer.
        \bigbreak \noindent 
        It is not a computer's operating system, but rather, the logic that guides a computer's instruction fetch and execution cycle (typically implemented in a computer's hardware).
        \bigbreak \noindent 
        The first stored-program computer was built by John von Neumann and his colleagues within years of Turing's work.
        \bigbreak \noindent 
        Contemporary computers implement the UTM differently (for efficiency) but they are all based on the UTM.
    \end{itemize}

    \pagebreak 
    \subsubsection{The halting problem}
    \bigbreak \noindent 
    \begin{itemize}
        \item \textbf{Revisiting Turing Machines}: If a TM has been processing some string for a long time, how do we know if this is a case when the TM will be looping forever?
            \bigbreak \noindent 
            If the TM is going to loop forever, then let's just stop now and declare the input string outside the language accepted by the TM!
        \item \textbf{The halting problem}: There is a problem, The Halting Problem, that states:
            \bigbreak \noindent 
            \begin{center}
                \textit{If we are given a Turing Machine $T$ and input string $w$, can we tell whether $T$ halts on $w$?}
            \end{center}
            \bigbreak \noindent 
            The Halting Problem question does not care if T accepts/rejects w, only whether T eventually stops (accept or crash/reject) or loops forever.
            \bigbreak \noindent 
            Obviously, we cannot simply start processing w with T to find the answer.
            \bigbreak \noindent 
            The UTM we presented cannot answer the question because it will loop forever if T loops on w.
            \bigbreak \noindent 
            What The Halting Problem is calling for is an algorithm (a TM) that
            \begin{itemize}
                \item Accepts as input an encoded TM T and input string w
                \item The Halting Problem TM would never loop forever (i.e., must be recursive, not merely recursively enumerable)
                \bigbreak \noindent 
            \item On all possible inputs it would always stop and report either "halts" or "loops" to indicate what T would do when processing w.
                \bigbreak \noindent 
            \end{itemize}
            So the question becomes does such an algorithm exist?... No!
            \bigbreak \noindent 
        \item \textbf{Halting problem proof}: Assume a TM exists that solves The Halting Problem, call it $T_{H}$.
            \bigbreak \noindent 
            That is, $T_{H}$ is a TM that takes as input and encoded TM T, followed by a \#, followed by some input string w.
            \bigbreak \noindent 
            $T_{H}$ would always terminate and print as its output on its tape
            \begin{itemize}
                \item "HALTS" if T will halt when processing w, or
                \item "LOOPS" if T will loop forever when processing w
            \end{itemize}
            \bigbreak \noindent 
            \fig{.7}{./figures/71.png}
            \bigbreak \noindent 
            We modify $T_{H}$ to create a $T_{H}^{\prime} $ by replacing the TH path from $q_{i}$ through its "writes HALTS" states, to HALT with a new state that loops forever.
            \bigbreak \noindent 
            \fig{.7}{./figures/72.png}
            \bigbreak \noindent 
            $T_{H}^{\prime} $ will either
            \begin{itemize}
                \item Loop forever if input $T$ would halt on its input $w$, or
                \item Halt with "LOOPS" written on the tape when input $T$ would loop forever on its input $w$
            \end{itemize}
            \bigbreak \noindent 
            Consider how $T_{H}^{\prime}$ would process input $T_{H}^{\prime}\#T_{H}^{\prime}$, that is, ask $T_{H}^{\prime}$ how it would process an encoding of itself
            \bigbreak \noindent 
            As we've seen, there are only two possible responses from $T_{H}^{\prime} $.
            \bigbreak \noindent 
            \textbf{Case I:} $T_{H}^{\prime} $ actually loops forever when processing $T_{H}^{\prime}\#T_{H}^{\prime}$ (i.e., enters $q_{h}^{\prime} $). $T_{H}^{\prime} $ only loops forever when its input T would halt on T's input w
            \bigbreak \noindent 
            Since we processed $T_{H}^{\prime}\#T_{H}^{\prime}$ that would mean that $T_{H}^{\prime}$ must halt when processing its own input, contradiction.
            \bigbreak \noindent 
            \textbf{Case II:} $T_{H}^{\prime}$ actually halts with "LOOPS" when processing on $T_{H}^{\prime}\#T_{H}^{\prime} $ (i.e., enters $q_{1}$). 
            $T_{H}^{\prime}$ only halts with "LOOPS" when its input T would loop forever on T's input w.
            \bigbreak \noindent 
            Since we processed $T_{H}^{\prime}\#T_{H}^{\prime}$ that would mean that $T_{H}^{\prime} $ must loop forever when processing its own input, contradiction.
            \bigbreak \noindent 
            This tells us that $T_{H}^{\prime}$ cannot exist, which in itself is not a proof that there is no solution to The Halting Problem
            \bigbreak \noindent 
            However, $T_{H}^{\prime}$ was a legitimate modification of $T_{H}$ which we assumed to exist as a solution to The Halting Problem.
            \bigbreak \noindent 
            Since $T_{H}^{\prime}$ cannot exist and it was based on a legitimate modification of $T_{H}$, we conclude that $T_{H}$ cannot exist
            \bigbreak \noindent 
            $\therefore$ There is no algorithm that can solve The Halting Problem.
    \end{itemize}

    \pagebreak 
    \subsection{Complexity theory}
    \begin{itemize}
        \item \textbf{Intro}: The field of computational complexity looks at the resources required to solve problems. Or, more generally, how much resources does it take: time, memory space, number of processors, bandwidth, and so fourth. It is a given that the problem is solvable.
        \item \textbf{$\mathcal{P}$ and $\mathcal{NP}$}: Regarding time, we define two sets of languages. The set $\mathcal{P}$ is those languages that can be solved in polynomial time, and $\mathcal{N}\mathcal{P}$ is the set of those languages that can be solved in polynomial time on a non-deterministic TM. The set $\mathcal{P}$ is somewhat viewed as the reasonably tractable problems, buy many problems of practical interest have been shown to be in $\mathcal{N}\mathcal{P}$. Thus, the question of whether all of $\mathcal{N}\mathcal{P}$ is in $\mathcal{P}$ (that is, whether $\mathcal{P} = \mathcal{N}\mathcal{P}$) is of fundamental importance. This question remains unsolved.
    \end{itemize}
    \bigbreak \noindent 
    \subsubsection{Time complexity}
    \begin{itemize}
        \item \textbf{Time complexity}: The time complexity of a set of problems is how much item is needed to solve them. The time complexity of a language is how much time is needed to decide membership in it.
            \bigbreak \noindent 
            The goal is to odetermine how the resources required depend on the size of the input. $n$ always denotes the size of the input. Then, the running time is the number of steps as a function of $n$.
            \bigbreak \noindent 
            It is important to note that we analyze the \textbf{worst case}. It doesn't matter that some instances can be done quickly, what matters is if \textit{every} instance can be done quickly. A TM is said to run in time $T(n) $ if for all inputs $w$, it halts within $T(\abs{w}) $ steps.
        \item \textbf{Big-O}: It is important to note that constants do not matter. If a machine runs in  $2n^{2}, 7n^{2}$, or $ 1000n^{2}$ steps is immaterial. In fact, we cannot care about the constants. For, the exact time will depend very heavily on what the atomic operations are. And nobody uses a laptop TM anyway. If the number of steps in a certain TM is proportional to $n^{2}$, we say the TM runs in $O(n^{2}) $ time, or "\textit{order} $n^{2} $" time, meaning there exists some constant $c$ such that the Tm runs in at most $cn^{2}$ steps for any input of length $n$. The order notation says how the worst case running time grows as $n$ gets large.
        \item \textbf{Polynomial time}: The collection of all problems that can be solved in polynomial time is called $\mathcal{P}$. That is, a language $L$ is in $\mathcal{P}$ if there exists a constant $k$ and a TM that decides $L$ that runs in time $O(n^{k})$.
        \item \textbf{Complexity class}: It is common to call a set of related languages a \textit{complexity class}, or just a class. The class $\mathcal{P}$ roughly captures the collection of practically solvable problems. 
        \item \textbf{Polynomially related}: Two models of computation are polynomially related if there is a polynomial $p$ such that if a language is decidable in $T(n)$ time on one model, the nit is decidable in time $p(T(n)) $ on the other.
        \item \textbf{Nondeterministic time}: The collection of all problems that can be solved in polynomial time by a non-deterministic machine is called $\mathcal{NP}$. That is, a language $L$ is in $\mathcal{NP}$ if there exists a constant $k$ and an NTM that decides $L$ that runs in time $O(n^{k})$. It should be clear that
            \begin{align*}
                \mathcal{P} \subseteq \mathcal{NP}
            .\end{align*}
        \item \textbf{Theorem}: Let $L$ be a recursive language. If there is an NTM for $L$ that runs in time $T(n)$, then there is a deterministic TM for $L$ that runs in time $O(C^{T(n)}) $ for some constant $C$
        \item \textbf{Conjecture}: $\mathcal{P} \ne \mathcal{NP}$, however, we do not know.
    \end{itemize}






    \pagebreak 
    \unsect{DSA}
    \bigbreak \noindent 
    \subsection{C++ Stuff}
    \bigbreak \noindent 
    \subsubsection{Type declarations}
    \begin{itemize}
        \item \textbf{Discern any type}: Some rules,
            \begin{enumerate}
                \item Start with the variable name, we read from inside to out
                \item const, \%, *, and basic types go on the left
                \item const refers to what is immediately on the left (except for \texttt{const int*}), but the standard form of this is actually \texttt{int const*}. Thus, the exception to this is const is at the very left, then it refers to what is immediately right.
                \item arrays and functions go on the right, function args are type declaration sub-problems
            \end{enumerate}
            The Algorithm:
            \begin{itemize}
                \item Start with the variable name, or the implied name position 
                \item Read right until end or )
                \item Read left until end or (
                \item If something still left to read, move out one level of parenthesis and go to 2, else done.
            \end{itemize}
            \bigbreak \noindent 
            Thus, using parenthesis allows us to change direction, this will come in handy.
            \bigbreak \noindent 
            \textbf{Examples}:
            \begin{itemize}
                \item  $a$ is an int $\implies$ \texttt{int a}
                \item $a$ is a pointer to an int $\implies$ \texttt{int * a}
                \item $a$ is a pointer to a constant int $\implies$ \texttt{int const * a} (also \texttt{const int * a})
                \item $a$ is a constant pointer to an int $\implies$ \texttt{int * const a}
                \item $a$ is a constant pointer to a constant int $\implies$ \texttt{int const * const a} (also \texttt{const int * const a})
                \item $a$ is an array of 5 ints $\implies$ \texttt{int a[5]}
                \item $a$ is an array of 5 pointers to constant ints $\implies$ \texttt{int const * a[5]}
                \item $a$ is a pointer to an array of 5 constant ints $\implies$ \texttt{int const (* a)[5]}
            \end{itemize}
        \item \textbf{Multi dimensional arrays (matrices)}: Think of multi-dimensional arrays as arrays of arrays. More indicative of what's happening internally. \texttt{float dat [3][4];} can be read as: "dat is an array of 3 arrays of 4 floats" (Using the algorithm from above).
            \bigbreak \noindent 
            \textbf{Examples}:
            \begin{itemize}
                \item \texttt{arg1} is a reference to an array of 25 constant pointers to arrays of 8 strings. $\implies$ \texttt{string (* const (\& arg1)[25])[8]}
                    \bigbreak \noindent 
                    \textbf{Note:} Notice how we use parenthesis to change direction
            \end{itemize}
        \item \textbf{Function Pointers}: Pointers point to bytes, which can be interpreted different ways. Pointers can point to bytes that can be interpreted as code, i.e. a function pointer.
            \bigbreak \noindent 
            \textbf{Examples}: 
            \begin{itemize}
                \item $f$ is a pointer to a function which takes an int and returns void. $\implies$ \texttt{void (* f) (int)}
            \end{itemize}
    \end{itemize}

   \pagebreak 
   \subsubsection{G++}
   \begin{itemize}
       \item \textbf{Compliation and linking}:
           Compilers turn source code into executable code.
           \begin{itemize}
               \item \textbf{Source code $\to$ object code (Compilation)}: Object code is almost executable. It contains pieces that it provides to other objects, and holes to be filled in. It is a slow process
               \item \textbf{Object code $\to$ executable (Linking)}: Connects pieces of object files together. This is a fast process
           \end{itemize}
           \bigbreak \noindent 
           \textbf{Note:}  Many "compilers" do both compiling and linking. Most programs are built in two stages:
           \begin{enumerate}
               \item Compile all the source code files
                \item Link the object code file into an executable
           \end{enumerate}
           This is the most efficient way to compile large projects.  Changing a single source code file requires a small number of compilations (slow), followed by linking (fast).
        \item \textbf{Standard unix c compiler}: The standard is GNU gcc
        \item \textbf{Standard unix cpp compiler}: The standard is GNU g++
        \item {g++ Options}: With no options, g++ will go from source to an executable named a.out
            \begin{itemize}
                \item \textbf{-o}: The -o option gives the name of the output file
                \item \textbf{-c}: The -c option makes the compiler stop after the compilation stage. No linking is done. The name of the object code file is the same as the source with the extension replaced with .o
                \item \textbf{-W[\textit{warning}]}: Tell the complier to look for a specific warning
                \item \textbf{-Wall (Warning all)}: There are many -W\textit{warning} options, which warn of various conditions. -Wall warns about all of them. The compiler keeps going through warnings
                    \bigbreak \noindent 
                    \textbf{Note:} A compiler warning is usually a bug waiting to happen. Do all you can to get rid of all warnings.
                \item \textbf{-Werror}: The -Werror option turns all warnings into errors. The compiler aborts on an error.
                \item \textbf{-g}: The -g option turns on debugging, and leaves much extra information in an object file. Executable is much larger, possibly slower.
                \item \textbf{-0}: The -O option turns on optimization. There are several different levels of optimization, e.g. -O0, -O1, -O2, -O3. 
                    \bigbreak \noindent 
                    \textbf{Note:} Optimization may break your code, and -O and -g don't always work well together
                \item \textbf{-I[\textit{directory}]}: The -I option specifies an additional directory to search for include files. No space between -I and directory
                    \bigbreak \noindent 
                    Thus, 
                    \begin{cppcode}
                    #include "./dir/headerfile" // Without -I
                    #include "headerfile" // With -I : g++ -I./dir ...
                    \end{cppcode}
                \item \textbf{-L[\textit{directory}]}: The -L option specifies an additional directory to search for libraries. No space between -L and directory.
                    \bigbreak \noindent 
                    \textbf{Note:} This option is meant for linking only. It has no effect in compilation.
                \item \textbf{-l[\textit{libraryname}]}: The -l option specifies a library for linking. No space between -l and library name. The library name is related to the libray file name, but it is not identical. Library names start with "lib" and end with ".so.*" or ".a". These are removed. For example
                    \begin{itemize}
                        \item The math library /lib/x86\_64-linux/gnu/libm.so.6 is linked as -lm
                        \item The X11 graphics library /usr/lib/x86\_64-linux-gnu/libX11.so is linked as -lX11
                    \end{itemize}
                    \bigbreak \noindent 
                    \textbf{Note:} This option is for linking only. It has no effect in compilation.  Libraries are the last things listed in a linking command.
                    \bigbreak \noindent 
                    If you're linking against a library that is located in a non-standard directory (a directory that is not automatically searched by the linker, such as ./libs), then you need to tell the linker where to find that library using the -L option. Thus, -L tells the compiler  where to look, -l specifies which one to grab.

            \end{itemize}
   \end{itemize}


    \pagebreak 
    \unsect{Databases}

    \bigbreak \noindent 
    \subsection{Introduction to databases (db concepts)}
    \bigbreak \noindent 
    \subsubsection{Definitions and theorems}
    \begin{itemize}
        \item \textbf{What is a database?}: A database is a collection of stored operational data used by the application systems of some particular enterprise, better yet a collection of related data.
        \item \textbf{What is an enterprise?}: a generic term for any reasonably large-scale commercial, scientific, technical, or other application. Such as
            \begin{itemize}
                \item Manufacturing
                \item Financial
                \item Medical
                \item University
                \item Government
            \end{itemize}
        \item \textbf{Operational data}: Data maintained about the operation of an enterprise, such as
            \begin{itemize}
                \item Products
                \item Accounts
                \item Patients
                \item Students
                \item Plans
            \end{itemize}
            \bigbreak \noindent 
            \textbf{Note:} Notice that this DOES NOT include input/output data
        \item \textbf{Database Management System (DBMS)}: A Database Management System (DBMS) is a collection of programs that enables users to create and maintain a database. Ie a general-purpose software system that facilitates
            \begin{itemize}
                \item Definition of databases
                \item Construction of databases
                \item Manipulation of data within a database
                \item Sharing of data between users/applications
            \end{itemize}
        \item \textbf{Defining a database}: For the data being stored in the database, defining the database specifies
            \begin{itemize}
                \item The data types
                \item The structures
                \item The constraints
            \end{itemize}
        \item \textbf{Constructing a Database}: Constructing a database is the process of storing the data itself on some storage device
            \bigbreak \noindent 
            \textbf{Note:} The storage device is controlled by the DBMS
        \item \textbf{Manipulating a Database}
            \begin{itemize}
                \item retrieve specific information in a query
                \item update the database to include changes
                \item generate reports from the data
            \end{itemize}
            \bigbreak \noindent 
            Most likely already defined by whatever dbms you choose
        \item \textbf{Sharing a Database}: Sharing a database Allows multiple users and programs to access the database at the same time, any conflicts between applications are handled by the DBMS
        \item \textbf{Other Important Functions of a Database}: Other important functions provided by a DBMS include
            \begin{itemize}
                \item Protection, system protection, security protection
                \item Maintenence, allows updates to be performed easily
            \end{itemize}
        \item \textbf{Simplified Database System Environment}:
            \bigbreak \noindent 
            \fig{.5}{./figures/1.png}
        \item \textbf{Main characteristics of a database system are:}
            \begin{itemize}
                \item Self-describing nature of a database system
                \item Insulation between programs and data, and data abstraction
                \item Support for multiple views of the data
                \item Sharing of data and multi-user transaction processing
            \end{itemize}
        \item \textbf{Other Capabilities of DBMS Systems}: Support for at least one data model through which the user can view the data, There is at least one abstract model of data that allows the user to see the "information" in the database, Relational, hierarchical, network, inverted list, or object-oriented
            \bigbreak \noindent 
            Support for at least one data model through which the user can view the data
            \begin{itemize}
                \item efficient file access which allows us to "find the boss of Susie Jones"
                \item allows us to "navigate" within the data
                \item allows us to combine values in 2 or more databases to obtain "information"
            \end{itemize}
            \bigbreak \noindent 
            Support for high-level languages that allow the user to define the structure of the data, access that data, and manipulate it
            \begin{itemize}
                \item Data Definition Language (DDL)
                \item Data Manipulation Language (DML)
                \item Data Control Language (DCL)
                \item query language access data
                \item operations such as add, delete, and replace
            \end{itemize}
        \item \textbf{Transaction Management}: Transaction management is a feature that provides correct, concurrent access to the database, possibly by many users at the same time, ability to simultaneously manage large numbers of \textit{transactions}
        \item \textbf{Access Control}: Access control is the ability to limit access to data by unauthorized users along with the capability to check the validity of the data. This is to protect against loss when database crashes and prevent unauthorized access to portions of the data
        \item \textbf{Resiliency}: Resiliency is the ability to recover from system failures without losing data, Ideally, should be able to recover from any type of failure, such as 
            \begin{itemize}
                \item sabotage
                \item acts of God
                \item hardware failure
                \item software failure
                \item etc.
            \end{itemize}
            \bigbreak \noindent 
            \textbf{Note}: Obviously, some of these would require more than just software - offsite backups, etc
        \item \textbf{Use of Conceptual Modeling}:
            \bigbreak \noindent 
            \fig{.5}{./figures/2.png}
        \item \textbf{Leveled Architecture of a DBMS}:
            \bigbreak \noindent 
            \fig{.5}{./figures/3.png}
        \item \textbf{External level}: a view or sub-schema, a portion of the logical database, may be in a higher level language
        \item \textbf{Logical Level}: abstraction of the real world as it pertains to the users of the database. DBMS provides a data definition language (DDL) to describe the logical schema in terms of a specific data model such as relational, hierarchical, network, inverted list, etc.
        \item \textbf{Physical Level}: The collection of files and indices, the collection of files and indices, this is the actual data
        \item \textbf{Instance}: An instance of the database is the actual contents of the data, it could be 
            \begin{itemize}
                \item the extension of the database
                \item current state of the database
                \item a snapshot of the data at a given point in time
            \end{itemize}
        \item \textbf{Schema}: The schema of a database is the data about what the data represents. Such as,
            \begin{itemize}
                \item plan of the database
                \item logical plan
                \item physical plan
                \item the intention of the database
            \end{itemize}
        \item \textbf{Schema vs Instance}:
            \bigbreak \noindent 
            \fig{.5}{./figures/4.png}
        \item \textbf{Data Independence}: Data Independence is a property of an appropriately designed database system,  it has to do with the mapping of logical level to physical level, and logical to external
            \begin{itemize}
                \item \textbf{Physical data independence}:  Physical schema can be changed without modifying logical schema
                \item \textbf{Logical data independence}: logical schema can be changed without having to modify any of the external views
            \end{itemize}
        \item \textbf{DCL (Control), DDL (Definition), DML (Manipulation)}: may be completely separate (example is IMS), may be intermixed (DB2), or may be a host language, for example an  application program in which DML commands are embedded such as COBOL or PL/I
        \item \textbf{DBMS Components}:
            \bigbreak \noindent 
            \fig{.5}{./figures/5.png}
        \item \textbf{Overall DBMS Usage Scenario}: Database Administrator (DBA) define the conceptual, logical, and physical levels using DDL.  DBMS software stores instances of these in schemas.  User defines views (External Schema) in DDL. User accesses database using DML
        \item \textbf{Advantages of a Database}:
            \begin{itemize}
                \item Controlled redundancy
                \item Reduced inconsistency in the data
                \item Shared access to data
                \item Standards enforced
                \item Security restrictions maintained
                \item Integrity maintained more easily
                \item Provides capability for backup and recovery
                \item Permitting inferences and actions using rules
            \end{itemize}
        \item \textbf{Disadvantages of a Database}:
            \begin{itemize}
                \item Increased complexity needed to implement concurrency control
                \item Increased complexity needed for centralized access control
                \item Security needed to allow the sharing of data
                \item Necessary redundancies can cause complexity when updating
            \end{itemize}
        \item \textbf{Data vs Information}:
            \begin{itemize}
                \item \textbf{Data}: Data refers to raw, unprocessed facts, figures, and details. It represents basic elements that have not been interpreted or given any meaning.
                \item \textbf{Information}: Information is processed, organized, or structured data that is meaningful and useful. It is data that has been interpreted or analyzed to provide context, relevance, and purpose.
            \end{itemize}
    \end{itemize}

    \pagebreak 
    \subsection{Conceptual Modeling and ER Diagrams}
    \bigbreak \noindent 
    \subsubsection{Definitions and theorems}
    \begin{itemize}
        \item \textbf{Data Models}: A means of describing the structure of data, we typically have A set of operations that manipulate the data (for data models that are implemented)
        \item \textbf{Types of data models}:
            \begin{itemize}
                \item Conceptual data model
                \item  Logical data models - relational, network, hierarchical, inverted list, or object-oriented
            \end{itemize}
        \item \textbf{Conceptual Data Model}:
            \begin{itemize}
                \item Shows the structure of the data including how things are related
                \item Communication tool
                \item Independent of commercial DBMSes
                \item Relatively easy to learn and use
                \item Helps show the semantics or meaning of the data
                \item Graphical representation
                \item Entity-Relationship Model is very common
            \end{itemize}
        \item \textbf{Logical Data Models - Relational}: Data is stored in relations (tables). These tables have one value per cell. Based upon a mathematical model.
        \item \textbf{Logical Data Models - Network}: Data is stored in records (vertices) and associations between them (edges), Based upon a model called CODASYL
        \item \textbf{Logical Data Models - Hierarchical}: Data is stored in a tree structure with parent/child relationships
        \item \textbf{Logical Data Models - Inverted List}: Tabular representation of the data using indices to access the tables, Almost relational, but it allows for non-atomic data values \footnote{ "Non-atomic data values" refer to data structures or values that are composed of multiple components, as opposed to atomic data values, which are indivisible and represent a single value.}, which are not allowed in relations
        \item \textbf{Logical Data Models - Object Oriented}: Data stored as objects which contain
            \begin{itemize}
                \item Identifier
                \item Name
                \item Lifetime
                \item Structure
            \end{itemize}
        \item \textbf{Entity-Relationship Model}: Meant to be simple and easy to read. Should be able to convey the design both to database designers and unsophisticated users
        \item \textbf{Entities}: Principle objects about which information is kept - These are the *things* we store data about. If you look at the ER Diagram like a spoken language, the entities are nouns - Person, place, thing, event. When drawn on the ER diagram, entities are shown as rectangles with the name of the entity inside.
            \bigbreak \noindent 
            \fig{.5}{./figures/7.png}
        \item \textbf{Relationships}:  Relationships connect one or more entities together to show an association. A relationship \textit{cannot} exist without at least one associated entity.  Graphically represented as a diamond with the name of the relationship inside, or just beside it
            \bigbreak \noindent 
            \fig{.7}{./figures/8.png}
        \item \textbf{Attributes}: Characteristics of entities \textbf{OR} of relationships, Represent some small piece of associated data, Represented by either a rounded rectangle or an oval.
            \bigbreak \noindent 
            \fig{.5}{./figures/9.png}
        \item \textbf{Attributes on Entities}: When an attribute is attached to an entity, it is expected to have a value for every instance of that entity, unless it is
            allowed to be null. For instance, in the diagram above, Name was an attribute of Person. Every person
            that we store data about will have a value for Name.
        \item \textbf{Attributes on Relationships}: When an attribute is attached to a relationship, it is only expected to have a value when the entities involved in the
            relationship come together in the appropriate way.
            In the diagram from before, the Amount attribute is attached to the donates relationship, which connects the
            Person and Charity entities. Amount will have one value for each time a Person donates to a Charity, denoting how
            much that person donated to the charity. It will not necessarily have a value for a given person, or a given charity.
            This can be referred to as the \textbf{intersection data}.
        \item \textbf{Types of attributes}:
            \fig{.5}{./figures/10.png}
        \item \textbf{Degree of a Relationship}: The degree of a relationship is defined as how many entities it associates. If one entity is associated more than once
            (such as with a recursive relationship), then the degree counts each time it is referenced.
            \bigbreak \noindent 
            \fig{.5}{./figures/11.png}
            \bigbreak \noindent 
            \textbf{Note:} There is no limit to how many entities there can be in a relationship. After binary, and ternary, we start to call the relationships $n$-ary, where $n$ is the degree
        \item \textbf{Connectivity of a Relationship}:
            \begin{itemize}
                \item A constraint of the mapping of associated entities
                \item Written as (minimum, maximum).
                \item Minimum is usually zero or one.
                \item Maximum is a number (commonly one) or can be a letter denoting many.
                \item The actual number is called the cardinality.
            \end{itemize}
            \bigbreak \noindent 
            \fig{.5}{./figures/12.png}
            \bigbreak \noindent 
            Together (from the image) both sides make up the connectivity, to refer to a single side, we use the term "cardinality", ie the cardinality of a person is (1,1). If we hold Address constant (We know a specific address and are therefore refering to that), how many persons may live at that address, in this case (1,1)
        \item \textbf{Attributes on Relationships (revisited)}: Must be on a many-to-many relationship. (1-many and 1-to-1 relationships should have the attribute on one of
the entities involved.  Someone needs to know all of the associated entities to access the attribute.
        \item \textbf{Reading Cardinalities}: For binary relationships:
            \begin{itemize}
                \item For each Thing that smurfs, there are a minimum of $c$, and a maximum of $d$ Objects.
                \item For each Object that smurfs/is smurfed, there is a minimum of $a$ and a maximum of $b$ Things
            \end{itemize}
            \bigbreak \noindent 
            \fig{.5}{./figures/13.png}
        \item \textbf{Weak Entities}: Sometimes you may run into an entity that depends upon another entity for its existence. The weak entity is a tool you can use to represent this.:w
            \bigbreak \noindent 
            Weak entities are written like normal entities, except that they have a double rectangle outline. The relationship
that connects the weak entity to the strong entity it depends upon will be written with a double diamond. This
does not mean that the relationship is weak. It is just to indicate upon which entity the weak entity depends.
\bigbreak \noindent 
\fig{.5}{./figures/14.png}
        \item \textbf{Recursive Relationships}: It is possible for an entity to have a relationship with itself. This is called a recursive relationship. It makes more sense if you think of entities as collections of objects of their appropriate type
        \item \textbf{Recursive Relationships - Many-To-Many}: A many-to-many recursive relationship means that the objects are arranged in a network structure, Notice that the minimum is 0 on both sides. This is important.
            \bigbreak \noindent 
            \fig{.5}{./figures/15.png}
        \item \textbf{Recursive Relationships - One-To-Many}: A one-to-many recursive relationship means that the objects are arranged in a tree structure, Notice that the minimum is still 0 on both sides. This is important.
            \bigbreak \noindent 
            \fig{.4}{./figures/16.png}
        \item \textbf{Entity or Attribute?}: 
            Sometimes it isn't clear whether something should be an entity or an attribute of some other entity. Usually the
decision will come down to how complicated it is to store the data, and how important it is. If it ends up being used
in multiple places, it might be a clue that you should use an entity
        \item \textbf{Inheritance}: Two types of inheritance available
            \begin{itemize}
                \item "is a" inheritance. This shows that the subtype IS a member of the supertype.
                \item "is part of " inheritance. This shows that the supertype contains, or is made up of members of the subtypes.
            \end{itemize}
            \bigbreak \noindent 
            All attributes of the supertype entity are inherited by the subtype entities. The identifier of the subtypes will be the same as the supertype
            \bigbreak \noindent 
        \item \textbf{IS A Inheritance}:  This type of inheritance happens when you have a supertype and one or more subtypes that are members
            of the supertype. Denoted by an upside-down triangle, with the supertype on top, and the subtypes coming out the bottom.
            \bigbreak \noindent 
            \fig{.5}{./figures/17.png}
        \item \textbf{Defining IS-A inheritance}: There are two things you need to choose when using IS-A inheritance:
            \begin{itemize}
                \item \textbf{Generalization (no) vs. specialization (yes)}: can the supertype occur without being a member of the specified subtypes?
                \item \textbf{Overlapped (yes) vs. disjoint subtypes (no)}: is it possible for a single occurrence of the supertype to be a member of more than one subtype?
            \end{itemize}
            \bigbreak \noindent 
            They are mutually exclusive so you need to pick one of each, ie. GO, GD, SO, SD
        \item \textbf{IS-A inheritance - Generalization}:  Supertype is the union of all of the subtypes, This means that an instance of the supertype CANNOT EXIST without belonging to at least one subtype.
        \item \textbf{IS-A inheritance - Specialization}: The subtype entities specialize the supertype, This means that an instance of the supertype CAN exist without being related to any of the subtypes
        \item \textbf{IS-A inheritance - Overlapping Subtypes}: It is possible for an instance of the supertype to be related to more than one of the subtypes
        \item \textbf{IS-A inheritance - Disjoint Subtypes}: the subtype entities are mutually exclusive, it is not possible for an instance of the supertype to be related to more than one subtype.
        \item \textbf{IS-PART-OF Inheritance}: "Is part of " inheritance indicates that the
supertype is constructed from instances of the
subtypes. It is shown on an ER diagram as a circle,
with the supertype on the top, and subtypes on
the bottom.
\bigbreak \noindent 
\fig{.5}{./figures/18.png}
    \item \textbf{Warning about IS-PART-OF}:  The IS PART OF inheritance operator does have its uses, but it is not very commonly used, If you see something involving a certain number of things being present, there are several possibilities
        \begin{itemize}
            \item Sometimes a number is specified that isn't actually important for what we are modeling. This won't even be represented on an ER Diagram. This is the case when changing the number wouldn't have any effect on the necessary structure of a database.
            \item If you need a certain number of items for a relationship to hold, you should explore using the connectivity of the relationship to express that.
            \item Finally, this IS PART OF inheritance might be useful. It is almost never necessary, however.
        \end{itemize}
    \item \textbf{Are you actually representing what you want to?}: Let's say you're running a business selling used cars. A simple ER diagram for the sales might look like the following:
        \bigbreak \noindent 
        \fig{.4}{./figures/19.png}
        \bigbreak \noindent 
        The resulting database would have one entry for each time a specific person buys a specific car. If the same person
buys the same car more than once (obviously selling it to someone else at some point), this model would no longer
be appropriate.
\bigbreak \noindent 
Adding a new entity to the relationship for the date/time of the purchase can fix this problem.
\bigbreak \noindent 
\fig{.5}{./figures/20.png}
\bigbreak \noindent 
Notice that the connectivities can change when you add new entities to the relationship.
\item \textbf{Weak Entities - Introduction}: So far, all of the entities we have used have been things that stand on their own. There are some situations where
we are modeling an object for which we certainly need to store data, but the items exist only in the context of some
other entity. Many of these examples can occur
\bigbreak \noindent 
One example of a time that an entity depends on another would be the idea of a city. Within a state, we can
generally be assured that cities will have unique names. If we were working only at that level, the City could be an
entity as we saw above. A good identifier for it would be the name of the city, so we would see the following:
\bigbreak \noindent 
\fig{.5}{./figures/21.png}
\bigbreak \noindent 
In some situations, this would be valid. The Name attribute can serve, in those circumstances, as an appropriate
identifier.
\bigbreak \noindent 
To indicate this sort of dependency, we can make the dependent entity a "weak" entity. This is drawn with a
double-edged rectangle, shown below.
\bigbreak \noindent 
\fig{.4}{./figures/22.png}
\bigbreak \noindent 
Notice that the City entity is now drawn as a weak entity, with a double border. The relationship between the weak
entity and the strong entity is also drawn with a double border. The relationship is not weak, per se, but it is used to
indicate which strong entity the weak entity depends upon.
    \item \textbf{Discriminant (partial key)}:  The discriminant, also known as the partial key, is an attribute (or a set of attributes) within the weak entity that can uniquely identify the weak entity, but only in combination with the primary key of the strong entity it is associated with. In other words, the discriminant helps to distinguish instances of the weak entity when they are tied to a particular instance of the strong entity.
    \item \textbf{Schema}: In databases, a schema is the structural definition of how data is organized in a database. It outlines the way data is stored

    \end{itemize}

    \pagebreak 
    \subsection{The Relational Model}
    \begin{itemize}
        \item \textbf{Basic Structure}: 
            \begin{itemize}
                \item \textbf{Relations}: In the relational data model, our database is made up of one or more \textbf{relations} (tables). Each relation should have a unique name.
                \item \textbf{Schema}: The schema of a relation is written as \textbf{Relation\_Name}($A_{1}, A_{2}, ...,A_{n} $), Where $A_{1}, A_{2}, ...,A_{n}$ are placeholders for the attribute names
                \item \textbf{Column headers (attributes)}: The attributes becomes the column headers of the relation.
                \item \textbf{Instance data, tuples}: When there is instance data, it will come in the form of \textbf{tuples} (rows), which have a value for each attribute, as shown below
            \end{itemize}
            \bigbreak \noindent 
            \textbf{Note:} No field may contain than one value.

            \begin{table}[h!]
                \centering
                \begin{tabular}{|>{\centering\arraybackslash}m{2cm}|>{\centering\arraybackslash}m{2cm}|>{\centering\arraybackslash}m{2cm}|c|>{\centering\arraybackslash}m{2cm}|}
                    \hline
                    \multicolumn{5}{|c|}{\textbf{Relation\_Name}} \\ \hline
                    $A_1$ & $A_2$ & $A_3$ & $\dots$ & $A_n$ \\ \hline
                    $x_1$ & $x_2$ & $x_3$ & $\dots$ & $x_n$ \\ \hline
                    $y_1$ & $y_2$ & $y_3$ & $\dots$ & $y_n$ \\ \hline
                    $\dots$ & $\dots$ & $\dots$ &  & $\dots$ \\ \hline
                \end{tabular}
            \end{table}
        \item \textbf{The domain of an attribute}: Each attribute becomes a column heading
            \bigbreak \noindent 
            Each attribute (column) also has an associated \textbf{domain}. The domain of an attribute is the set of all valid values for it.  The domain may be looked at as a data type, but may have additional constraints.
        \item \textbf{The domain of a set of attributes}: The domain of a set of attributes is the set of all possible combinations of values for the attributes in the set.
        \item \textbf{Tuples (Rows)}: A tuple is a special type of (mathematical) set containing values for each attribute within the relation. Tuples are shown as rows in the table, with the value for each attribute under the appropriate column
        \item \textbf{Atomic tuples}:  The values are required to be atomic; there can be only one value per tuple per attribute
        \item \textbf{Relation vs relationship}: Though they have similar names, A relation (table) and a relationship (from an ER diagram) \textbf{ARE NOT} the same thing.
            \begin{itemize}
                \item \textbf{Degree of relation}: The degree of a relation is the number of attributes present.
                \item \textbf{Cardinality of a Relation}: The cardinality of a relation is the number of tuples present.
            \end{itemize}
        \item \textbf{Keys}: Speaking generally, the purpose of a key is to uniquely identify a tuple in some relation.
            \begin{itemize}
                \item \textbf{Super keys}: A super key within a relation is an attribute or set of attributes whose values can uniquely identify any tuple within that relation
                \item \textbf{The trivial key}: Every relation has at least one - the set of all attributes in the relation 
                \item \textbf{Candidate Keys}: A candidate key is a minimal super key - the minimum set of attributes necessary to uniquely identify a tuple within the relation
                \item \textbf{Primary Key}: The primary key for a relation is chosen by the database designer from among the relation's candidate keys. It becomes the "official" key that is used to reference tuples within the relation. There can be only one
                \item \textbf{Prime, non-prime attributes}: Once a primary key is chosen, each of the attributes in the relation will be either \textbf{prime} or \textbf{non-prime} with respect to the relation. A prime attribute is one of the attributes that can be found in any of the candidate keys. A non-prime attribute is one of the attributes not found in any of the candidate keys
                    \bigbreak \noindent 
                    Once a primary key is chosen for it, the schema of a relation is written with the primary key's attributes underlined
                \item \textbf{Foreign Keys}: A foreign key is a tool used to link relations within a database. Since every relation has a primary key that uniquely identifies each tuple, the values of those key attributes can be used from another relation to reference individual tuples.
                    \bigbreak \noindent 
                    The relation whose primary key is being used is the \textbf{home relation}
            \end{itemize}
        \item \textbf{Order Independence}: In relations, the order things appear doesn't matter. There are ways to force them to sort later when we're working with SQL, but the relation itself has no order for either rows or attributes...
        \item \textbf{Order Independence - Attributes}: It doesn't matter what order the attributes appear in, if two relational schemas have the same name, the same attributes, and the same primary key, then they are equivalent.
        \item \textbf{Order Independence - Tuples}: Tuples are stored unordered. If you need to have them appear in some order later, you will be able to sort based on the values inside of them using SQL.
        \item \textbf{Constraints}: Constraints are limits imposed on the domains of various attributes. These can come from the system your database is modeling
        \item \textbf{Entity Integrity Constraint}: The entity integrity constraint applies to all relations. It states that no tuple may exist within a relation that has null value for any of attributes that make up the primary key. This is a consequence of the primary key being a candidate key, which is minimal and cannot do its job with any less data.
        \item \textbf{Referential Integrity Constraint}: It constrains the values of foreign keys in relations to values that actually exist as primary keys for tuples within the home relation. If the foreign key is otherwise allowed to be NULL, then that is also an acceptable value.
        \item \textbf{Summary: Terms}:
            \begin{itemize}
                \item \textbf{Relations}: Tables
                \item \textbf{Colums}: Attributes
                \item \textbf{Tuples}: The rows in the relation that holds the instance datae
                \item \textbf{Domain of an attribute}: Set of all possible values for the attribute
                \item \textbf{Domain of a set of attributes}: Set of all possible combinations of values for the attributes in the set
                \item \textbf{Degree of relation:} The degree of a relation is the number of attributes present.
                \item \textbf{Cardinality of a Relation:} The cardinality of a relation is the number of tuples present.
            \end{itemize}
    \end{itemize}

    \pagebreak 
    \subsection{Relational Model Normalization}
    \begin{itemize}
        \item \textbf{Designing Relational Databases}: There are a large number of possible ways to represent each problem with using relations. Some choices will perform better than others for various reasons. The option chosen should be the best one, but how do we know which one that is?
            \bigbreak \noindent 
            We should study:
            \begin{itemize}
                \item Problems that can come up
                \item How to avoid them
                \item Desirable properties
                \item How to guarantee them
            \end{itemize}
        \item \textbf{Basic Example}: If our database is a single relation with schema \textbf{SP}(\underline{SuppName}, SuppAddr, \underline{Item}, Price) with the instance data:
            \bigbreak \noindent 
            \begin{center}
                \begin{tabular}{|l|l|l|l|}
                    \hline
                    \textbf{SuppName} & \textbf{SuppAddr} & \textbf{Item} & \textbf{Price} \\ \hline
                    John    & 10 Main   & Apple   & \$2.00  \\ \hline
                    John    & 10 Main   & Orange  & \$2.50  \\ \hline
                    Jane    & 20 State  & Grape   & \$1.25  \\ \hline
                    Jane    & 20 State  & Apple   & \$2.25  \\ \hline
                    Frank   & 30 Elm    & Mango   & \$6.00  \\ \hline
                \end{tabular}
            \end{center}
            \bigbreak \noindent 
            There are some common things that we might want to do that would cause issues
            \begin{itemize}
                \item \textbf{Insertion Anomaly}: Let's say we want to add a new vendor, "Sally", and store her address, "40 Pine", but she is not selling anything yet. Can this be inserted into the relation SP?
                    \bigbreak \noindent 
                    \textbf{NO}. The primary key is (SuppName, Item), but we only have SuppName. The entity integrity constraint is violated if we try to insert the data as a tuple in this relation. It cannot fit. We call this an insertion anomaly.
                \item \textbf{Deletion Anomaly}: This time, let's say that Frank no longer sells Mango. We want to take that out of the database so nobody can order a mango that is not available. Can this tuple remain in the relation with the Mango information removed?
                    \bigbreak \noindent 
                    \textbf{NO}. The primary key is (SuppName, Item), and the Item is going away. The entity integrity constraint is violated if we remove the data from the tuple in this relation. We can either keep the whole tuple, advertising fake mango, or delete the whole tuple and lose the information on Frank, which doesn't exist in any other tuples. We call this a deletion anomaly.
                \item \textbf{Update Anomaly}: Next, let's say that John is moving to a different address. We would have to change it once for every item John is selling. This isn't a big deal with only two items, but as John's list of supplied items grows, so does the amount of database work that needs to be done every time he moves. If any of the SuppAddr values for John don't agree, then it may not be clear which is the right address for John. This is an update anomaly.
                \item \textbf{Redundancy}: Redundancy is when values are repeated.
                    \bigbreak \noindent 
                    It can be
                    \begin{itemize}
                        \item \textbf{Good:} If you have an off-site backup of your entire database, the redundancy is useful, and can be used to restore in case of a failure.
                        \item \textbf{Bad:} Redundancy on the same physical device is unnecessary. It wastes space and comes with the potential for update anomalies.
                    \end{itemize}
                \item \textbf{Note:} The good redundancy is something the DBA/IT department should handle. When we talk about redundancy in the design of our database, we will be talking about the bad kind.
            \end{itemize}
        \item \textbf{Anomolies summarized}:
            \bigbreak \noindent 
            \textbf{Insertion anomalies}:
            \begin{itemize}
                \item When a piece of data cannot be inserted because it violates some constraint of the relation.
                \item Usually this is the entity integrity constraint being violated, but not always. See the Sally example
            \end{itemize}
            \bigbreak \noindent 
            \textbf{Deletion anomalies}: 
            \begin{itemize}
                \item When deleting some piece of data, a deletion anomaly is when more data is lost than intended
                \item Usually this is caused when the data removed is part of the primary key, which would cause a violation of the entity integrity constraint. See the Frank example
            \end{itemize}
            \bigbreak \noindent 
            \textbf{Update anomalies}:
            \begin{itemize}
                \item When updating a single value requires changes to multiple tuples, this is an update anomaly. See the John example.
                \item This is caused by unnecessary redundancies in the data.
                \item These cause inefficiency, and potential inconsistencies.
            \end{itemize}
            \bigbreak \noindent 
        \item \textbf{Decomposition}: There is no rule that says that a relational database must be made up of a single relation. The way we will solve these anomalies is to add new relations to our database and change the old ones. This is called decomposition.
            \bigbreak \noindent 
            Using the example from above, we can remove the anomalies by decomposing the database into two relations.
            \bigbreak \noindent 
            \textbf{SP}(\underline{SuppName}, \underline{Item}, Price)
            \bigbreak \noindent 
            \begin{center}
                \begin{tabular}{|l|l|l|}
                    \hline
                    \textbf{SuppName} & \textbf{Item} & \textbf{Price} \\ \hline
                    John    & Apple   & \$2.00  \\ \hline
                    John    & Orange  & \$2.50  \\ \hline
                    Jane    & Grape   & \$1.25  \\ \hline
                    Jane    & Apple   & \$2.25  \\ \hline
                \end{tabular}
            \end{center}
            \bigbreak \noindent 
            \textbf{S}(SuppName, SuppAddr)
            \bigbreak \noindent 
            \begin{center}
                \begin{tabular}{|l|l|}
                    \hline
                    \textbf{SuppName} & \textbf{SuppAddr} \\ \hline
                    John    & 10 Main   \\ \hline
                    Jane    & 20 State  \\ \hline
                    Frank   & 30 Elm    \\ \hline
                    Sally   & 40 Pine   \\ \hline
                \end{tabular}

            \end{center}
        \item \textbf{When to decompose}: One way of designing a database could be to list all of the possible anomalies and then decompose to fix each of them. The problem with this is that any anomalies you don't see coming will not be fixed.
            \bigbreak \noindent 
            We will look at a systematic method of identifying the potential for anomalies. This method is called normalization
        \item \textbf{Normalization}: Normalization involves making sure that each of your relations follows certain rules. Depending on which rules are followed, each of the relations in your database will be in one or more normal forms. These rules are based on functional dependencies
        \item \textbf{Functional Dependencies}: A functional dependency is a statement about which attributes can be inferred from other attributes. If we take $X$ and $Y$ as sets of attributes, we can write:
            \begin{align*}
                X \to Y
            .\end{align*}
            \bigbreak \noindent 
            Which means, if, whenever unique values for \textbf{all} of the attributes in $X$ are known, unique values for \textbf{each} of the attributes of $Y$ are guaranteed to be possible to look up or to infer using those values.
            \bigbreak \noindent 
            This is read either as:
            \begin{itemize}
                \item $X$ functionally determines $Y$
                \item $Y$ is functionally dependent upon $X$
            \end{itemize}
        \item \textbf{Functional Dependencies: Real-life Examples}: 
            \begin{itemize}
                \item \textbf{ZID $\to$ StudentFirstName, StudentLastName, Birthday}: If I identify a student using their ZID, that student has one first name, last name, and birthday
                \item \textbf{StudentFirstName \not \to ZID}: The first name is not enough to determine a single ZID, as there are multiple students with the same first name
                \item \textbf{ZID, CourseID, Semester \to Grade}: If I know which student, which course, and which semester, I can find a single grade
            \end{itemize}
        \item \textbf{Functional Dependencies: Keep In Mind}: FDs are constraints present within the operational data your database models. They don't necessarily describe how things work in the real world, but they do have to accurately describe any data you will store in your database
            \bigbreak \noindent 
            FDs \textbf{must} hold for all possible data values. Attempts to add data that does not obey the FDs will result in anomalies.
            \bigbreak \noindent 
            FDs can be enforced during insertion if the database is set up properly
        \item \textbf{Armstrong's Axioms}: Armstrong's Axioms are a set of rules for operations that are permissible when manipulating functional dependencies
            \begin{itemize}
                \item \textbf{Reflexivity}: If $Y \subseteq X$, then $X \to Y $
                \item \textbf{Augmentation}: If $X \to Y$, the $XZ \to YZ$ for any $Z$
                \item \textbf{Transitivity}: If $X \to Y$ and $Y \to Z$, then $X \to Z $
                \item \textbf{Decomposition}: If $X \to YZ$, then $X \to Y$ and $X \to Z $
                \item \textbf{Composition}: If $X \to Y$ and $A \to B$, then $XA \to YB $
                \item \textbf{Union (Notation)}: If $X\to Y$ and $Y \to Z$, then $X \to YZ $
                \item \textbf{Pseudo-transitivity}: If $X \to Y$ and $YZ \to W$, then $XZ \to W $
                \item \textbf{Self-determination}: $I \to I $ for any $I$
            \end{itemize}
        \item \textbf{Functional Dependencies: Keys Revisited}: Now that we know about functional dependencies (FDs), we can assert:
            \bigbreak \noindent 
            \begin{center}
                The attributes of a superkey must functionally determine all of the attributes of the relation.
            \end{center}
            \bigbreak \noindent 
            Candidate keys and primary keys are superkeys, so this is true of them as well, and they also satisfy additional requirements.
            \bigbreak \noindent 
            \textbf{Example:} As an example, say we have the relation \textbf{R}(\underline{a},b,c,d,e,f). We can say
            \begin{align*}
                a &\to a,b,c,d,e,f \\
                \implies a&\to b,c,d,e,f
            .\end{align*}
        \item \textbf{First Normal Form (1NF)}: You should recall from the introduction to relations that all of the values in a tuple with a relation must be atomic. This means that there is a maximum of one value per attribute per tuple
            \bigbreak \noindent 
            The requirement for a relation to be in First Normal Form (1NF) is this same requirement that all of the values must be atomic
            \bigbreak \noindent 
            What this usually looks like is a table with mutltiple values in a single cell. A non-1NF relation would not even technically count as a relation.
            \bigbreak \noindent 
            Given the table:
            \bigbreak \noindent 
            \begin{center}
                \begin{tabular}{|c|c|c|}
                    \hline
                    X  & Y  & Z  \\ \hline
                    x1 & y1 & z1 \\ 
                       &    & z2 \\ 
                       &    & z3 \\ \hline
                    x2 & y2 & z4 \\ \hline
                    x3 & y2 & z5 \\ \hline
                \end{tabular}
            \end{center}
            \bigbreak \noindent 
            It looks like $X$ would have been the primary key, but it's not doing its job of uniquely determining $Z$, which is showing as a repeating group so $X$ can't be a key
            \bigbreak \noindent 
            What usually causes this is not having the correct primary key
            \bigbreak \noindent 
            The table above has the following function dependencies:
            \begin{align*}
                X &\to Y \\
               X,Z &\to Z
            .\end{align*}
            \bigbreak \noindent 
            To move this pseudo-relation into an actual relation that doesn't violate 1NF, we need to choose a real primary key that meets the requirements. We do that using the FDs. In this case, ($X$ , $Z$) works.
            \bigbreak \noindent 
            Changing the primary key yields: - $R$($X$, $Y$ , $Z$))
            \bigbreak \noindent 
            \begin{center}
                \begin{tabular}{c|c|c}
                    X  & Y  & Z  \\ \hline
                    x1 & y1 & z1 \\ \hline
                    x1 & y1 & z2 \\ \hline
                    x1 & y1 & z3 \\ \hline
                    x2 & y2 & z4 \\ \hline
                    x3 &y2 & z5 
                \end{tabular}
            \end{center}
        \item \textbf{Pseudo-relation}: The notation for a "pseudo-relation" like the one above would be to use inner parenthesis on the repeating group, ie. \textbf{R}($X$, $Y$ , ($Z$))
        \item \textbf{Second Normal Form (2NF)}: Second Normal Form (2NF) has to do with the concept of full dependence.
            \bigbreak \noindent 
            Given two sets of attributes, $X$ and $Y$ , we can say that $Y$  is fully dependent on $X$, if (and only if)
            \begin{align*}
                X \to Y
            .\end{align*}
            And no subset of $X$ determines $Y$
            \bigbreak \noindent 
            A relation is in 2NF if:
            \begin{itemize}
                \item It already meets the requirements of 1NF, and
                \item All non-prime attributes of the relation are fully dependent upon the entire primary key
            \end{itemize}
            \bigbreak \noindent 
            What breaks 2NF is when attributes are dependent upon only part of the primary key. To fix 2NF violations once we're in 1NF, decomposition is the solution.
            \bigbreak \noindent 
            \textbf{Example:} Going back to our earlier example: \textbf{EmpProj}(\underline{EmpID}, \underline{Project}, Supv, Dept, Case)
            \bigbreak \noindent 
            \begin{center}
            \begin{tabular}{|c|c|c|c|c|}
                \hline
                EmpID & Project & Supv & Dept & Case \\ \hline
                e1    & p1      & s1   & d1   & c1   \\ \hline
                e2    & p2      & s2   & d2   & c2   \\ \hline
                e1    & p3      & s1   & d1   & c3   \\ \hline
                e3    & p3      & s1   & d1   & c3   \\ \hline
            \end{tabular}
        \end{center}
        \bigbreak \noindent 
            \textbf{Functional Dependencies}:
            \begin{center}
                EmpID, Project $\to$ Supv, Dept, Case \\
                EmpID $\to$ Supv, Dept \\
                Supv $\to$ Dept
            \end{center}
            \bigbreak \noindent 
            A quick glance confirms all of the values are atomic, so 1NF is confirmed.
            \bigbreak \noindent 
            There is a 2NF violation caused by (EmpID $\to$ Supv, Dept) because the primary key is (EmpID, Project), but only EmpID is on the LHS.
            \bigbreak \noindent 
            Observing the instance data, you should easily see that the attributes of the RHS cause update anomalies in this
            table. We also can't insert a new employee with no project (insertion anomaly), and removing e2 from p2 would
            remove e2 from the database entirely (deletion anomaly). These are symptoms of the 2NF violation.
            \bigbreak \noindent 
            \textbf{Decomposition Pattern}: There is a pattern to follow for the decomposition. Start with the original relation, and the FD that causes the violation.
            \begin{align*}
                &\text{\textbf{EmpProj}(\underline{EmpID}, \underline{Project}, Supv, Dept, Case)} \\
                &\text{\textbf{EmpID} $\to$ Supv, Dept}
            .\end{align*}
            \bigbreak \noindent 
            The attributes on the RHS of the FD are removed from the original relation and placed into a newly created relation that has the FD's LHS as its primary key. A foreign key links the attribute from the LHS in the original table (the LHS is not removed) to the corresponding tuple in the new table, where it is the primary key.
            \bigbreak \noindent 
            \begin{align*}
                &\text{\textbf{EmpProj}(EmpID, Project, Case)} \\
                &\text{\textbf{Employee}(EmpID, Supv, Dept)}
            .\end{align*}
            \bigbreak \noindent 
            \textbf{Instance of 2NF Version}:
            \bigbreak \noindent 
            \textbf{EmpProj}(EmpID, Project, Case)                           
            \begin{center}
                \begin{tabular}{|c|c|c|}
                    \hline
                    EmpID & Project & Case \\ \hline
                    e1    & p1      & c1   \\ \hline
                    e2    & p2      & c2   \\ \hline
                    e1    & p3      & c3   \\ \hline
                    e3    & p3      & c3   \\ \hline
                \end{tabular}
            \end{center}
            \bigbreak \noindent 
            \textbf{Employee} (EmpID, Supv, Dept
            \bigbreak \noindent 
            \begin{center}
                \begin{tabular}{|c|c|c|}
                    \hline
                    EmpID & Supv & Dept \\ \hline
                    e1    & s1   & d1   \\ \hline
                    e2    & s2   & d2   \\ \hline
                    e3    & s1   & d1   \\ \hline
                \end{tabular}
            \end{center}
            \bigbreak \noindent 
        \item \textbf{Third Normal Form (3NF)}: To be in Third Normal Form (3NF), a relation must
            \begin{enumerate}
                \item already qualify to be in 2NF
                \item none of the non-prime attributes may be transitively dependent upon the primary key
            \end{enumerate}
            \bigbreak \noindent 
            By definition, all non-prime attribute are functionally dependent upon the primary key. What makes a transitive dependency is that there is also some non-prime attribute (which also depends on the key) that also functionally determines the attribute.
            \bigbreak \noindent 
            To quickly identify the transitive dependencies from the list of FDs, look on the LHS for attributes that are non-prime in the context of the current relation.
            \bigbreak \noindent 
            \textbf{Example:}
            \begin{align*}
                &\text{\textbf{EmpProj}(EmpID, Project, Case)} \\
                &\text{\textbf{Employee} (EmpID, Supv, Dept} \\
                &\text{EmpID, Project $\to$ Supv, Dept, Case} \\
                &\text{EmpID $\to$ Supv, Dept} \\
                &\text{Supv $\to$ Dept}
            .\end{align*}
            \bigbreak \noindent 
            In this case, the FD that causes our relations to violate 3NF is (Supv $\to$ Dept), and the violation happens in the
            Employee relation. If you refer back to the instance data of that in the 2NF solution, you can see that the
            violation can cause anomalies, so we want to fix it.
            \bigbreak \noindent 
            Just like 2NF, we fix 3NF by decomposing using the FD that causes the violation to occur. \textbf{AT NO POINT DO WE CHANGE THE FDs}
            \bigbreak \noindent 
            \textbf{Decomposition Pattern}: We follow the same pattern for decomposition in 3NF as we did in 2NF. Start with the relation that has the violation, and the FD that causes the violation to occur.
            \begin{align*}
                &\text{\textbf{Employee} (EmpID, Supv, Dept)} \\
                &\text{Supv $\to$ Dept}
            .\end{align*}
            \bigbreak \noindent 
            The attributes on the RHS of the FD are removed from the violating relation and placed into a newly created
            relation that has the FD's LHS as its primary key. A foreign key links the attribute from the LHS in the original table
            (the LHS is not removed) to the corresponding tuple in the new table, where it is the primary key.
            \begin{align*}
                &\text{\textbf{Employee}(EmpID, Supv)} \\
                &\text{\textbf{SupvDept}(Supv, Dept)}
            .\end{align*}
            \bigbreak \noindent 
            The RHS (Dept) that was a violation when it was in Employee because the LHS (Supv) was non-prime is no longer
            there to cause the problem. It is in the new relation where the LHS (Supv) is the primary key, and therefore we
            don't have a transitive dependency. These two relations no longer have the 3NF violation.
        \item \textbf{Summary of the normalization forms}:
            \bigbreak \noindent 
            \textbf{First Normal Form (1NF):}
            \begin{itemize}
                \item No repeating groups. All values are atomic.
                \item A primary key must have been chosen, and this primary key must be a proper superkey - it needs to be able to functionally determine every attribute in the relation.
            \end{itemize}
            1NF violations are fixed by choosing an appropriate primary key
            \bigbreak \noindent 
            \textbf{Second Normal Form (2NF) - To be in Second Normal Form, a relation must conform to 1NF and:}
            \begin{itemize}
                \item All of the non-prime attributes must be fully dependent upon the entire primary key.
                \item No non-prime attribute may be functionally determined by any subset of the primary key.
                \item No partial key dependencies
            \end{itemize}
            \bigbreak \noindent 
            2NF violations are fixed by decomposition.
            \bigbreak \noindent 
            \textbf{Third Normal Form (3NF) - To be in Third Normal Form, a relation must conform to 2NF and:}
            \begin{itemize}
                \item There may be no transitive dependencies.
                \item No non-prime attribute may functionally determine another non-prime attribute.
            \end{itemize}
            \bigbreak \noindent 
            3NF violations are fixed by decomposition.




    \end{itemize}

    \pagebreak 
    \subsection{ERD to Relations (Conceptual to logical)}
    \begin{itemize}
        \item \textbf{The basic outline (steps)}
            \begin{enumerate}
                \item Handle all of the entities
                \item Handle all of the relationships
            \end{enumerate}
        \item \textbf{Entity handling}: We will start with entities, because they can stand on their own, unlike relationships or attributes. In general, each entity will get its own relation. The attributes of the entity will become attributes in the schema of the relation created. There are some special cases to take into account, which will be handled from most independent to least, so:
            \begin{enumerate}[label=\alph*.]
                \item Strong (non-weak) entities that are not subtypes
                \item Strong.(non-weak) entities that are subtypes
                \item Weak entities
            \end{enumerate}
        \item \textbf{Entities like date}: there is no reason to make a relation for a "Date" entity or similar. The single value for the date is enough to determine it, and any other data associated with it is generally happening through a relationship anyway. Think about what data would go into such a table and how little use there would be for storing it separately.
        \item \textbf{Handling strong, non subtype entities}: Make a new relation, whose name will be the same as the name of the entity  
            \bigbreak \noindent 
            The primary key of the relation will be all of the identifier attributes, taken together  All attributes of the entity become attributes of the relation  Every instance of the entity gets the relevant values put into a new tuple in the relation
            \bigbreak \noindent 
            \textbf{Example:} Suppose we had an entity $A$ with attributes $\underline{ID}$, and other:
            \bigbreak \noindent 
            Then, we would make a relation $A(\underline{ID}, other)$
        \item \textbf{Handling strong, subtype entities}: Suppose
            \bigbreak \noindent 
            \fig{.5}{./figures/42.png}
            \bigbreak \noindent 
            \textbf{Employee} is a supertype (not subtype) so it gets handled in the previous step
            \begin{center}
                \textbf{Employee}(\underline{EmpId}, name)
            \end{center}
            \bigbreak \noindent 
            \textbf{Hourly} and \textbf{Salaried} are each strong, but they are subtypes (each is a type of Employee), so they are handled here
            \bigbreak \noindent 
            This type of inheritance means that the subtypes are types of the supertype, so they are identified by \textbf{Employee's} EmpID
            \bigbreak \noindent 
            There are two methods of handling these.
            \begin{enumerate}
                \item \textbf{Big table}: The first method involves putting the attributes of the subtypes into the relation made for the supertype. So, the original relation:
                    \begin{center}
                        \textbf{Employee}(\underline{EmpID}, Name)
                    \end{center}
                    Would become something like:
                    \begin{center}
                        \textbf{Employee}(\underline{EmpID}, Name,Wage, Salary)
                    \end{center}
                    but it would need to be modified to indicate which subtypes a given employee belongs to. Let's examine that on the next page.
                    \bigbreak \noindent 
                    The big table method needs a way to know which of the subtypes the current instance of the supertype belongs to, which is handled differently depending on the IS-A's configuration.
                    \bigbreak \noindent 
                    For \textbf{disjoint subtypes}, where an instance of the supertype can only be one of the subtypes at a timei, we can add an attribute, EmpType that has a value indicating which type this employee is.:
                    \begin{center}
                        \textbf{Employee}(\underline{EmpID}, Name, EmpType,Wage, Salary)
                    \end{center}
                    \bigbreak \noindent 
                    For generalization, EmpType would not allow NULL. For specialization, it would be allowed.
                    \bigbreak \noindent 
                    For \textbf{overlapping subtypes}, it is possible to be more than one at a time, so we need an individual true/false answer for each type:
                    \bigbreak \noindent 
                    \begin{center}
                        \textbf{Employee}(\underline{EmpID}, Name, IsHourly,Wage, IsSalaried, Salary)
                    \end{center}
                    \bigbreak \noindent 
                    In this case, nothing about the schema would indicate generalization vs. specialization
                \item \textbf{New relation}: Method 2 involves creating a new relation for the subtype entity
                    The name of new relation would be the same as the name of the entity.
                    \bigbreak \noindent 
                    The primary key of the new relation would be the same as the primary key for the supertype's relation.
                    \bigbreak \noindent 
                    The primary key is also a foreign key to the existing table.
                    \bigbreak \noindent 
                    An instance of the supertype entity will only have a tuple in the subtype relation if it is a member of that subtype, so we will not need any extra attributes like we did in method 1.
                    \bigbreak \noindent 
                    The foreign key can be used to look up any of the attributes that are being inherited from the supertype
                    \bigbreak \noindent 
                    Thus, we would have 
                    \begin{align*}
                        &\text{\textbf{Employee}(\underline{EmpID}, Name} \\
                        &\text{\textbf{Hourly}(\underline{EmpID\dag},Wage)} \\
                        &\text{\textbf{Salaried}(\underline{EmpID\dag}, Salary)}
                    .\end{align*}
                    \bigbreak \noindent 
                    \textbf{Note:} The (\dag) (dagger symbol) will be used in these slides to indicate that the attribute is part of a foreign key (and, in this example, the whole thing).
            \end{enumerate}
        \item \textbf{Handling weak entities}:  Suppose
            \bigbreak \noindent 
            \fig{.5}{./figures/43.png}
            \bigbreak \noindent 
            The strong entity would already have a relation. 
            \begin{center}
                \textbf{Strong}(\underline{id}, x)
            \end{center}
            \bigbreak \noindent 
            The weak entity gets its own relation. The primary key will be the concatenation of the weak entity's discriminator with the strong entity's identifier. The other attributes of the entity are brought in as non-prime attributes.
            \begin{center}
                \textbf{Weak}(\underline{id}\dag, \underline{disc}, y)
            \end{center}
            \bigbreak \noindent 
            The \underline{id} portion is a foreign key to the Strong relation
        \item \textbf{Entities: Functional Dependencies}: The only functional dependencies introduced by the entities of an ER diagram are the ones introduced when the identifiers become primary keys. Remember that a primary key has to functionally determine all of the other attributes in a relation
        \item \textbf{Handle relationships}: The relationships will be handled in order from lowest degree to highest degree, and within that, from simplest cardinality (one-to-one) to more complicated cardinalities (many-to-many, etc.).
            \bigbreak \noindent 
            The purpose of a relationship is to form connections between entities. We know that we are using relations to represent our entities, so we will need to use a tool that can link those relations to each other.
            \bigbreak \noindent 
            The tool best suited to linking tuples from relations together is the foreign key.
            \bigbreak \noindent 
            Every relationship we model in the relational model will have one or more foreign key involved. Where we put these foreign keys will depend on the cardinality, and the decisions are motivated by the normal forms we discussed.
            \begin{enumerate}
                \item \textbf{Binary one-to-one Relationships}: In a binary relationship, we will already have made a relation for each of the entities involved.
                    \bigbreak \noindent 
                    \fig{.5}{./figures/44.png}
                    \begin{center}
                        \textbf{A}(\underline{a},p) $\quad$ \textit{and} $\quad$ \textbf{B}(\underline{b}, q)
                    \end{center}
                    \bigbreak \noindent 
                    Since each instance of B will have one of A, and each instance of A will have one of B through C, we can represent this one-to-one relationship by putting a new foreign key into the entity for either side. Choose either:
                    \begin{center}
                        \textbf{A}(\underline{a},p,b\dag) $\quad$ \textit{or} $\quad$ \textbf{B}(\underline{b}, q, a\dag)
                    \end{center}
                    \bigbreak \noindent 
                    The relationship implies the functional dependencies:
                    \begin{align*}
                        a \to b \\
                        b \to a
                    .\end{align*}
                \item \textbf{Binary one-to-many Relationships}: In a binary relationship, we will already have made a relation for each of the entities involved.
                    \begin{center}
                        \textbf{A}(\underline{a},p) $\quad$ \textit{and} $\quad$ \textbf{B}(\underline{b}, q)
                    \end{center}
                    \bigbreak \noindent 
                    For this one-to-many relationship, there can be many instances of B for each of A, so we can't have the foreign key in the A table (wouldn't be atomic, so 1NF would be violated). We still do have the option of putting a foreign key in the B table pointing to the corresponding A, so our only option is: \begin{center}
                        \textbf{B}(\underline{b},q,a\dag)
                    \end{center}
                    \bigbreak \noindent 
                    The only FD is 
                    \begin{align*}
                        b \to a 
                    .\end{align*}
                    \bigbreak \noindent 
                \item \textbf{Binary many-to-many Relationships}: In a binary relationship, we will already have made a relation for each of the entities involved.
                    \begin{center}
                        \textbf{A}(\underline{a},p) $\quad$ \textit{and} $\quad$ \textbf{B}(\underline{b}, q)
                    \end{center}
                    \bigbreak \noindent 
                    There are no new functional dependencies introduced by the relationship, and putting a foreign key into either relation would not be atomic (1NF violation). The many-to-many relationship requires a new relation. Its foreign key will be the concatenation of the primary keys of each of the entity relations, which will be used as foreign keys to the corresponding tables. Any intersection data is put into this new relation as a non-prime attribute.
                    \bigbreak \noindent 
                    \begin{center}
                        \textbf{C}(\underline{a}\dag, \underline{b}\dag, x)
                    \end{center}
                \item \textbf{Relationships Greater than Binary: one-to-one-to-one}:
                    \bigbreak \noindent 
                    \fig{.7}{./figures/45.png}
                    \bigbreak \noindent 
                    So we have 
                    \begin{center}
                        \textbf{A}(\underline{a}, p) and \textbf{B}(\underline{b}, q) and \textbf{C}(\underline{c},r
                    \end{center}
                    \bigbreak \noindent 
                    Each of the "one legs" represents a functional dependency, and each of them gives us a potential relation to choose from for our relation.
                    \bigbreak \noindent 
                    \begin{table}[h!]
                        \centering
                        \begin{tabular}{ll}
                            \toprule
                            \textbf{Functional Dependency} & \textbf{Potential Relation for \textbf{D}} \\ 
                            \midrule
                            $a, b \rightarrow c$ & $\mathbf{D} \left( a^\dagger, b^\dagger, c^\dagger, x \right)$ \\[8pt]
                            $b, c \rightarrow a$ & $\mathbf{D} \left( a^\dagger, b, c^\dagger, x \right)$ \\[8pt]
                            $a, c \rightarrow b$ & $\mathbf{D} \left( a^\dagger, b, c^\dagger, x \right)$ \\ 
                            \bottomrule
                        \end{tabular}
                    \end{table}
                    \textbf{Note:} If we have say only two ones, like a one to one to many relationship, we would just have less functional dependencies and therefore less options to choose from (see table above)
                \item \textbf{Greater than Binary without any "ones"}:
                    No functional dependencies are implied by this relationship. To stay in 3NF, the relation we must use is:
                    \begin{center}
                        \textbf{D}(\underline{a}\dag, \underline{b}\dag, \underline{c}\dag, x)
                    \end{center}
                \item \textbf{Date entities (and similar)}: For relationships that have a "Date" entity (or the equivalent), recall that we did not make a relation for that entity. The only change necessary for your relationship involving that entity is that the date value is used instead of a foreign key, and that attribute will not be a foreign key, because the home relation would not exist
                    \bigbreak \noindent 
                    As an example, if the C entity in the ternary relationship with no "ones" ER diagram were a Date entity, we would not create the C relation for it, and the relation to represent the relationship would be modified. Notice that the $c$ attribute is still part of the primary key, but no longer a foreign key.
                    \begin{align*}
                        &\text{From: \textbf{D}(\underline{a}\dag, \underline{b}\dag, \underline{c}\dag, x)} \\
                        &\text{To: \textbf{D}(\underline{a}\dag, \underline{b}\dag, \underline{c}, x)} \\
                   .\end{align*}
                \item \textbf{Recursive Relationships: one-to-many}: Recursive relationships will be handled as if they were normal relationships of the same degree and cardinality. The practical difference is that the entity that is linked multiple times will still only have one relation, so multiple foreign keys might go to the same table.
                    \bigbreak \noindent 
                    Suppose: 
                    \bigbreak \noindent 
                    \fig{.7}{./figures/46.png}
                    \bigbreak \noindent 
                    There should obviously only be one relation for the entity Department, because it is only a single entity.
                    \begin{center}
                        \textbf{Department}(\underline{DeptNo}, Name)
                    \end{center}
                    \bigbreak \noindent 
                    With a non-recursive one-to-many binary relationship, we would have put a foreign key to the relation for the one
                    side into the relation on the one side. In this version, we only have one table, so the decision is easy. We will need
                    to come up with another name for the foreign key, as we cannot have two attributes with the same name inside the
                    same relation. Thus, we grow Department into the following:
                    \begin{center}
                        \textbf{Department}(\underline{DeptNo}, Name, ReportsToDept\dag)
                    \end{center}
                    \bigbreak \noindent 
                    Where the home relation for the new attribute, ReportsToDept, is that same relation, Department. The tuple of
                    the department that the current department reports to will be have a DeptNo that equals the ReportsToDept in
                    the current tuple. Alternatively, ReportsToDept can be NULL if the department does not report to another.
                \item \textbf{Recursive Relationships, many-to-many}: Suppose 
                    \bigbreak \noindent 
                    \fig{.87}{./figures/47.png}
                    \bigbreak \noindent 
                    Like non-recursive many-to-many relationships, we will need to create a new relation. Unlike the non-recursive
                    version, we only have one home relation for our two foreign keys. As in the one-to-many version, we will need to
                    choose a new name for at least one copy of the foreign key, since they can't share the same name. The relation for
                    our Person entity would be \textbf{Person}(\underline{ID}, Name)
                    \bigbreak \noindent 
                    The new relation created to represent the relationship would be in the following form:
                    \begin{center}
                        \textbf{Friends}(\underline{activeFriend}\dag, \underline{passiveFriend}\dag)
                    \end{center}
                    \bigbreak \noindent 
                    ActiveFriend and PassiveFriend are foreign keys to the tuple in Person with data for the person that is taking part in
                    the relationship. This can be done in a directed or undirected way, and you probably want to put a comment
                    somewhere about which way you intend to use it.
                    \bigbreak \noindent 
                    directed: (Person1, Person2) would not imply (Person2, Person1)
                    \bigbreak \noindent 
                    undirected: (Person1, Person2) does imply (Person2, Person1)
                    \bigbreak \noindent 
                    \fig{.6}{./figures/48.png}
            \end{enumerate}
        \item \textbf{Summary}:
            \begin{enumerate}
                \item Strong, non-subtype entities
                    \begin{itemize}
                        \item New relation, PK is entities identifiers
                    \end{itemize}
                \item Sub-type entities
                    \begin{itemize}
                        \item New relations, PK is supertype identifiers, which are foreign keys to supertype relation
                    \end{itemize}
                \item Weak entities
                    \begin{itemize}
                        \item New relation, PK is concatenation of strong identifier and discriminator. Strong Id from the concat is FK to strong relation.
                    \end{itemize}
                \item \textbf{Relationships: Binary 1-1}
                    \begin{itemize}
                        \item Put foreign key in either side
                    \end{itemize}
                \item \textbf{Binary 1-m}
                    \begin{itemize}
                        \item Put foreign key to one side in the many side                     
                    \end{itemize}
                \item \textbf{Binary m-m}
                    \begin{itemize}
                        \item New relation, PK is concatenation of both entities keys, which also serves as foreign keys to entities. 
                    \end{itemize}
                \item \textbf{n-ary 1-1-...-1 (all ones)}
                    \begin{itemize}
                        \item New relation, choose n-1 entities for PK, put remaining entity ID as non-prime, but foreign.
                    \end{itemize}
                \item \textbf{n-ary 1-1-...-m (Two ones)}
                    \begin{itemize}
                        \item New relation, choose all many legs and one of the one legs for PK, put remaining one leg as non-prime but foreign
                    \end{itemize}
                \item \textbf{n-ary 1-m-...--m (Single one)}
                    \begin{itemize}
                        \item New relation, choose all many legs for PK, put remaining one leg as non-prime but foreign 
                    \end{itemize}
                \item \textbf{n-ary m-m-...-m (No ones)}
                    \begin{itemize}
                        \item New relation, all legs are PK
                    \end{itemize}
                \item \textbf{Handle date entities, and things of that nature}
                    \begin{itemize}
                        \item Since we do not create relations for these types of entities, we cannot make them foreign keys, because the home relation will not exist. They can still be part of the PK.
                    \end{itemize}
                \item \textbf{Recursive relationships}
                    \begin{itemize}
                        \item We handle these the same, but the foregin key will link to the same relation. Make sure to put a comment somewhere to specify directed or undirected.
                    \end{itemize}
            \end{enumerate}

    \end{itemize}

    \pagebreak 
    \subsection{MariaDB, SQL}
    \begin{itemize}
        \item \textbf{DDL}:
            \begin{itemize}
                \item CREATE TABLE
                \item ALTER TABLE
                \item DROP TABLE
            \end{itemize}
        \item \textbf{DML}:
            \begin{itemize}
                \item INSERT 
                \item UPDATE
                \item DELETE
            \end{itemize}
        \item \textbf{MariaDB navigation}:
            \begin{itemize}
                \item \textbf{USE <x>}: select the database <x>
                \item \textbf{SHOW TABLES} list all of the tables in the current database
                \item \textbf{DESCRIBE <x>} show the properties of each column of table <x>
                \item \textbf{SHOW CREATE TABLE} <x> show a CREATE TABLE statement that can be used to reconstruct table <x>
            \end{itemize}
        \end{itemize}
        \pagebreak \bigbreak \noindent 
        \subsubsection{DDL}
        \begin{itemize}
        \item \textbf{Creating a new table with CREATE TABLE}: The basic format of a CREATE TABLE statement. []'s and <>'s are not to be typed. [] indicates that the contents are optional, and the <>'s indicate placeholders:
            \bigbreak \noindent 
            \begin{sqlcode}
                CREATE TABLE <table_name> (
                    <attribute> <type> [NOT NULL] [UNIQUE] [PRIMARY KEY], [ ... ]
                    [PRIMARY KEY(<pkattrs>),]
                    [FOREIGN KEY(<attr_here>) REFERENCES <home_table>(<attr_home>)]
                );
            \end{sqlcode}
            \begin{itemize}
                \item <table\_name> name of the table
                \item <attribute> name of the current attribute
                \item <type> data type of the current attribute
                \item <pkattrs> comma-separated list of the attributes makeing up the table's primary key
                \item <attr\_here> comma-separated list of attributes in the current table forming a foreign key
                \item <home\_table> name of the home table
                \item <attr\_home> comma-separated list of attributes in the home table, matching the attributes in <attr\_here>
            \end{itemize}
        \item \textbf{Table / Column names}: When choosing a name for a table or a column, we can use the following characters:
            \begin{itemize}
                \item any of the normal upper or lower case letters (regexp: [A-Za-z]
                \item an underscore - \_
                \item a dollar sign - \$
                \item digits, but only after the first character
            \end{itemize}
            \bigbreak \noindent 
            The following limits are in place:
            \begin{itemize}
                \item Table names must be unique within the database. They share the same namespace with views.
                \item Attribute/column names must be unique with each table.
                \item Unless quoted properly with backticks, reserved keywords cannot be used as identifier
            \end{itemize}
            \bigbreak \noindent 
            \textbf{Note:} These identifiers may or may not be case sensitive, depending on the locale setting of the server.
            \bigbreak \noindent 
            Generally, the maximum length of an identifier is 64 characters.
        \item \textbf{Data types}
            \begin{itemize}
                \item \textbf{INT/INTEGER}: integer values
                \item \textbf{FLOAT}: single precision floating point numbers
                \item \textbf{DOUBLE/REAL}: double precision floating point numbers
                \item \textbf{DECIMAL(i,j)}: decimal numbers, i digits total, j after the decimal point .
                \item \textbf{CHAR(n)}: character string exactly n characters long
                \item \textbf{VARCHAR(n)}: variable-length character string up to n characters long
                \item \textbf{DATE}: date in 'YYYY-MM-DD' format
                \item \textbf{TIME}: time in 'HH:MM:SS' format
                \item \textbf{DATETIME}: date/time in 'YYYY-MM-DD HH:MM:SS' format, no timezone conversion
                \item \textbf{TIMESTAMP}: date/time in 'YYYY-MM-DD HH:MM:SS' format, timezone conversion
            \end{itemize}
        \item \textbf{Column Options}: Here are some common options that can be applied to a column/attribute. They are written right after the type when defining a new column in a CREATE TABLE statement
            \begin{itemize}
                \item \textbf{NULL}: allows NULL to be stored as the value for this attribute (default)
                \item \textbf{NOT NULL}: prevents NULL from being stored as the value for this attribute
                \item \textbf{UNIQUE}: ensures that no two tuples have the same value for this attribute
                \item \textbf{PRIMARY KEY}: declares this attribute to be the entire primary key
                \item \textbf{AUTO\_INCREMENT}: next-available value auto-assigned for this attribute when not provided
                \item \textbf{DEFAULT <x>}: sets the default value of the attribute to <x> when not supplied
            \end{itemize}
        \item \textbf{Setting the Primary Key}: There are two ways to set the primary key:
            \begin{enumerate}
                \item For single-attribute primary keys, you can use the PRIMARY KEY column option. The option may only be used once, and proclaims that the single attribute is the entirety of the primary key.
                \item If you have multiple attributes in the primary key, the only way is to add the separate constraint:
                    \begin{center}
                        \texttt{PRIMARY KEY(<x>,<y>,<z>,<etc>)}
                    \end{center}
                    \bigbreak \noindent 
                    This can also be used for single attribute primary keys.
                    \bigbreak \noindent 
                    \textbf{Note:} It should be obvious that only one primary key can be set.
            \end{enumerate}
        \item \textbf{Comments}: MariaDB supports the following comment syntax
            \begin{enumerate}
                \item \textbf{Pound (\#)}:
                \item \textbf{Double hypen (\texttt{--})}: This is the standard style
                \item \textbf{C-style multiline comments (\textbackslash * ... *\textbackslash)}
            \end{enumerate}
        \item \textbf{Quotes}: There are two types of quotes that you may encounter in SQL.
            \begin{enumerate}
                \item \textbf{Quotes for values - single quotes 'value'}: not necessary for numeric values, but can be used without breaking them, always required for string values. If it is ambiguous whether something is a value or an identifier, use these quotes
                \item \textbf{Quotes for identifiers - backticks `identifier}: not necessary for identifiers that follow the rules from above, but can be used anyway, can allow identifier names to contain characters not otherwise allowed.  Can allow identifiers to use names that would normally be reserved keywords
            \end{enumerate}
            \bigbreak \noindent 
            \textbf{Note:} Notice that identifier in the SQL context is a different thing than an identifier in an ER diagram. Here, identifier will mean the name of some table, column, variable, etc.
        \item \textbf{An example of CREATE TABLE}: Let's go ahead and make the SQL CREATE TABLE statement to create a table for the relation:
            \begin{center}
                \textbf{Person}(\underline{SSN}, FNAME, LNAME, PHONE)
            \end{center}
            \bigbreak \noindent 
            \begin{sqlcode}
                CREATE TABLE Person(
                    SSN CHAR(9) PRIMARY KEY, # SSN BAD IDEA, PK on same line (1)
                    FNAME CHAR(20) NOT NULL, # First name
                    LNAME CHAR(20) NOT NULL, # Last name
                    PHONE CHAR(10) # Phone number
                ); 
            \end{sqlcode}
            \bigbreak \noindent 
            The relational schema we started with does not have information on data types or column options other than PRIMARY KEY, so we choose them while creating the table.
        \item \textbf{Setting up a foreign key}: A foreign key links the current table to another table, which we call the home relation.
            \begin{enumerate}
                \item The foreign key must contain all of the attributes of the primary key of the home relation.
                \item They may have different names in each of the tables, but there needs to be a match for each.
                \item Each of these attributes must have the exact same data type as its counterpart in the home table.
            \end{enumerate}
            \bigbreak \noindent 
            If a table is to contain a foreign key, we include a constraint in our CREATE TABLE statement like the following:
            \begin{sqlcode}
                FOREIGN KEY (<localnames>) REFERENCES <home_table>(<homenames>)
            \end{sqlcode}
            This can be done for multiple foreign keys, filling in the placeholders <localnames>, <home\_table>, and <homenames> appropriately for each.
    \item \textbf{Table with foreign key example}: Let's make a table for a subtype of Person, Student:
        \begin{center}
            \textbf{Student}(\underline{SSN}\dag, CLSYEAR, GPA, TOTALHRS)
        \end{center}
        \bigbreak \noindent 
        \begin{sqlcode}
            CREATE TABLE Student (
                SSN CHAR(9) NOT NULL, -- SSN is BAD IDEA
                CLSYEAR CHAR(9), -- fresh/soph/junior/senior
                GPA DECIMAL(4.3), -- 4.000, we hope
                TOTALHRS INT,

                PRIMARY KEY (SSN), -- set up the primary key separately (2)
                FOREIGN KEY (SSN) REFERENCES Person(SSN) -- a Student is a Person
            );
        \end{sqlcode}
        \bigbreak \noindent 
        \textbf{Note:} We need to use SHOW CREATE TABLE to show the get information of the foreign keys of a table.
    \item \textbf{Change existing table schema: ALTER TABLE}: An ALTER TABLE statement will allow you to have the DBMS make changes to the schema of a table that has already been created. It works with various subcommands. The three we will cover are:
        \begin{enumerate}
            \item ALTER TABLE ADD
            \item ALTER TABLE MODIFY
            \item ALTER TABLE DROP
        \end{enumerate}
    \item \textbf{ALTER TABLE ADD}: The ALTER TABLE ADD command can be used to add a new column or new columns to the schema of an existing table.
        \bigbreak \noindent 
        To add a single column/attribute
        \bigbreak \noindent 
        \begin{sqlcode}
            ALTER TABLE <table_name> ADD <attribute> <type>; 
        \end{sqlcode}
        \bigbreak \noindent 
        To add multiple columns/attributes:
        \bigbreak \noindent 
        \begin{sqlcode}
            ALTER TABLE <table_name> ADD (<attribute> <type>, ...);
        \end{sqlcode}
    \item \textbf{ALTER TABLE MODIFY}: The ALTER TABLE MODIFY command can be used to change properties of a column/attribute (including type, length, and other column options) in a table that already exists.
        \bigbreak \noindent 
        \begin{sqlcode}
            ALTER TABLE <table_name> MODIFY <col_name> <new_options>;
        \end{sqlcode}
    \item \textbf{ALTER TABLE DROP}: The ALTER TABLE DROP command can be used to remove a column/attribute from the schema of a table.
        \bigbreak \noindent 
        \begin{sqlcode}
            ALTER TABLE <table_name> DROP <col_name>;
        \end{sqlcode}
    \item \textbf{SHOW TABLES}: In MariaDB/MySQL, if you want to see a list of the tables present in the current database, you can use the command:
        \bigbreak \noindent 
        \begin{sqlcode}
            SHOW TABLES;
        \end{sqlcode}
    \item \textbf{DROP TABLE}: To remove a table from the database, we can use the DROP TABLE command.
        \bigbreak \noindent 
        \begin{sqlcode}
        DROP TABLE <table_name>;
        \end{sqlcode}
    \item \textbf{Termination of commands (;)}: Notice in all sql code examples we have a semi colon after the command / line, this is needed to execute the command.








    \end{itemize}

    \pagebreak 
    \subsubsection{DML except SELECT}
    \begin{itemize}
        \item \textbf{DML Introduction}: The Data Manipulation Language (DML) is the language used to work with the instance data. In SQL, this means doing things with the rows contained by tables, rather than to the tables themselves. We have
            \begin{itemize}
                \item \textbf{INSERT}: Add a new row to a table
                \item \textbf{UPDATE}: Change values in an existing row
                \item \textbf{DELETE}: Remove rows from the table
                \item \textbf{SELECT}: Display the data stored in rows (In the next subsection)
            \end{itemize}
        \item \textbf{INSERT}:
            \begin{sqlcode}
                INSERT INTO <table_name>
                    VALUES (<value_list>);

                INSERT INTO <table_name>
                    (<attr_list>)
                    VALUES (<value_list>);

                INSERT INTO <table_name>
                    <another_query>;
            \end{sqlcode}
            \bigbreak \noindent 
            Where 
            \begin{itemize}
                    \item \textbf{<table\_name>}: The name of the table where the row should be added.
                    \item \textbf{<value\_list>}: A list of values for the new row. If no <attr\_list> is given, then the values are for each of the columns of the table, in order.
                    \item \textbf{<attr\_list>}: A list of names of attributes that match up with the values in <value\_list>. This allows us to omit optional columns or change the order.
                    \item \textbf{<another\_query>}: A query that returns rows, like a SELECT statement. The rows returned are inserted into the table.
            \end{itemize}
            \bigbreak \noindent 
            \textbf{Notes}: Without the attribute list, there must be a value in the VALUES() for every column, and they have to be in the same order as they had in the table.
            \bigbreak \noindent 
            Columns not in the attribute list are set to their default value if possible. This is why PHONE is NULL. This version of the INSERT statement is better if you're making SQL that needs to be in a script that is to be run later, as it tolerates more changes to the table schema than the other version.
        \item \textbf{The WHERE clause}:
            \bigbreak \noindent 
            \begin{sqlcode}
            ... WHERE <expression> ...
            \end{sqlcode}
            \bigbreak \noindent 
            When working with DML statements, it will be desirable to be able to work only with specific rows. This can be accomplished using a WHERE clause.
            \bigbreak \noindent 
            The WHERE clause is the keyword WHERE followed by an expression that evaluates to either true or false. It is included in an SQL query to control which rows are affected by the query
            \bigbreak \noindent 
            The expression after WHERE is evaluated one time per row. Rows where the expression evaluates as true are included in the operation. Rows where the expression evaluates to false are excluded from the operation
            \bigbreak \noindent 
            WHERE clauses are generally used in UPDATE, DELETE, and SELECT statements.
        \item \textbf{UPDATE}:
            \bigbreak \noindent 
            \begin{sqlcode}
                UPDATE <table_name>
                    SET <attr> = <value> [, <attr> = <value> ...]
                    [ WHERE <expression> ];
            \end{sqlcode}
            \bigbreak \noindent 
            Where
            \begin{itemize}
                \item \textbf{<attr>}: name of a column to change
                \item \textbf{<value>}: value to assign to <attr>
                \item \textbf{<expression>}: expression evaluated for each row to determine if the row is affected
            \end{itemize}
        \item \textbf{DELETE}: To delete the rows without getting rid of the table, use a DELETE statement.
            \bigbreak \noindent 
            \begin{sqlcode}
            DELETE FROM <table_name>
                [ WHERE <expression> ];
            \end{sqlcode}
            \bigbreak \noindent 
            It is important to realize that all rows are affected by default, so if a WHERE clause is not supplied, all of the rows will be deleted.
        \item \textbf{Views in SQL}: A view in SQL is a virtual table. It does not store its own data, but rather derives it from the other tables (or views) via a query that is a part of its definition.
            \bigbreak \noindent 
                Views do not contain their own data. They dynamically grab their data from the base tables on demand. Thus, changes to the data in the base tables will be reflected in the views that derive from them automatically
        \item \textbf{CREATE VIEW}:
            \bigbreak \noindent 
            \begin{sqlcode}
                CREATE VIEW <view_name>
                    [( <view_col_name> [, <view_col_name>]...)] # can rename columns here
                    AS SELECT <attr_name> [, <attr_name>] ...
                        FROM <source_table_or_view> [, ...]
                        WHERE <condition>;
                    \end{sqlcode}
                    \bigbreak \noindent 
                    The portion after the AS keyword is a SELECT statement, part of the DML that is used to ask the DBMS to show
                    portions of instance data (rows from tables). 
                    \bigbreak \noindent 
                    Once the view is created, it supports DML queries in most of the same ways a non-virtual table can be. Writing to a
                    view is sometimes possible, but depends on how the SELECT statement that constructed it was formulated. It is
                    generally a better idea to write directly to the base tables.
            \bigbreak \noindent 
            \textbf{Example:}
            \bigbreak \noindent 
            \begin{sqlcode}
                CREATE VIEW dekalb_people
                    (SSN, first_name, last_name) # control the names of the columns as seen in the view
                    AS SELECT SSN, FNAME, LNAME # control which columns are returned by SELECT
                        FROM Person # get rows from the Person table
                        WHERE ZIP = '60115'; # control which rows make it into the view
            \end{sqlcode}
        \item \textbf{DROP VIEW}: Although tables and views share the same namespace (so it is not possible to have a view and a table with the same name) and work the same in a lot of queries, DROP TABLE is one of the exceptions and will not work to delete a view. It will give you an error message
            \bigbreak \noindent 
            Instead, use DROP VIEW, which has generally the same syntax:
            \bigbreak \noindent 
            \begin{sqlcode}
            DROP VIEW <viewname>;
            \end{sqlcode}
        \item \textbf{Advantages of Views}:
            \begin{itemize}
                \item Base tables should always be designed in Third Normal Form or better. Views allow us to access them in possibly more convenient ways while still having the benefits of 3NF.
                \item Views can free users from complicated DML operations, such as joins.
                \item Users can be denied direct access to base tables, but given access to portions of them through the views. This enhances security
            \end{itemize}
    \end{itemize}


    \pagebreak 
    \subsubsection{DML SELECT}
    \begin{itemize}
        \item \textbf{SELECT Statement Format}: Two versions of the basic format of a SELECT statement follow.
            \bigbreak \noindent 
            \begin{sqlcode}
                SELECT [DISTINCT|ALL] <column_list> # most common, show row data
                    FROM <table_list>
                    [ WHERE <where_exp> ]
                    [ GROUP BY <group_key> ]
                    [ HAVING <having_exp> ]
                    [ ORDER BY <sortcols> ] ;

                SELECT <anyexpression> ; # show results of the supplied expression
            \end{sqlcode}
            Where
            \begin{itemize}
                \item \textbf{<column\_list>}: comma separated list of the columns to show in the results, * for all columns
                \item \textbf{<where\_exp>}: boolean expression evaluated once per row to determine whether the row is included
                \item \textbf{<group\_key>}: comma-separated list of the columns to use when grouping the rows
                \item \textbf{<having\_exp>}: boolean expression evaluated once per group to determine whether the group is included
                \item \textbf{<sortcols>}: comma-separated list of the columns to sort by (most important comes first)
                \item \textbf{<anyexpression>}: the expression whose results should be displayed
            \end{itemize}
        \item \textbf{Example data}: Here we have a simple database to use for the examples that follow. It tracks suppliers and the parts they supply.
            \bigbreak \noindent 
            \fig{.5}{./figures/49.png}
            \begin{align*}
                &\text{Supplier Info S(\underline{S}, SNAME, STATUS, CITY)} \\
                &\text{Part Info P(\underline{P}, PNAME, COLOR,WEIGHT)} \\
                &\text{Supplied Parts SP(\underline{S}$^{\dag}$, \underline{P}$^{\dag}$, QTY)}
            .\end{align*}
            \bigbreak \noindent 
            The S table contains the information on the suppliers themselves.
            \bigbreak \noindent 
            \begin{center}
                \begin{tabular}{c|c|c|c}
                    S &SNAME &STATUS &CITY \\
                    \hline
                    S1 &Smith &20 &London \\
                    S2 &Jones &10 &Paris \\
                    S3 &Blake &30 &Paris \\
                    S4 &Clark &20 &London \\
                    S5 &Adams &30 &Athens  
                \end{tabular}
            \end{center}
            \bigbreak \noindent 
            The P table contains information on parts.
            \bigbreak \noindent 
            \begin{center}
                \begin{tabular}{c|c|c|c}
                    P &PNAME &COLOR &WEIGHT \\
                    \hline
                    P1 &Nut &Red &12 \\
                    P2 &Bolt &Green &17 \\
                    P3 &Screw &Blue &17 \\
                    P4 &Screw &Red &14 \\
                    P5 &Cam &Blue &12 \\
                    P6 &Cog &Red &19 
                \end{tabular}
            \end{center}
            \bigbreak \noindent 
            The SP table contains information on which suppliers supply which parts, and how many.
            \bigbreak \noindent 
            \begin{center}
                \begin{tabular}{c|c|c}
                    S& P& QTY \\
                    \hline
                    S1 &P1 &300 \\
                    S1 &P2 &200 \\
                    S1 &P3 &400 \\
                    S1 &P4 &200 \\
                    S1 &P5 &100 \\
                    S1 &P6 &100 \\
                    S2 &P1 &300 \\
                    S2 &P2 &400 \\
                    S3 &P2 &200 \\
                    S4 &P2 &200 \\
                    S4 &P4 &300 \\
                    S4 &P5 &400
                \end{tabular}
            \end{center}
        \item \textbf{Example query}: 
            \bigbreak \noindent 
            Get supplier numbers and status for suppliers in Paris.
            \bigbreak \noindent 
            \begin{sqlcode}
            SELECT S,STATUS
                FROM S
                WHERE CITY = 'paris';
            \end{sqlcode}
            \bigbreak \noindent 
            Get part numbers for all parts supplied
            \bigbreak \noindent 
            \begin{sqlcode}
            SELECT P FROM SP;
            \end{sqlcode}
            \bigbreak \noindent 
            Adding the DISTINCT keyword can prevent duplicate output rows from being shown.
            \bigbreak \noindent 
            \begin{sqlcode}
            SELECT DISTINC P
                FROM SP;
            \end{sqlcode}
            \bigbreak \noindent 
            List the full details of all suppliers.
            \bigbreak \noindent 
            \begin{sqlcode}
            SELECT * FROM S;
            \end{sqlcode}
            \bigbreak \noindent 
            List supplier numbers for all suppliers in Paris with a STATUS greater than 20.
            \bigbreak \noindent 
            \begin{sqlcode}
            SELECT * FROM S 
                WHERE CITY = 'Paris' AND 
                    STATUS > 20;
            \end{sqlcode}
        \item \textbf{Relational Operators in SQL}:
            \begin{itemize}
                \item $=$ is equal to
                \item $<$ less than
                \item $<=$ less than or equal to
                \item $>$ greater than
                \item $>=$ greater than or equal to
                \item $<>$ or $!=$ not equal to
            \end{itemize}
        \item \textbf{Compound Logical Operators}:
            \begin{itemize}
                \item \textbf{AND}
                \item \textbf{OR}
                \item \textbf{NOT}
            \end{itemize}
        \item \textbf{The ORDER BY clause}: Adding the ORDER BY clause allows us to enforce a sorting order upon our results.
            \bigbreak \noindent 
            \begin{sqlcode}
            ORDER BY <attrs>
            \end{sqlcode}
            \bigbreak \noindent 
            Where <attrs> is a comma-separated list of the attributes to base our sorting upon.
            \bigbreak \noindent 
            After each attribute, you have the option to add either DESC (for descending) or ASC (for ascending) to affect the sort direction for each attribute. The default sort direction is ascending, if not specified.
            \bigbreak \noindent 
            The first attribute listed is the most important, and any subsequent attributes is only sorted upon if there are multiple rows in which the values for the previous attributes before them were all the same.
    \item \textbf{ORDER BY example}: List the supplier numbers and status for suppliers in Paris in descending order of status.
        \bigbreak \noindent 
        \begin{sqlcode}
            SELECT S,STATUS FROM S
                WHERE CITY = 'Paris'
                ORDER BY STATUS DESC;
        \end{sqlcode}
    \item \textbf{Cartesian Product in SQL}: For two sets $A = \{a,b,c\}$, and $B=\{d,e,f\} $
        \begin{align*}
            A \times B = \{(a, d), (a, e), (a, f), (b , d), (b , e), (b ,f), (c, d), (c, e), (c, f)\}
        .\end{align*}
        \bigbreak \noindent 
        This is relevant because the Cartesian Product is used in SQL when we SELECT from multiple tables. When this happens, the sets (like $A$ and $B$) to be combined are the tables, and the items inside of them are the tuples/rows they contain.
        \bigbreak \noindent 
        When the Cartesian Product is done on two tables,
        \begin{itemize}
            \item The width of the result is the sum of the widths (in columns) of both of the tables.
            \item The length (in rows) of the result will be the product of the lengths of both of the tables.
        \end{itemize}
        \bigbreak \noindent 
        \textbf{Note:} The Cartesian Product is an associative operation
    \item \textbf{Aliases and the dot operator}: When we select certain relations, we can give aliases to them and then reference their attributes with a dot (similar to how you access C++ member functions)
        \bigbreak \noindent 
        This is important because if we take the cartesian product of the same relations, we need to give them aliases (they would otherwise have the same name)
    \item \textbf{Aliases with AS}: We can also use AS to assign aliases
        \bigbreak \noindent 
        \begin{sqlcode}
        SELECT col AS c1
            FROM relation1 AS r1
            ...
        \end{sqlcode}
    \item \textbf{Using the dot in general}: In general, even without aliases, we can use the dot to refer to relations attributes, this is crucial if we select on two relations with matching attribute names.
    \item \textbf{Cartesian Product Example}: S has 5 rows of 4 columns. The Cartesian product, SELECT * FROM S T1, S T2; returns 25 rows, each with 2 sets of the columns in S, for a total of 8 columns. They don't all fit on the page here.
        \bigbreak \noindent 
        \begin{center}
            \begin{tabular}{c|c|c|c|c|c|c|c}
                T1.S &T1.SNAME &T1.STATUS &T1.CITY &T2.S &T2.SNAME &T2.STATUS &T2.CITY \\
                \hline
                S1 &Smith &20 &London &S1 &Smith& 20 &London \\
                S2 &Jones &10 &Paris &S1 &Smith &20 &London \\
                S3 &Blake &30 &Paris &S1 &Smith &20 &London \\
                S4 &Clark &20 &London &S1 &Smith &20& London \\
                S5 &Adams &30 &Athens &S1 &Smith &20& London \\
                S1 &Smith &20 &London &S2 &Jones &10& Paris \\
                S2 &Jones &10 &Paris &S2 &Jones &10 &Paris \\
                S3 &Blake &30 &Paris &S2 &Jones &10 &Paris \\
                S4 &Clark &20 &London &S2& Jones &10& Paris \\
                S5 &Adams &30 &Athens &S2& Jones &10& Paris \\
                S1 &Smith &20 &London &S3& Blake &30& Paris \\
                S2 &Jones &10 &Paris &S3 &Blake &30 &Paris \\
                S3 &Blake &30 &Paris &S3 &Blake &30 &Paris \\
                S4 &Clark &20 &London &S3& Blake &30 &Paris \\
                S5 &Adams &30 &Athens &S3& Blake &30 &Paris 
            \end{tabular}
        \end{center}
    \item \textbf{Cartesian product example}: For each part supplied, get the part number and names of all the cities supplying the part. (This is a join which pulls together data from multiple tables.)
        \bigbreak \noindent 
        \begin{sqlcode}
        SELECT DISTINC P, CITY
            FROM SP,S
            WHERE SP.S = S.S
        \end{sqlcode}
        \bigbreak \noindent 
        List the supplier numbers for all pairs of suppliers such that two suppliers are located in the same city.
        \bigbreak \noindent 
        \begin{sqlcode}
            SELECT T1.S, T2.S /* one S from each side */
                FROM S T1, S T2 /* cartesian product of S with S, giving name to each side */
                WHERE T1.CITY = T2.CITY /* same city for both suppliers */
                    AND T1.S < T2.S; /* avoid duplicate pairs; lower S on left */
        \end{sqlcode}
    \item \textbf{Multiple-row subqueries}: Multiple-row subqueries are nested queries that have the potential to return more than one row of results to
        the parent query. Most commonly used in WHERE and HAVING clauses
        \bigbreak \noindent 
        \textbf{Note:} Must be used  with multiple-row operators.
    \item \textbf{SQL Sets}: In an SQL statement, we can denote a set with a list of values inside parentheses.
    \item \textbf{Multiple Row Subqueries: IN Set Operator}: IN is a set operator used to test membership.
        \bigbreak \noindent 
        The IN operator will have value on its left, and a set on its right. It will evaluate to true if the value from the left hand side is present in the set provided on the right.
        \bigbreak \noindent 
        \begin{center}
            \begin{center}
                \begin{tabular}{c|c}
                    Example & Evaluates to \\
                    \hline
                    'S1' IN ('S2','S3','S1') & true \\
                    'S1' IN ('S2','S3','S4') &false \\
                    4 IN (2,1,6,4,5) &true \\
                    3 IN (1,5,6,10) &false
                \end{tabular}
            \end{center}
        \end{center}
        \bigbreak \noindent 
        When a multiple-row subquery is evaluated, its results are inserted into its parent query as a set. We can use set operations like IN to fit those results into our query
    \item \textbf{Multiple-row subqueries: Example}: List the supplier names for suppliers who supply part P2. (This time using a subquery.)
        \bigbreak \noindent 
        \begin{sqlcode}
            SELECT SNAME
                FROM S
                WHERE S IN              # IN operator used to check current S against the list
                    ( SELECT S          # this is the subquery
                    FROM SP             # which returns a list (set)
                    WHERE P = 'P2' );   # containing all the suppliers that supply part P2.
        \end{sqlcode}
        \bigbreak \noindent 
        \textbf{Note:} The innermost subqueries are the first to run. It returns
        \bigbreak \noindent 
        \begin{center}
            \begin{tabular}{c}
                S \\
                \hline
                S1 \\
                S2 \\
                S3 \\
                S4
            \end{tabular}
            \end{center}
            \bigbreak \noindent 
            Which is inserted into the parent query as ('S1', 'S2', 'S3', 'S4'), in the position where the subquery that returned the results was found.
            \bigbreak \noindent 
            Thus, after the subquery is run, the outer query effectively becomes:
            \bigbreak \noindent 
            \begin{sqlcode}
                SELECT SNAME
                    FROM S
                    WHERE S IN                  # IN operator used to check current S against the list
                        ('S1', 'S2', 'S3', 'S4');   # <-- results of subquery inserted in place
            \end{sqlcode}
            \bigbreak \noindent 
            Which would have the following results:
            \bigbreak \noindent 
            \begin{center}
                \begin{tabular}{c|c}
                    SNAME \\
                    \hline
                    Smith \\
                    Jones \\
                    Blake \\
                    Clark  \\
                \end{tabular}
            \end{center}
        \item \textbf{Set Operators: ALL and ANY}: The ALL and ANY operators modify the normal relational (in the comparison sense) operators to work on sets.
            \bigbreak \noindent 
            If we want to compare a value with every item in the set and reduce the answers to a single true/false using the AND operation, we can use ALL
            \bigbreak \noindent 
            \begin{sqlcode}
            <value> <relop> ALL (set)
            \end{sqlcode}
            \bigbreak \noindent 
            If we want to compare a value with every item in the set and reduce the answers to a single true/false using the OR operation, we can use ANY.
            \bigbreak \noindent 
            \begin{sqlcode}
            <value> <relop> ANY (set)
            \end{sqlcode}
        \item \textbf{Set Operator: EXISTS}: The EXISTS operator is a unary operator working on sets that is used to determine whether the set supplied is non-empty. Once again <set> is either an explicitly written set or a multi-row subquery
            \bigbreak \noindent 
            \begin{sqlcode}
            EXISTS (set)
            \end{sqlcode}
            \begin{itemize}
                \item Evaluates to true if the set is non-empty (contains at least one element)
                \item Evaluates to false if the set is empty (no elements inside)
            \end{itemize}
        \item \textbf{Set operator: NOT EXISTS}: EXISTS, when used in conjunction with the logical inversion operator, NOT, enables two types of queries that were difficult before
            \begin{itemize}
                \item Queries involving the set difference operation $\{a,b,c,d,e\} - \{b,c\} = \{a,d,e\}$
                \item Queries that involve the concept of every
            \end{itemize}
            \bigbreak \noindent 
            only include rows where the subquery is EMPTY
        \item \textbf{Union}: The UNION operator causes two sets to be merged, the set union.
            \bigbreak \noindent 
            \begin{sqlcode}
                SELECT P
                    FROM P
                    WHERE WEIGHT > 18 # first SELECT returns only P6
                UNION
                    SELECT P
                    FROM SP
                    WHERE S = 'S2'; # second query returns P1, P2
            \end{sqlcode}
        \item \textbf{Union caveat}: You should be careful in situations where the domain of a column matters, as UNION will put rows together whether the columns match in type/purpose or not.
        \item \textbf{Group Functions}: Group functions are sometimes referred to as aggregate or multiple-row functions. They take a list of columns as an argument, with an optional DISTINCT or ALL inside before those columns are listed.
            \begin{itemize}
                \item \textbf{SUM(<x>)}: add up the value of column <x> in all of the rows of each group
                \item \textbf{AVG(<x>)}: find the average value of column for each group
                \item \textbf{COUNT(<x>)}: count how many rows there are (usually <x> is a * here.)
                \item \textbf{MAX(<x>)}: returns the maximum value of column <x> for each group
                \item \textbf{MIN(<x>)}: returns the minimum value of column for each group
                \item \textbf{STDDEV(<x>)}: returns the standard deviation of column <x> for each group
                \item \textbf{VARIANCE(<x>)}: returns the variance of column for each group
            \end{itemize}
            All of these functions will return a single value for each group present.
            \bigbreak \noindent 
            If no GROUP BY clause is included, then there is only a single group, which contains all the rows of the query. The GROUP BY clause will allow that to be divided into subgroups.
        \item \textbf{Group function example: COUNT}: Find out the number of suppliers
            \bigbreak \noindent 
            \begin{sqlcode}
                SELECT COUNT(*) FROM S;
            \end{sqlcode}
        \item \textbf{DISTINCT with group functions}: Get the total number of suppliers currently supplying parts
            \bigbreak \noindent 
            If you want to count only distinct values, we can do that with DISTINCT
            \bigbreak \noindent 
            \begin{sqlcode}
            SELECT COUNT(DISTINCT S) FROM SP;
            \end{sqlcode}
            \bigbreak \noindent
        \item \textbf{WHERE clause with group functions}: The WHERE clause is evaluated BEFORE any groups are formed.
            \bigbreak \noindent 
            \begin{sqlcode}
                SELECT COUNT(*)
                    FROM SP
                    WHERE P = 'P2'; # the value of P is known before grouping, so WHERE works
            \end{sqlcode}
        \item \textbf{The GROUP BY clause}: The GROUP BY clause in a SELECT statement takes the following form:
            \bigbreak \noindent 
            \begin{sqlcode}
            GROUP BY <attrs>
            \end{sqlcode}
            \bigbreak \noindent 
            It will cause the SELECT statement to examine the rows in its result set, and gather the ones that match on their values for the columns in <attrs> into subgroups.
            \bigbreak \noindent 
            \begin{sqlcode}
            SELECT SUM(QTY) FROM SP
                GROUP BY P; # make a subgroups for each part

            SELECT P, SUM(QTY) FROM SP # added P to be shown
                GROUP BY P; # make a subgroup for each part
            \end{sqlcode}
        \item \textbf{Group by caveat}: However, if we try to display columns that aren't part of the <attrs> of the GROUP BY and aren't calculated by a group function, we begin to have problems.
            \bigbreak \noindent 
            \begin{sqlcode}
            SELECT P, S, QTY, SUM(QTY) FROM SP GROUP BY P; # P is good, but look at S and QTY
            \end{sqlcode}
            \bigbreak \noindent 
            \begin{center}
                \begin{tabular}{c|c|c|c}
                    P &S &QTY &SUM(QTY) \\
                    \hline
                    P1 &S1 &300 &600 \\
                    P2 &S1 &200 &1000 \\
                    P3 &S1 &400 &400 \\
                    P4 &S1 &200 &500 \\
                    P5 &S1 &100 &500 \\
                    P6 &S1 &100 &100 
                \end{tabular}
            \end{center}
            \bigbreak \noindent 
            What do the values of S and QTY mean in this grouped context? Nothing! They are not relevant or correct. Is S1 the
            only supplier for all of the groups? The SP table indicates no. Is the QTY there valid for P1? No, the correct answer
            for total P1 supplied is in the SUM(QTY).
            \bigbreak \noindent 
            There is a distinct value of S and a value of QTY for every row in each subgroup. That is many values, and only one
            place to show them in - it's not atomic. Unfortunately the DBMS is just choosing one to show anyway, but it has no
            meaning, and such situations should be avoided.
        \item \textbf{HAVING clause}: Just as the WHERE clause could be used to filter individual rows based on whether they evaluated true for its expression, the HAVING clause allows us to filter out groups based on values that pertain to the group.
            \bigbreak \noindent 
            \begin{sqlcode}
            HAVING <expr>
            \end{sqlcode}
            \bigbreak \noindent 
            For each group in the results, the HAVING expression, <expr> is evaluated, and only groups where <expr> is true will be included in the final output.
            \bigbreak \noindent 
            The reason HAVING is necessary is that the WHERE clause is evaluated BEFORE the groups are formed, and is not able to work with values that don't exist until after it has already finished.
        \item \textbf{Example with HAVING}: List the part numbers for all parts supplied by more than one supplier.
            \bigbreak \noindent 
            \begin{sqlcode}
                SELECT P
                    FROM SP
                    GROUP BY P
                    HAVING COUNT(*) > 1;
            \end{sqlcode}
        \item \textbf{Single-Row Subqueries}: Single-row subqueries are subqueries that return a single value (ONE ROW with ONE COLUMN).
            \bigbreak \noindent 
            Like the multiple-row subqueries, they are evaluated and then their results are used in the parent query that contained them.
            \bigbreak \noindent 
            They don't need to use the multiple-row operators to work.
            \bigbreak \noindent 
        \item \textbf{Single-Row Subquery as a column}: SELECT Title, Retail, (SELECT AVG(Retail) FROM Books) \# third column will have result 'Overall Average' \# with a changed title
            \bigbreak \noindent 
            \begin{center}
                \begin{tabular}{c|c|c}
                    Title &Retail Overall &Average \\
                    \hline
                    The Princess Bride &39.99 &42.00  \\
                    The Life of Pi &3.14 &42.00 \\
                    The Hitchhiker's Guide &29.50 &42.00 \\
                    $\cdots$ & $\cdots$ & $\cdots $
                \end{tabular}
            \end{center}
            \bigbreak \noindent 
            Having the call to the group function AVG would normally reduce the results to a single row per group, but it happened inside a subquery, so it did not change the outer query. This can be useful when you really want to know an aggregate value but don't want to condense your rows.
        \item \textbf{Single-Row Subquery in a WHERE clause}: Let's use a bookstore as an example. If you knew the ISBN of a book and wanted to run a query to find all of the books that are more expensive than it, you could use a subquery to find out the cost of the book with that ISBN and then compare that value with its result.
            \bigbreak \noindent 
            \begin{sqlcode}
            SELECT Title, Cost
                FROM Books
                WHERE Cost > # compare the cost of current row with result of subquery
                    (SELECT Cost # only the Cost returned -- single column
                    FROM Books
                    WHERE ISBN = '1328948854'); # ISBN is PK -- single row
            \end{sqlcode}
        \item \textbf{Single-Row Subquery in a HAVING clause}: Since the result of the subquery is inserted in place, it will work anywhere a single value makes sense. This includes use as part of a HAVING clause. Using the same book database from the previous slide:
            \bigbreak \noindent 
            \begin{sqlcode}
                SELECT Category,
                    AVG(Retail - Cost) 'Average Profit' # calculate average profit of all books, change the label
                    FROM Books
                    GROUP BY Category
                    HAVING AVG(Retail - Cost) > # compare cost of each group with result of the subquery
                        ( SELECT AVG(Retail - Cost) # finds the average profit for books in LIT
                        FROM Books
                        WHERE Category = 'LIT' );
            \end{sqlcode}
        \item \textbf{Single-Row Subquery example 1}: List the supplier numbers for suppliers who are located in the same city as supplier S1
            \bigbreak \noindent 
            \begin{sqlcode}
                SELECT S
                    FROM S
                    WHERE CITY = # compare each row with result of subquery
                        ( SELECT CITY # find out which city S1 is in
                        FROM S
                        WHERE S = 'S1' );
                    \end{sqlcode}
        \item \textbf{The LIKE operator}: So far, all of the string comparisons we've done have been with the = operator, which tests for strict equality. (Locale
            settings determine whether it's a case sensitive or case insensitive match.)
            \bigbreak \noindent 
            Using just =, we'd have to have a lot of OR's strung together to have any kind of flexibility.
            \bigbreak \noindent 
            If we have a pattern to be matched, we generally won't use =, but rather the LIKE operator.
            \bigbreak \noindent 
            \begin{sqlcode}
                <val> LIKE <pattern>
            \end{sqlcode}
            The LIKE operator will return true when <val> matches the pattern specified in <pattern>.
        \item \textbf{Patterns with LIKE}: The patterns that LIKE uses to check your values against are defined using these special characters. 
            \begin{itemize}
                \item \% $\quad$ any zero or more characters can fit here without breaking the match
                \item \_ $\quad$ any single character can fit here without breaking the match
                \item \textbackslash $\quad$escape the next character
                \item \% $\quad$escaped \%, so only match the actual \% character here
                \item \textbackslash\_ $\quad$ escaped \textbackslash\_, so only match the actual \_ character here
                \item \textbackslash\textbackslash $\quad $ escaped \textbackslash, so only match the actual \textbackslash character here
            \end{itemize}
            \bigbreak \noindent 
            Any characters not in this list will only match themselves.
            \bigbreak \noindent 
        \item \textbf{LIKE: Character classes and union (or)}: You can specify a list or range of characters with square brackets.
            \bigbreak \noindent 
            \begin{sqlcode}
            SELECT ...
                WHERE ... LIKE "_[abc]%"
                WHERE ... LIKE "_[a-z]%"
            \end{sqlcode}
        \item \textbf{Negating character classes:} To invert a character class, we can use !
            \bigbreak \noindent 
            \begin{sqlcode}
            SELECT ...
                WHERE ... LIKE "_[!abc]%"
            \end{sqlcode}
        \item \textbf{List suppliers whose name starts with the letter 'S'}:
            \bigbreak \noindent 
            \begin{sqlcode}
                SELECT * FROM S
                    WHERE SNAME LIKE `S%`;
            \end{sqlcode}
        \item \textbf{Single-valued (non-group) functions}: Unlike the aggregate functions, these functions won't make your results collapse based on groups. They are evaluated, and their value is inserted in place.
            \begin{itemize}
                \item \textbf{LOWER(<str>)}: Returns copy of <str> but all lowercase
                \item \textbf{UPPER(<str>)}: Returns a copy of <str> but all uppercase
                \item \textbf{SUBSTR(<str>, <pos>, <len>)}: Returns a copy of the substring of <str> starting at its <pos>th position, <len> characters long
                \item \textbf{LENGTH(<str>)}: Returns the length in characters of the string, <str>
                \item \textbf{LPAD(<str>, <len>, <sp>)}: Returns <str> fit into <len> characters, padding with <sp> on the left if necessary
                \item \textbf{RPAD(<str>, <len>, <sp>)}: Returns <str> fit into <len> characters, padding with <sp> on the right if necessary
                \item \textbf{ROUND(<num>,<pos>)}: Returns the number <num>, rounded to <pos> digits after the decimal point
                \item \textbf{CONCAT(<str>, [...])} Returns a the concatenation of the strings <str>, in order.
                \item \textbf{SOUNDEX(<str>)}: Returns a string containing a code that can be used to compare how <str> sounds like other strings.
            \end{itemize}
            \bigbreak \noindent 
            These functions can be nested however you'd like. Just like C++, they're evaluated from the inside out.
        \item \textbf{JOINS}: We've seen joins in some of our examples already.
            \bigbreak \noindent 
            A join is an operation that takes information from separate tables and combines it into one set of results.
            \bigbreak \noindent 
            There are two basic types of join:
            \begin{enumerate}
                \item \textbf{Inner join}
                \item \textbf{Outer join}
            \end{enumerate}
            For either of these types of join, it is possible to join a table with itself, in which case we call it a self join.
        \item \textbf{Change to S}: To make things interesting in these joins, let's add a new supplier, S7, to our S table. They won't supply anything yet so leave P and SP unchanged.
            \bigbreak \noindent 
            We have already done a few inner joins in the earlier examples,
            \bigbreak \noindent 
        \item \textbf{Inner join}: With an inner join, only lines that match up with each other in both tables will be a part of the result. Earlier, we accomplished this by putting together two tables with the Cartesian Product, and then using a WHERE clause to make sure only things that matched on the foreign key were retained.
            \bigbreak \noindent 
            \begin{sqlcode}
                SELECT S.S,P,SNAME,QTY
                    FROM S,SP # Cartesian product of S with SP
                    WHERE S.S = SP.S; # only keep rows matching S=S
            \end{sqlcode}
            \bigbreak \noindent 
            Can be written as
            \bigbreak \noindent 
            \begin{sqlcode}
            SELECT S.S,P,SNAME,QTY
                FROM S JOIN SP # replace the comma with the keyword JOIN
                ON S.S = SP.S; # WHERE becomes ON for the foreign key
            \end{sqlcode}
        \item \textbf{Outer Join}: An outer join can be more flexible than an inner join. It will contain everything that the inner join contained, but one or both of the two tables involved will be special, and will have at least one row in the results whether it matched the other side or not. The values for the missing side will be filled with NULL since there is no relevant value.
            \bigbreak \noindent 
            Here we have the same query from before, but as an outer join instead of an inner join.
            \bigbreak \noindent 
            \begin{sqlcode}
                SELECT S.S,P,SNAME,QTY
                    FROM S LEFT JOIN SP # LEFT means table on LHS of JOIN is the strong one
                    ON S.S=SP.S;
            \end{sqlcode}
            \bigbreak \noindent 
            LEFT means the table on the left-hand side of the JOIN keyword is the strong one. RIGHT would mean the RHS is strong. In some dialects of SQL, you can use FULL to make both strong, but this does not work in MariaDB. You can accomplish something similar with a UNION if needed.
        \item \textbf{The LIMIT Clause}: the LIMIT clause is used to restrict the number of rows returned by a query. When you specify LIMIT 50, it tells the database to return only the first 50 rows of the result set.
            \bigbreak \noindent 
            \begin{sqlcode}
            SELECT * from sometable LIMIT 50; // Queries the first 50 rows
            \end{sqlcode}
            \bigbreak \noindent 
            \textbf{Note:} Redundant if the number of rows in the relation is less than the limit restriction
        \item \textbf{IS NULL and IS NOT NULL}: A field with a NULL value is a field with no value.
            \bigbreak \noindent 
            If a field in a table is optional, it is possible to insert a new record or update a record without adding a value to this field. Then, the field will be saved with a NULL value.
            \bigbreak \noindent 
            It is not possible to test for NULL values with comparison operators, We will have to use the IS NULL and IS NOT NULL operators instead.
            \bigbreak \noindent 
            \begin{sqlcode}
            SELECT ...
                WHERE ... IS NULL;

            SELECT ...
                WHERE ... IS NOT NULL;
            \end{sqlcode}
        \item \textbf{BETWEEN operator}: the BETWEEN operator is used to filter the result set within a specified range. It works for numbers, dates, and text values.
            \bigbreak \noindent 
            \begin{sqlcode}
            SELECT ...
                WHERE ... BETWEEN val1 AND val2;
            \end{sqlcode}
            \bigbreak \noindent 
            We can of course negate this with NOT
            \bigbreak \noindent 
            \begin{sqlcode}
            SELECT ...
                WHERE ... NOT BETWEEN val1 AND val2;
            \end{sqlcode}
    \end{itemize}

    \pagebreak 
    \subsection{Frontend: Html}
    \begin{itemize}
        \item \textbf{Why HTML for databases?}: HTML is a relatively easy way to provide a graphical user interface to a user. This interface can solicit from the user the data that is needed to perform operations on the database, as well as present to the user the data that is returned in a structured way
        \item \textbf{HTML vs. XHTML}: As a web browser goes through evaluating all of the HTML code it encounters, it has several possible modes of operation.
            \begin{itemize}
                \item The HTML-only parser mode is very loose. If you make a mistake, it will try to guess what you meant and continue anyway. This can cause weird behavior with no warning that anything was wrong.
                \item The XHTML parser performs more checks, which can help you locate and fix your problem areas more quickly.
            \end{itemize}
        \item \textbf{Tree-Like Structure}: A well-formed HTML document should have a structure based upon the data structure known as a tree. This tree is formed by the elements present in the document, which can be used as containers of one or more other elements.
        \item \textbf{Elements}: There are three basic types of elements that you will see in HTML.
            \begin{enumerate}
                \item \textbf{Plain Text}: This is just normal text, in whatever encoding (ASCII, EBCDIC, Unicode) the document is using. Note that HTML-specific characters will need to be properly escaped.
                \item \textbf{Comments}: These start with $<$!\texttt{--} and end with \texttt{--}$>$
                \item  \textbf{HTML tags}: These are special keywords, surrounded by angle brackets ( $<$ and $>$). For each tag element, there will be a start tag and an end tag
                    \bigbreak \noindent 
                    The start, or open tag is, in its most simple form, just the keyword in angle brackets, eg. $<$tag$>$.
                    \bigbreak \noindent 
                    The end, or close tag, when written separately, has a forward slash / after the first angle bracket, eg. $<$/tag$>$.
            \end{enumerate}
        \item \textbf{Plain Text}: Plain text elements will be displayed (or not displayed) in a manner appropriate to where they occur in the document.
        \item \textbf{A Note on Comments}: Comments are nice, and should still be used when they would be helpful. However, since the comments are sent to the user from the server every time the page is loaded, it might be beneficial, from a performance standpoint, to keep them to a minimum.
        \item \textbf{HTML Tags}: They can be written separately, usually when they are surrounding something, such as <div>Text Here</div> - note the / at the beginning of the end tag.
            \bigbreak \noindent 
             If there is nothing between the start and end tag, there is a shorthand, <meta/> - the / at the end of the tag means that the end tag is included with the opening tag, so it is equivalent to <meta></meta>, with nothing inside.
             \bigbreak \noindent 
             If the element is generally expected to have content, you should not use this shortcut.
            \item \textbf{Attributes}: It is possible to specify attributes of an HTML tag. These get written as part of the start, or opening tag, and they can either be name=value pairs (attrname1 or attrname2 below), or just the name of an attribute if it is boolean if it is desired to set it to true (boolattr below).
                \bigbreak \noindent 
                \begin{htmlcode}
                <tag attrname1="attr1value" attrname2="attr2value" boolattr>
                \end{htmlcode}
                \bigbreak \noindent 
                Each type of tag will have a different set of applicable attributes. The values specified for these attributes will control the behavior of the element.
    \item \textbf{Basic HTML Document}:
        \bigbreak \noindent 
        \begin{htmlcode}
            <html>
                <head>
                    <title>Page Title Here</title>
                    <meta charset="UTF-8"/>
                </head>
                <body>
                    <h1>Big Heading Here</h1>
                    <p>A paragraph here.</p>
                </body>
            </html>
        \end{htmlcode}
    \item \textbf{<html> Element}: The <html> element should be the one of the first elements to appear, and will contain all of the other elements in your document.
        \bigbreak \noindent 
        There should be only one per document.
    \item \textbf{<head> Element}: The <head> element is meant to contain header information. Among other things, this can include:
        \begin{itemize}
            \item <title> The title that will be displayed for this document in either a tab button or the window caption. The title is the text between the start and the end tag, not an attribute.
            \item <meta> Various information about the document that you would like to make available to search engines, etc.
        \end{itemize}
        Notice that any text you put in the <head> tag will not show up when the document is rendered.
    \item \textbf{<body> Element}: The <body> element will contain the body of your document. Any text that you would like to display to the user should be contained by this element
    \item \textbf{Working with Text}: Before we start working with text, there are a few things we should address.
        \begin{enumerate}
            \item whitespace (spaces, tabs) is mostly ignored after the first space
            \item newline character in the source do not get shown as newlines when the document is displayed
            \item characters that could be interpreted as part of the HTML markup may need to be escaped to show up as text-only
        \end{enumerate}
    \item \textbf{Special Characters and Escape Codes}:
        \begin{itemize}
            \item \texttt{\&gt;} greater than symbol (because used to identify tags)
            \item \texttt{\&lt;} less than symbol (because used to identify tags)
            \item \texttt{\&amp;} ampersand symbol (because used for these codes)
            \item \texttt{\&nbsp;} non-breaking space (this is a space that will not be ignored)
            \item \texttt{\&copy;} copyright symbol, \textcopyright\ (not a part of ASCII alphabet)
            \item \texttt{\&\#9999;} Unicode character with decimal value 9999
            \item \texttt{\&\#xffff;} Unicode character with hexadecimal value ffff
        \end{itemize}
    \item \textbf{Headings - <h1> to <h6>}: The tags, <h1> through <h6>, provide various levels of headings. <h1> is the first heading, and will have the largest text, with the others getting progressively smaller in importance.
    \item \textbf{Paragraphs - <p>}: The <p> element is used to indicate that the text it contains is meant to be a paragraph. In general, this means that
        it will be printed as a single block of text, wrapping automatically as its text reaches the end of the line.
        \bigbreak \noindent 
        What if we wanted to format an address, with the name on one line, the street address on another, and the city/state/zipcode below that?
        \bigbreak \noindent 
        \begin{htmlcode}
            <p>Northern Illinois University</p>
            <p>1425 West Lincoln Highway</p>
            <p>DeKalb, IL 60115</p>
        \end{htmlcode}
    \item \textbf{Line breaks}: There is a special tag, <br />, the line break, that can help in this situation.
        \bigbreak \noindent 
        \begin{htmlcode}
            <p>Northern Illinois University<br />
                1425 West Lincoln Highway<br />
                DeKalb, IL 60115</p>
        \end{htmlcode}
        \bigbreak \noindent 
        \textbf{Note:} Some purists will argue that the <br /> tag should not be used, because it's purely visual, and the purpose of HTML is to denote structure. They are not entirely wrong, but it's more work to do it other ways, so it will work in a pinch.
    \item \textbf{Unordered Lists - <ul>}: It is possible to create unordered (bulleted) lists with <ul> and <li> elements.
        \bigbreak \noindent 
        \begin{htmlcode}
            <ul>
                <li>No</li>
                <li>Particular</li>
                <li>Order</li>
            </ul>
        \end{htmlcode}
    \item \textbf{Ordered Lists - <ol>}: It is possible to create ordered (numbered) lists with <ol> and <li> elements.
        \bigbreak \noindent 
        \begin{htmlcode}
            <ol>
                <li>First</li>
                <li>Second</li>
                <li>Third</li>
            </ol>
        \end{htmlcode}
        \bigbreak \noindent 
        The type of numeral used for ordered lists can be controlled by CSS, or via the type attribute of the <ol> tag. Its possible values are (1, A, a, I, and i).
    \item \textbf{Images - <img>}: It is possible to embed images into your HTML documents. This is done with the <img> tag. On its own, the <img>
        element cannot do much, because it needs more information. It will be necessary to supply the path to the file
        containing the image to display as the src attribute.
        \bigbreak \noindent 
        In HTML, paths can be specified in any of three ways:
        \begin{itemize}
            \item \textbf{relative paths} - these are paths starting from the directory the current HTML document is being loaded from. These will never begin with a forward slash, \texttt{/}.
            \item \textbf{absolute paths (local)} - these are paths starting from the root of the webserver that is serving the current document, these will always begin with a forward slash, \texttt{/}.
            \item \textbf{absolute paths (remote)} - these are paths that are on other servers. They will begin with the protocol specifier of the server. (\texttt{http://}, \texttt{https://}, \texttt{ftp://}, etc.)
        \end{itemize}
        With all three of these types of paths, the steps along the path will be separated by a forward slash, \texttt{/}.
    \item \textbf{Images - Useful Attributes}: Some useful attributes for the \texttt{<img>} element:
        \begin{itemize}
            \item \textbf{height} - height of image (in pixels) or as a percentage of screen (with a \texttt{\%} at the end)
            \item \textbf{width} - width of image (in pixels) or as a percentage of screen (with a \texttt{\%} at the end)
            \item \textbf{alt} - alternate text, which will show when hovering, or may be read aloud for accessibility purposes
        \end{itemize}
        The \texttt{height} and \texttt{width} attributes can be used to scale the image, but the same image file is downloaded either way. It is best to make the image exactly the size you intend to display it using some image editing tool, rather than resizing it on the page. The best use for \texttt{height} and \texttt{width} is to allow the browser to know the size of the image before downloading it, so the page doesn't have to resize as it loads.
    \item \textbf{Links}: One of the features that originally made HTML popular was the idea of the hyperlink. Now we just call them links.
        \bigbreak \noindent 
        They are portions of the document that can be clicked to navigate to somewhere else.
        \bigbreak \noindent 
        They are added to the HTML using the anchor tag, <a>. The path for the destination of the link is specified with the href attribute, and the text that forms the link will be whatever is contained by the anchor element.
        \bigbreak \noindent 
        \begin{htmlcode}
            <a href="other.html">Relative Path Link</a>
            <a href="/some.html">Absolute Path Link</a>
            <a href="http://www.niu.edu"> Link on another server</a>
        \end{htmlcode}
    \item \textbf{Tables}: Something that comes up very often when dealing with relational databases is the table. There is support for drawing them in HTML. This is done with a set of four elements that work together:
        \begin{itemize}
            \item \texttt{<table>} - This element starts a table, the others should only occur inside one of these.
            \item \texttt{<tr> (table row)} - This element is a row within a table. It will contain one or more of either of the following two.
            \item \texttt{<th> (table heading)} - This element is a table heading cell. The text to display as a column label should go inside.
            \item \texttt{<td> (table data)} - This element is a table data cell. The text inside should be the data to be shown in the cell.
        \end{itemize}
        Tables in HTML are always specified in row-major order. The cells go from left to right inside of rows. There is not a way to specify it columnwise.
        \bigbreak \noindent 
        \begin{htmlcode}
            <table>
                <tr>
                    <th>Topping</th>
                    <th>Price</th>
                </tr>
                <tr>
                    <td>Sausage</td>
                    <td>$0.90</td>
                </tr>
                <tr>
                    <td>Pepperoni</td>
                    <td>$1.00</td>
                </tr>
            </table>
        \end{htmlcode}
    \item \textbf{Forms}: In HTML, forms are used to collect user input and submit it to a server for processing. They are fundamental to creating interactive web applications, allowing users to input information, like login credentials, feedback, or search queries
        \bigbreak \noindent 
        \begin{htmlcode}
            <form action="submit_page.php" method="post">
                <!-- Form elements like input fields, buttons, etc. go here -->
            </form>
        \end{htmlcode}
        \begin{itemize}
            \item \textbf{action:} Specifies the URL where form data should be submitted.
            \item \textbf{method:} Defines the HTTP method for submitting the form. Common values are:
                \begin{enumerate}
                    \item \textbf{GET:} Sends form data as URL parameters (query strings), typically used for non-sensitive data.
                    \item \textbf{POST:} Sends form data in the request body, more secure for sensitive data.
                \end{enumerate}
        \end{itemize}
    \item \textbf{Input}: The <input> tag in HTML is used to create interactive fields in forms that collect user input. The <input> tag is versatile and supports various types of inputs by setting the type attribute to specify the kind of data the field should accept.
        \begin{enumerate}
            \item \textbf{Text Input:} Collects single-line text input, like names or emails.
                \bigbreak \noindent 
                \begin{htmlcode}
                    <input type="text" name="username" placeholder="Enter your name">
                \end{htmlcode}
            \item \textbf{Password Input:} Similar to text but hides characters as they're typed.
                \bigbreak \noindent 
                \begin{htmlcode}
                 <input type="password" name="password" placeholder="Enter your password">
                \end{htmlcode}
            \item \textbf{Radio Buttons:} Allow users to select a single option from a set.
                \bigbreak \noindent 
                \begin{htmlcode}
                    <input type="radio" name="gender" value="male"> Male
                    <input type="radio" name="gender" value="female"> Female
                \end{htmlcode}
            \item \textbf{Checkboxes:} Let users select multiple options independently.
                \bigbreak \noindent 
                \begin{htmlcode}
                 <input type="checkbox" name="subscribe" value="newsletter"> Subscribe to newsletter
                \end{htmlcode}
            \item \textbf{Submit Button:} Sends the form data to the server.
                \bigbreak \noindent 
                \begin{htmlcode}
                    <input type="submit" value="Submit">
                \end{htmlcode}
            \item \textbf{Color Picker:} Allows users to pick a color from a color wheel or enter a color code.
                \bigbreak \noindent 
                \begin{htmlcode}
                    <input type="color" name="favcolor" value="#ff0000">
                \end{htmlcode}
                
            \item \textbf{Date Picker:} Provides a calendar interface for selecting a date.
                \bigbreak \noindent 
                \begin{htmlcode}
                    <input type="date" name="birthdate">
                \end{htmlcode}

            \item \textbf{Email Input:} Accepts and validates email addresses.
                \bigbreak \noindent 
                \begin{htmlcode}
                    <input type="email" name="email" placeholder="Enter your email" required>
                \end{htmlcode}
                
            \item \textbf{File Upload:} Allows users to upload files from their device.
                \bigbreak \noindent 
                \begin{htmlcode}
                    <input type="file" name="profilePicture">
                \end{htmlcode}

            \item \textbf{Hidden Input:} Stores data that isn't visible to the user but is submitted with the form.
                \bigbreak \noindent 
                \begin{htmlcode}
                    <input type="hidden" name="userID" value="12345">
                \end{htmlcode}
                
            \item \textbf{Image Button:} Acts as a submit button but uses an image instead of text.
                \bigbreak \noindent 
                \begin{htmlcode}
                    <input type="image" src="submit_button.png" alt="Submit" width="50" height="50">
                \end{htmlcode}
                
            \item \textbf{Month Picker:} Allows users to select a specific month and year.
                \bigbreak \noindent 
                \begin{htmlcode}
                    <input type="month" name="startmonth">
                \end{htmlcode}
                
            \item \textbf{Number Input:} Accepts numeric values, optionally within a specified range.
                \bigbreak \noindent 
                \begin{htmlcode}
                    <input type="number" name="quantity" min="1" max="10" step="1">
                \end{htmlcode}
                
            \item \textbf{Range Slider:} Allows users to select a number within a specified range by sliding a handle.
                \bigbreak \noindent 
                \begin{htmlcode}
                    <input type="range" name="volume" min="0" max="100" step="10">
                \end{htmlcode}

            \item \textbf{Reset Button:} Clears all input fields in the form to their default values.
                \bigbreak \noindent 
                \begin{htmlcode}
                    <input type="reset" value="Reset">
                \end{htmlcode}

            \item \textbf{Search Field:} Provides a field for entering search queries.
                \bigbreak \noindent 
                \begin{htmlcode}
                    <input type="search" name="query" placeholder="Search...">
                \end{htmlcode}
                
            \item \textbf{Telephone Input:} Used for entering telephone numbers, with optional pattern for specific formats.
                \bigbreak \noindent 
                \begin{htmlcode}
                    <input type="tel" name="phone" placeholder="123-456-7890" pattern="[0-9]{3}-[0-9]{3}-[0-9]{4}">
                \end{htmlcode}

            \item \textbf{URL Input:} Accepts and validates URLs.
                \bigbreak \noindent 
                \begin{htmlcode}
                    <input type="url" name="website" placeholder="https://example.com">
                \end{htmlcode}
        \end{enumerate}
    \item \textbf{Form validation}: HTML5 provides several attributes for form validation:
        \begin{itemize}
            \item \textbf{required:} Ensures a field must be filled out.
                \bigbreak \noindent 
                \begin{htmlcode}
                    <input type="email" name="email" required>
                \end{htmlcode}
            \item \textbf{pattern:} Specifies a regular expression pattern for input validation.
                \bigbreak \noindent 
                \begin{htmlcode}
                    <input type="text" name="username" pattern="[A-Za-z]{3,}">
                \end{htmlcode}
            \item \textbf{min and max:} Define minimum and maximum values for number fields.
                \bigbreak \noindent 
                \begin{htmlcode}
                    <input type="number" name="age" min="1" max="100">
                \end{htmlcode}
        \end{itemize}
    \item \textbf{Dropdown List (Select):} Allows users to select from a list of options.
        \bigbreak \noindent 
        \begin{htmlcode}
            <select name="country">
                <option value="usa">USA</option>
                <option value="canada">Canada</option>
            </select>
        \end{htmlcode}
    \item \textbf{Textarea:} For multi-line text input, such as comments or descriptions.
        \bigbreak \noindent 
        \begin{htmlcode}
            <textarea name="comments" placeholder="Enter your comments"></textarea>
        \end{htmlcode}
    \item \textbf{Buttons in html}: We can create buttons with the button tag
        \bigbreak \noindent 
        \begin{htmlcode}
            <button> </button>
        \end{htmlcode}
    \item \textbf{Nav tags}: The <nav> tag in HTML is used to define a section of navigation links on a web page. It's a semantic element introduced in HTML5 that helps improve the structure and accessibility of a website by explicitly marking sections dedicated to navigation. This tag is typically used for primary navigation elements like menus, tables of contents, or other links that users can use to navigate a site.
        \bigbreak \noindent 
        \begin{htmlcode}
            <nav>
                <ul>
                    <li><a href="index.html">Home</a></li>
                    <li><a href="about.html">About Us</a></li>
                    <li><a href="services.html">Services</a></li>
                    <li><a href="contact.html">Contact</a></li>
                </ul>
            </nav>
        \end{htmlcode}
    \item \textbf{Div containers}: The most commonly used container in HTML, <div> stands for "division."
        \bigbreak \noindent 
        It's a generic container that can hold any other HTML elements.
        \bigbreak \noindent 
        Used for grouping elements to apply CSS styles or JavaScript functionality.
        \bigbreak \noindent 
        \begin{htmlcode}
            <div class="container">
                <h1>Welcome to My Website</h1>
                <p>This is a sample container.</p>
            </div>
        \end{htmlcode}
    \item \textbf{Section container}: A semantic container used to group related content.
        \bigbreak \noindent 
        Useful for sections like articles, services, or product listings.
        \bigbreak \noindent 
        \begin{htmlcode}
            <section class="features">
                <h2>Features</h2>
                <p>Our product offers the following features:</p>
            </section>
        \end{htmlcode}
    \item \textbf{<header>, <footer>, and <aside> Tags:}
        \begin{itemize}
            \item \textbf{<header>:} For header content, like logos or navigation.
            \item \textbf{<footer>:} For footer content, like copyright or contact information.
            \item \textbf{<aside>:} For content related to the main content, like sidebars or ads.
        \end{itemize}
    \item \textbf{Span}: The <span> element, can be used to group inline elements only. So, if you have a part of a sentence or paragraph which you want to group together
    \end{itemize}

    \pagebreak 
    \subsection{PHP}
    \begin{itemize}
        \item \textbf{Components of web services}: 
            \begin{itemize}
                \item \textbf{Web browser}: formats and displays Web pages
                \item \textbf{Web server}: sends Web pages to browsers and lets site visitors enter and request information
                \item \textbf{Information server}: accepts requests from the Web server and uses its stored data to respond appropriately
            \end{itemize}
        \item \textbf{Database server}: A database server is a kind of information server, it stores information in databases
            \bigbreak \noindent 
            Web servers, etc. connect to the database server to send queries or update data
        \item \textbf{Web pages and HTML}: An HTML document is used to create the format and structure of a Web page
            \bigbreak \noindent 
                HTTP is a communication protocol that specifies how two or more things are expected to interact on the Web
        \item \textbf{Web server}: a computer system that responds to requests for Web pages by processing the requests and returning a new Web page to the browser
        \item \textbf{Preparing to use php}: The php language was designed to help developers create dynamic and data-driven web pages
            \bigbreak \noindent 
            php interacts with one main external tool, the MySQL database management system, to access data stored in a database
            \bigbreak \noindent 
            MySQL must be installed on a functional Web server to interact with php, but this is a relatively easy step in setting up the php environment
            \bigbreak \noindent 
            php is a server-side scripting language that you can embed into HTML documents. You can also embed HTML in php scripts
            \bigbreak \noindent 
            php scripts are parsed and interpreted on the server side of a Web application
            \bigbreak \noindent 
            php is popular with web developers and web designers alike, and is powerful and easy to use
            \bigbreak \noindent 
            To start working with php, you can create a script that contains HTML code
        \item \textbf{Methods of using php in html}: There are two ways 
            \begin{bashcode}
            <?php 
                ...
            ?>

            <SCRIPT LANGUAGE="php"
                ...
            </SCRIPT>
            \end{bashcode}
        \item \textbf{PHP example}: We can create a file called example.php, that is essentially an html file, but can contain php code in <?php ... ?> blocks. For example,
            \bigbreak \noindent 
            \begin{bashcode}
            <body>
                <?php 
                    print("Hello world");
                ?>
            </body>
            \end{bashcode}
            \bigbreak \noindent 
            Note that lines in php are terminated with a semicolon
        \item \textbf{Displaying php Output}: php has two functions that allow display: echo and print
            \bigbreak \noindent 
            The only difference between echo and print is that the print function returns a 1 or 0 integer (denoting success or failure, respectively), for the contents of the function being displayed
            \bigbreak \noindent 
            Also, be aware that if you want to send php reserved characters (such as double quotations) to the Web browser within the echo command, you must use the backslash character
        \item \textbf{Defining php Variables}:  Variables in php are preceded with a dollar sign (\$) and contain either letters or numbers
            \bigbreak \noindent 
            php is called a loosely typed programming language, meaning that you don't have to predefine your variables; you can define and use them as needed
            \bigbreak \noindent 
            You do have to follow certain rules for naming a variable:
            \begin{itemize}
                \item Precede the variable name with a dollar sign
                \item Assign the variable a meaningful name that you can remember in the future
                \item Name the variable with uppercase or lowercase letters, numbers, or the underscore (\_) character
                \item Do not allow the first character after the (\$) to be a number
                \item Variable names are case sensitive
                \item Assign the variable an initial value with a single equals (=) sign
            \end{itemize}
            \bigbreak \noindent 
            \begin{bashcode}
                <?php 
                    $x = 3.14;
                    echo $x;
                    echo "<br/>";
                    print($x);
                ?>
            \end{bashcode}
        \item \textbf{Variable scope}:  If a variable is defined at the start of a php file, it stays in memory until the end of that file. This is known as the variable's scope
            \bigbreak \noindent 
            If a variable is assigned a value of 5 in one php file, and that file calls another php file that has a variable of the same name, then the first variable is terminated and its value is lost
            \bigbreak \noindent 
            One major distinction that relates to a variable's scope involves the processing of web-based forms
            \bigbreak \noindent 
            Any variables that are defined within a php/HTML form and sent to the server with the form's Post method are automatically sent with the called Post action and named in php by the same name used in the HTML form
        \item \textbf{Variable data types}: 
            \begin{itemize}
                \item \textbf{Boolean}: 
                \item \textbf{Integer}:
                \item \textbf{Float or double}:
                \item \textbf{String }:
                \item \textbf{Object}:  An instance of a class
                \item \textbf{Array}:  Ordered set of keys and values
                \item \textbf{Resource}: Reference to a thirdparty resource (a database for example)
                \item \textbf{NULL}: An unintialized variable
            \end{itemize}
        \item \textbf{Gettype function}: Test the type of a variable by using the built-in php function gettype().
        \item \textbf{Functions for numbers}: There are also many functions you can use with numbers. Two nice ones are round() and number\_format().
            \begin{itemize}
                \item \textbf{round()}: rounds a decimal to either the nearest integer or to a specified number of digits. Round(\$n,2) will give 2 digits to the right of the decimal point.
                \item \textbf{number\_format()}: makes a number appear in the more commonly written format (adding commas where appropriate) and you can specify digits to the right of the decimal point.
            \end{itemize}
        \item \textbf{Super Global Variables}: Super Global Variables - pre defined in php, these are always present and their values available to all your scripts
            \begin{itemize}
                \item \textbf{\$\_GET}: contains any variables provided to a script through the GET method
                \item \textbf{\$\_POST}: contains any variables provided to a script through the POST method
                \item \textbf{\$\_COOKIE}: contains any variables provided to a script through a cookie
                \item \textbf{\$\_FILES}: contains any variables provided to a script through file uploads
                \item \textbf{\$\_SERVER}: contains information such as
                \item \textbf{\$\_ENV}: contains any variables provided to a script as part of the server environment
                \item \textbf{\$\_REQUEST}: contains any variables provided to a script via any user input mechanism
                \item \textbf{\$\_SESSION}: contains any variables that are currently registered to a session
            \end{itemize}
        \item \textbf{php operators}: 
            \begin{itemize}
                \item \textbf{Arithmetic Operators} (used to perform mathematical calculations)
                    \begin{itemize}
                        \item \texttt{+} (Addition): Adds two values, e.g., \texttt{5 + 3} results in \texttt{8}.
                        \item \texttt{-} (Subtraction): Subtracts one value from another, e.g., \texttt{5 - 3} results in \texttt{2}.
                        \item \texttt{*} (Multiplication): Multiplies two values, e.g., \texttt{5 * 3} results in \texttt{15}.
                        \item \texttt{/} (Division): Divides one value by another, e.g., \texttt{15 / 3} results in \texttt{5}.
                        \item \texttt{\%} (Modulus): Returns the remainder of division, e.g., \texttt{5 \% 2} results in \texttt{1}.
                        \item \texttt{**} (Exponentiation): Raises a number to the power of another, e.g., \texttt{2 ** 3} results in \texttt{8}.
                    \end{itemize}

                \item \textbf{Assignment Operators} (used to assign values to variables)
                    \begin{itemize}
                        \item \texttt{=} (Basic assignment): Assigns a value to a variable, e.g., \texttt{\$x = 5}.
                        \item \texttt{+=} (Addition assignment): Adds and assigns a value, e.g., \texttt{\$x += 5} is equivalent to \texttt{\$x = \$x + 5}.
                        \item \texttt{-=} (Subtraction assignment): Subtracts and assigns a value, e.g., \texttt{\$x -= 5} is equivalent to \texttt{\$x = \$x - 5}.
                        \item \texttt{*=} (Multiplication assignment): Multiplies and assigns a value, e.g., \texttt{\$x *= 5} is equivalent to \texttt{\$x = \$x * 5}.
                        \item \texttt{/=} (Division assignment): Divides and assigns a value, e.g., \texttt{\$x /= 5} is equivalent to \texttt{\$x = \$x / 5}.
                        \item \texttt{\%=} (Modulus assignment): Takes modulus and assigns a value, e.g., \texttt{\$x \%= 5} is equivalent to \texttt{\$x = \$x \% 5}.
                    \end{itemize}

                \item \textbf{Comparison Operators} (used to compare two values)
                    \begin{itemize}
                        \item \texttt{==} (Equal): Checks if values are equal, e.g., \texttt{5 == 5} results in \texttt{true}.
                        \item \texttt{===} (Identical): Checks if values and types are identical, e.g., \texttt{5 === "5"} results in \texttt{false}.
                        \item \texttt{!=} (Not equal): Checks if values are not equal, e.g., \texttt{5 != 3} results in \texttt{true}.
                        \item \texttt{<>} (Not equal): Alternative to \texttt{!=}.
                        \item \texttt{!==} (Not identical): Checks if values and types are not identical, e.g., \texttt{5 !== "5"} results in \texttt{true}.
                        \item \texttt{>} (Greater than): Checks if one value is greater than another, e.g., \texttt{5 > 3} results in \texttt{true}.
                        \item \texttt{<} (Less than): Checks if one value is less than another, e.g., \texttt{3 < 5} results in \texttt{true}.
                        \item \texttt{>=} (Greater than or equal to): Checks if one value is greater than or equal to another, e.g., \texttt{5 >= 5} results in \texttt{true}.
                        \item \texttt{<=} (Less than or equal to): Checks if one value is less than or equal to another, e.g., \texttt{3 <= 5} results in \texttt{true}.
                        \item \texttt{<=>} (Spaceship operator): Returns \texttt{-1} if left is less, \texttt{0} if equal, \texttt{1} if greater, e.g., \texttt{5 <=> 3} results in \texttt{1}.
                    \end{itemize}

                \item \textbf{Logical Operators} (used to combine boolean expressions)
                    \begin{itemize}
                        \item \texttt{\&\&} (And): Returns \texttt{true} if both operands are true, e.g., \texttt{true \&\& false} results in \texttt{false}.
                        \item \texttt{||} (Or): Returns \texttt{true} if either operand is true, e.g., \texttt{true || false} results in \texttt{true}.
                        \item \texttt{!} (Not): Inverts the boolean value, e.g., \texttt{!true} results in \texttt{false}.
                        \item \texttt{xor} (Exclusive or): Returns \texttt{true} if only one operand is true, e.g., \texttt{true xor false} results in \texttt{true}.
                    \end{itemize}

                \item \textbf{Increment/Decrement Operators} (used to increase or decrease a variable's value by 1)
                    \begin{itemize}
                        \item \texttt{++} (Increment): Increases the variable's value by 1, e.g., \texttt{\$x++}.
                        \item \texttt{--} (Decrement): Decreases the variable's value by 1, e.g., \texttt{\$x--}.
                    \end{itemize}

                \item \textbf{String Operators} (used to work with strings)
                    \begin{itemize}
                        \item \texttt{.} (Concatenation): Joins two strings, e.g., \texttt{"Hello " . "world"} results in \texttt{"Hello world"}.
                        \item \texttt{.=} (Concatenation assignment): Appends a string to a variable, e.g., \texttt{\$x .= "world"}.
                    \end{itemize}

                \item \textbf{Array Operators} (used to work with arrays)
                    \begin{itemize}
                        \item \texttt{+} (Union): Combines two arrays, e.g., \texttt{\$a + \$b} merges arrays \$a and \$b.
                        \item \texttt{==} (Equality): Checks if arrays have the same key-value pairs, e.g., \texttt{\$a == \$b}.
                        \item \texttt{===} (Identity): Checks if arrays have identical key-value pairs in the same order and of the same types.
                        \item \texttt{!=} (Inequality): Checks if arrays do not have the same key-value pairs.
                        \item \texttt{!==} (Non-identity): Checks if arrays are not identical in either key-value pairs, order, or types.
                    \end{itemize}

                \item \textbf{Type Operators} (used to check or specify types)
                    \begin{itemize}
                        \item \texttt{instanceof}: Checks if an object is an instance of a specific class, e.g., \texttt{\$object instanceof ClassName}.
                    \end{itemize}

                \item \textbf{Error Control Operator} (used to suppress errors)
                    \begin{itemize}
                        \item \texttt{@} (Error suppression): Suppresses error messages for an expression, e.g., \texttt{@file\_get\_contents("nonexistentfile.txt")}.
                    \end{itemize}
            \end{itemize}
        \item \textbf{Comments}: Like most computer languages, php allows you to add explanations to the code in the form of comments
            These comments are ignored by the php parser
            \bigbreak \noindent 
            Comments should be added whenever necessary to explain code that is hard to follow
            \bigbreak \noindent 
            To insert a comment in a single line of php code, you preface the comment with either a pound symbol (\#) or two forward slashes (//)
        \item \textbf{Define()}:
            use php's builtin define() function
            \bigbreak \noindent 
            \begin{bashcode}
                define("YOUR_CONSTANT_NAME", value)
            \end{bashcode}
            You can set your constant to a number, a string, or a boolean.
            \bigbreak \noindent 
            By convention, use all caps for name of a constant.
            \bigbreak \noindent 
            You don't use a \$ when accessing a constant.
        \item \textbf{If statements}: If statements in php are the same as c++
            \bigbreak \noindent 
            \begin{bashcode}
                if (expression) {
                    ...
                } else if (expression) {
                    ...
                } else {
                    ...
                }
            \end{bashcode}
        \item \textbf{Switch-Case}: Switch statements are also the same as c++
            \bigbreak \noindent 
            \begin{bashcode}
                switch (expression) {
                    case 1:
                        ...
                        break;
                    case 2: 
                        ...
                        break;
                    default: // If no break was encountered
                        ...
                        break;
                }
            \end{bashcode}
        \item \textbf{For Loop}: Same as c++
            \bigbreak \noindent 
            \begin{bashcode}
                for (init expression; test expression; modification expression) {
                    //code to execute
                }
            \end{bashcode}
            \bigbreak \noindent 
            \textbf{Note:} zero, an undefined variable or an empty string will all evaluate to false, all others will evaluate to true.
        \item \textbf{False evaluations}: In PHP, several values evaluate to false when used in a boolean context. 
            \begin{itemize}
                \item \textbf{Boolean:} false itself.
                \item \textbf{Integer:} 0 (zero).
                \item \textbf{Float:} 0.0 (zero as a floating-point number).
                \item \textbf{String:} An empty string "" and the string "0".
                \item \textbf{Array:} An empty array [].
                \item \textbf{NULL:} The NULL type.
                \item \textbf{Objects without methods or properties:} Objects that do not have any methods or properties defined in them may also evaluate to false when checked with empty().
            \end{itemize}
        \item \textbf{While and do while}: Same as c++
            \bigbreak \noindent 
            \begin{bashcode}
                while (expression) {
                    ...
                }

                do {
                    ...
                } while(expression);
            \end{bashcode}
        \item \textbf{Arrays}: There are two ways to define an array in php.
            \bigbreak \noindent 
            \begin{bashcode}
                $colors = array("red","green","blue");

                $colors[0] = "red";
                $colors[1] = "green";
                $colors[2] = "blue";
            \end{bashcode}
            \bigbreak \noindent 
            These are both numerically indexed arrays.
        \item \textbf{Associative arrays}: You can also have associative arrays which have named keys.
            \bigbreak \noindent 
            \begin{bashcode}
                $character = array(
                    "name" => "Monk",
                    "occupation" => "detective"
                );
            \end{bashcode}
            \bigbreak \noindent 
            You access an element of an associative array by using the key name rather than a number.
        \item \textbf{Array functions}: There are approximately 60 array functions built into php. Some important ones are
            \begin{itemize}
                \item count() and sizeof() return the number of elements in the array.
                \item foreach() steps through an array
                \item each() and list() usually appear together in the context of stepping through an array and returning keys and values
                \item reset() rewinds the pointer to the beginning of the array
                \item array\_push() adds elements at the end of an existing array
                \item array\_pop() removes and returns the last element in an existing array
                \item array\_merge() combines two or more existing arrays
                \item shuffle() randomizes the elements of a given
            \end{itemize}
        \item \textbf{Including Files}:
            When developing more than a single home page for the Internet, you probably want the pages to have a common look and feel
            \bigbreak \noindent 
            To make this possible, php has provided a method called include files
            \bigbreak \noindent 
            These files let you incorporate common artwork, contact information, and menu and link options into your Web pages with a minimum of code
            \bigbreak \noindent 
            \begin{bashcode}
                // vars.php
                <?php
                    $color = 'green';
                    $fruit = 'apple';
                ?>

                // test.php
                <?php
                    echo "A $color $fruit"; // Output - A
                    include 'vars.php';
                    echo "A $color $fruit"; // Output - A green apple
                ?> 
            \end{bashcode}
        \item \textbf{Try except}: Same as c++
            \bigbreak \noindent 
            \begin{cppcode}
            try {
                ...
            }  catch (exceptionname e) {
                ...
            }
            \end{cppcode}
        \item \textbf{Creating objects with the new keyword}: In PHP, the new keyword is used to create an instance of a class, which is known as an object. When you use new, PHP allocates memory for the object and calls its constructor method (if defined) to initialize it
            \bigbreak \noindent 
            \begin{bashcode}
                $object = new ClassName();
            \end{bashcode}
        \item \textbf{Scope resolution operator (::)}: In PHP, the :: operator is called the Scope Resolution Operator, also known as the Paamayim Nekudotayim (Hebrew for "double colon"). It is used to access static, constant, or overridden properties and methods of a class, without needing to instantiate the class.
        \item \textbf{arrow operator (->)}: In PHP, the -> operator is used to access properties and methods of an object instance. It allows you to call non-static methods and access non-static properties on an instantiated object.
        \item \textbf{print\_r}: The print\_r function in PHP is used for outputting human-readable information about a variable. It's particularly useful for displaying the contents of arrays and objects. Unlike echo or print, which are mainly used to display strings, print\_r provides a structured format for complex data types, making it easier to visualize the nested structure of an array or object.
            \bigbreak \noindent 
            \begin{cppcode}
                print_r(mixed $expression, bool $return = false): mixed
            \end{cppcode}
            \bigbreak \noindent 
            \begin{cppcode}
                $array = array("name" => "John", "age" => 30, "city" => "New York");
                print_r($array);
            \end{cppcode}
        \item \textbf{foreach()}:
            \bigbreak \noindent 
            \begin{cppcode}
                $a = array(1 => 1 ,2,3);
                foreach($a as $item) {
                    echo $item;

                }

                foreach($a as $key => $value) {
                    echo "key: $key    value: $value";
                    echo "<br/>";
                }
            \end{cppcode}
        \item \textbf{Functions}: Functions in php are made with the \textit{function} keyword
            \bigbreak \noindent 
            \begin{cppcode}
                function functionName($param1, $param2) {
                    // Code to execute
                    return $result; // Optional
                }
            \end{cppcode}
        \item \textbf{Anonymous functions}: Functions with no name, often used as variables or passed as arguments to other functions.
            \bigbreak \noindent 
            \begin{cppcode}
                $sayHello = function($name) {
                    return "Hello, $name!";
                };

                echo $sayHello("Bob"); // Outputs: Hello, Bob!
            \end{cppcode}
        \item \textbf{Arrow Functions}: Shorter syntax for one-liner anonymous functions.
            \bigbreak \noindent 
            \begin{phpcode}
                $multiply = fn($a, $b) => $a * $b;
                echo $multiply(2, 3); // Outputs: 6
            \end{phpcode}
    \item \textbf{Optional Parameters:} Parameters with default values, which are used if no argument is provided.
        \bigbreak \noindent 
        \begin{cppcode}
            function greet($name = "Guest") {
                return "Hello, $name!";
            }
            echo greet(); // Outputs: Hello, Guest!
        \end{cppcode}
    \item \textbf{Passing by reference}: By default, arguments are passed by value, meaning changes within the function don't affect the original variable. You can pass by reference using \&.
        \bigbreak \noindent 
        \begin{cppcode}
            function addOne(&$number) {
                $number += 1;
            }
            $num = 5;
            addOne($num);
            echo $num; // Outputs: 6
        \end{cppcode}







    \end{itemize}

    \pagebreak 
    \subsection{SQL via PHP: PDO}
    \begin{itemize}
        \item \textbf{Why SQL via PHP?}: Can provide an interface for the user that does not require them to worry about database design specifics.
            \bigbreak \noindent 
            Although there are other ways to provide this interface, the web-based interface is very common, and PHP is a common and relatively easy way of making it work.
        \item \textbf{Application Programming Interface (API)}: In order for our PHP application to interface with our DBMS, we will need to use an appropriate API (Application Programming Interface) An API is the set of function calls and other resources that are provided to allow you to interface with a given application via your program code.
        \item \textbf{Which API?}: Even for a given DBMS, there can be many API's present. PHP has been around for a while, so many things have evolved and died out. Many of the API's still work. Some work but are considered deprecated, and others are no longer supported at all. In this class, we will be working with the PDO library.
        \item \textbf{PDO Library}: The PHP Data Objects (PDO) library is an object oriented API to connect PHP to SQL servers. It allows you to use a common interface to interact with many different DBMS programs.
            \bigbreak \noindent 
            It supports most of the popular relational DBMSes, including MySQL, Postgresql, and SQLite, in a fairly transparent way, so it is more portable than using the other, DBMS specific API's.
            \textbf{Note:} Because PDO is object oriented, it requires at least version 5 of PHP. If you need to use a lower version, you'll need to look into the other API's available.
        \item \textbf{PHP Data Objects}: The PDO library is object oriented. That means that our interactions with the database will be done using objects and their members. There are three basic objects that we will be concerned with:
            \begin{itemize}
                \item \textbf{PDO}: this object handles the connection to the database
                \item \textbf{PDOStatement}: this object handles prepared statements, and is used to work with result sets
                \item \textbf{PDOException}: this object is used to store information on errors that have occurred
            \end{itemize}
        \item \textbf{Specifying a Database with a DSN}:  Although, once properly initialized, PDO should function the same for any DBMS, you need to properly initialize it by telling it which type of server you are connecting to. To do this, you need to make a DSN string.
            \begin{center}
                \begin{tabular}{c|c}
                    DBMS &DSN Format \\
                    \hline
                    MySQL& mysql:host=HOSTNAME;dbname=DBNAME \\
                    Postgresql &pgsql:host=HOSTNAME;dbname=DBNAME \\
                    SQLite 3 &sqlite:PATHTODB \\
                    SQLite 2 &sqlite2:PATHTODB
                \end{tabular}
            \end{center}
            \bigbreak \noindent 
            MariaDB will use the MySQL interface.
        \item \textbf{Initializing PDO}: Once you've chosen the DBMS you'll be using and you've chosen an appropriate DSN string, you can use that DSN to construct an instance of the PDO class. This object represents a connection to the database specified by the DSN
            \bigbreak \noindent 
            \begin{bashcode}
                try { // if something goes wrong, an exception is thrown
                    $dsn = "mysql:host=courses;dbname=z123456";
                    $pdo = new PDO($dsn, $username, $password);
                }
                catch(PDOexception $e) { // handle that exception
                    echo "Connection to database failed: " . $e->getMessage();
                }
            \end{bashcode}
        \item \textbf{Using PDO to Talk to DBMS}: The PDO library provides three basic ways of running queries for a database, once connected:
            \begin{itemize}
                \item the PDO::exec() function is used to execute an SQL query that does not return a result (INSERT, UPDATE, etc.)
                \item the PDO::query() function is used to execute an SQL query that will return a result (SELECT)
                \item the PDO::prepare() function should be used when constructing a query from user input.
            \end{itemize}
        \item \textbf{Using PDO::exec()}: PDO::exec() is used to run a query that does not return any results. Instead of returning a result set, it will tell you how many rows were affected by your query.
            \bigbreak \noindent 
            \begin{bashcode}
                // Three examples follow.
                $n = $pdo->exec("INSERT INTO Students (FName) VALUES ('Victor');");
                $n = $pdo->exec("UPDATE Students SET LName='Husky' WHERE FName='Victor';");
                $n = $pdo->exec("DELETE FROM OldJunk;");
            \end{bashcode}
        \item \textbf{Using PDO::query()}: PDO::query() is used to run a query that does return a result. The result set is returned as a PDOStatement object.
            \bigbreak \noindent 
            \begin{bashcode}
                # Defining $sql as the query you'd like to run; here's one from classicmodels
                $sql = "SELECT phone FROM Customer;";

                # Run the query - the results are stored into the $result object on success
                $result = $pdo->query($sql);
            \end{bashcode}
        \item \textbf{Using PDO::prepare()}: The third option is to use the PDO::prepare() command. This is useful for situations where the same query is run multiple times in the same script, and can also help you to avoid some SQL Injection issues. Once prepare() succeeds, you run the query with the execute() method of the PDOStatement returned by prepare()
            \bigbreak \noindent 
            You can use a colon before a value name in your query to denote where the execute statement will insert the value of the given name:
            \bigbreak \noindent 
            \begin{bashcode}
                <?php
                    # Notice that we use :color below in our SQL template
                    $sql = 'SELECT name, color, calories
                    FROM fruit
                    WHERE calories < :calories AND color = :color';

                    $prepared = $pdo->prepare($sql, array(PDO::ATTR_CURSOR => PDO::CURSOR_FWDONLY));
                    # The value associated with the :color key will be used when executing

                    $success = $prepared->execute(array(':calories' => 150, ':color' => 'red'));
                    # if($success==true) then $prepared will be ready to be used as a result set
                    # with fetch() or fetchAll() -- just like the object returned by query()
                ?>
            \end{bashcode}
            \bigbreak \noindent 
            It is also possible to use a ? in your query as a positional parameter.
            \bigbreak \noindent 
            \begin{bashcode}
                <?php
                    # Execute a prepared statement by passing an array of values
                    $prepared = $pdo->prepare('SELECT name, color, calories
                                            FROM fruit
                                            WHERE calories < ? AND color = ?');

                    # Here we execute the query twice with different parameters.
                    # The ?'s will be replaced with the values in the array specified,
                    # in the order they are specified.
                    $prepared->execute(array(150, 'red'));
                    $red = $prepared->fetchAll();
                    $prepared->execute(array(175, 'yellow'));
                    $yellow = $prepared->fetchAll();
                ?>
            \end{bashcode}
        \item \textbf{Named placeholders}: In PDO (PHP Data Objects), named placeholders are a way to represent dynamic values in SQL statements using identifiable names instead of anonymous ? placeholders. Named placeholders make SQL statements easier to read and maintain, especially when there are multiple variables or parameters.
            \bigbreak \noindent 
            A named placeholder is written as :name in an SQL query, where name is an identifier you choose. When you prepare the SQL statement, you can bind actual values to these placeholders by their names, allowing you to execute the query with specific data values.
        \item \textbf{anonymous placeholders}: In PDO (PHP Data Objects), anonymous placeholders (also called positional or unnamed placeholders) are represented by question marks ? in SQL statements. These placeholders are used to represent values that will be dynamically bound to the SQL query at execution time.
            \bigbreak \noindent 
            When using anonymous placeholders, each placeholder corresponds to a specific value based on its position in the SQL query. When you prepare and execute the query, you provide an array of values that will replace each ? in the order they appear.
        \item \textbf{Execute}: In PHP's PDO (PHP Data Objects) extension, the execute() method is used to run a prepared SQL statement with specific values for placeholders. It's part of the PDOStatement class and is essential for safely executing queries that include dynamic data, protecting against SQL injection attacks.
            \bigbreak \noindent 
            After preparing a statement with PDO::prepare(), you call execute() to supply any values for placeholders (either named or anonymous) and run the query.
        \item \textbf{Dealing with result sets}: Once you have a result set (stored in a PDOStatement from query() you can use its PDOStatement::fetch() or PDOStatement::fetchAll() methods to get the data returned.
            \bigbreak \noindent 
            If you would like to work on one row at a time, as if using the mysql\_fetch\_array() function from the original MySQL API, use fetch().
            \bigbreak \noindent 
            To grab all of the rows at the same time, use fetchAll().
            \bigbreak \noindent 
            \begin{bashcode}
                <?php
                    # FETCH_BOTH means that you will get both position indices and the column names
                    # as keys in the array returned
                    $row = $result->fetch(PDO::FETCH_BOTH);

                    # this returns all of the rows at once in a two-dimensional array
                    $allrows = $result->fetchAll();
                ?>
            \end{bashcode}
        \item \textbf{Handling Errors}: As of the time of this writing, there are three modes PDO can use to handle errors.
            \begin{itemize}
                \item \textbf{PDO::ERRMODE\_SILENT}: the default mode. PDO will set the error code for you to inspect using the errorCode and errorInfo methods on your PDO and PDOStatement objects
                \item \textbf{PDO::ERRMODE\_WARNING}: in addition to setting the error code, emits a traditional E\_WARNING message. Use for debugging/testing when you just want to see what problems occurred without interrupting the flow of the application.
                \item \textbf{PDO::ERRMODE\_EXCEPTION}: in addition to setting the error code, also throw PDOException when an error occurs.
            \end{itemize}
            \bigbreak \noindent 
            You can set which one you'd like PDO to use with the setAttribute method of the PDO object you're using, as below:
            \bigbreak \noindent 
            \begin{bashcode}
                $pdo->setAttribute(PDO::ATTR_ERRMODE, PDO::ERRMODE_SILENT);
                $pdo->setAttribute(PDO::ATTR_ERRMODE, PDO::ERRMODE_WARNING);
                $pdo->setAttribute(PDO::ATTR_ERRMODE, PDO::ERRMODE_EXCEPTION);
            \end{bashcode}
        \item \textbf{Members of the PDO object}
            \bigbreak \noindent 
            \begin{cppcode}
                PDO {
                    // constructor has a bunch of optional parameters, check reference if needed
                    public __construct ( string $dsn [, string $username [, string $password [,
                    array $options ]]] )
                    public bool beginTransaction ( void )
                    public bool commit ( void )
                    public mixed errorCode ( void )
                    public array errorInfo ( void )
                    public int exec ( string $statement )
                    public mixed getAttribute ( int $attribute )
                    public static array getAvailableDrivers ( void )
                    public bool inTransaction ( void )
                    public string lastInsertId ([ string $name = NULL ] )
                    public PDOStatement prepare ( string $statement [, array $driver_options ] )
                    public PDOStatement query ( string $statement )
                    public string quote ( string $string [, int $parameter_type = PDO::PARAM_STR ] )
                    public bool rollBack ( void )
                    public bool setAttribute ( int $attribute , mixed $value )
                }
            \end{cppcode}
        \item \textbf{Members of the PDOStatement object}: 
            \bigbreak \noindent 
            \begin{cppcode}
            PDOStatement implements Traversable {
                /* Properties */
                readonly string $queryString;
                /* Methods */
                public bool bindColumn ( mixed $column , mixed &$param [, int $type [, int $maxlen [, mixed $driverdata ]]] )
                public bool bindParam ( mixed $parameter , mixed &$variable [,
                int $data_type = PDO::PARAM_STR [, int $length [, mixed $driver_options ]]] )
                public bool bindValue ( mixed $parameter , mixed $value [, int $data_type = PDO::PARAM_STR ] )
                public bool closeCursor ( void )
                public int columnCount ( void )
                public void debugDumpParams ( void )
                public string errorCode ( void )
                public array errorInfo ( void )
                public bool execute ([ array $input_parameters ] )
                public mixed fetch ([ int $fetch_style [, int $cursor_orientation = PDO::FETCH_ORI_NEXT [,
                int $cursor_offset = 0 ]]] )
                public array fetchAll ([ int $fetch_style [, mixed $fetch_argument [, array $ctor_args = array() ]]] )
                public mixed fetchColumn ([ int $column_number = 0 ] )
                public mixed fetchObject ([ string $class_name = "stdClass" [, array $ctor_args ]] )
                public mixed getAttribute ( int $attribute )
                public array getColumnMeta ( int $column )
                public bool nextRowset ( void )
                public int rowCount ( void )
                public bool setAttribute ( int $attribute , mixed $value )
                public bool setFetchMode ( int $mode )
            }
            \end{cppcode}
        \item \textbf{Members of the PDOException object}
            \bigbreak \noindent 
            \begin{cppcode}
                PDOException extends RuntimeException {
                    /* Properties */
                    public array $errorInfo ;
                    protected string $code ;
                    /* Inherited properties */
                    protected string $message ;
                    protected int $code ;
                    protected string $file ;
                    protected int $line ;
                    /* Inherited methods */
                    final public string Exception::getMessage ( void )
                    final public Throwable Exception::getPrevious ( void )
                    final public mixed Exception::getCode ( void )
                    final public string Exception::getFile ( void )
                    final public int Exception::getLine ( void )
                    final public array Exception::getTrace ( void )
                    final public string Exception::getTraceAsString ( void )
                    public string Exception::__toString ( void )
                    final private void Exception::__clone ( void )
                }
            \end{cppcode}




    \end{itemize}

    \pagebreak 
    \subsection{Transactions and concurrency control}
    \begin{itemize}
        \item \textbf{Transaction}: A logical unit of work, a unit of program execution that access and possibly updates various data items
            \bigbreak \noindent 
            is initiated by user program written in a high-level data-manipulation language like  SQL, COBOL, C, C++, or Java
            \bigbreak \noindent 
            Transactions are delimited by statements such as \textit{begin transaction} \textit{end transaction} or \textit{COMMIT} or \textit{ROLLBACK}
        \item \textbf{Sample Transaction}: 
            \bigbreak \noindent 
            \begin{cppcode}
                BEGIN TRANSACTION;
                    UPDATE ACC 123 { BALANCE := BALANCE - 100 };
                    IF (ANY ERROR) THEN
                        GO TO UNDO;
                    ENDIF;
                    UPDATE ACC 456 { BALANCE := BALANCE + 100 };
                    UPDATE ACC 456 { BALANCE := BALANCE + 100 };
                    IF (ANY ERROR) THEN
                        GO TO UNDO;
                    ENDIF;
                    COMMIT;
                    GO TO FINISH;
                UNDO:
                    ROLLBACK;
                FINISH:
                    RETURN;
            \end{cppcode}
        \item \textbf{ACID properties of transactions:}
            \begin{itemize}
                \item \textbf{Atomicity:} either all operations of the transaction are implemented properly or none are.
                \item \textbf{Consistency:} execution of a transaction in isolation preserves the consistency of the database (with no other transaction executing concurrently).
                \item \textbf{Isolation:} each transaction is unaware of other transactions executing concurrently in the system.
                \item \textbf{Durability:} after a transaction completes successfully, the changes it has made to the database persist, even if the system fails.
            \end{itemize}
        \item \textbf{Transaction States}: In the absence of any failures, all transactions complete successfully.
            \bigbreak \noindent 
            However, there is not always an absence of any failures.
            \bigbreak \noindent 
            Thus we have different "states" in which a transaction may reside.
            \bigbreak \noindent 
            \fig{.5}{./figures/1.jpg}
            \begin{itemize}
                \item \textbf{Active:} the initial state and one in which the transaction stays while it is executing.
                \item \textbf{Partially committed:} a state after the final statement has been executed.
                \item \textbf{Failed:} the state after the discovery that normal execution can no longer proceed.
                \item \textbf{Aborted:} the state after the transaction has been rolled back and the database restored to its condition prior to the start of the transaction.
                \item \textbf{Committed:} the state after successful execution.
            \end{itemize}
        \item \textbf{Committed Transaction}:
            A transaction is considered committed when it has performed updates that transforms the database into a new consistent state.
            \bigbreak \noindent 
            Once a transaction is committed, its effects cannot be undone by a system failure.
            \bigbreak \noindent 
            Only way to undo a committed transaction is to execute a compensating transaction.
            \bigbreak \noindent 
            However it is not always possible to create a compensating transaction.
        \item \textbf{Failed \& Aborted Transaction}: When a transaction cannot be completed (due to some kind of system failure), the transaction must be "rolled back".
            \bigbreak \noindent 
             It also enters the aborted state where the system has two options:
             \begin{itemize}
                 \item Restart the transaction (but only if aborted due to some hardware or software error).
                 \item Kill the transaction which is usually done due to some internal logical error.
             \end{itemize}
        \item \textbf{Concurrent Executions of Transactions}: Concurrent executions are good:
            \begin{itemize}
                \item Improved throughput and resource utilization
                \item Reduced waiting time
            \end{itemize}
             In transaction processing multiple transactions are allowed to run concurrently.
             \bigbreak \noindent 
             Updating within concurrent transactions causes several complications with consistency of the data.
        \item \textbf{Execution sequences}: Represent the chronological order in which instructions are executed in the system
        \item \textbf{Serial schedules}: consists of a sequence of instructions from various transactions where the instructions belonging to one single transaction appear together in that schedule
        \item \textbf{Concurrency control}: Concurrency control: the task of ensuring that any schedule that gets executed will leave the database in a consistent state
            \bigbreak \noindent 
            The concurrency-control component of the DBMS carries out setting up the schedules to ensure consistency during the execution of multiple transactions
        \item \textbf{Serializability}: a schedule that is equivalent to a serial schedule.
            \bigbreak \noindent 
            In discussing serializability, only two operations are important
            \begin{itemize}
                \item read(Q)
                \item write(Q)
            \end{itemize}
             We also assume that the transaction may perform an arbitrary sequence of operations on the copy of Q between the read and write operations
        \item \textbf{Concurrency control introduction}: Fundamental Property of a transaction is isolation
            \bigbreak \noindent 
            To ensure that isolation property is preserved when several transactions are running concurrently, the system controls the interaction among the concurrent transactions
            \bigbreak \noindent 
            These schemes are called concurrency control
        \item \textbf{Lock-Based Protocols}: One way to ensure serializability
            \begin{itemize}
                \item Require data items be accessed in a mutually exclusive manner
                \item That is when one transaction is accessing the data item, no other transaction can modify that data item
                \item Common method to implement is that transaction must hold a lock on an item
            \end{itemize}
        \item \textbf{Modes in which data item may be locked}:
            \begin{itemize}
                \item \textbf{Shared}: If a transaction T1 has obtained a shared-mode lock on item Q, then T1 can read, but cannot write, Q
                \item \textbf{Exclusive}: If a transaction T1 has obtained an exclusivemode lock on item Q, then T1 can both read or write Q
            \end{itemize}
            \bigbreak \noindent 
            Every transaction is required to request a lock in an appropriate mode on each data item
            \bigbreak \noindent 
            Request is made to the concurrencycontrol manager
            \bigbreak \noindent 
            Concurrency-control manager must grant the lock to the transaction before it can proceed.
        \item \textbf{Compatibility Function}: Given set of lock modes can define compatibility function
            \begin{itemize}
                \item A \& B represent arbitrary lock modes
                \item Transaction T1 request a lock of mode A on item Q
                \item Transaction T2 currently holds a lock of mode B
                \item If T1 can be granted a lock on Q immediately in spite of the presence of the mode B lock, then mode A is compatible with mode B
            \end{itemize}
            \bigbreak \noindent 
            Can represent with a Lock-compatibility matrix
            \bigbreak \noindent 
            \begin{center}
                \begin{tabular}{p{4cm}|p{4cm}|p{4cm}}
                    & shared & exclusive \\
                    \hline
                    shared & true & false \\ 
                    exclusive & false & false
                \end{tabular}
            \end{center}
            \bigbreak \noindent 
            When a transaction requests a lock that is incompatible, it enters a wait state until all incompatible locks have been released
            \bigbreak \noindent 
            Transactions cannot execute until concurrency-control manager grants the requested locks
        \item \textbf{Deadlock}: Locking can lead to an undesirable situation where no transaction can proceed with normal execution. This situation is called deadlock
            \bigbreak \noindent 
            When this occurs
            \begin{itemize}
                \item The system must roll back one of the transactions
                \item Unlocking transactions until execution can be continued
            \end{itemize}
            \bigbreak \noindent 
            If locking is not used, or if a data item is unlocked as soon as possible after reading or writing, inconsistent states may occur
            \bigbreak \noindent 
            On the other hand, if a data item is not unlocked before requesting a lock on some other data item deadlocks may occur
            \bigbreak \noindent 
            In general, deadlocks are a necessary evil associated with locking which is necessary to avoid inconsistent states
            \bigbreak \noindent 
            Deadlocks are preferable to inconsistent states
            \begin{itemize}
                \item since they can be handled via rolling back transactions
                \item inconsistent states cannot be handled by database
            \end{itemize}
        \item \textbf{Locking Protocol}: a set of rules that each transaction must follow
            \bigbreak \noindent 
            Indicates when a transaction may lock or unlock each data item
            \bigbreak \noindent 
            A schedule is legal under a given locking protocol if it follows the rules
            \bigbreak \noindent 
            A locking protocol ensures conflict serializability if and only if all legal schedules are conflict serializable
        \item \textbf{Granting of Locks}: Grant can take place if
            \begin{itemize}
                \item a transaction requests a lock on a data item in some mode
                \item and no other transaction has a lock on the same data item in a conflicting mode
            \end{itemize}
            \bigbreak \noindent 
             Care must be take to avoid certain situations
             \bigbreak \noindent 
             Suppose transaction T2 has a shared-mode lock on a data item
             \bigbreak \noindent 
             Transaction T1 requests an exclusivemode lock on the data item
             \bigbreak \noindent 
             Clearly, T1 has to wait for T2 to release the shared-mode lock
             \bigbreak \noindent 
             Meanwhile, transaction T3 may request a shared-mode lock on the same data item 
             \bigbreak \noindent 
             The lock request is compatible with the lock granted to T2 so T3 may be granted the shared-mode lock
             \bigbreak \noindent 
             At this point, T2 releases the lock but T1 has to still wait for T3 to finish
             \bigbreak \noindent 
             But again, there are other transactions Ti, that requests a shared-mode lock
             \bigbreak \noindent 
             In fact it is possible that there is a sequence of transactions that each requests a shared-mode lock on the data item and T1 NEVER gets the exclusive-mode lock
             \bigbreak \noindent 
             T1 is then said to be starved
        \item \textbf{Avoiding starvation of transactions}: T1 requests a lock on a data item Q in some mode M. the lock is granted provided that
            \begin{itemize}
                \item there is no other transaction holding a lock on Q in a mode that conflicts with M
                \item there is no other transaction that is waiting for a lock on Q and that made its lock request before T1
            \end{itemize}
        \item \textbf{Two-Phase Locking Protocol}: A protocol that ensures serializability is the two-phase locking protoco
            \bigbreak \noindent 
             Each transaction issues lock and unlock requests in two phases
             \begin{enumerate}
                 \item \textbf{Growing phase}: a transaction may obtain locks but may not release any lock
                 \item \textbf{Shrinking phase}: a transaction may release locks but may not obtain any new locks
             \end{enumerate}
             \bigbreak \noindent 
             Initially a transaction is in the growing phase
             \bigbreak \noindent 
            Once the transaction releases a lock, it enters shrinking phase and cannot request more locks
            \bigbreak \noindent 
            The Two-phase locking protocol
            \begin{enumerate}
                \item Ensures conflict serializability
                \item Does not ensure freedom from deadlock
            \end{enumerate}
        \item \textbf{Variations of two-phase locking protocol}:
            \begin{enumerate}
                \item \textbf{Strict two-phase locking protocol}: requires not only that locking be two phase, but that all exclusive-mode locks taken by a transaction be held until that transaction commits
                \item \textbf{Rigorous two-phase locking protocol}: requires that all locks be held until the transaction commits
            \end{enumerate}
        \item \textbf{Refinement of basic two-phase locking protocol}: lock conversions are allowed
            \begin{itemize}
                \item a mechanism is allowed for upgrading a shared lock to an exclusive lock
                \item a mechanism is allowed for downgrading an exclusive lock to a shared lock
            \end{itemize}
            \bigbreak \noindent 
            Strict two-phase and rigorous twophase locking (with lock conversions) are extensively used in DBMSs
            \bigbreak \noindent 
            A simple but widely used scheme automatically generates the appropriate lock and unlock instructions for a transaction on the basis of read and write requests
            \bigbreak \noindent 
            When a transaction T1 issues a read(Q), the system issues a lock-S(Q) instruction followed by the read(Q) instruction
            \bigbreak \noindent 
            When T1 issues a write(Q) operation, the system checks to see whether T1 already holds a shared lock on Q.
            \begin{itemize}
                \item If yes, system issues an upgrade(Q) followed by a write(Q)
                \item If no, system issues a lock-X(Q) followed by a write(Q)
            \end{itemize}
            \bigbreak \noindent 
            All locks obtained by a transaction are unlocked after that transaction commits or aborts
        \item \textbf{Other Locking Protocols}:
            \begin{itemize}
                \item \textbf{Graph-based protocols}
                \item \textbf{Timestamp-based protocols}
                \item \textbf{Validation-based protocols}: majority of transactions are read-only
                \item \textbf{Multiversion schemes}: each write(Q) creates a new version of Q
            \end{itemize}
        \item \textbf{Deadlock Handling}: Deadlock state... there exists a set of transactions such that every transaction in the set is waiting for another transaction in the set
            \bigbreak \noindent 
            Two principal methods for dealing with deadlocks
            \begin{enumerate}
                \item deadlock prevention
                \item deadlock detection and deadlock recovery
            \end{enumerate}
        \item \textbf{Deadlock prevention}: two approaches
            \begin{enumerate}
                \item \textbf{one approach:} ensure that no cyclic waits can occur by
                    \begin{itemize}
                        \item ordering the requests for locks
                        \item or requiring all locks be acquired together 
                    \end{itemize}
                \item \textbf{second approach:} impose an ordering of all data items and require that a transaction lock data items only in a sequence consistent with the ordering
            \end{enumerate}
        \item \textbf{Deadlock detection and recovery}: an algorithm that examines the state of the system is invoked periodically to see if a deadlock has occurred, if one has, then the system must recover
            \bigbreak \noindent 
            To recover from a deadlock the system must
            \begin{itemize}
                \item maintain information about the current allocation of data items to transactions as well as any outstanding data item requests
                \item provide an algorithm that uses this information to determine whether the system has entered a deadlock state
                \item recover from the deadlock when the detection algorithm determines that a deadlock exists.
            \end{itemize}
    \end{itemize}

    \pagebreak 
    \subsection{Mariadb in C++}
    \begin{itemize}
        \item \textbf{Motivation}: We have already looked at how to interact with a MariaDB Database from PHP using the PHP Data Objects (PDO) API. Now we will look at the API you can use to do the same thing in C/C++ programs.
            \bigbreak \noindent 
            The MariaDB API is implemented in C, so it is provided as a set of functions to call, as opposed to a set of classes, as may be found in more object oriented programming
        \item \textbf{Header file}: To use any of these functions, you need to include the appropriate header file.
            \bigbreak \noindent 
            \begin{cppcode}
                #include <mysql/mysql.h>
            \end{cppcode}
            \bigbreak \noindent 
            This file provides declarations for the functions and new types that make up the MariaDB API. Later, during the linking stage, you will specify the mariadb library, where the implementations of the functions are found.
        \item \textbf{Some new variable types}: The MariaDB library defines some new types. Some of them are data structures that store more complicated data, which is dealt with using the library functions, others are just a different way of referring to types you already know
            \begin{itemize}
                \item \textbf{MYSQL}: a data structure containing data about a connection to a MariaDB server
                \item \textbf{MYSQL\_RES}: a data structure that contains data about a result set
                \item \textbf{MYSQL\_ROW} a data structure that contains data about a single row from a result set
                \item \textbf{my\_ulonglong}: used to store large unsigned integers
            \end{itemize}
        \item \textbf{Handling errors}: Various functions in the MariaDB API have specific return codes for when an error has occurred. In other situations, it may not be clear.
            \bigbreak \noindent 
            When an error occurs, there are two functions you can use to find out what happened:
            \bigbreak \noindent 
            \begin{cppcode}
            unsigned int mysql_errno(MYSQL *mysql);
            const char *mysql_error(MYSQL *mysql);
            \end{cppcode}
            \bigbreak \noindent 
            \begin{itemize}
                \item mysql\_errno - returns a numeric error code that identifies the last error that occurred
                \item mysql\_error - returns a human-readable error message describing the last error that occurred.
            \end{itemize}
            The mysql parameter is a pointer to the connection object you would like to check the error for. We will discuss that later.
        \item \textbf{Library Initialization}: 
            \bigbreak \noindent 
            \begin{cppcode}
                int mysql_library_init(int argc, char **argv, char **groups)
            \end{cppcode}
            \bigbreak \noindent 
            This function initializes the MySQL library. It is generally only needed if you want your application to be
            threadsafe, but you should call it in your programs for this class anyway.
            \begin{itemize}
                \item \textbf{argc}: argument count, use 0 unless you have a reason not to
                \item \textbf{argv}: argument vector, use NULL unless you have a reason not to
                \item \textbf{groups}: NULL-terminated array of strings listing which options to read, use NULL unless you have a reason not to
            \end{itemize}
        \item \textbf{Library Deinitialization}:
            \bigbreak \noindent 
            \begin{cppcode}
                void mysql_library_end(void)
            \end{cppcode}
            \bigbreak \noindent 
            This function cleans up any memory allocated during the use of the MySQL library. You should make sure to call it after you are completely finished using the library to communicate with the DBMS.
        \item \textbf{Initializing a Connection Object}:
            \bigbreak \noindent 
            \begin{cppcode}
            MYSQL *mysql_init(MYSQL *mysql)
            \end{cppcode}
            \bigbreak \noindent 
            This function initializes (allocating memory as well, if needed) a MYSQL object, which is suitable for use with mysql\_real\_connect(). If mysql is NULL, then function will dynamically allocate memory for the object. If it is a pointer to an already allocated MYSQL object, it will set it up in that place. Either way, it will return a pointer to where the initialized object is. If there is an error, NULL will be returned.
            \bigbreak \noindent 
            \begin{itemize}
                \item \textbf{mysql}: a pointer to a MYSQL object to initialize, or NULL
            \end{itemize}
        \item \textbf{Establishing a DB Connection}:
            \bigbreak \noindent 
            \begin{cppcode}
                MYSQL *mysql_real_connect(MYSQL *mysql, const char *host,
                    const char *user, const char *passwd,
                    const char *db, unsigned int port,
                    const char *unix_socket, unsigned long client_flag)
            \end{cppcode}
            \bigbreak \noindent 
            Establishes a connection to the specified MySQL database.
            \begin{itemize}
                \item \textbf{mysql}: pointer to an existing MYSQL object, can be used to set options
                \item \textbf{host}: the hostname of the server to connect to item user - the username to authenticate with
                \item \textbf{passwd}: the password for the specified user
                \item \textbf{db}: the name of the database to use on that MySQL server
                \item \textbf{port}: set to zero unless you want to use a non-default port
                \item \textbf{unix\_socket}: set this to NULL
                \item \textbf{client\_flag}: set this to zero (can be used for special flags)
            \end{itemize}
            \bigbreak \noindent 
            Returns a pointer to a MYSQL connection object on success, NULL on failure.
        \item \textbf{Closing a DB Connection}:
            \bigbreak \noindent 
            \begin{cppcode}
            void mysql_close(MYSQL *mysql)
            \end{cppcode}
            \bigbreak \noindent 
            This function closes a previously opened connection. It also deallocates the connection handler pointed to by mysql if the handler was allocated automatically in either mysql\_init() or mysql\_connect().
            \bigbreak \noindent 
            \begin{itemize}
                \item \textbf{mysql}: a pointer to the MYSQL object for the connection to be closed
            \end{itemize}
        \item \textbf{Running SQL Queries}: After you have initialized the API and connected to the server, you can begin to run SQL queries on it. There are a couple of functions that are designed to do this.
            \begin{itemize}
                \item \textbf{mysql\_query()}: allows you to send a query as a null-terminated string
                \item \textbf{mysql\_real\_query()}: allows you to send a query as a string of a specified length
            \end{itemize}
        \item \textbf{Queries using null-terminated string}:
            \bigbreak \noindent 
            \begin{cppcode}
            int mysql_query(MYSQL *mysql, const char *stmt_str)
            \end{cppcode}
            \bigbreak \noindent 
            This function executes the SQL statement stored in the null-terminated string pointed to by stmt\_str. Normally the string must consist of a single SQL statment without a terminating semicolon. If you explicitly enable multiple-statement execution, it can contain several statements separated by semicolons. It is recommended not to do this, normally, because of the potential for SQL injection.
            \bigbreak \noindent 
            \begin{itemize}
                \item \textbf{stmt\_str}: a pointer to the beginning of a null-terminated string containing the SQL statement(s) to run
            \end{itemize}
            \bigbreak \noindent 
            Returns zero on success. Non-zero if an error has occurred.
        \item \textbf{Queries using string with length}:
            \bigbreak \noindent 
            \begin{cppcode}
                int mysql_real_query(MYSQL *mysql,
                        const char *stmt_str,
                        unsigned long length)
            \end{cppcode}
            \bigbreak \noindent 
            This function executes the SQL statement pointed to by stmt\_lstr, which is interpreted to be length bytes long. The use of a length as opposed to a terminating null character is what distinguishes this from the mysql\_lquery() function from before. This allows binary data, which may contain null characters as valid data, to be sent.
            \begin{itemize}
                \item \textbf{mysql}: pointer to the MySQL connection to run the query through
                \item \textbf{stmt\_lstr}: pointer to the beginning of the string containing the SQL statement(s)
                \item \textbf{length}: the length, in bytes, of the string, stmt\_lstr
            \end{itemize}
            \bigbreak \noindent 
            It returns zero on success, non-zero if an error has occurred.
        \item \textbf{Get information on a query's results}:  After a query is run, there may be a result set, which you would expect from a query with, for example, a SELECT statement. There are a couple of ways to access the results of such a query.
            \bigbreak \noindent 
            Other queries might not generate a result set, such as INSERT, UPDATE, or DELETE. There are some things you may want to know about them even though no data is returned
            \bigbreak \noindent 
            For queries that don't generate a result set, you may want to know some or all of the following
            \begin{itemize}
                \item If you want to know how many rows were returned in a result set, you can use mysql\_lnum\_lrows().
                \item If you want to know how many attributes (columns) exist in a result set, you can use mysql\_lfield\_lcount().
                \item If you want to know what value was chosen for an AUTO\_lINCREMENT field, you can use mysql\_linsert\_lid().
            \end{itemize}
        \item \textbf{How many rows were affected?}:
            \bigbreak \noindent 
            \begin{cppcode}
                my_ulonglong mysql_affected_rows(MYSQL *mysql)
            \end{cppcode}
            \bigbreak \noindent 
            This function can be called immediately after running a query on the server.
            \begin{itemize}
                \item \textbf{mysql}: a pointer to the server connection object you ran the query through
            \end{itemize}
            \bigbreak \noindent 
            Greater than zero indicates number of rows affected or retrieved. Zero indicates no rows. Errors are indicated with a -1 return code.
        \item \textbf{How many rows were returned?}:
            \bigbreak \noindent 
            \begin{cppcode}
                my_ulonglong mysql_num_rows(MYSQL_RES *result)
            \end{cppcode}
            \bigbreak \noindent 
            This function will return the number of rows in the specified result set. If you want to use this before doing things with your data, you will need to use mysql\_lstore\_lresult() instead of mysql\_luse\_lresult() to retrieve the result set.
            \begin{itemize}
                \item \textbf{result}: a pointer to the result set object to inquire about
            \end{itemize}
        \item \textbf{How many columns in the result?}:
            \bigbreak \noindent 
            \begin{cppcode}
                unsigned int mysql_field_count(MYSQL *mysql)
            \end{cppcode}
            \bigbreak \noindent 
            This function will return the number of columns were in the result set of the most recent query on the specified connection.
            \begin{itemize}
                \item \textbf{mysql}: a pointer to the connection object you just queried
            \end{itemize}
        \item \textbf{AUTO\_INCREMENT $\to$ What is the key of inserted row?}:
            \bigbreak \noindent 
            \begin{cppcode}
                my_ulonglong mysql_insert_id(MYSQL *mysql)
            \end{cppcode}
            \bigbreak \noindent 
            This function returns the value generated for an AUTO\_INCREMENT column by the previous INSERT or UPDATE statement.
            \begin{itemize}
                \item \textbf{mysql}: a pointer to the connection object we ran the query through
            \end{itemize}
            \bigbreak \noindent 
            Will return zero unless a value has been stored in an AUTO\_INCREMENT field. If multiple rows were affected, only the first one will be returned.
        \item \textbf{Getting info from result sets}: There are two basic functions that you can use to retrieve the data from a result set.
            \begin{itemize}
                \item The mysql\_lstore\_lresult() will function similarly to the way PDOStatement::fetchAll() function did, in that it will retrieve all the results at once.
                \item The mysql\_luse\_lresult() works more similarly to the PDOStatement::fetch() function, grabbing the results one row at a time.
                \item Neither of these two will directly give you the values in the row; that can be done with the mysql\_lfetch\_lrow() function.
                \item You should make a call to mysql\_lfree\_lresult() after finishing with the result sets, to free up memory used to store the results.
            \end{itemize}
        \item \textbf{Download all rows of the result set immediately}
            \bigbreak \noindent 
            \begin{cppcode}
                MYSQL_RES *mysql_store_result(MYSQL *mysql)
            \end{cppcode}
            \bigbreak \noindent 
            This function is used to retrieve the result set generated by a query. It will download the whole result set from the server immediately and store it in your program's memory. The values can then be obtained row-by-row with mysql\_lfetch\_lrow() or you can use mysql\_lrow\_lseek() to jump to specific rows.
            \begin{itemize}
                \item \textbf{mysql}: pointer to the connection object we used to run the query
            \end{itemize}
            \bigbreak \noindent 
            Returns a pointer to a result set structure if successful, NULL on error.
        \item \textbf{Set up to download one row at a time}:
            \bigbreak \noindent 
            \begin{cppcode}
            MYSQL_RES *mysql_use_result(MYSQL *mysql)
            \end{cppcode}
            \bigbreak \noindent 
            This function sets up a result set that will fetch its rows one at a time from the server. Individual rows are obtained with calls to mysql\_lfetch\_lrow(). This is more memory efficient than using mysql\_lstore\_lresult() would be.
            \begin{itemize}
                \item \textbf{mysql}: a pointer to the connection object that ran the query
            \end{itemize}
            \bigbreak \noindent 
            Returns a pointer to the result set structure on success. NULL is returned on failure.
        \item \textbf{Get the row data}:
            \bigbreak \noindent 
            \begin{cppcode}
            MYSQL_ROW mysql_fetch_row(MYSQL_RES *result)
            \end{cppcode}
            \bigbreak \noindent 
            This will fetch the next row in the result set as a MYSQL\_lROW structure, if there is one to fetch. It will start at the beginning and advance by one row each time it is called. If you call this after you've fetched the last row, it will return NULL. It will also return NULL if an error has occurred.
            \bigbreak \noindent 
            \begin{itemize}
                \item \textbf{result}: a pointer to the result set object to fetch rows from
            \end{itemize}
            \bigbreak \noindent 
            If you have any binary data in any of your fields, you will need to use mysql\_lfetch\_llengths() to get the lengths, as they may contain the null character as part of their valid data. Otherwise, you can treat a MYSQL\_lROW as an array of null-terminated strings. The fields can be addressed as the elements of the array, in the order they appear in the result set, starting from element 0.
        \item \textbf{Get byte-lengths for the fields in a row}:
            \bigbreak \noindent 
            \begin{cppcode}
            unsigned long *mysql_fetch_lengths(MYSQL_RES *result)
            \end{cppcode}
            \bigbreak \noindent 
            This function returns an array of integers that contains the lengths of each of the fields in a row returned by a previous call to mysql\_lfetch\_lrow(). The integer in a given element of the array returned is the length of the corresponding element in the MYSQL\_lROW
            \bigbreak \noindent 
            \begin{itemize}
                \item \textbf{result}:  a pointer to the result set you just got a row from
            \end{itemize}
            \bigbreak \noindent 
            Returns NULL on error. The most common error will be that you haven't fetched a row, or the last fetch failed.
        \item \textbf{Compiling without Linking}: To compile without linking, you can specify the -c flag to gcc or g++. This will take your source code, program.cc or program.c in this example, and make an object file from the code within it.
            \bigbreak \noindent 
            \begin{cppcode}
                g++ -c -I/usr/include/mariadb program.cc
                gcc -c -I/usr/include/mariadb program.c
            \end{cppcode}
            \bigbreak \noindent 
            Either of these statements will yield a new file called program.o, which is the object file compiled from the original source.
        \item \textbf{Compilation Errors}: The compilation stage is concerned with declarations. If you have an error that says something is undeclared, that failure is happening in the compilation stage.
            \bigbreak \noindent 
            If you have an error during compilation, it is usually one of the following types
            \begin{itemize}
                \item You forgot to include a header file that contains necessary declarations
                \item You made a syntax error in your code, which the compiler's error message should help you correct
                \item You made a typo somewhere (misspelled identifiers, etc.)
            \end{itemize}
        \item \textbf{Linking Alone}: After your object files have been created through compilation, you can perform the linking stage with one of the following commands.
            \bigbreak \noindent 
            \begin{cppcode}
                g++ -o program program.o -lmariadb
                gcc -o program program.o -lmariadb
            \end{cppcode}
            \begin{itemize}
                \item The -o flag say to name the program whatever the next word is. In this case, your program.o would be linked to make an executable file named program. If your project had multiple source code files, you'd include all of the object files' names, separated by spaces, where program.o is now.
                \item the -l flag is used to specify the name of the library to link in.
            \end{itemize}
        \item \textbf{Linking Errors}: The linking stage is concerned with finding the compiled code in object files/libraries. If you see an error about an "undefined reference", then the failure is happening during the linking stage. If an error occurs during the linking phase, it is usually one of the following
            \begin{itemize}
                \item You failed to list one of the object files that contains the implementations of your functions
                \item You failed to tell the linker to include a library that is needed
                \item The linker path does not include the directory your library is located in.
                \item Your header file has a different declaration than its implementation uses.
            \end{itemize}
        \item \textbf{Compilation and Linking Together}: It is possible to perform compilation and linking with the same command. This may be easier to type, but doing it separately allows you to skip recompiling files that haven't changed. The following commands are an example of how to do both stages together.
            \bigbreak \noindent 
            \begin{cppcode}
                g++ -o program -I/usr/include/mariadb -lmariadb program.cc
                gcc -o program -I/usr/include/mariadb -lmariadb program.c
            \end{cppcode}

    \end{itemize}

    \pagebreak 
    \unsect{Software engineering}
    \bigbreak \noindent 
    \subsection{Introduction}
    \begin{itemize}
        \item \textbf{Software engineering}: Software engineering is concerned with theories, methods, and tools for professional software, expenditure on software represents a significant fraction of GNP in all developed countries 
            \bigbreak \noindent 
            Software engineering is concerned with cost effective software development
            \bigbreak \noindent 
            SWE is an engineering discipline that is concerned with all aspects of software product from the early stages of system specification through to maintaining the system after it has gone into use
            \begin{itemize}
                \item \textbf{Engineering discipline}: Using appropriate theories and methods to solve problems bearing in mind organizational and financial constraints
                \item \textbf{All aspects of software product}: Not just technical process of development. Also project management and the development of tools, methods, etc to support software production.
            \end{itemize}
        \item \textbf{Software costs}: Software costs dominate computer system costs, the costs of software on a PC are often greater than the hardware cost
            \bigbreak \noindent 
            Software costs more to maintain than it does to develop, maintenance costs may be several times development costs
            \bigbreak \noindent 
            It is usually cheaper to use swe methods for software systems rather than just write the programs as if it was a personal programming project. Majority of costs are the costs of changing the software after its gone into use.
        \item \textbf{Software project failure}: New software eng techniques help us build larger, more complex systems. Although the demands are changing, larger more complex systems are required.
            \bigbreak \noindent 
            Software that does not use effective software eng techniques are often more expensive and less reliable than it should be.
        \item \textbf{Good software}: Good software should deliver the required functionality and performance to the user and should be maintainable, dependable and useable
        \item \textbf{Fundamental software engineering activities}: Software specification, software development, software validation and software evolution,
            \begin{itemize}
                \item \textbf{Software specification:} Customers and engineers define the software to be produced and the constraints on its operation.
                \item \textbf{Software development:} The software is designed and programmed.
                \item \textbf{Software validation:} The software is checked to ensure it meets customer requirements.
                \item \textbf{Software evolution:} The software is modified to reflect changing customer and market requirements.
            \end{itemize}
        \item \textbf{CS vs software engineering}: CS focuses on theory and fundamentals, SWE is concerned with the practicalities of developing and delivering useful software
        \item \textbf{SWE vs system engineering}: System engineering is concerned with all aspects of computer based systems development  including hardware, software and process engineering. SWE is part of this more general process.
        \item \textbf{Key challenges for SWE}: Coping with increasing diversity, demands for reduced delivery times, and developing trustworthy software.
        \item \textbf{What are the costs of software engineering}: 60\% of software costs are development  costs, 40\% are testing costs.
        \item \textbf{What differences has the web made to software engineering}: The web has led to the availability of software services and the possibility of developing highly distributed service based systems. Web based systems development has  led to important advances in programming languages and software reuse.
        \item \textbf{Software products}:
            \begin{itemize}
                \item \textbf{Generic products}: Standalone systems that are marketed and sold to any customer who wants to buy them.
                    \bigbreak \noindent 
                    The specification of what the software should do is owned by the software developer and decisions on software change are made by the dev
                \item \textbf{Customized products}: Software that is commissioned by a specific customer to meet their own needs.
                    \bigbreak \noindent 
                    Specification of what the software should do is owned by the customer for the software and they make decisions on software changes that are required.
            \end{itemize}
        \item \textbf{Good software}:
            \begin{center}
                \begin{tabular}{|p{4cm}|p{10cm}|}
                    \hline
                    \textbf{Product Characteristic} & \textbf{Description} \\ \hline
                    Maintainability & Software should be written to evolve with changing customer needs, as software change is a critical requirement in a dynamic business environment. \\ \hline
                    Dependability and Security & Includes reliability, security, and safety. Dependable software prevents physical or economic damage and restricts access from malicious users. \\ \hline
                    Efficiency & Avoids wasteful use of resources like memory and processor cycles, emphasizing responsiveness, processing time, and memory utilization. \\ \hline
                    Acceptability & Software must be understandable, usable, and compatible with systems used by its intended audience. \\ \hline
                \end{tabular}
            \end{center}
        \item \textbf{General issues that affect software}:
            \begin{itemize}[noitemsep]
                \item \textbf{Heterogeneity:} Systems must operate as distributed networks across various types of devices, including computers and mobile devices.
                \item \textbf{Business and Social Change:} Rapid societal and economic changes, driven by emerging technologies, require adaptable and quickly developed software solutions.
                \item \textbf{Security and Trust:} Software is deeply integrated into our lives, making it essential to trust its reliability and security.
                \item \textbf{Scale:} Software must accommodate a vast range of applications, from small embedded systems in portable devices to large-scale, cloud-based systems serving global communities.
            \end{itemize}
        \item \textbf{Application types}:
            \begin{itemize}[noitemsep]
                \item \textbf{Stand-alone applications:} These are systems that run on a local computer, such as a PC, with all necessary functionality and no network connection required.
                \item \textbf{Interactive transaction-based applications:} Applications executed on a remote computer, accessed by users through their PCs or terminals. Examples include web applications like e-commerce systems.
                \item \textbf{Embedded control systems:} Software control systems managing hardware devices. These are numerous and found in a wide variety of hardware applications.
                \item \textbf{Batch processing systems:} Business systems designed to process large amounts of data in batches, converting multiple inputs into corresponding outputs.
                \item \textbf{Entertainment systems:} Primarily for personal use, these systems are intended to entertain users.
                \item \textbf{Systems for modeling and simulation:} Developed by scientists and engineers to model physical processes or situations, involving multiple separate and interacting objects.
                \item \textbf{Data collection systems:} Systems that collect data from their environment using sensors and send it to other systems for processing.
                \item \textbf{Systems of systems:} These are composed of multiple software systems working together.
            \end{itemize}
        \item \textbf{Web services (internet SWE)}: Web services allow application functionality to be accessed over the web
        \item \textbf{Cloud computing}: An approach to the provision of computer services where applications run remotely on the 'cloud'. Users do not buy software but pay to use
        \item \textbf{Web based SWE}: Web-based systems are complex distributed systems but the fundamental principles of SWE are as applicable to them as they are to any other types of system
            \begin{itemize}[noitemsep]
                \item \textbf{Software reuse:} The dominant approach for constructing web-based systems. It involves assembling systems from pre-existing software components and systems.
                \item \textbf{Incremental and agile development:} Web-based systems should be developed and delivered incrementally. It is now recognized as impractical to specify all requirements in advance.
                \item \textbf{Service-oriented systems:} Implemented using service-oriented software engineering, where components are standalone web services.
                \item \textbf{Rich interfaces:} Technologies such as AJAX and HTML5 enable the creation of rich interfaces within a web browser.
            \end{itemize}
        \item \textbf{SWE ethics}:
            \begin{itemize}[noitemsep]
                \item \textbf{Confidentiality:} Engineers must respect the confidentiality of their employers or clients, regardless of whether a formal confidentiality agreement has been signed.
                \item \textbf{Competence:} Engineers should not misrepresent their level of competence and must avoid knowingly accepting work beyond their expertise.
                \item \textbf{Intellectual property rights:} Engineers should be aware of local laws related to intellectual property, including patents and copyrights. They must ensure the intellectual property of employers and clients is protected.
                \item \textbf{Computer misuse:} Engineers must not use their technical skills to misuse others' computers. Misuse can range from trivial actions, such as playing games on an employer's machine, to serious offenses like disseminating viruses.
            \end{itemize}
        \item \textbf{ACM/IEEE code of ethics}:
            \begin{itemize}[noitemsep]
                \item The professional societies in the US have collaborated to create a code of ethical practice.
                \item Members of these organizations agree to abide by the code of practice upon joining.
                \item The code includes eight principles guiding the behavior and decisions of professional software engineers, including practitioners, educators, managers, supervisors, policymakers, trainees, and students of the profession.
            \end{itemize}
            Software engineers shall commit themselves to making the analysis, specification, design, development, testing, and maintenance of software a beneficial and respected profession. 
            \bigbreak \noindent 
            In accordance with their commitment to the health, safety, and welfare of the public, software engineers shall adhere to the following Eight Principles:
            \begin{enumerate}[noitemsep]
                \item \textbf{Public:} Software engineers shall act consistently with the public interest.
                \item \textbf{Client and Employer:} Software engineers shall act in the best interests of their client and employer, consistent with the public interest.
                \item \textbf{Product:} Software engineers shall ensure that their products and related modifications meet the highest professional standards possible.
                \item \textbf{Judgment:} Software engineers shall maintain integrity and independence in their professional judgment.
                \item \textbf{Management:} Software engineering managers and leaders shall subscribe to and promote an ethical approach to the management of software development and maintenance.
                \item \textbf{Profession:} Software engineers shall advance the integrity and reputation of the profession consistent with the public interest.
                \item \textbf{Colleagues:} Software engineers shall be fair to and supportive of their colleagues.
                \item \textbf{Self:} Software engineers shall participate in lifelong learning regarding the practice of their profession and shall promote an ethical approach to the practice of the profession.
            \end{enumerate}




    \end{itemize}

    \pagebreak 
    \unsect{Assembler}
    \subsection{Introduction to the mainframe, assist, and TSO/ISPF}
    \begin{itemize}
        \item \textbf{The mainframe}: A mainframe is a powerful, high-performance computer system designed for large-scale data processing and critical applications. It is widely used in industries that require high reliability, scalability, and security, such as banking, healthcare, government, retail, and telecommunications.
            \begin{itemize}
                \item \textbf{High Reliability (Fault Tolerance)}: Mainframes are designed for continuous operation, often running without interruption for years. They feature redundant components and sophisticated error detection and correction systems to minimize downtime.
                \item \textbf{Massive Processing Power}: Mainframes can process vast amounts of data and handle thousands to millions of transactions per second, making them ideal for enterprise-scale applications like banking systems or airline reservations.
                \item \textbf{Scalability}: Mainframes can handle an increasing workload by adding more processing power or storage without needing major architectural changes.
                \item \textbf{Virtualization and Multitasking}: They support multiple operating systems and can run thousands of virtual servers simultaneously. This capability enables diverse workloads and maximizes resource utilization.
                \item \textbf{Data Security and Compliance}: Mainframes have robust security features, including encryption, access control, and auditing, making them suitable for industries with stringent compliance requirements (e.g., financial regulations like PCI DSS).
                \item \textbf{Batch and Online Processing}: They excel in both batch processing (processing large data sets at scheduled times) and online transaction processing (OLTP) for real-time applications.
            \end{itemize}
        \item \textbf{Examples of mainframes}:
            \begin{itemize}
                \item \textbf{IBM Z Series:} A popular series of mainframes developed by IBM, including the IBM z15 and z16, known for advanced encryption and AI integration.
                \item \textbf{UNIVAC and BULL:} Earlier examples of mainframes used for enterprise computing.
            \end{itemize}
        \item \textbf{z/OS}: z/OS is IBM's mainframe operating system designed to handle large-scale computing for enterprise applications. It is part of IBM's Z series mainframe platform and is known for its high reliability, scalability, and security. z/OS supports both legacy and modern workloads, making it the backbone for many mission-critical operations in industries like banking, healthcare, and government.
        \item \textbf{Assembly and assembler}: Assembly, or assembly language, is a low-level programming language that is closely tied to a computer's hardware architecture. It serves as a human-readable representation of the machine language instructions executed by a computer's central processing unit (CPU).
            \bigbreak \noindent 
            Assembly is one step above machine code, making it hardware-specific. Each instruction corresponds closely to a machine code instruction
            \bigbreak \noindent 
            Assembly uses mnemonics (e.g., MOV, ADD, SUB) to represent instructions, making it easier to understand compared to binary machine code.
            \bigbreak \noindent 
            Assembly code is specific to a particular CPU architecture (e.g., x86, ARM, IBM System/360). Code written for one architecture typically won't work on another without modification.
            \bigbreak \noindent 
            Provides direct access to hardware components such as CPU registers, memory addresses, and I/O ports, allowing precise control over the system.
            \bigbreak \noindent 
            An assembler converts the assembly code into machine code (binary) that the CPU can execute. The machine code is linked with libraries and loaded into memory for execution.
        \item \textbf{ASSIST}: ASSIST (Assembler System for Student Instruction and Systems Teaching) is an educational assembler designed to help students learn IBM Assembly Language programming (also known as System/360 or System/370 Assembly Language). It is a simplified, user-friendly tool tailored for academic settings, offering an approachable environment for understanding low-level programming concepts and mainframe assembly.
        \item \textbf{TSO (Time sharing option)}: TSO, or Time Sharing Option, is an interactive environment in IBM's z/OS mainframe operating system. It allows users to access and interact with the system in a time-shared manner, as opposed to batch processing, where jobs are submitted and processed sequentially. TSO enables users to:
            \begin{itemize}
                \item Enter commands interactively.
                \item Run programs or scripts directly.
                \item Edit and manage datasets.
                \item Submit batch jobs for execution.
            \end{itemize}
        \item \textbf{ISPF (Interactive System Productivity Facility)}: ISPF, or Interactive System Productivity Facility, is a user-friendly, menu-driven interface that runs on top of TSO. It is essentially an application within TSO that provides a structured and more intuitive way to perform tasks on the z/OS system. ISPF offers:
            \begin{itemize}
                \item \textbf{Menu Navigation:} A hierarchical menu system for accessing different functionalities.
                \item \textbf{Dataset Management:} Create, edit, delete, and view datasets (files).
                \item \textbf{Utilities:} Tools for sorting, copying, and comparing data.
                \item \textbf{Programming Support:} Edit and compile source code, debug programs, and manage jobs.
                \item \textbf{Custom Applications:} Organizations can add their own ISPF applications.
            \end{itemize}
            ISPF makes working with TSO more accessible by abstracting many complexities into guided menus and forms.
        \item \textbf{Datasets}: In the context of IBM mainframes and z/OS, a dataset is a structured collection of data stored on a storage medium. Datasets are the main way data is organized, stored, and accessed on mainframe systems. They are similar to files in other operating systems, but with more specific formats, attributes, and access methods.
            \bigbreak \noindent 
            Datasets in z/OS are categorized by how they are organized and accessed:
            \begin{enumerate}
                \item \textbf{Sequential Datasets (PS - Physical Sequential)}:
                    \begin{itemize}
                        \item Data is stored in a linear fashion, similar to text files in other systems.
                        \item Access is sequential (record-by-record in order).
                        \item Example: Log files or batch job outputs.
                    \end{itemize}
                \item \textbf{Partitioned Datasets (PDS) or Extended PDS (PDSE)}:
                    \begin{itemize}
                        \item Contains multiple members (like a directory of files).
                        \item Each member acts as a separate file.
                        \item Commonly used for source code, JCL, and configuration data.
                        \item Example: A PDS might contain different COBOL program files as separate members.
                    \end{itemize}
            \end{enumerate}
            \bigbreak \noindent 
            \textbf{Note:} A PDSE is sometimes called a "library." This is only because a PDSE, unlike a sequential data set, or "flat file," is separated into different members which are, in themselves, sequential files. Each of these members of a PDSE is somewhat like a book on a bookshelf in a library, hence the alias "library." So, a PDSE is a collection of members
    \end{itemize}

    \pagebreak 
    \subsection{Using ISPF}
    \begin{itemize}
        \item \textbf{Logging off TSO/ISPF}: To properly sign off TSO/ISPF, press F3 while in the ISPF Primary Option Menu. If you have made changes while signed on, you will be presented the following screen:
            \bigbreak \noindent 
            \fig{.5}{./figures/77.png}
            \bigbreak \noindent 
            Type the number 2 and press Enter and you will then be presented with a screen with a red-lettered READY displayed.
            \bigbreak \noindent 
            Type the word logoff or LOGOFF and press Enter and you will now be logged off.
        \item \textbf{Making a PDSE}: To allocate a PDSE, enter 3 for Utilities and press Enter. On this screen, enter 2 (Data Set) and press Enter 
            \bigbreak \noindent 
            Move the cursor by tabbing or with your mouse to the line to the right of Project under ISPF Library: Type your project name. Tab again to the line to the right of Group and type your group name. Tab again to the line to the right of Type and type your type name. Your library will be \textit{projectname.groupname.typename}
            \bigbreak \noindent 
            Move your cursor to the option line and type $a$
            \bigbreak \noindent 
            Only change or fill in the specific fields mentioned here: First, tab or use your mouse to move the cursor to the line to the right of Space units and enter TRKS for tracks. Tab twice and enter 10 for Primary Quantity. Tab again and enter 10 for Secondary quantity. Tab again and enter 5 for Directory blocks. Tab again and enter FB for Record format. Tab again and enter 80 for Record length. Tab again and enter 880 for Block size. Tab again and enter LIBRARY for Data set name type.
            \bigbreak \noindent 
            After pressing Enter and, if the data set does not already exist (and it shouldn't!), you will be presented the Data Set Utility panel again with the message Data set allocated in white lettering in the upper right hand corner of the panel: This indicates success! At this point, F3 back to the option page.
        \item \textbf{Naming things on the mainframe}: It is first important to know that names of entities on the mainframe can have 1 to 8 characters. They can only contain letters A-Z (upper case only), digits 0-9, and international characters \$, \# and @ (dollar sign, pound sign/hash tag, and at sign). They can only begin with a letter or one of the three international characters
        \item \textbf{Creating PDSE members}: Enter 2 (Edit) at the Option ===> line and press Enter. The first time you enter this panel, you will need to fill in some fields that will be pre-filled the next time you come back to it.
            \bigbreak \noindent 
            Fill in the project, group, and type of the created PDSE. Enter a name for the member you would like to create and begin editing. If you type an existing member, you can begin editing that as well.
        \item \textbf{Editing PDSE members}: Follow the same instructions from above (creating pdse members) or you can press Enter in the Edit Entry Panel (with the member field blank) and move the cursor to the dot across from the name of the member you wish to edit and type either s, S, e, or E and press Enter.
            \bigbreak \noindent 
            \textbf{To insert a line while editing:} Move the cursor with the tab key or your mouse to the line numbers on the left hand side of the screen and anywhere within those 6 digits type the letter i or I.  Press Enter to have the line inserted
            \bigbreak \noindent 
            To insert multiple lines while editing, do the above but follow the letter i or I with an integer between 2 and n. It will insert the number of lines you have requested but it will not scroll to show all of your inserted lines. It will fit as many on the panel as it can depending on where you began inserting the lines.
            \bigbreak \noindent 
            \textbf{To delete a line while editing:} Move the cursor with the tab key or your mouse to the line numbers on the left hand side of the screen and anywhere within those 6 digits type the letter d or D
            \bigbreak \noindent 
            \textbf{To move a line while editing:} Move the cursor with the tab key or your mouse to the line numbers on the left hand side of the screen and anywhere within those 6 digits type the letter m or M. Then, move the cursor with the tab key or your mouse to where you want the line moved and type either a or A for inserting the line you are moving after the line where your cursor is or type either b or B for inserting the line you are moving before the line where your cursor is. note that, when you start a move, you must complete it before you can go on editing. In other words, if you change your mind, you will still have to move the line but you can then delete it if necessary.
            \bigbreak \noindent 
            \textbf{To copy a line while editing}: Move the cursor with the tab key or your mouse to the line numbers on the left hand side of the screen and anywhere within those 6 digits type the letter c or C. Then, move the cursor with the tab key or your mouse to where you want the line copied and type either a or A for inserting the line you are moving after the line where your cursor is or type either b or B for inserting the line you are moving before the line where your cursor is.
            \bigbreak \noindent 
            \textbf{Deleting, moving or copying blocks, or multiple lines while editing:} To delete a block of contiguous lines, type dd or DD on the first line you want to delete and type dd or DD on  the last line you want to delete. These two lines and every line in between will be deleted.
            \bigbreak \noindent 
            To move a block, use mm or MM on both the first line and the last line you want to move and the a or A for after or b or B for before as in item K above. The block of contiguous lines will be deleted from its original place.
            \bigbreak \noindent 
            To copy a block, use cc or CC on both the first line and the last line you want to copy and the a or A for after or b or B for before as in item K above. The block of contiguous lines will remain in its original place and a copy will be inserted where you indicated it to be inserted.
            \bigbreak \noindent 
            \textbf{To split a line of text:} To move the end of a line to the next line, or split the text, move the cursor with the tab key or your mouse to the line numbers on the left hand side of the screen and anywhere within those 6 digits type ts or TS for 'text split'. Then, before you press Enter, move your cursor to the character where you want to begin the split Now, press Enter. Everything from the character you indicated as the beginning of the split will be pushed and inserted two lines down with a new line inserted in between
            \bigbreak \noindent 
            \textbf{To collapse (hide) lines while editing:} To collapse one or more lines, move the cursor with the tab key or your mouse to the line numbers on the left hand side of the screen and anywhere within those 6 digits type x or X to collapse one line or xn or Xn to collapse n lines.
            \bigbreak \noindent 
            To collapse a block, type xx or XX on the first line you want to collapse and scroll to the last line you want to collapse and type xx or XX. Press Enter and a dashed line will appear telling you how many lines are collapsed.
            \bigbreak \noindent 
            \textbf{To uncollapse (reveal) lines while editing: } Type res or RES or reset or RESET on the command line and press enter and all of the lines will be uncollapsed or revealed.
            \bigbreak \noindent 
            To uncollapse or reveal some lines but not others, go to the dashed line and type f or F to reveal the first collapsed line or fn or Fn to reveal the first n collapsed lines. You can also type l or L to reveal the last collapsed line or ln or Ln to reveal the last n collapsed lines.
            \bigbreak \noindent 
            \textbf{Useful Primary Editing Primary Commands}:
            \begin{itemize}
                \item \textbf{LOCATE line-number}: Moves to the indicated line.
                \item \textbf{FIND string}: Finds the first occurrence of \textit{string}, starting from the current line.  
                    \bigbreak \noindent 
                    To find the next occurrence, press \textbf{PF5}.
                \item \textbf{CHANGE string-1 string-2 [ALL]}: Finds the first occurrence of \textit{string-1}, starting from the current line,  and changes it to \textit{string-2}.  
                    \bigbreak \noindent 
                    To find the next occurrence, press \textbf{PF5}.  
                    \bigbreak \noindent 
                    To change the next occurrence, press \textbf{PF6}.  
                    \bigbreak \noindent 
                    To change all occurrences, include the \textbf{ALL} option.
                \item \textbf{COPY member-name}: Retrieves data from the specified member; use an \textbf{A} or \textbf{B} line command  to specify where the data should be placed.
                \item \textbf{PROFILE}: Displays the profile settings for the edit session.
                \item \textbf{RECOVERY [ON | OFF]}: Determines whether edit recovery mode is on.
                \item \textbf{UNDO}: Reverses the last editing change.
                \item \textbf{SAVE}: Saves changes and continues the edit session.
                \item \textbf{END (PF3/15)}: Saves changes and returns to the Edit Entry panel.
                \item \textbf{RETURN (PF4/16)}: Saves changes and returns to the Primary Option Menu.
                \item \textbf{CANCEL}: Returns to the Edit Entry panel without saving changes.
            \end{itemize}
        \item \textbf{Saving  members}: Type the word save or SAVE anywhere on the command line
        \item \textbf{Setting the Scroll ===> value:} It is strongly suggested that you change Scroll ===> PAGE to Scroll ===> CSR on every panel that you can in ISPF. ISPF will retain this setting if you exit the panel normally. 
        \item \textbf{SDSF}: SDSF stands for System Display and Search Facility. It is a powerful interactive tool used on IBM z/OS mainframes for monitoring, managing, and controlling system resources and jobs. SDSF provides users with a user-friendly interface to view and interact with the output of batch jobs, system logs, and other critical system information.
            \bigbreak \noindent 
            Here are some typical commands used within SDSF
            \begin{itemize}
            \item \textbf{? or H:} Help.
            \item \textbf{OWNER *:} Displays all jobs owned by the user.
            \item \textbf{PREFIX jobname:} Filters jobs by their name or prefix.
            \item \textbf{FIND string:} Searches for a string in the output.
            \item \textbf{SORT column:} Sorts the list by the specified column (e.g., jobname, priority).
            \item \textbf{P jobname:} Purges a job from the queue.
            \item \textbf{H jobname:} Holds a job in the queue.
            \item \textbf{R jobname:} Releases a held job.
            \end{itemize}
        \item \textbf{Jobs}: In the context of IBM mainframes and z/OS, a job refers to a unit of work submitted to the system for processing. Jobs are typically associated with batch processing, where multiple tasks or programs are executed without direct user interaction. Jobs are defined and controlled using Job Control Language (JCL), which specifies the tasks to be performed and the resources required.
        \item \textbf{Batch processing}: Batch processing is a method of executing a series of tasks or jobs on a computer without requiring user interaction during the process. It is commonly used in environments where large volumes of data need to be processed or repetitive tasks need to be automated. Batch processing is a fundamental concept in mainframe systems like IBM z/OS but is also used in other computing environments.
        \item \textbf{Viewing results in SDSF}: To review output in the output queue in SDSF on TSO/ISPF at Marist, enter SD (for SDSF) from the ISPF Primary Option Menu command line. From the SDSF Primary Option Menu enter ST for status on the command line. This will display the queue of completed jobs, both successful and unsuccessful. 
            \bigbreak \noindent 
            Be sure not to let these completed jobs pile up. To get rid of jobs in the queue, put a P on the line in the margin just to the left of the job to be purged and press enter. A P can be entered on multiple jobs at once but the user may have to press enter a few times to get the jobs to roll off.
            \bigbreak \noindent 
            By the way, the user can enter SD.ST from the ISPF Primary Option Menu to go directly to the status queue. If somewhere else within TSO, the user can enter =SD.ST to go directly to the status queue. For example, if editing a PDS member and it has been saved, the user can submit his or her job by typing SUB on the command line. A red message will pop up if a successful submission has been made. Press enter again and then enter =SD.ST and press enter to go to the SDSF Status queue to see the results of the recently submitted job.
            \bigbreak \noindent 
            Once in SDSF status queue, select the job to be reviewed in the queue by typing a letter S in the margin just to the left of the job. It is important to review at least the first few 'pages' of output.
            \bigbreak \noindent 
            \textbf{Action Characters Used in SDSF Status}
            \begin{itemize}
                \item \textbf{?}: Displays a list of the output data sets for a job.
                \item \textbf{S}: Displays one or more output data sets.
                \item \textbf{H}: Holds a job.
                \item \textbf{A}: Releases a held job.
                \item \textbf{O}: Releases held output and makes it available for printing.
                \item \textbf{C}: Cancels a job.
                \item \textbf{P}: Purges a job and its output
            \end{itemize}
            \bigbreak \noindent 
            \textbf{Action Characters Used in the SDSF Held Output Queue}
            \begin{itemize}
                \item \textbf{?}: Displays a list of the output data sets for a job.
                \item \textbf{S}: Displays one or more output data sets.
                \item \textbf{O}: Releases output and makes it available for printing.
                \item \textbf{P}: Purges output data sets.
            \end{itemize}
    \end{itemize}


    \pagebreak 
    \subsection{Number systems and computer storage}
    \begin{itemize}
        \item \textbf{The Decimal number system}: In the decimal system, any natural number $m$ can represented by use of the ten symbols $0,1,2,3,...9$. These are the decimal digits, $m$ can be represented as
            \begin{align*}
                d_{n}d_{n-1}d_{n-2}...d_{1}d_{0}
            \end{align*}
            Where $m \geq 0$. This same number $m$ can be represented as 
            \begin{align*}
                d_{n} \times 10^{n} + d_{n-1} \times 10^{n-1} + d_{n-2} \times 10^{n-2} + ... + d_{1} \times 10^{1} + d_{0} \times 10^{0}
            \end{align*}
            For example, the natural number $123$ can be represented as 
            \begin{align*}
                1 \times 10^{2} + 2\times 10^{1} + 3 \times 10^{0}
            \end{align*}
            The decimal system is also called the base ten system since ten digits are utiltized in the number representations. 
            \bigbreak \noindent 
            There is, however, nothing sacred about the base ten since the notion of a positional number system can be easily generalized to any given base \( b \) where \( b \) is a natural number greater than or equal to tw
        \item \textbf{Binary}: The binary number system is a base two number system, where any natural number $m$ can be represented with the digits $0$ and $1$. Observe since
            \begin{align*}
                123 &= 1 \times 2^{6} + 1\times 2^{5} +1\times 2^{4} + 1\times 2^{3} + 0\times 2^{2} +  1\times 2^{1} + 1\times 2^{0}
            \end{align*}
            The number 123 would be represented in the binary system as 
            \begin{align*}
                1111011
            \end{align*}
        \item \textbf{Hexadecimal}: The hexadecimal number system is a base 16 number system, which uses $0,1,2,...9,A,B,C,...F$, where $A,B,C,...,F$ represent $10,11,12,...,15$. Observe
            \begin{align*}
                123 = 7 \times 16^{1} + 11\times 16^{0}
            \end{align*}
            Thus 123 has the decimal representation $7B $
        \item \textbf{Numbers 1-30 in each system}: The following table gives the representations of the numbers zero through thirty-two 
            in each of these three number systems.
            \begin{center}
                \begin{tabular}{|c|c|c|c|c|c|}
                    \hline
                    \textbf{Decimal} & \textbf{Hexadecimal} & \textbf{Binary} & \textbf{Decimal} & \textbf{Hexadecimal} & \textbf{Binary} \\ \hline
                    0 & 0 & 0 & 17 & 11 & 10001 \\ \hline
                    1 & 1 & 1 & 18 & 12 & 10010 \\ \hline
                    2 & 2 & 10 & 19 & 13 & 10011 \\ \hline
                    3 & 3 & 11 & 20 & 14 & 10100 \\ \hline
                    4 & 4 & 100 & 21 & 15 & 10101 \\ \hline
                    5 & 5 & 101 & 22 & 16 & 10110 \\ \hline
                    6 & 6 & 110 & 23 & 17 & 10111 \\ \hline
                    7 & 7 & 111 & 24 & 18 & 11000 \\ \hline
                    8 & 8 & 1000 & 25 & 19 & 11001 \\ \hline
                    9 & 9 & 1001 & 26 & 1A & 11010 \\ \hline
                    10 & A & 1010 & 27 & 1B & 11011 \\ \hline
                    11 & B & 1011 & 28 & 1C & 11100 \\ \hline
                    12 & C & 1100 & 29 & 1D & 11101 \\ \hline
                    13 & D & 1101 & 30 & 1E & 11110 \\ \hline
                    14 & E & 1110 & 31 & 1F & 11111 \\ \hline
                    15 & F & 1111 & 32 & 20 & 100000 \\ \hline
                    16 & 10 & 10000 & & & \\ \hline
                \end{tabular}
            \end{center}
        \item \textbf{Base notation}: For clarity, when it is not clear which system a given number is represented in, a subscript with the base of the system will be used. For example, 
            \begin{align*}
                (123)_{10} &= 7B_{16} = 1111011_{2}
            \end{align*}
        \item \textbf{Binary and hex to decimal}: Let $a_{n}a_{n-1}...a_{2}a_{1}a_{0} $ be the representation of a number $m$ in base $b$. Then, the decimal representation of $m$ is given by the sum
            \begin{align*}
                d_{n}b_{n} + d_{n-1}b_{n-1} + ... + d_{2}b_{2} +d_{1}b_{1} + d_{0}b_{0}
            \end{align*}
            Where each $d_{i}$ is the decimal equivalent of the corresponding $a_{i}$
            \bigbreak \noindent 
            For example, consider the binary number $1011_{2}$, then
            \begin{align*}
                1\cdot 2^{0} + 1\cdot 2^{1} + 0 \cdot 2^{2} + 1\cdot 2^{3} = 11
            \end{align*}
            Next, consider the hex number $A61$. Observe
            \begin{align*}
                1 \cdot 16^{0} + 6 \cdot 16^{1} + A \cdot 16^{2} &= 1 \cdot 16^{0} + 6 \cdot 16^{1} + 10 \cdot 16^{2} \\
                                                                 &=2657
            \end{align*}
        \item \textbf{Decimal to binary or hex}: Obtain the representation of a number $n$ in a given base $b$ from the representation of $n$ in the decimal system by using the following steps
            \begin{enumerate}
                \item Divide $n$ by $b$, giving a quotient $q$ and remainder $r$
                \item Write the representation of $r$ in the base $b$  as the rightmost digit or as the digit to the immediate left of the one last written.
                \item If $q$ is zero, stop. Otherwise set $n$ equal to $q$ and go to step 1
            \end{enumerate}
            For example, consider $123_{10} \to h_{16} $, where $h$ is the hexadecimal representation. We follow the above procedure.
            \begin{align*}
                123 &= 16(7) + 11 \ : \ B_{16}\\
                7&= 16(0) + 7 \ : \ 7_{16}
            \end{align*}
            Since we hit a $q=0$, we stop. The hexadecimal representation is then $7B_{16}$. Next, consider $123 \to b_{2} $
            \begin{align*}
                123 &= 2(61) + 1 \ :\ 1_{2} \\
                61 &=2(30) + 1 \ : \ 1_{2} \\
                30 &= 2(15) + 0 \ : \ 0_{2} \\
                15 &= 2(7) + 1 \ : \ 1_{2} \\
                7 &= 2(3) + 1 \ : \ 1_{2} \\
                3&= 2(1) + 1 \ : \ 1_{2} \\
                1 &= 2(0) + 1 \ : \ 1_{2}
            \end{align*}
            Thus, $123_{10} = 111011_{2}$
        \item \textbf{Conversion between binary and hex}: Because 16 digits are required in the hexadecimal system and $2^{4} = 16$, a very simple algorithm exists for converting binary representations to hexadecimal representations, and vice versa.
            \bigbreak \noindent 
            The algorithm may be stated as follows
            \begin{enumerate}
                \item Starting at the right of a binary representation \(n\), separate the digits into groups of four. If there are fewer than four digits in the last (leftmost) group, add as many zeros as may be necessary to the left of the leftmost digit to fill out the group. For example, if \(n = 101101\), the digits should be separated into two groups depicted as follows: \(10 \, 1101\). Since the leftmost group does not contain four digits, two leading zeros are added to give \(0010 \, 1101\).
                \item Convert each group of four binary digits to a hexadecimal digit. The result is the hexadecimal representation of \(n\).
            \end{enumerate}
            Consider $n=101101$. Splitting into groups of four, we get the two groups
            \begin{align*}
                0010 \quad 1101
            \end{align*}
            Since $0010 =2_{10}$, and $1101 = 13_{10} = D_{16}$, we get $0010_{2} = 2D_{16} $
        \item \textbf{Addition of binary and hexadecimal numbers}: The algorithm for the addition of unsigned integer numbers is as follows
            \begin{enumerate}
                \item Write the two addends one above the other with the rightmost digits of these numbers aligned.
                \item Add the two rightmost digits; if a 1 appears above these digits, indicating a carry, add 1 to the result. Write the integer portion of this result to the immediate left of the last recorded digit in the sum; If a carry is part of the result, write a 1 above the next higher order pair of digits. (If one or both of the digits do not exist, assume a value of 0 for the missing digits.)
                \item Delete the rightmost digits of the two addends. If the digits of the addends are exhausted, stop; otherwise, go to Step 2.
            \end{enumerate}
            We start we a decimal system example. Suppose $743_{10}$ and  $864_{10}$ are to be added. Then,
            \bigbreak \noindent 
            \fig{.6}{./figures/73.png}
            \bigbreak \noindent 
            Collecting the results yields $1607_{10}$. The carry table for binary is simple, since there are only two digits involved.
            \begin{center}
                \begin{tabular}{c|c|c}
                    $+$ & $0$ & $1$ \\
                    \hline
                    0 & 0 & 1 \\
                    \hline
                    1 & 1 & $0+c$
                \end{tabular}
            \end{center}
            The result of using this table and the algorithm to find the sum of $10110$ and $1011$ is shown below
            \bigbreak \noindent 
            \fig{.6}{./figures/74.png}
            \bigbreak \noindent 
            Thus, $10110_{2} + 1011_{2} = 100001_{2} $
            \bigbreak \noindent 
            The carry table for hexadecimal addition is a bit more complex.
            \bigbreak \noindent 
            \fig{.6}{./figures/76.png}
            \bigbreak \noindent 
            Let's add $FCDE$ and $9A05 $
            \bigbreak \noindent 
            \fig{.7}{./figures/75.png}
            \bigbreak \noindent 
            Thus, the result is $196E3_{16} $
        \item \textbf{Subtraction of binary and hexadecimal numbers}: In the subtraction algorithm it is assumed that for $a-b$, $a \geq b$.
            \begin{enumerate}
                \item Write the minuend above the subtrahend with the rightmost digits of these numbers aligned.
                \item \begin{enumerate}[label=(\alph*)]
                        \item If the rightmost digit in the minuend is greater than or equal 
                            to the corresponding digit in the subtrahend, subtract the digit in 
                            the subtrahend from the corresponding digit in the minuend and 
                            write the result to the immediate left of the last recorded digit in 
                            the difference; otherwise
                        \item If the rightmost digit \(d\) in the minuend is less than the 
                            corresponding digit in the subtrahend, replace \(d\) by \(d + c\), decrease the 
                            next-higher-order nonzero digit in the minuend by 1, replace any 
                            intervening zero digits in the minuend by the digit corresponding in 
                            value to the base minus 1. Then subtract the rightmost digit in the 
                            subtrahend from \(d + c\) and write the result to the immediate left of 
                            the last recorded digit in the difference.
                    \end{enumerate}
            \item Delete the rightmost digits in the minuend and subtrahend. If the digits of these two numbers are exhausted, stop; otherwise, go to Step 2.
            \end{enumerate}
        \item \textbf{Main storage (RAM)}: A computer's main storage is typically RAM (Random Access Memory). It is the primary, volatile memory that temporarily stores data and instructions that the CPU needs while a computer is running. RAM is fast and allows quick access to data, but it loses its content when the computer is turned off.
            \bigbreak \noindent 
            Main storage can also refer to primary storage components, which include:
            \begin{itemize}
                \item \textbf{RAM}: Used for active processes and running applications.
                \item \textbf{Cache Memory}: High-speed memory located near or on the CPU for frequently accessed data.
                \item \textbf{Registers}: Small storage areas directly within the CPU for immediate operations.
            \end{itemize}
            It is distinct from secondary storage, such as hard drives (HDDs) or solid-state drives (SSDs), which provide long-term, non-volatile storage for data and programs.
        \item \textbf{Main storage organization}: The computers memory contains a certain amount of bits, say $2^{30}$ bits. These bits are organized into groups of eight contiguous bits, called a \textit{byte}. Bytes are assigned consecutive increasing hexadecimal numbers starting with the number zero. For example,
            \begin{align*}
                \underbrace{11111111}_{\text{Byte 0}} \quad \underbrace{11111111}_{Byte 1} \quad ...
            \end{align*}
            The number given to a byte is called the absolute \textit{address} of the byte. It is often necessary to reference \textit{fields} that contain several bytes. The address of such a field is considered to be the address of the first (leftmost) byte in the field.
            \bigbreak \noindent 
            If all of the bytes in storage are thought of as being grouped into consecutive non-overlapping pairs beginning with byte zero, the resulting pairs of bytes are referred to as \textit{halfwords}. From this, we note that each address that is an even number is the address of a halfword, and only even numbers are halfword addresses. For this reason, each even address is called a \textit{halfword boundary}.
            \bigbreak \noindent 
            Storage is also thought of as being organized into groups of four bytes, called \textit{fullwords (or just words)}, and groups of eight bytes, called \textit{doublewords}
        \item \textbf{Binary representation of integer numbers}: Let $n$ be the binary representation of any integer. Then, the \textit{one's complement} of $n$ is the result of replacing each 0 with a 1, and each 1 with a 0.  For example, consider
            \begin{align*}
                1001
            \end{align*}
            Then, the one's complement is
            \begin{align*}
                0110
            \end{align*}
            Further, if $n$ is the binary representation of any integer, then the \textit{two's complement} of $n$ is formed by doing both of the following
            \begin{enumerate}
                \item Find the one's complement of $n$
                \item Add 1 to the result
            \end{enumerate}
            Consider the binary number $0101_{2} = 5_{10}$. The one's complement is $1010$, adding one gives $1010 + 1 = 1011_{2}$. Thus, $1011$ is the two's complement of $0101$.
            \bigbreak \noindent 
            The 32-bit fullword is the unit of storage that was chosen in IBM computers for representing integer numbers. The method used for encoding integer numbers in fullwords is as follows
            \begin{enumerate}
                \item Any integer in the range $0$ to $2^{31} - 1$, where $2^{31}-1 = 2,147,483,647$, is represented in a fullword in its exact binary representation. Note that each 32-bit binary number in this range has a value of 0 in the leftmost bit position. This bit is called the \textit{sign bit}, and all positive integer representations are characterized by having a sign bit value of 0.  
                    \bigbreak \noindent 
                    If we have 32 bits, where the leftmost is the sign bit, then
                    \begin{align*}
                        1 + 2 + 4  + ... + 2^{30} &= 2^{31} -1
                    \end{align*}
                    Since 
                    \begin{align*}
                        1 + 2 + 4 + ... + 2^{n} &= 2^{n+1}-1
                    \end{align*}
                    And a $k$-bit binary number has $n = k-1$.
                \item Any integer in the range $-1$ to $-2^{31} + 1$ is encoded by taking the two's complement of the encoded form of its absolute value. A representation of $-2^{31}$ is also allowed, but this representation (a 1 followed by 31 0's) is not the two's complement of the representation of any positive integer. The sign bit in the encoded form of each negative integer has a value of 1.
            \end{enumerate}
            For example, the encoded form of $+1$ is 
            \begin{align*}
                00000000 \quad 00000000 \quad 00000000 \quad 00000001
            \end{align*}
            The encoded form of $-1$ is then found by taking the two's complement of $+1$. That is,
            \begin{align*}
                (&11111111 \quad 11111111 \quad 11111111 \quad 11111110) + 1 \\
                = &11111111 \quad 11111111 \quad 11111111 \quad 11111111
            \end{align*}
            There are two integers with encoded forms that are identical to their two's complement. These integers are 0 and $-2^{31}$.
            \bigbreak \noindent 
            Some rather remarkable things are true of the binary representations of integers in this scheme
            \begin{itemize}
                \item The two's complement of the representation of a negative integer (with the exception of $-2^{31}$) is the representation of the absolute value of that integer
                \item When binary addition is performed on the representations of integer numbers (whether the signs are mixed or not), the result has the correct value in the sign bit (provided the result is in the range $-2^{31} to 2^{31} - 1$)
            \end{itemize}
            \bigbreak \noindent 
            \textbf{Note:} The range of a fullword is $-2^{31}$ to $2^{31} -1$. $-2^{31}$ is represented as 
            \begin{align*}
                10000000 \quad 00000000 \quad 00000000 \quad 00000000
            \end{align*}
        \item \textbf{Integer addition and subtraction}: Integer addition is performed by simply performing the usual binary addition. Integer subtraction is performed by first taking the two's complement of the subtrahend and then adding the result to the minuend. There is no need for the sign bits of the integer representations of the numbers involved in these operations to be checked before the operations are performed, for the sign bit of the result will be correct as noted above.
            \bigbreak \noindent 
            To conserve space and since 32-bit binary numbers are all but impossible to read at a glance, the printouts of the conditions of memory locations are always given in hexadecimal form. The printed forms of the representations of 1 and $-1$ are therefore
            \begin{align*}
                00000001
            \end{align*}
            and 
            \begin{align*}
                FFFFFFFF
            \end{align*}
        \item \textbf{Overflow}: Overflow occurs when operations are performed on the representations of integer numbers with the effect that the carry into the sign bit of the result differs from the carry out of that position. Consider the following addition operations on integers coded in five-bit binary numbers
            \begin{align*}
                \begin{array}{cccccc}
                    0&1&&&& \\
                    &0&1&0&0&0\\
                    +&0&1&0&0&1 \\
                    \hline 
                     &1&0&0&0&1
                \end{array}
            \end{align*}
            Notice that the carry into the sign bit differs from the carry out of the sign bit... \textit{overflow!}. If these carries matched, the result would be valid.
        \item \textbf{Understanding signed numbers in the two's complement system}: Consider an 8-bit integer. Recall that for an $n$-bit signed integer, the range is
            \begin{align*}
                -2^{n-1} \text{ to } 2^{n-1}-1
            \end{align*}
            The maximum occurs when the sign bit is zero, and all other bits are one. Observe
            \begin{align*}
                01111111_{2} = 127_{10}
            \end{align*}
            The minimum occurs when the sign bit is one and all other bits are zero
            \begin{align*}
                10000000_{2} = -128_{10}
            \end{align*}
            If we have $8$ bits to play with, then the total number of 8-bit permutations is $2^{8} = 256$. Dividing this by two gives $128$ non-negative numbers and $128$ negative numbers. Let's consider these sets. 
            \begin{align*}
                &\text{Non-negative } \{0,1,2,...,127\}
                &\text{Negative: } \{-1,-2,-3,...,-128\}
            \end{align*}
            If we consider the cardinalities of these sets,
            \begin{align*}
                \abs{\{0,1,2,...,127\}} = 128 \\
                \abs{\{-1,-2,-3,...,-128\}} = 128
            \end{align*}
            The union of these two sets give us our 256 total 8-bit permutations.
            \bigbreak \noindent 
            \textbf{Note:} To interpret a signed hexadecimal value as an integer in a two's complement system, you must first convert it to binary. This is because the two's complement representation inherently operates on binary numbers
            \bigbreak \noindent 
            If the leftmost hex digit is $8-F$, the value is negative, whereas $0-7$  is positive.
        \item \textbf{$n$-bit addressing}: $n$-bit addressing refers to a memory addressing system where each memory address is represented using 24 bits. This addressing scheme allows a system to address up $2^{n}$ to unique memory locations.
            \bigbreak \noindent 
            The IBM System/370 uses 24-bit addressing. Thus, each memory address is represented using 24 bits (3 bytes/6 hex digits), and there are $2^{24} = 16,777,216$ unique memory locations (16mb)
            \bigbreak \noindent 
            The first byte of memory has the hex address $00 \ 00 \ 00$, the second has $00 \ 00 \ 01$, up to the address with value $2^{24}-1 $, which is $FF \ FF \ FF $
        \item \textbf{General purpose registers and relative addressing}: Main storage (memory) was introduced earlier as a hardware unit into which data could be stored in the form of binary numbers. Modern IBM computers also contain 16 general purpose registers as units for storing and manipulating data. Each register is a 32-bit storage unit whose contents can be altered or accessed in much less time than it would take to alter or access a field in main storage. Because of the difference in response time, the registers are often used as storage for frequently referenced data items or as operands in arithmetic operations.
            \bigbreak \noindent 
            The 16 registers are numbered $0,1,2,...,15$ and will be referred to as $R0, R1, ...,R15 $
            \bigbreak \noindent 
            One important use of registers involves the concept of addressing. Every byte of storage has associated with it an absolute address, and every field is addressed by the address of its leftmost byte. Recall that  the number given to a byte is called the absolute \textit{address} of the byte. It is often necessary to reference \textit{fields} that contain several bytes. The address of such a field is considered to be the address of the first (leftmost) byte in the field.
            \bigbreak \noindent 
            Whenever a program is run, it must be stored somewhere in main storage. Thus, each instruction and each item of data in the program will have an absolute address. In a program, when a reference is made to a data item or to an instruction, the computer must have an absolute address to find what is referred to.
            \bigbreak \noindent 
            A difficulty arises immediately: The writer of a program does not know where in storage his program will be stored when it is run and therefore does not  any of the relevant absolute addresses. Some other means has to be provided for referring to data or instructions within the program.
            \bigbreak \noindent 
            The means that is provided is called \textit{base-displacement addressing} or  \textit{explicit addressing}. The idea is that within  a program the relative positions of any two statements are fixed, such that if the absolute address of any one statement in the program were known, then the address of any other statement could be calculated. To do this, one would compute the distance between the statement whose absolute address is known and the one whose address is to be found. Such a distance, measured as a number of bytes, is called a \textit{displacement}.
            \bigbreak \noindent 
            But how, you ask, can the absolute address of even one statement in the program be known? For now, it is sufficient to know, when a program is executed, such an address will be contained in at least one of the general purpose registers. A register that holds an absolute address, from which the addresses of other statements can be calculated, is called a \textit{base register}.
            \bigbreak \noindent 
            The standard way, then, to refer to any data item or instruction in a program is to specify a base register and a displacement. When the instruction containing the reference is executed, the sum of the displacement and the absolute address in the base register will be calculated to obtain the absolute address of the item or instruction.
            \bigbreak \noindent 
            As an example, if the displacement specified were 4, the base register specified were $R1$, and $R1$ contained 0000007C, the absolute address formed would be 000080
            \bigbreak \noindent 
            Note that in a 24-bit environment (with 24-bit addressing), the contents of the registers will always be 32-bits, but the absolute addresses will remain 24-bit.
            \bigbreak \noindent 
            Base-displacement addresses are often specified in the format
            \begin{align*}
                D(B)
            \end{align*}
            Where $D$ is the displacement, expressed as a decimal number in the range 0 to 4095, and $B$ is the number of the base register.
            \bigbreak \noindent 
            For example, consider the base-displacement addresses.
            \begin{align*}
               4(1) \quad 20(13) \quad 0(11) 
            \end{align*}
            In the first example $4_{10} = 4_{16}$ is added to the contents of $R1$. In the second example, $20_{10}$ is converted to $14_{16}$ and added to the contents of $R13$. In the third example, the contents of $R11$ give the desired address without modification.
            \bigbreak \noindent 
            Occasionally, it is convenient to express an address as the sum of a displacement, the contents of a base register, and the contents of a second register as well. The additional register is called an \textit{index register}; the format for an address that includes an index register is
            \begin{align*}
                D(X,B)
            \end{align*}
            Where $D$ and $B$ have the same meaning as above, and $X$ is the number of the index register. Consider 
            \begin{align*}
                4(7,1)
            \end{align*}
            Is a $D(X,B)$ address, using $R7$ as the index register, $R1$ as the base register, and $4$ as the displacement. If $R1$ contains $0000007C$ and $R7$ contains $00000010$, the address calculated will be the sum of the contents of the two registers plus the displacement, or $000090$
            \bigbreak \noindent 
            It should be noted that $D(B)$ or $D(X,B)$ addresses never stand alone, they are always used as part of a particular instruction. The rules for the instruction always specify which of the two addressing formats should be used.
            \bigbreak \noindent 
            There are three details that qualify the above remarks
            \begin{enumerate}
                \item When the contents of an index register or a base register are used to calculate an address, only the value represented by the rightmost 24 bits of the register is used in the calculation. Thus, for the purpose of calculating an address, the value added from a register must always be in the range \texttt{000000} to \texttt{FFFFFF}. As an example, suppose that R1 contained \texttt{FFFFFFFF} (the encoding of $-1$). Then the effective address derived from \texttt{0(R1)} is not negative; rather it is \texttt{FFFFFF}, an extremely large absolute address.
                \item In either the \texttt{D(B)} or the \texttt{D(X,B)} address format, the registers may be omitted. This means that when a \texttt{D(B)} address is required, the programmer may use either of the following forms:
                    \begin{center}
                        \begin{tabular}{ll}
                            \textbf{Form} & \textbf{Example} \\
                            \texttt{D(B)} & \texttt{42(15)} \\
                            \texttt{D}    & \texttt{42}
                        \end{tabular}
                    \end{center}
                    Similarly, when a \texttt{D(X,B)} address is required, the programmer may use:
                    \begin{center}
                        \begin{tabular}{ll}
                            \textbf{Form} & \textbf{Example} \\
                            \texttt{D(X,B)} & \texttt{42(1,14)} \\
                            \texttt{D(X)}   & \texttt{42(1)} \\
                            \texttt{D(,B)}  & \texttt{42(,14)} \\
                            \texttt{D}      & \texttt{42}
                        \end{tabular}
                    \end{center}
                    Note that if only one register is specified in the \texttt{D(X,B)} address format, the presence or absence of a comma determines whether the register given is considered an index register or a base register.
                    \bigbreak \noindent 
                    When registers are omitted from a \texttt{D(B)} or \texttt{D(X,B)} address, the calculation of the corresponding absolute address is performed by using 0 in place of the contents of the omitted registers. Thus, either \texttt{42(14)} or \texttt{42(,14)} results in the same absolute address.
                    \texttt{42(,14)} is converted to an absolute address by adding the binary equivalent of decimal 42 to the binary number represented by the rightmost three bytes of the contents of R14.
                \item $R0$ should not ordinarily be used as an index register or as a base register. If R0 is specified in an address, it will be taken to mean that the corresponding register has been omitted. (As an example, both \texttt{4(,15)} and \texttt{4(0,15)} refer to the same absolute address.)
                \item \textbf{Note about register zero as index or base}: In assembly language (e.g., for IBM System/360 or z/Architecture), the address specification $D(X,B)$ refers to a displacement $D$ an index register $x$, and a base register $B$. When an index or base register is omitted, it is implicitly treated as 0.
                    \bigbreak \noindent 
                    Register 0 (R0) is often special in many architectures. When used as an index register (x) or base register (B), it is ignored in the address calculation, effectively contributing 0 to the address.
            \end{enumerate}
        \end{itemize}

            \pagebreak 
            \subsection{Basic concepts}
            \begin{itemize}
                \item \textbf{The function of a program}: The modern computer can perform millions of operations per second. If these operations occur in the proper sequence, meaningful results will be obtained that might have been totally inaccessible otherwise. The function of a program is to direct the order in which operations are executed within the computer.
                    \bigbreak \noindent 
                    The creation of programs that correctly direct the sequence of operations is, therefore of fundamental importance if computers are to be successfully utilized.
                \item \textbf{Type of assembly on IBM mainframes}: The language used in this text is a symbolic language called \textit{Basic Assembler Language}, this is the assembler language developed for use on IBM mainframe computers.
                    \bigbreak \noindent 
                    Note that the computer cannot directly carry out or execute assembler language instructions. Each of these instructions must be translated into a machine-language instruction before the computer can begin to perform the task that the program was intended to accomplish.
                    \bigbreak \noindent 
                    The machine language instructions that the computer can execute directly are strings of binary digits that are difficult for a human to decipher. A program consisting entirely of machine-language instructions is called an \textit{object program}.
                    \bigbreak \noindent 
                \item \textbf{The \textit{Job Control Language} (JCL)}: After an assembler language program has been prepared, a few additional instructions in JCL are added to the program and the resulting program is presented to the computer through an input device, such as a card reader or a terminal. The JCL instructions invoke a program called an \textit{assembler}, which translates the source program instructions into machine-language instructions and thus produces an object program.
                    \bigbreak \noindent 
                    If no syntax errors are detected, the object program is loaded into the main storage of the computer and execution of the program begins.
                \item \textbf{Execution of the program}: The actual execution of the program is depicted by the following algorithm
                    \begin{enumerate}
                         \item Initially, the absolute address of the first instruction of the program to be executed is inserted into a special pointer called the \textbf{Program Status Word (PSW)}.
                         \item The machine retrieves from storage the instruction that is pointed to by the PSW.
                         \item The machine then updates the contents of the PSW to point to the next instruction.
                         \item The machine executes the operation indicated by the previously retrieved instruction. If the instruction did not cause a branch to occur (a branch is caused by a basic operation analogous to a \texttt{GOTO} statement in a higher-level language), then go to \textbf{Step 2}. Otherwise, put the absolute address that is to be branched to into the PSW, and go to \textbf{Step 2}.
                    \end{enumerate}
                    \bigbreak \noindent 
                \item \textbf{R14 and R15} In ASSIST, general purpose register 15 (R15) is used to hold the address of the first byte of the program. R14 holds the return address, the address to return to after the program ends. In ASSIST virtual memory, it is simulated that the program starts at address  0
                \item \textbf{Explicit addressing}: $D(X,B)$ displacement addressing is \textit{explicit} addressing
                \item \textbf{Types of Instructions}: The process of encoding instructions can be clarified by considering some specific instructions in their symbolic and encoded forms. Instructions are encoded according to five distinct formats, RR, RX, RS, SI, and SS. The set of instructions encoded in the RR format is referred to as the \textit{RR instructions}.
                \item \textbf{RR Instructions, add register (AR)}: An RR instruction is most often used to cause an operation involving two registers.
                    \bigbreak \noindent 
                    Consider the format of the Add Register instruction
                    \begin{align*}
                        AR \quad r1,r2
                    \end{align*}
                    Execution of this instruction causes the contents of r2 to be added to the contents of r1; the contents of r2 are unaltered, unless r2 is r1. If an overflow occurs in the process of addition, a fixed point overflow condition will exist, this will normal cause termination of the program.
                    \bigbreak \noindent 
                    This symbolic form must be translated into an encoded form before the instruction is actually executed. The encoded form of any RR instruction is a 16-bit binary number that will occupy a halfword of storage in the machine when the program is executed. The encoded instruction can therefore be represented as a four digit hexadecimal number. The precise format of an encoded RR instruction is
                    \begin{align*}
                        h_{0}h_{0}h_{r1}h_{r2}
                    \end{align*}
                    Where
                    \begin{itemize}
                        \item $h_0h_0$ is a two-digit operation code specifying the purpose of the instruction.
                        \item $h_r$ is the number of the register (0-F) that is to be used as the first operand.
                        \item $h_r$ is the number of the register that is to be used as the second operand.
                    \end{itemize}
                    \bigbreak \noindent 
                    The operation code (\textbf{op code}) for an \textbf{AR} instruction is \texttt{1A}. Therefore, the encoding of:
                    \[
                        \text{AR } 14,0
                    \]
                    is \texttt{1AE0}. The \texttt{1A} means that the encoded instruction is an \textbf{AR}, the \texttt{E} indicates that the first operand is \texttt{R14}, and the \texttt{0} indicates that \texttt{R0} is the second operand.
                \item \textbf{RR Instructions, SR (Subtract register) and LR (Load register)}: Consider the instruction
                    \begin{align*}
                        SR \quad r1,r2
                    \end{align*}
                    Execution causes the number represented by the contents of r2 to be subtracted from the number represented by the contents of r1. Just as for AR, r2 is unaltered. Fixed-point overflow can occur. Next, consider
                    \begin{align*}
                        LR \quad r1,r2
                    \end{align*}
                    Causes the replacement of the original contents of r1 by the contents of r2, r2 remains unaltered.
                    \bigbreak \noindent 
                    The opcode for SR is 1B, and the opcode for LR is 18
                \item \textbf{RX instructions: L (load)}: RX instructions usually cause an operation that involves a register and main storage to be performed. For example, consider the Load instruction
                    \begin{align*}
                        r,D(X,B)
                    \end{align*}
                    Which causes the fullword in storage starting at the effective address derived from $D(X,B)$ to be loaded into $r$. The original contents of $r$ are replaced, while the fullword in storage remains unaltered
                    \bigbreak \noindent 
                    Three errors might occur during the execution of a load instruction
                    \begin{enumerate}
                        \item If the absolute address calculated from \( D(X, B) \) is not a multiple of 4 (not a fullword boundary), a \textit{specification exception} occurs.
                        \item If \( D(X, B) \) is an address greater than any actual storage address, an \textit{addressing exception} will occur. This exception can happen only if either the \( X \) or the \( B \) register contains an excessively large number.
                        \item If \( D(X, B) \) is the actual address of an area, but the address is not within the area of storage allocated to the program, a \textit{protection error} will occur.
                    \end{enumerate}
                    The encoded form of an RX instruction occupies two halfwords and conforms to the following format:
                    \[
                        h_0 h_0 h_r h_X \quad h_B h_D h_D
                    \]
                    where
                    \begin{itemize}
                        \item $h_0 h_0$ is the op code indicating the particular instruction
                        \item $h_r$ is the number of the register used as the first operand
                        \item $h_X$ is the number of the index register (0 if omitted)
                        \item $h_B$ is the number of the base register (0 if omitted)
                        \item $h_D h_D h_D$ is the displacement
                    \end{itemize}
                    Thus, since the op code of L is 58,
                    \[
                        L \quad 2, 12(1, 10)
                    \]
                    is encoded as $5821 \, \text{A00C}$. Note that the displacement used in a relative address must be in the range $0$ to $4095$ simply because $FFF_{16} = 4095_{10}$ is the largest three-digit hexadecimal number (and only three digits are provided in the encoded form of the instruction).

                \item \textbf{RX: ST (store)}: The \texttt{ST} instruction performs the inverse of the operation of the \texttt{L} instruction:
                    \[
                        \texttt{ST} \quad r, D(X, B) \quad \text{(Store)}
                    \]
                    Execution of this instruction causes the contents of \texttt{r} to replace the fullword at the location determined by \texttt{D(X, B)}; the condition of \texttt{r} remains unaltered. The absolute address corresponding to \texttt{D(X, B)} must, therefore, be on a fullword boundary. The same exceptions (errors) that can occur for an \texttt{L} instruction can also occur for an \texttt{ST} instruction.
                    \bigbreak \noindent 
                    The opcode of $ST$ is $80$
                \item \textbf{A complete program}: Given a choice, a programming would naturally choose to write programs using the symbolic instructions. The role of an assembler program is to accept as input a source program written using symbolic notations and to encode the instructions into executable  format. The precise format used to represent a symbolic instruction that can be submitted to the assembler is as follows.
                    \begin{enumerate}
                        \item A label may start in column 1. A label is a string of from one to eight characters such that:
                            \begin{enumerate}
                                \item The first character is either a letter from \texttt{A} to \texttt{Z}, a \texttt{\$}, a \texttt{\#}, or an \texttt{@}.
                                \item Each of the characters following may be any of the above or any one of the decimal digits (0, 1, 2, \dots, 9).
                            \end{enumerate}
                            Thus,
                            \begin{verbatim}
                                WORD1
                                LOOP
                                MYROUTNE
                                BRANCH01
                            \end{verbatim}
                            are valid labels. 
                        \item A mnemonic for the operation code, such as \texttt{AR} for Add Register, starts in column 10. (Appendix C contains a list of valid op codes and the corresponding mnemonics.)
                        \item The operands start in column 16.
                        \item Any comment that the user wishes to append to the instruction can occur after allowing at least one blank following the last character of the operands. The only restriction on the comment is that it cannot extend past column 71 of the input record.
                    \end{enumerate}
                    The following example of a symbolic instruction, with indications of where the fields would appear, illustrates the above points:
                    \begin{verbatim}
    1          10       16
    LOADUP     L        1,0(2,3)       LOAD THE WORD INTO R1
                    \end{verbatim}
                    \bigbreak \noindent 
                    Up to this point, the only symbolic instructions that have been discussed are those that are to be encoded into executable instructions. There are, however, several instructions, called \textit{assembler instructions}, that are not used to generate machine instructions. Rather, these instructions are used to communicate to the assembler information about how the user's program is to be processed and to direct the generation of constants and storage areas. The following assembler instructions are of particular importance:
                    \begin{enumerate}
                        \item The \texttt{CSECT} instruction is used to begin a program and must appear before any executable instructions in the program. The format of the \texttt{CSECT} instruction is:
                            \begin{verbatim}
                    label      CSECT
                            \end{verbatim}
                        \item The end of a program is signified by an \texttt{END} statement. The \texttt{END} statement has as an operand the label of the place in the program where execution of the program should begin. Normally this label is the label of the \texttt{CSECT} statement. For example, the \texttt{CSECT} and \texttt{END} statements would typically occur as:
                            \begin{verbatim}
                    MYPROG      CSECT
                                .
                                .
                                .
                                END  MYPROG
                            \end{verbatim}
                        \item A \texttt{DC} statement is used to define constants. The only form of a \texttt{DC} statement that is needed at this point is:
                            \begin{verbatim}
                    label      DC      mF'n'
                            \end{verbatim}
                            where 
                            \begin{itemize}
                                \item \texttt{m} is a nonnegative decimal integer (duplication factor)
                                \item \texttt{n} is a decimal integer
                            \end{itemize}

                            If \texttt{m} is omitted, it is assumed to be 1. The effect of this \texttt{DC} statement is to cause \texttt{m} consecutive fullwords, each containing the number \texttt{n}, to be generated. Thus,
                            \begin{verbatim}
                    TWO       DC      F'2'
                            \end{verbatim}
                            causes one fullword containing \texttt{00000002} to be generated and makes it accessible to the program by its symbolic name \texttt{TWO}. \texttt{DC} statements can appear anywhere between the \texttt{CSECT} and \texttt{END} statements in the program. The fullwords that are generated as the result of such \texttt{DC} statements will always begin on fullword boundaries.
\item A \texttt{DS} statement is used to set aside areas of storage for fullwords that require no initial values. The format of the \texttt{DS} statement is:
    \begin{verbatim}
                    label      DS      mF
    \end{verbatim}
    The above statement causes the assembler to set aside the next \texttt{m} fullwords as a storage area. An area of \texttt{n} bytes, which may or may not begin on a fullword boundary, can be set aside by use of a \texttt{DS} statement in the following format:
    \begin{verbatim}
                    label      DS      CLn
    \end{verbatim}
    Here, \texttt{CLn} stands for Character, Length \texttt{n}. Again, if more than one such area is required, a duplication factor can be used. For example, the statement:
    \begin{verbatim}
                    label      DS      mCLn
    \end{verbatim}
    causes \texttt{m} fields, each of which has a length of \texttt{n} bytes, to be set aside.

\end{enumerate}
    Before looking at a sample program, two additional details should be mentioned:
    \begin{itemize}
        \item When execution of a program begins, \texttt{R15} will always contain the absolute address of the beginning of the program. For now, this can be assumed to mean that \texttt{R15} contains the address of the first generated instruction in the program. This fact is significant since it allows all relative addresses to be created using R15 as a base register
        \item When execution of a program begins, R14 contains the absolute address of the routine that should be branched to when execution of the program has been completed. That is, when execution of a program terminates, branch to the address in R14 should be effected. This is referred to as \textit{exiting} from the program. Although branching instructions are not covered until later in this text, the format of the branch instruction that causes the exit is
            \bigbreak \noindent 
            \begin{center}
    label BCR B'1111',14     
            \end{center}
    \end{itemize}
    \textbf{Note:} The encoding of BCR B'1111',14 is 07FE (important when we look at dumps)
                \item \textbf{Sample program}:
                    \begin{verbatim}
    ADD2     CSECT 
             L     1,16(,15)
             L     2,20(15)
             AR    1,2
             ST    1,24(15)

             BCR   B'1111',14
             DC    F'4'
             DC    F'6'
             DS    F
             END   ADD2
                    \end{verbatim}
                   Note that any input record with an asterisk in column 1 is a comment. The comment can be anywhere form column 2 through column 71. 
                   \bigbreak \noindent 
                   The actual instructions in this program that will be translated into executable code are
                   \bigbreak \noindent 
                   \begin{verbatim}
                   L     1,16(,15)
                   L     2,20(15)
                   AR    1,2
                   ST    1,24(15)
                   \end{verbatim}
                   \bigbreak \noindent 
                   The assembler will convert these into machine code and produce a listing of the following form.
                   \begin{center}
                       \begin{tabular}{|c|c|c|l|}
                           \hline
                           \textbf{LOC} & \textbf{OBJ CODE} & \textbf{SOURCE} & \textbf{STATEMENT} \\ \hline
                           000000       & 5810 F010         & L               & 1,16(15)           \\ \hline
                           000004       & 582F 0014         & L               & 2,20(15)           \\ \hline
                           000008       & 1A12              & AR              & 1,2                \\ \hline
                           00000A       & 501F 0018         & ST              & 1,24(15)           \\ \hline
                           00000E       & 07FE              & BCR             & B'1111',14         \\ \hline
                       \end{tabular}
                   \end{center}
                   \bigbreak \noindent 
                   In the first instruction, R15 is used as a base register. In the other RX instructions no base register is used, and R15 is used as an index register. When only one of the two registers is specified in a D(X,B) address, it makes no difference in the execution of the program whether it is used as the X or the B register.
                   \bigbreak \noindent 
                   The constants are generated starting at location \texttt{000010} and the listing produced from the \texttt{DC} and \texttt{DS} statements is:
                   \begin{center}
                       \begin{tabular}{|c|c|c|l|}
                           \hline
                           \textbf{LOC}   & \textbf{OBJ CODE} & \textbf{SOURCE} & \textbf{STATEMENT} \\ \hline
                           000010         & 00000004          & DC              & F'4'               \\ \hline
                           000014         & 00000006          & DC              & F'6'               \\ \hline
                           000018         &                   & DS              & F                  \\ \hline
                       \end{tabular}
                   \end{center}
                   Note that the \texttt{DS} statement generates no object code.
                   \bigbreak \noindent 
                   The \texttt{END} statement marks the end of the program and contains the label of the starting point as the operand. Normally, this is the label that is used in the \texttt{CSECT} statement.
                \item \textbf{The using statement}: The USING directive is used in IBM's System/360 Assembly Language (including the ASSIST simulator) to tell the assembler which register should be used as a base register for address resolution. It enables the assembler to generate base-displacement addressing automatically.
                    \bigbreak \noindent 
                    \begin{verbatim}
                    PROGRAM1  CSECT 
                              USING  PROGRAM1,15
                    \end{verbatim}
                    \bigbreak \noindent 
                    Sets us \textit{addressability} off of $R15$. In this way, we can use named vars instead of $D(X,B)$
                \item \textbf{Intro to LTORG and using named variables}: The LTORG directive forces the assembler to dump the current literal pool (a set of constants stored in memory) at that point in the program.
                    \bigbreak \noindent 
                    If we wrote the sample program above as
                    \bigbreak \noindent 
                    \begin{verbatim}
                ADD2     CSECT 
                         USING ADD2,15

                         L     1,NUM1
                         L     2,NUM2
                         AR    1,2
                         ST    1,RESULT

                         BCR   B'1111',14

                         LTORG

                NUM1     DC    F'4'
                NUM2     DC    F'4'
                RESULT   DS    F

                         END   ADD2
                    \end{verbatim}
                    \bigbreak \noindent 
                    Then we see we can used the named variables instead of directly calculating the $D(X,B)$ address. This form of addressing is called \textit{implicit} addressing. In doing this, any additional instructions that may shift the bytes of our constants won't force us to recalculate the $D(X,B)$ addresses
                \item \textbf{Note about LTORG}: The LTORG directive needs to start at a double word boundary. If LTORG does not naturally start at a double word boundary, it will go to the next available. This means we might get "slack bytes", which are unused bytes from the end of the instruction before the LTORG up until the double word boundary where the LTORG begins.
                \item \textbf{Implicit and explicit addressing}: A $D(X,B)$ displacement (also called relative) address is termed \textit{explicit}, whereas using a label is \textit{implicit}
                \item \textbf{The location counter (LOC)}: The location counter is a hidden variable maintained by the assembler that tracks the current address where instructions and data are assembled in memory.
                    \bigbreak \noindent 
                    In ASSIST, the LOC starts at 000000, it increments as instructions and data are defined.
                \item \textbf{Literals in IBM Z/OS and ASSIST}: Consider the add RX instruction, we could write
                    \bigbreak \noindent 
                    \begin{cppcode}
                    A R,D(X,B)
                    \end{cppcode}
                    \bigbreak \noindent 
                    Adds the contents found at the relative address $D(X,B)$ to the contents of the given register $R$. But can also replace the relative address with a \textit{literal}. Consider 
                    \bigbreak \noindent 
                    \begin{cppcode}
                    A 5,=F'3'
                    \end{cppcode}
                    \bigbreak \noindent 
                    F'3' specifies a fullword (4-byte) integer constant with the value 3. The assembler automatically places literals in a literal pool (usually at the end of a program block). The instruction fetches the value from memory, not from a register.
                    \bigbreak \noindent 
                    If you need to use the same literal multiple times, you have two options:
                    \begin{enumerate}
                        \item Use the Same Literal Again (Assembler Handles It) The assembler will store only one instance of =F'3' in the literal pool, even if you use it multiple times:
                            \bigbreak \noindent 
                            \begin{cppcode}
                                A 6,=F'3'   ; Adds 3 to Register 6
                                A 7,=F'3'   ; Adds 3 to Register 7 (same literal, no duplicate in memory)
                            \end{cppcode}
                            \bigbreak \noindent 
                        \item If you need direct access to the constant, define it explicitly with a DC statement
                    \end{enumerate}
                    \bigbreak \noindent 
                    \textbf{Note:} We note that the $F$ in a DC statement like \texttt{DC F'...'} guarantees that the full word will be placed at the beginning of a fullword boundary
                \item \textbf{A few basic RR and RX instructions}
                    \bigbreak \noindent 
                    Suppose we have the named variable
                    \bigbreak \noindent 
                    \begin{cppcode}
                    NUM1    DC F'36'
                    \end{cppcode}
                    \bigbreak \noindent 
                    \textbf{RR (2 Byte instructions):}
                    \begin{itemize}
                        \item \textbf{LR (Load register)}: \texttt{LR 1,2} load contents of register two into register one
                        \item \textbf{AR (Add register)}: \texttt{AR 1,2} Add contents of R2 into and over contents of R1
                        \item \textbf{SR (Subtract register)} \texttt{SR 1,2} Add contents of R2 into and over contents of R1
                    \end{itemize}
                    \bigbreak \noindent 
                    \textbf{RX (4 Byte instructions):}
                    \begin{itemize}
                        \item \textbf{L (Load)}: \texttt{L 7,NUM1} Load contents of NUM1 into register 7
                        \item \textbf{ST (Store)}: \texttt{ST 10,NUM1} Store contents of R10 into NUM1
                        \item \textbf{LA (Load address)}: \texttt{LA 3,NUM1} Load address of the first byte of NUM1 into R3
                        \item \textbf{A (Add)}: \texttt{A 4,NUM1} add contents of NUM1 to contents of R4
                        \item \textbf{S (Subtract)}: \texttt{S 9,NUM1} subtract NUM1 from R9
                    \end{itemize}
                    
                \item \textbf{Note about opcodes}: If the opcode has rightmost hex digit
                    \begin{itemize}
                        \item 0-3: 2 byte instruction
                        \item 4-B: 4 byte instruction
                        \item C-F: 6 byte instruction
                    \end{itemize}
                \item \textbf{Note about fullword boundarys}: An address is at the start of a fullword boundary if the last digit is $0,4,8,$ or $C $
                \item \textbf{Note about double word boundarys}: A double word boundary is an address that ends with either 0 or 8.
                    \bigbreak \noindent 
                    CSECTS (program start) and LTORGS always begin on a dword boundary. In ASSIST CSECT will begin at a dword boundary no matter what because our program starts virtually at address zero (which is a doubleword boundary).
                \item \textbf{A note about literals}: Consider the instruction
                    \begin{cppcode}
                    L 3,=F'235'
                    \end{cppcode}
                    This takes the decimal literal 235 and loads it into register 3. However, this is a sloppy way to do it. This instruction uses 8 bytes, 4 for the encoded instruction, and 4 for the fullword literal. Instead, we can do
                    \bigbreak \noindent 
                    \begin{cppcode}
                    LA 3,235
                    \end{cppcode}
                    Since LA needs a $D(X,B)$ address, 235 is shorthand for $235(0,0)$. Since we ignore R0 in the address calculation, the above instruction loads the "address" $235_{10} = EB_{16}$ (plus nothing because we ignore R0) into R3. In this way, we skip needing to store the literal 235 in storage. Therefore, the above instruction only uses 4 bytes instead of 8.
                    \bigbreak \noindent 
                    Further, consider a situation where we need to "zero out" a register. We could write
                    \bigbreak \noindent 
                    \begin{cppcode}
                    L 8,=F'0'
                    \end{cppcode}
                    \bigbreak \noindent 
                    Which replaces the contents of R8 with zero, effectively "zeroing out" the registers contents. Again, the above instruction uses 8 bytes. Instead, we could write
                    \bigbreak \noindent 
                    \begin{cppcode}
                    SR 8,8
                    \end{cppcode}
                    \bigbreak \noindent 
                    Which subtracts the contents of register 8 from register 8. This instruction therefore only uses two bytes... The two bytes required to encode the instruction.
                \item \textbf{A note about define storage}: If we specify a value in a DS instruction, the value wont be placed in storage. It will behave just like a regular DS.
                \item \textbf{Note about instructions Load and Store}: The instructions L and ST require arguments at fwb's (full word boundarys), failure to supply a value at a fwd results in a ABEND (SOC6), which is a \textit{specification exception}
                \item \textbf{Note about assist}: The ASSIST assembler fills are bytes in storage with a default value $F5$, and non special registers with value $F4F4F4$
                \item \textbf{XDUMP}: If we write an XDUMP instruction with no operands, we get a output dump of the contents of all the registers. If we specify a location in memory, and a number of bytes after that location, we get those requested bytes but they will be buried within the 32 byte line in which they are contained. (Dump gives us 32 byte line starting at previous 32 byte boundary).
                \item \textbf{Assist and ABENDS}: In ASSIST, ABENDS are called \textit{assist completion dumps}
                    \bigbreak \noindent 
                    Note that a return code of zero means that assist finished successfully, the return code will always be zero even if the program abends
                \item \textbf{Assembler (compilation) errors and execution (runtime errors)}: An abend is always an execution (runtime) error, we cannot get abends during compilation.
                    \bigbreak \noindent 
                    The first pass of the assembler looks for syntax errors and addressability errors. The program will not compile if everything in the program is not addressable.
                \item \textbf{Dumps and the PSW}: A dump of storage will begin by specifying the address of the first byte of the dumped storage
                    \bigbreak \noindent 
                    An ABEND dump will dump the registers contents, the storage that our program occupies, and the PSW (program status word). The PSW is 8 bytes. There is one PSW per cpu and each instruction changes the contents. 
                    \bigbreak \noindent 
                    The first two bytes of the PSW we should not worry about. The last two bytes of the first four bytes contain the interruption code, the ones we should encounter at this point are the following
                    \begin{itemize}
                        \item \textbf{S0C 1 (Operation exception)}: Invalid opcode / instruction
                        \item \textbf{S0C 4 (Protection exception)}: Good address, outside the scope of the program
                        \item \textbf{S0C 5 (Addressing exception)}: Bad address
                        \item \textbf{S0C 6 (Specification exception)}: Happens for a number of reasons, most commonly when you an instruction requires storage on a fullword boundary, but we give storage that is not on a fullword boundary. Load and store are examples of instructions that require the second operand to be on a fullword boundary
                        \item \textbf{S0C 9 (Fixed-point divide exception)}: divide by zero
                        \item \textbf{S0C B (Decimal-divide exception)}: divide by zero
                    \end{itemize}
                    \bigbreak \noindent 
                    The next piece of information that is of concern to us is the first 4 bits of the fifth byte of the PSW. The first two bits are ILC (instruction length count), which is the length of the instruction that caused the abend (in halfwords). Therefore, to get the length of the instruction in bytes, we multiply by two. The last two bits is the condition code set by the last successful instruction that sets the condition code. 
                    \bigbreak \noindent 
                    To process the information of the first four bits of the 5th byte of the PSW, we take the hex number and convert to binary, the split into two different 2 bit binary numbers. Since the ILC has max value $11$, the biggest an instruction is therefore $3(2) = 6$ bytes. Suppose we our four bits are
                    \begin{align*}
                        1000
                    \end{align*}
                    We then split into two different 2-bit binary numbers
                    \begin{align*}
                        10 \quad 00
                    \end{align*}
                    The ILC is therefore $10$, which means instruction length $2(2) = 4$ bytes. And, the condition code is 0.
                    \bigbreak \noindent 
                    The last 3 bytes of the 8 byte PSW is the next instruction address, which is the address of the instruction following the one that caused the abend. 
                \item \textbf{Uses a register as a counter}: Suppose we want to register 3 as a counter variable. To begin, we clear out its contents 
                    \bigbreak \noindent 
                    \begin{cppcode}
                    SR 3,3 // SR R,R to clear contents of R
                    \end{cppcode}
                    \bigbreak \noindent 
                    Then, to increment the count. That is, to add one to its contents, we do
                    \bigbreak \noindent 
                    \begin{cppcode}
                    LA 3,1(,3)  // General: LA R,1(,R)
                    \end{cppcode}
                    \bigbreak \noindent 
                    Suppose this is the first time we increment the registers contents. Thus, before the LA, its contents will be zero. LA 3,1(,3) loads the "address" one byte off of register 3. Since the contents are zero, one byte off of register three is 1.
                    \bigbreak \noindent 
                    After the first increment, the registers contents are now two. A second LA will load the "address" 1 byte off of 1 (R3s contents), which is two.
                \item \textbf{Condition codes and the bitmasks in branches}: Some instructions set the condition code. Recall from analysis of the PSW that the condition code can either be 0,1,2, or 3 (cc is 2 bits). When we see something like B'0111' in a branch, what we are looking at is a \textit{bitmask}. The bitmask B'0111' deals with the condition code, it says that if the condition code is zero, don't branch. If the condition code is 1,2, or 3, branch.
                    \begin{align*}
                        B' \underbrace{0}_{\begin{array}{c}
                                \text{Don't} \\
                                \text{branch} \\
                                \text{if cc}\\
                                \text{is zero}
                        \end{array}}
                        \underbrace{1}_{\begin{array}{c}
                                \text{Branch } \\
                                \text{if cc} \\
                                \text{is one}
                        \end{array}}
                        \underbrace{1}_{\begin{array}{c}
                                \text{Branch } \\
                                \text{if cc} \\
                                \text{is two}
                        \end{array}}
                        \underbrace{1}_{\begin{array}{c}
                                \text{Branch } \\
                                \text{if cc} \\
                                \text{is three}
                        \end{array}} \ '
                    \end{align*}
                \item \textbf{Read from files and process data}: In ASSIST assembly, to read data from files we use three X type (assist) instructions
                    \begin{itemize}
                        \item \textbf{XREAD}
                        \item \textbf{XPRNT}
                        \item \textbf{XDECI}
                    \end{itemize}
                    \bigbreak \noindent 
                    Consider the following assembler program that reads data from a file and outputs
                    \bigbreak \noindent 
                    \begin{cppcode}
                            XREAD RECORD,80
                    LOOP1   BC    B'0111', ENDLOOP1
                            XPRNT DETAIL,133
                            XREAD RECORD,80
                            BC    B'1111',LOOP1
                    ENDLOOP1
                            LTORG
                    RECORD  DS    CL80
                    \end{cppcode}
                    \bigbreak \noindent 
                    Note that XREAD sets the condition code. Zero if successful, 1,2, or 3 if we are at the end of the file (no more reading to do)
                    \bigbreak \noindent 
                    We begin by doing a preliminary read, This ensures our file is not empty, if the file is empty, we don't loop. The condition code will be nonzero, and we take the branch to ENDLOOP1 immediately
                    \bigbreak \noindent 
                    The LOOP1 is a label, which allows us to execute branch instructions to that label. Notice that we defined storage named record, which is character length 80, each character is one byte, the total buffer is therefore 80 bytes (or 80 characters) long. For each XREAD, this buffer reads 80 bytes of the file, and overwrites the contents of the RECORD buffer. At the beginning of each loop iteration, we check to see if the condition code set by the last XREAD was zero. If it was, we do not branch to ENDLOOP1 (thus continuing the loop). 
                    \bigbreak \noindent 
                    If the loop enters, we XPRNT (more on xprnt later), then read again, and branch unconditionally to the loop condition check
                    \bigbreak \noindent 
                    After each XREAD, our RECORD buffer is filled with space separated data. We use XDECI to process the data in the RECORD buffer. XDECI allows us to read numbers from an input record and put the numbers into registers.
                    \bigbreak \noindent 
                    \begin{cppcode}
                    XDECI 2,RECORD
                    \end{cppcode}
                    \bigbreak \noindent 
                    XDECI starts at the specified address (RECORD in this case), and reads digits until it finds a space character. It then converts what it found to binary and stores in the specified register (R2 in this case). It also puts the address of the space it found into R1. Thus, to continue, we would do
                    \bigbreak \noindent 
                    \begin{cppcode}
                    XDECI 3,0(1)
                    \end{cppcode}
                    \bigbreak \noindent 
                    Start the next XDECI zero bytes off of R1 (the address of the space it found), and store findings in R3 this time.
                \item \textbf{XDECO and XPRNT}: We use XDECO and XPRNT to create \textit{print lines}. Consider the following example
                    \bigbreak \noindent 
                    \begin{cppcode}
                            L      3,NUM1
                            L      4,NUM2
                            XDECO  3,ONUM1
                            XDECO  4,ONUM2
                            AR     4,3
                            XDECO  4,OSUM1
                            XPRNT  DETAIL,133
                    *
                            LTORG
                    DETAIL  DC     C'0'       Carriage control char
                    ONUM1   DS     CL12       Character length 12 storage
                            DC     5C' '      Repeat 5 spaces
                    ONUM2   DS     CL12
                            DC     5C' '
                    OSUM    DS     CL12
                            DC     86C' '     Fill the remaining of 133 byte print line with space
                    \end{cppcode}
                    \bigbreak \noindent 
                    Note that print lines should always occupy 133 bytes. $1 + 12 + 5 + 12 + 5 + 12 + 86 = 133$, note that the first byte is the carriage control. A carriage control character zero is to double space the print lines, two blanks between each line
                    \bigbreak \noindent 
                    \textbf{Carriage controls:}
                    \begin{itemize}
                        \item \textbf{C' '}: Single space 
                        \item \textbf{C'0'}: Double space 
                        \item \textbf{C'-'}: Triple space
                        \item \textbf{C'1'}: Top of next page
                    \end{itemize}
                    \bigbreak \noindent 
                    Let's first discuss the storage. The first byte specifies the carriage control. We define 12 character bytes for each output variable, with 5 space characters in between each output variable. After the 12 character bytes are allocated for OSUM1, and the 5 padding bytes after OSUM1, we have 86 remaining bytes to fill, so we use spaces.
                    \bigbreak \noindent 
                    Regarding the top instructions assume NUM1, NUM2 are defined, we load NUM1 and NUM2 into registers 3 and 4. Then, we XDECO R3 into ONUM1, R4 into ONUM2, and after AR modifies R4, we XDECO the new contents of 4 into OSUM1. What XDECO does is converts the contents to decimal, and puts them into the character buffer right justified.
                    \bigbreak \noindent 
                    XPRNT then starts at the carriage control (DETAIL), and prints a total of 133 bytes formatted.
                    \bigbreak \noindent 
                    We could also define headers and column names
                    \bigbreak \noindent 
                    \begin{cppcode}
                        XPRNT HEADER1,133 
                        XPRNT COLHDR1,133     PRINT COLUMN HEADER 1 
                        XPRNT HYPHENS1,133    PRINT HYPHENS 
                        XPRNT DETAIL,133      PRINT DETAIL LINE 

                        HEADER1  DC    C'1' 
                        DC    57C' ' 
                        DC    C'HERE IS MY REPORT' 
                        DC    58C' ' 
                        * 
                        COLHDR1  DC    C'0'              CARRIAGE CONTROL CHARACTER 
                        DC    C'        NUM1'   OUTPUT AREA FOR NUM1 
                        DC    5C' '             SPACES 
                        DC    C'        NUM2'   OUTPUT AREA FOR NUM2 
                        DC    5C' '             SPACES 
                        DC    C'         SUM'   OUTPUT AREA FOR THE SUM 
                        DC    86C' '            SPACES 
                        * 
                        HYPHENS1 DC    C' '              CARRIAGE CONTROL CHARACTER 
                        DC    C'------------'   OUTPUT AREA FOR NUM1 
                        DC    5C' '             SPACES 
                        DC    C'------------'   OUTPUT AREA FOR NUM2 
                        DC    5C' '             SPACES 
                        DC    C'------------'   OUTPUT AREA FOR THE SUM 
                        DC    86C' '            SPACES
                    \end{cppcode}
                    \bigbreak \noindent 
                    \textbf{Note:} XDECO requires 12 bytes.
                \item \textbf{Even odd pairs}: An even odd pair of registers are two adjacent registers in which the number of the first register is even and the second is odd, we have the even odd pairs
                    \begin{align*}
                        0,1
                        2,3
                        4,5
                        6,7
                        8,9
                        10,11
                        12,13
                        14,15
                    \end{align*}
                    Even odd pairs are used in some instructions in order to 64-bit "Register" in which the two registers are combined
                \item \textbf{Multiplication and division}: We have the instructions
                    \begin{cppcode}
                    MR  R,R
                    M   R,D(X,B)
                    DR  R,R
                    D   R,D(X,B)
                    \end{cppcode}
                    \bigbreak \noindent 
                    Note that the first operand in all the above instructions are even odd pair registers. Thus, the left operand is an even register in $0-14$. The second operand is either an fullword boundary address (M and D), or any register (MR, DR)
                    \bigbreak \noindent 
                    Let's first consider the MR instruction.
                    \bigbreak \noindent 
                    \begin{cppcode}
                    LA   3,4
                    LA   7,3
                    MR   2,7      Multiply 4 by 3
                    \end{cppcode}
                    \bigbreak \noindent 
                    First to note that the even odd pair register R2 used as the first operand to the MR instruction takes R2 and R3, and combines them to make a 64-bit register in which to store the result of the multiplication. Thus, even though the target register is R2, the result will be completely contained within R3 unless the result was big enough to also need to use R2. If the result was not big enough to use R2, the contents of R2 are zeroed out.
                    \bigbreak \noindent 
                    Next, we consider division. Let's divide 24 by 6
                    \bigbreak \noindent 
                    \begin{cppcode}
                    LA  5,24
                    LA  9,6
                    M   4,=F'1'   Need to use M to extend the sign bit of R5 into R4
                    DR  4,9
                    \end{cppcode}
                    \bigbreak \noindent 
                    The remainder in this case will be stored in the even register, and the quotient in the odd register.
                \item \textbf{Note about multiplication and division}: In multiplication, the even register can be junk. But, with division, the even reg must have extension of the sign bit of the odd register. Thus, the even register must be filled with all zeros or all ones. This is why we multiply the even register by one before we divide. 
                \item \textbf{Defining label that does not take up storage}: 
                    \bigbreak \noindent 
                    \begin{cppcode}
                    LB   DS  0H
                    \end{cppcode}
                    \bigbreak \noindent 
                    We define storage of zero halfwords... Ie it will not take up any storage and will not increase the LOC. Note that LB is simply the name of the label
                    \bigbreak \noindent 
                    you're not actually reserving any memory; you're simply associating the label LB with the current location counter value. This label is stored in the assembler's symbol table. The symbol table holds the address (which is the current location counter at the time of definition) for use during assembly and linking, but it does not correspond to any allocated memory in the final object code.
                \item \textbf{Compares and branching}: We have the two instructions
                    \begin{itemize}
                        \item \textbf{RX: C (compare)}:
                            \bigbreak \noindent 
                            \begin{cppcode}
                            C   R,D(X,B)
                            \end{cppcode}
                            \bigbreak \noindent 
                            Compares the contents of R and the storage at address D(X,B)
                            \bigbreak \noindent 
                            Note that D(X,B) must be on a fullword boundary (FWB)
                        \item \textbf{RR: CR (compare register)} 
                            \bigbreak \noindent 
                            \begin{cppcode}
                            CR  R,R 
                            \end{cppcode}
                            \bigbreak \noindent 
                            Compares the contents of the specified registers
                    \end{itemize}
                    \bigbreak \noindent 
                    The condition code will be
                    \begin{itemize}
                        \item \textbf{Zero:} if $a=b$
                        \item \textbf{One} if $a<b$
                        \item \textbf{Two} if $a>b$
                    \end{itemize}
                    Note that the condition code can be at most 3, but C and CR never set the CC to 3.
                    \bigbreak \noindent 
                    Suppose we want to compare two registers, and branch if the condition code is note zero (values were not equal), we would write
                    \bigbreak \noindent 
                    \begin{cppcode}
                            CR      2,3             Compare registers 2 and 3
                            BC      B'0111',NOTEQ

                    NOTEQ   DS      0H
                    \end{cppcode}
                    \bigbreak \noindent 
                    Or, using BCR, we could write
                    \bigbreak \noindent 
                    \begin{cppcode}
                            LA      7,NOTEQ
                            CR      2,3             Compare registers 2 and 3
                            BCR      B'0111',7

                    NOTEQ   DS      0H
                    \end{cppcode}
                \item \textbf{Defining or decaling character strings}: We have
                    \bigbreak \noindent 
                    \begin{cppcode}
                    NAME        DC      CLn'...'   String of length n
                    NAME        DC      C'...'     Single char
                    NAME        DC      C'...'     Char str, size is the size of the specified string
                    NAME        DC      nC'...'    Repeat Single char n times
                    \end{cppcode}
                    \bigbreak \noindent 
                    And of course we also have their DS counterparts
                    \bigbreak \noindent 
                    The encoding of these strings follows EBCDIC char encodings. The string \textit{smith} is encoded to E2 D4 C9 E3 C8
                \item \textbf{Copying and comparing strings}: We do not use C or CR on character strings, instead we use the 6 byte SS (storage to storage) instructions
                    \begin{itemize}
                        \item \textbf{MVC (Move characters)}: Copies the characters from source to destination
                            \bigbreak \noindent 
                            \begin{cppcode}
                            MVC     D(L,B),D(B)
                            MVC     DEST(len),SRC
                            \end{cppcode}
                            \bigbreak \noindent 
                            Where the second operand is the source, and the first is the destination. Note that the provided length is used for \textbf{both} operands
                            \bigbreak \noindent 
                            \textbf{Note:} In this form it is assumed that the length of both operands is the same, more on the other forms later.
                        \item \textbf{CLC (Compare logical characters)}: Byte by byte lexicographical compare based on the EBCDIC HEX encodings, sets the condition code same as C and CR
                            \bigbreak \noindent 
                            \begin{cppcode}
                            CLC     D(L,B),D(B)
                            CLC     LABEL1(length),LABEL2
                            \end{cppcode}
                    \end{itemize}
                \item \textbf{Encoding of MVC and CLC}: The encoding of the forms described above is 
                    \bigbreak \noindent 
                    \begin{align*}
                        h_{0}h_{0}h_{L}h_{L} \quad h_{B1}h_{D1}h_{D1}h_{D1} \quad h_{B2}h_{D2}h_{D2}h_{D2}
                    \end{align*}
                    Where
                    \begin{itemize}
                        \item $h_{0}h_{0}$ is the opcode
                        \item $h_{L}h_{L}$ is the length minus one 
                        \item $h_{B1}h_{D1}h_{D1}h_{D1}$ Base and displacement of the first operand
                        \item $h_{B2}h_{D2}h_{D2}h_{D2}$ Base and displacement of the second operand
                    \end{itemize}
            \item \textbf{Important EBCDIC encodings to remember}:
                \begin{itemize}
                    \item \textbf{Space character}: 40
                \end{itemize}
            \item \textbf{LTR (load and test register)}: The LTR instruction loads into a register and sets the condition code based on its value. The CC will be 
                \begin{enumerate}
                    \item \textbf{Zero}: If the loaded value is zero
                    \item \textbf{One} If the loaded value is negative.
                    \item \textbf{Two}: If the loaded value is positive
                    \item \textbf{Three:} If there is overflow
                \end{enumerate}
                Thus, we can test if a register has a value of zero with
                \bigbreak \noindent 
                \begin{cppcode}
                LTR R,R
                \end{cppcode}
                \bigbreak \noindent 
                Then, we can test the condition code accordingly.
            \item \textbf{If-endif statements}: We can write an "if-endif" statement by using branches
                \bigbreak \noindent 
                \begin{cppcode}
                        BC      B'bbbb', OVER
                                ...
                                ...
                                ...
                OVER    DS      0H
                \end{cppcode}
                \bigbreak \noindent 
                This code branches if register 8 contains zero.
            \item \textbf{If-else-endif statements}: Consider
                \bigbreak \noindent 
                \begin{cppcode}
                        BC      B'bbbb',ELSE
                                ...
                                ...
                                ...
                        BC      B'1111',ENDIF
                ELSE    DS      0H
                                ...
                                ...
                                ...
                ENDIF   DS      0H
                \end{cppcode}
            \item \textbf{Extended mnemonics}: Shorten common instructions. Note that these are not real instructions, but translate to equivalent instructions. A few common ones are
                \bigbreak \noindent 
                \begin{cppcode}
                BCR     B'1111',14
                BR      14
                \end{cppcode}
                \bigbreak \noindent 
                Branch non-zero:
                \bigbreak \noindent 
                \begin{cppcode}
                BC    B'0111',LABEL
                BNZ   LABEL
                \end{cppcode}
                \bigbreak \noindent 
                Branch zero:
                \bigbreak \noindent 
                \begin{cppcode}
                BC    B'1000',LABEL
                BZ    LABEL
                \end{cppcode}
                \bigbreak \noindent 
                Branch:
                \bigbreak \noindent 
                \begin{cppcode}
                BC    B'1111'LABEL
                B     LABEL
                \end{cppcode}
        \item \textbf{For loop}: We can use BCTR or BCT...
        \item \textbf{Decrementing a register}: We use bctr
            \bigbreak \noindent 
            \begin{cppcode}
            BCTR  R,0
            \end{cppcode}
            \bigbreak \noindent 
            Decrements R by 1
        \item \textbf{EQU}: EQU, or EQUates, assigns a value to a label
            \bigbreak \noindent 
            \begin{cppcode}
            label    EQU    Expression
            \end{cppcode}
            \bigbreak \noindent 
            gives a label the value of the expression. Every occurrence of label will be treated as if it was the expression.
            \bigbreak \noindent 
            Equates are either typed above the CSECT or below the END
            \bigbreak \noindent 
            \begin{cppcode}
            LOAD  EQU  L
            \end{cppcode}
            \bigbreak \noindent 
            Then, a load instruction can be written as
            \bigbreak \noindent 
            \begin{cppcode}
            LOAD   3,NUM1
            \end{cppcode}
            \bigbreak \noindent 
            Before assembling the code, the Assembler replaces the label of the equates with the expression.
    \end{itemize}

    \pagebreak 
    \subsection{Decimal arithmetic}
    \begin{itemize}
        \item \textbf{Intro}: We are now going to learn to do arithmetic in decimal instead of binary arithmetic using registers and fullwords in storage
            \bigbreak \noindent 
            To do this, we use a numeric format called packed decimal which is also referred to as "packed numbers" or simply "decimal" for short
            \bigbreak \noindent 
            Numbers are in storage in decimal so numbers in dumps are easier to read. We can use larger numbers - up to 31 decimal digits! 
            \bigbreak \noindent 
            Also, both operands are in storage so we don't need to use registers. Storage to storage arithmetic means fewer instructions to do arithmetic, i.e., no loading values into registers, storing them, etc.
        \item \textbf{Two decimal formats}:
            \begin{itemize}
                \item \textbf{Zoned Decimal Format}: The number is represented in almost character format (EBCDIC), similar to what we get after an XDECO
                    \bigbreak \noindent 
                    Used for getting input and for formatting output
                    \bigbreak \noindent 
                    For example, 4-byte zoned decimal number: F1F2F8F9 = 1289$_{10}$. One zoned decimal digit per byte. Within the byte, the left hex digit is the zone digit, and the right hex digit is the numeric digit.
                    \bigbreak \noindent 
                    The zone digit of the rightmost byte is used to represent the sign
                    \begin{itemize}
                        \item If the sign digit is A, C, E or F, the number is positive.
                        \item If the sign digit is B or D, the number is negative
                    \end{itemize}
                    \bigbreak \noindent 
                    F3F8F4F3 is equivalent to 3843. A zoned decimal number looks a lot like the character, or EBCDIC, representation of the number. The difference is the sign zone digit in the first hex digit of the last byte.
                    \bigbreak \noindent 
                    \textbf{Note:} You cannot do arithmetic with zoned decimal numbers! Zoned Decimal Numbers
                \item \textbf{Packed Decimal Format}: the format used for doing arithmetic in decimal.
            \end{itemize}
        \item \textbf{Declaring zoned decimal numbers}
            \bigbreak \noindent 
            \begin{cppcode}
            label    dc    mZLn'p'
            \end{cppcode}
            \bigbreak \noindent 
            Where $n$ is the length in bytes, and $m$ is how many times to repeat (1 if left blank).
            \bigbreak \noindent 
            \begin{center}
                \begin{tabular}{p{4cm}|p{4cm}}
                    Declaration &Generates\\
                    \hline
                    NUM1 DC 3ZL3'123' &F1F2C3 F1F2C3 F1F2C3 \\
                    NUM2 DC 3Z'123' &F1F2C3 F1F2C3 F1F2C3 \\
                    NUM3 DC ZL3'-123' &F1F2D3 \\
                    NUM4 DC ZL6'-123' &F0F0F0F1F2D3 \\
                    NUM5 DC ZL3'1.23' &F1F2C3
                \end{tabular}
            \end{center}
        \item \textbf{Packed decimal format}: Each number translates to a digit in the packed number, with the sign represented by the hex digit at the end of the representation. 
            \bigbreak \noindent 
            For example, $F1F2C3$ in zoned format translates to $123C$ in packed format. For the bytes leading up to the byte with the sign digit, we simply drop the zoned digit. Then, we reverse the order of the last byte (sign digit + number)
            \bigbreak \noindent 
            Packed decimal numbers take fewer bytes to store than zoned. Packed decimal numbers are actually stored as decimal numbers.
            \bigbreak \noindent 
            In a dump, packed decimal numbers appear as decimal numbers.
            \bigbreak \noindent 
            For example, the positive number 4657 would be 04657C. The sign digit of a packed decimal number is the last digit (highlighted in red above).
            \bigbreak \noindent 
            A sign digit of A, C, E or F indicates a positive packed decimal number
            \bigbreak \noindent 
            We often get numbers from an input data set or from some other source in a quasi-zoned decimal format like EBCDIC.
            \bigbreak \noindent 
            We can take the zoned decimal numbers as input and convert them to packed decimal numbers. Packed decimal allows us to do arithmetic. We can even do real number division in packed decimal!
            \bigbreak \noindent 
            \textbf{Note:} Leading zeros are fine, they stay.
        \item \textbf{Declaring packed decimal variables:}
            \bigbreak \noindent 
            \begin{cppcode}
            label DC mPLn'p'
            \end{cppcode}
            \bigbreak \noindent 
            \begin{center}
                \begin{tabular}{p{4cm}|p{7cm}}
                    Declaration &Generates \\
                    \hline
                    NUM1 DC 3PL3'123' &00123C 00123C 00123C \\
                    NUM2 DC 2PL4'123' &0000123C 0000123C \\
                    NUM3 DC PL3'-1.23' &00123D (decimal not stored!) \\
                    NUM4 DC PL6'-123' &00000000123D \\
                    NUM5 DC PL2'12345' &345C (truncated on left!) \\
                \end{tabular}
            \end{center}
        \item \textbf{Default sign digits}:
            \begin{itemize}
                \item \textbf{Positive:} C
                \item \textbf{Negative:} D
            \end{itemize}
        \item \textbf{Note about decimal points}: Decimal points are not stored, it is up to the developer to know where the decimal should be placed.
        \item \textbf{EBCDIC vs zoned decimal}: As long as we have a positive number, the EBCDIC representation is the same as the zoned decimal representation.
        \item \textbf{The case of packed decimals and extra leading zeros}: Consider the zoned decimal to packed decimal conversion
            \begin{align*}
                F0F0F0F3F9F7F1F2 \to 000039712F
            .\end{align*}
            \bigbreak \noindent 
            Notice the extra leading zero that appeared in the packed decimal. If not for this extra added zero, there would be an odd number (9) bytes. There is no such thing as an empty halfbyte.
        \item \textbf{Determine how many bytes of packed decimal storage you will need to declare to hold a converted zoned decimal number}: We take
            \begin{align*}
                \lfloor\text{len(zoned decimal bytes)}/2\rfloor + 1
            .\end{align*}
            \bigbreak \noindent 
            So, to store an 8-byte (8 digit) zoned decimal number in packed decimal, you would need a minimum of 5 bytes of packed decimal storage
            \bigbreak \noindent 
            It is a good general rule to declare packed decimal fields no larger than is absolutely necessary
        \item \textbf{Instructions involving packed decimals}:
            \begin{itemize}
                \item \textbf{Pack (F2) (SS) (Zoned to packed conversion)}: A little like XDECI, It translates numbers in character format into a format which can can be used for arithmetic
                    \bigbreak \noindent 
                    \begin{cppcode}
                        label PACK D1(L1,B1),D2(L2,B2)
                        label PACK label1(L1),label2(L2)
                    \end{cppcode}
                    \bigbreak \noindent 
                    PACK converts zoned decimal numbers into packed decimal numbers. Consider the zoned decimal F1F2C3, It does the following  
                    \begin{itemize}
                        \item The rightmost byte of the second operand is placed in the rightmost byte of the first operand, with zone and numeric digits reversed
                        \item The zone digits are stripped away and the remaining numeric digits from the second operand are moved to the first operand, right to left
                    \end{itemize}
                    \bigbreak \noindent 
                    If the length of the first operand, the target packed decimal field, is not long enough, the number is truncated on the left.
                    \bigbreak \noindent 
                    If the length of the first operand, the target packed decimal field, is too long, the number is padded on the left with zeros.
                    \bigbreak \noindent 
                    The maximum length for a packed decimal field is 16 bytes. 16 bytes can hold a number with 31 digits!
                    \bigbreak \noindent 
                    PACK does not verify that the second operand holds a valid zoned decimal number.
                    \bigbreak \noindent 
                    PACK does not verify that a valid sign is converted into the packed decimal field.
                    \bigbreak \noindent 
                    PACK does not cause a Data Exception (S0C7).
                    \bigbreak \noindent 
                    If you code the length on the second operand incorrectly, the resulting packed decimal number will definitely NOT have the correct sign.
                \item \textbf{UNPK (F3) (SS) (Packed to zoned conversion)}: A little like XDECO, It translates numbers in a format which can be used for arithmetic to numbers in character format
                    \bigbreak \noindent 
                    \begin{cppcode}
                        label UNPK D1(L1,B1),D2(L2,B2)
                        label UNPK label1(L1),label2(L2)
                    \end{cppcode}
                    \bigbreak \noindent 
                    UNPK converts packed decimal numbers into zoned decimal numbers. Consider the number 123C. It does the following to this number, as an example
                    \begin{itemize}
                        \item The rightmost byte of the second operand is placed in the rightmost byte of the first operand, with the zone (sign) and numeric digits reversed
                        \item Zone digit F is added to the remaining digits from right to left, each now taking a byte. 
                    \end{itemize}
                \item \textbf{AP (FA) (SS) (Add)}: Add one packed decimal field to another
                    \bigbreak \noindent 
                    \begin{cppcode}
                    AP PFIELD1(5),PFIELD2(2)
                    \end{cppcode}
                    \bigbreak \noindent 
                    adds the packed decimal number in PFIELD2 to the packed decimal number in PFIELD1. PFIELD2 is unchanged
                    \bigbreak \noindent 
                    \begin{cppcode}
                        label AP D1(L1,B1),D2(L2,B2) Explicit addr.
                        label AP PNUM1(L1),PNUM2(L2) Implicit addr.
                    \end{cppcode}
                    \bigbreak \noindent 
                    L1 and L2 represent lengths coded on both operands, respectively, with 16 being the greatest
                \item \textbf{SP (FB) (SS) (Subtract)}: Subtract one packed decimal field from another
                    \bigbreak \noindent 
                    \begin{cppcode}
                    SP PFIELD1(5),PFIELD2(2)
                    \end{cppcode}
                    \bigbreak \noindent 
                    subtracts the packed decimal number in PFIELD2 from the packed decimal number in PFIELD1. PFIELD2 is unchanged
                    \bigbreak \noindent 
                    \begin{cppcode}
                        label SP D1(L1,B1),D2(L2,B2) Explicit addr.
                        label SP PNUM1(L1),PNUM2(L2) Implicit addr
                    \end{cppcode}
                    \bigbreak \noindent 
                    L1 and L2 represent lengths coded on both operands, respectively, with 16 being the greatest.
                \item \textbf{ZAP (F8) (SS) (Zero and add packed)}: Copy one packed decimal field to
                    \bigbreak \noindent 
                    \begin{cppcode}
                    ZAP PFIELD1(5),PFIELD2(2)
                    \end{cppcode}
                    \bigbreak \noindent 
                    copies the packed decimal number in PFIELD2 to the packed decimal field named PFIELD1
                    \bigbreak \noindent 
                    In reality, PFIELD1 is zero'd out and PFIELD2 is added into PFIELD1. PFIELD2 is unchanged
                    \bigbreak \noindent 
                    Often used to copy a number into a larger field preparing for arithmetic.
                    \bigbreak \noindent 
                    \begin{cppcode}
                        label ZAP D1(L1,B1),D2(L2,B2) Explicit addr.
                        label ZAP PNUM1(L1),PNUM2(L2) Implicit addr
                    \end{cppcode}
                    \bigbreak \noindent 
                    L1 and L2 represent lengths coded on both operands, respectively, with 16 being the greatest.
                \item \textbf{MP (FC) (SS) (Multiply)}: multiply one packed decimal field by another.
                    \bigbreak \noindent 
                    \begin{cppcode}
                        label MP D1(L1,B1),D2(L2,B2) Explicit addr.
                        label MP PNUM1(L1),PNUM2(L2) Implicit addr
                    \end{cppcode}
                    \bigbreak \noindent 
                    L1 and L2 represent lengths coded on both operands, respectively, with 16 being the greatest.
                    \bigbreak \noindent 
                    Does not set the condition code
                    \bigbreak \noindent 
                    \begin{cppcode}
                    MP PNUM1(6),PNUM2(3)
                    \end{cppcode}
                    \bigbreak \noindent 
                    Where
                    \bigbreak \noindent 
                    \begin{cppcode}
                        PNUM1 DC PL6'12233'    00 00 00 12 23 3C
                        PNUM2 DC PL3'15'       00 01 5C
                    \end{cppcode}
                    \bigbreak \noindent 
                    Becomes
                    \bigbreak \noindent 
                    \begin{cppcode}
                        PNUM1 DC PL6'12233'    00 00 01 83 49 5C
                        PNUM2 DC PL3'15'       00 01 5C
                    \end{cppcode}
                \item \textbf{DP (FD) (SS) (Divide)}: divide one packed decimal field by another
                    \bigbreak \noindent 
                    \begin{cppcode}
                        label DP D1(L1,B1),D2(L2,B2) Explicit addr.
                        label DP PNUM1(L1),PNUM2(L2) Implicit addr.
                    \end{cppcode}
                    \bigbreak \noindent 
                    Both quotient and remainder are stored in the first operand. Also does not set the condition code.
                    \bigbreak \noindent 
                    \textbf{Note:} Remainder will be in the last $n$ bytes, where $n$ is the size of the second operand (divisor).
                    \bigbreak \noindent 
                    \begin{cppcode}
                    DP PNUM1(5),PNUM2(2)
                    \end{cppcode}
                    \bigbreak \noindent 
                    Where
                    \bigbreak \noindent 
                    \begin{cppcode}
                        PNUM1 DC PL5'27'   00 00 00 02 7C
                        PNUM2 DC PL2'5'    00 5C
                    \end{cppcode}
                    \bigbreak \noindent 
                    Becomes
                    \bigbreak \noindent 
                    \begin{cppcode}
                        PNUM1 DC PL5'27'   00 00 5C 00 2C
                        PNUM2 DC PL2'5'    00 5C
                    \end{cppcode}
                    \bigbreak \noindent 
                    Observe that the last two bytes (002C) is the remainder. Thus, 00005C is the quotient.
                \item \textbf{CP (F9) (SS) (Compare)}: compare one packed decimal field with another.
                    \bigbreak \noindent 
                    \begin{cppcode}
                        label CP D1(L1,B1),D2(L2,B2) Explicit addr.
                        label CP PNUM1(L1),PNUM2(L2) Implicit addr.
                    \end{cppcode}
                    \bigbreak \noindent 
                    Always use CP when comparing one packed decimal field with another and not some other compare instruction.
                    \bigbreak \noindent 
                    Does a numeric comparison so the value of the field is important, not the lengths of the operands.
                    \bigbreak \noindent 
                    Does NOT change either operand.
                    \bigbreak \noindent 
                    \begin{cppcode}
                    CP PNUM1(5),PNUM2(2)
                    \end{cppcode}
                    \bigbreak \noindent 
                    Where
                    \bigbreak \noindent 
                    \begin{cppcode}
                        PNUM1 DC PL5'27'     00 00 00 02 7C
                        PNUM2 DC PL2'5'      00 5C
                    \end{cppcode}
                    \bigbreak \noindent 
                    condition code is set to two.
                \item \textbf{SRP (F0) (SS) (Shift and round)}: Shift a packed decimal by decimal digits left or right
                    \bigbreak \noindent 
                    \begin{cppcode}
                        label SRP D1(L,B1),D2(B2),i Explicit addr.
                        label SRP PNUM1(L),D2(B2),i Implicit addr.
                    \end{cppcode}
                    \bigbreak \noindent 
                    Used to multiply and divide packed decimal numbers by factors of 10.
                    \bigbreak \noindent 
                    Often used to add extra decimal places to a number preparing it for division and then to get rid of one or more decimal places and round.
                    \bigbreak \noindent 
                    \begin{cppcode}
                    SRP PNUM2(11),4,0 
                    \end{cppcode}
                    \bigbreak \noindent 
                    The first operand is the packed decimal field with the number being shifted, the second indicates a left shift by 4 digits and the third, the rounding factor, is set to 0 (rounding is unnecessary when shifting left).
                    \bigbreak \noindent 
                    Equivalent to multiplying PNUM2 for 11 bytes by $10^{4}$.
                    \bigbreak \noindent 
                    \begin{cppcode}
                    SRP PNUM3(10),64-3,5
                    \end{cppcode}
                    \bigbreak \noindent 
                    The first operand is the packed decimal field with the number being shifted, the second indicates a right shift by 3 digits and the third, the rounding factor, is set to 5. 5 is standard rounding. In other words, if the number being rounded off is 5, 6, 7, 8 or 9, round up by adding 1 to the digit to the left.
                    \bigbreak \noindent 
                    Shifting right by 3 digits with standard rounding is like dividing the packed decimal number by $10^{3}$
                    \bigbreak \noindent 
                    \textbf{Note:} The mainframe knows it is a shift left if the second operand is between 1 and 31 and, conversely, knows it is a shift right if the second operand is between 32 and 63.
                    \bigbreak \noindent 
                    The second operand of the previous example could have been coded as 61 but 64-3 is far more easy to understand and is self-documenting
                    \bigbreak \noindent 
                    The sign of the number shifted is not changed unless the result becomes 0, in which case a negative sign, B or D, is made positive.
                \item \textbf{CVB (4F) (RX) (Convert to binary)}: converts a packed decimal number in a doubleword of storage on a doubleword boundary to its binary equivalent and stores it in a register
                    \bigbreak \noindent 
                    \begin{cppcode}
                        label CVB R1,D2(X2,B2) Explicit addr.
                        label CVB R1,DWORD Implicit addr.
                    \end{cppcode}
                \item \textbf{CVD (4E) (RX) (Convert to decimal)}: converts a binary number in the first operand register to its packed decimal equivalent in a doubleword on a doubleword boundary
                    \bigbreak \noindent 
                    \begin{cppcode}
                        label CVD R1,D2(X2,B2) Explicit addr.
                        label CVD R1,DWORD Implicit addr
                    \end{cppcode}
            \end{itemize}
        \item \textbf{More SRP examples}: Example of a left shift by 3, equivalent to a multiply by $10^{3}$
            \bigbreak \noindent 
            \begin{cppcode}
            SRP PNUM1(5),3,0
            \end{cppcode}
            \bigbreak \noindent 
            Where 
            \bigbreak \noindent 
            \begin{cppcode}
            PNUM1 DC PL5'3227'     00 00 03 22 7C
            \end{cppcode}
            \bigbreak \noindent 
            Becomes
            \bigbreak \noindent 
            \begin{cppcode}
            PNUM1 DC PL5'3227'     00 32 27 00 0C
            \end{cppcode}
            \bigbreak \noindent 
            Example of a right shift by 2, equivalent to a divide by $10^{2}$ , with standard rounding:
            \bigbreak \noindent 
            \begin{cppcode}
            SRP PNUM1(5),64-2,5 
            \end{cppcode}
            \bigbreak \noindent 
            Where
            \bigbreak \noindent 
            \begin{cppcode}
            PNUM1 DC PL5'3277'     00 00 03 27 7C
            \end{cppcode}
            \bigbreak \noindent 
            Becomes
            \bigbreak \noindent 
            \begin{cppcode}
            PNUM1 DC PL5'3277' 00 00 00 03 3C
            \end{cppcode}
        \item \textbf{Recall an important fact in arithmetic}: When multiplying two decimals, you add together the number of digits after the decimal point in each number. For example, if one number has 2 decimal places and the other has 3, the product will have 2 + 3 = 5 decimal places (ignoring any trailing zeros that might be dropped).
            \bigbreak \noindent 
            Dividing decimals works a bit differently. There isn't a direct rule like "add" or "subtract" the decimal places. Instead, you often adjust the numbers by multiplying both the dividend and divisor by the same power of 10 to make the divisor a whole number, then perform the division. The number of decimal places in the quotient depends on the specific numbers and the precision you need (sometimes resulting in a terminating decimal, other times in a repeating or rounded decimal).
        \item \textbf{Real number divide example}: Now, let's do an example of a divide pack that provides us a real number result instead of just a quotient and remainder.
            \bigbreak \noindent 
            Note the following fields defined in storage and their initial values:
            \bigbreak \noindent 
            \begin{cppcode}
                PDEPAMT DC PL7'456929.87'    00 00 04 56 92 98 7C
                PSHRPRC DC PL3'12.35'        01 23 5C
                PCALCSHR DC PL10'0'          00 00 00 00 00 00 00 00 00 0C
            \end{cppcode}
            \bigbreak \noindent 
            We want to divide PDEPAMT by PSHRPRC and get a result with three decimal places just like we would get if using a calculator to do the divide
            \bigbreak \noindent 
            But, to do so, we will have to "fake" real number division using the DP instruction which gives us only integer division results.
            \bigbreak \noindent 
            So, as long as the number being divided, the dividend, has the same number of decimal places as the divisor, no pre-shifting necessary
            \bigbreak \noindent 
            To prepare for the division, we first need to ZAP the number being divided into a larger field.
            \bigbreak \noindent 
            How many bytes long should that larger field be? A simple way to determine that is to add the length of the divisor to the length of the number being divided. In this case it is 7 + 3 = 10
            \bigbreak \noindent 
            PCALCSHR is defined as a 10-byte packed decimal field initialized to 0.
            \bigbreak \noindent 
            Here are the instructions to accomplish the task
            \bigbreak \noindent 
            \begin{cppcode}
                ZAP PCALCSHR(10),PDEPAMT(7) COPY TO LARGER FIELD
                SRP PCALCSHR(10),3,0 ADD 3 FAKE DECIMAL PLACES
                DP PCALCSHR(10),PSHRPRC(3) DIVIDE DEP BY PRC
                SRP PCALCSHR(7),64-1,5 SHIFT AND ROUND QUOTIENT TO TWO PLACES
            \end{cppcode}
            \bigbreak \noindent 
            The result is rounded to two decimal places and is in the quotient part of PCALCSHR, i.e., the first 7 bytes. We can ignore the last three bytes of PCALCSHR, the remainder
        \item \textbf{Encoding of the SS instructions above}: We use the SS encoding format (there are three)
            \begin{align*}
                h_{0}h_{0}h_{L_{1}}h_{L_{2}} \quad h_{B_{1}}h_{D_{1}}h_{D_{1}}h_{D_{1}} \quad h_{B_{2}}h_{D_{2}}h_{D_{2}}h_{D_{2}}
            .\end{align*}
            where 
            \begin{itemize}
                \item \textbf{$h_{0}h_{0}$}: Specifies the opcode
                \item \textbf{$h_{L_{1}}h_{L_{2}}$}: Specifies the length-1 of the arguments lengths
                \item \textbf{$h_{B_{1}}h_{D_{1}}h_{D_{1}}h_{D_{1}} $}: Address of the first operand
                \item \textbf{$h_{B_{2}}h_{D_{2}}h_{D_{2}}h_{D_{2}} $}: Address of the second operand
            \end{itemize}
        \item \textbf{Decker's Rules for Packed Decimal Instructions}:
            \begin{itemize}
                \item Begin the label, or name, of ALL packed decimal fields with the letter P, for Packed
                \item Declare ALL packed decimal fields with a DC, a specific length in bytes and initialized to 0 (unless to some other value)
                \item NEVER let lengths default! Code a length on every packed operand where it CAN be coded!
            \end{itemize}
        \item \textbf{Errors with packed decimals}:
            \begin{itemize}
                \item \textbf{Data Exception (S0C7)}: At least one of the operands is not a valid packed decimal representation
                \item \textbf{Decimal-overflow Exception (S0CA)}: The result is too large for the receiving field
            \end{itemize}
        \item \textbf{Errors that occur with MP}:
            \begin{itemize}
                \item \textbf{Specification Exception (S0C6)}: - if length of second operand is greater than 8 or if length of the second operand, the multiplier, is greater than length of the first operand, the multiplicand
                \item \textbf{Data Exception - (S0C7)}: if the first $n$ bytes of the first operand are not all zeros where $n$ is the length of the second operand or if at least one of the operands is not a valid packed decimal number
            \end{itemize}
        \item \textbf{Errors that occur with DP}:
            \begin{itemize}
                \item \textbf{Specification Exception (S0C6)}: if length of second operand is greater than 8 or if length of the second operand is greater than or equal to the length of the first operand.
                \item \textbf{Decimal-divide Exception  (S0CB)}: if quotient will not fit in n bytes where n is the length of the first operand minus the length of the second operand.
                \item \textbf{Data Exception (S0C7)}: if at least one of the operands is not a valid packed decimal number 
            \end{itemize}
        \item \textbf{Errors that occur with SRP}: 
            \begin{itemize}
                \item \textbf{Decimal-overflow Exception  (S0CA)}: if a left shift results in losing nonzero digits.
            \end{itemize}
        \item \textbf{Errors that occur with CVB}: 
            \begin{itemize}
                \item \textbf{Specification Exception - S0C6 -}: if the second operand is not on a doubleword boundary.
                \item \textbf{Data Exception - S0C7 -}: if the doubleword does not hold a valid packed decimal number.
                \item \textbf{Fixed-point Divide Exception - S0C9 -} if the packed decimal number at the D(X,B) address is too large to be represented in 32 SIGNED bits in the register.
            \end{itemize}
        \item \textbf{Errors that occur with CVD}: 
            \begin{itemize}
                \item \textbf{Specification Exception - S0C6 -}: if the second operand is not on a doubleword boundary.
            \end{itemize}
        \item \textbf{Packed decimal literals}: Literals can be used in the instructions, for example
            \bigbreak \noindent 
            \begin{cppcode}
            AP PNUM1(5),=PL2'35'
            \end{cppcode}
            \bigbreak \noindent 
            Don't let lengths default in literals either
        \item \textbf{Formatting Numeric Output}: There are two instructions that can be used to format and print packed decimal numbers
            \bigbreak \noindent 
            The formatting can - as desired - include a floating dollar sign, a floating positive sign, a floating negative sign, commas between every three digits and/or a decimal point.
            \bigbreak \noindent 
            Two instructions can be used to accomplish this editing:
            \begin{itemize}
                \item ED - Edit Instruction
                \item EDMK - Edit and Mark Instruction
            \end{itemize}
        \item \textbf{ED (DE) (SS) (Edit)}: converts a packed decimal number to its printable EBCDIC equivalent.
            \bigbreak \noindent 
            \begin{cppcode}
                label ED D1(L1,B1),D2(B2) Explicit addr.
                label ED ONUM1(15),PNUM1 Implicit addr.
            \end{cppcode}
            \bigbreak \noindent 
            Takes the packed decimal number at D2(B2), converts it to EBCDIC and places it at D1(L1,B1) according to a pre-placed edit pattern.
            \bigbreak \noindent 
            Note that we do not code a length for the second operand.
            \bigbreak \noindent 
            It can edit the number inserting special characters as desired.
            \bigbreak \noindent 
            A hexadecimal edit pattern must be moved into the output field prior to executing the ED instruction.
            \bigbreak \noindent 
            The edit pattern can insert commas and/or a decimal point as desired.
            \bigbreak \noindent 
            The edit pattern can also supply a character with which we can suppress leading zeros in the number displayed in the print line.
            \bigbreak \noindent 
            \begin{cppcode}
                MVC ODEPAMT(10),=X'4020206B2021204B2020'
                ED ODEPAMT(10),PDEPAMT
            \end{cppcode}
            \bigbreak \noindent 
            Consider the storage (print line)
            \bigbreak \noindent 
            \begin{cppcode}
                PDEPAMT DC PL4'9435.75' 09 43 57 5C
                DETAIL DC C'0' DOUBLE SPACING
                ODEPAMT DS CL10 OUTPUT SPACE FOR PDEPAMT
                DC 122C' ' SPACES
            \end{cppcode}
            \bigbreak \noindent 
            Remember that each byte consists of two hexadecimal digits, if looking at storage in hex, that is.
            \bigbreak \noindent 
            This edit pattern will fill 10 bytes, or columns, of the print line. 
            \bigbreak \noindent 
            The first character in the edit pattern is almost always a fill character.
            \bigbreak \noindent 
            In this example, the fill character is X'40', a space in EBCDIC.
            \bigbreak \noindent 
            X'20' and X'21' are digit selectors.
            \bigbreak \noindent 
            X'21' is a digit selector but it is different from X'20'.
            \bigbreak \noindent 
            X'21' sets significance on in the byte, or column, following it.
            \bigbreak \noindent 
            X'6B' represents a comma in EBCDIC.
            \bigbreak \noindent 
            X'4B' represents a decimal point in EBCDIC
            \bigbreak \noindent 
            The result in EBCDIC in print line: $\bar{b}\bar{b}$9,435.75 ($\bar{b} $ = a space)
            \bigbreak \noindent 
            If significance is off, the character to the right of the 20 will be suppressed if there are no nonzero digits to the left. This is not the case for significance on (21)
        \item \textbf{EDMK (DF) (SS)}: Edit and Mark (EDMK) is used exactly the same as Edit (ED). The only difference is that EDMK places the address of the first
            non-zero digit - from left to right in the output field - in register 1
            \bigbreak \noindent 
            If EDMK reaches a significance on digit selector (X'21') before
            reaching a non-zero digit, the address of the byte following the X'21'
            is placed into register 1.
            \bigbreak \noindent 
            The address in register 1 can then be used to place a dollar sign,
            positive sign, negative sign or some other character to the immediate
            left of the first non-zero digit or where significance is turned on.
            \bigbreak \noindent 
            \begin{cppcode}
                MVC ODEPAMT(10),=X'4020206B2021204B2020'
                EDMK ODEPAMT(10),PDEPAMT
                BCTR 1,0 DECREMENT REGISTER 1 BY 1
                MVI 0(1),C'$' PLACE THE DOLLAR SIGN
            \end{cppcode}
            \bigbreak \noindent 
            \begin{cppcode}
                PDEPAMT DC PL4'9435.75' 09 43 57 5C
                PRTLINE DC C'0' DOUBLE SPACING
                ODEPAMT DS CL10 OUTPUT SPACE FOR PDEPAMT
                DC 122C' ' SPACES
            \end{cppcode}
            \bigbreak \noindent 
            ODEPAMT after the above four instructions and XPRNT: $\bar{b}$\$9,435.75 ($\bar{b}$ = a space)
            \bigbreak \noindent 
            Note that, if PDEPAMT is 0, register 1 is not changed.
            \bigbreak \noindent 
            Therefore, we need to add a level of insurance to print \$0.00 if PDEPAMT is 0 by presetting register 1 to point to the byte immediately to the left of the decimal point byte:
            \bigbreak \noindent 
            \begin{cppcode}
                LA 1,ODEPAMT+6
                MVC ODEPAMT(10),=X'4020206B2021204B2020'
                EDMK ODEPAMT(10),PDEPAMT
                BCTR 1,0 DECREMENT REG 1 BY 1
                MVI 0(1),C'$' PLACE THE DOLLAR SIGN
            \end{cppcode}
        \item \textbf{Notes about ED and EDMK}: Digits are moved from the source field, one at a time, and from left to right.
            \bigbreak \noindent 
            If not enough digit selectors, the result will be truncated on the right
            \bigbreak \noindent 
            A Data Exception - S0C7 - can occur if the source field is not a valid packed decimal number.
            \bigbreak \noindent 
        \item \textbf{Packed instruction condition codes}
            \begin{itemize}
                \item \textbf{AP, SP, ZAP}:
                    \begin{itemize}
                        \item 0 = the result is zero
                        \item 1 = the result is negative
                        \item 2 = the result is positive
                        \item 3 = overflow
                    \end{itemize}
                    \bigbreak \noindent 
                    \textbf{Note:} Note that overflow does not automatically occur if the first operand is shorter than the second, only when the result of the arithmetic operation does not fit into the first operand
                \item \textbf{CP}:
                    \begin{itemize}
                        \item 0 = the two values compared are equal
                        \item 1 = the first operand's value is less than the second operand's
                        \item 2 = the first operand's value is greater than the second operand's
                        \item 3 - not used
                    \end{itemize}
                \item \textbf{SRP}:
                    \begin{itemize}
                        \item 0 = result is zero
                        \item 1 = result is negative
                        \item 2 = result is positive
                        \item 3 = overflow has occurred
                    \end{itemize}
                \item \textbf{ED, EDMK}:
                    \begin{itemize}
                        \item 0 = source field is zero 
                        \item 1 = source field is negative 
                        \item 2 = source field is positive
                        \item 3 = unused
                    \end{itemize}
            \end{itemize}
    \end{itemize}

    \pagebreak 
    \subsection{Internal Subroutines}
    \begin{itemize}
        \item \textbf{STM (90) (RS) (Store Multiple) Instruction}
            \bigbreak \noindent 
            \begin{cppcode}
                label STM R1,R2,D(B)
            \end{cppcode}
            Stores all registers from R1 through R2 to contiguous fullwords in that order in storage starting at D(B).
            \bigbreak \noindent 
            If R2 is less than R1 , then R1 through register 15 and register 0 through R2 are stored in that order in storage starting at D(B).
        \item \textbf{LM (98) (RS) (Load Multiple) Instruction}:
            \bigbreak \noindent 
            \begin{cppcode}
            label LM R1,R2,D(B)
            \end{cppcode}
            \bigbreak \noindent 
            Loads all registers from R1 through R2 from contiguous fullwords in that order from storage starting at D(B).
            \bigbreak \noindent 
            If R2 is less than R1 , then R1 through register 15 and register 0 through R2 are loaded in that order from storage starting at D(B)
        \item \textbf{RS instructions and encodings}: RS instructions are registers to storage, they have the form
            \bigbreak \noindent 
            \begin{cppcode}
            NAME   R1,R2,D(B)
            \end{cppcode}
            \bigbreak \noindent 
            With encoding
            \bigbreak \noindent 
            \begin{align*}
                h_{0}h_{0}h_{r_{1}}h_{r_{2}}h_{B}h_{D}h_{D}h_{D}
            .\end{align*}
        \item \textbf{Why subroutines?}: subdividing our programs into functions makes a complicated program easier to understand, enhance and debug. They reduce the amount of inline repeated code; we can put that repeated code in a subroutine and simply call and execute that subroutine when the code is needed again.
        \item \textbf{Internal vs external subroutines}: An internal subroutine in Assembler is one that is located within a single control section, or CSECT.
            \bigbreak \noindent 
            \textbf{External subroutines}
            \begin{itemize}
                \item It is located outside of the caller's CSECT.
                \item It has its own CSECT.
                \item Is actually a subprogram (although it's often referred to as a subroutine).
                \item External subprograms can be written in C++, Java, etc., on the mainframe
            \end{itemize}
        \item \textbf{Notes about internal subroutines}: There are specific standards that Assembler programmers worldwide must follow to call subroutines and subprograms.
        \item \textbf{Internal subroutine standards}:
            \begin{itemize}
                \item On entrance to a subroutine, register 1 holds the address of a parameter list (if parameters need to be passed in)
                \item On entrance to a subroutine, the initial value in any register that will be altered by the subroutine should be saved using ST and/or STM as necessary
                \item Before exiting a subroutine, the initial value in any register that was altered by the subroutine should be restored using L and/or LM as necessary
            \end{itemize}
        \item \textbf{Defining an Internal Subroutine}: The following puts a label in storage without moving the location counter or actually declaring storage:
            \bigbreak \noindent 
            \begin{cppcode}
            rtnName DS 0H
            \end{cppcode}
            Note that the name of the subroutine can also be placed on the subroutine's first instruction.
            \bigbreak \noindent 
            \begin{cppcode}
            rtnName STM 3,7,SUBSAVE
            \end{cppcode}
        \item \textbf{Parameter list}: Standards dictate that a parameter list must be declared in a very specific manner in Assembler.
            \bigbreak \noindent 
            Parameters are only passed by reference in Assembler and NEVER by value.
            \bigbreak \noindent 
            A parameter list is a set of contiguous fullwords, each containing the address of a parameter, or variable, to be passed.
            \bigbreak \noindent 
            To declare a parameter list that allows the addresses in the fullwords in the parameter list to be virtual, we use Address Constants, or ADCONs, defined as
            \bigbreak \noindent 
            \begin{cppcode}
            label DC A(expression)
            \end{cppcode}
            \bigbreak \noindent 
            If \textbf{expression} is a non-negative integer, the generated fullword will contain the binary representation of that integer, which is the same as declaring
            \bigbreak \noindent 
            \begin{cppcode}
            label DC F'expression'
            \end{cppcode}
            \bigbreak \noindent 
            If expression is a $label$ or $label+n$, the generated fullword will contain the address of label or $label+n$.
            \bigbreak \noindent 
            \begin{cppcode}
            PARM DC A(5)
            \end{cppcode}
            \bigbreak \noindent 
            declares a fullword at label PARM in storage as:
            \bigbreak \noindent 
            \begin{center}
                00000005
            \end{center}
            \bigbreak \noindent 
            \begin{cppcode}
            PARMLIST DC A(FIELD1)
            \end{cppcode}
            \bigbreak \noindent 
            declares a fullword at label PARMLIST in storage as: 00000148, if FIELD1 is declared in storage and is at location counter value 000148 after assembly
            \bigbreak \noindent 
            \begin{cppcode}
            000148 FIELD1 DC F'34220'
            \end{cppcode}
            \bigbreak \noindent 
            \begin{cppcode}
            PARMLIST DC A(45,FIELD2)
            \end{cppcode}
            declares two fullwords at label PARMLIST in storage as:
            \bigbreak \noindent 
            \begin{center}
                0000002D00000274
            \end{center}
            if the following is declared in storage and is at location counter value 000274 after assembly
            \bigbreak \noindent 
            \begin{cppcode}
            000274 FIELD2 DC F'2301.00'
            \end{cppcode}
            \bigbreak \noindent 
            Another example declaration:
            \bigbreak \noindent 
            \begin{cppcode}
                PARMLIST DC A(FIELD3)
                         DC A(34)
            \end{cppcode}
            \bigbreak \noindent 
            Standards dictate that, if the internal subroutine needs parameters passed into it, set up the parameter list similarly to what is shown immediately above
            \bigbreak \noindent 
            Then load the address of PARM or PARMLIST into register 1 before calling the subroutine.
            \bigbreak \noindent 
            By the way, it's the same for passing parameters to external subprograms!
            
        \item \textbf{BAL (45) (RX) (Branch and Link) Instruction}:
            \bigbreak \noindent 
            \begin{cppcode}
            label BAL R1,D2(X2,B2)
            \end{cppcode}
            \bigbreak \noindent 
            Would be more appropriately named the Link and Branch Instruction.
            \bigbreak \noindent 
            \begin{itemize}
                \item \textbf{Link Part of the Instruction:} Puts the last three bytes of the PSW into register R1 and zeros out the first byte of R1 . Why?
                \bigbreak \noindent 
                Remember that the last three bytes of the PSW hold the address of the next instruction, i.e., where to return to after the subroutine.
                \item \textbf{Branch Part of the Instruction:} After the link part described immediately above, the BAL instruction then takes an unconditional branch to the D(X,B) address, i.e., the address of the subroutine.
            \end{itemize}
        \item \textbf{Passing control to subroutine}: Here is the sequence of instructions calling an internal subroutine using a parameter lis
            \bigbreak \noindent 
            \begin{cppcode}
                LA 1,PARMLIST POINT R1 AT PARMLIST
                BAL 11,SUBRTN BRANCH AND LINK TO SUBRTN
            \end{cppcode}
            \bigbreak \noindent 
            Then, when the subroutine is finished, branch back to the instruction in the caller immediately following the call, or BAL:
            \bigbreak \noindent 
            \begin{cppcode}
            BR 11 RETURN TO CALLER
            \end{cppcode}
        \item \textbf{Using parameters}: In the subroutine, after we save the caller's registers, we dereference the parms
            \bigbreak \noindent 
            \begin{cppcode}
                STM   2,4,SAVEREGS   STORE REGS TO BE USED
                LM    2,4,0(1)
            \end{cppcode}
        \item \textbf{Returning control to caller}: When the subroutine completes its task and is ready to return to the caller, it must:
            \bigbreak \noindent 
            \begin{itemize}
                \item Restore the caller's registers using either a Load (L) if only a single register needs to be restored or Load Multiple (LM) if more than one
                \item Return to the caller using: BR 11
            \end{itemize}

    \end{itemize}

    \pagebreak 
    \subsection{Standard linkage and external subroutines}
    \begin{itemize}
        \item \textbf{External subprograms}: Subprograms are similar to internal subroutines but they are outside, or external, to our program. 
            \bigbreak \noindent 
            In Assembler, they have their own CSECT
            \bigbreak \noindent 
            An external subprogram is a type of subroutine but
            \begin{itemize}
                \item It is located outside of the caller's CSECT.
                \item It has its own CSECT.
                \item Is actually a subprogram (although it's often mistakenly referred to as a subroutine).
                \item External subprograms can be written in COBOL, C, C++, Metal C, Java, etc., on the mainframe
            \end{itemize}
            \bigbreak \noindent 
            Just like defining internal subroutines, there are standards and conventions that must be followed.
            \bigbreak \noindent 
            In ASSIST, subprograms are not really "external". Simulated by adding a new CSECT below the storage of the CSECT above
            \bigbreak \noindent 
            Must have full standard entry and exit linkage. One END statement refers to first CSECT at the top of the file
            \bigbreak \noindent 
            Standards dictate that a parameter list must be declared in a very specific manner in Assembler. Parameters are only passed by reference in Assembler and NEVER by value.  A parameter list is a set of contiguous fullwords, each containing the address of a parameter, or variable, to be passed.
            \bigbreak \noindent 
            Here is the sequence of instructions calling an external subprogram using a parameter list:
            \bigbreak \noindent 
            \begin{cppcode}
                LA   1,PARMLIST    POINT R1 AT PARMLIST
                L    15,=V(SUBPGM) LOAD 15 WITH ADDR OF SUBPGM
                BALR 14,15         BRANCH AND LINK TO SUBRTN
            \end{cppcode}
            \bigbreak \noindent 
            Then, when the subprogram is finished, branch back to the instruction in the caller immediately following the call, or BALR:
            \bigbreak \noindent 
            \begin{cppcode}
            BR 14 RETURN TO CALLER 
            \end{cppcode}
            When the subprogram is completes its task and is ready to return to the caller, it must:
            \begin{enumerate}
                \item The standard exit linkage restores the caller's registers
                \item Return to the caller using: BR 14
            \end{enumerate}
        \item \textbf{DROP Statement}:
            \bigbreak \noindent 
            \begin{cppcode}
            DROP R
            \end{cppcode}
            \bigbreak \noindent 
            Or
            \bigbreak \noindent 
            \begin{cppcode}
            DROP R1,R2,...,RN
            \end{cppcode}
            \bigbreak \noindent 
            It ends the "domain" of a USING statement. The DROP informs the Assembler that register R or registers R1,R2,...,Rn are no longer to be associated with label.
            \bigbreak \noindent 
            Or, that the specified register is no longer (for instructions below) supposed to be used to convert implicit addresses to explicit addresses for encoding instructions.
        \item \textbf{Dummy SECTions, or DSECTs}: A dummy section is used to specify a format that can be associated with a particular area in storage without producing any object code.
            \bigbreak \noindent 
            The end of a dummy section is signaled by the occurrence of a CSECT statement, another DSECT statement or an END statement
            \bigbreak \noindent 
            An example DSECT definition:
            \bigbreak \noindent 
            \begin{cppcode}
                $TABELEM DSECT
                $STCKNUM DS    F
                $ARTIST  DS    CL24
                $TITLE   DS    CL24
                $INSTOCK DS    F
                $PRICE   DS    F
            \end{cppcode}
            \bigbreak \noindent 
            specifies the format of a table element. The labels \$STCKNUM, \$ARTIST, etc., can be used rather than displacements into the element itself.
            \bigbreak \noindent 
            Note the convention to use the \$ to denote the name of a DSECT and its fields
            \bigbreak \noindent 
            Before a DSECT can be used, a USING statement must be coded
            \bigbreak \noindent 
            \begin{cppcode}
            USING $TABELEM,3
            \end{cppcode}
        \item \textbf{Standard Entry and Exit Linkage Conventions}:
            \begin{itemize}
                \item Conventions about how to call a subprogram and return from it were standardized many years ago by a group of Assembler developers. What follows is a description of those conventions.
                \item Standards dictate that, when control is passed to an external subprogram, register 15 contains the address of the subprogram.
                \item Standards dictate that register 14 contains the address of the next instruction to execute in the caller program, i.e., the one following the call to the subprogram, i.e., the instruction.
                \item Standards dictate that register 13 contains the address of an 18-fullword save area in the caller's storage in which its own registers will be saved by a called subprogram(!).
                \item Just as was presented in the previous chapter when calling internal subroutines, parameters are passed to an external subprogram in exactly the same manner. Standards dictate that register 1 contains the address of the beginning fullword of the parameter list if parameters are passed.
                \item Return codes are passed back to the caller in register 15.
                \item A simple calculated value can be passed back to the caller in register 0 (rare, and should be avoided).
                \item As hinted above while talking about register 13, the subprogram is responsible for storing the contents of the caller's registers upon entry to the routine and restoring those values before returning control to the caller.
            \end{itemize}
        \item \textbf{Format of 18F save area}
            \bigbreak \noindent 
            \begin{center}
                \begin{tabular}{p{4cm}}
                    Unused \\
                    Backward Pointer \\
                    Forward pointer \\
                    R14 \\
                    R15 \\
                    R1\\
                    R2\\
                    R3\\
                    R4\\
                    R5\\
                    R6\\
                    R7\\
                    R8\\
                    R9\\
                    R10\\
                    R11\\
                    R12
                \end{tabular}
            \end{center}
            \bigbreak \noindent 
            Notice that we do not include R13
        \item \textbf{Standard entry linkage}: The following code should be included as the first lines of each CSECT
            \bigbreak \noindent 
            \begin{cppcode}
            pgmName    CSECT
                       STM   14,12,12(13)
                       LR    12,15
                       USING pgmName,12
                       LA    14,name_of_18F_save_area_in_pgmName
                       ST    13,4(,14)
                       ST    14,8(,13)
                       LR    13,14
            \end{cppcode}
            \bigbreak \noindent 
            \begin{cppcode}
            STM   14,12,12(13) 
            \end{cppcode}
            \bigbreak \noindent 
            Saves all of the caller's registers, except for register 13, in the caller's 18-fullword register save area.
            \bigbreak \noindent 
            \begin{cppcode}
            LR    12,15
            Using pgmName,12
            \end{cppcode}
            Puts the address of pgmName in R12, then establishes R12 as the base register for pgmName
            \bigbreak \noindent 
            \begin{cppcode}
            LA    14,name_of_18F_save_area_in_pgmName
            \end{cppcode}
            points register 14 to an 18-fullword register save area in the current program, pgmName's, storage where its own registers will be saved if it calls a subprogram itself.
            \bigbreak \noindent 
            \begin{cppcode}
            ST    13,4(,14)
            \end{cppcode}
            \bigbreak \noindent 
            stores the address of the caller's 18-fullword register save area in the current program, pgmName's, own 18- fullword register save area. This value in register 13 is known as the \textbf{backward pointer.}
            \bigbreak \noindent 
            \begin{cppcode}
            ST    14,8(,13)
            \end{cppcode}
            stores the address of the current program, pgmName's, 18-fullword register save area in the caller's 18- fullword register save area. This value in register 14 is known as the \textbf{forward pointer}.
            \bigbreak \noindent 
            \begin{cppcode}
            LR    13,14
            \end{cppcode}
            \bigbreak \noindent 
            now points register 13 to the current program, pgmName's, 18-fullword register save area in case it calls a subprogram itself.
        \item \textbf{Standard Exit Linkage}: The following code should be included as the last lines of executable code in each CSECT if no return code is being passed back in register 15 (and no calculated value is being returned in register 0:
            \bigbreak \noindent 
            \begin{cppcode}
            L    13,4(,13)     
            LM   14,12,12(13)
            BR   14
            \end{cppcode}
            \bigbreak \noindent 
            \begin{cppcode}
            L    13,4(,13)
            \end{cppcode}
            

    \end{itemize}

    \pagebreak 
    \subsection{Tables}
    \begin{itemize}
        \item \textbf{Tables}: tables in assembler are like arrays in higher-level languages. Like arrays, Assembler tables are defined in storage - and given a name - to store related data items.
            \bigbreak \noindent 
            Examples of "related" data items or data items of "similar character"
            \begin{itemize}
                \item the ages of each each individual student in class
                \item the account balances of checking accounts of bank customers
                \item the daily average temperature in New York City for the calendar year
                \item the names of the students in an Assembler class
            \end{itemize}
            \bigbreak \noindent 
            In an array, we call the storage for an individual data item an element. In Assembler, this storage is called an entry. In arrays, we can only store one data item per element and, unless an array of objects, all data items must be of the same primitive data type. In Assembler tables, we have complete freedom to store any combination of data items and types in a single entry of a table.
            \bigbreak \noindent 
            In Assembler, we can define a table in storage using any storage class. We only need to be sure that the table is big enough in bytes to hold all of the data we need it to hold.
            \bigbreak \noindent 
            Here is an example of an Assembler table definition that can hold integer test scores for up to 50 students
            \bigbreak \noindent 
            \begin{cppcode}
            SCORES DC 50F'0'
            \end{cppcode}
            \bigbreak \noindent 
            Each of the 50 fullwords can hold a single student's test score
            \bigbreak \noindent 
            We could store the grades and then process the table like an array to calculate an average and/or find the minimum and maximum scores.
            \bigbreak \noindent 
            With the table declared on the previous slide, the following slide displays the code for a read loop that reads records from a file with up to 50 records
            \bigbreak \noindent 
            \begin{cppcode}
                LA 2,SCORES R2 -> SCORES TABLE (OUR TABLE POINTER)
                SR 3,3 R3 = 0 (OUR COUNT OF FILLED TABLE ENTRIES)
                *
                XREAD RECORD,80 READ FIRST RECORD
                *
            LOOP1 BNZ ENDLOOP1 BRANCH TO ENDLOOP1 IF NO MORE RECORDS
                *
                LA 3,1(,3) ADD 1 TO FILLED ENTRY COUNTER
                XDECI 4,RECORD GET SCORE FROM RECORD
                ST 4,0(2) STORE SCORE INTO TABLE ENTRY
                *
                LA 2,4(,2) R2 -> NEXT FULLWORD TABLE ENTRY
                *
                XREAD RECORD,80 READ NEXT RECORD
                B LOOP1 BRANCH BACK TO TOP OF LOOP1
                *
                ENDLOOP1 DS 0H
            \end{cppcode}
    \end{itemize}


    \pagebreak 
    \unsect{Computer Architecture and Syst Org}
    \subsection{Chapter 1: History of computers}
    \begin{itemize}
        \item \textbf{Evolution of computers}:
            \bigbreak \noindent 
            \fig{.5}{./figures/1002.png}
            \bigbreak \noindent 
            Although computers have become faster and more reliable, the same principal components have been present since the beginning of the stored program computer.
        \item \textbf{Stored program computer}: A stored program computer stores the program so that you can change the program without changing the hardware.
        \item \textbf{Mechanical calculators}:
            The opposite of a stored program computer is a device like a calculator where you have to input each step one at a time.
            \bigbreak \noindent 
            Mechanical calculators have existed since the 1600’s. Development limited by ability to machine the parts.
            \bigbreak \noindent 
            The progress of mechanical calculators (in the 1600s and later) depended on how precisely people could manufacture and shape the physical components (gears, levers, cams, wheels, screws, etc.) using the tools and machining techniques of the time.
            \bigbreak \noindent 
            Early lathes, milling machines, and other metalworking tools weren’t advanced enough to create very small, precise, and reliable parts. Without that precision, the calculators would jam, wear down, or give incorrect results.
            \bigbreak \noindent 
            Handheld electronic calculators were invented in the 1970’s, overlapping stored program technology
            \bigbreak \noindent 
        \item \textbf{The Pescaline (first commercially produced mechanical calculator)}:
            The Pascaline (1640s) was a mechanical calculator invented by Blaise Pascal to help his father with tax work.
            \bigbreak \noindent 
            It could perform addition and subtraction directly, and multiplication/division by repeated operations.
            \bigbreak \noindent 
            It used a series of interlocking gears and wheels. Each wheel represented a digit (0–9). Turning one wheel past 9 automatically carried over to the next wheel (like an odometer).
            \bigbreak \noindent 
            It was accurate but expensive, fragile, and limited to basic arithmetic.
            \bigbreak \noindent 
            the Pascaline was the first commercially produced mechanical calculator, using gear-driven wheels to automate carrying in arithmetic.
        \item \textbf{Jacquard loom}:
            The Jacquard loom (1804, Joseph-Marie Jacquard) was a mechanical loom that revolutionized weaving:
            \bigbreak \noindent 
            It used punched cards to control the raising and lowering of warp threads.
            \bigbreak \noindent 
            This let weavers produce complex, repeatable textile patterns automatically, rather than moving threads by hand.
            \bigbreak \noindent 
            It’s considered an early form of programmable machine, directly inspiring later computing ideas (like punch-card programming in Babbage’s engines and early computers).
            \bigbreak \noindent 
            \textbf{Note:} Possibly the first stored program computer. It is Still used today in less industrialized parts of the world
        \item \textbf{Four generations}:
            The evolution of the stored program computer is usually classified into four generations according to the underlying technology.
            \begin{enumerate}
                \item \textbf{First Generation: 1940s-1950s}: Vacuum tubes
                \item \textbf{Second Generation: 1950s-1960s}: Transistors
                \item \textbf{Third Generation: 1960s-1980s}: Integrated circuits
                \item \textbf{Fourth Generation: 1980s-present}: Microprocessors
            \end{enumerate}
        \item \textbf{Generation zero (Card tabulator) (Punched card tabulating machines)}:
            The punched card tabulating machine (1880s–1890s, invented by Herman Hollerith) was an early data-processing device:
            \bigbreak \noindent 
            Built for the 1890 U.S. Census, it sped up counting and sorting huge amounts of population data.
            \bigbreak \noindent 
            Information was encoded as holes in cards (each position represented a data field). The machine had electrical contacts—when a pin passed through a hole, it completed a circuit, advancing a counter or sorter.
            \bigbreak \noindent 
            It cut census processing from nearly a decade (1880) to just a couple of years.
            \bigbreak \noindent 
            Hollerith’s company later became part of IBM, making this a direct ancestor of modern computing.
            \bigbreak \noindent 
            The cards used EBCDIC format which is still used in IBM machines today. The cards had 80 columns, each row represents an 80 character input line.
        \item \textbf{Vacuum tubes}:
            Vacuum tubes (early 1900s) are electronic devices used to control the flow of electricity in a vacuum-sealed glass tube.
            \bigbreak \noindent 
            A typical tube has a cathode (heated filament that releases electrons), an anode (plate that collects electrons), and often one or more grids to control the flow.
            \bigbreak \noindent 
            Functions:
            \begin{itemize}
                \item Amplification (make weak signals stronger)
                \item Switching (on/off, like early transistors)
                \item Rectification (convert AC to DC)
            \end{itemize}
             They formed the basis of the first generation of computers (1940s–1950s), but they were large, power-hungry, produced heat, and failed often.
             \bigbreak \noindent 
            They were eventually replaced by transistors in the late 1950s, which were smaller, faster, and more reliable.
        \item \textbf{John Atanasoff and Clifford Berry of Iowa State University}: Created the Atanasoff Berry Computer (1937 - 1938), which solved systems of linear equations.
            \bigbreak \noindent 
            It was the first totally electronic machine, destroyed during department cleanup
        \item \textbf{John Mauchly and J. Presper Eckert (ENIAC)}: Created the Electronic Numerical Integrator and Computer (ENIAC) at the University of Pennsylvania, 1946
            \bigbreak \noindent 
            Lawsuits over whether Atanasoff or Mauchly and Eckert should get credit for inventing the computer
            \bigbreak \noindent 
            The ENIAC wasn’t the first machine ever, but it was the first large-scale electronic digital machine officially named a "computer." Before ENIAC, the term "computer" referred to people who did calculations by hand.
            \bigbreak \noindent 
            The ENIAC was the first general-purpose computer, it had 17000 vacuum tubes which made up two large rooms (one for air-conditioning). It weighed roughly 30 tons. A few vacuum tubes burnt out every day
            \bigbreak \noindent 
            It has approximately 125 bytes of memory
        \item \textbf{IBM 650}:
           The IBM 650 was the first mass-produced computer (1955). It had a magnetic drum, not a disk. It used decimal, not binary. Phased out in 1969.
        \item \textbf{The Transistor}: A transistor (invented in 1947 at Bell Labs) is a tiny semiconductor device that can amplify or switch electronic signals, just like a vacuum tube but without the drawbacks.
            \bigbreak \noindent 
            A transistor functions as an electronic switch, controlling the flow of electrical current through a circuit or amplifying a weak signal into a stronger one.
            \begin{itemize}
                \item \textbf{Switching:} Transistors can be turned "on" or "off" by their input signal, creating a digital "1" or "0". This on-off behavior is essential for logic gates, which perform calculations in computer processors
                \item \textbf{Amplification:} A small input signal can control a larger output signal, increasing its power or amplitude. This is crucial in applications like radio receivers and audio systems, where weak signals need to be strengthened for detection or output. 
            \end{itemize}
            \bigbreak \noindent 
            \textbf{Why transistors replaced vacuum tubes}:
            \begin{itemize}
                \item \textbf{Size:} Transistors are extremely small compared to bulky glass tubes.
                \item \textbf{Power \& Heat:} They consume far less power and generate much less heat.
                \item \textbf{Reliability:} Transistors are solid-state (no fragile filaments or vacuum glass), so they rarely burn out.
                \item \textbf{Speed:} They can switch on and off much faster, enabling quicker computations.
                \item \textbf{Cost:} Mass production on silicon chips made them cheaper over time.
            \end{itemize}
            \textbf{Note:} Solid-state means that a device’s components are made entirely from solid materials (like semiconductors) and do not rely on moving parts or vacuum tubes to function.
        \item \textbf{Second generation (Transistorized computers) ((1954-1965))}: These computers used transistors, which are far more reliable than vacuum tubes
            \bigbreak \noindent 
            Examples include the IBM 7094 (scientific) and 1401 (business), and the Digital Equipment Corporation (DEC) PDP-1
            \bigbreak \noindent 
            Each of these computers had a different architecture, and less than 1MB of memory.
        \item \textbf{Size comparison}: Relative sizes (clockwise from top)
            \begin{itemize}
                \item Vacuum tube
                \item Transistor
                \item Chip with 2000 NAND gates
                \item Integrated circuit package
            \end{itemize}
            \bigbreak \noindent 
            \fig{.5}{./figures/1003.png}
        \item \textbf{Integrated circuit}: An integrated circuit (IC) is a tiny chip of semiconductor (usually silicon) that contains many electronic components—transistors, resistors, capacitors, and wiring—all fabricated together on a single piece of material.
            \bigbreak \noindent 
            \textbf{Why ICs were an improvement (over vacuum tubes and even discrete transistors)}:
            \begin{itemize}
                \item \textbf{Miniaturization:} Instead of wiring thousands of individual transistors by hand, many could be built directly onto one chip → much smaller devices.
                \item \textbf{Reliability:} Fewer soldered connections meant fewer points of failure compared to vacuum tubes or discrete transistor circuits.
                \item \textbf{Speed:} Signals travel much faster across a tiny chip than between separate components on a circuit board.
                \item \textbf{Power efficiency:} ICs consume very little power compared to vacuum tubes (which required heating filaments).
                \item \textbf{Cost:} Mass production made ICs cheap to manufacture, lowering the cost of electronics.
            \end{itemize}
        \item \textbf{The Third Generation: Integrated Circuit Computers (1965-1980)}:
            An integrated circuit computer is a computer built using integrated circuits (ICs) instead of vacuum tubes or discrete transistors.
            \bigbreak \noindent 
            Thousands of transistors, resistors, and capacitors were fabricated on small silicon chips, then connected together to form CPUs and memory systems.
            \bigbreak \noindent 
            Examples of third generation computers include the IBM 360/370, DEC PDP-8 and PDP-11, and the Cray-1 supercomputer
            \bigbreak \noindent 
            Beginning of the modern era, System software (operating systems and compilers), and Application software
        \item \textbf{Why IBM succeeded}: By using the same architecture, IBM provided an upgrade path
            \bigbreak \noindent 
            Same architecture but not necessarily same hardware design
            \bigbreak \noindent 
            Scientific and business machines. Scientists need floating point. Business people don’t want roundoff error
            \bigbreak \noindent 
            IBM was the Microsoft of its day, but it no longer is
        \item \textbf{Fourth generation VLSI computers}: Very large scale integrated circuits (VLSI) have more than 10,000 components per chip
            \bigbreak \noindent 
            Enabled the creation of microprocessors.
            \bigbreak \noindent 
            8080, 8086, and 8088 chips used in first personal computers in early 80’s
            \bigbreak \noindent 
            Earliest PCs had no hard disk they had a 360KB per floppy disk (later 1.4MB) which ran only one application at a time
        \item \textbf{Backwards compatibility}: Backwards compatibility is the ability of a new machine to run software intended for the previous generation
            \bigbreak \noindent 
            Essential to modern software industry because users want to spend their money on their core business, not on rewriting or converting software
            \bigbreak \noindent 
            Current PC architecture became near-universal not because it was a good one but because it was created at the right time
            \bigbreak \noindent 
            Newer machines needed backward compatibility, so we are stuck with that architecture (and its successors) forever
            \bigbreak \noindent 
            IBM has also provided backwards compatibility for its mainframe line
        \item \textbf{Moore's Law}: Gordon Moore (founder of Intel) in 1965: The density of transistors in an integrated circuit will double every year.
            \bigbreak \noindent 
            The more recent version: Density of silicon chips doubles every 18 months.
            \bigbreak \noindent 
            This is just an approximation, not a real law like that law of gravity.
            \bigbreak \noindent 
            This can’t last forever, eventually you’d have transistors smaller than an atom!
        \item \textbf{Summary}:
            \bigbreak \noindent 
            First Generation (1940s-1950s)
            \begin{itemize}
                \item \textbf{Technology:} Vacuum tubes
                \item Characteristics
                    \begin{itemize}
                        \item Large, slow, and expensive
                        \item High power consumption
                    \end{itemize}
                \item \textbf{Example:} ENIAC
            \end{itemize}
            Second Generation (1950s-1960s)
            \begin{itemize}
                \item \textbf{Technology:} Transistors
                \item Characteristics
                \begin{itemize}
                    \item Smaller, faster, and more reliable than vacuum tubes
                    \item Reduced power usage
                \end{itemize}
                \item \textbf{Example:} IBM 1401
            \end{itemize}
            Third Generation (1960s-1980s)
            \begin{itemize}
                \item \textbf{Technology:} Integrated Circuits (ICs).
                \item Characteristics:
                    \begin{itemize}
                        \item Increased speed and efficiency.
                        \item Cost-effective and compact design.
                    \end{itemize}
                \item \textbf{Example:} IBM System/360
            \end{itemize}
            Fourth Generation (1980s-present)
            \begin{itemize}
                \item \textbf{Technology:} Microprocessors
                \item Characteristics
                \begin{itemize}
                    \item Entire processing unit on a single chip
                    \item Further miniaturization and mass production
            \end{itemize}
            \item \textbf{Example:} Personal Computers (PCs)
    \end{itemize}

    \end{itemize}

    \pagebreak \bigbreak \noindent 
    \subsection{Chapter 3}
    \bigbreak \noindent 
    \subsubsection{Boolean algebra}
    \begin{itemize}
        \item \textbf{Boolean expressions}: Boolean expressions are built by using Boolean operators to combine Boolean variables
            \bigbreak \noindent 
            Most people consider the fundamental Boolean operators to be AND, OR, and NOT.
            \begin{itemize}
                \item \textbf{Not}: Unary operator, we can write NOT $A$
                \item \textbf{And, Or}: Binary operators
            \end{itemize}
        \item \textbf{And}: AND plays the role of "multiply" in Boolean algebra, so it can be written using any symbol for "times". It is called the Boolean product.
            \begin{align*}
                AB = A \text{ AND } B
            .\end{align*}
            Can also be written with $*$, $\land$, $\times$
        \item \textbf{Or}: OR plays the role of "add" in Boolean algebra, so it can be written using any symbol for "plus". It is called the Boolean sum.
            \begin{align*}
                A + B = A \text{ OR } B
            .\end{align*}
            Can also be written with $\lor$
        \item \textbf{NOT}: NOT plays the role of negation, so it can be written with any symbol for negation
            \bigbreak \noindent 
            Can be written with a minus sign, overbar, prime mark, elbow $(\neg)$ or the exclamation mark (!)
        \item \textbf{Order of operations}:
            \begin{itemize}
                \item Parentheses have top priority.
                \item NOT has the next priority.
                \item AND is next.
                \item NAND, NOR is lower than and but higher than xor and or
                \item XOR is lower than AND,NAND, and NOR but higher than OR
                \item OR has the lowest priority
            \end{itemize}
        \item \textbf{AND binds tighter than OR}: This is another way of saying that AND has priority over OR
        \item \textbf{Boolean function}: A Boolean function is a function of zero or more Boolean variables.
            \bigbreak \noindent 
            The output of a Boolean function is also a Boolean value, i.e., 0 or 1.
        \item \textbf{Canonical form}: There are numerous ways of stating the same Boolean expression.
            \begin{itemize}
                \item These synonymous forms are logically equivalent.
                \item Logically equivalent expressions have identical truth tables.
            \end{itemize}
            \bigbreak \noindent 
            To make it easier to compare Boolean expressions, we use standardized forms called canonical forms.
            \bigbreak \noindent 
            The most useful canonical form for Boolean expressions is the sum-of-products form
            \bigbreak \noindent 
            In the sum-of-products form, terms are connected by OR.
            \bigbreak \noindent 
            The terms used to make up sum-of-products form are called minterms
            \bigbreak \noindent 
            For a function of $n$ variables, each minterm contains exactly $n$ items AND'ed together..
            \bigbreak \noindent 
            For example, the minterms with 2 variables are
            \begin{align*}
                (-x)(-y),\ (-x)(y),\ (x)(-y),\ (x)(y)
            .\end{align*}
            \bigbreak \noindent 
            Some or all of these terms are OR'd together to create the sum-of-products form for any function of two variables.
            \bigbreak \noindent 
            To make the result unique, we need a way to ensure that the terms always come in the same order.
            \bigbreak \noindent 
            In this class we will use the following rule:
            \begin{enumerate}
                \item Alphabetize the variables in each term.
                \item Alphabetize the terms, considering the negative form of a variable to precede the positive one, i.e., $-x$ comes before $x$.
            \end{enumerate}
            \bigbreak \noindent 
            So whichever of the four minterms
            \begin{align*}
                (-x)(-y),\ (-x)(y),\ (x)(-y),\ (x)(y)
            .\end{align*}
            are used, they will always be listed in that order.
            \bigbreak \noindent 
            To convert a function to sum-of products form, we start with its truth table
            \bigbreak \noindent 
            Every row that results in a 1 contains a combination of values of variables that makes the function true.
            \bigbreak \noindent 
            So we AND together that set of values.
            \bigbreak \noindent 
             The function is true if any one of those sets of values occurs, so we OR the sets together
             \bigbreak \noindent 
             Consider $F(x,y,z) = x \bar{z} + y$, where $ \bar{z}$ denotes the negation of $z$
             \bigbreak \noindent 
             The truth table is 
             \bigbreak \noindent 
             \begin{align*}
                 \begin{array}{|c|c|c||c|}
                     \hline
                     x & y & z & x\overline{z} + y \\
                     \hline
                     0 & 0 & 0 & 0 \\
                     0 & 0 & 1 & 0 \\
                     0 & 1 & 0 & 1 \\
                     0 & 1 & 1 & 1 \\
                     1 & 0 & 0 & 1 \\
                     1 & 0 & 1 & 0 \\
                     1 & 1 & 0 & 1 \\
                     1 & 1 & 1 & 1 \\
                     \hline
                 \end{array}
             .\end{align*}
             \bigbreak \noindent 
             This function has 5 rows that return true, so our sum-of-products representation will have 5 terms
             \bigbreak \noindent 
             Each term will show the values of $x$, $y$ and $z$ that are needed to make that term true. Remember that if $x$ is false (0), then $\bar{x}$ is true.
             \bigbreak \noindent 
             Thus, 
             \begin{align*}
                 F(x,y,z) = \bar{x}y \bar{z} + \bar{x}yz + x \bar{y} \bar{z} + xy \bar{z} + xyz
             .\end{align*}
             \bigbreak \noindent 
             You can check that you have correctly generated the sum-of-products form by building a truth table for the sum-of-products form. The result column should match the result column of the original function.
    \end{itemize}

    \pagebreak 
    \subsubsection{Boolean identities}
    \begin{itemize}
        \item \textbf{Intro}: The simpler that we can make a Boolean function, the smaller the circuit that will result.
            \bigbreak \noindent 
            Simpler circuits are cheaper to build, consume less power, and run faster than complex circuits.
            \bigbreak \noindent 
            However, there are other criteria involved
            \begin{enumerate}
                \item How the circuit involved fits with other circuits in the machine, e.g., can common activities be factored out.
                \item General criteria such as providing redundancy, a safety margin of error, and others.
                \item Consistency with other machines in the same line.
            \end{enumerate}
            \bigbreak \noindent 
            Still, we often want to reduce our Boolean functions to a simpler form
            \bigbreak \noindent 
            We can use Boolean identities (equations) for this.
            \bigbreak \noindent 
            In a real-world situation, this would be accomplished with the help of a hardware compiler that would also enforce standards.
            \bigbreak \noindent 
        \item \textbf{Boolean identities in one variable}
            Most Boolean identities have an AND (product) form as well as an OR (sum) form.
            \bigbreak \noindent 
            The first group contains identities involving only one variable. They enable you to reduce the number of terms in an expression, and sometimes also the number of variables.
            \begin{center}
                \begin{tabular}{c|c|c}
                    Identity Name & AND form & OR form \\ 
                    \hline
                    Identity Law & $1x =x$ & 0 + x = x\\
                    Null Law & $0x = 0 $ & $1 + x = 1 $ \\
                    Idempotent Law & $xx = x$ & $x + x = x$ \\
                    Inverse Law & $x\bar{x} = 0$ & $x + \bar{x} = 1$
                \end{tabular}
            \end{center}
            \bigbreak \noindent 
            The AND form and the OR form are duals. The dual of a law is created as follows:
            \begin{enumerate}
                \item Switch AND and OR.
                \item Switch 0 and 1
            \end{enumerate}
            \bigbreak \noindent 
            If a Boolean statement is valid (its truth table contains all TRUEs), its dual is valid also. This is different from ordinary arithmetic, it is a characteristic feature of Boolean algebra
            \bigbreak \noindent 
            Note that laws are good for variables and expressions. 
            The laws are expressed in terms of variables like $X$, $Y$ and $Z$.
            \bigbreak \noindent 
            You can substitute any Boolean expression in place of a variable.
        \item \textbf{Identities}:
            \begin{center}
                \begin{tabular}{|l|l|l|}
                    \hline
                    \textbf{Identity Name}       & \textbf{AND Form}              & \textbf{OR Form} \\ \hline
                    Identity Law                 & $1x = x$                       & $0 + x = x$ \\ \hline
                    Null (or Dominance) Law      & $0x = 0$                       & $1 + x = 1$ \\ \hline
                    Idempotent Law               & $xx = x$                       & $x + x = x$ \\ \hline
                    Inverse Law                  & $x\bar{x} = 0$                 & $x + \bar{x} = 1$ \\ \hline
                    Commutative Law              & $xy = yx$                      & $x + y = y + x$ \\ \hline
                    Associative Law              & $(xy)z = x(yz)$                & $(x + y) + z = x + (y + z)$ \\ \hline
                    Distributive Law             & $x + yz = (x + y)(x + z)$      & $x(y + z) = xy + xz$ \\ \hline
                    Absorption Law               & $x(x + y) = x$                 & $x + xy = x$ \\ \hline
                    DeMorgan's Law               & $\overline{(xy)} = \bar{x} + \bar{y}$ & $\overline{(x + y)} = \bar{x}\bar{y}$ \\ \hline
                    Double Complement Law        & $\bar{\bar{x}} = x$            &  \\ \hline
                \end{tabular}
            \end{center}
        \item \textbf{Example}: Simplify the following function
            \begin{align*}
                F(x,y,z) = (x+y)(x+\bar{y})(\overline{x \bar{z}})
            \end{align*}
            \begin{align*}
                F(x,y,z) &= (x+y)(x+ \bar{y})(\overline{x \overline{z}}) \quad \text{ (Given)} \\
                         &= (x+y)(x+ \bar{y})( \bar{x} + \overline{ \overline{z}}) \quad \text{(DeMorgan)} \\
                         &= (x+y)(x+ \bar{y})( \bar{x} +  z) \quad \text{(Double Complement)} \\
                         &= (xx + x \bar{y} + xy + y \bar{y})( \bar{x} + z) \quad \text{(Distributive (OR))} \\
                         &= (x + x \bar{y} + xy + y \bar{y})( \bar{x} + z) \quad \text{(Idempotent)} \\ 
                         &= (x + x \bar{y} + xy)( \bar{x} + z) \quad \text{(Inverse)} \\
                         &= (x + x (y + \bar{y}))( \bar{x} + z) \quad \text{(Distributive)} \\
                         &= (x + x)( \bar{x} + z) \quad \text{(Inverse)} \\
                         &= x( \bar{x} + z) \quad \text{(Idempotent)} \\
                         &= x \bar{x} + xz \quad \text{(Distributive)} \\
                         &=  xz \quad \text{(Inverse)}
            \end{align*}

    \end{itemize}

    \pagebreak 
    \subsubsection{Logic gates}
    \begin{itemize}
        \item \textbf{Gate:} A gate is an electronic device that produces a result based on two or more input values
            \bigbreak \noindent 
            Technically, one or more input values, since a NOT gate only has one input.
            \bigbreak \noindent 
            In reality, gates consist of one to six transistors, but digital designers think of them as a single unit.
            \bigbreak \noindent 
            Integrated circuits contain collections of gates suited to a particular purpose.
            \bigbreak \noindent 
            Because of the associative law, AND-gates and OR-gates can also take in 3 or more inputs, but NAND and NOR are not associative
        \item \textbf{And, or, and not gates}:
            \bigbreak \noindent 
            \fig{.5}{./figures/andornot.png}
            \bigbreak \noindent 
        \item \textbf{XOR gate}: Another very useful gate is the exclusive OR (XOR) gate
            \bigbreak \noindent 
            \fig{.8}{./figures/xor.png}
            \bigbreak \noindent 
            The XOR gate is defined as 
            \begin{align*}
                x\bar{y} + \bar{x}y
            .\end{align*}
        \item \textbf{Properties of XOR}
            \begin{itemize}
                \item \textbf{Commutative}: $A \oplus B = B \oplus A $
                \item \textbf{Associative}: $(A \oplus B) \oplus C = A \oplus (B \oplus C) $
                \item \textbf{Identity element (0)}: $A \oplus 0 = A $
                \item \textbf{Self inverse}: $A \oplus A = 0 $
                \item \textbf{Inverse element (1)}: $A \oplus 1 = \bar{A} $
                \item \textbf{Distributive over AND (but not over OR)}:
                    \begin{align*}
                        A(B \oplus C) = AB \oplus AC
                    .\end{align*}
                \item \textbf{Non-Idempotent}: $A \oplus A = 0 \ne A $
                \item \textbf{Relation to equality}:
                    \begin{itemize}
                        \item $A \oplus B = 1$ means $A = B $
                        \item $A \oplus B = 0$ means $A \ne B $
                    \end{itemize}
            \end{itemize}
        \item \textbf{NAND Gate}:
            \bigbreak \noindent 
            \fig{.5}{./figures/nand.png}
            \begin{align*}
                x \text{ NAND } y = \overline{xy} = \bar{x} + \bar{y}
            .\end{align*}
        \item \textbf{NOR Gate}:
            \bigbreak \noindent 
            \fig{.5}{./figures/nor.png}
            \begin{align*}
                x \text{ NOR } y = \overline{x+y} = \bar{x}\bar{y}
            .\end{align*}
        \item \textbf{Notation for NAND and NOR}:
            \begin{itemize}
                \item \textbf{NAND (Sheffer stroke)}: $x \uparrow y $
                \item \textbf{NOR (Peirce arrow)}: $x \downarrow y$
            \end{itemize}
        \item \textbf{NAND, NOR - Universal}: NAND and NOR are universal gates.
            \bigbreak \noindent 
            A universal gate means that any Boolean function can be constructed using all NAND gates or all NOR gates
            \bigbreak \noindent 
            This makes NAND and NOR different from gates like AND and OR, which are not universal gates
            \bigbreak \noindent 
            Since NAND and NOR are inexpensive to manufacture, this is a useful property
            \bigbreak \noindent 
            To show that NAND is a universal gate, we need to show that the basic gates AND, OR and NOT can be built from NANDs
        \item \textbf{Building AND, OR, and NOT from NAND}: First, we remark on a key property of the NAND gate
            \begin{align*}
                x \uparrow x   = \bar{x}     
            .\end{align*}
            Thus, we have our first gate: the NOT gate.
            \bigbreak \noindent 
            The two remaining gates are built in the following way
            \begin{align*}
                (x \uparrow y) \uparrow (x \uparrow y) &= xy
            .\end{align*}
            \textbf{\textit{Proof.}} Since $x \uparrow x = \bar{x}$, and $x \uparrow  y = \overline{xy}$, we have
            \begin{align*}
                (x \uparrow y) \uparrow (x \uparrow y) = \overline{(\overline{xy})(\overline{xy})} = \overline{\overline{xy}} = xy
            .\end{align*}
            Finally, the OR gate is 
            \begin{align*}
                (x \uparrow x) \uparrow ( y \uparrow y) &= x + y
            .\end{align*}
            \textbf{\textit{Proof.}} Since $ x \uparrow x = \bar{x}$, and $ y \uparrow y = \bar{y} $, we have
            \begin{align*}
                (x \uparrow x) \uparrow ( y \uparrow y) &= \overline{\bar{x}\bar{y}} = \overline{\overline{x+y}} =  x+y
            .\end{align*}
            $\endpf$
        \item \textbf{Building AND, OR, and NOT from NOR}: For NOR, we have the same property,
            \begin{align*}
                x \downarrow x = \bar{x}
            .\end{align*}
            So, the NOT gate is $x \downarrow x$. The AND and OR  gates are built with
            \begin{align*}
                (x \downarrow x) \downarrow (y \downarrow y) &= xy \\
                (x \downarrow y) \downarrow ( x \downarrow y) &= x + y
            .\end{align*}
            Proofs left as an exercise to the reader.
        \item \textbf{Properties of NAND gates}
            \begin{itemize}
                \item \textbf{Commutative}: $A \uparrow B = B \uparrow A $
                \item \textbf{Non-Associative}: $ (A \uparrow B) \uparrow C \ne A \uparrow (B \uparrow C )$
                \item \textbf{Non-Idempotent}: $A \uparrow A = \bar{A } \ne A $
                \item \textbf{Absorption like behavior}
                    \begin{align*}
                        A \uparrow 1 &= \bar{A} \\
                        A \uparrow 0 &= 1
                    .\end{align*}
                \item \textbf{Universal}
            \end{itemize}
        \item \textbf{Properties of NOR gates}
            \begin{itemize}
                \item \textbf{Commutative}: $A \downarrow B  = B \downarrow A $
                \item \textbf{Non-Associative}: $ (A \downarrow B ) \downarrow C \ne A \downarrow ( B \downarrow C ) $
                \item \textbf{Non-Idempotent}: $A \downarrow A = \bar{A} \ne A $ 
                \item \textbf{Absorption like behavior}
                    \begin{align*}
                        A \downarrow 0 &= \bar{A} \\
                        A \downarrow 1 &= 0 
                    .\end{align*}
                \item \textbf{Universal}:
            \end{itemize}
        \item \textbf{Generalizing Gates}:
            Gates can have multiple inputs and more than one output
            \bigbreak \noindent 
            A gate showing 3 or more inputs is really an abbreviation for a series of gates. This is made possible by the associative and commutative laws for AND and OR
            \bigbreak \noindent 
            \fig{.5}{./figures/hello1.png}
            \bigbreak \noindent 
        A gate with more than one output is an abbreviation for a set of gates with the same input, where each of them provides one of the outputs.
        \bigbreak \noindent 
        \fig{.5}{./figures/hello2.png}
    \item \textbf{Implementing Functions}:
        Boolean functions are implemented with
        combinations of gates.
        \bigbreak \noindent 
        The circuit below implements:
        \begin{align*}
            F(x,y,z) = x + \bar{y}z
        .\end{align*}
        \bigbreak \noindent 
        \fig{.5}{./figures/hello3.png}

    \end{itemize}

    \pagebreak 
    \subsubsection{Combinatorial circuits}
    \begin{itemize}
        \item \textbf{Combinatorial circuit}: We have designed a circuit to implement the Boolean function
            \begin{align*}
                F(x,y,z) = x + \bar{y}z
            \end{align*}
            This circuit is an example of a \textit{combinational} or \textit{combinatorial logic} circuit
            \bigbreak \noindent 
            Combinational logic circuits produce a specified output at the instant when input values are applied.
        \item \textbf{Half Adder}: Combinational logic circuits give us many useful devices.
            \bigbreak \noindent 
            One of the simplest is the half adder, which finds the sum of two bits.
            \bigbreak \noindent 
            To build a half adder, look at its truth table, shown at the right
            \bigbreak \noindent 
            It has two inputs and two outputs.
            \begin{center}
                \begin{center}
                    \begin{tabular}{cc|cc}
                        $x$&$y$&Sum&Carry \\ 
                        \hline
                        0 & 0 & 0 & 0 \\
                        0 & 1 & 1 & 0 \\
                        1 & 0 & 1 & 0 \\
                        1 & 1 & 0 & 1
                    \end{tabular}
                \end{center}
            \end{center}
            The sum can be calculated with XOR
            \begin{align*}
                \text{SUM}(x,y) = x \oplus y
            \end{align*}
            The carry can be calculated with AND
            \begin{align*}
                \text{CARRY}(x,y) = xy
            \end{align*}
            The half adder circuit is 
            \bigbreak \noindent 
            \fig{.5}{./figures/ha.png}
            \bigbreak \noindent 
            \textbf{Note:} The dot denotes intersection of wires, no dot denotes no intersection.
            \bigbreak \noindent 
            We have used gates to build an arithmetic function. We needed to use the binary number system to do so. Every aspect of building a computer derives from this ability to use gates to build arithmetic functions.
        \item \textbf{Full Adder}: A full adder handles the complete calculation of one column of a binary addition.
            \bigbreak \noindent 
            It has 3 inputs and 2 outputs.
            \bigbreak \noindent 
            The truth table for a full adder is shown below
            \bigbreak \noindent 
            \begin{center}
                \begin{tabular}{|c c c|c c|}
                    \hline
                    \multicolumn{3}{|c|}{\textbf{Inputs}} & \multicolumn{2}{c|}{\textbf{Outputs}} \\
                    \hline
                    X & Y & Carry In & Sum & Carry Out \\
                    \hline
                    0 & 0 & 0 & 0 & 0 \\
                    0 & 0 & 1 & 1 & 0 \\
                    0 & 1 & 0 & 1 & 0 \\
                    0 & 1 & 1 & 0 & 1 \\
                    1 & 0 & 0 & 1 & 0 \\
                    1 & 0 & 1 & 0 & 1 \\
                    1 & 1 & 0 & 0 & 1 \\
                    1 & 1 & 1 & 1 & 1 \\
                    \hline
                \end{tabular} 
            \end{center}
            \bigbreak \noindent 
            \begin{align*}
                \text{SUM-FULL}(x,y,c) = \text{SUM(SUM($x,y$), $c$)} = (x\oplus y) \oplus c
            \end{align*}
            \bigbreak \noindent 
            \fig{.5}{./figures/fa.png}
            \bigbreak \noindent 
            If either sums has a carry, we output one. If neither sums has a carry, we output zero.
            \begin{align*}
                \text{CARRY-OUT}(x,y,c) &= \text{CARRY}(x,y) \text{ OR } \text{CARRY}(\text{SUM}(x,y,c) \\
                &=xy + (x \oplus y)c
            \end{align*}
        \item \textbf{Ripple-Carry Adder}: 
            Full adders can be connected in a series. The carry bit "ripples" from one adder to the next; hence, this circuit is called a ripple-carry adder.
            \bigbreak \noindent 
            \fig{.5}{./figures/rca.png}
            \bigbreak \noindent 
            The right-most full adder does not need a carry-in. However, from the point of view of manufacturing efficiency, it does not hurt to use a full adder with the carry-in set to 0
            \bigbreak \noindent 
            You can do that by having no input to that item. Or you could replace the right-most full adder by a halfadder
        \item \textbf{Block diagrams}: Describe the external features of a circuit, the inputs and outputs.
        \item \textbf{Decoders}:
            Decoders are another important type of combinational circuit
            \bigbreak \noindent 
            Among other things, they are useful in selecting a memory location according a binary value placed on the address lines of a memory bus
            \bigbreak \noindent 
            Address decoders with $n$ inputs can select any of $2^{n}$ locations.
            \bigbreak \noindent 
            \fig{.5}{./figures/block.png}
            \fc{This is a block diagram for a decoder}
            \bigbreak \noindent 
            This diagram shows the internals of a 2-to-4 decoder
            \bigbreak \noindent 
            \fig{.5}{./figures/dc1.png}
            \bigbreak \noindent 
            For example, the $xy$ output is only true if $x$ is true and $y$ is true.
            \bigbreak \noindent 
            A decoder is used to convert a binary number ($B'xy'$) to a value (0th, 1st, 2nd, 3rd output is true, counting from the bottom of the diagram.
            \bigbreak \noindent 
            Given an address, a decoder can be used to access the memory cell at that address.
        \item \textbf{Multiplexers}: A multiplexer, abbreviated mux, is the opposite of a decoder. It selects a single output from several inputs. The particular input chosen for output is determined by the value of the multiplexer's control lines
            \bigbreak \noindent 
            To be able to select among $n$ inputs, $\log_{2}(n) $ control lines are needed.
            \bigbreak \noindent 
            \fig{.5}{./figures/mp.png}
            \bigbreak \noindent 
            This diagram shows the internals of a 4-to-1 multiplexer.
            \bigbreak \noindent 
            \fig{.5}{./figures/mp2.png}
            \bigbreak \noindent 
            Note that only one of the AND-gates can be true at any given time.
            \bigbreak \noindent 
            Each input $I_{n}$ is only connected to one of the AND-gates. That AND-gate is only true for a specific pair of values of $S_{0}$ and $S_{1} $ . For example, the top one is true when $S_{1}S_{0} = 0b11$.
            \bigbreak \noindent 
            Input $I_{n}$ is connected to the AND-gate where the binary value $S_{1}S_{0} = n$. From top to bottom, the AND-gates are connected to inputs \# 3, 2, 1 and 0
            \bigbreak \noindent 
            Each control input, $S_{0}$ and $S_{1}$ , is provided in its original and negated form to make it easy to build the four AND-gates
            \bigbreak \noindent 
            Since only one of the AND-gates is true at any one time, we can OR together their values to get the value of the input ($I_{0} $ , $I_{1} $, $I_{2} $ , or $I_{3}$) selected by the two $S$ lines.
            \bigbreak \noindent 
            A multiplexer can be used to send data from a given memory cell to a register. The memory address is given in $S_{1}S_{0}$ .
        \item \textbf{Shifter}: This shifter moves the bits of a nibble one position to the left or right.
            \bigbreak \noindent 
            If $S = 1$, we have a right shift. If $S = 0$, we have a left shift.
            \bigbreak \noindent 
            \fig{.5}{./figures/mane.png}
            \bigbreak \noindent 
            \begin{align*}
                G_{1} &= I_{3}S\\
                G_{2} &= I_{2}\bar{S}\\
                G_{3} &= I_{2}S\\
                G_{4} &= I_{1}\bar{S}\\
                G_{5} &= I_{1}S\\
                G_{6} &= I_{0}\bar{S}\\
                G_{7} &= G_{1} + G_{4} = I_{3}S + I_{1}\bar{S} \\
                G_{8} &= G_{3} + G_{6} = I_{2}S + I_{3}S
            \end{align*}
        \item \textbf{Two bit ALU}: The ALU, or arithmetic-logic unit, is a core part of the CPU that handles arithmetic and logic
            \bigbreak \noindent 
            We can use the combinational circuits we have learned to build an ALU. Our sample ALU will handle four operations - AND, OR, NOT and addition - for two-bit numbers.
            \bigbreak \noindent 
            The inputs are $A_{1}A_{0}$ and $B_{1}B_{0}$. The output is $C_{1}C_{0} $.
            \bigbreak \noindent 
            The control lines $I_{0}I_{1}$ determine which operation will be done
            \begin{itemize}
                \item 00 = Addition
                \item 01 = Not
                \item 10 = Or 
                \item 11 = And
            \end{itemize}
            \bigbreak \noindent 
            \fig{.7}{./figures/alu.png}
    \end{itemize}

    \pagebreak 
    \subsubsection{Sequential Circuits}
    \begin{itemize}
        \item \textbf{Intro}: Combinational logic circuits are perfect for situations when we require the immediate application of a Boolean function to a set of inputs
            \bigbreak \noindent 
            There are other times, however, when we need a circuit to change its value with consideration to its current state as well as its inputs.
            \bigbreak \noindent 
            These circuits have to "remember" their current state.
            \bigbreak \noindent 
            \textit{Sequential logic circuits} provide this functionality for us.
        \item \textbf{Sequential circuits}: As the name implies, sequential logic circuits require a means by which events can be sequenced.
            \bigbreak \noindent 
            State changes are controlled by clocks.
            \bigbreak \noindent 
            Sequential circuits are used anytime that we have a "stateful" application.
            \bigbreak \noindent 
            A stateful application is one where the next state of the machine depends on the current state of the machine and the input
            \bigbreak \noindent 
            A stateful application requires both combinational and sequential logic.
        \item \textbf{Clock}: A "clock" is a special circuit that sends electrical pulses through a circuit.
            \bigbreak \noindent 
            Clocks produce electrical waveforms such as the one shown below
            \bigbreak \noindent 
            \fig{.5}{./figures/clock.png}
        \item \textbf{State changes}: State changes occur in sequential circuits only when the clock ticks
            \bigbreak \noindent 
            Circuits can change state on the rising edge, falling edge, or when the clock pulse reaches its highest voltage
            \bigbreak \noindent 
            \fig{.5}{./figures/100.png}
            \bigbreak \noindent 
            Circuits that change state on the rising edge, or falling edge of the clock pulse are called edge-triggered
            \bigbreak \noindent 
            Level-triggered circuits change state when the clock voltage reaches its highest or lowest level.
            \bigbreak \noindent 
            In general, the circuits that we will look at will get their inputs on the rising edge.
            \bigbreak \noindent 
             By the time of the falling edge, the output will be stable and we can use it.
        \item \textbf{Feedback}: To retain their state values, sequential circuits rely on feedback.
            \bigbreak \noindent 
            Feedback in digital circuits occurs when an output is looped back to the input
            \bigbreak \noindent 
            A simple example of this concept is shown below.
        \item \textbf{SR flipflop (Set/Reset)}; You can see how feedback works by examining the most basic sequential logic components, the SR flipflop
            \bigbreak \noindent 
            The block diagram for an SR flipflop is shown below along with its characteristic table.
            \bigbreak \noindent 
            \fig{.5}{./figures/sr.png}
            \bigbreak \noindent 
            \begin{center}
                \begin{tabular}{cc|c}
                    $S$&$R$&$Q(t+1)$ \\ 
                    \hline
                    0 & 0 & $Q(t)$ (no change) \\
                    0 & 1 & 0 (reset to 0)\\ 
                    1 & 0 & 1 (set to 1)\\
                    1 & 1 & undefined
                \end{tabular}
            \end{center}
            \bigbreak \noindent 
            The characteristic table describes its behavior. 
            \bigbreak \noindent 
            $Q(t)$ means the value of the output at time $t$. $Q(t+1)$ is the value of $Q$ after the next clock pulse.
            \bigbreak \noindent 
            The internals are
            \bigbreak \noindent 
            \fig{.5}{./figures/sr2.png}
            \bigbreak \noindent 
            We see that the top gate has output $(\bar{Q}(t))$, and the bottom has output $Q(t)$. The top gate gives
            \begin{align*}
                \bar{Q}(t+1) = \overline{S + Q(t)}
            \end{align*}
            and the bottom gives
            \begin{align*}
                Q(t+1) = \overline{R + \bar{Q}(t)}
            .\end{align*}
            At the clock tick, the top gets input $S$ and $Q(t)$, and the bottom gate gets input $R$ and $\bar{Q}(t)$.
            \bigbreak \noindent 
            For example, if $S = 1,\; R= 0$, the outputs for the top and bottom gates, respectively are
            \begin{align*}
                \bar{Q}(t+1) &= \overline{S + Q(t)} = \overline{1 + Q(t)} = \bar{1} = 0, \\
                Q(t+1) &= \overline{R + \bar{Q}(t)} = \overline{0 + \bar{Q}(t)} = \overline{\bar{Q}(t)}
            .\end{align*}
            But, we have that the complement of $Q(t+1)$, $\bar{Q}(t+1)$ is zero, so $Q(t+1)$ is one. 
            \bigbreak \noindent 
            If $S = R = 0$, then the gates give (top to bottom)
            \begin{align*}
                \bar{Q}(t+1) &= \overline{0 + Q(t)} = \overline{Q(t)}, \\
                Q(t+1) &= \overline{0 + \bar{Q}(t)} = \overline{\bar{Q}(t)}
            .\end{align*}
            From here we can deduce that if $Q(t)=0$ or $Q(t) = 1$, $Q(t+1) = 0$, which is expected.
            \bigbreak \noindent 
            \textbf{Note:} To understand basic computer architecture, we can consider flipflops as fundamental units, just as we treat gates, i.e., we don't need to know their internals.
            \bigbreak \noindent 
            The SR flipflop actually has three inputs: $S$, $R $, and its current output, $Q$.
            \bigbreak \noindent 
            Thus, we can construct a truth table for this circuit, as shown at the right.
            \bigbreak \noindent 
            Notice the two undefined values. When both $S$ and $R$ are 1, the SR flipflop is unstable
            \bigbreak \noindent 
            \[
                \begin{array}{|c|c|c||c|}
                    \hline
                    \textbf{S} & \textbf{R} & \textbf{Q(t)} & \textbf{Q(t+1)} \\
                    \hline
                    0 & 0 & 0 & 0 \\
                    0 & 0 & 1 & 1 \\
                    0 & 1 & 0 & 0 \\
                    0 & 1 & 1 & 0 \\
                    1 & 0 & 0 & 1 \\
                    1 & 0 & 1 & 1 \\
                    1 & 1 & 0 & \text{undefined} \\
                    1 & 1 & 1 & \text{undefined} \\
                    \hline
                \end{array}
            \]
        \item \textbf{JK Flipflop}: If we can be sure that the inputs to an SR flipflop will never both be 1, we will never have an unstable circuit
            \bigbreak \noindent 
            The SR flipflop can be modified to provide a stable state when both inputs are 1.
            \bigbreak \noindent 
             This modified flipflop is called a JK flipflop, shown at the right.
             \bigbreak \noindent 
             \textbf{Note:} The "JK" is named after \textit{Jack Kilby.}
             \bigbreak \noindent 
             \fig{.7}{./figures/jk.png}
             \bigbreak \noindent 
             At the right, we see how an SR flipflop can be modified to create a JK flipflop
             \bigbreak \noindent 
             \fig{.7}{./figures/jk2.png}
             \bigbreak \noindent 
             The characteristic table indicates that the flipflop is stable.
             \bigbreak \noindent 
             \begin{center}
                 \begin{tabular}{cc|c}
                     $J$ & $K$ & $Q(t+1)$  \\
                     \hline
                     0 & 0 & $Q(t)$ \\
                     0 & 1 & 0 \\
                     1 & 0  & 1\\
                     1 & 1 & $\bar{Q}(t)$
                 \end{tabular}
             \end{center}
             \bigbreak \noindent 
             When $J$ and $K$ are both 1, the JK flipflop changes its output from the preceding value.

            \item \textbf{D Flipflop}: The output of the D flipflop remains the same during subsequent clock pulses. The output changes only when the value of $D$ changes.
                \bigbreak \noindent 
                A $D$ flipflop is built by sending $D$ and $D^{\prime}$ into the $S$ and $R$ inputs of an SR flipflop.
                \bigbreak \noindent 
                The characteristic table is 
                \bigbreak \noindent 
                \begin{center}
                    \begin{tabular}{c|c}
                         $D$&$Q(t+1)$ \\ 
                        \hline
                         0 & 0 \\ 
                         1 & 1
                    \end{tabular}
                \end{center}
                \bigbreak \noindent 
                With externals
                \bigbreak \noindent 
                \fig{.5}{./figures/d.png}
                \bigbreak \noindent 
                Since its value is constant unless the input changes, the $D$ flipflop is the fundamental circuit of computer memory.
                \bigbreak \noindent 
                $D$ flipflops are usually illustrated using the block diagram shown below.
                \bigbreak \noindent 
                 In addition to $D$, the clock input is needed as for any sequential circuit.
                 \bigbreak \noindent 
                 It is convenient but not required to output both $Q$ and $Q^{\prime} $.
                 \bigbreak \noindent 
                 \fig{.5}{./figures/d2.png}
            \item \textbf{Register}: This illustration shows a 4-bit register consisting of $D$ flipflops. You will usually see its block diagram (below) instead.
                \bigbreak \noindent 
                \fig{.5}{./figures/reg.png}
                \bigbreak \noindent 
                \fig{.5}{./figures/reg2.png}
                \bigbreak \noindent 
                This register maintains its state as the clock ticks because the value of $\text{In}_{n}$ is equal to the value of $\text{out}_{n}$ .
            \item \textbf{Memory}: The following image represents four words of memory. Each word has 3 bits, represented in one column
                \bigbreak \noindent 
                A 4-word memory needs 2 selector lines to address the four words. We need to know which word the data is going to or from
                \bigbreak \noindent 
                Note that $S_{1}$ is the high-order bit even though it’s shown on the right. Look carefully at the AND gates $S_{0}$ and $S_{1}$ feed into
                \bigbreak \noindent 
                When data is being written into memory, it comes from the 3 input bits on the left.
                \bigbreak \noindent 
                $\text{In}_{0}$ is connected to bit 0 of each word, but the update only happens if the corresponding selector line is 1 (and the write enable bit is 1) when the clock ticks.
                \bigbreak \noindent 
                \fig{.5}{./figures/memory.png}
            \item \textbf{Memory read}: When data is being read from memory (i.e., output), it goes into the 3 output bits on the right.
                \bigbreak \noindent 
                Each output bit contains the ‘OR’ of the corresponding output bits from the four words.
                \bigbreak \noindent 
                Only one of those bits can be true at any one time. That’s the output bit from the word that is selected, i.e., the word whose number (address) is represented by the selector lines.
                \bigbreak \noindent 
                Memory read (i.e., output) does not require a clock tick. The output is always there. To get the data from the $Q$ signals for each memory cell to Out only requires combinational logic. The clock isn't needed for combinational logic.
            \item \textbf{Memory Read example}: We will use bit 0 as an example again. Bit 0 of each word is AND-ed to the selector for that word number
                \bigbreak \noindent 
                So bit 0 of a given word is only output from the $D$ flipflop it’s stored in when the selector for that word is on.
                \bigbreak \noindent 
                All of the bit 0’s from the different words are OR’d together to form output bit $\text{Out}_{0}$ , but only one of them (the one from the selected word) will have a 1 in it.
                \bigbreak \noindent 
                So the value of $\text{Out}_{0}$ will contain the value of bit 0 of the selected word, i.e., whatever is in the $D$ flipflop. If the word had been updated during this clock cycle. $\text{Out}_{0} $ will contain the new value. Otherwise it will contain the old value. Again, that’s what we expect from a computer memory.
            \item \textbf{Memory write}
                \bigbreak \noindent 
                memory update (i.e., input) only happens when the clock ticks. That’s the only time the clock bit in the lower left is on. Input is sampled on the rising edge and the output is stable by the time of the falling edge.
                \bigbreak \noindent 
                Also, memory update (i.e., input) only happens when the write enable bit is on.
                \bigbreak \noindent 
                Memory can only be updated when the "write enable" bit is on. There is a write enable line that is AND-ed with the clock
                \bigbreak \noindent 
                The clock ticks regularly, and there is always some value on the input lines (every bit is either 0 or 1), but we don’t want to update memory every time the clock ticks.
                \bigbreak \noindent 
                So we only turn the write enable bit on when we want to update memory
            \item \textbf{Memory write example}: We will use bit 0 as an example. All 3 bits act the same way
                \bigbreak \noindent 
                Input $\text{In}_{0}$ is connected to bit 0 of each of the words, but the data in bit $\text{In}_{0}$ does not flow to bit 0 of each word. It only flows to the word represented by the selector.
                \bigbreak \noindent 
                Furthermore, the new value from $\text{In}_{0}$ is only written when the clock ticks and write enable is on. (You can see the clock line and the write enable line AND-ed together, and that result AND-ed together with the selector line for the column.)
                \bigbreak \noindent 
                If write enable is off or the particular word is not selected, the data in the memory remains the same. That’s how a $D$ flipflop works, and it’s also what you’d expect from a computer memory. We want values to be stable unless updated.



            \item \textbf{Implementing $y=x$}: Which is \texttt{LOAD $x$; STORE $y$}
                \bigbreak \noindent 
                First,  read the value of $x$ from memory:
                \begin{enumerate}
                    \item Load the address of $x$ on the selector lines
                    \item Now the value of $x$ will be available on the Out lines.
                \end{enumerate}
                Then, write the value of $y$ into memory:
                \begin{enumerate}
                    \item Load the address of $y$ on the selector lines
                    \item Load the new value of $y$ (i.e., the value of $x$) on the In lines.
                    \item Turn on the write enable bit
                    \item The next time the clock ticks, the new value of $y$ will be loaded into the cells pointed to by the address of $y$
                \end{enumerate}

                \bigbreak \noindent 
                \textbf{Note:} To implement something like $y=x+1$, we basically do the same thing as above, but between outputting the value of $x$ and loading that value int $y$, we feed the output of $x$ into an ALU to get $x+1 $

            \item \textbf{Counter}: A binary counter is another a sequential circuit. Among other things, it can be used to keep track of how many cycles the clock has ticked.
                \bigbreak \noindent 
                This is a 4-bit counter. It represents the 4-bit binary number $B_{3}B_{2}B_{1}B_{0}$.  
                \bigbreak \noindent 
                \fig{.5}{./figures/78.png}
                \bigbreak \noindent 
                It counts $B'0000'$, $B'0001'$, etc. Note that the high-order bits are at the bottom of the diagram.  
                \bigbreak \noindent 
                After it hits $B'1111'$, it returns to $B'0000'$. Now it can count forever.
                \bigbreak \noindent 
                When the count enable line is on, a pulse is produced every time the clock ticks.
                \bigbreak \noindent 
                The input is sampled on the rising clock edge and the output is stable at the time of the falling clock edge.
                \bigbreak \noindent 
                The count enable line feeds into both $J$ and $K$ of the top flip-flop.  
                \bigbreak \noindent 
                When $J = 1$ and $K = 1$ at a clock tick, the JK flip-flop changes state, i.e., the output is the inverse of the previous output. Assuming the original value of $B_{0}$ was $0$, it is now $1$. The value of $B_{0}$ is also sent to the AND gate below it. So $B_{3}B_{2}B_{1}B_{0}$ now equals $B'0001'$.
                \bigbreak \noindent 
                When will $B_{1}$ flip to $1$?  On the next clock pulse, the $J = 1$, $K = 1$ input causes $B_{0}$ to flip from $1$ to $0$. At the same time, the $1$ from $B_{0}$ is AND-ed with the $1$ from the count enable line and sent to $B_{1}$, causing it to flip from $0$ to $1$. So $B_{3}B_{2}B_{1}B_{0}$ now equals $B'0010'$.
                \bigbreak \noindent 
                When will $B_{2}$ flip to $1$?  Its input comes from an AND gate. One of the inputs is $B_{1}$, and the other is the AND of the clock enable line and $B_{0}$.  
                \bigbreak \noindent 
                The clock enable is $1$ during counting, so the relevant part is $B_{0}$.  
                \bigbreak \noindent 
                In other words, $B_{2}$ will become $1$ when $B_{1} = 1$ and $B_{0} = 1$, as we expect from a counter. So the value after $B'0010'$ is $B'0011'$, and the following value is $B'0100'$.
                \bigbreak \noindent 
                When $B_{3}B_{2}B_{1}B_{0} = B'1111'$, the largest value that the counter can hold has been reached.  
                \bigbreak \noindent 
                The next cycle will reset $B_{3}B_{2}B_{1}B_{0}$ to $B'0000'$ (i.e., all four bits will flip).  
                \bigbreak \noindent 
                You can use a combinational circuit to add $1$ to a number, but you need a sequential circuit, i.e., one that knows its previous value, to implement a counter.

    \end{itemize}
    
    \pagebreak 
    \subsection{Chapter 2}
    \bigbreak \noindent 
    \subsubsection{Binary Fractions}
    \begin{itemize}
        \item \textbf{Intro}: Fractions do not necessarily have exact representations in all bases, i.e., with a fixed number of digits.
            \bigbreak \noindent 
            In base 10, if the denominator contains powers of 2 and/or 5, the fraction can be exactly represented.
            \bigbreak \noindent 
            If a fraction cannot be exactly represented, it can always be represented by a fixed part plus a repeating part
            \bigbreak \noindent 
            In binary, only sums of powers of 2 can be exactly represented.
            \begin{align*}
                \frac{1}{2} &= 0b.1\\
                \frac{1}{4} &= 0b.01 \\
                \frac{1}{8} &= 0b.001 \\
                \frac{3}{4} &= \frac{1}{2} + \frac{1}{4} = 0b.11 \\
                \frac{1}{2^{20}} &= 0b.00000000000000000001
            \end{align*}
            And here are some fractions that can’t be exactly represented:
            \begin{align*}
                \frac{1}{3} &= 0b.010101...
                \frac{1}{12} &= 0b.00010101...
                \frac{1}{5} &= 0b.00110011...
                \frac{1}{10} &= 0b.000110011...
            \end{align*}
            Again, there is a repeating part, possibly preceded by a non-repeating part.
            \bigbreak \noindent 
            The fact that most fractions can’t be exactly represented in a given number of bits is extremely important in computer science.
        \item \textbf{Truncation error}:  Since any field has a finite width, we need to round off or truncate after that width is reached.
            \bigbreak \noindent 
            Usually we truncate, since that is easier for computer systems to do
            \bigbreak \noindent 
            That means that binary fractions are dangerous for representing money, for example, where we need an exact representation
            \bigbreak \noindent 
            For non-financial systems, we generally just need to know what the maximum possible error is.
        \item \textbf{Fractional representation}: Digits to the right of the radix point represent negative powers of the radix.
            \bigbreak \noindent 
            The radix point is called the decimal point or binary point in decimal and binary, respectively.
            \bigbreak \noindent 
            In base 10,
            \begin{align*}
                0.47_{10} = 4 \times 10^{-1} + 7 \times 10^{-2}
            \end{align*}
            In base 2,
            \begin{align*}
                0.11_{2} &= 1 \times 2^{-1} + 1 \times 2^{-2} \\
                         &= \frac{1}{2} + \frac{1}{4} = 0.75
            \end{align*}
        \item \textbf{Decimal fractions to binary (subtraction method)}: For converting decimal fractions to binary, there is a subtraction method and a multiplication method.
            \bigbreak \noindent 
            We always start with the largest value first, $2^{-1}$ , which is .5. Then we subtract $2^{-2} $, which is .25, etc
            \bigbreak \noindent 
            For example, suppose we want to convert $0.8125$ to binary. We subtract $\frac{1}{2}$, then $\frac{1}{4}$, etc. If the number is too small to subtract the current fraction, that bit gets a zero.
            \bigbreak \noindent 
            \fig{.5}{./figures/101.png}
            \bigbreak \noindent 
            Our result, reading from top to bottom is
            \begin{align*}
                0.8125_{10} = 0.1101_{2}
            \end{align*}
            \bigbreak \noindent 
            We stop when we reach 0, or when we reach the desired number of bits.
            \bigbreak \noindent 
            For example, if we were converting .8126 to a binary number with a width of 4 bits, we’d still get $0b.1101$, with an error term of $.0001_{10}$ (because there would be a 1 instead of a 0 at the end.)
            \bigbreak \noindent 
            \textbf{Note:} The subtraction method is not always convenient because it requires a table of negative powers of 2: .5, .25, etc.
        \item \textbf{Inverses of powers of two (first 8)}
            \begin{align*}
                2^{-1} &=  \frac{1}{2} = 0.5 = 0b.1\\
                2^{-2} &= \frac{1}{4} = 0.25 = 0b.01\\
                2^{-3} &= \frac{1}{8} = 0.125 = 0b.001\\
                2^{-4} &= \frac{1}{16} = 0.0625 = 0b.0001\\
                2^{-5} &= \frac{1}{32} = 0.03125 = 0b.00001\\
                2^{-6} &= \frac{1}{64} = 0.015625 = 0b.000001\\
                2^{-7} &= \frac{1}{128} = 0.0078125 = 0b.0000001\\
                2^{-8} &= \frac{1}{256} = 0.00390625 = 0b.00000001\\
            \end{align*}
        \item \textbf{Number of output bits}: To know the number of output bits, you must look at the desired output data structure
            \bigbreak \noindent 
            Number of bits requested to the right of the point is often required to be a multiple of 2, 4 (nibble) or 8 (byte).
            \bigbreak \noindent 
            You can add as many 0’s to the right as desired.
        \item \textbf{Decimal fractions to binary (multiplication method)}: In this method, we repeatedly multiply by 2
            \bigbreak \noindent 
            Suppose we again want to convert decimal $0.8125$, we begin by multiply by two.
            \begin{align*}
                0.8125 \times 2 = 1.6250
            \end{align*}
            Whenever we get a part that’s no longer a fraction (like the 1 in the example), we have finished converting that part, so we drop it from our calculation and save it for the binary result
            \bigbreak \noindent 
            continue multiplying each fractional part by 2.
            \bigbreak \noindent 
            We drop the 1, then multiply the remaining fraction (.6250) by 2.
            \bigbreak \noindent 
            \fig{.8}{./figures/102.png}
            \bigbreak \noindent 
            We are finished when the remaining piece of the input is zero, as in this example, or until we have reached the desired number of binary places.
            \bigbreak \noindent 
            Our result, reading from top to bottom is
            \begin{align*}
                0.8125_{10} = 0b.1101
            \end{align*}
        \item \textbf{Decimal fraction to hex}: Convert it to a binary fraction with the desired number of bits
            \bigbreak \noindent 
            Even if the conversion terminates earlier, fill out the binary fraction on the right with 0’s to a full number of bytes.
            \bigbreak \noindent 
            Convert each nibble to hex
            \bigbreak \noindent 
            For example, $\frac{7}{8} = 0b.11100000  = 0x.e0$
        \item \textbf{Converting binary fractions to decimal}:
            Converting binary fractions to decimal is a two-step process:
            \begin{enumerate}
                \item Convert the binary number to the right of the point to decimal.
                \item Divide it by the appropriate power of 2
            \end{enumerate}
            For example, if we wanted to convert $0b.1101$ to decimal, $1101 = 13$, and there are four bits after the binary point, so the denominator is $2^{4}$. So, the result is $\frac{13}{2^{4}} =\frac{13}{16}$
        \item \textbf{Hex fractions to decimal}: Hex fractions can be converted to decimal in the same way
            \bigbreak \noindent 
            For example, $0x.d1$, $0xd1 = 209$, and there are two hex digits after the decimal, so we divide by $16^{-2} $
            \bigbreak \noindent 
            You can also convert the hex fraction to binary first
        
    \end{itemize}

    \pagebreak 
    \subsubsection{Twos complement}
    \begin{itemize}
        \item \textbf{Signed integers}: The conversions we have so far presented have involved only positive numbers.
            \bigbreak \noindent 
            There are three representation systems for signed integers:
            \begin{enumerate}
                \item Signed magnitude
                \item One’s complement
                \item Two’s complement
            \end{enumerate}
            We only use the last one. We only look at the first two to show their disadvantages.
            \bigbreak \noindent 
            For signed numbers, it is essential to know the width of the field, e.g., 8, 16, 32 or 64 bits. Other widths are occasionally used.
        \item \textbf{Signed magnitude}: The signed magnitude system uses the highorder bit to indicate the sign of a value.
            \bigbreak \noindent 
            The high-order bit is the leftmost bit in a byte.
            \bigbreak \noindent 
            It is also called the most significant bit (MSB) because in an unsigned number, it represents the largest value.
            \bigbreak \noindent 
            We will occasionally also use the term LSB, least significant bit, for the rightmost bit, or the bit in the 1’s position.
            \bigbreak \noindent 
            An 8-bit signed magnitude number contains:
            \begin{itemize}
                \item \textbf{bit 1:} sign bit (leftmost bit)
                \item \textbf{bits 2-8:} absolute value of the number (7 bits)
            \end{itemize}
            \bigbreak \noindent 
            For example, in 8-bit signed magnitude,
            \begin{align*}
                3 &= 0000 \;\; 0011 \\
                -3 &= 1000 \;\; 0011 
            \end{align*}
            \bigbreak \noindent 
            Signed magnitude has two significant disadvantages.
            \begin{itemize}
                \item It requires multiple rules to do arithmetic with signed magnitude numbers.
                \item There are two different reps. for zero, e.g., 0000 0000 and 1000 0000 in an 8-bit system.
                    \bigbreak \noindent 
                    But we want each number to have a unique representation so we can do comparisons.
            \end{itemize}
            For these reasons many computer systems employ complement systems to represent numeric values.
        \item \textbf{One's complement}: To calculate the one’s complement of a number, we flip each bit.
            \bigbreak \noindent 
            For example, in 8-bit one’s complement:
            \begin{align*}
                +3 &= 0000 \;\; 0011 \\
                -3 &= 1111 \;\; 1100
            \end{align*}
            In an 8-bit one’s complement system, the largest absolute value uses 7 bits.
            \begin{align*}
                +127 &= 0111 \;\; 1111 \quad \text{(Largest)} \\
                -127 &= 1000 \;\; 0000 \quad \text{(Smallest)}
            \end{align*}
            The high-order bit is the sign bit. As in the examples above, it identifies the sign and is not part of the magnitude.
            \begin{itemize}
                \item Positive numbers always have a 0.
                \item Negative numbers always have a 1.
            \end{itemize}
        \item \textbf{Propagating the sign bit}: Again, notice the importance of the field width. In 8-bit one’s complement:
            \begin{align*}
                +3 &= 0000 0011 \\
                -3 &= 1111 1100
            \end{align*}
            In 16-bit one’s complement:
            \begin{align*}
                +3 &= 0000 0000 0000 0011 \\
                -3 &= 1111 1111 1111 1100
            \end{align*}
            So you can’t left fill a negative number with 0’s to move it to a larger field. You have to left fill negative numbers with 1’s. This is called propagating the sign bit
        \item \textbf{Bit flipping}: Suppose you have a number $b$ in binary, flipping all the bits of $b$ is equivalent to computing
            \begin{align*}
                (2^{N} - 1) - b
            \end{align*}
            \bigbreak \noindent 
            For an $N$ bit binary number, the maximum value is $2^{N}-1$, which would be all $N$ bits set to one. Take each bit of $b$
            \begin{itemize}
                \item If the bit is 0, subtracting it from a 1 leaves 1.
                \item If the bit is 1, subtracting it from a 1 leaves 0.
            \end{itemize}
            That's exactly bit flipping
        \item \textbf{One's complement addition}:  In one’s complement addition, the carry bit is "carried around" and added to the sum.
            \bigbreak \noindent 
            \fig{.8}{./figures/103.png}
        \item \textbf{Why does end-around carry work?}: 
            Suppose $a$ and $b$ are positive numbers. How would you add $a + (-b)$? You would calculate $a-b$
            \bigbreak \noindent 
            Now suppose $b $ is represented in one’s complement as $(2^{N}-1) - b$.
            \bigbreak \noindent 
            Then doing unsigned arithmentic would give $a + (2^{N}-1 - b) = 2^{N}-1 + (a-b)$.
            \bigbreak \noindent 
            But the correct value is $2^{N} + (a-b)$, so we need to add 1
            \bigbreak \noindent 
            Although end-around carry adds some complexity, one’s complement addition is simpler to implement than signed magnitude.
            \bigbreak \noindent 
            But it still has the disadvantage of having two different representations for zero:
            \begin{align*}
                0000 0000 &= +0 \\
                1111 1111 &= -0
            \end{align*}
            Two’s complement solves this problem.
        \item \textbf{Two's complement}: To express a value in two’s complement:
            \begin{itemize}
                \item If the number is positive, just convert it to binary.
                \item If the number is negative, find the one’s complement of the number and then add 1.
            \end{itemize}
            With two’s complement arithmetic, we can add binary numbers directly. We discard any carries emitted from the high order bit.
        \item \textbf{MSB in two's complement}: 
            \begin{itemize}
                \item \textbf{Non-negative number:} MSB = 0 
                \item \textbf{Negative number:} MSB = 1
            \end{itemize}
        \item \textbf{Range of signed (two's complement) numbers}:  For an $N$ bit number, where $n = N-1$, the range is
            \begin{align*}
                \text{Range} &= -2^{N-1} \text{ to } 2^{N-1}-1 \\
                             &= -2^{n} \text{ to } 2^{n} - 1
            .\end{align*}
            For example, the range of an $8-bit$ number is $-2^{7} \text{ to } 2^{7} - 1 = -128 \text{ to } 127$. Thus, $127 + 128 + 1 =  256$ possible numbers. Observe that we add one to include the value zero.
            \bigbreak \noindent 
            \textbf{Note:} The asymmetry is the price of having a unique zero.
        \item \textbf{Some representations (8-bit)}:
            \begin{itemize}
                \item $0$ and $-0$: $00000000$
                \item $-1$: $11111111$
                \item $+127$: $01111111$
                \item $-127$: $10000001 $
                \item $-128$: $10000000 $
            \end{itemize}
    \end{itemize}

    \pagebreak 
    \subsubsection{Overflow}
    \begin{itemize}
        \item \textbf{Intro}: What happens when the result of a calcuation doesn’t fit in the given field width?
            \bigbreak \noindent 
            When we detect overflow, we can compensate for it and/or notify the user.
            \bigbreak \noindent 
            In complement arithmetic, an overflow condition is easy to detect
        \item \textbf{Rule for detecting signed two’s complement overflow}: When the "carry in" and the "carry out" of the sign bit differ, overflow has occurred.
            \bigbreak \noindent 
            Adding two numbers of opposite sign never causes overflow.
        \item \textbf{Twos complement to decimal}: To convert from two’s complement to decimal:
            \begin{itemize}
                \item \textbf{If the sign bit is zero}: Just convert to decimal the unsigned way.
                \item \textbf{If the sign bit is one}: There are two methods
                    \begin{enumerate}
                        \item Subtract one, flip bits, convert to decimal, add minus sign
                        \item Flip bits, add one, convert to decimal, add minus sign
                    \end{enumerate}
            \end{itemize}
    \end{itemize}

    \pagebreak 
    \subsubsection{Binary math}
    \begin{itemize}
        \item \textbf{Binary multiplication}: 
            Can do binary multiplication the way people multiply by hand.
            \bigbreak \noindent 
            Easy to do in hardware because the algorithm is made up of adds and shifts.
            \bigbreak \noindent 
            \fig{.8}{./figures/1001.png}
        \item \textbf{Binary multiplication with Booth's algorithm}:
            Research into finding better arithmetic algorithms has continued for over 50 years.
            \bigbreak \noindent 
            The methods we use by hand are not the most efficient possible.
            \bigbreak \noindent 
            Booth’s algorithm carries out multiplication faster and more accurately than our usual algorithm.
            \bigbreak \noindent 
            The general idea is to replace arithmetic operations with bit shifting to the extent possible.
            \bigbreak \noindent 
            Booth's algorithm is based on this idea:
            \begin{align*}
                n \times 1001 &= n \times (1000 + 1) \\
                n \times 999 &= n \times (1000 - 1)
            .\end{align*}
            In binary, this involves thinking of 1111 in the middle of a number as $10000 - 1$ in the correct position.
    \end{itemize}


    \pagebreak 
    \subsection{Chapter 4}
    \bigbreak \noindent 
    \subsubsection{The Von Neumann Model}
    \begin{itemize}
        \item \textbf{Introduction}: On the ENIAC, all programming was done at the digital logic level. Programming the computer involved moving plugs and wires. A different hardware configuration was needed to solve every unique problem type. 
            \bigbreak \noindent 
            Configuring the ENIAC to solve a simple problem required many days of labor by skilled technicians
            \bigbreak \noindent 
            Inventors of the ENIAC, John Mauchley and J. Presper Eckert, conceived of a computer that could store instructions in memory. The invention of this idea has since been ascribed to a mathematician, John von Neumann, who was a contemporary of Mauchley and Eckert.
            \bigbreak \noindent 
            Stored-program computers have become known as von Neumann architecture systems
        \item \textbf{Characteristics of a stored program computer}: Today’s stored-program computers have the following characteristics:
            \bigbreak \noindent 
            \begin{itemize}
                \item Three hardware systems: 
                    \begin{enumerate}
                        \item A central processing unit (CPU)
                        \item A main memory system
                        \item An I/O system
                    \end{enumerate}
                \item The capacity to carry out sequential instruction processing.
                \item A single data path between the CPU and main memory.
            \end{itemize}
            \bigbreak \noindent 
            \textbf{Note:} This single path is known as the von Neumann bottleneck.
        \item \textbf{The von Neumann model}: This is a general depiction of a von Neumann system:
            \bigbreak \noindent 
            \fig{.7}{./figures/1004.png}
            \bigbreak \noindent 
            These computers employ a fetch-decode-execute cycle to run programs
            \bigbreak \noindent 
            The control unit fetches the next instruction from memory using the program counter to determine where the instruction is located. The instruction is decoded into a language that the ALU can understand.
            \bigbreak \noindent 
            Data operands required to execute the instruction are fetched from memory and placed into registers in the CPU.
            \bigbreak \noindent 
            The ALU executes the instruction and places results in registers or memory.
        \item \textbf{Buses}: Buses allow data to flow between components.
            \bigbreak \noindent 
            \fig{.7}{./figures/1005.png}
            \bigbreak \noindent 
            As we see, there are three buses
            \begin{enumerate}
                \item Data bus
                \item Address bus
                \item Control bus
            \end{enumerate}
        \item \textbf{Modern improvements}:
            Conventional stored-program computers have undergone many incremental improvements over the years. These improvements include adding specialized buses, floating-point units, and cache memories, to name only a few.
            \bigbreak \noindent 
            But, enormous improvements in computational power require departure from the classic von Neumann architecture. Adding processors is one approach.
            \bigbreak \noindent 
            In the late 1960s, high-performance computer systems were equipped with dual processors to increase computational throughput. In the 1970s supercomputer systems were introduced with 32 processors.
            \bigbreak \noindent 
            Supercomputers with 1,000 processors were built in the 1980s. In 1999, IBM announced its Blue Gene system containing over 1 million processors.
    \end{itemize}


    \pagebreak 
    \subsubsection{CPU basics and the Bus}
    \begin{itemize}
        \item \textbf{CPU basics}: The computer’s CPU fetches, decodes, and executes program instructions
            \bigbreak \noindent 
        \item \textbf{Principal parts of the CPU}: The two principal parts of the CPU are the datapath and the control unit
            \begin{enumerate}
                \item The datapath consists of an arithmetic-logic unit (ALU) and storage units (registers) that are connected by a data bus that is also connected to main memory. 
                \item Various CPU components perform sequenced operations according to signals provided by its control unit.
            \end{enumerate}
        \item \textbf{CPU basics (2)}: Registers hold data that can be readily accessed by the CPU. They can be implemented using D flip-flops
            \bigbreak \noindent 
            The arithmetic-logic unit (ALU) carries out logical and arithmetic operations as directed by the control unit.
            \bigbreak \noindent 
            The control unit determines which actions to carry out according to the values in a program counter register and a status register
        \item \textbf{The Bus}: The CPU shares data with other system components by way of a data bus. A bus is a set of wires that simultaneously convey a single bit along each line.
        \item \textbf{Two types of buses}: Two types of buses are commonly found in computer systems: point-to-point, and multipoint buses
        \item \textbf{Data, control, and address lines}: Buses consist of data lines, control lines, and address lines.
            \begin{itemize}
                \item Data lines convey bits from one device to another
                \item Control lines determine the direction of data flow, and when each device can access the bus.
                \item Address lines determine the location of the source or destination of the data.
            \end{itemize}
        \item \textbf{Address bus}: The address bus is used by the CPU to send addresses to the main memory for lookup
        \item \textbf{Data bus}: Memory uses the data bus to send back data from the desired address. The CPU also uses the data bus to send data to memory for updating the value at an address.
            \bigbreak \noindent 
            \textbf{Note:} The CPU has a similar relationship to the I/O subsystem. All devices need the control bus.
        \item \textbf{The bus (figure)}:
            \bigbreak \noindent 
            \fig{.7}{./figures/1006.png}
        \item \textbf{Multipoint bus}: A multipoint bus is shown below. 
            \bigbreak \noindent 
            Because a multipoint bus is a shared resource, access to it is controlled through protocols, which are built into the hardware.
            \bigbreak \noindent 
            \fig{.7}{./figures/1007.png}
        \item \textbf{Bus Arbitration}: What happens when more than one device wants to use the bus at the same time? \textit{Arbitration} refers to the hardware algorithm (protocol) for resolving conflicts.
            \begin{itemize}
                \item \textbf{Daisy chain:} Permissions are passed from the highestpriority device to the lowest.
                \item \textbf{Centralized parallel:} Each device is directly connected to an arbitration circuit.
                \item \textbf{Distributed using self-detection:} Devices decide which gets the bus among themselves.
                \item \textbf{Distributed using collisiondetection:} Any device can try to use the bus. If its data collides with the data of another device, it tries again.
            \end{itemize}
    \end{itemize}

    \pagebreak 
    \subsubsection{Marie}
    \begin{itemize}
        \item \textbf{Intro}: We can now bring together many of the ideas that we have discussed to this point using a very simple model computer 
            \bigbreak \noindent 
            Our model computer, the Machine Architecture that is Really Intuitive and Easy, MARIE, was designed to illustrate basic architecture concepts.
            \bigbreak \noindent 
            While this system is too simple to do anything useful in the real world, understanding its functions will make it easier to understand more complex system architectures.
        \item \textbf{Characteristics}: The MARIE architecture has the following characteristics:
            \begin{enumerate}
                \item Binary, two's complement data representation.
                \item Stored program, fixed word length data and instructions.
                \item 4K words of word-addressable main memory.
                \item 16-bit data words.
                \item 16-bit instructions, 4 for the opcode and 12 for the address.
                \item 12-bit addresses
                \item A 16-bit arithmetic logic unit (ALU).
                \item Seven registers for control and data movement.
            \end{enumerate}
            \textbf{Note:} There are no general purpose registers in the Marie model. There are seven special (named) registers
        \item \textbf{Seven registers}:
            \begin{enumerate}
                \item \textbf{Accumulator, AC}: a 16-bit register that holds a conditional operator (e.g., "less than") or one operand of a two-operand instruction.
                \item \textbf{Memory buffer register, MBR}: a 16-bit register that holds the data after its retrieval from, or before its placement in memory.
                \item \textbf{Instruction register, IR}: a 16-bit register which holds an instruction immediately preceding its execution.
                \item \textbf{Memory address register, MAR}: a 12-bit register that holds the memory address of an instruction or the operand of an instruction.
                \item \textbf{Program counter, PC}: a 12-bit register that holds the address of the next program instruction to be executed
                \item \textbf{Input register, InREG}: an 8-bit register that holds data read from an input device.
                \item \textbf{Output register, OutREG}: an 8-bit register, that holds data that is ready for the output device.
            \end{enumerate}
        \item \textbf{Marie architecture graphic}
            \bigbreak \noindent 
            \fig{.6}{./figures/1021.png}
            \bigbreak \noindent 
            The registers are interconnected, and connected with main memory through a common data bus.
            \bigbreak \noindent 
            Each device on the bus is identified by a unique number that is set on the control lines whenever that device is required to carry out an operation.
            \bigbreak \noindent 
            Separate connections are also provided between the accumulator and the memory buffer register, and the ALU and the accumulator and memory buffer register.
            \bigbreak \noindent 
            This permits data transfer between these devices without use of the main data bus.
        \item \textbf{Bus addresses}:
            \bigbreak \noindent 
            The bus address refers to the address lines on the address bus — the part of the bus system that carries the memory address from the CPU to main memory.
            \bigbreak \noindent 
            MARIE uses a single shared 16-bit bus for both data and addresses (time-multiplexed). That means:
            \begin{itemize}
                \item The same 16 wires carry either addresses or data at different times.
                \item Control signals decide what’s currently being sent.
            \end{itemize}
            Each "Bus Address" (0–7) identifies which device (register or memory) is connected to that bus line.
            \bigbreak \noindent 
            When one component needs to send or receive data:
            \bigbreak \noindent 
            \begin{itemize}
                \item The Control Unit activates that device’s bus address line.
                \item That component either places data on the bus (output) or reads from it (input).
            \end{itemize}
            So "Bus Address 3" doesn’t mean memory address 3 — it means "the bus line connected to the MBR."

        \item \textbf{Marie data path}:
            \bigbreak \noindent 
            \fig{.6}{./figures/1022.png}
            \bigbreak \noindent 
            All the bits of the control bus enter and leave each component. The bus address on the diagram shows which bit configuration needs to be represented (e.g., 1 = 001, 2 = 010, etc.) in order for that component to know that it is being spoken to.
        \item \textbf{Marie instruction set (ISA)}:
            A computer’s instruction set architecture (ISA) specifies the format of its instructions and the primitive operations that the machine can perform.
            \bigbreak \noindent 
            The ISA is an interface between a computer’s hardware and its software. Some ISAs include hundreds of different instructions for processing data and controlling program execution.
            \bigbreak \noindent 
            The MARIE ISA has only thirteen instructions
        \item \textbf{Instruction format}:
            \bigbreak \noindent 
            \fig{.7}{./figures/1023.png}
        \item \textbf{Fundamental instructions}:
            \bigbreak \noindent 
            \fig{.7}{./figures/1024.png}
        \item \textbf{Labels in Marie}: In MARIE assembly, a label is used to mark a memory address so you can refer to it by name instead of by number.             You define a label like this:
            \bigbreak \noindent 
            \begin{cppcode}
                LOOP,   LOAD    X
            \end{cppcode}
            \bigbreak \noindent 
            The comma tells the assembler: "This is a label that identifies this memory address." The comma itself does not occupy memory and is not executed - it's purely syntactic.
            \bigbreak \noindent 
            We have three types of labels
            \begin{enumerate}
                \item Pure
                \item Data
                \item Subroutine
            \end{enumerate}
        \item \textbf{Defining storage in Marie}: In MARIE, you define storage (memory locations for data) at the end of your program using assembler directives like DEC or HEX
            \bigbreak \noindent 
            \begin{cppcode}
            LABEL,  DEC  value
            \end{cppcode}
        \item \textbf{Marie's characteristics}: Marie has
            \begin{itemize}
                \item One accumulator (AC)
                \item No stack
                \item No registers other than AC, MAR, MBR, IR, PC
                \item No call frame / local scope 
            \end{itemize}
        \item \textbf{Pure labels}: A pure label is simply a name that marks a memory address but does not define data
            \bigbreak \noindent 
            A pure label is just a symbolic name used to refer to a specific location in your program, such as a jump target.
            \bigbreak \noindent 
            \begin{cppcode}
            START,  LOAD   X
            \end{cppcode}
            \bigbreak \noindent 
            START, is a pure label - it marks the address of the first instruction (LOAD X).
        \item \textbf{Marie RTL}: Each of our instructions actually consists of a sequence of smaller instructions called \textbf{microoperations}.
            \bigbreak \noindent 
            The exact sequence of microoperations that are carried out by an instruction can be specified using register transfer language (RTL)
            \bigbreak \noindent 
            In the MARIE RTL, we use the notation $M[X]$ to indicate the actual data value stored in memory location $X$, and $\leftarrow $ to indicate the transfer of bytes to a register or memory location
        \item \textbf{Skipcond}: Skipcond allows us to skip the next instruction bases on the value of the accumulator. Recall that if a Marie instruction requires an operand, the rightmost 12 bits are the operand. In Skipcond, if we allowed say an address as an operand, then there would be no room for the condition. Thus, we do not have an address as an operand, instead we decide to use bits 10 and 11 as the condition. If the rightmost 12 bits of the skipcond operand are
            \begin{align*}
                b_{11}b_{10}b_{9}b_{8} \quad ... \quad b_{1}b_{0}
            \end{align*}
            then we evaluate skipcond as follows
            \begin{itemize}
                \item If $b_{11}b_{10} = 00$, skip the next instruction if the $AC$ is negative. (\texttt{skipcond 000})
                \item If $b_{11}b_{10} = 01$, skip the next instruction if the $AC$ is zero. (\texttt{skipcond 400})
                \item If $b_{11}b_{10} = 10$, skip the next instruction if the $AC$ is positive. (\texttt{skipcond 800})
            \end{itemize}
            So, if we wanted to skip the next instruction if the $AC$ is zero, our instruction would be
            \begin{align*}
                1000 \quad 0100 \quad 0000 \quad 0000
            .\end{align*}
            Which, is
            \begin{align*}
                8400
            \end{align*}
            in Hex, or just \texttt{skipcond 400}.
        \item \textbf{Looping with branch and skipcond}: Suppose we want to repeat a block of code five times, the general structure is
            \bigbreak \noindent 
            \begin{cppcode}
                    LOAD     five
            LOOP    ...
            .       .
            .       .
            .       .
                    SUBT     one
                    SKIPCOND 400
                    JUMP     LOOP
                    JUMP     END
            END     ...
            .       .
            .       .
            .       .
            \end{cppcode}
            \bigbreak \noindent 
            Provided that five and one are defined in memory.
        \item \textbf{If else statement}: Suppose we want to write an \texttt{if (x AND y)} statement, we could do
            \bigbreak \noindent 
            \begin{cppcode}
                    LOAD     x
                    SKIPCOND 400        / skip next if x == 0
                    JUMP     CHECK_Y    / x != 0, check y
                    JUMP     ELSE     / x == 0, skip entire if

            CHECK_Y, LOAD     y
                    SKIPCOND 400        / skip next if y == 0
                    JUMP     DO_IF      / both x and y nonzero
                    JUMP     ELSE     / y == 0, skip if body

            DO_IF,  ...                 / body of the if(x and y)
            .       .
            .       .
            .       .
                    
                    JUMP     FI
            ELSE,   ...
            .       . 
            .       .
            .       .

            FI
            \end{cppcode}
        \item \textbf{If $(x+y)$ else}: 
            \begin{cppcode}
                     LOAD     x
                     SKIPCOND 400       
                     JUMP     DO_IF     

            CHECK_Y, LOAD     y
                     SKIPCOND 400       
                     JUMP     DO_IF      
                     JUMP     ELSE     

            DO_IF,   ...              
            .        .
            .        .
            .        .
                    
                     JUMP     FI
            ELSE,    ...
            .        . 
            .        .
            .        .

            FI
            \end{cppcode}
        \item \textbf{If elseif, else}: Suppose we have some value \texttt{x}, a mystery value, and we want to write \texttt{if (x == 0) {} else if (x < 0) {} else {}}
            \bigbreak \noindent 
            \begin{cppcode}
                if (x == 0) {
                    ...
                } else if (x < 0) {
                    ...
                } else {
                    ...
                }
            \end{cppcode}
            \bigbreak \noindent 
            In Marie, we could write
            \bigbreak \noindent 
            \begin{cppcode}
                        LOAD     x
                        SKIPCOND 400
                        JUMP     ELSEIF
                DO_IF   ...
                .       .
                .       .
                .       .
                        JUMP     FI
                ELSEIF  SKIPCOND 000
                        JUMP     ELSE                       
                DO_ELIF ...
                .       .
                .       .
                .       .
                        JUMP     FI
                ELSE    ...
                .       .
                .       .
                .       .
                FI      ...
            \end{cppcode}
        \item \textbf{RTL for LOAD}:  The RTL for the LOAD instruction is:
            \bigbreak \noindent 
            \begin{align*}
                \text{MAR} &\leftarrow X \\
                \text{MBR} &\leftarrow M[\text{MAR}] \\
                \text{AC} &\leftarrow \text{MBR}
            .\end{align*}
        \item \textbf{RTL for STORE $X$}: The RTL for the STORE instruction is 
            \begin{align*}
                \text{MAR} &\leftarrow X \\
                \text{MBR} &\leftarrow \text{AC} \\
                M[\text{MAR}] &\leftarrow \text{MBR}
            .\end{align*}
        \item \textbf{RTL for ADD $X$}: Similarly, the RTL for the ADD instruction is
            \begin{align*}
                \text{MAR} &\leftarrow X \\
                \text{MBR} &\leftarrow M[\text{MAR}] \\
                \text{AC} &\leftarrow \text{AC} + \text{MBR}
            .\end{align*}
        \item \textbf{RTL for SUBT $X$}:
            \begin{align*}
                \text{MAR} &\leftarrow X \\
                \text{MBR} &\leftarrow M[\text{MAR}] \\
                \text{AC} &\leftarrow \text{AC} - \text{MBR}
            .\end{align*}
        \item \textbf{RTL for INPUT}:
            \begin{align*}
                \text{AC} &\leftarrow \text{inREG}
            .\end{align*}
        \item \textbf{RTL for OUTPUT}:
            \begin{align*}
                \text{outREG} &\leftarrow AC
            .\end{align*}
        \item \textbf{RTL for HALT}: No operations needed, no RTL
        \item \textbf{RTL for SKIPCOND}:
            \bigbreak \noindent 
            \begin{cppcode}
                if |$IR[11-10] = 00$| then
                    if |$AC < 0$| then 
                        |$PC \leftarrow PC + 1$|
                    endif

                else if |$IR[11-10] = 01$|
                    if |$AC = 0$| then
                        |$PC \leftarrow PC + 1$|
                    endif

                else if |$IR[11-10] = 10$| then
                    if |$AC > 0$| then
                        |$PC \leftarrow PC + 1$|
                    endif
                endif

            \end{cppcode}
        \item \textbf{RTL for JUMP $X$}:
            \begin{align*}
                \text{PC} &\leftarrow \text{IR}[11-0]
            .\end{align*}


        \item \textbf{Instruction Processing}: The fetch-decode-execute cycle is the series of steps that a computer carries out when it runs a program.
            \bigbreak \noindent 
            \begin{enumerate}
                \item We first have to fetch an instruction from memory, and place it into the IR.
                \item Once in the IR, it is decoded to determine what needs to be done next.
                \item If a memory value (operand) is involved in the operation, it is retrieved and placed into the MBR
                \item With everything in place, the instruction is executed.
            \end{enumerate}
            \bigbreak \noindent 
            \fig{.6}{./figures/1025.png}
            \bigbreak \noindent 
            \textbf{Note:} The fetch-decode-execute cycle is sometimes called the fetch-decode-fetch-execute cycle.
        \item \textbf{Example program}: Consider the simple MARIE program given below. We show a set of mnemonic instructions stored at addresses 100 - 106 (hex):
            \bigbreak \noindent 
            \fig{.6}{./figures/1026.png}
            \bigbreak \noindent 
            Let’s look at what happens inside the computer when our program runs. This is the LOAD 104 instruction.
            \bigbreak \noindent 
            \fig{.6}{./figures/1027.png}
            \bigbreak \noindent 
            The line in parentheses means that the RTN "branches" according to which instruction was decoded. That line needs no RTN because it uses combinational logic That instruction will be executed in the execute step
            \bigbreak \noindent 
            Our second instruction is ADD 105.
            \bigbreak \noindent 
            The fetch, decode and "get operand" (i.e., fetch the data) steps are the same. The only thing that is different is the execute, since the instruction is 'add' instead of 'load'.
            \bigbreak \noindent 
            \fig{.6}{./figures/1028.png}
            \bigbreak \noindent 
            The AC now has the value of AC + MBR. Remember that MARIE has 2’s complement addition.
        \item \textbf{Register transfer notation (RTL)}: RTN = register transfer notation, a synonym for RTL.
        \item \textbf{Indirect addressing}: So far, all of the MARIE instructions that we have discussed use direct addressing.
            This means that the address of the operand is explicitly stated in the instruction.
            \bigbreak \noindent 
            In indirect addressing, the address of the address of the operand is given in the instruction. In other words, the instruction gives the address of a memory location that contains the address of the operand.
            \bigbreak \noindent 
            Indirect addresses are needed to point at different values in a table so you can use a loop to process them, i.e., to get values from an address that changes during the run.
        \item \textbf{Indirect instructions}: MARIE contains two indirect instructions:
            \begin{itemize}
                \item ADDI $x$ (op code B): Look at address $x$, lookup the address stored there, and add that value to the AC.
                \item JUMPI $x$ (op code C): Branch to the address in memory cell $x$. Can be used to return from a subroutine implemented with JNS (see next slide for JNS).
            \end{itemize}
        \item \textbf{Additional instructions}: MARIE also contains two other useful instructions that we have not described before:
            \begin{itemize}
                \item JNS $x$ (jump and store) (op code 0): Store return address (i.e., next sequential address) at $x$, then jump to $x+1$.
                \item CLEAR (op code A) sets the contents of the accumulator to zero.
            \end{itemize}
        \item \textbf{JNS $x$ (jump and store)}: JNS $x $ does the following:
            \begin{enumerate}
                \item Stores the return address (i.e., the address of the next sequential instruction) at address $x$.
                \item Jumps to the address specified at address $x+1$
            \end{enumerate}
        \item \textbf{RTL for ADDI $x$}: The ADDI instruction specifies the address of the address of the operand. The following RTL shows us what is happening at the register level: 
            \begin{align*}
                \text{MAR} &\leftarrow x \\
                \text{MBR} &\leftarrow M[\text{MAR}] \\
                \text{MAR} &\leftarrow \text{MBR} \\
                \text{MBR} &\leftarrow M\text{MAR} \\
                \text{AC} &\leftarrow \text{AC} + \text{MBR}
            .\end{align*}
        \item \textbf{RTL for JNS $x$}
            \begin{align*}
                \text{MBR} &\leftarrow \text{PC} \\
                \text{MAR} &\leftarrow x \\
                M[\text{MAR}] &\leftarrow \text{MBR} \\
                \text{MBR} &\leftarrow x \\
                \text{AC} &\leftarrow 1\\
                \text{AC} &\leftarrow \text{AC} + \text{MBR} \\
                \text{PC} &\leftarrow \text{AC}
            .\end{align*}
        \item \textbf{RTL for CLEAR}: All it does is set the contents of the accumulator to all zeroes
            \begin{align*}
                \text{AC} &\leftarrow 0
            .\end{align*}
        \item \textbf{Subroutines}: The jump-and-store instruction, JNS, gives us limited subroutine functionality.
            \bigbreak \noindent 
            We jump to a subroutine with JNS $x$, where the subroutine begins at address $x+1$. That is, one word after the label used in JNS $(x)$
            \bigbreak \noindent 
            After the subroutine is finished, JUMPI $x$ can be used to return to the original address, since JUMPI jumps to the address stored at $x$.
            \bigbreak \noindent 
            \textbf{Notes}
            \begin{itemize}
                \item MARIE does not support parameter passing directly. You always work through global memory (shared variables in main memory).
                \item JNS cannot handle recursive calls because JNS $x$ always uses the same address, namely $x$, to store the return address. In other words, the return address is stored in the first word of the subroutine. (Of course, that is not good programming practice today.)
            \end{itemize}
            \bigbreak \noindent 
            Subroutine labels need to be defined as
            \bigbreak \noindent 
            \begin{cppcode}
            NAME,  HEX 0
            \end{cppcode}
            \bigbreak \noindent 
            Strictly because of the fact that JNS NAME jumps to the location NAME+1.
        \item \textbf{The control unit}: A computer’s control unit ensures that each instruction is executed in sequence, making sure that data flows to the correct components as each instruction is executed..
            \bigbreak \noindent 
            There are two general ways in which a control unit can be implemented: hardwired control and microprogrammed control.
            \begin{enumerate}
                \item In a hardwired machine, digital logic components are used to select the "from" and "to" components and ensure that data flows to them at the correct time.
                \item A microprogrammed machine contains a small program in a piece of read-only memory in a component called the microcontroller. The microprogram contains the RTL for each instruction. In order to execute an instruction, the microcontroller interprets the corresponding piece of this program.
            \end{enumerate}
        \item \textbf{Marie is a hardwired machine}: The operation of MARIE’s control unit is defined by the RTL for each instruction..
            \bigbreak \noindent 
            Each line of RTL is implemented with digital logic components that cause the appropriate data to flow from the component on the right-hand side of the RTL line to the left-hand side..
            \bigbreak \noindent 
            For example, the RTL for the ADD instruction is:
            \begin{align*}
                \text{MAR} &\leftarrow x \\
                \text{MBR} &\leftarrow M[\text{MAR}] \\
                \text{AC} &\leftarrow \text{AC} + \text{MBR}
            .\end{align*}
            We will soon see how each line of RTL is implemented.
        \item \textbf{Logical component}: When we call something a logical component, we mean it’s a functional part of the CPU — defined by what it does, not necessarily by where it physically sits.
            \begin{itemize}
                \item The Arithmetic Logic Unit (ALU) is the logical component that performs arithmetic and logic.
                \item The Control Unit (CU) is the logical component that issues control signals and manages sequencing.
                \item The Registers (like MAR, PC, AC, IR) are logical components that store data or addresses.
            \end{itemize}
        \item \textbf{More on Marie's control unit (CU)}: In MARIE, the Control Unit (CU) is part of the CPU, and its main role is to generate and coordinate the control signals that tell the rest of the system what to do during each phase of the fetch–decode–execute cycle.
            \bigbreak \noindent 
            It’s a logical component (a section of the CPU) responsible for directing the operation of the processor. It interprets the instruction currently in the Instruction Register (IR) and activates the correct control signals in the right sequence. These control signals go to components like the ALU, registers (AC, MAR, MBR, PC, IR), and the bus, ensuring data moves and operations occur correctly.
            \bigbreak \noindent 
            It’s not just a “storage area” that contains control signals — control signals are generated dynamically by the CU’s logic circuits, not stored like data. It doesn’t hold data like registers or memory do. Instead, it’s composed of decoding and timing logic that responds to the current instruction and clock cycle
            \bigbreak \noindent 
            When we say the control unit generates control signals, we mean it produces specific voltage patterns on wires (control lines) inside the CPU. Each instruction (like LOAD or ADD) has a unique combination of these voltage signals that tell different hardware components what to do that clock cycle.
            \bigbreak \noindent 
            Each control line is literally a wire connecting the control unit to some component (registers, ALU, memory interface, etc.).
        \item \textbf{Bus address signals}: Each of MARIE’s registers has a unique address along the datapath, the addresses take the form of signals issued by the control unit. MARIE needs 3 lines to identify addresses 0 - 7 on the bus.
            \bigbreak \noindent 
            Many machines would use 8 signals, one of which would be high at any given time.
            \bigbreak \noindent 
            \fig{.7}{./figures/1029.png}
            \bigbreak \noindent 
            The MARIE CPU contains two sets of three signals. $P_{2}$, $P_{1}$, and $P_{0}$ determine which component data will be read from. $P_{5}$, $P_{4}$, and $P_{3}$ determine which component data will be written to
            \bigbreak \noindent 
            For example, the MBR has address 3. Since the MBR is a 12-bit register, we can build it as a set of 12 D flipflops. We add gates implementing $P_{3}$ AND $P_{4}$ AND NOT $P_{5}$ before the input to these flipflops so that data will only flow into the MBR (i.e., writing to the MBR) when address 3 is activated.
            \bigbreak \noindent 
            Similarly, we add gates implementing $P_{0}$ AND $P_{1}$ AND NOT $P_{2}$ after the output of these flipflops so that data will only flow out of the MBR (i.e., reading from the MBR) to the bus when those signals are activated.
        \item \textbf{The MBR}:
            \bigbreak \noindent 
            \fig{.6}{./figures/1030.png}
        \item \textbf{First line of ADD RTL}: Let’s look again at the first line of the RTL for ADD:
            \begin{align*}
                \text{MAR} &\leftarrow x
            .\end{align*}
            After an ADD instruction is fetched, the address field, $x$, is in the rightmost 12 bits of the IR. The IR has a datapath address of 7.
            According to this line of RTL, $x$ must be copied to the MAR. The MAR has a datapath address of 1.
            Thus, we need to raise signals $P_{2}$, $P_{1}$, and $P_{0}$ to read from the IR, and signal $P_{3}$ to write to the MAR.
            \bigbreak \noindent 
            When the high order bits of the IR contain the opcode for ADD, these signals must be set high to implement the first line of the RTL. In succeeding clock ticks, the remaining lines of the RTL for ADD will be implemented.
        \item \textbf{Timing signals}: Most instructions need multiple clock ticks, one for each line of RTL. A binary counter is used to identify which line of RTL is being executed.
            \bigbreak \noindent 
            MARIE has a maximum of 3 lines of RTL for an instruction (7 if you include the advanced instructions). Therefore 4 timing signals (8 including the advanced instructions) are needed, one to indicate each line of RTL and one to reset the counter for the next instruction. These timing signals are named
            \begin{align*}
                T_{0}, T_{1}, T_{2}, ..., T_{7}
            .\end{align*}
            The $C_{r}$ signal is used to reset the counter.
            \bigbreak \noindent 
            Theoretically, we could use combinations of two timing signals to obtain 4 clock times, or combine 3 timing signals to get 8 clock times, but we won’t do it that way
        \item \textbf{ALU signals}: Looking at the RTL chart, we can see that the ALU has only three operations: add, subtract, and clear.
            \bigbreak \noindent 
            We have four ALU signals, $A_{0}$ through $A_{3}$, available to indicate these operations. We will use
            \begin{align*}
                A_{0} = \text{ add}, \quad A_{1} = \text{ subtract}, \quad A_{2} = \text{ clear}
            .\end{align*}
            We could use $A_{3}=$ no operation, but we won't. Another design alternative would be to use combinations of two signals to generate four values When we want to signal an ALU instruction, we raise the corresponding signal.
            \bigbreak \noindent 
            \textbf{Note:} These signals don’t need to match the ALU control values
        \item \textbf{Marie's control signals}: The entire set of MARIE’s control signals contains
            \begin{itemize}
                \item Register controls $P_{0}$ through $P_{5}$
                \item ALU controls $A_{0} $ through $A_{3}$
                \item Timing signals $T_{0} $ through $T_{7} $ and counter reset $C_{r} $
            \end{itemize}
            All of the RTL lines needed for MARIE’s instructions can be implemented by ensuring that the correct register and/or ALU control signals are raised during the correct clock ticks for a specific instruction.
            \bigbreak \noindent 
            Here is the full signal sequence for ADD. It shows which control signals must be set high at each clock tick
            \begin{align*}
                P_{2}\; P_{1}\; P_{0}\; P_{3}\; T_{0}&:\; \text{MAR} \leftarrow x \\
                P_{4}\; P_{3}\; T_{1}&:\; MBR \leftarrow M[MAR] \\
                A_{0}\; P_{1}\; P_{0}\; P_{5}\; T_{2}&:\; AC \leftarrow AC + MBR \\
                C_{r}\; T_{3}&:\; [Reset counter]
            .\end{align*}
            \begin{itemize}
                \item \textbf{Line 1:} input is address 7 (low order bits of the IR) and output is address 1 (MAR).
                \item \textbf{Line 2:} input is address 0 (main memory) and output is address 3 (MBR).
                \item \textbf{Line 3:} input is address 3 (MBR), output is address 4 (AC), and the operation happening inside the ALU (i.e., what happens to the AC) is addition.
            \end{itemize}
        \item \textbf{Timing diagram}: This diagram is called a timing diagram ($P_{4}$ is not shown). You can see which signals, including the timing signals, are high at each clock tick. This instruction has 4 clock ticks, $C_{0} - C_{3}$.
            \begin{align*}
                P_{0}\; P_{1}\; P_{2}\; P_{3}\; T_{0}\; \text{ are high at } C_{0}\; \\
                P_{0}\; P_{2}\; T_{1}\; \text{ are high at } C_{1}\; \\
                A_{0}\; P_{0}\; P_{1}\; P_{5}\; T_{2}\; \text{ are high at } C_{2}\; \\
                C_{r}\; T_{3}\; \text{ are high at } C_{3}\;
            .\end{align*}
        \item \textbf{More on the control unit}:
            The RTL and the control signals are the same whether the machine is hardwired or microprogrammed. 
            \bigbreak \noindent 
            In hardwired control, the bit pattern of an instruction feeds directly into the combinational logic of the control unit.
            \bigbreak \noindent 
            \fig{.7}{./figures/1031.png}
        \item \textbf{Control unit circuit (only shows the ADD instruction portion)}
            \bigbreak \noindent 
            \fig{.6}{./figures/1032.png}
        \item \textbf{ROM}:
            \begin{itemize}
                \item A non-volatile memory chip — it keeps its contents even when power is off.
                \item Stores fixed data or instructions that rarely (or never) change.
                \item The CPU can read from it but normally can’t write to it.
                \item Used to hold essential low-level code like a bootloader or microprogram for the control unit.
            \end{itemize}
            Think of ROM as permanent storage for instructions the hardware needs to start or operate.
        \item \textbf{Firmware}:
            \begin{itemize}
                \item The software stored in ROM (or Flash) that controls the hardware.
                \item It’s the middle layer between hardware and higher-level software (OS, apps).
                \item Examples: BIOS/UEFI in PCs, router firmware, or the microcode inside a CPU.
            \end{itemize}
        \item \textbf{PROM (Programmable read only memory)}:
            \begin{itemize}
                \item Stands for Programmable ROM.
                \item It’s manufactured blank, and you can program it once using a special device called a PROM programmer (it literally burns tiny fuses inside).
                \item After that, it’s permanent — can only be read, not erased or re-written.
            \end{itemize}
            \begin{align*}
                \text{PROM } = \text{ one-time programmable ROM.}
            .\end{align*}
        \item \textbf{Intro to microprogramming}: Now we switch to looking at microprogramming. In microprogrammed control, execution of microcode instructions produces control signal changes. The microprogram converts each microcode instruction into control signals.
            \bigbreak \noindent 
            The microprogram is stored in firmware, which is a small piece of read-only memory also called the control store. One microcode instruction is retrieved during each clock cycle.
            \bigbreak \noindent 
            in a microprogrammed CPU, the microprogram replaces (or substitutes for) the hardwired control circuit. However, it doesn’t remove the control unit entirely — it changes its implementation.
            \bigbreak \noindent 
            So instead of a big network of AND/OR gates and flip-flops generating control signals directly, you have:
            \begin{itemize}
                \item A small, special-purpose memory (the control store) containing microinstructions.
                \item A microsequencer that fetches, decodes, and executes those microinstructions — just like the CPU itself fetches and executes regular instructions.
            \end{itemize}
            \begin{center}
                \begin{tabular}{p{6cm}|p{6cm}}
                    Hardwired Control	&Microprogrammed Control \\
                    \hline \\[0.01cm]
                    Uses combinational logic (gates, decoders, timing signals) to generate control signals directly.	&Uses microinstructions stored in memory to specify which control signals go high each cycle. \\&\\
                    Faster, but less flexible — to change instruction behavior, you must redesign the logic.	&Slower, but more flexible — you can change or extend the instruction set by modifying the microcode. \\&\\
                    Example: MARIE’s control unit circuit with AND/OR gates.	&Example: A microprogram stored in ROM that defines each instruction’s control steps 
                \end{tabular}
            \end{center}
        \item \textbf{Generic microprogrammed control unit}:
            \bigbreak \noindent 
            \fig{.6}{./figures/1033.png}
        \item \textbf{Microinstruction format}: If MARIE were microprogrammed, the microinstruction format might look like this:
            \bigbreak \noindent 
            \fig{.6}{./figures/1034.png}
            \bigbreak \noindent 
            We see that a microinstruction is 18 bits long. 
            \begin{itemize}
                \item \textbf{Opcodes}: 5 bits
                \item \textbf{Jump bit}: 1 bit
                \item \textbf{Dest}: 7 bits
            \end{itemize}
            Thus, addresses are 7 bits.
            \bigbreak \noindent 
            MicroOp1 and MicroOp2 contain binary codes for each instruction. Jump is a single bit indicating that the value in the Dest field is an address that should be placed in the microsequencer, indicating which microprogram instruction should be executed next
        \item \textbf{Marie micro opcodes}: The table below contains MARIE’s micro-opcodes along with their corresponding RTL:
            \bigbreak \noindent 
            \begin{tabular}{|c|l||c|l|}
                \hline
                \textbf{MicroOp Code} & \textbf{Microoperation} & \textbf{MicroOp Code} & \textbf{Microoperation} \\
                \hline
                00000 & NOP & 01100 & MBR $\leftarrow$ M[MAR] \\
                00001 & AC $\leftarrow 0$ & 01101 & OutREG $\leftarrow$ AC \\
                00010 & AC $\leftarrow$ AC $-$ MBR & 01110 & PC $\leftarrow$ IR[11–0] \\
                00011 & AC $\leftarrow$ AC + MBR & 01111 & PC $\leftarrow$ MBR \\
                00100 & AC $\leftarrow$ InREG & 10000 & PC $\leftarrow$ PC + 1 \\
                00101 & IR $\leftarrow$ M[MAR] & 10001 & If AC = 00 \\
                00110 & M[MAR] $\leftarrow$ MBR & 10010 & If AC $>$ 0 \\
                00111 & MAR $\leftarrow$ IR[11–0] & 10011 & If AC $<$ 0 \\
                01000 & MAR $\leftarrow$ MBR & 10100 & If IR[11–10] = 00 \\
                01001 & MAR $\leftarrow$ PC & 10101 & If IR[11–10] = 01 \\
                01010 & MAR $\leftarrow$ X & 10110 & If IR[11–10] = 10 \\
                01011 & MBR $\leftarrow$ AC & 10111 & If IR[15–12] = MicroOp2[4–1] \\
                11000 &MBR $\leftarrow$ PC &11001 &MBR $\leftarrow$ $X$ \\
                11010 &AC $\leftarrow$ 1 &11011 &PC $\leftarrow$ AC \\
                11100 &AC $\leftarrow$ MBR & &  \\
                \hline
            \end{tabular}
        \item \textbf{First lines of Marie's microprogram}:
            \bigbreak \noindent 
            \fig{.6}{./figures/1035.png}
            \bigbreak \noindent 
            The first four lines contain the fetch (first 3 lines) and decode (line 4) steps of the execution cycle. The remaining lines are the beginning of a jump table.
            \bigbreak \noindent 
            The jump table entries indicate that if the high order bits of the IR (i.e., the opcode) have a given value, the microprogram will jump to the address where the remaining microoperations for that opcode will be found.
            \bigbreak \noindent 
            For example, LOAD has opcode 1. Looking through the jump table, we can see that a LOAD instruction will cause the microcode processor to jump to 010 0111.
            \bigbreak \noindent 
            At that address, it will find the fetch operand step for that instruction (if required) followed by the execute step, i.e., the RTL in microcode format:
            \begin{align*}
                010\; 0111\; \text{MAR} &\leftarrow X \text{ NOP}\; 0\; 000\; 0000 \\
                010\; 1000\; \text{MBR} &\leftarrow M[\text{MAR}]\; \text{NOP}\; 0\; 000\; 0000 \\
                010\; 1001\; \text{AC} &\leftarrow \text{MBR}\; \text{NOP}\; 1\; 000\; 0000
            .\end{align*}
            The last step of the encoded RTL will have a 1 in the JUMP field with a DEST of 000 0000. That will cause the system to jump to 000 0000, i.e., the beginning of the fetch cycle for the next operation.
            \bigbreak \noindent 
            The chart above shows the microcode in RTL for convenience, but we could also show it in binary.
            \begin{align*}
                010\; 0111\; 01010\; 00000\; 0\; 000\; 0000\\
                010\; 1000\; 01100\; 00000\; 0\; 000\; 0000\\
                010\; 1001\; 11100\; 00000\; 1\; 000\; 0000
            .\end{align*}
        \item \textbf{Microprogramming}:
            A microprogrammed control unit works like a system-in-miniature. Microinstructions are fetched, decoded, and executed just like regular instructions. This extra level of interpretation makes microprogrammed control slower than hardwired, but this is not important since computers are getting faster every day.
            \bigbreak \noindent 
            The advantage of microprogrammed control is that only the microprogram needs to be changed if the instruction set changes. In fact, all modern machines are microcoded. We study hardwired machines first just to make it easier to understand the microcode architecture.


    \end{itemize}


    \pagebreak 
    \unsect{Android Development with Java and XML}
    \bigbreak \noindent 
    \subsection{The Basics}
    \begin{itemize}
        \item \textbf{Layout file for apps "look"}: Note that the XML layout file for our app's "look" is named activity\_main.xml
        \item \textbf{Creating a new project}: Click on the Empty Views Activity template to highlight it and click Next
            \begin{enumerate}
                \item Enter the name of the project
                \item Enter the package name as edu.niu.zid.projectname
                \item Set language to Java
                \item Choose API 22 ("Lollipop"; Android 5.1) for the minimum SDK
                \item Click finish
            \end{enumerate}
        \item \textbf{Directory structure}: please be sure that you have no additional layers than what you see here.
            \bigbreak \noindent 
            \fig{.5}{./figures/mi.png}
            \bigbreak \noindent 
            Notice at the highest level in the hierarchy of our project folders that there are two main "pieces" to the Android project:
            \begin{enumerate}
                \item app
                \item Gradle Scripts
            \end{enumerate}
            \bigbreak \noindent 
            Within the app, there are really three folders we are going to be concerned with for now:
            \begin{enumerate}
                \item manifests
                \item java
                \item res
            \end{enumerate}
        \item \textbf{Manifests}: The directory named manifests holds one xml file - so far. It is the AndroidManifest.xml file.
        \item \textbf{Java (directory):} The directory named java holds one Java source files - so far. It is the MainActivity.java file
        \item \textbf{Res (directory)}: The directory named res holds resource files, such as utility xml files for defining strings, themes, menus, layouts, dimensions, images, sounds, etc
        \item \textbf{MainActivity.java}: This is the logic/controller file. It's the Java class that defines what your app does when it runs. It extends AppCompatActivity (or another activity class), and inside it you set up event handling, lifecycle methods (like onCreate), and the code that reacts to user interactions. For example, if a button is clicked, the code for what happens lives here.
            \bigbreak \noindent 
            The simplest MainActivity.java file is as follows
            \bigbreak \noindent 
            \begin{javacode}
                package YOURPACKAGENAME

                import androidx.appcompat.app.AppCompatActivity;

                import android.os.Bundle;

                public class MainActivity extends AppCompatActivity {

                    @Override
                    protected void onCreate(Bundle savedInstanceState) {
                        super.onCreate(savedInstanceState);
                        setContentView(R.layout.activity_main);
                    }
                }           
            \end{javacode}
            \bigbreak \noindent 
            In this code, the most important import statement is the import for class android.appcompat.app.AppCompatActivity
            \bigbreak \noindent 
            MainActivity.java class extends, or inherits from, superclass AppCompatActivity class found in Android API package androidx.appcompat.app.AppCompatActivity
            \bigbreak \noindent 
            AppCompatActivity itself is a subclass itself of Activity.
            \bigbreak \noindent 
            An Activity provides the screen with which users can interact, in other words, the User Interface, or UI.
            \bigbreak \noindent 
            The MainActivity class is meant to be used as the Controller for every app we develop
            \bigbreak \noindent 
            Method onCreate() of an Android app is called automatically upon launch just like the main() method is called automatically by the Java Virtual Machine, or JVM, upon launch of a Java applications.
            \bigbreak \noindent 
            Inside method onCreate(), we create the initial View for the app, controlled by this Activity.
            \bigbreak \noindent 
            In method onCreate(), the superclass' onCreate() method is first called.
            \bigbreak \noindent 
            Then the View, or opening screen, for this Activity is set by calling method setContentView().
            \bigbreak \noindent 
            Method setContentView() is inherited from the AppCompatActivity , or Activity class.
            \bigbreak \noindent 
            Its API is 
            \bigbreak \noindent 
            \begin{javacode}
            void setContentView(int layoutResID)
            \end{javacode}
            \bigbreak \noindent 
            layoutResID is a resource found in the subfolder named layout in the folder named res.
            \bigbreak \noindent 
            This ID defaults to: \textit{R.layout.activity\_main.}
            \bigbreak \noindent 
            In MainActivity.java, we refer to and access activity\_main.xml using the syntax
            \bigbreak \noindent 
            \begin{javacode}
            R.layout.activity_main
            \end{javacode}
            \bigbreak \noindent 
            activity\_main is a static constant of the static final class layout defined inside the R.java class or file.
            \bigbreak \noindent 
            The R.java file was automatically created when we began our project
            \bigbreak \noindent 
            R stands for resources and "represents" the res directory.
        \item \textbf{activity\_main.xml}: This is the UI/layout file. It defines the visual structure of your screen using XML: buttons, text fields, images, layouts, etc. You don't put behavior here—only the arrangement and styling of the interface elements.
            \bigbreak \noindent 
            activity\_main is an xml file, a resource located in the layout subdirectory of the res directory
            \bigbreak \noindent 
            It was also automatically created when we began our project using the template we chose.
            \bigbreak \noindent 
            We can think of XML as a markup language similar to HTML, but with user-defined tags
        \item \textbf{Intro to design editor}: With the activity\_main.xml file focused, click on this small button near the upper right
            \bigbreak \noindent 
            \fig{1}{./figures/im2.png}
            \bigbreak \noindent 
            Note that this opens the Design editor in which we can drag and drop widgets into our app. More on this to come!
        \item \textbf{Intro to XML syntax}:
            In XML, an non-empty element uses the general syntax shown here
            \bigbreak \noindent 
            \begin{xmlcode}
            <tagName attribute1="value1">Element Content</tagName> 
            \end{xmlcode}
            \bigbreak \noindent 
            For example:
            \bigbreak \noindent 
            \begin{xmlcode}
            <color name="red">#FFFF0000</color>
            \end{xmlcode}
            which can be interpreted as there is an element color that has the name red and corresponds to the RGB value \#FFFF0000.
            \bigbreak \noindent 
            If an element has no content yet or will never have any content, we can use this syntax
            \bigbreak \noindent 
            \begin{xmlcode}
            <tagName attribute1="value1" attribute2="value2".../>
            \end{xmlcode}
        \item \textbf{XML Naming Rules}: XML elements must follow these naming rules:
            \begin{enumerate}
                \item Names can contain letters, numbers, and other characters.
                \item Names cannot start with a number or punctuation character.
                \item Names cannot start with the letters xml (or XML, or Xml, etc).
                \item Names cannot contain spaces.
            \end{enumerate}
        \item \textbf{XML Comments}: Documentation in XML is done with
            \bigbreak \noindent 
            \begin{xmlcode}
            <!-- comment text -->
            \end{xmlcode}
        \item \textbf{Simple activity\_main.xml Example, TextView and ConstraintLayout}: Let's consider a simple XML example
            \bigbreak \noindent 
            \begin{xmlcode}
                <?xml version="1.0" encoding="utf-8"?>
                <androidx.constraintlayout.widget.ConstraintLayout xmlns:android="http://schemas.android.com/apk/res/android"
                    xmlns:app="http://schemas.android.com/apk/res-auto"
                    xmlns:tools="http://schemas.android.com/tools"
                    android:layout_width="match_parent"
                    android:layout_height="match_parent"
                    tools:context=".MainActivity">

                    <TextView
                        android:layout_width="wrap_content"
                        android:layout_height="wrap_content"
                        android:text="@string/app_name"
                        app:layout_constraintBottom_toBottomOf="parent"
                        app:layout_constraintEnd_toEndOf="parent"
                        app:layout_constraintStart_toStartOf="parent"
                        app:layout_constraintTop_toTopOf="parent" />

                </androidx.constraintlayout.widget.ConstraintLayout>
            \end{xmlcode}
            \bigbreak \noindent 
            There are two elements in our current activity\_main.xml:
            \begin{enumerate}
                \item ConstraintLayout
                \item TextView
            \end{enumerate}
            \bigbreak \noindent 
            The TextView element is nested inside the ConstraintLayout element
            \bigbreak \noindent 
            A ConstraintLayout allows us to position and size elements in a flexible way.
            \bigbreak \noindent 
            One way is relative positioning, or where an element is placed on the screen, or view, relative to another. 
            \bigbreak \noindent 
            The ConstraintLayout element has several attributes.
            \bigbreak \noindent 
            The android:layout\_width and android:layout\_height attributes define the size of the ConstraintLayout.
            \bigbreak \noindent 
            Their value match\_parent means that it will be as big as its parent element, which is the screen of the Android device in this case.
            \bigbreak \noindent 
            The TextView element is empty, or, in other words, has no content, and has, by default, seven attributes.
            \bigbreak \noindent 
            The first three are layout\_width, layout\_height and text.
            \bigbreak \noindent 
            The android:text attribute specifies the content, or the words, of the TextView to be shown on the screen of the Android device.
            \bigbreak \noindent 
            By default, the four attributes that follow layout\_width, layout\_height and text are for positioning of the TextView within the ConstraintLayout that will be discussed later
            \bigbreak \noindent 
            A TextView element is an instance of the TextView class, which puts a label on the screen.
            \bigbreak \noindent 
            The android:layout\_width and android:layout\_height attributes define the size of the TextView.
            \bigbreak \noindent 
            The value wrap\_content means that it will be as small as possible so that their contents (the text or words) fit inside it.
            \bigbreak \noindent 
            In other words, the element "wraps" around its content.
            \bigbreak \noindent 
            By default, the TextView appears in the middle of the screen
            \bigbreak \noindent 
            The four app:layout attributes specify the relative position of the TextView element's four sides
            \bigbreak \noindent 
            In this example, they are relative to the "parent", or the ConstraintLayout in which the Textview is nested
        \item \textbf{strings.xml}: is the resource file for text values in your app. Stores text like titles, labels, button names, or messages as key–value pairs.
            keep all text centralized for easy updates, consistency, and support for localization
            \bigbreak \noindent 
            You reference them in code (getString(R.string.welcome\_message)) or XML (@string/welcome\_message).
            \bigbreak \noindent 
            One string is defined in the strings.xml file by default
            \bigbreak \noindent 
            \begin{xmlcode}
                <resources>
                    <string name="app_name">Hello, Android!</string>
                <resources>
            \end{xmlcode}
            \bigbreak \noindent 
            The syntax for defining a string is
            \bigbreak \noindent 
            \begin{xmlcode}
                <string name="stringName">valueOfString</string>
            \end{xmlcode}
        \item \textbf{themes.xml}: defines the overall visual style of your Android app.
            \bigbreak \noindent 
            Central place to configure app-wide appearance, like colors, fonts, backgrounds, status bar style, and widget looks.
            \bigbreak \noindent 
            A theme is a collection of style attributes applied globally.
        \item \textbf{themes.xml example}:
            \bigbreak \noindent 
            \begin{xmlcode}
             <resources xmlns:tools="http://schemas.android.com/tools">
                 <!-- Base application theme. -->

                 <style name="Base.Theme.MyApplication" parent="Theme.Material3.DayNight.NoActionBar">
                     <!-- Customize your light theme here. -->
                     <!-- <item name="colorPrimary">@color/my_light_primary</item> -->
                 </style>

                 <style name="Theme.MyApplication" parent="Base.Theme.MyApplication" />
            </resources>
            \end{xmlcode}
            \bigbreak \noindent 
            It is used to define styles that the app uses.
            \bigbreak \noindent 
            We can modify a theme by adding an item element using this syntax
            \bigbreak \noindent 
            \begin{xmlcode}
            <item name="styleAttribute">valueOfItem</item>
            \end{xmlcode}
            \bigbreak \noindent 
            The name of the theme attribute that specifies the text size inside a TextView is android:textSize.
            \bigbreak \noindent 
            Let us change the default text size to 40 by adding the following code to the themes.xml file just above the line </style>
            \bigbreak \noindent 
            \begin{xmlcode}
            <item name="android:textSize">40sp</item>
            \end{xmlcode}
        \item \textbf{Removing and adding action bar (themes.xml)}: Also, this line removes the commonly shown Action Bar:
            \bigbreak \noindent 
            \begin{xmlcode}
            <style name="Base.Theme.HelloAndroid" parent="Theme.Material3.DayNight.NoActionBar">
            \end{xmlcode}
            \bigbreak \noindent 
            \textbf{Note:} The Empty Views Activity template does not provide a default Action Bar
            \bigbreak \noindent 
            If we change the line to
            \bigbreak \noindent 
            \begin{xmlcode}
            <style name="Base.Theme.HelloAndroid" parent="Theme.AppCompat.Light.DarkActionBar">
            \end{xmlcode}
            \bigbreak \noindent 
            And make two other changes, we will see the Action Bar with the name of our app displayed
        \item \textbf{colors.xml}: The colors.xml file in the res > values directory is another automatically generated XML file.
            \bigbreak \noindent 
            It is used to define colors we want to use in our apps
            \bigbreak \noindent 
            The syntax for defining a color is
            \bigbreak \noindent 
            \begin{xmlcode}
            <color name="colorName">valueOfColor</color>
            \end{xmlcode}
            \bigbreak \noindent 
            The color value is defined using hexadecimal, or base 16, notation
            \bigbreak \noindent 
            \#FFrrggbb (uses RGB color system) where:
            \begin{itemize}
                \item rr = amount of red in color
                \item gg = amount of green in color
                \item bb = amount of blue in color
            \end{itemize}
        \item \textbf{colors.xml example}: If we create the color in colors.xml
            \bigbreak \noindent 
            \begin{xmlcode}
            <color name="colorPrimary"> #FF3F51B5 </color>
            \end{xmlcode}
            \bigbreak \noindent 
            Then, in themes.xml, we can add or uncomment the line
            \bigbreak \noindent 
            \begin{xmlcode}
            <item name="colorPrimary">@color/colorPrimary</item>
            \end{xmlcode}
            \bigbreak \noindent 
            inside of the action bar tag
            \bigbreak \noindent 
            \begin{xmlcode}
                <style name="Base.Theme.Test3" parent="Theme.AppCompat.Light.DarkActionBar">
                    <item name="colorPrimary">@color/colorPrimary</item>
                </style> 
            \end{xmlcode}
            \bigbreak \noindent 
            Now we have a blue action bar.
        \item \textbf{AndroidManifest.xml}: The AndroidManifest.xml file is located in the manifests directory.
            \bigbreak \noindent 
            It specifies the resources that the app uses, such as activities, the file system, the Internet, and hardware resources.
            \bigbreak \noindent 
            Before a user downloads an app on Google Play, the user is notified about these details
        \item \textbf{Changing the launcher icon}: When your app is installed on a device, its icon and name appear with all other installed apps in the launcher
            \bigbreak \noindent 
            Users can use the icon to launch our app on their device.
            \bigbreak \noindent 
            We should always supply a launcher icon for our app
            \bigbreak \noindent 
            A launcher icon for a mobile device should be 48 x 48 dp.
            \bigbreak \noindent 
            Because various devices can have different screen densities, we can supply several launcher icons, one for each density
            \bigbreak \noindent 
            If we provide these different versions of our icon, we need to follow the 2/3/4/6/8 scaling ratios between the various densities from medium (2) to xxx-high (8)
            \bigbreak \noindent 
            If we supply only one icon - as we will do here - Android Studio will use that icon and expand its density as necessary.
            \bigbreak \noindent 
            If we plan to publish our app, we should provide a 512 x 512 launcher icon for display in Google's app store.
            \bigbreak \noindent 
            Right mouse click on mipmap in the project structure along the right side and choose New and then Image Asset.
            \bigbreak \noindent 
            To set the launch icon for the app to hi.png, we assign the
            String @mipmap/hi to the android:icon attribute and the
            String @mipmap/hi\_round to the android:roundIcon
            attribute of the application element in the
            AndroidManifest.xml file.
            \bigbreak \noindent 
            The @mipmap/hi expression defines the resource in the mipmap
            directory (of the res directory) whose name is hi (note that we do
            not include the extension) and the same for hi\_round
            \bigbreak \noindent 
            \begin{xmlcode}
            android:icon="@mipmap/hi" android:roundIcon="@mipmap/hi_round"
            \end{xmlcode}
        \item \textbf{Orientation}: Sometimes, we want the app to run in only one orientation
            \bigbreak \noindent 
            In other words, we do not want the app to rotate when the user rotates his or her device.
            \bigbreak \noindent 
            Inside the activity element (in the manifest), we can add the
            android:screenOrientation attribute and specify either
            portrait or landscape as its value
            \bigbreak \noindent 
            For example, if we want our app to run in vertical orientation only, we add
            \bigbreak \noindent 
            \begin{xmlcode}
            android:screenOrientation="portrait"
            \end{xmlcode}
            \bigbreak \noindent 
            Note that we can control the behavior of the app on a per activity basis
        \item \textbf{Gradle build system}: Android Application Package, or APK, is the file format for distributing applications that run on the Android operating system. 
            \bigbreak \noindent 
            The file extension is .apk
            \bigbreak \noindent 
            To create an apk file, the project is compiled and its various parts are packaged into the apk file.
            \bigbreak \noindent 
            The apk file can be found in the project directory:
            \begin{center}
                \textit{projectName/app/build/outputs/apk}
            \end{center}
            \bigbreak \noindent 
            These .apk files are built using the gradle build system, which is integrated in the Android Studio environment.
            \bigbreak \noindent 
            When we start creating an app, the gradle build scripts are automatically created
            \bigbreak \noindent 
            They can be modified to build apps that require custom building
        \item \textbf{Debugging}: Just like in a regular Java program, we can send output to the console in addition to displaying data on the screen.
            \bigbreak \noindent 
             For this, we can use one of the static methods of the Log class.
             \bigbreak \noindent 
             The Log class is part of the android.util package
             \bigbreak \noindent 
             Selected methods of the Log class include d, e, i, v, w.
             \bigbreak \noindent 
             They all have the same parameter list and return type; for example, the API of d is
             \bigbreak \noindent 
             \begin{javacode}
             public static void d(String tag, String message)
             \end{javacode}
             \begin{itemize}
                 \item \textbf{Log.d(String tag, String msg):} Debug: Used for debugging messages. These are usually filtered out in release builds.
                 \item \textbf{Log.e(String tag, String msg):} Error: Used to report error conditions. This is the highest-severity logging method.
                 \item \textbf{Log.i(String tag, String msg):} Info: Used for general information messages that highlight the progress of the application.
                 \item \textbf{Log.v(String tag, String msg):} Verbose: Used for the most detailed log messages, often too much for normal use, but helpful during deep debugging.
                 \item \textbf{Log.w(String tag, String msg):} Warning: Used to report potential issues or unexpected states that aren't necessarily errors.
             \end{itemize}
             \bigbreak \noindent 
             \textit{tag} identifies the source of the message and can be associated with a "filter" (described in a few slides).
             \bigbreak \noindent 
             \textit{message} is the String to be output.
             \bigbreak \noindent 
             To run the app in debug mode, we click on the debug icon on the toolbar.
             \bigbreak \noindent 
             The app runs and stops at the first breakpoint.
             \bigbreak \noindent 
             The Debug tab will open in the panel at the bottom of the screen.
             \bigbreak \noindent 
             Here we will see some debugging information and tools.
             \bigbreak \noindent 
             Under Frames, we can see where in the code we are currently executing.
             \bigbreak \noindent 
             To resume the app, we click on the green Resume icon at the top left of the panel.
             \bigbreak \noindent 
             As we resume, stop at breakpoints, and resume the app a few times, the values of the various variables in our app are displayed under Variables.
             \bigbreak \noindent 
             Under Variables, we can check the values of the various variables
             \bigbreak \noindent 
             If the app is running on a device, we can still log output statements in Logcat
             \bigbreak \noindent 
             This is much faster than starting the emulator.
         \item \textbf{Logcat}: Output from logging statements show up in the Logcat
             \bigbreak \noindent 
             To open the Logcat, click on the Logcat tab at the bottom of the IDE.
             \bigbreak \noindent 
             We can filter the output listed in the Logcat by clicking in the dropdown in the upper right of the Logcat
             \bigbreak \noindent 
             Suppose we create a new filter named \textit{f\_mainactivity}, with a tag MainActivity.
             \bigbreak \noindent 
             Now that a filter has been created along with its tag, we can output messages to Logcat.
             \bigbreak \noindent 
             Add the following line to the onCreate method of MainActivity.java but after the call to super.onCreate
             \bigbreak \noindent 
             \begin{javacode}
                 Log.w("MainActivity", "Inside onCreate!");
             \end{javacode}
             \bigbreak \noindent 
             The output we will see in Logcat is: Inside onCreate!
             \bigbreak \noindent 
             Or another way to do it
             \bigbreak \noindent 
             \begin{javacode}
                 ...
                 public static String MA = "MainActivity";
                 ...
                 setContentView(R.layout.activity_main);
                 Log.w(MA, "View resource: " +
                 R.layout.activity_main);
                 ... 
             \end{javacode}
    \end{itemize}

    \pagebreak 
    \subsection{Tip Calculator with constraint layout}
    \begin{itemize}
        \item \textbf{Intro}: In building the app for this chapter, we will make use of the Android ConstraintLayout.
            \bigbreak \noindent 
            In this chapter, we will learn how to...
            \begin{enumerate}
                \item structure our code using the Model-View-Controller, or MVC, architecture,
                \item use a variety of UI components,
                \item define and use styles and themes and
                \item handle events
            \end{enumerate}
        \item \textbf{MVC}:
            \begin{enumerate}
                \item The \textbf{Model} represents the functionality of the app
                \item The \textbf{View} represents the user interface, or UI.
                \item The \textbf{Controller} represents the middleman between the Model and the View.
            \end{enumerate}
            The user interacts with the View, the Controller sends userentered data to the Model and asks for some calculations, then displays the results returned by the Model in the View.
            \bigbreak \noindent 
            The MVC design of the Tip Calculator App will consist of...
            \bigbreak \noindent 
            TipCalculator.java, or the TipCalculator class, the Model, written in Java.
            \bigbreak \noindent 
            activity\_main.xml, the View, written in XML.
            \bigbreak \noindent 
            MainActivity.java, or the MainActivity class, the Controller, also written in Java
            \bigbreak \noindent 
            In this app, we will keep the Model to the barest minimum of code.
            \bigbreak \noindent 
            It includes a bill, a tip percentage and methods to compute the tip and the total bill based on the bill and tip percentage amounts.
            \bigbreak \noindent 
            Like any Java program, the model should be platform independent.
            \bigbreak \noindent 
            It should not only be platform independent but also be coded in such a way that it could be used in an app other than one in Android.
            \bigbreak \noindent 
            The app's View reflects, or shows, the data items that are stored in the app's Model.
            \bigbreak \noindent 
            It includes UI components, also simply called components, elements or widgets, where the user will enter the bill amount and desired tip percentage
            \bigbreak \noindent 
            It is also where the calculated tip and total bill are eventually displayed
            \bigbreak \noindent 
            In our app, we will use activity\_main.xml to define our View.
            \bigbreak \noindent 
            It is the file in which we define the UI components that are displayed on the screen.
            \bigbreak \noindent 
            Note that we can define and manipulate UI components entirely by code using XML or we can use the click and drag method.
            \bigbreak \noindent 
            In this app, we are going to both write XML code and click and drag to edit the activity\_main.xml file.
            \bigbreak \noindent 
            We use UI components to...
            \bigbreak \noindent 
            \begin{enumerate}
                \item display data,
                \item capture user input, or allow for user data entry and
                \item allow the user to interact with the components and trigger some action such as a call to a method to do something
            \end{enumerate}
            \bigbreak \noindent 
            In this app, the call will be to calculate the tip amount and the total bill using the bill amount and desired tip percentage entered.
        \item \textbf{Some components}
            \begin{center}
                \begin{center}
                    \begin{tabular}{p{4cm}|p{4cm}}
                        Component &Class Representing It \\
                        \hline
                        Panel &View \\
                        Keyboard &KeyBoardView \\
                        Label& TextView \\
                        Text Field &EditText \\
                        Button &Button \\
                        Radio Button &RadioButton \\
                        Checkbox &CheckBox \\
                        2-State Button &ToggleButton \\
                        On-Off Switch &Switch 
                    \end{tabular}
                \end{center}
            \end{center}
            Programmatically speaking, the View class is the root, or lowest level, UI component.
            \bigbreak \noindent 
            It occupies the rectangular area of the screen, it can be "drawn" on and can respond to events such as pushing a button on the screen.
            \bigbreak \noindent 
            The other UI components inherit from View, either directly or indirectly
            \bigbreak \noindent 
            View is in the package android.view
            \bigbreak \noindent 
            Most UI components are in the android.widget package
        \item \textbf{ConstraintLayout width and height}:
            The default ConstraintLayout has the following two attributes and values:
            \bigbreak \noindent 
            \begin{javacode}
                android:layout_width="match_parent"
                android:layout_height="match_parent"
            \end{javacode}
            \bigbreak \noindent 
            we could specify absolute values:
            \bigbreak \noindent 
            \begin{javacode}
                android:layout_width="200dp"
                android:layout_height="50dp"
            \end{javacode}
            \bigbreak \noindent 
            But, as the app will run on many different devices, it is always best to specify relative positioning like "match\_parent" instead of specifying absolute values.
        \item \textbf{Layout manager classes}:
            With the Empty Views Activity template, the default layout manager used in activity\_main.xml is ConstraintLayout.
        \item \textbf{ConstraintLayout margins}: Let us add some margins for the ConstraintLayout.
            \bigbreak \noindent 
            Right mouse click on values in the res folder of the project
            \bigbreak \noindent 
            Click New and then choose Values Resource File.
            \bigbreak \noindent 
            In the New Resource File dialog that opens, enter dimens for the File name and click OK.
            \bigbreak \noindent 
            Items in dimens.xml use the syntax
            \bigbreak \noindent 
            \begin{xmlcode}
            <dimen name=""> </dimen>
            \end{xmlcode}
            \bigbreak \noindent 
            For example,
            \bigbreak \noindent 
            \begin{xmlcode}
                <dimen name="activity_horizontal_margin">16dp</dimen>
            \end{xmlcode}
    \end{itemize}


    \pagebreak 
    \subsection{Reference}
    \bigbreak \noindent 
    \subsubsection{Includes}
    \begin{itemize}
        \item androidx.appcompat.app.AppCompatActivity;
        \item android.os.Bundle;
        \item android.util.Log;
        \item android.view.View;
        \item android.view.Gravity;
        \item android.graphics.Color;
        \item android.widget.EditText;
        \item android.widget.TextView;
        \item java.text.NumberFormat;
        \item android.text.TextWatcher: 
        \item android.text.Editable
        \item java.lang.CharSequence
        \item android.graphics.Point;
        \item android.widget.Button;
        \item android.widget.GridLayout;
        \item androidx.constraintlayout.widget.ConstraintLayout;
        \item androidx.constraintlayout.widget.ConstraintSet;
        \item androidx.constraintlayout.widget.Guideline
        \item import androidx.constraintlayout.widget.Barrier;
        \item android.view.ViewGroup;
        \item android.content.Context;
        \item android.content.DialogInterface;
        \item androidx.appcompat.app.AlertDialog;
        \item android.graphics.Typeface;
        \item android.view.Display;
        \item android.widget.RelativeLayout
    \end{itemize}

    \pagebreak 
    \subsubsection{Important information}
    \begin{itemize}
        \item \textbf{AVD (Android virtual device)}: It's basically an emulated device configuration that runs inside the Android Emulator. An AVD defines things like:
            \begin{itemize}
                \item \textbf{Device model}: (screen size, resolution, density, RAM, etc.)
                \item \textbf{System image}: (Android version, API level, Google Play support, etc.)
                \item \textbf{Hardware features}: (camera, GPS, sensors, etc.)
            \end{itemize}
            So when you launch an emulator in Android Studio, you're really starting up an AVD that behaves like a specific Android phone or tablet.
        \item \textbf{Screen Density}: screen density means how many pixels are packed into a physical area of the screen, usually measured as dots per inch (dpi).
            \bigbreak \noindent 
            A screen with high density has many pixels in a small space, making things look sharper but also smaller if not scaled.
            \bigbreak \noindent 
            A screen with low density has fewer pixels in the same area, so things look larger but less sharp.
            \bigbreak \noindent 
            Android groups devices into "density buckets" so developers don't have to manually calculate for every possible screen:
            \bigbreak \noindent 
            \begin{itemize}
                \item \textbf{ldpi (low)}: $\sim$120 dpi
                \item \textbf{mdpi (medium)}: $\sim$160 dpi (baseline)
                \item \textbf{hdpi (high)}: $\sim$240 dpi
                \item \textbf{xhdpi (extra-high)}: $\sim$320 dpi
                \item \textbf{xxhdpi (extra extra high)}: $\sim$480 dpi
                \item \textbf{xxxhdpi (extra extra extra high)}: $\sim$640 dpi
            \end{itemize}
            \bigbreak \noindent 
            Android uses this formula under the hood:
            \begin{align*}
                px = \; dp \times \frac{dpi}{160}
            \end{align*}
            Where
            \begin{itemize}
                \item dp = your value in density-independent pixels (e.g., 20)
                \item dpi = the device's actual screen density in dots-per-inch
                \item 160 = the baseline density (mdpi)
            \end{itemize}
            \bigbreak \noindent 
            Even though the number of pixels changes, the physical size (in inches or mm) stays about the same. That's because on a denser screen, pixels are smaller, so Android uses more of them to maintain the same real-world size.
            \bigbreak \noindent 
            So your 20dp button will look like the same size button whether it's on a cheap low-res phone or a modern super high-res one.
        \item \textbf{Layout Params}: Generally, a layout class like ConstraintLayout or RelativeLayout contains a static inner class that contains XML attributes that allow us to arrange the components within the layout. This static inner class is often named \textbf{LayoutParams}
            \bigbreak \noindent 
            Every child view inside a ViewGroup needs layout parameters (LayoutParams).
            \bigbreak \noindent 
            These tell the parent how big the child should be and how it should be positioned.
            \bigbreak \noindent 
            Different ViewGroups define their own rules:
            \begin{itemize}
                
                \item \textbf{LinearLayout.LayoutParams}: weight, gravity
                \item \textbf{FrameLayout.LayoutParams}: gravity
                \item \textbf{ConstraintLayout.LayoutParams}: constraints like topToBottom, leftToLeft, etc.
            \end{itemize}
        \item \textbf{Positioning components inside ConstraintLayout}: We will use the attributes of ConstraintLayout.LayoutParams to position the XML elements on the screen
        \item \textbf{Ids}: The android:id attribute allows us to give an id to an XML element
            \bigbreak \noindent 
            The syntax for assigning an id to an XML element is
            \bigbreak \noindent 
            \begin{xmlcode}
            android:id = "@+id/idValue"
            \end{xmlcode}
        \item \textbf{Default colors (defined in @android:color)}:
            \begin{itemize}
                \item \textbf{@android:color/black}: \#FF000000
                \item \textbf{@android:color/darker\_gray}: \#FF444444
                \item \textbf{@android:color/dim\_gray}: \#FF696969
                \item \textbf{@android:color/gray}: \#FF888888
                \item \textbf{@android:color/light\_gray}: \#FFCCCCCC
                \item \textbf{@android:color/white}: \#FFFFFFFF
                \item \textbf{@android:color/red}: \#FFFF0000
                \item \textbf{@android:color/green}: \#FF00FF00
                \item \textbf{@android:color/blue}: \#FF0000FF
                \item \textbf{@android:color/yellow}: \#FFFFFF00
                \item \textbf{@android:color/cyan}: \#FF00FFFF
                \item \textbf{@android:color/magenta}: \#FFFF00FF
                \item \textbf{@android:color/transparent}: \#00000000 (fully transparent)
            \end{itemize}
        \item \textbf{ConstraintLayout Enable Autoconnection to Parent}: When you enable it, every new view you drag into the ConstraintLayout will automatically get constraints to its closest edges of the parent (top, bottom, start, end).
        \item \textbf{Styles vs Themes}: A style relates to a UI component or a View. A theme relates to an activity; it can even relate to the whole app.
            \bigbreak \noindent 
            We can also "theme" an app with a style by editing the AndroidManifest.xml file, changing the android:theme attribute of its application element using this syntax:
            \bigbreak \noindent 
            \begin{xmlcode}
                android:theme="@style/nameOfStyle"
            \end{xmlcode}
        \item \textbf{findViewById}: findViewById is a fundamental method in Android development used to obtain a reference to a View object defined in your XML layout files. This method allows you to interact with UI elements programmatically, such as setting text, handling clicks, or changing visibility.
            \bigbreak \noindent 
            For example,
            \bigbreak \noindent
            \begin{javacode}
            EditText billEditText = (EditText)findViewById(R.id.amount_bill);
            \end{javacode}
            \bigbreak \noindent 
            findViewById returns the View that is part of the layout xml file that was inflated in method onCreate() of the Activity class.
            \bigbreak \noindent 
            If we expect to retrieve a TextView and we want to assign the View retrieved to a TextView, we need to cast the View returned by method findViewById() to a TextView:
        \item \textbf{View Event handling with XML (only onClick)}: To set up a click event on a View, we use the android:onClick attribute of the View and assign to it a method using the form:
            \bigbreak \noindent 
            \begin{xmlcode}
            android:onClick="methodName"
            \end{xmlcode}
            \bigbreak \noindent 
            The event will be handled inside that method, which must have the following API
            \bigbreak \noindent 
            \begin{javacode}
            public void methodName(View v) 
            \end{javacode}
            \bigbreak \noindent 
            $v$ is the View where the event happened.
            \bigbreak \noindent 
            When the user clicks on the button, we execute inside calculate, and its View parameter is the Button; if we have the following statement inside calculate
            \bigbreak \noindent 
            \begin{javacode}
            Log.w("MainActivity", "v = " + v);
            \end{javacode}
            \bigbreak \noindent 
            Inside LogCat, we have something like:
            \bigbreak \noindent 
            \begin{javacode}
            v=android.widget.Button@425a2e60
            \end{javacode}
            \bigbreak \noindent 
            The value above identifies the Button
        \item \textbf{IME}: It’s the software keyboard or other input system (like voice typing, handwriting, predictive text) that allows users to enter text into your app.
            \bigbreak \noindent 
            Every EditText communicates with the IME through the Input Method Framework (IMF) in Android.
            \bigbreak \noindent 
            When you tap into an EditText, the IME pops up (usually the on-screen keyboard).
        \item \textbf{Get size of screen}: 
            \bigbreak \noindent 
            \begin{javacode}
                import android.graphics.Point;

                Point size = new Point();
                getWindowManager().getDefaultDisplay().getSize(size);
            \end{javacode}
        \item \textbf{Create grid layout in java (controller)}:
            \bigbreak \noindent 
            \begin{javacode}
                GridLayout gridLayout = new GridLayout(this);
                gridLayout.setRowCount(int m);
                gridLayout.setColumnCount(int n);
            \end{javacode}
        \item \textbf{Adding views to GridLayout}:
            \bigbreak \noindent 
            \begin{javacode}
                gridLayout.addView(view, w, h);

            \end{javacode}
        \item \textbf{setContentView}: the method setContentView(...) is used inside an Activity to specify which layout file (XML) should be used as the UI for that screen.
            \bigbreak \noindent 
            Calling \texttt{setContentView(R.layout.some\_layout)} tells the activity:
            \begin{center}
                "Use the layout resource some\_layout.xml from the \textit{res/layout} folder as the UI for this screen."
            \end{center}
            \bigbreak \noindent 
            If you’re using programmatic UI (creating Views in Java instead of XML), you can pass a View object directly,
        \item \textbf{ViewGroup}:
            In Android, everything you see on screen is either a View or a ViewGroup.
            \bigbreak \noindent 
            A View is a basic UI element — e.g. Button, TextView, EditText, ImageView.
            \bigbreak \noindent 
            A ViewGroup is a container that holds other Views (and even other ViewGroups). 
            \bigbreak \noindent 
            So, a ViewGroup is essentially a layout manager. It defines how child views are positioned, sized, and arranged inside it.
            \begin{itemize}
                \item \textbf{LinearLayout}: places children in a row or column.
                \item \textbf{RelativeLayout (older, mostly replaced by ConstraintLayout)}: places children relative to each other.
                \item \textbf{ConstraintLayout}: a flexible, modern layout system.
                \item \textbf{FrameLayout}: stacks children on top of each other.
                \item \textbf{RecyclerView}: a powerful scrolling container for lists/grids.
            \end{itemize}
            \textbf{Note:} A ViewGroup is a subclass of View.
        \item \textbf{Giving id to view in java}: We can use
            \bigbreak \noindent 
            \begin{javacode}
            View.generateViewId()
            \end{javacode}
            as an argument to \texttt{view.setId()}. The method \texttt{generateViewId()} is a static method in the View class. It creates a unique integer ID at runtime that’s guaranteed not to collide with other view IDs.
        \item \textbf{The finish() method}: finish() is a method of the Activity class. It tells Android:
            \begin{quote}
             "I’m done with this Activity, please close it and remove it from the back stack."
            \end{quote}
            So if you call finish() inside an Activity, that activity will end and control will go back to the previous one (or exit the app if it was the only activity).
        \item \textbf{Converting px to dp in java}: In Android, px (pixels) and dp (density-independent pixels) are not the same thing. You almost always want to work in dp because it scales properly across different screen densities.
            \bigbreak \noindent 
            \begin{javacode}
                float dp = px / context.getResources().getDisplayMetrics().density; // px to dp

                float px = dp * context.getResources().getDisplayMetrics().density; // dp to px
            \end{javacode}
            \bigbreak \noindent 
            when a Java function in Android Studio takes an int representing a dimension (like width, height, margin, radius, stroke width, etc.), it is almost always in raw pixels (px) — not dp.
        \item \textbf{xmlns:android="http://schemas.android.com/apk/res/android"}: One of the most important pieces of XML in Android layouts.
            \bigbreak \noindent 
            It tells the XML parser: "Whenever you see an attribute that starts with android:, it belongs to the Android system’s XML namespace, located at this URI."
            \bigbreak \noindent 
            You must include this in:
            \begin{itemize}
                \item Any layout XML (e.g. activity\_main.xml)
                \item Any drawable XML, menu XML, or style XML that uses android: attributes
            \end{itemize}
            Basically, any file where you use attributes like android:layout\_width, android:padding, android:text, etc.
            \bigbreak \noindent 
            you only need to declare that line once, and it always goes on the root element of your XML layout (like <ConstraintLayout>, <RelativeLayout>, <LinearLayout>, etc.).

    \end{itemize}

    \pagebreak 
    \subsubsection{Units}
    \begin{itemize}
        \item \textbf{dp}: stands for density-independent pixels, or "dips" for short.
            \bigbreak \noindent 
            The most common unit for layout dimensions (width, height, margins, padding). Scales with screen density so UI looks consistent across devices.
        \item \textbf{sp}: stands for scalable pixels. Maybe we can call them "sips"?
            \bigbreak \noindent 
            Primarily for text size. Like dp, but also respects the user's font size settings (accessibility).
        \item \textbf{px (pixels)}: Actual screen pixels. Avoid using directly because it doesn't scale across different densities.
        \item \textbf{pt (points):} 1/72 of an inch. Rarely used, but supported.
        \item \textbf{in (inches)}: Physical size in inches (based on the screen's dpi).
        \item \textbf{mm (millimeter)}: Physical size in millimeters.
    \end{itemize}

    \pagebreak 
    \subsubsection{Files}
    \begin{itemize}
        \item \textbf{AndroidManifest.xml}: The app's blueprint. Declares package name, permissions, minimum SDK, app components (activities, services, etc.), and the launcher activity.
        \item \textbf{MainActivity.java}: The main Java class that runs when the app starts. Controls app logic and connects the UI (XML layouts) with backend code.
        \item \textbf{activity\_main.xml}: The layout file for MainActivity. Defines the UI elements (buttons, text, etc.) using XML.
        \item \textbf{colors.xml}: Central place for defining reusable color values. Used for themes, buttons, backgrounds, etc
        \item \textbf{strings.xml}: Stores all text strings (app name, labels, messages). Helps with reusability and localization (multi-language support).
        \item \textbf{dimens.xml}:  Defines dimensions like margins, padding, font sizes (e.g., 16dp). Ensures consistent spacing and scaling.
        \item \textbf{themes.xml}: Holds styles and themes (colors, fonts, appearances) applied app-wide or to individual components.
    \end{itemize}

    \pagebreak 
    \subsubsection{XML tags}
    \begin{itemize}
        \item \textbf{XML Declaration}: Android's resource compiler can usually parse XML without it. But it is strongly recommended to include it for consistency and to avoid encoding issues 
            \bigbreak \noindent 
            \begin{xmlcode}
                <?xml version="1.0" encoding="utf-8"?>
            \end{xmlcode}
        \item \textbf{Resources}: The root element of an XML resource file in the res/values/ directory (like strings.xml, colors.xml, styles.xml, etc.).
            \bigbreak \noindent 
            \begin{xmlcode}
            <resources>
                ...
            </resources>
            \end{xmlcode}
        \item \textbf{String}: 
            \bigbreak \noindent 
            \begin{xmlcode}
            <string name=""> ... </string>
            \end{xmlcode}
        \item \textbf{Color}:
            \bigbreak \noindent 
            \begin{xmlcode}
            <color name=""> hex </color>
            \end{xmlcode}
        \item \textbf{Dimen}:
            \bigbreak \noindent 
            \begin{xmlcode}
            <dimen name=""> ... </dimen>
            \end{xmlcode}
        \item \textbf{Manifest}
            \bigbreak \noindent 
            \begin{xmlcode}
            <manifest>
                ...
            </manifest>
            \end{xmlcode}
        \item \textbf{Style}: used to define styles and themes for your Android app
            \bigbreak \noindent 
            \begin{xmlcode}
            <style name="" parent="">
                <item name=""> ... </item>
            </style>
            \end{xmlcode}
    \end{itemize}

    \pagebreak 
    \subsubsection{Components}
    \begin{itemize}
        \item \textbf{ConstraintLayout}: In Android Studio, ConstraintLayout is a powerful and flexible layout manager used to design UIs. It's often the default choice in modern Android projects. 
            \bigbreak \noindent 
            It is a \textbf{ViewGroup} (container) that positions and sizes its child views based on constraints you define.
            \begin{itemize}
                \item \textbf{Constraints:} Each view needs at least one horizontal and one vertical constraint (e.g., align left to parent, center in parent, align top to another view).
                \item \textbf{No deep nesting:} Unlike LinearLayout or RelativeLayout, you can achieve complex designs without nesting multiple layouts, which improves performance.
                \item \textbf{Flexible positioning:} You can center, chain, bias (percent-based positioning), or even set aspect ratios.
            \end{itemize}
            \bigbreak \noindent 
            \begin{xmlcode}
            <androidx.constraintlayout.widget.ConstraintLayout 
            xmlns:android="http://schemas.android.com/apk/res/android" 
            xmlns:app="http://schemas.android.com/apk/res-auto"
            xmlns:tools="http://schemas.android.com/tools"
            ...
            tools:context=".MainActivity">
                ...
            </androidx.constraintlayout.widget.ConstraintLayout>
            \end{xmlcode}
            \bigbreak \noindent 
            \textbf{Note:} xmlns stands for \textbf{XML Namespace}. 
            \bigbreak \noindent 
            \texttt{tools:context=".MainActivity"} tells the Layout Editor which Activity will load this layout. It's a design-time hint for Android Studio (not used at runtime).
        \item \textbf{TextView}: is a basic Android UI widget used to display text to the user. It's read-only by default (unlike EditText, which allows input). It can show plain text, styled text, or even links.
            \bigbreak \noindent 
            \begin{xmlcode}
                <TextView ...>
                    ...
                </TextView
            \end{xmlcode}
        \item \textbf{EditText}: Subclass of TextView that allows the user to enter and edit text. It's the standard widget for text input in Android (like forms, search boxes, chat inputs)
            \bigbreak \noindent 
            \begin{xmlcode}
                <EditText ...>
                    ...
                </EditText>
            \end{xmlcode}

    \end{itemize}

    \pagebreak 
    \subsubsection{Attributes}
    \begin{itemize}
        \item \textbf{ConstraintLayout}:
            \begin{itemize}
                \item \textbf{android:layout\_width}: Defines the width of the view inside its parent.
                \item \textbf{android:layout\_height}: Same as above, but for height.
                    \begin{itemize}
                        \item \textbf{wrap\_content}: Size just big enough for its content.
                        \item \textbf{match\_parent}: Fill the entire parent width.
                        \item \textbf{Specific size}: Like 20dp
                    \end{itemize}
                \item \textbf{android:paddingBottom}: space inside the view, between its boundary and its content (like text or an image).
                \item \textbf{android:paddingLeft}: space inside the view, between its boundary and its content (like text or an image).
                \item \textbf{android:paddingRight}: space inside the view, between its boundary and its content (like text or an image).
                \item \textbf{android:paddingTop}: space inside the view, between its boundary and its content (like text or an image).
            \end{itemize}
            The following layout params are placed as attributes in the components nested inside ConstraintLayouts
            \begin{itemize}
                \item \textbf{app:layout\_constraintStart\_toStartOf="targetId"}: Aligns the start edge (left in LTR layouts, right in RTL) of this view to the start edge of the targetId.
                \item \textbf{app:layout\_constraintTop\_toTopOf="targetId"}: Aligns the top edge of this view to the top edge of the targetId.
                \item \textbf{app:layout\_constraintBottom\_toBottomOf="targetId"}: Aligns the bottom edge of this view to the bottom edge of the targetId.
                \item \textbf{app:layout\_constraintLeft\_toRightOf="targetId"}: Places the left edge of this view aligned to the right edge of the targetId.
                \item \textbf{app:layout\_constraintRight\_toRightOf="targetId"}: Aligns the right edge of this view to the right edge of the targetId.
                \item \textbf{app:layout\_constraintLeft\_toLeftOf="targetId"}: Aligns the left edge of this view to the left edge of the targetId.
                \item \textbf{app:layout\_constraintHorizontal\_bias (no units, value 0-1.0)}:
                \item \textbf{app:layout\_constraintVertical\_bias (no units, value 0-1.0)}:
            \end{itemize}
            \textbf{Note:} Instead of \textit{targetId}, we can specify \textit{parent}
            \bigbreak \noindent 
            Bias only works if you constrain both sides (e.g. start and end, or top and bottom). If there's only one constraint, the bias has no effect.
        \item \textbf{RelativeLayout}
            \begin{itemize}
                \item \textbf{android:layout\_alignParentTop="true"}:	Stick to top edge
                \item \textbf{android:layout\_alignParentBottom="true"}:	Stick to bottom edge
                \item \textbf{android:layout\_alignParentStart="true"}:	Stick to left (or start) edge
                \item \textbf{android:layout\_alignParentEnd="true"}:	Stick to right (or end) edge
                \item \textbf{android:layout\_centerInParent="true"}:	Center both vertically and horizontally
                \item \textbf{android:layout\_centerHorizontal="true"}:	Center horizontally only
                \item \textbf{android:layout\_centerVertical="true"}:	Center vertically only
                \item \textbf{android:layout\_above="@id/viewId"}:	Place above another view
                \item \textbf{android:layout\_below="@id/viewId"}:	Place below another view
                \item \textbf{android:layout\_toStartOf="@id/viewId"}:	Place to the left of another view
                \item \textbf{android:layout\_toEndOf="@id/viewId"}:	Place to the right of another view
                \item \textbf{android:layout\_alignStart="@id/viewId"}:	Align left edges
                \item \textbf{android:layout\_alignEnd="@id/viewId"}:	Align right edges
                \item \textbf{android:layout\_alignTop="@id/viewId"}:	Align top edges
                \item \textbf{android:layout\_alignBottom="@id/viewId"}:	Align bottom edges
                \item \textbf{android:layout\_marginStart / android:layout\_marginEnd}:	Logical left/right margins (RTL aware)
                \item \textbf{android:layout\_marginLeft / android:layout\_marginRight}:	Physical left/right margins (legacy)
                \item \textbf{android:layout\_marginTop / android:layout\_marginBottom}:	Vertical margins
                \item \textbf{android:padding*}	Padding inside the view (applies to content, not position)
                \item \textbf{android:layout\_alignBaseline="@id/viewId"}:	Align text baselines of two views
                \item \textbf{android:layout\_alignWithParentIfMissing="true"}:	If referenced ID is missing, align with parent instead (rarely used)
            \end{itemize}
        \item \textbf{TextView}: 
            \begin{itemize}
                \item \textbf{android:text}: the actual text string (not just visual, it's the content).
                \item \textbf{android:hint}: placeholder shown when empty.
                \item \textbf{android:ellipsize}: controls truncation (end, marquee, etc.).
                \item \textbf{android:scrollHorizontally}: enables horizontal scrolling.
                \item \textbf{android:marqueeRepeatLimit}: how many times marquee scroll repeats.
                \item \textbf{android:inputType}: defines the type of expected text (password, email, number, etc.).
                \item \textbf{android:digits}: restricts input to specific characters.
                \item \textbf{android:editable}: (deprecated, use EditText).
                \item \textbf{android:ems}: sets width in units of "M" characters.
                \item \textbf{android:freezesText}: whether text is preserved on screen rotation.
                \item \textbf{android:phoneNumber}: (deprecated, use inputType="phone").
                \item \textbf{android:selectAllOnFocus}: selects all text when focused.
                \item \textbf{android:linksClickable}: whether links are clickable.
                \item \textbf{android:autoLink}: auto-detect links (web, email, phone).
                \item \textbf{android:focusable}: can this view take focus?
                \item \textbf{android:focusableInTouchMode}: can it take focus during touch mode?
                \item \textbf{android:longClickable}: whether it supports long-press actions.
                \item \textbf{android:cursorVisible}: whether the text cursor is shown.
                \item \textbf{android:inputMethod}: IME options (soft keyboard).
                \item \textbf{android:imeOptions}: extra options for keyboard (e.g., actionDone).
                \item \textbf{android:imeActionId}: action ID for IME.
                \item \textbf{android:imeActionLabel}: label for IME action key.
                \item \textbf{android:contentDescription}: for accessibility services (screen readers).
                \item \textbf{android:autoLink}: auto-detect links.
                \item \textbf{android:linksClickable}: enable link clicks.
            \end{itemize}
        \item \textbf{EditText}: 
            \begin{itemize}
                \item \textbf{android:hint}:  A gray placeholder text shown when the field is empty.
                \item \textbf{android:inputType}:  Controls what kind of text can be entered and how the keyboard looks:
                \begin{itemize}
                    \item \textbf{text}: normal text
                    \item \textbf{textPassword}: hidden input (\bullet\bullet\bullet\bullet)
                    \item \textbf{number}: numeric keyboard
                    \item \textbf{numberDecimal}: real numbers
                    \item \textbf{phone}: phone keypad
                    \item \textbf{textEmailAddress}: email-optimized keyboard
                    \item \textbf{android:ems}: Sets the default width in terms of characters.
                \end{itemize}
                \item \textbf{android:maxLines / android:lines}: Control number of visible lines.
                \item \textbf{android:gravity}: Aligns the text inside the box.
                \item \textbf{android:drawableLeft / drawableRight}: Add icons inside the field.
                \item \textbf{android:textColor}: Sets the color of the text
                \item \textbf{android:textColorHint}: Sets the color of the hint text
            \end{itemize}
        \item \textbf{Button}:
            \begin{itemize}
                \item \textbf{android:clickable}: whether the button responds to clicks.
                \item \textbf{android:longClickable}: whether the button responds to long presses.
                \item \textbf{android:focusable}: can the button take focus.
                \item \textbf{android:focusableInTouchMode}: focusable via touch navigation.
                \item \textbf{android:soundEffectsEnabled}: enable/disable click sound.
                \item \textbf{android:hapticFeedbackEnabled}: enable/disable vibration feedback.
                \item \textbf{android:contentDescription}: spoken description for screen readers.
                \item \textbf{android:importantForAccessibility}: whether this button should be exposed to accessibility services.
                \item \textbf{android:labelFor}: associates this button as a label for another view.
                \item \textbf{android:enabled}: whether the button can be interacted with.
                \item \textbf{android:nextFocusUp / nextFocusDown / nextFocusLeft / nextFocusRight}: custom focus navigation.
                \item \textbf{android:checkable (for ToggleButton/MaterialButton)}: whether it can act like a checkbox.
                \item \textbf{android:checked (for toggleable buttons)}: initial checked state.
                \item \textbf{android:duplicateParentState}: inherits enabled/pressed/selected state from parent.
                \item \textbf{android:visibility}: visible, invisible, or gone.
                \item \textbf{android:keepScreenOn}: keep the screen on while this button is visible.
            \end{itemize}
    \end{itemize}

    \pagebreak 
    \subsubsection{Styles}
    \begin{itemize}
        \item \textbf{How to style}: We can use many attributes to specify how a UI component will look. For example, we can specify the background color of a component.
            \bigbreak \noindent 
            We can specify its text size and color and whether the text is centered or not. We can also specify the component's size and the padding inside it
            \bigbreak \noindent 
            We use styles and themes to organize our project better
            \bigbreak \noindent 
            Themes and styles enable us to separate the look and feel of a View from its content. This is similar to the concept of CSS (Cascading Style Sheets) in web design.
            \bigbreak \noindent 
            We can define more styles in the file named themes.xml in the res > values directory.
        \item \textbf{Define a style} The syntax for defining a style is 
            \bigbreak \noindent 
            \begin{xmlcode}
                <style name="nameOfStyle"
                    [parent="styleThisStyleInheritsFrom"]>
                    <item name="attributeName" parent="styleThisStyleInheritsFrom">attributeValue</item>
                    ...
                </style>
            \end{xmlcode}
            \bigbreak \noindent 
            The (optional) parent attribute allows us to create a hierarchy of styles, i.e., styles can inherit from other styles.
        \item \textbf{Example style}: For example, this style specifies width, height, text size, and padding
            \bigbreak \noindent 
            \begin{xmlcode}
                <style name="TextStyle"
                    parent="@android:style/TextAppearance">
                    <item name="android:layout_width">wrap_content</item>
                    <item name="android:layout_height">wrap_content</item>
                    <item name="android:textSize">32sp</item>
                    <item name="android:padding">10dp</item>
                </style>
            \end{xmlcode}
        \item \textbf{Apply a style}: We give the style attribute to a component. For example,
            \bigbreak \noindent 
            \begin{xmlcode}
            <button style="@style/styleName" ...> ... </button> 
            \end{xmlcode}
        \item \textbf{TextView Styles}:
            \begin{itemize}
                \item \textbf{android:layout\_width}
                \item \textbf{android:layout\_height}
                \item \textbf{android:layout\_margin}
                \item \textbf{android:layout\_gravity}
                \item \textbf{android:ellipsize}: how text is cut off (none, start, middle, end, marquee)
                \item \textbf{android:singleLine}: (deprecated, use maxLines="1")
                \item \textbf{android:textSize}: font size (e.g. 16sp)
                \item \textbf{android:textColor}: text color (e.g. @color/black)
                \item \textbf{android:textColorHint}: color of the hint text
                \item \textbf{android:textColorHighlight}: color of text selection highlight
                \item \textbf{android:textColorLink}: color of hyperlinks
                \item \textbf{android:textStyle}: normal, bold, italic
                \item \textbf{android:fontFamily}: font family (e.g. sans-serif, serif, or custom font from res/font/)
                \item \textbf{android:typeface}: older way of setting (normal, sans, serif, monospace)
                \item \textbf{android:lineSpacingExtra}: add extra space between lines
                \item \textbf{android:lineSpacingMultiplier}: scale spacing between lines
                \item \textbf{android:letterSpacing}: adjust space between characters
                \item \textbf{android:gravity}: how text is positioned inside its box (top, bottom, left, right, center)
                \item \textbf{android:textAlignment}: alignment relative to parent (gravity, center, viewStart, etc.)
                \item \textbf{android:includeFontPadding}: extra font padding (default true)
                \item \textbf{android:scrollHorizontally}: allow horizontal scroll if needed
                \item \textbf{android:ems}: width in "M" units
                \item \textbf{android:shadowColor}: color of text shadow
                \item \textbf{android:shadowDx, android:shadowDy}: shadow offset
                \item \textbf{android:shadowRadius}: shadow blur radius
            \end{itemize}
        \item \textbf{EditText}: Since EditText is a subclass of TextView, it inherits all of TextView's styling attributes
            \begin{itemize}
                \item \textbf{android:textCursorDrawable}: lets you set a custom drawable for the blinking cursor.
                \item \textbf{android:textSelectHandle}: base selection handle drawable.
                \item \textbf{android:textSelectHandleLeft}: left handle for text selection.
                \item \textbf{android:textSelectHandleRight}: right handle for text selection.
                \item \textbf{android:colorControlActivated}: (theme attr, but affects EditText focus underline in Material/AppCompat).
                \item \textbf{android:colorControlNormal}: normal underline/tint when unfocused.
                \item \textbf{android:colorControlHighlight}: ripple/highlight color when focused.
            \end{itemize}
        \item \textbf{Button}: Button is another subclass of TextView, so it inherits all of TextView’s styling attributes
            \begin{itemize}
                \item \textbf{android:text}: label text.
                \item \textbf{android:textSize}: text size (14sp, 18sp).
                \item \textbf{android:textColor}: text color.
                \item \textbf{android:textStyle}: normal, bold, italic.
                \item \textbf{android:fontFamily}: custom font (@font/roboto\_bold).
                \item \textbf{android:letterSpacing}: adjust spacing between characters.
                \item \textbf{android:lineSpacingExtra / android:lineSpacingMultiplier}: line spacing.
                \item \textbf{android:textAllCaps}: force all caps (default true on Material buttons).
                \item \textbf{android:ellipsize}: how text is cut off if too long.
                \item \textbf{android:background}: drawable for button background (e.g. custom shape).
                \item \textbf{android:foreground}: optional overlay (e.g. ripple).
                \item \textbf{android:insetLeft, android:insetRight, android:insetTop, android:insetBottom}: padding inside button background (mostly for legacy Button).
                \item \textbf{android:padding, android:paddingStart, android:paddingEnd}: space inside button.
                \item \textbf{android:layout\_width, android:layout\_height}: sizing.
                \item \textbf{android:minWidth, android:minHeight}: minimum size (Material buttons have built-in minimums).
                \item \textbf{android:backgroundTint}: tint for the button background.
                \item \textbf{android:backgroundTintMode}: blend mode for tint.
                \item \textbf{android:drawableTint}: tint icons/drawables inside button.
                \item \textbf{android:drawableTintMode}: blending for icon tint.
                \item \textbf{android:drawableStart / android:drawableLeft}: icon at start.
                \item \textbf{android:drawableEnd / android:drawableRight}: icon at end.
                \item \textbf{android:drawableTop, android:drawableBottom}: icons above/below text.
                \item \textbf{android:drawablePadding}: space between icon and text
                \item \textbf{android:elevation}: z-depth shadow (Material design).
                \item \textbf{android:translationZ}: raised elevation when pressed.
                \item \textbf{android:shadowColor, android:shadowDx, android:shadowDy, android:shadowRadius}: text shadow.
                \item \textbf{app:cornerRadius}: rounded corners.
                \item \textbf{app:strokeColor}: outline color.
                \item \textbf{app:strokeWidth}: outline width.
                \item \textbf{app:icon}: set an icon.
                \item \textbf{app:iconPadding}: space between icon and text.
                \item \textbf{app:iconGravity}: where the icon appears (start, end, textStart, textEnd).
                \item \textbf{app:iconTint}: tint icon color.
            \end{itemize}
    \end{itemize}

    \pagebreak 
    \subsubsection{Events}
    \begin{itemize}
        \item \textbf{View (XML)}
            \begin{itemize}
                \item \textbf{android:onClick}: name of a method in your Activity that gets called when the button is clicked
            \end{itemize}
    \end{itemize}

    \pagebreak 
    \subsubsection{Java event listeners}
    \begin{itemize}
        \item \textbf{Views:}
            \begin{itemize}
                \item \textbf{void setOnClickListener(OnClickListener l)}: Registers a listener to be invoked when the view is clicked (short tap).
                \item \textbf{void setOnLongClickListener(OnLongClickListener l)}: Registers a listener to be invoked when the view is long-pressed.
                \item \textbf{void setOnFocusChangeListener(OnFocusChangeListener l)}: Sets a listener that is called whenever the view gains or loses input focus.
                \item \textbf{void setOnTouchListener(OnTouchListener l)}: Sets a listener to receive touch events (e.g., finger down, move, lift) before they are processed by \texttt{onTouchEvent()}.
                \item \textbf{void setOnKeyListener(OnKeyListener l)}: Sets a listener to receive key events (e.g., hardware button presses) before they are passed to \texttt{onKeyDown()} or \texttt{onKeyUp()}.
                \item \textbf{boolean onKeyDown(int keyCode, KeyEvent event)}: Called when a hardware key is pressed down while the view has focus.
                \item \textbf{boolean onKeyUp(int keyCode, KeyEvent event)}: Called when a hardware key is released while the view has focus.
                \item \textbf{boolean onTouchEvent(MotionEvent event)}: Handles touch screen interaction with the view (default implementation supports clicks, scrolls, etc.).
                \item \textbf{boolean onGenericMotionEvent(MotionEvent event)}: Handles non-touch input events such as mouse, joystick, or stylus actions.
                \item \textbf{boolean onKeyPreIme(int keyCode, KeyEvent event)}: Called when a key event occurs before it reaches the input method (useful for intercepting Back button presses while an IME is visible).
                \item \textbf{boolean onTrackballEvent(MotionEvent event)}: Handles trackball events (legacy input for older devices).
            \end{itemize}
        \item \textbf{TextView} 
            \begin{itemize}
                \item \textbf{void addTextChangedListener(TextWatcher watcher)}: Registers a TextWatcher to receive callbacks when the text changes (before, during, or after editing).
                \item \textbf{void removeTextChangedListener(TextWatcher watcher)}: Unregisters a previously added TextWatcher so it no longer receives callbacks.
                \item \textbf{void setOnEditorActionListener(TextView.OnEditorActionListener l)}: Sets a listener to handle editor actions from the soft keyboard (e.g., pressing "Done", "Next", or "Search").
                \item \textbf{void setOnClickListener(View.OnClickListener l)}: Assigns a listener to handle click events when the view is tapped.
                \item \textbf{void setOnLongClickListener(View.OnLongClickListener l)}: Assigns a listener to handle long-click (press-and-hold) events on the view.
                \item \textbf{void setOnFocusChangeListener(View.OnFocusChangeListener l)}: Sets a listener that is triggered when the view gains or loses input focus.
            \end{itemize}
        \item \textbf{EditText}:
            \begin{itemize}
                \item \textbf{void addTextChangedListener(TextWatcher watcher)}: Registers a listener for text changes.
                \item \textbf{void removeTextChangedListener(TextWatcher watcher)}: Unregisters a text change listener.
                \item \textbf{void setOnEditorActionListener(TextView.OnEditorActionListener l)}: Sets a listener for handling IME action events (e.g., pressing Enter, Done, Search).
            \end{itemize}
        \item \textbf{Button:} Subclass of TextView, which is a subclass of View, so all the ones listed above
    \end{itemize}

    \pagebreak 
    \subsubsection{Java event listeners (2)}
    \begin{itemize}
        \item \textbf{TextWatcher}: The \texttt{TextWatcher} interface, from the android.text package, provides three methods to handle key events.
            \bigbreak \noindent 
            You can attach a TextWatcher to any view that implements Editable text content — meaning subclasses of TextView that allow editing.
            \bigbreak \noindent 
            \begin{javacode}
                import android.text.TextWatcher;
                import android.text.Editable;
                import java.lang.CharSequence;

                .
                .
                .


                private class TextChangeHandler implements TextWatcher {
                    public void beforeTextChanged(CharSequence s, int start, int count, int after )

                    public void onTextChanged(CharSequence s, int start, int before, int after)

                    public void afterTextChanged(Editable e)
                }
            \end{javacode}
            Where (\texttt{beforeTextChanged})
            \begin{itemize}
                \item \textbf{CharSequence s}: The text before the change.
                \item \textbf{int start}: The position (index) in the text where the change will begin.
                \item \textbf{int count}: How many characters are about to be replaced (i.e., how many will be removed).
                \item \textbf{int after}: How many characters will replace the old ones (i.e., how many will be added).
            \end{itemize}
            (onTextChanged)
            \begin{itemize}
                \item \textbf{CharSequence s}: The text after the change (current state).
                \item \textbf{int start}: The position in the text where the change happened.
                \item \textbf{int before}: Number of characters that were replaced (removed).
                \item \textbf{int count}: Number of new characters added.
            \end{itemize}
            \bigbreak \noindent 
            (afterTextChanged)
            \begin{itemize}
                \item \textbf{Editable:}  Represents the text content of the EditText after a change has occurred.
            \end{itemize}
            \bigbreak \noindent 
            Instantiate an instance of the class and attach it to a view with \texttt{.addTextChangedListener()}

        \item \textbf{onClick listener}: Create a private inner class that implements View.OnClickListener, and overrides the function onClick with signature
            \bigbreak \noindent 
            \begin{javacode}
            public void onClick(View v)
            \end{javacode}
            \bigbreak \noindent 
            Create an instance of the inner class and use the function \texttt{setOnClickListener()} to add it to a view.
            \bigbreak \noindent 
            Or, use an anonymous inner class
            \bigbreak \noindent 
            \begin{javacode}
                myView.setOnClickListener(new View.OnClickListener() {
                    @Override
                    public void onClick(View v) {
                        ...
                    }
                });
            \end{javacode}

    \end{itemize}


\pagebreak 
    \subsubsection{Code examples}
    \begin{itemize}
        \item \textbf{Minimum code for controller}:
            \bigbreak \noindent 
            \begin{javacode}
                package com.example.test3;

                import androidx.appcompat.app.AppCompatActivity;

                import android.os.Bundle;

                public class MainActivity extends AppCompatActivity {

                    @Override
                    protected void onCreate(Bundle savedInstanceState) {
                        super.onCreate(savedInstanceState);
                        setContentView(R.layout.activity_main);
                    }
                }
            \end{javacode}
        \item \textbf{Base theme:} 
            \bigbreak \noindent 
            \begin{xmlcode}
                <style name="Base.Theme.TipCalculator" parent="Theme.Material3.DayNight.NoActionBar">
                    <!-- Customize your light theme here. -->
                    <!-- <item name="colorPrimary">@color/my_light_primary</item> -->
                </style>
            \end{xmlcode}
    \end{itemize}

    \pagebreak 
    \subsection{Java reference}
    \bigbreak \noindent 
    \subsubsection{android.content.Context}
    \begin{itemize}
        \item \textbf{What is it}: Context is an interface to global information about your application environment. It gives you access to:
            \begin{itemize}
                \item App resources (colors, strings, layouts, drawables, etc.)
                \item System services (e.g. LayoutInflater, PowerManager, NotificationManager, etc.)
                \item Permissions
                \item Starting new Activities, Services, etc.
            \end{itemize}
            \bigbreak \noindent 
            When you create a View in code:
            \bigbreak \noindent 
            \begin{javacode}
                Button btn = new Button(this);
                GridLayout grid = new GridLayout(this);
            \end{javacode}
            \bigbreak \noindent 
            That this is a Context. In an Activity, this works because Activity is a subclass of Context.
            \bigbreak \noindent 
            The View needs a Context to:
            \begin{itemize}
                \item Know which theme/style to apply
                \item Access resources (e.g., strings, colors, dimensions)
                \item Hook into the Android system for rendering
            \end{itemize}
            Without a Context, a View has no "connection" to the running Android app environment.
    \end{itemize}


    \pagebreak \bigbreak \noindent 
    \subsubsection{Constraint layout in java (and ConstraintSet)}
    \begin{itemize}
        \item \textbf{Needed includes}:
            \bigbreak \noindent 
            \begin{javacode}
                import androidx.constraintlayout.widget.ConstraintLayout;
                import androidx.constraintlayout.widget.ConstraintSet;
                import android.view.ViewGroup;
            \end{javacode}
        \item \textbf{Create ConstraintLayout}:
            \bigbreak \noindent 
            \begin{javacode}
                ConstraintLayout layout = new ConstraintLayout(this);
            \end{javacode}
            The argument \textit{this} is the \textbf{context}, which in this case is the activity.
        \item \textbf{Constraints}: Constraints are essentially sets of rules that dictate the way in which a widget is aligned and distanced in relation to other widgets. The sides of the containing ConstraintLayout and special elements are called \textbf{guidelines}.
            \bigbreak \noindent 
            Constraints also dictate how the user interface layout of an activity will respond to changes in device orientation, or when displayed on devices of differing screen sizes. In order to be adequately configured, a widget must have sufficient constraint connections such that it’s position can be resolved by the ConstraintLayout layout engine in both the horizontal and vertical planes.
        \item \textbf{Margins}: A margin is a form of constraint that specifies a fixed distance. 
        \item \textbf{Opposing Constraints}: Two constraints operating along the same axis on a single widget are referred to as opposing constraints. In other words, a widget with constraints on both its left and right-hand sides is considered to have horizontally opposing constraints. 
            \bigbreak \noindent 
            The key point to understand here is that once opposing constraints are implemented on a particular axis, the positioning of the widget is now based on percentages of the overall dimensions of the ConstraintLayout rather than coordinate based.
            \bigbreak \noindent 
            \fig{.9}{./figures/1008.png}
            \bigbreak \noindent 
            Instead of being fixed at 20dp from the top of the layout, for example, the widget is now positioned at a point 30\% from the top of the layout, regardless of the physical dimensions of the container, or parent layout.
            \bigbreak \noindent 
            In different orientations and when running on larger or smaller screens, the Button will always be in the same location relative to the dimensions of the parent layout.
        \item \textbf{Constraint Bias}: By default, opposing constraints are equal, resulting in the corresponding widget being centered along the axis of opposition. 
            \bigbreak \noindent 
            To allow for the adjustment of widget position in the case of opposing constraints, the ConstraintLayout implements a feature known as constraint bias. Constraint bias allows the positioning of a widget along the axis of opposition to be biased by a specified percentage in favor of one constraint.
            \bigbreak \noindent 
            \fig{.9}{./figures/1009.png}
            \bigbreak \noindent 
            Figure 2b-4, for example, shows the previous constraint layout with a 75\% horizontal bias and 10\% vertical bias:
        \item \textbf{Chains}: ConstraintLayout chains provide a way for the layout behavior of two or more widgets to be defined as a group.
            \bigbreak \noindent 
            Chains can be declared in either the vertical or horizontal axis and configured to define how the widgets in the chain are spaced and sized. Widgets are chained when connected together by bi-directional constraints.
            \bigbreak \noindent 
            \fig{.9}{./figures/1010.png}
            \bigbreak \noindent 
            Figure 2b-5, for example, illustrates three widgets chained in this way
            \bigbreak \noindent 
            The first element in the chain is the chain head which translates to the top widget in a vertical chain or, in the case of a horizontal chain, the left-most widget.
            \bigbreak \noindent 
            The layout behavior of the entire chain is primarily configured by setting attributes on the chain head widget.
        \item \textbf{Chain Styles}: The layout behavior of a ConstraintLayout chain is dictated by the chain style setting applied to the chain head widget; these are as described next.
            \begin{itemize}
                \item \textbf{Spread Chain}: The widgets contained within the chain are distributed evenly across the available space which is the default behavior for chains.
                    \bigbreak \noindent 
                    \fig{.8}{./figures/1011.png}
                \item \textbf{Spread Inside Chain}: The widgets contained within the chain are spread evenly between the chain head and the last widget in the chain. The head and last widgets are not included in the distribution of spacing
                    \fig{.8}{./figures/1012.png}
                \item \textbf{Weighted Chain}: Allows the space taken up by each widget in the chain to be defined via weighting properties.
                    \fig{.8}{./figures/1013.png}
                \item \textbf{Packed Chain}: The widgets that make up the chain are packed together without any spacing. A bias may be applied to control the horizontal or vertical positioning of the chain in relation to the parent container.
                    \fig{.8}{./figures/1014.png}
            \end{itemize}


        \item \textbf{Creating chains with java code}: You do it by manually setting constraints directly on each view’s ConstraintLayout.LayoutParams, using the leftToRight, rightToLeft, topToBottom, etc. fields.
            \bigbreak \noindent 
            Suppose we have three buttons in a ConstraintLayout, 
            \bigbreak \noindent 
            \begin{javacode}
                // --- Button 1 constraints ---
                ConstraintLayout.LayoutParams lp1 = new ConstraintLayout.LayoutParams(
                        ViewGroup.LayoutParams.WRAP_CONTENT,
                        ViewGroup.LayoutParams.WRAP_CONTENT
                );
                lp1.leftToLeft = ConstraintLayout.LayoutParams.PARENT_ID;
                lp1.rightToLeft = btn2.getId();    // Chain with Button 2
                lp1.topToTop = ConstraintLayout.LayoutParams.PARENT_ID;
                lp1.bottomToBottom = ConstraintLayout.LayoutParams.PARENT_ID;
                lp1.horizontalChainStyle = ConstraintLayout.LayoutParams.CHAIN_SPREAD;
                btn1.setLayoutParams(lp1);

                // --- Button 2 constraints ---
                ConstraintLayout.LayoutParams lp2 = new ConstraintLayout.LayoutParams(
                        ViewGroup.LayoutParams.WRAP_CONTENT,
                        ViewGroup.LayoutParams.WRAP_CONTENT
                );
                lp2.leftToRight = btn1.getId();
                lp2.rightToLeft = btn3.getId();
                lp2.topToTop = ConstraintLayout.LayoutParams.PARENT_ID;
                lp2.bottomToBottom = ConstraintLayout.LayoutParams.PARENT_ID;
                btn2.setLayoutParams(lp2);

                // --- Button 3 constraints ---
                ConstraintLayout.LayoutParams lp3 = new ConstraintLayout.LayoutParams(
                        ViewGroup.LayoutParams.WRAP_CONTENT,
                        ViewGroup.LayoutParams.WRAP_CONTENT
                );
                lp3.leftToRight = btn2.getId();
                lp3.rightToRight = ConstraintLayout.LayoutParams.PARENT_ID;
                lp3.topToTop = ConstraintLayout.LayoutParams.PARENT_ID;
                lp3.bottomToBottom = ConstraintLayout.LayoutParams.PARENT_ID;
                btn3.setLayoutParams(lp3);
            \end{javacode}
        \item \textbf{Chain styles in java}:
            \bigbreak \noindent 
            \begin{javacode}
                ConstraintLayout.LayoutParams.CHAIN_SPREAD
                ConstraintLayout.LayoutParams.CHAIN_SPREAD_INSIDE
                ConstraintLayout.LayoutParams.CHAIN_PACKED
            \end{javacode}
            \bigbreak \noindent 
            Notice that in the above example, we give \texttt{ConstraintLayout.LayoutParams.CHAIN\_SPREAD} to the head button in the chain.
        \item \textbf{Weighted chain in java}: To make a weighted chain, you must:
            \begin{enumerate}
                \item Use \texttt{MATCH\_CONSTRAINT} (0dp) for the dimension you want to weight (width in a horizontal chain, height in a vertical one).
                \item Assign a weight to each view using layoutParams.horizontalWeight or layoutParams.verticalWeight.
                \item Use either \texttt{CHAIN\_SPREAD} or \texttt{CHAIN\_SPREAD\_INSIDE} style.
            \end{enumerate}
            \bigbreak \noindent 
            For example, consider again those three buttons
            \bigbreak \noindent 
            \begin{javacode}
                // ---- Button 1 ----
                ConstraintLayout.LayoutParams lp1 = new ConstraintLayout.LayoutParams(
                        0,  // MATCH_CONSTRAINT for weighted width
                        ViewGroup.LayoutParams.WRAP_CONTENT
                );
                lp1.leftToLeft = ConstraintLayout.LayoutParams.PARENT_ID;
                lp1.rightToLeft = btn2.getId();
                lp1.horizontalWeight = 1f; // weight = 1
                lp1.horizontalChainStyle = ConstraintLayout.LayoutParams.CHAIN_SPREAD;
                btn1.setLayoutParams(lp1);

                // ---- Button 2 ----
                ConstraintLayout.LayoutParams lp2 = new ConstraintLayout.LayoutParams(
                        0,
                        ViewGroup.LayoutParams.WRAP_CONTENT
                );
                lp2.leftToRight = btn1.getId();
                lp2.rightToLeft = btn3.getId();
                lp2.horizontalWeight = 2f; // weight = 2
                btn2.setLayoutParams(lp2);

                // ---- Button 3 ----
                ConstraintLayout.LayoutParams lp3 = new ConstraintLayout.LayoutParams(
                        0,
                        ViewGroup.LayoutParams.WRAP_CONTENT
                );
                lp3.leftToRight = btn2.getId();
                lp3.rightToRight = ConstraintLayout.LayoutParams.PARENT_ID;
                lp3.horizontalWeight = 1f; // weight = 1
                btn3.setLayoutParams(lp3);
            \end{javacode}
        \item \textbf{MATCH\_CONSTRAINT}: Let’s break down what MATCH\_CONSTRAINT means, how it differs from WRAP\_CONTENT and MATCH\_PARENT, and when you use it.
            \bigbreak \noindent 
            MATCH\_CONSTRAINT is a special size mode in ConstraintLayout that tells a view:
            \bigbreak \noindent 
            \begin{center}
                "Size yourself dynamically based on your constraints, rather than fixed content or parent size."
            \end{center}
            \bigbreak \noindent 
            In Java, it’s specified by setting the dimension (width or height) to 0 in the layout params:
            \bigbreak \noindent 
            \begin{javacode}
                ConstraintLayout.LayoutParams params = new ConstraintLayout.LayoutParams(0, WRAP_CONTENT);
            \end{javacode}
            \bigbreak \noindent 
            View size is flexible and determined by the constraints and optionally by weights or ratios
        \item \textbf{Baseline Alignment}: So far,  we have only  referred to constraints that dictate alignment relative to the sides of a widget (typically referred to as side constraints).
            \bigbreak \noindent 
            A common requirement, however, is for a widget to be aligned relative to the content that it displays rather than the boundaries of the widget itself. To address this need, ConstraintLayout provides baseline alignment support.
            \bigbreak \noindent 
            Every view that displays text (e.g. TextView, EditText, Button) has a \textbf{text baseline} - the imaginary horizontal line upon which the text "sits."
            \bigbreak \noindent 
            Baseline alignment means you’re aligning two or more views based on that text baseline, instead of their tops or bottoms.
            \bigbreak \noindent 
            Suppose we require a TextView widget to be positioned 40dp to the left of the Button. 
            \bigbreak \noindent 
            In this case, the TextView needs to be baseline aligned with the Button view.
            \bigbreak \noindent 
            This means that the text within the Button needs to be vertically aligned with the text within the TextView.
            \bigbreak \noindent 
            \fig{.8}{./figures/1015.png}
            \bigbreak \noindent 
            The TextView is now aligned vertically along the baseline of the Button and positioned 40dp horizontally from the Button object’s left-hand edge.
            \bigbreak \noindent 
            In Java, you connect one view’s baseline to another view’s baseline using:
            \bigbreak \noindent 
            \begin{javacode}
            params.baselineToBaseline = otherView.getId();
            \end{javacode}
            \bigbreak \noindent 
            You can combine this with other constraints like leftToLeft, topToTop, etc.
        \item \textbf{Working with Guidelines}: Guidelines are special elements available within the ConstraintLayout that provide an additional target to which constraints may be connected.
            \bigbreak \noindent 
            Multiple guidelines may be added to a ConstraintLayout instance which may, in turn, be configured in horizontal or vertical orientations.
            \bigbreak \noindent 
            Once added, constraint connections may be established from widgets in the layout to the guidelines. This is particularly useful when multiple widgets need to be aligned along an axis. 
            \bigbreak \noindent 
            \fig{.8}{./figures/1016.png}
        \item \textbf{Guidelines in java}: First we create a \texttt{Guideline} object, then configure \texttt{ConstraintLayout.LayoutParams} for it.
            \bigbreak \noindent 
            \begin{javacode}
                import androidx.constraintlayout.widget.Guideline 

                // 1) Make the guideline
                Guideline vGuide = new Guideline(this);
                vGuide.setId(View.generateViewId());

                ConstraintLayout.LayoutParams vgLP = new ConstraintLayout.LayoutParams(
                        ConstraintLayout.LayoutParams.WRAP_CONTENT,
                        ConstraintLayout.LayoutParams.WRAP_CONTENT
                );
                vgLP.orientation = ConstraintLayout.LayoutParams.VERTICAL;
                // Choose ONE of these three ways to position it:
                vgLP.guidePercent = 0.30f;   // 30% from the left (0..1)
                // vgLP.guideBegin  = 120;    // 120px from the left
                // vgLP.guideEnd    = 16;     // 16px from the right

                vGuide.setLayoutParams(vgLP);
                layout.addView(vGuide);

                // 2) Constrain a view to the guideline
                TextView tv = new TextView(this);
                tv.setId(View.generateViewId());
                tv.setText("Hello");
                ConstraintLayout.LayoutParams tvLP = new ConstraintLayout.LayoutParams(
                0,  // MATCH_CONSTRAINT width so it can expand between guideline and parent
                ConstraintLayout.LayoutParams.WRAP_CONTENT
                );
                tvLP.leftToRight = vGuide.getId();                       // to the right of guideline
                tvLP.rightToRight = ConstraintLayout.LayoutParams.PARENT_ID;
                tvLP.topToTop = ConstraintLayout.LayoutParams.PARENT_ID;

                tv.setLayoutParams(tvLP);
                layout.addView(tv);

                setContentView(layout);
            \end{javacode}
            \bigbreak \noindent 
            \textbf{Note:} \texttt{ConstraintLayout.LayoutParams.orientation} is a property that only applies to Guidelines, not to regular Views.
            \bigbreak \noindent 
            \texttt{ConstraintLayout.LayoutParams.orientation} tells the ConstraintLayout whether a Guideline is:
            \begin{itemize}
                \item \textbf{Vertical}: divides the layout left $\leftrightarrow $ right
                \item \textbf{Horizontal}: divides the layout top $\leftrightarrow$ bottom
            \end{itemize}
        \item \textbf{Working with Barriers}: Rather like guidelines, barriers are virtual views that can be used to constrain views within a layout
            \bigbreak \noindent 
            As with guidelines, a barrier can be vertical or horizontal and one or more views may be constrained to it (to avoid confusion, these will be referred to as constrained views).
            \bigbreak \noindent 
            Unlike guidelines where the guideline remains at a fixed position within the layout, however, the position of a barrier is defined by a set of so called reference views.
            \bigbreak \noindent 
            Barriers were introduced to address an issue that occurs with some frequency involving overlapping views.
            \bigbreak \noindent 
            Consider the following example
            \bigbreak \noindent 
            \fig{.8}{./figures/1017.png}
            \bigbreak \noindent 
            The key points to note about the above layout is that the width of View 3 is set to match constraint mode, and the left-hand edge of the view is connected to the right hand edge of View 1.
            \bigbreak \noindent 
            As currently implemented, an increase in width of View 1 will have the desired effect of reducing the width of View 3:
            \bigbreak \noindent 
            \fig{.8}{./figures/1018.png}
            \bigbreak \noindent 
            A problem arises, however, if View 2 increases in width instead of View 1:
            \bigbreak \noindent 
            \fig{.8}{./figures/1019.png}
            \bigbreak \noindent 
            Clearly because View 3 is only constrained by View 1, it does not resize to accommodate the increase in width of View 2 causing the views to overlap.
            \bigbreak \noindent 
            A solution to this problem is to add a vertical barrier and assign Views 1 and 2 as the barrier’s reference views so that they control the barrier position.
            \bigbreak \noindent 
            The left-hand edge of View 3 will then be constrained in relation to the barrier, making it a constrained view.
            \bigbreak \noindent 
            Now when either View 1 or View 2 increase in width, the barrier will move to accommodate the widest of the two views, causing the width of View 3 change in relation to the new barrier position:
            \bigbreak \noindent 
            \fig{.8}{./figures/1020.png}
            \bigbreak \noindent 
            When working with barriers there is no limit to the number of reference views and constrained views that can be associated with a single barrier.
        \item \textbf{Barriers in java}: A Barrier is a virtual helper object in ConstraintLayout that positions itself dynamically based on the position of other views. It’s like a movable guideline that automatically adjusts to the furthest edge of a group of views.
            \bigbreak \noindent 
            Assume we have two TextViews
            \bigbreak \noindent 
            \begin{javacode}
                import androidx.constraintlayout.widget.Barrier;

                // --- Create a barrier ---
                Barrier barrier = new Barrier(this);
                barrier.setId(View.generateViewId());
                barrier.setType(Barrier.END); // or LEFT, RIGHT, START, TOP, BOTTOM
                barrier.setReferencedIds(new int[]{label1.getId(), label2.getId()});
                layout.addView(barrier);

                // Create a button that stays to the right of both text views
                Button btn = new Button(this);
                btn.setId(View.generateViewId());
                btn.setText("Aligned with longest");
                ConstraintLayout.LayoutParams lp3 = new ConstraintLayout.LayoutParams(
                        ConstraintLayout.LayoutParams.WRAP_CONTENT,
                        ConstraintLayout.LayoutParams.WRAP_CONTENT
                );
                lp3.startToEnd = barrier.getId(); // use barrier directly
                lp3.topToTop = t1.getId();       // align top with first label
                lp3.leftMargin = 16;
                btn.setLayoutParams(lp3);

                layout.addView(btn);

                ...
            \end{javacode}

        \item \textbf{Barrier types}: Barrier types you can use:
            \begin{itemize}
                \item Barrier.START
                \item Barrier.END
                \item Barrier.LEFT
                \item Barrier.RIGHT
                \item Barrier.TOP
                \item Barrier.BOTTOM
            \end{itemize}
            Which are explained in the following table
            \bigbreak \noindent 
            \begin{center}
                \begin{tabular}{p{4cm}|p{3cm}|p{2cm}|p{3cm}}
                    Constant	&Tracks the...	&Moves toward...	&Typical Use \\
                    \hline \\
                    Barrier.START	&Start edges (left in LTR, right in RTL)	&Start side	&Keep something to the left of the leftmost view \\ && \\
                    Barrier.END	End &edges (right in LTR, left in RTL)	&End side	&Keep something to the right of the rightmost view \\&& \\ 
                    Barrier.LEFT	&Left edges (absolute)	&Left side	&Same as START but ignores layout direction \\ && \\
                    Barrier.RIGHT	&Right edges (absolute)	&Right side	&Same as END but ignores layout direction \\ && \\
                    Barrier.TOP	&Top edges	&Upward	&Keep something below the topmost/tallest view \\ && \\
                    Barrier.BOTTOM	&Bottom edges	&Downward	&Keep something below the lowest/bottommost view 
                \end{tabular}
            \end{center}






        \item \textbf{ConstraintLayout.LayoutParams}: Has constructor
            \bigbreak \noindent 
            \begin{javacode}
            ConstraintLayout.LayoutParams(int width, int height)
            \end{javacode}
            Lets you define width and height (MATCH\_PARENT, WRAP\_CONTENT, or px).
            \bigbreak \noindent 
            For example,
            \bigbreak \noindent 
            \begin{javacode}
                ConstraintLayout.LayoutParams lp =
                    new ConstraintLayout.LayoutParams(
                    ViewGroup.LayoutParams.WRAP_CONTENT,
                    ViewGroup.LayoutParams.WRAP_CONTENT
                );
            \end{javacode}
            Then, we can add constraints, Each field takes an ID of another view (or PARENT\_ID):
            \begin{itemize}
                \item \textbf{leftToLeft}: Align the left edge of this view to the left edge of another view (or parent).
                \item \textbf{leftToRight}: Align the left edge of this view to the right edge of another view.
                \item \textbf{rightToLeft}: Align the right edge of this view to the left edge of another view.
                \item \textbf{rightToRight}: Align the right edge of this view to the right edge of another view.
                \item \textbf{topToTop}: Align the top edge of this view to the top edge of another view.
                \item \textbf{topToBottom}: Align the top edge of this view to the bottom edge of another view.
                \item \textbf{bottomToTop}: Align the bottom edge of this view to the top edge of another view.
                \item \textbf{bottomToBottom}: Align the bottom edge of this view to the bottom edge of another view.
                \item \textbf{startToStart}: Align the start edge of this view to the start edge of another view.
                \item \textbf{startToEnd}: Align the start edge of this view to the end edge of another view.
                \item \textbf{endToStart}: Align the end edge of this view to the start edge of another view.
                \item \textbf{endToEnd}: Align the end edge of this view to the end edge of another view.
                \item \textbf{horizontalBias(float b)}: $b \times 100$ percent from left
                \item \textbf{verticalBias(float b)}: $b \times 100$ percent from top
            \end{itemize}
            \bigbreak \noindent 
            \textbf{Note:}
            \begin{itemize}
                \item left/right = physical edges (always left/right of the screen).
                \item start/end = logical edges (switch meaning in RTL layouts).
            \end{itemize}
            \bigbreak \noindent 
            For example,
            \bigbreak \noindent 
            \begin{javacode}
            lp.topToTop = ConstraintLayout.LayoutParams.PARENT_ID;
            \end{javacode}
            \bigbreak \noindent 
            We also have chains, for arranging multiple views in a line with flexible spacing:
        \item \textbf{Retrieve and update LayoutParams}:
            \bigbreak \noindent 
            \begin{javacode}
            ConstraintLayout.LayoutParams lp = (ConstraintLayout.LayoutParams) view.getLayoutParams();
            ...
            view.setLayoutParams(lp);
            \end{javacode}





    \end{itemize}

    \pagebreak 
    \subsubsection{Grid layout in java}
    \bigbreak \noindent 
    GridLayout is a type of ViewGroup that arranges its children in a grid of rows and columns.
    \bigbreak \noindent 
    Each child view is placed into a "cell" defined by its row and column. You can make a child span multiple rows or columns. It’s similar to a table layout but more flexible (alignments, spans, etc.).
    \begin{itemize}
        \item \textbf{Needed Includes}:
            \bigbreak \noindent 
            \begin{javacode}
                import android.content.Context;
                import android.view.ViewGroup;
                import android.widget.GridLayout;
            \end{javacode}
        \item \textbf{Create GridLayout}: You can make a GridLayout in code just like any other layout:
            \bigbreak \noindent 
            \begin{javacode}
                GridLayout grid = new GridLayout(this);   // "this" = Context, usually Activity
                grid.setRowCount(3);                      // number of rows
                grid.setColumnCount(3);                   // number of columns
            \end{javacode}
        \item \textbf{Setting LayoutParams for GridLayout}: We use ViewGroup.LayoutParams to set the layout params for the entire grid container.
            \bigbreak \noindent 
            \begin{javacode}
                grid.setLayoutParams(
                    new ViewGroup.LayoutParams(
                        ViewGroup.LayoutParams.MATCH_PARENT,
                        ViewGroup.LayoutParams.MATCH_PARENT
                    )
                );
            \end{javacode}
            
        \item \textbf{GridLayout.LayoutParams}: You need GridLayout.LayoutParams for child views placed inside a GridLayout. GridLayout.LayoutParams is the special layout parameter class that children of a GridLayout must use.
            \bigbreak \noindent 
         It tells the GridLayout parent:
            \begin{itemize}
                \item Which row and column this child belongs to
                \item How many cells to span
                \item How to align inside those cells
                \item How much margin space it should have
            \end{itemize}
            For example,
            \bigbreak \noindent 
            \begin{javacode}
                Button btn = new Button(this);
                btn.setText("Hi");

                GridLayout.LayoutParams btnLp = new GridLayout.LayoutParams(
                        GridLayout.spec(0),  // row 0
                        GridLayout.spec(1)   // column 1
                );
                btnLp.width = GridLayout.LayoutParams.WRAP_CONTENT;
                btnLp.height = GridLayout.LayoutParams.WRAP_CONTENT;

                btn.setLayoutParams(btnLp);
                grid.addView(btn);
            \end{javacode}
            \bigbreak \noindent 
            \texttt{GridLayout.spec(int index)} creates a Spec object. A Spec describes a position in either rows or columns. Here, GridLayout.spec(0) means row 0 (the first row), and GridLayout.spec(1) means the first column.
        \item \textbf{GridLayout.spec overloads}:
            \bigbreak \noindent 
            \begin{javacode}
                GridLayout.spec(int index) // Single index, default span=1, default alignment = UNDEFINED
                GridLayout.spec(int index, int size) // Index + span
                GridLayout.spec(int index, Alignment align) // Index + alignment
                GridLayout.spec(int index, int size, Alignment align) // Index + span + alignment
            \end{javacode}
            You can pass these alignment constants:
            \begin{itemize}
                \item \textbf{GridLayout.START}: align to start (left or top).
                \item \textbf{GridLayout.END}: align to end (right or bottom).
                \item \textbf{GridLayout.CENTER}: center inside the cell.
                \item \textbf{GridLayout.FILL}: expand to fill the cell.
            \end{itemize}
            (BASELINE exists for aligning text baselines in rows.)

        \item \textbf{Rule of thumb}: 
            Use ViewGroup.LayoutParams when sizing a container relative to its parent.
            \bigbreak \noindent 
            Use the container’s own LayoutParams subclass (GridLayout.LayoutParams, LinearLayout.LayoutParams, etc.) when adding children inside that container.
        \item \textbf{Using MarginLayoutParams on GridLayout.setLayoutParams}: This affects the GridLayout as a whole, not its children.
            \bigbreak \noindent 
            For example, 
            \bigbreak \noindent 
            \begin{javacode}
                ViewGroup.MarginLayoutParams lp =
                new ViewGroup.MarginLayoutParams(
                    ViewGroup.LayoutParams.MATCH_PARENT,
                    ViewGroup.LayoutParams.WRAP_CONTENT
                );

                lp.setMargins(20, 40, 20, 40); // left, top, right, bottom in px

                grid.setLayoutParams(lp);
            \end{javacode}

    \end{itemize}

    \pagebreak 
    \subsubsection{Buttons in java}
    \begin{itemize}
        \item \textbf{Subclass}: Buttons are a subclass of TextView, which is a subclass of View 
        \item \textbf{Creating buttons}: There are basically 3 parts:
            \begin{enumerate}
                \item Construct the Button (needs a Context)
                \item Customize it (text, size, colors, etc.)
                \item Add it to a parent layout (like GridLayout, LinearLayout, etc.) with proper LayoutParams.
            \end{enumerate}
            To create the button,
            \bigbreak \noindent 
            \begin{javacode}
            Button myButton = new Button(context);
            \end{javacode}
        \item \textbf{Button methods}:
            \begin{itemize}
                \item setTextSize(int size)
                \item setOnClickListener(listener)
                \item setEnabled(boolean status)
                \item setText(string text)
                \item setLayoutParams(lp)
            \end{itemize}
    \end{itemize}

    \pagebreak 
    \subsubsection{TextView and EditText}
    \begin{itemize}
        \item \textbf{Creating TextView}: TextView is a subclass of View
            \bigbreak \noindent 
            \begin{javacode}
            TextView textView = new TextView(context);
            \end{javacode}
        \item \textbf{TextView methods}:
            \begin{itemize}
                \item setWidth(int w)
                \item setHeight(int h)
                \item setGravity(Gravity g)
                \item setBackgroundColor(Color c)
                \item setTextSize(int size)
                \item setText(string text)
                \item setBackgroundColor(Color c)
                \item setLayoutParams(lp)
                \item setTypeFace(Typeface font, int style)
                    \begin{itemize}
                        
                    \end{itemize}
            \end{itemize}
        \item \textbf{Creating EditText}: EditText is a subclass of TextView, which is a subclass of View
            \bigbreak \noindent 
            \begin{javacode}
            EditText editText = new EditText(context);
            \end{javacode}
        \item \textbf{EditText methods}:
            \begin{itemize}
                \item setWidth(int w)
                \item setHeight(int h)
                \item setHint(String)
                \item setGravity(Gravity g)
                \item setBackgroundColor(Color c)
                \item setTextSize(int size)
                \item setText(string text)
                \item setBackgroundColor(Color c)
                \item setLayoutParams(lp)
            \end{itemize}
    \end{itemize}

    \pagebreak 
    \subsubsection{android.graphics.Color}
    \begin{itemize}
        \item \textbf{Color constants}: This class defines color constants and helper methods for working with colors.
            \bigbreak \noindent 
            \begin{javacode}
            Color.RED
            Color.BLUE
            Color.BLACK
            Color.WHITE
            \end{javacode}
            \bigbreak \noindent 
            These are just integer values representing ARGB colors.
        \item \textbf{Create colors}:
            \begin{itemize}
                \item ARGB values (alpha, red, green, blue):
                \bigbreak \noindent 
                \begin{javacode}
                int myColor = Color.argb(255, 100, 200, 150);
                \end{javacode}
                \bigbreak \noindent 
                Here, 255 = fully opaque.
            \item RGB values (no alpha, alpha = 255):
                \bigbreak \noindent 
                \begin{javacode}
                int myColor = Color.rgb(100, 200, 150);
                \end{javacode}
            \item Parse from string:
                \bigbreak \noindent 
                \begin{javacode}
                int myColor = Color.parseColor("#FF5722");  // hex code
                \end{javacode}
            \end{itemize}

    \end{itemize}



    \pagebreak 
    \subsubsection{android.view.Gravity}
    \begin{itemize}
        \item \textbf{Intro}: This class defines constants used to position or align content inside a View.
            \bigbreak \noindent 
            It doesn’t move the view itself — it controls how the content inside a view (like text in a TextView) or a child inside a parent layout is aligned.
            \begin{itemize}
                \item Gravity.LEFT / Gravity.RIGHT
                \item Gravity.TOP / Gravity.BOTTOM
                \item Gravity.CENTER (both horizontally + vertically)
                \item Gravity.CENTER\_HORIZONTAL
                \item Gravity.CENTER\_VERTICAL
                \item Gravity.FILL (stretch to fill)
            \end{itemize}
            Use bitwise OR (|) to combine:
            \bigbreak \noindent 
            \begin{javacode}
            textView.setGravity(Gravity.CENTER | Gravity.BOTTOM);
            \end{javacode}
    \end{itemize}

    \pagebreak 
    \subsubsection{DiaglogInterface and AlertDialog}
    \begin{itemize}
        \item \textbf{AlertDialog}: A subclass of Dialog. Used to show a modal pop-up window on top of the activity — typically for alerts, confirmations, or choices
            It can have:
            \begin{itemize}
                \item A title
                \item A message
                \item Optional icon
                \item Up to 3 buttons (Positive, Negative, Neutral)
                \item Custom layouts (if you want more than just text)
            \end{itemize}
        \item \textbf{Create an alert}:
            \bigbreak \noindent 
            \begin{javacode}
            AlertDialog.Builder alert = new AlertDialog.Builder(this);
            ...
            AlertDialog dialog = alert.create();
            dialog.show();
            \end{javacode}
        \item \textbf{AlertDialog.builder methods}
            \begin{itemize}
                \item alert.setTitle(string title);
                \item alert.setMessage(string message);
                \item alert.setPositiveButton(string buttonText, DialogInterface.onClickListener);
                \item alert.setNegativeButton(string buttonText, DialogInterface.onClickListener);
                \item alert.show();
            \end{itemize}
        \item \textbf{DialogInterface}: This is just an interface. Many dialog-related classes (including AlertDialog) implement it.
            \bigbreak \noindent 
            \begin{javacode}
                public interface DialogInterface {
                    void cancel();
                    void dismiss();

                    interface OnCancelListener {
                        void onCancel(DialogInterface dialog);
                    }

                    interface OnDismissListener {
                        void onDismiss(DialogInterface dialog);
                    }

                    interface OnClickListener {
                        void onClick(DialogInterface dialog, int which);
                    }

                    // ... and others like OnKeyListener, OnMultiChoiceClickListener
                }
            \end{javacode}
            \bigbreak \noindent 
            It gives you common methods to control the dialog:
            \begin{itemize}
                \item \textbf{dismiss()}: close the dialog
                \item \textbf{cancel()}: cancel the dialog (triggers onCancel() callback)
            \end{itemize}
            It’s also used in listeners for button clicks.
        \item \textbf{DialogInterface.OnClickListener}: We can use this interface to show the alert when something is clicked.
            \bigbreak \noindent 
            \begin{javacode}
                private class MyDialog implements DialogInterface.OnClickListener
                {

                    public void onClick(DialogInterface dialog, int which)
                    {
                        
                    }
                }
            \end{javacode}
            \bigbreak \noindent 
            The which parameter in listeners
            \begin{itemize}
                \item DialogInterface.BUTTON\_POSITIVE (-1)
                \item DialogInterface.BUTTON\_NEGATIVE (-2)
                \item DialogInterface.BUTTON\_NEUTRAL (-3)
            \end{itemize}
            So you know which button was pressed.
            \bigbreak \noindent 
            The dialog parameter is a reference to the dialog that triggered the click. Its type is the interface DialogInterface, but in practice it will usually be an instance of a concrete class like AlertDialog.
            You can use this reference to control the dialog inside the callback:
            \begin{itemize}
                \item \textbf{dialog.dismiss()}: close the dialog immediately.
                \item \textbf{dialog.cancel()}: cancel the dialog (triggers OnCancelListener if one is set).
            \end{itemize}
            \bigbreak \noindent 
            \begin{javacode}
                builder.setPositiveButton("OK", new DialogInterface.OnClickListener() {
                    @Override
                    public void onClick(DialogInterface dialog, int id) {
                        // dialog is the AlertDialog, typed as DialogInterface
                        dialog.dismiss();   // closes it
                    }
                });
            \end{javacode}

    \end{itemize}

    \pagebreak 
    \subsubsection{GradientDrawable}
    \begin{itemize}
        \item \textbf{Includes}:
            \bigbreak \noindent 
            \begin{javacode}
                import android.graphics.drawable.GradientDrawable;
            \end{javacode}
        \item \textbf{GradientDrawable}: A GradientDrawable is a drawable object (something you can use as a background or graphic) that can display:
            \begin{itemize}
                \item Solid colors, or Gradients (color transitions),
            \end{itemize}
            and can have:
            \begin{itemize}
                \item Rounded corners,
                \item Borders (strokes),
                \item Different shapes (rectangle, oval, line, ring).
            \end{itemize}
            Essentially, it’s Android’s built-in shape painter
        \item \textbf{Creation}:
            \bigbreak \noindent 
            \begin{javacode}
            GradientDrawable shape = new GradientDrawable();
            \end{javacode}
        \item \textbf{Methods}:
            \begin{center}
                \begin{tabular}{p{4cm}|p{4cm}|p{4cm}}
                    \hline
                    \textbf{Method} & \textbf{Description} & \textbf{Example Usage} \\
                    \hline
                    \texttt{setShape(int shape)} & Sets the type: \texttt{RECTANGLE}, \texttt{OVAL}, \texttt{LINE}, \texttt{RING} & \texttt{setShape(GradientDrawable.OVAL)} \\
                    \hline
                    \texttt{setCornerRadius(float radius)} & Rounds the corners (only works for rectangles) & \texttt{setCornerRadius(30f)} \\
                    \hline
                    \texttt{setCornerRadii(float[] radii)} & Gives each corner a different roundness & \texttt{setCornerRadii(new float[]{20,20,0,0,20,20,0,0})} \\
                    \hline
                    \texttt{setColor(int color)} & Fills with a solid color & \texttt{setColor(Color.BLUE)} \\
                    \hline
                    \texttt{setStroke(int width, int color)} & Adds a border & \texttt{setStroke(3, Color.WHITE)} \\
                    \hline
                    \texttt{setGradientType(int type)} & Chooses gradient: \texttt{LINEAR}, \texttt{RADIAL}, \texttt{SWEEP} & \texttt{setGradientType(GradientDrawable.LINEAR\_GRADIENT)} \\
                    \hline
                    \texttt{setColors(int[] colors)} & Defines colors for gradient transitions & \texttt{setColors(new int[]{Color.RED, Color.YELLOW})} \\
                    \hline
                \end{tabular}
            \end{center}
        \item \textbf{Using GradientDrawable}: We can then call .setBackground() on a view, passing in our GradientDrawable.
        \item \textbf{Giving a button rounded edges}:
            \bigbreak \noindent 
            \begin{javacode}
                Button b = new Button();

                GradientDrawable button_shape = new GradientDrawable();
                button_shape.setShape(GradientDrawable.RECTANGLE);
                button_shape.setCornerRadius(35f);
                button_shape.setColor(PURPLE);

                b.setBackground(button_shape);
            \end{javacode}
    \end{itemize}

    \pagebreak 
    \subsubsection{android.graphics.Typeface}
    \begin{itemize}
        \item \textbf{Typeface class}: the Typeface class in Android is the foundation for all text styling related to fonts and weight (bold, italic, etc.). It represents the font face used to render text on screen — in a TextView, Canvas, or anywhere text is drawn.
            \bigbreak \noindent 
            A \texttt{Typeface} is an object that describes the style and family of a font. It defines how text looks — e.g. whether it’s serif or sans-serif, bold or italic, or even a custom font you loaded.
        \item \textbf{Font family constants}:
            \bigbreak \noindent 
            \begin{center}
                \begin{tabular}{p{4cm}|p{4cm}|p{4cm}}
                    Constant	&Font family	&Appearance \\
                    \hline \\[0.01cm]
                    Typeface.DEFAULT	&Default system font (usually Roboto on newer Androids)	&Plain \\[2ex]
                    Typeface.SANS\_SERIF	&Sans-serif	&Clean, modern \\[2ex]
                    Typeface.SERIF	&Serif	&Classic (like Times New Roman) \\[2ex]
                    Typeface.MONOSPACE	&Monospace	&Fixed-width (like Courier New)
                \end{tabular}
            \end{center}
        \item \textbf{Weight constants}:
            \bigbreak \noindent 
            \begin{center}
                \begin{tabular}{p{4cm}|p{1cm}}
                    Constant	&Style \\ 
                    \hline \\[0.01cm]
                    Typeface.NORMAL	&0 \\[2ex]
                    Typeface.BOLD	&1 \\[2ex]
                    Typeface.ITALIC	&2 \\[2ex]
                    Typeface.BOLD\_ITALIC	&3
                \end{tabular}
            \end{center}
        \item \textbf{Using with setTypeFace}
            \bigbreak \noindent 
            \begin{javacode}
            label.setTypeface(Typeface.SERIF, Typeface.BOLD);
            \end{javacode}
        \item \textbf{Null as a parameter}: We can set the font family but not weight
            \bigbreak \noindent 
            \begin{javacode}
            label.setTypeface(null, Typeface.BOLD);
            \end{javacode}
            \bigbreak \noindent 
            We can set the family but not weight by using the overload that only accepts a family
            \bigbreak \noindent 
            \begin{javacode}
            label.setTypeface(Typeface.SERIF);
            \end{javacode}
        \item \textbf{Loading custom font from res/font}: If you have \texttt{res/font/roboto\_bold.ttf}, you can load it like:
            \bigbreak \noindent 
            \begin{javacode}
                Typeface roboto = ResourcesCompat.getFont(this, R.font.roboto_bold);
                textView.setTypeface(roboto);
            \end{javacode}
            \bigbreak \noindent 
            or programmatically from assets:
            \bigbreak \noindent 
            \begin{javacode}
                Typeface tf = Typeface.createFromAsset(getAssets(), "fonts/CustomFont.ttf");
                textView.setTypeface(tf);
            \end{javacode}
        \item \textbf{Methods}:
            \bigbreak \noindent 
            \begin{center}
                \begin{tabular}{p{6cm}|p{6cm}}
                    Method	&Description \\
                    \hline \\[0.01cm]
                    \texttt{create(Typeface family, int style)}	&Returns a new Typeface based on an existing family and style. \\[2ex]
                    \texttt{createFromAsset(AssetManager mgr, String path)}:	&Loads a Typeface from an asset file (assets/fonts/...). \\[2ex]
                    \texttt{createFromFile(File path)}:	&Loads from a file on disk. \\[2ex]
                    \texttt{defaultFromStyle(int style)}:	&Returns the default Typeface for a style (e.g. Typeface.defaultFromStyle(Typeface.BOLD)). \\[2ex]
                    \texttt{equals(Object obj)}:	&Compares two typefaces. \\[2ex]
                    \texttt{hashCode()}:	&Hash for comparison.
                \end{tabular}
            \end{center}

    \end{itemize}

    \pagebreak 
    \subsubsection{Relative layout}
    \begin{itemize}
        \item \textbf{Include}
            \bigbreak \noindent 
            \begin{javacode}
                android.widget.RelativeLayout
            \end{javacode}
        \item \textbf{Relative layout}: RelativeLayout is a ViewGroup that lets you position child views relative to each other or to the parent container.
            \bigbreak \noindent 
            Each child view can be placed relative to the parent (top, bottom, left, right, center, etc.) or relative to another view (above, below, to the left/right of another widget).
        \item \textbf{Create a RelativeLayout}:
            \bigbreak \noindent 
            \begin{javacode}
                RelativeLayout layout = new RelativeLayout(this);
                layout.setLayoutParams(new ViewGroup.LayoutParams(
                    ViewGroup.LayoutParams.MATCH_PARENT,
                    ViewGroup.LayoutParams.MATCH_PARENT
                ));
            \end{javacode}
        \item \textbf{Create RelativeLayout.LayoutParams}
            \bigbreak \noindent 
            \begin{javacode}
            RelativeLayout.LayoutParams rlp = new RelativeLayout.LayoutParams(new ViewGroup.LayoutParams(
                RelativeLayout.LayoutParams.WRAP_CONTENT,
                RelativeLayout.LayoutParams.WRAP_CONTENT
            ));
            \end{javacode}
        \item \textbf{.addRule()}: Adds a layout rule to be interpreted by the RelativeLayout. There are two versions
            \bigbreak \noindent 
            \begin{javacode}
                void addRule(int verb, int subject)
                void addRule(int verb)
            \end{javacode}
            \bigbreak \noindent 
            The first version applies a standalone rule - one that does not reference another view.
            \bigbreak \noindent 
            \begin{javacode}
                params.addRule(RelativeLayout.CENTER_IN_PARENT);
            \end{javacode}
            The rule means: "center this view both horizontally and vertically inside the parent"
            \bigbreak \noindent 
            The second version defines a relationship between this view and another view (by ID).


    \end{itemize}

    \pagebreak 
    \subsubsection{Linear layout}
    \begin{itemize}
        \item 
    \end{itemize}

    \pagebreak 
    \subsubsection{Table layout}
    \begin{itemize}
        \item \relax 
    \end{itemize}

    \pagebreak 
    \subsubsection{Frame layout}

    \pagebreak 
    \subsubsection{List view}

    \pagebreak 
    \subsubsection{Image view}

    \pagebreak 
    \subsubsection{Compound Button}

    \pagebreak 
    \subsubsection{Check box}

    \pagebreak 
    \subsubsection{Radio Button}

    \pagebreak 
    \subsubsection{Adapter view}

    \pagebreak 
    \subsubsection{Abs spinner}

    \pagebreak 
    \subsubsection{Spinner}

    \pagebreak 
    \subsubsection{Progress bar}

    \pagebreak 
    \subsubsection{Abs seek bar}

    \pagebreak 
    \subsubsection{Seek bar}

    \pagebreak 
    \subsubsection{AttributeSet}

    \pagebreak 
    \subsubsection{Constraint set}

    \pagebreak 
    \subsubsection{defStyleAttr}

    \pagebreak 
    \subsubsection{defStyleRes}


    \pagebreak 
    \subsubsection{android.view.Display}
    \begin{itemize}
        \item \textbf{Display class}: The android.view.Display class represents a physical screen or display device that your app’s UI can be shown on. It provides detailed information about the screen your app is running on — such as its size, refresh rate, orientation, and pixel density.
            \bigbreak \noindent 
            It is useful when adapting layouts, scaling graphics, or handling multi-screen setups
        \item \textbf{Getting a display object}: You usually don’t create Display yourself. Instead, you retrieve it from a system service like \texttt{WindowManager}
            \bigbreak \noindent 
            \begin{javacode}
                WindowManager wm = (WindowManager) getSystemService(Context.WINDOW_SERVICE);
                Display display = wm.getDefaultDisplay();
            \end{javacode}
            \bigbreak \noindent 
            Or, with
            \bigbreak \noindent 
            \begin{javacode}
            Display display = getWindowManager().getDefaultDisplay();
            \end{javacode}
            \bigbreak \noindent 
            or, in newer Android versions (API 30+):
            \bigbreak \noindent 
            \begin{javacode}
            Display display = getDisplay();  // available from any Activity or View
            \end{javacode}
            \bigbreak \noindent 
            If you’re in a non-Activity class (like a helper or controller class), you can get it through a Context:
            \bigbreak \noindent 
            \begin{javacode}
                WindowManager wm = (WindowManager) context.getSystemService(Context.WINDOW_SERVICE);
                Display display = wm.getDefaultDisplay();
            \end{javacode}

        \item \textbf{Getting display size}: To get the size of the usable screen in pixels:
            \bigbreak \noindent 
            \begin{javacode}
                Display display = getWindowManager().getDefaultDisplay();

                Point size = new Point();
                display.getSize(size);

                int width = size.x;
                int height = size.y;
            \end{javacode}
        \item \textbf{Getting Real Screen Size}: To include everything (status bar, navigation bar):
            \bigbreak \noindent 
            \begin{javacode}
                Point realSize = new Point();
                display.getRealSize(realSize);
            \end{javacode}
        \item \textbf{Getting Refresh Rate}: Returns how many times per second the screen updates:
            \bigbreak \noindent 
            \begin{javacode}
                float refreshRate = display.getRefreshRate();
            \end{javacode}
        \item \textbf{Getting Display Rotation}: Tells you how the screen is currently rotated relative to its "natural" orientation:
            \bigbreak \noindent 
            \begin{javacode}
            int rotation = display.getRotation();
            \end{javacode}
            \bigbreak \noindent 
            where the possible values are
            \begin{itemize}
                \item \textbf{Surface.ROTATION\_0}:  natural orientation
                \item \textbf{Surface.ROTATION\_90}:  rotated right
                \item \textbf{Surface.ROTATION\_180}:  upside down
                \item \textbf{Surface.ROTATION\_270}:  rotated left
            \end{itemize}
        \item \textbf{Getting Display Metrics}: To obtain screen density and scaling info:
            \bigbreak \noindent 
            \begin{javacode}
                DisplayMetrics metrics = new DisplayMetrics();
                display.getMetrics(metrics);

                int densityDpi = metrics.densityDpi;
                float density = metrics.density;  // Scale factor for dp → px
                float scaledDensity = metrics.scaledDensity;  // Scale for sp → px
            \end{javacode}
        \item \textbf{In Multi-Display or External Display Scenarios}: Starting from Android 4.2+, a device can have multiple displays (like casting to a TV or projector). You can access all of them with:
            \bigbreak \noindent 
            \begin{javacode}
                DisplayManager dm = (DisplayManager) getSystemService(Context.DISPLAY_SERVICE);
                Display[] displays = dm.getDisplays();
            \end{javacode}
        \item \textbf{Methods}:
            \begin{itemize}
                \item \textbf{getSize(Point)}:	Gets the app-usable screen size (in pixels).
                \item \textbf{getRealSize(Point)}:	Gets the full physical display size.
                \item \textbf{getRotation()}:	Returns the screen rotation (0, 90, 180, 270).
                \item \textbf{getRefreshRate()}:	Returns display refresh rate in Hz.
                \item \textbf{getMetrics(DisplayMetrics)}:	Returns logical density and scaling info.
                \item \textbf{getName()}:	Returns display name (useful in multi-display setups).
            \end{itemize}
    \end{itemize}

    % \pagebreak 
    % \subsubsection{android.hardware.display.DisplayManager}
    % \begin{itemize}
    %     
    % \end{itemize}

    % \pagebreak 
    % \subsubsection{android.view.WindowManager}
    % \begin{itemize}
    %     
    % \end{itemize}

    % \pagebreak 
    % \subsubsection{android.util.DisplayMetrics}
    % \begin{itemize}
    %     
    % \end{itemize}

    \pagebreak 
    \subsubsection{android.view.KeyEvent}
    \begin{itemize}
        \item \textbf{KeyEvent}: \texttt{android.view.KeyEvent} represents a hardware key press or release event — like when the user presses or releases a key on the device’s keyboard, a game controller, or a button such as Volume Up, Back, or Enter.
            \bigbreak \noindent 
            KeyEvent objects are delivered to your app whenever a key action happens.
            \bigbreak \noindent 
            They describe:
            \begin{itemize}
                \item Which key was pressed (keyCode)
                \item Whether it was a press or release
                \item The time, source, modifiers (Shift, Ctrl, etc.)
                \item The Unicode character it represents (if any)
            \end{itemize}
        \item \textbf{Lifecycle}
            \begin{center}
                \begin{tabular}{p{4cm}|p{4cm}|p{4cm}}
                    Stage	&Event Type	&Constant \\
                    \hline \\[0.01cm]
                    Key pressed down	&ACTION\_DOWN	&KeyEvent.ACTION\_DOWN \\[2ex]
                    Key released	&ACTION\_UP	&KeyEvent.ACTION\_UP \\[2ex]
                    Key held (repeats)	&multiple ACTION\_DOWN events	&with getRepeatCount() > 0
                \end{tabular}
            \end{center}
        \item \textbf{Getting a KeyEvent object}: In Android, you don’t manually create KeyEvent objects in most cases. Instead, the Android framework automatically provides them to your app when a user presses or releases a hardware or software key.
            \bigbreak \noindent 
            When a key is pressed, the system calls your Activity, View, or Dialog methods and passes a KeyEvent object as a parameter.
            \begin{itemize}
                \item In an Activity:
                    \bigbreak \noindent 
                    \begin{javacode}
                        @Override
                        public boolean onKeyDown(int keyCode, KeyEvent event) {
                            Log.d("KeyEvent", "Pressed key: " + event.getKeyCode());
                            return super.onKeyDown(keyCode, event);
                        }

                        @Override
                        public boolean onKeyUp(int keyCode, KeyEvent event) {
                            Log.d("KeyEvent", "Released: " + event.getKeyCode());
                            return true;
                        }
                    \end{javacode}
                    \bigbreak \noindent 
                    Here, the system creates and passes the KeyEvent object automatically.
                \item In a View:
                    \bigbreak \noindent 
                    \begin{javacode}
                        @Override
                        public boolean onKeyDown(int keyCode, KeyEvent event) {
                            // Handle key press inside your custom view
                            return true;
                        }
                    \end{javacode}
                    \bigbreak \noindent 
                    Or use a listener:
                    \bigbreak \noindent 
                    \begin{javacode}
                        view.setOnKeyListener(new View.OnKeyListener() {
                            @Override
                            public boolean onKey(View v, int keyCode, KeyEvent event) {
                                if (event.getAction() == KeyEvent.ACTION_DOWN) {
                                    Log.d("KeyEvent", "Key pressed: " + event.getKeyCode());
                                    return true;
                                }
                                return false;
                            }
                        });
                    \end{javacode}

            \end{itemize}
            \bigbreak \noindent 
            \textbf{Note:}  onKeyDown() and onKeyUp() return a boolean because the return value tells the Android framework whether your code has consumed (handled) the key event or no
            \bigbreak \noindent 
            When you return true, it means "I've handled this key event — don’t send it anywhere else."
            \bigbreak \noindent 
            Returning false means: "I didn't handle this — let the system or another component handle it." Then Android passes the event along:
            \begin{itemize}
                \item From the current View up to its parent
                \item From the Activity to the Window
                \item Or eventually to the system (for default behavior)
            \end{itemize}
            super.onKeyDown() calls the default handler in the base Activity class, which performs standard Android behaviors (like handling BACK or MENU keys).
        \item \textbf{Getting the action}
            \bigbreak \noindent 
            \begin{javacode}
            MotionEvent event.getAction()
            \end{javacode}
        \item \textbf{Action constants}:
            \begin{itemize}
                \item \textbf{ACTION\_DOWN}: Key was pressed down
                \item \textbf{ACTION\_UP}: Key was released
                \item \textbf{ACTION\_MULTIPLE}: Multiple repeated key events (e.g., long press)
            \end{itemize}
            \bigbreak \noindent 
            For example,
            \bigbreak \noindent 
            \begin{javacode}
            if (event.getAction() == KeyEvent.ACTION_DOWN) { ... }
            \end{javacode}
        \item \textbf{Getting the key code}:
            \bigbreak \noindent 
            \begin{javacode}
            int event.getKeyCode();
            \end{javacode}
        \item \textbf{Key Code Constants}: These tell you which key was pressed.
            \bigbreak \noindent 
            There are hundreds of these - a few common groups:
            \bigbreak \noindent 
            \begin{javacode}
                KEYCODE_A, KEYCODE_B, ..., KEYCODE_Z
                KEYCODE_0, KEYCODE_1, ..., KEYCODE_9
            \end{javacode}
            \bigbreak \noindent 
            \begin{javacode}
                KEYCODE_ENTER
                KEYCODE_DEL           // Backspace
                KEYCODE_TAB
                KEYCODE_ESCAPE
                KEYCODE_SPACE
                KEYCODE_BACK
                KEYCODE_MENU
                KEYCODE_HOME
            \end{javacode}
            \bigbreak \noindent 
            \begin{javacode}
                KEYCODE_VOLUME_UP
                KEYCODE_VOLUME_DOWN
                KEYCODE_MUTE
                KEYCODE_MEDIA_PLAY_PAUSE
                KEYCODE_MEDIA_NEXT
                KEYCODE_MEDIA_PREVIOUS
            \end{javacode}
            \bigbreak \noindent 
            \begin{javacode}
                KEYCODE_DPAD_UP
                KEYCODE_DPAD_DOWN
                KEYCODE_DPAD_LEFT
                KEYCODE_DPAD_RIGHT
                KEYCODE_BUTTON_A
                KEYCODE_BUTTON_B
            \end{javacode}
            \bigbreak \noindent 
            \begin{javacode}
                KEYCODE_POWER
                KEYCODE_SLEEP
                KEYCODE_WAKEUP
            \end{javacode}
        \item \textbf{Checking if shift was pressed}:
            \bigbreak \noindent 
            \begin{javacode}
            boolean event.isShiftPressed()
            \end{javacode}
        \item \textbf{Getting meta state}
            \bigbreak \noindent 
            \begin{javacode}
            event.getMetaState()
            \end{javacode}
        \item \textbf{Meta / modifier key flags}: Used for Shift, Alt, Ctrl, etc. These can be combined using bitwise OR (|).
            \begin{itemize}
                \item \textbf{META\_SHIFT\_ON}:	Shift key active
                \item \textbf{META\_ALT\_ON}:	Alt key active
                \item \textbf{META\_CTRL\_ON}:	Control key active
                \item \textbf{META\_META\_ON}:	Meta/Command key active
                \item \textbf{META\_SYM\_ON}:	Symbol modifier active
                \item \textbf{META\_CAPS\_LOCK\_ON}:	Caps lock active
                \item \textbf{META\_NUM\_LOCK\_ON}:	Num lock active
                \item \textbf{META\_SCROLL\_LOCK\_ON}:	Scroll lock active
            \end{itemize}
        \item \textbf{KeyEvent Methods}
            \begin{itemize}
                \item \textbf{getAction()}:	int	Down, Up, or Multiple
                \item \textbf{getKeyCode()}:	int	Which key was pressed
                \item \textbf{getMetaState()}:	int	Modifier flags
                \item \textbf{getRepeatCount()}:	int	How many times repeated
                \item \textbf{getEventTime()}:	long	Time when event occurred
                \item \textbf{getDownTime()}:	long	Time when key was first pressed
                \item \textbf{getDeviceId()}:	int	ID of the input device (keyboard/gamepad)
                \item \textbf{getScanCode()}:	int	Raw hardware scan code
                \item \textbf{getUnicodeChar()}:	int	Unicode value (e.g., 'A' → 65)
                \item \textbf{getFlags()}:	int	Internal system flags
                \item \textbf{getSource()}:	int	Input source (keyboard, gamepad, etc.)
                \item \textbf{isShiftPressed()}:	boolean	True if Shift active
                \item \textbf{isCtrlPressed()}:	boolean	True if Ctrl active
                \item \textbf{isAltPressed()}:	boolean	True if Alt active 
            \end{itemize}
        \item \textbf{Other useful constants}
            \begin{itemize}
                \item \textbf{Action Constants}:	ACTION\_DOWN, ACTION\_UP
                \item \textbf{Key Codes}:	KEYCODE\_A, KEYCODE\_ENTER, KEYCODE\_BACK
                \item \textbf{Meta Flags}:	META\_SHIFT\_ON, META\_CTRL\_ON
                \item \textbf{Event Data Members}:	getAction(), getKeyCode(), getDownTime(), getRepeatCount(), etc.
                \item \textbf{Flags}:	FLAG\_LONG\_PRESS, FLAG\_SOFT\_KEYBOARD, etc.
            \end{itemize}




    \end{itemize}

    \pagebreak 
    \subsection{Java Documentation}
    \subsubsection{View}
    \begin{itemize}
        \item \textbf{Hierarchy} 
            \begin{center}
                java.lang.Object $\to$	android.view.View
            \end{center}
        \item \textbf{Include}
            \bigbreak \noindent 
            \begin{javacode}
                android.view.View
            \end{javacode}
        \item \textbf{Constructors}
            \bigbreak \noindent 
            \begin{javacode}
                View(Context context)
                View(Context context, AttributeSet attrs)
                View(Context context, AttributeSet attrs, int defStyleAttr)
                View(Context context, AttributeSet attrs, int defStyleAttr, int defStyleRes)
            \end{javacode}

        \item \textbf{Public methods (Only most important)}
            \begin{itemize}
                % --- Core View Functionality ---
                \item \textbf{void setId(int id)}: Sets the unique identifier for the view.
                \item \textbf{void setVisibility(int visibility)}: Sets whether the view is visible, invisible, or gone.
                \item \textbf{int getVisibility()}: Returns the current visibility state.
                \item \textbf{void setEnabled(boolean enabled)}: Enables or disables user interaction.
                \item \textbf{boolean isEnabled()}: Returns whether the view is currently enabled.
                \item \textbf{void setFocusable(boolean focusable)}: Controls whether the view can gain focus.
                \item \textbf{void requestFocus()}: Requests focus for this view.
                \item \textbf{boolean hasFocus()}: Returns true if this view currently has focus.
                \item \textbf{void invalidate()}: Redraws the view on screen.
                \item \textbf{void requestLayout()}: Requests a new layout pass for this view.
                \item \textbf{void layout(int l, int t, int r, int b)}: Assigns size and position to the view.

                    % --- Event Handling ---
                \item \textbf{void setOnClickListener(View.OnClickListener l)}: Sets a callback to handle click events.
                \item \textbf{void setOnLongClickListener(View.OnLongClickListener l)}: Sets a listener for long press events.
                \item \textbf{boolean performClick()}: Programmatically triggers the click listener.
                \item \textbf{boolean onTouchEvent(MotionEvent event)}: Handles touch interactions.
                \item \textbf{boolean onKeyDown(int keyCode, KeyEvent event)}: Called when a hardware key is pressed.
                \item \textbf{void onDraw(Canvas canvas)}: Called to render the view’s visual content.

                    % --- Layout & Measurement ---
                \item \textbf{void measure(int widthMeasureSpec, int heightMeasureSpec)}: Determines the measured size of the view.
                \item \textbf{int getWidth() / int getHeight()}: Return the current dimensions of the view.
                \item \textbf{int getLeft() / getTop() / getRight() / getBottom()}: Return the view’s position relative to its parent.
                \item \textbf{ViewGroup.LayoutParams getLayoutParams()}: Returns layout parameters assigned to this view.
                \item \textbf{void setLayoutParams(ViewGroup.LayoutParams params)}: Updates the layout parameters.

                    % --- Animation & Position ---
                \item \textbf{ViewPropertyAnimator animate()}: Starts an animation for view properties (translation, alpha, rotation, etc.).
                \item \textbf{void setAlpha(float alpha)}: Sets the transparency (0.0 = fully transparent, 1.0 = opaque).
                \item \textbf{void setTranslationX(float translationX)}: Moves the view horizontally relative to its position.
                \item \textbf{void setTranslationY(float translationY)}: Moves the view vertically relative to its position.
                \item \textbf{void setRotation(float rotation)}: Rotates the view around its pivot point.
                \item \textbf{void setScaleX(float scaleX) / setScaleY(float scaleY)}: Scales the view’s size in X or Y direction.

                    % --- Focus and Interaction ---
                \item \textbf{void setClickable(boolean clickable)}: Enables or disables clickability.
                \item \textbf{void setLongClickable(boolean longClickable)}: Enables long-click behavior.
                \item \textbf{boolean isClickable()}: Returns whether the view handles clicks.
                \item \textbf{boolean isLongClickable()}: Returns whether the view handles long clicks.
                \item \textbf{void setPressed(boolean pressed)}: Sets the pressed state for visual feedback.
                \item \textbf{boolean isPressed()}: Returns whether the view is currently pressed.

                    % --- Accessibility & Description ---
                \item \textbf{void setContentDescription(CharSequence contentDescription)}: Sets a description for accessibility tools.
                \item \textbf{CharSequence getContentDescription()}: Returns the view’s accessibility description.
                \item \textbf{void announceForAccessibility(CharSequence text)}: Announces a message for accessibility services.
            \end{itemize}
        \item \textbf{Protected methods}
            \begin{itemize}
                % --- Scrollbars and Drawing ---
                \item \textbf{boolean awakenScrollBars(int startDelay, boolean invalidate)}:  
                    Trigger the scrollbars to draw.

                \item \textbf{boolean awakenScrollBars(int startDelay)}:  
                    Trigger the scrollbars to draw.

                \item \textbf{boolean awakenScrollBars()}:  
                    Trigger the scrollbars to draw.

                \item \textbf{int computeHorizontalScrollExtent()}:  
                    Compute the horizontal extent of the scrollbar thumb within its range.

                \item \textbf{int computeHorizontalScrollOffset()}:  
                    Compute the horizontal offset of the scrollbar thumb within its range.

                \item \textbf{int computeHorizontalScrollRange()}:  
                    Compute the horizontal scrollable range.

                \item \textbf{int computeVerticalScrollExtent()}:  
                    Compute the vertical extent of the scrollbar thumb within its range.

                \item \textbf{int computeVerticalScrollOffset()}:  
                    Compute the vertical offset of the scrollbar thumb within its range.

                \item \textbf{int computeVerticalScrollRange()}:  
                    Compute the vertical scrollable range.

                \item \textbf{void dispatchDraw(Canvas canvas)}:  
                    Called by \texttt{draw()} to draw child views.

                    % --- Event Dispatch ---
                \item \textbf{boolean dispatchGenericFocusedEvent(MotionEvent event)}:  
                    Dispatch a generic motion event to the currently focused view.

                \item \textbf{boolean dispatchGenericPointerEvent(MotionEvent event)}:  
                    Dispatch a generic motion event to the view under the first pointer.

                \item \textbf{boolean dispatchHoverEvent(MotionEvent event)}:  
                    Dispatch a hover event.

                \item \textbf{void dispatchRestoreInstanceState(SparseArray<Parcelable> container)}:  
                    Restores the state for this view and its children.

                \item \textbf{void dispatchSaveInstanceState(SparseArray<Parcelable> container)}:  
                    Saves the state for this view and its children.

                \item \textbf{void dispatchSetActivated(boolean activated)}:  
                    Dispatches activation to all of this view’s children.

                \item \textbf{void dispatchSetPressed(boolean pressed)}:  
                    Dispatches pressed state to all of this view’s children.

                \item \textbf{void dispatchSetSelected(boolean selected)}:  
                    Dispatches selected state to all of this view’s children.

                \item \textbf{void dispatchVisibilityChanged(View changedView, int visibility)}:  
                    Propagates a visibility change down the hierarchy.

                    % --- Drawable and State ---
                \item \textbf{void drawableStateChanged()}:  
                    Called whenever the view’s state changes in a way that affects its drawables.

                \item \textbf{boolean fitSystemWindows(Rect insets)}:  
                    (Deprecated) Apply window insets to adjust for system decorations.

                    % --- Fading Edges and Padding Offsets ---
                \item \textbf{float getBottomFadingEdgeStrength()}:  
                    Returns the intensity of the bottom faded edge.

                \item \textbf{int getBottomPaddingOffset()}:  
                    Amount by which to extend the bottom fading region.

                \item \textbf{float getLeftFadingEdgeStrength()}:  
                    Returns the intensity of the left faded edge.

                \item \textbf{int getLeftPaddingOffset()}:  
                    Amount by which to extend the left fading region.

                \item \textbf{float getRightFadingEdgeStrength()}:  
                    Returns the intensity of the right faded edge.

                \item \textbf{int getRightPaddingOffset()}:  
                    Amount by which to extend the right fading region.

                \item \textbf{float getTopFadingEdgeStrength()}:  
                    Returns the intensity of the top faded edge.

                \item \textbf{int getTopPaddingOffset()}:  
                    Amount by which to extend the top fading region.

                \item \textbf{boolean isPaddingOffsetRequired()}:  
                    Returns true if this view draws inside its padding and requires offset support.

                    % --- Scroll, Layout, and Sizing ---
                \item \textbf{int getSuggestedMinimumHeight()}:  
                    Returns the suggested minimum height for this view.

                \item \textbf{int getSuggestedMinimumWidth()}:  
                    Returns the suggested minimum width for this view.

                \item \textbf{boolean overScrollBy(int deltaX, int deltaY, int scrollX, int scrollY, int scrollRangeX, int scrollRangeY, int maxOverScrollX, int maxOverScrollY, boolean isTouchEvent)}:  
                    Scrolls the view with standard over-scroll behavior.

                \item \textbf{void onOverScrolled(int scrollX, int scrollY, boolean clampedX, boolean clampedY)}:  
                    Responds to an over-scroll operation.

                \item \textbf{void onScrollChanged(int l, int t, int oldl, int oldt)}:  
                    Called when the view scrolls its own content.

                    % --- Window and Configuration ---
                \item \textbf{void onAttachedToWindow()}:  
                    Called when the view is attached to a window.

                \item \textbf{void onDetachedFromWindow()}:  Called when the view is detached from a window.
                \item \textbf{void onConfigurationChanged(Configuration newConfig)}:  Called when the app configuration changes (e.g., orientation).
                \item \textbf{void onDisplayHint(int hint)}:  Receives a hint about whether the view is displayed or not.
                \item \textbf{int getWindowAttachCount()}:  Returns how many times this view has been attached to a window.
                    % --- Layout and Measurement ---
                \item \textbf{void onLayout(boolean changed, int left, int top, int right, int bottom)}:  
                    Called when assigning size and position to child views.

                \item \textbf{void onMeasure(int widthMeasureSpec, int heightMeasureSpec)}:  
                    Measures the view’s width and height.

                \item \textbf{final void setMeasuredDimension(int measuredWidth, int measuredHeight)}:  
                    Must be called in \texttt{onMeasure()} to store measured dimensions.

                \item \textbf{void onSizeChanged(int w, int h, int oldw, int oldh)}:  
                    Called when the size of this view changes during layout.

                    % --- Drawing and Animation ---
                \item \textbf{void onDraw(Canvas canvas)}:  
                    Implement this to perform custom drawing.

                \item \textbf{final void onDrawScrollBars(Canvas canvas)}:  
                    Draws horizontal and vertical scrollbars.

                \item \textbf{void onAnimationStart()}:  
                    Called when an animation starts.

                \item \textbf{void onAnimationEnd()}:  
                    Called when an animation ends.

                \item \textbf{boolean onSetAlpha(int alpha)}:  
                    Called when a transform involving alpha occurs.

                    % --- Context and Menu ---
                \item \textbf{void onCreateContextMenu(ContextMenu menu)}:  
                    Implement if the view contributes items to a context menu.

                \item \textbf{ContextMenu.ContextMenuInfo getContextMenuInfo()}:  
                    Returns extra info associated with the context menu.

                    % --- Drawable States and Verification ---
                \item \textbf{int[] onCreateDrawableState(int extraSpace)}:  
                    Generates the new drawable state array for this view.

                \item \textbf{boolean verifyDrawable(Drawable who)}:  
                    Override if the view displays custom drawables; return true for those drawables.

                \item \textbf{static int[] mergeDrawableStates(int[] baseState, int[] additionalState)}:  
                    Merges additional drawable states into the base state.

                    % --- Focus and Visibility ---
                \item \textbf{void onFocusChanged(boolean gainFocus, int direction, Rect previouslyFocusedRect)}:  
                    Called when the focus state of this view changes.

                \item \textbf{void onVisibilityChanged(View changedView, int visibility)}:  
                    Called when the visibility of this view or its ancestor changes.

                \item \textbf{void onWindowVisibilityChanged(int visibility)}:  
                    Called when the containing window’s visibility changes (GONE, INVISIBLE, or VISIBLE).

                    % --- State Saving and Restoration ---
                \item \textbf{Parcelable onSaveInstanceState()}:  
                    Generates a representation of this view’s internal state for later restoration.

                \item \textbf{void onRestoreInstanceState(Parcelable state)}:  
                    Restores the view’s internal state from a saved instance.

                \item \textbf{void dispatchRestoreInstanceState(SparseArray<Parcelable> container)}:  
                    Restores hierarchy state for this view and its children.

                \item \textbf{void dispatchSaveInstanceState(SparseArray<Parcelable> container)}:  
                    Saves hierarchy state for this view and its children.
            \end{itemize}

        \item \textbf{Fields}
            \begin{itemize}
                % --- Property Wrappers ---
                \item \textbf{public static final Property<View, Float> ALPHA}:  
                    A Property wrapper around the alpha functionality handled by the \texttt{View.setAlpha(float)} and \texttt{View.getAlpha()} methods.

                \item \textbf{public static final Property<View, Float> ROTATION}:  
                    A Property wrapper around the rotation functionality handled by \texttt{View.setRotation(float)} and \texttt{View.getRotation()}.

                \item \textbf{public static final Property<View, Float> ROTATION\_X}:  
                    A Property wrapper around the rotationX functionality handled by \texttt{View.setRotationX(float)} and \texttt{View.getRotationX()}.

                \item \textbf{public static final Property<View, Float> ROTATION\_Y}:  
                    A Property wrapper around the rotationY functionality handled by \texttt{View.setRotationY(float)} and \texttt{View.getRotationY()}.

                \item \textbf{public static final Property<View, Float> SCALE\_X}:  
                    A Property wrapper around the scaleX functionality handled by \texttt{View.setScaleX(float)} and \texttt{View.getScaleX()}.

                \item \textbf{public static final Property<View, Float> SCALE\_Y}:  
                    A Property wrapper around the scaleY functionality handled by \texttt{View.setScaleY(float)} and \texttt{View.getScaleY()}.

                \item \textbf{public static final Property<View, Float> TRANSLATION\_X}:  
                    A Property wrapper around the translationX functionality handled by \texttt{View.setTranslationX(float)} and \texttt{View.getTranslationX()}.

                \item \textbf{public static final Property<View, Float> TRANSLATION\_Y}:  
                    A Property wrapper around the translationY functionality handled by \texttt{View.setTranslationY(float)} and \texttt{View.getTranslationY()}.

                \item \textbf{public static final Property<View, Float> TRANSLATION\_Z}:  
                    A Property wrapper around the translationZ functionality handled by \texttt{View.setTranslationZ(float)} and \texttt{View.getTranslationZ()}.

                \item \textbf{public static final Property<View, Float> X}:  
                    A Property wrapper around the x-position handled by \texttt{View.setX(float)} and \texttt{View.getX()}.

                \item \textbf{public static final Property<View, Float> Y}:  
                    A Property wrapper around the y-position handled by \texttt{View.setY(float)} and \texttt{View.getY()}.

                \item \textbf{public static final Property<View, Float> Z}:  
                    A Property wrapper around the z-position handled by \texttt{View.setZ(float)} and \texttt{View.getZ()}.

                    % --- View State Sets ---
                \item \textbf{protected static final int[] EMPTY\_STATE\_SET}:  
                    Indicates the view has no states set.

                \item \textbf{protected static final int[] ENABLED\_STATE\_SET}:  
                    Indicates the view is enabled.

                \item \textbf{protected static final int[] ENABLED\_FOCUSED\_STATE\_SET}:  
                    Indicates the view is enabled and has focus.

                \item \textbf{protected static final int[] ENABLED\_SELECTED\_STATE\_SET}:  
                    Indicates the view is enabled and selected.

                \item \textbf{protected static final int[] ENABLED\_FOCUSED\_SELECTED\_STATE\_SET}:  
                    Indicates the view is enabled, focused, and selected.

                \item \textbf{protected static final int[] ENABLED\_WINDOW\_FOCUSED\_STATE\_SET}:  
                    Indicates the view is enabled and its window has focus.

                \item \textbf{protected static final int[] ENABLED\_FOCUSED\_WINDOW\_FOCUSED\_STATE\_SET}:  
                    Indicates the view is enabled, focused, and its window has focus.

                \item \textbf{protected static final int[] ENABLED\_SELECTED\_WINDOW\_FOCUSED\_STATE\_SET}:  
                    Indicates the view is enabled, selected, and its window has focus.

                \item \textbf{protected static final int[] ENABLED\_FOCUSED\_SELECTED\_WINDOW\_FOCUSED\_STATE\_SET}:  
                    Indicates the view is enabled, focused, selected, and its window has focus.

                \item \textbf{protected static final int[] FOCUSED\_STATE\_SET}:  
                    Indicates the view is focused.

                \item \textbf{protected static final int[] SELECTED\_STATE\_SET}:  
                    Indicates the view is selected.

                \item \textbf{protected static final int[] WINDOW\_FOCUSED\_STATE\_SET}:  
                    Indicates the view's window has focus.

                \item \textbf{protected static final int[] PRESSED\_STATE\_SET}:  
                    Indicates the view is pressed.

                    % --- Complex State Combinations ---
                \item \textbf{protected static final int[] PRESSED\_ENABLED\_STATE\_SET}:  
                    Indicates the view is pressed and enabled.

                \item \textbf{protected static final int[] PRESSED\_ENABLED\_FOCUSED\_STATE\_SET}:  
                    Indicates the view is pressed, enabled, and focused.

                \item \textbf{protected static final int[] PRESSED\_ENABLED\_SELECTED\_STATE\_SET}:  
                    Indicates the view is pressed, enabled, and selected.

                \item \textbf{protected static final int[] PRESSED\_ENABLED\_FOCUSED\_SELECTED\_STATE\_SET}:  
                    Indicates the view is pressed, enabled, focused, and selected.

                \item \textbf{protected static final int[] PRESSED\_FOCUSED\_STATE\_SET}:  
                    Indicates the view is pressed and focused.

                \item \textbf{protected static final int[] PRESSED\_SELECTED\_STATE\_SET}:  
                    Indicates the view is pressed and selected.

                \item \textbf{protected static final int[] FOCUSED\_SELECTED\_STATE\_SET}:  
                    Indicates the view is focused and selected.

                \item \textbf{protected static final int[] PRESSED\_ENABLED\_FOCUSED\_SELECTED\_WINDOW\_FOCUSED\_STATE\_SET}:  
                    Indicates the view is pressed, enabled, focused, selected, and its window has the focus.

                \item \textbf{protected static final int[] PRESSED\_ENABLED\_WINDOW\_FOCUSED\_STATE\_SET}:  
                    Indicates the view is pressed, enabled, and its window has focus.

                \item \textbf{protected static final int[] PRESSED\_FOCUSED\_WINDOW\_FOCUSED\_STATE\_SET}:  
                    Indicates the view is pressed, focused, and its window has focus.

                \item \textbf{protected static final int[] PRESSED\_SELECTED\_WINDOW\_FOCUSED\_STATE\_SET}:  
                    Indicates the view is pressed, selected, and its window has focus.

                \item \textbf{protected static final int[] SELECTED\_WINDOW\_FOCUSED\_STATE\_SET}:  
                    Indicates the view is selected and its window has focus.

                \item \textbf{protected static final int[] FOCUSED\_WINDOW\_FOCUSED\_STATE\_SET}:  
                    Indicates the view is focused and its window has focus.

                \item \textbf{protected static final int[] PRESSED\_FOCUSED\_SELECTED\_STATE\_SET}:  
                    Indicates the view is pressed, focused, and selected.

                \item \textbf{protected static final int[] PRESSED\_FOCUSED\_SELECTED\_WINDOW\_FOCUSED\_STATE\_SET}:  
                    Indicates the view is pressed, focused, selected, and its window has focus.
            \end{itemize}

        \item \textbf{Constants}
            \begin{itemize}
                \item \textbf{int ACCESSIBILITY\_DATA\_SENSITIVE\_AUTO}: Automatically determine whether only accessibility tools may interact with this view.
                \item \textbf{int ACCESSIBILITY\_DATA\_SENSITIVE\_NO}: Allow interactions from all AccessibilityServices.
                \item \textbf{int ACCESSIBILITY\_DATA\_SENSITIVE\_YES}: Only allow interactions from AccessibilityServices marked as tools.
                \item \textbf{int ACCESSIBILITY\_LIVE\_REGION\_ASSERTIVE}: Announce changes immediately.
                \item \textbf{int ACCESSIBILITY\_LIVE\_REGION\_NONE}: Do not automatically announce changes.
                \item \textbf{int ACCESSIBILITY\_LIVE\_REGION\_POLITE}: Announce changes politely.
                \item \textbf{int AUTOFILL\_FLAG\_INCLUDE\_NOT\_IMPORTANT\_VIEWS}: Include not-important-for-autofill views in \texttt{ViewStructure}.
                \item \textbf{String AUTOFILL\_HINT\_CREDIT\_CARD\_EXPIRATION\_DATE}: Hint for credit card expiration date.
                \item \textbf{String AUTOFILL\_HINT\_CREDIT\_CARD\_EXPIRATION\_DAY}: Hint for credit card expiration day.
                \item \textbf{String AUTOFILL\_HINT\_CREDIT\_CARD\_EXPIRATION\_MONTH}: Hint for credit card expiration month.
                \item \textbf{String AUTOFILL\_HINT\_CREDIT\_CARD\_EXPIRATION\_YEAR}: Hint for credit card expiration year.
                \item \textbf{String AUTOFILL\_HINT\_CREDIT\_CARD\_NUMBER}: Hint for credit card number.
                \item \textbf{String AUTOFILL\_HINT\_CREDIT\_CARD\_SECURITY\_CODE}: Hint for credit card CVC/CVV.
                \item \textbf{String AUTOFILL\_HINT\_EMAIL\_ADDRESS}: Hint for email address.
                \item \textbf{String AUTOFILL\_HINT\_NAME}: Hint for real name.
                \item \textbf{String AUTOFILL\_HINT\_PASSWORD}: Hint for password.
                \item \textbf{String AUTOFILL\_HINT\_PHONE}: Hint for phone number.
                \item \textbf{String AUTOFILL\_HINT\_POSTAL\_ADDRESS}: Hint for postal address.
                \item \textbf{String AUTOFILL\_HINT\_POSTAL\_CODE}: Hint for postal/ZIP code.
                \item \textbf{String AUTOFILL\_HINT\_USERNAME}: Hint for username.
                \item \textbf{int AUTOFILL\_TYPE\_DATE}: Field is a date (millis since epoch).
                \item \textbf{int AUTOFILL\_TYPE\_LIST}: Field is a selection list (int index).
                \item \textbf{int AUTOFILL\_TYPE\_NONE}: Not autofillable.
                \item \textbf{int AUTOFILL\_TYPE\_TEXT}: Field is text.
                \item \textbf{int AUTOFILL\_TYPE\_TOGGLE}: Field is boolean/toggle.
                \item \textbf{int CONTENT\_SENSITIVITY\_AUTO}: Framework determines content sensitivity.
                \item \textbf{int CONTENT\_SENSITIVITY\_NOT\_SENSITIVE}: Content not sensitive.
                \item \textbf{int CONTENT\_SENSITIVITY\_SENSITIVE}: Content is sensitive.
                \item \textbf{int DRAG\_FLAG\_ACCESSIBILITY\_ACTION}: Drag initiated via accessibility action.
                \item \textbf{int DRAG\_FLAG\_GLOBAL}: Drag can cross window boundaries.
                \item \textbf{int DRAG\_FLAG\_GLOBAL\_PERSISTABLE\_URI\_PERMISSION}: Persist granted URI permissions across reboots.
                \item \textbf{int DRAG\_FLAG\_GLOBAL\_PREFIX\_URI\_PERMISSION}: URI permission applies to prefix matches.
                \item \textbf{int DRAG\_FLAG\_GLOBAL\_SAME\_APPLICATION}: Drag can cross windows within same app.
                \item \textbf{int DRAG\_FLAG\_GLOBAL\_URI\_READ}: Recipient may request read access to URIs.
                \item \textbf{int DRAG\_FLAG\_GLOBAL\_URI\_WRITE}: Recipient may request write access to URIs.
                \item \textbf{int DRAG\_FLAG\_HIDE\_CALLING\_TASK\_ON\_DRAG\_START}: Hide caller task during drag.
                \item \textbf{int DRAG\_FLAG\_OPAQUE}: Drag shadow is opaque.
                \item \textbf{int DRAG\_FLAG\_START\_INTENT\_SENDER\_ON\_UNHANDLED\_DRAG}: Delegate unhandled drag to system to start.
                \item \textbf{int DRAWING\_CACHE\_QUALITY\_AUTO}: \textit{Deprecated}. Auto drawing cache quality.
                \item \textbf{int DRAWING\_CACHE\_QUALITY\_HIGH}: \textit{Deprecated}. High drawing cache quality.
                \item \textbf{int DRAWING\_CACHE\_QUALITY\_LOW}: \textit{Deprecated}. Low drawing cache quality.
                \item \textbf{int FIND\_VIEWS\_WITH\_CONTENT\_DESCRIPTION}: Find by content description.
                \item \textbf{int FIND\_VIEWS\_WITH\_TEXT}: Find by text.
                \item \textbf{int FOCUSABLE}: View wants keystrokes.
                \item \textbf{int FOCUSABLES\_ALL}: Add all focusables, regardless of touch mode.
                \item \textbf{int FOCUSABLES\_TOUCH\_MODE}: Add only focusables in touch mode.
                \item \textbf{int FOCUSABLE\_AUTO}: Determine focusability automatically.
                \item \textbf{int FOCUS\_BACKWARD}: Focus search backward.
                \item \textbf{int FOCUS\_DOWN}: Focus search down.
                \item \textbf{int FOCUS\_FORWARD}: Focus search forward.
                \item \textbf{int FOCUS\_LEFT}: Focus search left.
                \item \textbf{int FOCUS\_RIGHT}: Focus search right.
                \item \textbf{int FOCUS\_UP}: Focus search up.
                \item \textbf{int GONE}: View is hidden and takes no space.
                \item \textbf{int HAPTIC\_FEEDBACK\_ENABLED}: Enable haptic feedback.
                \item \textbf{int IMPORTANT\_FOR\_ACCESSIBILITY\_AUTO}: Determine importance for accessibility automatically.
                \item \textbf{int IMPORTANT\_FOR\_ACCESSIBILITY\_NO}: Not important for accessibility.
                \item \textbf{int IMPORTANT\_FOR\_ACCESSIBILITY\_NO\_HIDE\_DESCENDANTS}: Neither view nor descendants are important.
                \item \textbf{int IMPORTANT\_FOR\_ACCESSIBILITY\_YES}: Important for accessibility.
                \item \textbf{int IMPORTANT\_FOR\_AUTOFILL\_AUTO}: Determine importance for autofill automatically.
                \item \textbf{int IMPORTANT\_FOR\_AUTOFILL\_NO}: Not important for autofill; traverse children.
                \item \textbf{int IMPORTANT\_FOR\_AUTOFILL\_NO\_EXCLUDE\_DESCENDANTS}: Not important; do not traverse children.
                \item \textbf{int IMPORTANT\_FOR\_AUTOFILL\_YES}: Important; traverse children.
                \item \textbf{int IMPORTANT\_FOR\_AUTOFILL\_YES\_EXCLUDE\_DESCENDANTS}: Important; do not traverse children.
                \item \textbf{int IMPORTANT\_FOR\_CONTENT\_CAPTURE\_AUTO}: Determine importance for content capture automatically.
                \item \textbf{int IMPORTANT\_FOR\_CONTENT\_CAPTURE\_NO}: Not important; traverse children.
                \item \textbf{int IMPORTANT\_FOR\_CONTENT\_CAPTURE\_NO\_EXCLUDE\_DESCENDANTS}: Not important; exclude children.
                \item \textbf{int IMPORTANT\_FOR\_CONTENT\_CAPTURE\_YES}: Important; traverse children.
                \item \textbf{int IMPORTANT\_FOR\_CONTENT\_CAPTURE\_YES\_EXCLUDE\_DESCENDANTS}: Important; exclude children.
                \item \textbf{int INVISIBLE}: View is invisible but takes space.
                \item \textbf{int KEEP\_SCREEN\_ON}: Keep screen on while visible.
                \item \textbf{int LAYER\_TYPE\_HARDWARE}: Hardware layer.
                \item \textbf{int LAYER\_TYPE\_NONE}: No layer.
                \item \textbf{int LAYER\_TYPE\_SOFTWARE}: Software layer.
                \item \textbf{int LAYOUT\_DIRECTION\_INHERIT}: Inherit layout direction from parent.
                \item \textbf{int LAYOUT\_DIRECTION\_LOCALE}: Layout direction from locale script.
                \item \textbf{int LAYOUT\_DIRECTION\_LTR}: Left-to-right layout.
                \item \textbf{int LAYOUT\_DIRECTION\_RTL}: Right-to-left layout.
                \item \textbf{int MEASURED\_HEIGHT\_STATE\_SHIFT}: Bit shift to height state.
                \item \textbf{int MEASURED\_SIZE\_MASK}: Mask for measured size bits.
                \item \textbf{int MEASURED\_STATE\_MASK}: Mask for measured state bits.
                \item \textbf{int MEASURED\_STATE\_TOO\_SMALL}: Measured size is smaller than desired.
                \item \textbf{int NOT\_FOCUSABLE}: View does not want keystrokes.
                \item \textbf{int NO\_ID}: Marks a view with no ID.
                \item \textbf{int OVER\_SCROLL\_ALWAYS}: Always allow overscroll.
                \item \textbf{int OVER\_SCROLL\_IF\_CONTENT\_SCROLLS}: Allow overscroll only if content can scroll.
                \item \textbf{int OVER\_SCROLL\_NEVER}: Never allow overscroll.
                \item \textbf{int RECTANGLE\_ON\_SCREEN\_REQUEST\_SOURCE\_INPUT\_FOCUS}: Request due to input focus.
                \item \textbf{int RECTANGLE\_ON\_SCREEN\_REQUEST\_SOURCE\_SCROLL\_ONLY}: Request only to scroll, not tied to cursor/focus.
                \item \textbf{int RECTANGLE\_ON\_SCREEN\_REQUEST\_SOURCE\_TEXT\_CURSOR}: Request due to text cursor.
                \item \textbf{int RECTANGLE\_ON\_SCREEN\_REQUEST\_SOURCE\_UNDEFINED}: Request via legacy APIs (undefined source).
                \item \textbf{float REQUESTED\_FRAME\_RATE\_CATEGORY\_DEFAULT}: Preferred frame rate: default.
                \item \textbf{float REQUESTED\_FRAME\_RATE\_CATEGORY\_HIGH}: Preferred frame rate: high.
                \item \textbf{float REQUESTED\_FRAME\_RATE\_CATEGORY\_LOW}: Preferred frame rate: low.
                \item \textbf{float REQUESTED\_FRAME\_RATE\_CATEGORY\_NORMAL}: Preferred frame rate: normal.
                \item \textbf{float REQUESTED\_FRAME\_RATE\_CATEGORY\_NO\_PREFERENCE}: No frame rate preference.
                \item \textbf{int SCREEN\_STATE\_OFF}: Screen is off.
                \item \textbf{int SCREEN\_STATE\_ON}: Screen is on.
                \item \textbf{int SCROLLBARS\_INSIDE\_INSET}: Scrollbars inside padded area; increases padding.
                \item \textbf{int SCROLLBARS\_INSIDE\_OVERLAY}: Scrollbars inside content; no padding increase.
                \item \textbf{int SCROLLBARS\_OUTSIDE\_INSET}: Scrollbars at edge; increases padding.
                \item \textbf{int SCROLLBARS\_OUTSIDE\_OVERLAY}: Scrollbars at edge; no padding increase.
                \item \textbf{int SCROLLBAR\_POSITION\_DEFAULT}: Scrollbar at system default position.
                \item \textbf{int SCROLLBAR\_POSITION\_LEFT}: Scrollbar on left edge.
                \item \textbf{int SCROLLBAR\_POSITION\_RIGHT}: Scrollbar on right edge.
                \item \textbf{int SCROLL\_AXIS\_HORIZONTAL}: Horizontal scroll axis.
                \item \textbf{int SCROLL\_AXIS\_NONE}: No scroll axis.
                \item \textbf{int SCROLL\_AXIS\_VERTICAL}: Vertical scroll axis.
                \item \textbf{int SCROLL\_CAPTURE\_HINT\_AUTO}: Consider for scroll capture if scrollable.
                \item \textbf{int SCROLL\_CAPTURE\_HINT\_EXCLUDE}: Exclude this view from scroll capture.
                \item \textbf{int SCROLL\_CAPTURE\_HINT\_EXCLUDE\_DESCENDANTS}: Exclude descendants from scroll capture.
                \item \textbf{int SCROLL\_CAPTURE\_HINT\_INCLUDE}: Include this view for scroll capture.
                \item \textbf{int SCROLL\_INDICATOR\_BOTTOM}: Scroll indicator on bottom edge.
                \item \textbf{int SCROLL\_INDICATOR\_END}: Scroll indicator on end edge.
                \item \textbf{int SCROLL\_INDICATOR\_LEFT}: Scroll indicator on left edge.
                \item \textbf{int SCROLL\_INDICATOR\_RIGHT}: Scroll indicator on right edge.
                \item \textbf{int SCROLL\_INDICATOR\_START}: Scroll indicator on start edge.
                \item \textbf{int SCROLL\_INDICATOR\_TOP}: Scroll indicator on top edge.
                \item \textbf{int SOUND\_EFFECTS\_ENABLED}: Enable click/touch sound effects.
                \item \textbf{int STATUS\_BAR\_HIDDEN}: \textit{Deprecated}. Use low profile instead.
                \item \textbf{int STATUS\_BAR\_VISIBLE}: \textit{Deprecated}. Use \texttt{SYSTEM\_UI\_FLAG\_VISIBLE}.
                \item \textbf{int SYSTEM\_UI\_FLAG\_FULLSCREEN}: \textit{Deprecated}. Use \texttt{WindowInsetsController.hide(Type.statusBars())}.
                \item \textbf{int SYSTEM\_UI\_FLAG\_HIDE\_NAVIGATION}: \textit{Deprecated}. Use \texttt{WindowInsetsController.hide(Type.navigationBars())}.
                \item \textbf{int SYSTEM\_UI\_FLAG\_IMMERSIVE}: \textit{Deprecated}. Use default behavior.
                \item \textbf{int SYSTEM\_UI\_FLAG\_IMMERSIVE\_STICKY}: \textit{Deprecated}. Use show-transient-bars-by-swipe behavior.
                \item \textbf{int SYSTEM\_UI\_FLAG\_LAYOUT\_FULLSCREEN}: \textit{Deprecated}. See modern insets APIs.
                \item \textbf{int SYSTEM\_UI\_FLAG\_LAYOUT\_HIDE\_NAVIGATION}: \textit{Deprecated}. See modern insets APIs.
                \item \textbf{int SYSTEM\_UI\_FLAG\_LAYOUT\_STABLE}: \textit{Deprecated}. Use \texttt{WindowInsets.getInsetsIgnoringVisibility}.
                \item \textbf{int SYSTEM\_UI\_FLAG\_LIGHT\_NAVIGATION\_BAR}: \textit{Deprecated}. Use appearance flags.
                \item \textbf{int SYSTEM\_UI\_FLAG\_LIGHT\_STATUS\_BAR}: \textit{Deprecated}. Use appearance flags.
                \item \textbf{int SYSTEM\_UI\_FLAG\_LOW\_PROFILE}: \textit{Deprecated}. Hide system bars instead.
                \item \textbf{int SYSTEM\_UI\_FLAG\_VISIBLE}: \textit{Deprecated}. Use \texttt{WindowInsetsController}.
                \item \textbf{int SYSTEM\_UI\_LAYOUT\_FLAGS}: \textit{Deprecated}. System UI layout flags deprecated.
                \item \textbf{int TEXT\_ALIGNMENT\_CENTER}: Center paragraph alignment.
                \item \textbf{int TEXT\_ALIGNMENT\_GRAVITY}: Default for root view (gravity).
                \item \textbf{int TEXT\_ALIGNMENT\_INHERIT}: Inherit text alignment.
                \item \textbf{int TEXT\_ALIGNMENT\_TEXT\_END}: Align to paragraph end.
                \item \textbf{int TEXT\_ALIGNMENT\_TEXT\_START}: Align to paragraph start.
                \item \textbf{int TEXT\_ALIGNMENT\_VIEW\_END}: Align to view end (RTL-aware).
                \item \textbf{int TEXT\_ALIGNMENT\_VIEW\_START}: Align to view start (RTL-aware).
                \item \textbf{int TEXT\_DIRECTION\_ANY\_RTL}: Any-RTL algorithm.
                \item \textbf{int TEXT\_DIRECTION\_FIRST\_STRONG}: First-strong algorithm.
                \item \textbf{int TEXT\_DIRECTION\_FIRST\_STRONG\_LTR}: First-strong, force LTR.
                \item \textbf{int TEXT\_DIRECTION\_FIRST\_STRONG\_RTL}: First-strong, force RTL.
                \item \textbf{int TEXT\_DIRECTION\_INHERIT}: Inherit text direction.
                \item \textbf{int TEXT\_DIRECTION\_LOCALE}: From system locale.
                \item \textbf{int TEXT\_DIRECTION\_LTR}: Force LTR.
                \item \textbf{int TEXT\_DIRECTION\_RTL}: Force RTL.
                \item \textbf{String VIEW\_LOG\_TAG}: Logging tag for this class.
                \item \textbf{int VISIBLE}: View is visible.
            \end{itemize}
    \end{itemize}

    \pagebreak 
    \subsubsection{ViewGroup}
    \begin{itemize}
        \item \textbf{Hierarchy} 
            \begin{center}
                java.lang.Object $\to$	android.view.View $\to$	android.view.ViewGroup 
            \end{center}
        \item \textbf{Include}
            \bigbreak \noindent 
            \begin{javacode}
                android.view.ViewGroup 
            \end{javacode}
        \item \textbf{Constructors}
            \bigbreak \noindent 
            \begin{javacode}
                ViewGroup(Context context)
                ViewGroup(Context context, AttributeSet attrs)
                ViewGroup(Context context, AttributeSet attrs, int defStyleAttr)
                ViewGroup(Context context, AttributeSet attrs, int defStyleAttr, int defStyleRes)
            \end{javacode}
        \item \textbf{Public methods}
            \begin{itemize}
                \item \textbf{void addView(View child)}: Adds a child view to this ViewGroup.
                \item \textbf{void addView(View child, int index)}: Inserts a child view at a specific position.
                \item \textbf{void addView(View child, ViewGroup.LayoutParams params)}: Adds a child with explicit layout params.
                \item \textbf{void addView(View child, int index, ViewGroup.LayoutParams params)}: Inserts a child at a position with params.
                \item \textbf{void addView(View child, int width, int height)}: Adds a child using default params plus width/height.

                \item \textbf{void removeView(View view)}: Removes the specified child view.
                \item \textbf{void removeViewAt(int index)}: Removes the child at the given index.
                \item \textbf{void removeViews(int start, int count)}: Removes a range of children.
                \item \textbf{void removeAllViews()} : Removes all child views.

                \item \textbf{int getChildCount()}: Returns the number of children in this ViewGroup.
                \item \textbf{View getChildAt(int index)}: Returns the child at the specified index.
                \item \textbf{int indexOfChild(View child)}: Returns this child's index within the group.
                \item \textbf{void bringChildToFront(View child)}: Moves a child to the top of the Z-order.

                \item \textbf{boolean dispatchTouchEvent(MotionEvent ev)}: Dispatches touch events down the hierarchy.
                \item \textbf{boolean onInterceptTouchEvent(MotionEvent ev)}: Intercepts touch events before children (gesture handling).
                \item \textbf{void requestDisallowInterceptTouchEvent(boolean disallow)}: Child requests parent not to intercept touch.

                \item \textbf{final void layout(int l, int t, int r, int b)}: Assigns size/position to this view and descendants.
                \item \textbf{static int getChildMeasureSpec(int spec, int padding, int childDimension)}: Computes a child’s MeasureSpec.
                \item \textbf{ViewGroup.LayoutParams generateLayoutParams(AttributeSet attrs)}: Creates layout params from XML.

                \item \textbf{void setClipToPadding(boolean clipToPadding)}: Controls clipping of children within padding area.
                \item \textbf{void setClipChildren(boolean clipChildren)}: Controls whether children are clipped to this ViewGroup’s bounds.
                \item \textbf{int getDescendantFocusability()} : Gets how focus is handled among descendants.
                \item \textbf{void setDescendantFocusability(int focusability)}: Sets focus behavior for descendants.
                \item \textbf{View getFocusedChild()} : Returns the currently focused child, if any.
                \item \textbf{View focusSearch(View focused, int direction)}: Finds the next focusable view in a direction.
                \item \textbf{void requestChildFocus(View child, View focused)}: Notifies parent that a child wants focus.

                \item \textbf{void setOnHierarchyChangeListener(ViewGroup.OnHierarchyChangeListener l)}: Listens for child add/remove events.
                \item \textbf{void setLayoutTransition(LayoutTransition transition)}: Animates child appearance/disappearance/changes.
                \item \textbf{void suppressLayout(boolean suppress)}: Temporarily defers layout passes for batched changes.
                \item \textbf{void updateViewLayout(View view, ViewGroup.LayoutParams params)}: Updates layout params for an existing child.
            \end{itemize}

        \item \textbf{Protected methods}
            \begin{itemize}
                \item \textbf{boolean addViewInLayout(View child, int index, ViewGroup.LayoutParams params, boolean preventRequestLayout)}: Adds a view during layout, optionally preventing a layout request.
                \item \textbf{boolean addViewInLayout(View child, int index, ViewGroup.LayoutParams params)}: Adds a view during layout.
                \item \textbf{void attachLayoutAnimationParameters(View child, ViewGroup.LayoutParams params, int index, int count)}: For subclasses to set layout animation parameters on the child.
                \item \textbf{void attachViewToParent(View child, int index, ViewGroup.LayoutParams params)}: Attaches a view to this ViewGroup.
                \item \textbf{boolean canAnimate()}: Returns whether this ViewGroup can animate its children after first layout.
                \item \textbf{boolean checkLayoutParams(ViewGroup.LayoutParams p)}: Checks if the given LayoutParams are valid for this ViewGroup.
                \item \textbf{void cleanupLayoutState(View child)}: Prevents the specified child from being laid out during the next layout pass.
                \item \textbf{void debug(int depth)}: Outputs debug information about this view hierarchy.
                \item \textbf{void detachAllViewsFromParent()}: Detaches all views from the parent.
                \item \textbf{void detachViewFromParent(int index)}: Detaches the child at the given index from its parent.
                \item \textbf{void detachViewFromParent(View child)}: Detaches the specified child from its parent.
                \item \textbf{void detachViewsFromParent(int start, int count)}: Detaches a range of children from their parent.
                \item \textbf{void dispatchDraw(Canvas canvas)}: Called by \texttt{draw} to draw the child views.
                \item \textbf{void dispatchFreezeSelfOnly(SparseArray<Parcelable> container)}: Saves state only for this view (not its children).
                \item \textbf{boolean dispatchGenericFocusedEvent(MotionEvent event)}: Dispatches a generic motion event to the currently focused view.
                \item \textbf{boolean dispatchGenericPointerEvent(MotionEvent event)}: Dispatches a generic motion event to the view under the first pointer.
                \item \textbf{boolean dispatchHoverEvent(MotionEvent event)}: Dispatches a hover event.
                \item \textbf{void dispatchRestoreInstanceState(SparseArray<Parcelable> container)}: Restores state for this view and its children.
                \item \textbf{void dispatchSaveInstanceState(SparseArray<Parcelable> container)}: Saves state for this view and its children.
                \item \textbf{void dispatchSetPressed(boolean pressed)}: Propagates the pressed state to all children.
                \item \textbf{void dispatchThawSelfOnly(SparseArray<Parcelable> container)}: Restores state only for this view (not its children).
                \item \textbf{void dispatchVisibilityChanged(View changedView, int visibility)}: Dispatches visibility changes down the hierarchy.
                \item \textbf{boolean drawChild(Canvas canvas, View child, long drawingTime)}: Draws a single child of this ViewGroup.
                \item \textbf{void drawableStateChanged()}: Called when the view's state changes in a way that affects shown drawables.
                \item \textbf{ViewGroup.LayoutParams generateDefaultLayoutParams()}: Returns a set of default layout parameters.
                \item \textbf{ViewGroup.LayoutParams generateLayoutParams(ViewGroup.LayoutParams p)}: Returns a safe set of layout params based on the supplied params.
                \item \textbf{int getChildDrawingOrder(int childCount, int drawingPosition)}: Maps drawing-order position to container position.
                \item \textbf{boolean getChildStaticTransformation(View child, Transformation t)}: Sets \texttt{t} to the child's static transform if present; returns true if set.
                \item \textbf{boolean isChildrenDrawingOrderEnabled()}: Returns whether children are drawn in the order from \texttt{getChildDrawingOrder}.
                \item \textbf{boolean isChildrenDrawnWithCacheEnabled()}:\footnotesize~Deprecated (API 23). Child caching forced by parents is ignored; use \texttt{View.setLayerType}. \normalsize
                \item \textbf{void measureChild(View child, int parentWidthMeasureSpec, int parentHeightMeasureSpec)}: Measures a child with this ViewGroup's specs and padding.
                \item \textbf{void measureChildWithMargins(View child, int parentWidthMeasureSpec, int widthUsed, int parentHeightMeasureSpec, int heightUsed)}: Measures a child accounting for margins and used space.
                \item \textbf{void measureChildren(int widthMeasureSpec, int heightMeasureSpec)}: Measures all children with the given specs and padding.
                \item \textbf{void onAttachedToWindow()} : Called when the view is attached to a window.
                \item \textbf{int[] onCreateDrawableState(int extraSpace)}: Generates the Drawable state for this view.
                \item \textbf{void onDetachedFromWindow()} : Called when the view is detached from a window.
                \item \textbf{abstract void onLayout(boolean changed, int l, int t, int r, int b)}: Assigns size and position to each child (subclasses must implement).
                \item \textbf{boolean onRequestFocusInDescendants(int direction, Rect previouslyFocusedRect)}: Requests focus on a suitable descendant.
                \item \textbf{void removeDetachedView(View child, boolean animate)}: Finishes removing a detached view, optionally with animation.
                \item \textbf{void setChildrenDrawingCacheEnabled(boolean enabled)}:\footnotesize~Deprecated (API 28). View drawing cache largely obsolete with hardware acceleration; prefer \texttt{View.setLayerType} or \texttt{PixelCopy} for screenshots. \normalsize
                \item \textbf{void setChildrenDrawingOrderEnabled(boolean enabled)}: Controls whether children are drawn using custom drawing order.
                \item \textbf{void setChildrenDrawnWithCacheEnabled(boolean enabled)}:\footnotesize~Deprecated (API 23). Forcing child render caching is ignored; use \texttt{View.setLayerType}. \normalsize
                \item \textbf{void setStaticTransformationsEnabled(boolean enabled)}: Enables static child transformations (invokes \texttt{getChildStaticTransformation} during draw).
            \end{itemize}

        \item \textbf{Constants}
            \begin{itemize}
                \item \textbf{int CLIP\_TO\_PADDING\_MASK}: Clips to padding when both \texttt{FLAG\_CLIP\_TO\_PADDING} and \texttt{FLAG\_PADDING\_NOT\_NULL} are set.
                \item \textbf{int FOCUS\_AFTER\_DESCENDANTS}: The view receives focus only if none of its descendants request it.
                \item \textbf{int FOCUS\_BEFORE\_DESCENDANTS}: The view receives focus before any of its descendants.
                \item \textbf{int FOCUS\_BLOCK\_DESCENDANTS}: Prevents any descendants from receiving focus, even if they are focusable.
                \item \textbf{int LAYOUT\_MODE\_CLIP\_BOUNDS}: Layout mode constant that aligns layout to the view’s clip bounds.
                \item \textbf{int LAYOUT\_MODE\_OPTICAL\_BOUNDS}: Layout mode constant that aligns layout to the view’s optical bounds.
                \item \textbf{int PERSISTENT\_ALL\_CACHES}: \textit{Deprecated in API 28.} Formerly kept all drawing caches (animation, scrolling, etc.). Superseded by hardware acceleration and \texttt{View.setLayerType(int, Paint)}.
                \item \textbf{int PERSISTENT\_ANIMATION\_CACHE}: \textit{Deprecated in API 28.} Formerly kept only animation caches. Hardware acceleration now handles such effects efficiently.
                \item \textbf{int PERSISTENT\_NO\_CACHE}: \textit{Deprecated in API 28.} Disabled view drawing caches. Replaced by hardware rendering mechanisms.
                \item \textbf{int PERSISTENT\_SCROLLING\_CACHE}: \textit{Deprecated in API 28.} Formerly maintained caches for scrolling operations; hardware acceleration now replaces this feature.
            \end{itemize}

    \end{itemize}

    \pagebreak 
    \subsubsection{Context}
    \begin{itemize}
         \item \textbf{Hierarchy} 
             \begin{center}
                 java.lang.Object $\to$	android.content.Context
             \end{center}
        \item \textbf{Include}
            \bigbreak \noindent 
            \begin{javacode}
                android.content.Context
            \end{javacode}
        \item \textbf{Constructors}
            \bigbreak \noindent 
            \begin{javacode}
                Context()
            \end{javacode}
        \item \textbf{Public methods}
            \begin{itemize}
                \item \textbf{Context getApplicationContext()}: Returns the global application context.
                \item \textbf{Resources getResources()}: Provides access to the app's resources (layouts, strings, drawables, etc.).
                \item \textbf{PackageManager getPackageManager()}: Returns a PackageManager for querying installed apps and permissions.
                \item \textbf{ContentResolver getContentResolver()}: Gives access to content providers (e.g., Contacts, MediaStore).
                \item \textbf{SharedPreferences getSharedPreferences(String name, int mode)}: Access or create a preferences file for storing key–value pairs.
                \item \textbf{File getFilesDir()}: Returns the app’s private file storage directory.
                \item \textbf{File getCacheDir()}: Returns the app’s private cache directory.
                \item \textbf{Drawable getDrawable(int id)}: Retrieves a drawable resource styled for the current theme.
                \item \textbf{int getColor(int id)}: Returns a color resource styled for the current theme.
                \item \textbf{String getString(int resId)}: Returns a localized string from resources.
                \item \textbf{String getString(int resId, Object... formatArgs)}: Returns a formatted localized string.
                \item \textbf{void startActivity(Intent intent)}: Launches a new activity.
                \item \textbf{ComponentName startService(Intent service)}: Starts a service.
                \item \textbf{boolean stopService(Intent service)}: Stops a running service.
                \item \textbf{boolean bindService(Intent service, ServiceConnection conn, int flags)}: Connects to a service for interaction.
                \item \textbf{void unbindService(ServiceConnection conn)}: Disconnects from a bound service.
                \item \textbf{void sendBroadcast(Intent intent)}: Sends a broadcast to all interested receivers.
                \item \textbf{Intent registerReceiver(BroadcastReceiver receiver, IntentFilter filter)}: Registers a broadcast receiver.
                \item \textbf{void unregisterReceiver(BroadcastReceiver receiver)}: Unregisters a previously registered receiver.
                \item \textbf{Object getSystemService(String name)}: Returns a handle to a system-level service (e.g., \texttt{LAYOUT\_INFLATER\_SERVICE}).
                \item \textbf{<T> T getSystemService(Class<T> serviceClass)}: Type-safe version of \texttt{getSystemService}.
                \item \textbf{Resources.Theme getTheme()}: Returns the current theme for styling and inflation.
                \item \textbf{TypedArray obtainStyledAttributes(int[] attrs)}: Retrieves styled attributes in the current theme.
                \item \textbf{FileInputStream openFileInput(String name)}: Opens a private file for reading.
                \item \textbf{FileOutputStream openFileOutput(String name, int mode)}: Opens a private file for writing.
            \end{itemize}
        \item \textbf{Constants}
            \begin{itemize}
                % --- Core system & UI services ---
                \item \textbf{String ACTIVITY\_SERVICE}: For \texttt{ActivityManager} (process/app state) via \texttt{getSystemService}.
                \item \textbf{String WINDOW\_SERVICE}: For \texttt{WindowManager} (windows, display metrics).
                \item \textbf{String DISPLAY\_SERVICE}: For \texttt{DisplayManager} (displays, modes).
                \item \textbf{String LAYOUT\_INFLATER\_SERVICE}: For \texttt{LayoutInflater} (inflate XML layouts).
                \item \textbf{String POWER\_SERVICE}: For \texttt{PowerManager} (wake locks, power state).
                \item \textbf{String UI\_MODE\_SERVICE}: For \texttt{UiModeManager} (night mode, car/TV mode).

                    % --- Connectivity / networking ---
                \item \textbf{String CONNECTIVITY\_SERVICE}: For \texttt{ConnectivityManager} (network state, requests).
                \item \textbf{String WIFI\_SERVICE}: For \texttt{WifiManager} (Wi-Fi control).
                \item \textbf{String WIFI\_P2P\_SERVICE}: For \texttt{WifiP2pManager} (Wi-Fi Direct).
                \item \textbf{String TETHERING\_SERVICE}: For \texttt{TetheringManager} (tethering APIs).
                \item \textbf{String USB\_SERVICE}: For \texttt{UsbManager} (USB host/device).
                \item \textbf{String NFC\_SERVICE}: For \texttt{NfcManager} (NFC features).
                \item \textbf{String VPN\_MANAGEMENT\_SERVICE}: For \texttt{VpnManager} (built-in VPN profiles).

                    % --- Location / sensors / hardware ---
                \item \textbf{String LOCATION\_SERVICE}: For \texttt{LocationManager} (location providers).
                \item \textbf{String SENSOR\_SERVICE}: For \texttt{SensorManager} (accelerometer, gyro, etc.).
                \item \textbf{String BLUETOOTH\_SERVICE}: For \texttt{BluetoothManager} (Bluetooth stack).
                \item \textbf{String CAMERA\_SERVICE}: For \texttt{CameraManager} (camera devices).
                \item \textbf{String AUDIO\_SERVICE}: For \texttt{AudioManager} (volume, routing).

                    % --- Media / notifications ---
                \item \textbf{String NOTIFICATION\_SERVICE}: For \texttt{NotificationManager} (post/cancel notifications).
                \item \textbf{String MEDIA\_SESSION\_SERVICE}: For \texttt{MediaSessionManager} (media controls).
                \item \textbf{String MEDIA\_PROJECTION\_SERVICE}: For \texttt{MediaProjectionManager} (screen capture).
                \item \textbf{String DOWNLOAD\_SERVICE}: For \texttt{DownloadManager} (HTTP downloads).

                    % --- Input / clipboard / keyguard / biometrics ---
                \item \textbf{String INPUT\_METHOD\_SERVICE}: For \texttt{InputMethodManager} (soft keyboard).
                \item \textbf{String CLIPBOARD\_SERVICE}: For \texttt{ClipboardManager} (global clipboard).
                \item \textbf{String KEYGUARD\_SERVICE}: For \texttt{KeyguardManager} (lock screen).
                \item \textbf{String BIOMETRIC\_SERVICE}: For \texttt{BiometricManager} (biometric auth).

                    % --- App data / users / scheduling ---
                \item \textbf{String STORAGE\_SERVICE}: For \texttt{StorageManager} (volumes, storage ops).
                \item \textbf{String USER\_SERVICE}: For \texttt{UserManager} (multi-user info).
                \item \textbf{String JOB\_SCHEDULER\_SERVICE}: For \texttt{JobScheduler} (deferrable background work).
                \item \textbf{String APP\_OPS\_SERVICE}: For \texttt{AppOpsManager} (app operation checks).

                    % --- Haptics ---
                \item \textbf{String VIBRATOR\_MANAGER\_SERVICE}: For \texttt{VibratorManager} (multi-vibrator control).

                    % ======================
                    % Common flags / modes
                    % ======================
                \item \textbf{int MODE\_PRIVATE}: Default file mode for \texttt{openFileOutput}; file private to the app.
                \item \textbf{int MODE\_APPEND}: Append mode for \texttt{openFileOutput}.
                \item \textbf{int MODE\_ENABLE\_WRITE\_AHEAD\_LOGGING}: DB flag to enable WAL by default.
                \item \textbf{int MODE\_NO\_LOCALIZED\_COLLATORS}: DB flag to omit localized collators.
                \item \textbf{int RECEIVER\_EXPORTED}: \texttt{registerReceiver} flag — receiver accepts broadcasts from other apps.
                \item \textbf{int RECEIVER\_NOT\_EXPORTED}: \texttt{registerReceiver} flag — receiver is app-internal only.
                \item \textbf{int RECEIVER\_VISIBLE\_TO\_INSTANT\_APPS}: \texttt{registerReceiver} flag — visible to Instant Apps.

                    % --- Service binding flags most seen in practice ---
                \item \textbf{int BIND\_AUTO\_CREATE}: Auto-create service while bound.
                \item \textbf{int BIND\_NOT\_FOREGROUND}: Do not raise target service to foreground priority.
                \item \textbf{int BIND\_IMPORTANT}: Treat service as important to the client.
                \item \textbf{int BIND\_DEBUG\_UNBIND}: Include debugging help for unbind mismatches.
                \item \textbf{int BIND\_WAIVE\_PRIORITY}: Do not affect service process priority.
        \end{itemize}


    \end{itemize}

    \pagebreak 
    \subsubsection{ConstraintLayout}
    \begin{itemize}
        \item \textbf{Hierarchy}: 
            \begin{center}
                java.lang.Object $\to $	android.view.View $\to $	android.view.ViewGroup $\to $	androidx.constraintlayout.widget.ConstraintLayout
            \end{center}
        \item \textbf{Include}:
            \bigbreak \noindent 
            \begin{javacode}
            androidx.constraintlayout.widget.ConstraintLayout
            \end{javacode}
        \item \textbf{Constructors}:
            \bigbreak \noindent 
            \begin{javacode}
                ConstraintLayout(@NonNull Context context)
                ConstraintLayout(@NonNull Context context, @Nullable AttributeSet attrs)
                ConstraintLayout( @NonNull Context context, @Nullable AttributeSet attrs, int defStyleAttr)

                @TargetApi(value = Build.VERSION_CODES.LOLLIPOP)
                ConstraintLayout( @NonNull Context context, @Nullable AttributeSet attrs, int defStyleAttr, int defStyleRes)
            \end{javacode}
        \item \textbf{Public methods}:
            \begin{itemize}
                \item \textbf{void addValueModifier(ConstraintLayout.ValueModifier modifier)}: Adds a \texttt{ValueModifier} to the \texttt{ConstraintLayout}.
                \item \textbf{void fillMetrics(Metrics metrics)}: Populates the provided \texttt{Metrics} object with performance and measurement data.
                \item \textbf{void forceLayout()}: Forces a layout pass, marking the layout as needing to be re-measured and re-laid out.
                \item \textbf{ConstraintLayout.LayoutParams generateLayoutParams(AttributeSet attrs)}: Returns a new set of layout parameters based on the supplied attributes set.
                \item \textbf{Object getDesignInformation(int type, Object value)}: Retrieves design-time information associated with the layout.
                \item \textbf{int getMaxHeight()}: Returns the maximum height of this view.
                \item \textbf{int getMaxWidth()}: Returns the maximum width of this view.
                \item \textbf{int getMinHeight()}: Returns the minimum height of this view.
                \item \textbf{int getMinWidth()}: Returns the minimum width of this view.
                \item \textbf{int getOptimizationLevel()}: Returns the current optimization level for the layout resolution.
                \item \textbf{String getSceneString()}: Returns a JSON5 string useful for debugging the constraints currently applied.
                \item \textbf{static SharedValues getSharedValues()}: Returns the \texttt{SharedValues} instance, creating it if it does not already exist.
                \item \textbf{View getViewById(int id)}: Returns the \texttt{View} corresponding to the given ID.
                \item \textbf{final ConstraintWidget getViewWidget(View view)}: Returns the internal \texttt{ConstraintWidget} associated with a given view.
                \item \textbf{void loadLayoutDescription(int layoutDescription)}: Loads a layout description file from the application's resources.
                \item \textbf{void onViewAdded(View view)}: Called when a child view is added to the layout.
                \item \textbf{void onViewRemoved(View view)}: Called when a child view is removed from the layout.
                \item \textbf{void requestLayout()}: Requests a re-layout of this view hierarchy.
                \item \textbf{void setConstraintSet(ConstraintSet set)}: Sets a \texttt{ConstraintSet} object to manage constraints.
                \item \textbf{void setDesignInformation(int type, Object value1, Object value2)}: Stores design-time information associated with the layout.
                \item \textbf{void setId(int id)}: Sets the ID for this view.
                \item \textbf{void setMaxHeight(int value)}: Sets the maximum height for this view.
                \item \textbf{void setMaxWidth(int value)}: Sets the maximum width for this view.
                \item \textbf{void setMinHeight(int value)}: Sets the minimum height for this view.
                \item \textbf{void setMinWidth(int value)}: Sets the minimum width for this view.
                \item \textbf{void setOnConstraintsChanged(ConstraintsChangedListener constraintsChangedListener)}: Registers a listener to be notified when constraints change.
                \item \textbf{void setOptimizationLevel(int level)}: Sets the optimization level for layout resolution.
                \item \textbf{void setState(int id, int screenWidth, int screenHeight)}: Sets the state of the \texttt{ConstraintLayout}, causing it to load a specific \texttt{ConstraintSet}.
                \item \textbf{boolean shouldDelayChildPressedState()}: Returns true if the pressed state should be delayed for children or descendants of this \texttt{ViewGroup}.
            \end{itemize}
        \item \textbf{Protected methods}:
            \begin{itemize}
                \item \textbf{void applyConstraintsFromLayoutParams(boolean isInEditMode, View child, ConstraintWidget widget, ConstraintLayout.LayoutParams layoutParams, SparseArray<ConstraintWidget> idToWidget)}: Applies constraints from the given layout parameters to the specified \texttt{ConstraintWidget}.
                \item \textbf{boolean checkLayoutParams(ViewGroup.LayoutParams p)}: Determines whether the supplied layout parameters are valid for this layout.
                \item \textbf{void dispatchDraw(Canvas canvas)}: Called to draw the layout’s children onto the provided \texttt{Canvas}.
                \item \textbf{boolean dynamicUpdateConstraints(int widthMeasureSpec, int heightMeasureSpec)}: Can be overridden to change how \texttt{ValueModifier}s are used during dynamic updates of constraints.
                \item \textbf{ConstraintLayout.LayoutParams generateDefaultLayoutParams()}: Returns a set of default layout parameters for this \texttt{ConstraintLayout}.
                \item \textbf{ViewGroup.LayoutParams generateLayoutParams(ViewGroup.LayoutParams p)}: Returns a safe set of layout parameters based on the supplied parameters.
                \item \textbf{boolean isRtl()}: Returns \texttt{true} if the layout direction is right-to-left (RTL).
                \item \textbf{void onLayout(boolean changed, int left, int top, int right, int bottom)}: Called during layout to assign a size and position to each child.
                \item \textbf{void onMeasure(int widthMeasureSpec, int heightMeasureSpec)}: Measures the layout and its children to determine width and height.
                \item \textbf{void parseLayoutDescription(int id)}: Called to handle layout descriptions; subclasses may override this method.
                \item \textbf{void resolveMeasuredDimension(int widthMeasureSpec, int heightMeasureSpec, int measuredWidth, int measuredHeight, boolean isWidthMeasuredTooSmall, boolean isHeightMeasuredTooSmall)}: Handles setting the measured dimensions for the layout.
                \item \textbf{void resolveSystem(ConstraintWidgetContainer layout, int optimizationLevel, int widthMeasureSpec, int heightMeasureSpec)}: Handles the measuring and constraint resolution of the layout.
                \item \textbf{void setSelfDimensionBehaviour(ConstraintWidgetContainer layout, int widthMode, int widthSize, int heightMode, int heightSize)}: Configures the layout’s own dimension behavior during constraint resolution.
            \end{itemize}
        \item \textbf{Constants}:
            \begin{itemize}
                \item \textbf{static final int	DESIGN\_INFO\_ID = 0}:
                \item \textbf{static final String VERSION = "ConstraintLayout-2.2.0-alpha04"}:
            \end{itemize}
        \item \textbf{Protected fields}:
            \begin{itemize}
                \item \textbf{ConstraintLayoutStates	mConstraintLayoutSpec}:
                \item \textbf{boolean	mDirtyHierarchy}:
                \item \textbf{ConstraintWidgetContainer	mLayoutWidget}:
            \end{itemize}
    \end{itemize}

    \pagebreak 
    \subsubsection{ConstraintLayout.LayoutParams}
    \begin{itemize}
        \item \textbf{Hierarchy}: 
            \begin{center}
                java.lang.Object $\to $	android.view.View $\to $	android.view.ViewGroup $\to $	androidx.constraintlayout.widget.ConstraintLayout
            \end{center}
        \item \textbf{Include}:
            \bigbreak \noindent 
            \begin{javacode}
            androidx.constraintlayout.widget.ConstraintLayout
            \end{javacode}
        \item \textbf{Constructors}:
            \bigbreak \noindent 
            \begin{javacode}
                LayoutParams(ViewGroup.LayoutParams params)
                LayoutParams(Context c, AttributeSet attrs)
                LayoutParams(int width, int height)
            \end{javacode}
        \item \textbf{Public methods}:
            \begin{itemize}
                \item \textbf{String getConstraintTag()}: Returns a tag that can be used to identify a view as being part of a constraint group.
                \item \textbf{ConstraintWidget getConstraintWidget()}: Returns the underlying \texttt{ConstraintWidget} object associated with this layout parameter or view.
                \item \textbf{void reset()}: Resets the associated \texttt{ConstraintWidget} to its default state.
                \item \textbf{void resolveLayoutDirection(int layoutDirection)}: Resolves layout direction–dependent constraints such as start/end alignment.
                \item \textbf{void setWidgetDebugName(String text)}: Sets a debug name for the \texttt{ConstraintWidget}, useful for diagnostics or logging.
                \item \textbf{void validate()}: Validates the layout and ensures that all parameters and constraints are consistent.
            \end{itemize}
        \item \textbf{Public fields}:
            \begin{itemize}
                \item \textbf{int baselineMargin}: The baseline margin.
                \item \textbf{int baselineToBaseline}: Constrains the baseline of a child to the baseline of a target child (contains the target child ID).
                \item \textbf{int baselineToBottom}: Constrains the baseline of a child to the bottom of a target child (contains the target child ID).
                \item \textbf{int baselineToTop}: Constrains the baseline of a child to the top of a target child (contains the target child ID).
                \item \textbf{int bottomToBottom}: Constrains the bottom side of a child to the bottom side of a target child (contains the target child ID).
                \item \textbf{int bottomToTop}: Constrains the bottom side of a child to the top side of a target child (contains the target child ID).
                \item \textbf{float circleAngle}: The angle used for a circular constraint.
                \item \textbf{int circleConstraint}: Constrains the center of a child to the center of a target child (contains the target child ID).
                \item \textbf{int circleRadius}: The radius used for a circular constraint.
                \item \textbf{boolean constrainedHeight}: Specifies if the vertical dimension is constrained when both top and bottom constraints are set and the dimension is not fixed.
                \item \textbf{boolean constrainedWidth}: Specifies if the horizontal dimension is constrained when both left and right constraints are set and the dimension is not fixed.
                \item \textbf{String constraintTag}: Defines a category of view to be used by helpers and \texttt{MotionLayout}.
                \item \textbf{String dimensionRatio}: The ratio information defining the aspect ratio of the view.
                \item \textbf{int editorAbsoluteX}: The design-time X coordinate (left position) of the child.
                \item \textbf{int editorAbsoluteY}: The design-time Y coordinate (top position) of the child.
                \item \textbf{int endToEnd}: Constrains the end side of a child to the end side of a target child (contains the target child ID).
                \item \textbf{int endToStart}: Constrains the end side of a child to the start side of a target child (contains the target child ID).
                \item \textbf{int goneBaselineMargin}: The baseline margin to use when the target is gone.
                \item \textbf{int goneBottomMargin}: The bottom margin to use when the target is gone.
                \item \textbf{int goneEndMargin}: The end margin to use when the target is gone.
                \item \textbf{int goneLeftMargin}: The left margin to use when the target is gone.
                \item \textbf{int goneRightMargin}: The right margin to use when the target is gone.
                \item \textbf{int goneStartMargin}: The start margin to use when the target is gone.
                \item \textbf{int goneTopMargin}: The top margin to use when the target is gone.
                \item \textbf{int guideBegin}: The distance of a guideline from the top or left edge of its parent.
                \item \textbf{int guideEnd}: The distance of a guideline from the bottom or right edge of its parent.
                \item \textbf{float guidePercent}: The ratio of the distance to the parent's sides.
                \item \textbf{boolean guidelineUseRtl}: Determines whether guideline position respects RTL layout direction.
                \item \textbf{boolean helped}: Indicates whether the view was modified by a helper.
                \item \textbf{float horizontalBias}: The ratio between two connections when left and right (or start and end) sides are constrained.
                \item \textbf{int horizontalChainStyle}: Defines how elements of a horizontal chain are positioned.
                \item \textbf{float horizontalWeight}: The child’s weight used to distribute available horizontal space in a chain when using \texttt{MATCH\_CONSTRAINT}.
                \item \textbf{int leftToLeft}: Constrains the left side of a child to the left side of a target child (contains the target child ID).
                \item \textbf{int leftToRight}: Constrains the left side of a child to the right side of a target child (contains the target child ID).
                \item \textbf{int matchConstraintDefaultHeight}: Defines how the widget’s vertical dimension is handled when set to \texttt{MATCH\_CONSTRAINT}.
                \item \textbf{int matchConstraintDefaultWidth}: Defines how the widget’s horizontal dimension is handled when set to \texttt{MATCH\_CONSTRAINT}.
                \item \textbf{int matchConstraintMaxHeight}: Specifies a maximum height for the widget.
                \item \textbf{int matchConstraintMaxWidth}: Specifies a maximum width for the widget.
                \item \textbf{int matchConstraintMinHeight}: Specifies a minimum height for the widget.
                \item \textbf{int matchConstraintMinWidth}: Specifies a minimum width for the widget.
                \item \textbf{float matchConstraintPercentHeight}: Specifies a percentage value when using the match-constraint percent mode for height.
                \item \textbf{float matchConstraintPercentWidth}: Specifies a percentage value when using the match-constraint percent mode for width.
                \item \textbf{int orientation}: The orientation of the layout (horizontal or vertical).
                \item \textbf{int rightToLeft}: Constrains the right side of a child to the left side of a target child (contains the target child ID).
                \item \textbf{int rightToRight}: Constrains the right side of a child to the right side of a target child (contains the target child ID).
                \item \textbf{int startToEnd}: Constrains the start side of a child to the end side of a target child (contains the target child ID).
                \item \textbf{int startToStart}: Constrains the start side of a child to the start side of a target child (contains the target child ID).
                \item \textbf{int topToBottom}: Constrains the top side of a child to the bottom side of a target child (contains the target child ID).
                \item \textbf{int topToTop}: Constrains the top side of a child to the top side of a target child (contains the target child ID).
                \item \textbf{float verticalBias}: The ratio between two connections when the top and bottom sides are constrained.
                \item \textbf{int verticalChainStyle}: Defines how elements of a vertical chain are positioned.
                \item \textbf{float verticalWeight}: The child’s weight used to distribute available vertical space in a chain when using \texttt{MATCH\_CONSTRAINT}.
                \item \textbf{int wrapBehaviorInParent}: Specifies how this view is considered during the parent's wrap computation:
                    \begin{itemize}
                        \item \texttt{WRAP\_BEHAVIOR\_INCLUDED}: Included in both directions (default).
                        \item \texttt{WRAP\_BEHAVIOR\_HORIZONTAL\_ONLY}: Included only horizontally.
                        \item \texttt{WRAP\_BEHAVIOR\_VERTICAL\_ONLY}: Included only vertically.
                        \item \texttt{WRAP\_BEHAVIOR\_SKIPPED}: Excluded from wrap computation.
                    \end{itemize}
            \end{itemize}

        \item \textbf{Constants}:
            \begin{itemize}
                \item \textbf{static final int BASELINE = 5}: The baseline of the text in a view.
                \item \textbf{static final int BOTTOM = 4}: The bottom side of a view.
                \item \textbf{static final int CHAIN\_PACKED = 2}: Chain packed style.
                \item \textbf{static final int CHAIN\_SPREAD = 0}: Chain spread style.
                \item \textbf{static final int CHAIN\_SPREAD\_INSIDE = 1}: Chain spread inside style.
                \item \textbf{static final int CIRCLE = 8}: Circle reference from a view.
                \item \textbf{static final int END = 7}: The right side of a view in left-to-right languages.
                \item \textbf{static final int GONE\_UNSET = -2147483648}: Defines an ID that is not set (default unset state).
                \item \textbf{static final int HORIZONTAL = 0}: The horizontal orientation.
                \item \textbf{static final int LEFT = 1}: The left side of a view.
                \item \textbf{static final int MATCH\_CONSTRAINT = 0}: Dimension will be controlled by constraints.
                \item \textbf{static final int MATCH\_CONSTRAINT\_PERCENT = 2}: Sets \texttt{matchConstraintDefault*} percent mode to be based on a percent of another dimension (usually the parent). Used for \texttt{matchConstraintDefaultWidth} and \texttt{matchConstraintDefaultHeight}.
                \item \textbf{static final int MATCH\_CONSTRAINT\_SPREAD = 0}: Sets \texttt{matchConstraintDefault*} to spread as much as possible within its constraints.
                \item \textbf{static final int MATCH\_CONSTRAINT\_WRAP = 1}: Sets \texttt{matchConstraintDefault*} to wrap content size.
                \item \textbf{static final int PARENT\_ID = 0}: References the ID of the parent layout.
                \item \textbf{static final int RIGHT = 2}: The right side of a view.
                \item \textbf{static final int START = 6}: The left side of a view in left-to-right languages.
                \item \textbf{static final int TOP = 3}: The top side of a view.
                \item \textbf{static final int UNSET = -1}: Defines an ID that is not set.
                \item \textbf{static final int VERTICAL = 1}: The vertical orientation.
                \item \textbf{static final int WRAP\_BEHAVIOR\_HORIZONTAL\_ONLY = 1}: Specifies that wrapping occurs only horizontally.
                \item \textbf{static final int WRAP\_BEHAVIOR\_INCLUDED = 0}: Specifies that wrapping includes both horizontal and vertical directions (default).
                \item \textbf{static final int WRAP\_BEHAVIOR\_SKIPPED = 3}: Specifies that the widget is excluded from wrap computation.
                \item \textbf{static final int WRAP\_BEHAVIOR\_VERTICAL\_ONLY = 2}: Specifies that wrapping occurs only vertically.
            \end{itemize}


    \end{itemize}



    \pagebreak 
    \subsubsection{RelativeLayout}
    \begin{itemize}
        \item \textbf{Hierarchy}:
            \begin{center}
                java.lang.Object $\to$ android.view.View $\to$	android.view.ViewGroup $\to$	android.widget.RelativeLayout
            \end{center}
        \item \textbf{Include}:
            \bigbreak \noindent 
            \begin{javacode}
                android.widget.RelativeLayout    
            \end{javacode}
        \item \textbf{Constructors}: 
            \bigbreak \noindent 
            \begin{javacode}
                RelativeLayout(Context context)
                RelativeLayout(Context context, AttributeSet attrs)
                RelativeLayout(Context context, AttributeSet attrs, int defStyleAttr)
                RelativeLayout(Context context, AttributeSet attrs, int defStyleAttr, int defStyleRes)
            \end{javacode}
        \item \textbf{Public methods}:
            \begin{itemize}
                \item \textbf{RelativeLayout.LayoutParams generateLayoutParams(AttributeSet attrs)}: Returns a new set of layout parameters based on the supplied attributes set.
                \item \textbf{CharSequence getAccessibilityClassName()}: Return the class name of this object to be used for accessibility purposes.
                \item \textbf{int getBaseline()}: Return the offset of the widget's text baseline from the widget's top boundary.
                \item \textbf{int getGravity()}: Describes how the child views are positioned.
                \item \textbf{int getIgnoreGravity()}: Get the ID of the \texttt{View} to be ignored by gravity.
                \item \textbf{void requestLayout()}: Call this when something has changed that invalidates the layout of this view.
                \item \textbf{void setGravity(int gravity)}: Describes how the child views are positioned.
                \item \textbf{void setHorizontalGravity(int horizontalGravity)}: Sets the horizontal gravity of the layout.
                \item \textbf{void setIgnoreGravity(int viewId)}: Defines which \texttt{View} is ignored when gravity is applied.
                \item \textbf{void setVerticalGravity(int verticalGravity)}: Sets the vertical gravity of the layout.
                \item \textbf{boolean shouldDelayChildPressedState()}: Returns true if the pressed state should be delayed for children or descendants of this \texttt{ViewGroup}.
            \end{itemize}
        \item \textbf{Protected methods}:
            \begin{itemize}
                \item \textbf{boolean checkLayoutParams(ViewGroup.LayoutParams p)}: Determines whether the supplied layout parameters are valid for this layout.
                \item \textbf{ViewGroup.LayoutParams generateDefaultLayoutParams()}: Returns a set of layout parameters with a width of \texttt{ViewGroup.LayoutParams.WRAP\_CONTENT}, a height of \texttt{ViewGroup.LayoutParams.WRAP\_CONTENT}, and no spanning.
                \item \textbf{ViewGroup.LayoutParams generateLayoutParams(ViewGroup.LayoutParams lp)}: Returns a safe set of layout parameters based on the supplied layout parameters.
                \item \textbf{void onLayout(boolean changed, int l, int t, int r, int b)}: Called from layout when this view should assign a size and position to each of its children.
                \item \textbf{void onMeasure(int widthMeasureSpec, int heightMeasureSpec)}: Measures the view and its content to determine the measured width and height.
            \end{itemize}
        \item \textbf{Constants}:
            \begin{itemize}
                \item \textbf{int ABOVE}: Rule that aligns a child's bottom edge with another child's top edge.
                \item \textbf{int ALIGN\_BASELINE}: Rule that aligns a child's baseline with another child's baseline.
                \item \textbf{int ALIGN\_BOTTOM}: Rule that aligns a child's bottom edge with another child's bottom edge.
                \item \textbf{int ALIGN\_END}: Rule that aligns a child's end edge with another child's end edge.
                \item \textbf{int ALIGN\_LEFT}: Rule that aligns a child's left edge with another child's left edge.
                \item \textbf{int ALIGN\_PARENT\_BOTTOM}: Rule that aligns the child's bottom edge with its \texttt{RelativeLayout} parent's bottom edge.
                \item \textbf{int ALIGN\_PARENT\_END}: Rule that aligns the child's end edge with its \texttt{RelativeLayout} parent's end edge.
                \item \textbf{int ALIGN\_PARENT\_LEFT}: Rule that aligns the child's left edge with its \texttt{RelativeLayout} parent's left edge.
                \item \textbf{int ALIGN\_PARENT\_RIGHT}: Rule that aligns the child's right edge with its \texttt{RelativeLayout} parent's right edge.
                \item \textbf{int ALIGN\_PARENT\_START}: Rule that aligns the child's start edge with its \texttt{RelativeLayout} parent's start edge.
                \item \textbf{int ALIGN\_PARENT\_TOP}: Rule that aligns the child's top edge with its \texttt{RelativeLayout} parent's top edge.
                \item \textbf{int ALIGN\_RIGHT}: Rule that aligns a child's right edge with another child's right edge.
                \item \textbf{int ALIGN\_START}: Rule that aligns a child's start edge with another child's start edge.
                \item \textbf{int ALIGN\_TOP}: Rule that aligns a child's top edge with another child's top edge.
                \item \textbf{int BELOW}: Rule that aligns a child's top edge with another child's bottom edge.
                \item \textbf{int CENTER\_HORIZONTAL}: Rule that centers the child horizontally with respect to the bounds of its \texttt{RelativeLayout} parent.
                \item \textbf{int CENTER\_IN\_PARENT}: Rule that centers the child with respect to the bounds of its \texttt{RelativeLayout} parent.
                \item \textbf{int CENTER\_VERTICAL}: Rule that centers the child vertically with respect to the bounds of its \texttt{RelativeLayout} parent.
                \item \textbf{int END\_OF}: Rule that aligns a child's start edge with another child's end edge.
                \item \textbf{int LEFT\_OF}: Rule that aligns a child's right edge with another child's left edge.
                \item \textbf{int RIGHT\_OF}: Rule that aligns a child's left edge with another child's right edge.
                \item \textbf{int START\_OF}: Rule that aligns a child's end edge with another child's start edge.
                \item \textbf{int TRUE}: Constant used for layout rules that take a boolean value.
            \end{itemize}

    \end{itemize}




    \pagebreak 
    \subsubsection{RelativeLayout.LayoutParams}
    \begin{itemize}
        \item \textbf{Hierarchy}:
            \begin{center}
                java.lang.Object $\to $	android.view.ViewGroup.LayoutParams $\to $	android.view.ViewGroup.MarginLayoutParams $\to $	android.widget.RelativeLayout.LayoutParams
            \end{center}
        \item \textbf{Include}
            \bigbreak \noindent 
            \begin{javacode}
                android.widget.RelativeLayout.LayoutParams
            \end{javacode}

        \item \textbf{Constructors}:
            \bigbreak \noindent 
            \begin{javacode}
                LayoutParams(Context c, AttributeSet attrs)
                LayoutParams(ViewGroup.LayoutParams source)
                LayoutParams(ViewGroup.MarginLayoutParams source)
                LayoutParams(RelativeLayout.LayoutParams source)
                LayoutParams(int w, int h)
            \end{javacode}
        \item \textbf{Public methods}:
            \begin{itemize}
                \item \textbf{void addRule(int verb, int subject)}: Adds a layout rule to be interpreted by the \texttt{RelativeLayout}, relative to another view.
                \item \textbf{void addRule(int verb)}: Adds a layout rule to be interpreted by the \texttt{RelativeLayout}.
                \item \textbf{String debug(String output)}: Returns a string representation of this set of layout parameters, typically used for debugging.
                \item \textbf{int getRule(int verb)}: Returns the layout rule associated with a specific verb.
                \item \textbf{int[] getRules()}: Retrieves a complete list of all supported rules, where each index represents a rule verb and each value represents the associated parameter (or \texttt{false} if not set).
                \item \textbf{void removeRule(int verb)}: Removes a layout rule from interpretation by the \texttt{RelativeLayout}.
                \item \textbf{void resolveLayoutDirection(int layoutDirection)}: Called by \texttt{View.requestLayout()} to resolve layout parameters that depend on layout direction (e.g., start/end alignment).
            \end{itemize}
        \item \textbf{Fields}:
            \begin{itemize}
                \item \textbf{public boolean	alignWithParent}: When true, uses the parent as the anchor if the anchor doesn't exist or if the anchor's visibility is GONE.
            \end{itemize}

    \end{itemize}
    

    \pagebreak 
    \subsubsection{LinearLayout}
    \begin{itemize}
        \item \textbf{Hierarchy}:
            \begin{center}
                java.lang.Object $\to$	android.view.View $\to $	android.view.ViewGroup $\to $	android.widget.LinearLayout
            \end{center}
        \item \textbf{Include}
            \bigbreak \noindent 
            \begin{javacode}
            android.widget.LinearLayout
            \end{javacode}
        \item \textbf{Constructors}:
            \bigbreak \noindent 
            \begin{javacode}
                LinearLayout(Context context)
                LinearLayout(Context context, AttributeSet attrs)
                LinearLayout(Context context, AttributeSet attrs, int defStyleAttr)
                LinearLayout(Context context, AttributeSet attrs, int defStyleAttr, int defStyleRes) 
            \end{javacode}
        \item \textbf{Public Methods}:
            \begin{itemize}
                \item \textbf{LinearLayout.LayoutParams generateLayoutParams(AttributeSet attrs)}: Returns a new set of layout parameters based on the supplied attributes set.
                \item \textbf{CharSequence getAccessibilityClassName()}: Return the class name of this object to be used for accessibility purposes.
                \item \textbf{int getBaseline()}: Return the offset of the widget's text baseline from the widget's top boundary.
                \item \textbf{int getBaselineAlignedChildIndex()}: Returns the index used for baseline alignment.
                \item \textbf{Drawable getDividerDrawable()}: Returns the drawable used as a divider between child views.
                \item \textbf{int getDividerPadding()}: Get the padding size used to inset dividers in pixels.
                \item \textbf{int getGravity()}: Returns the current gravity.
                \item \textbf{int getOrientation()}: Returns the current orientation.
                \item \textbf{int getShowDividers()}: Returns how dividers are displayed between items.
                \item \textbf{float getWeightSum()}: Returns the desired weights sum.
                \item \textbf{boolean isBaselineAligned()}: Indicates whether widgets contained within this layout are aligned on their baseline or not.
                \item \textbf{boolean isMeasureWithLargestChildEnabled()}: When true, all children with a weight will be considered having the minimum size of the largest child.
                \item \textbf{void onRtlPropertiesChanged(int layoutDirection)}: Called when any RTL property (layout direction or text direction or text alignment) has been changed.
                \item \textbf{void setBaselineAligned(boolean baselineAligned)}: Defines whether widgets contained in this layout are baseline-aligned or not.
                \item \textbf{void setBaselineAlignedChildIndex(int i)}: Sets which child is used for baseline alignment.
                \item \textbf{void setDividerDrawable(Drawable divider)}: Set a drawable to be used as a divider between items.
                \item \textbf{void setDividerPadding(int padding)}: Set padding displayed on both ends of dividers.
                \item \textbf{void setGravity(int gravity)}: Describes how the child views are positioned.
                \item \textbf{void setHorizontalGravity(int horizontalGravity)}: Sets the horizontal gravity of the layout.
                \item \textbf{void setMeasureWithLargestChildEnabled(boolean enabled)}: When set to true, all children with a weight will be considered having the minimum size of the largest child.
                \item \textbf{void setOrientation(int orientation)}: Should the layout be a column or a row.
                \item \textbf{void setShowDividers(int showDividers)}: Set how dividers should be shown between items in this layout.
                \item \textbf{void setVerticalGravity(int verticalGravity)}: Sets the vertical gravity of the layout.
                \item \textbf{void setWeightSum(float weightSum)}: Defines the desired weights sum.
                \item \textbf{boolean shouldDelayChildPressedState()}: Return true if the pressed state should be delayed for children or descendants of this ViewGroup.
            \end{itemize}
        \item \textbf{Protected Methods}:
            \begin{itemize}
                \item \textbf{boolean checkLayoutParams(ViewGroup.LayoutParams p)}: Determines whether the supplied layout parameters are valid for this layout.
                \item \textbf{LinearLayout.LayoutParams generateDefaultLayoutParams()}: Returns a set of layout parameters with a width of \texttt{ViewGroup.LayoutParams.MATCH\_PARENT} and a height of \texttt{ViewGroup.LayoutParams.WRAP\_CONTENT} when the layout's orientation is vertical.
                \item \textbf{LinearLayout.LayoutParams generateLayoutParams(ViewGroup.LayoutParams lp)}: Returns a safe set of layout parameters based on the supplied layout parameters.
                \item \textbf{void onDraw(Canvas canvas)}: Implement this method to perform custom drawing operations on the layout.
                \item \textbf{void onLayout(boolean changed, int l, int t, int r, int b)}: Called from layout when this view should assign a size and position to each of its children.
                \item \textbf{void onMeasure(int widthMeasureSpec, int heightMeasureSpec)}: Measures the view and its content to determine the measured width and height.
            \end{itemize}

        \item \textbf{Constants}:
            \begin{itemize}
                \item \textbf{int HORIZONTAL}: Constant indicating a horizontal orientation for the layout.
                \item \textbf{int SHOW\_DIVIDER\_BEGINNING}: Show a divider at the beginning of the group.
                \item \textbf{int SHOW\_DIVIDER\_END}: Show a divider at the end of the group.
                \item \textbf{int SHOW\_DIVIDER\_MIDDLE}: Show dividers between each item in the group.
                \item \textbf{int SHOW\_DIVIDER\_NONE}: Do not show any dividers.
                \item \textbf{int VERTICAL}: Constant indicating a vertical orientation for the layout.
            \end{itemize}


    \end{itemize}

    \pagebreak 
    \subsubsection{LinearLayout.LayoutParams}
    \begin{itemize}
        \item \textbf{Hierarchy}:
            \begin{center}
                java.lang.Object $\to $	android.view.ViewGroup.LayoutParams $\to $	android.view.ViewGroup.MarginLayoutParams $\to $	android.widget.LinearLayout.LayoutParams
            \end{center}
        \item \textbf{Include}
            \bigbreak \noindent 
            \begin{javacode}
                android.widget.LinearLayout.LayoutParams
            \end{javacode}
        \item \textbf{Constructors}:
            \bigbreak \noindent 
            \begin{javacode}
                LayoutParams(Context c, AttributeSet attrs)
                LayoutParams(ViewGroup.LayoutParams p)
                LayoutParams(ViewGroup.MarginLayoutParams source)
                LayoutParams(LinearLayout.LayoutParams source)
                LayoutParams(int width, int height)
                LayoutParams(int width, int height, float weight)
            \end{javacode}
        \item \textbf{Public methods}:
            \begin{itemize}
                \item \textbf{String	debug(String output)}:
            \end{itemize}
        \item \textbf{Fields}:
            \begin{itemize}
                \item \textbf{public int	gravity}: Gravity for the view associated with these LayoutParams.
                \item \textbf{public float	weight}: Indicates how much of the extra space in the LinearLayout will be allocated to the view associated with these LayoutParams.
            \end{itemize}
    \end{itemize}

    \pagebreak 
    \subsubsection{GridLayout}
    \begin{itemize}
        \item \textbf{Hierarchy}: 
            \begin{center}
                java.lang.Object $\to$	android.view.View $\to$	android.view.ViewGroup $\to$	android.widget.GridLayout
            \end{center}
        \item \textbf{Include}:
            \bigbreak \noindent 
            \begin{javacode}
            android.widget.GridLayout
            \end{javacode}
        \item \textbf{Constructors}
            \bigbreak \noindent 
            \begin{javacode}
                GridLayout(Context context)
                GridLayout(Context context, AttributeSet attrs)
                GridLayout(Context context, AttributeSet attrs, int defStyleAttr)
                GridLayout(Context context, AttributeSet attrs, int defStyleAttr, int defStyleRes)
            \end{javacode}
        \item \textbf{Public methods}
            \begin{itemize}
                \item \textbf{GridLayout.LayoutParams generateLayoutParams(AttributeSet attrs)}: Returns a new set of layout parameters based on the supplied attribute set.
                \item \textbf{CharSequence getAccessibilityClassName()}: Returns the class name of this object to be used for accessibility purposes.
                \item \textbf{int getAlignmentMode()}: Returns the current alignment mode used for positioning children within their grid cells.
                \item \textbf{int getColumnCount()}: Returns the current number of columns.
                \item \textbf{int getOrientation()}: Returns the current orientation (horizontal or vertical).
                \item \textbf{int getRowCount()}: Returns the current number of rows.
                \item \textbf{boolean getUseDefaultMargins()}: Returns whether this \texttt{GridLayout} will allocate default margins when none are defined in layout parameters.
                \item \textbf{boolean isColumnOrderPreserved()}: Returns whether column boundaries are ordered by their grid indices.
                \item \textbf{boolean isRowOrderPreserved()}: Returns whether row boundaries are ordered by their grid indices.
                \item \textbf{void onViewAdded(View child)}: Called when a new child view is added to this \texttt{ViewGroup}.
                \item \textbf{void onViewRemoved(View child)}: Called when a child view is removed from this \texttt{ViewGroup}.
                \item \textbf{void requestLayout()}: Call this when something has changed that invalidates the layout of this view.
                \item \textbf{void setAlignmentMode(int alignmentMode)}: Sets the alignment mode used for alignments between children of this container.
                \item \textbf{void setColumnCount(int columnCount)}: Sets the total number of columns. Used to generate default column indices when none are specified.
                \item \textbf{void setColumnOrderPreserved(boolean columnOrderPreserved)}: When true, forces \texttt{GridLayout} to place column boundaries so their grid indices appear in ascending order.
                \item \textbf{void setOrientation(int orientation)}: Sets the layout’s orientation. This controls:
                    \begin{itemize}
                        \item The direction in which default row/column indices are generated when not specified.
                        \item Whether the grid is treated as row-major or column-major.
                    \end{itemize}
                \item \textbf{void setRowCount(int rowCount)}: Sets the total number of rows. Used to generate default row indices when none are specified.
                \item \textbf{void setRowOrderPreserved(boolean rowOrderPreserved)}: When true, forces \texttt{GridLayout} to place row boundaries in ascending order of their grid indices.
                \item \textbf{void setUseDefaultMargins(boolean useDefaultMargins)}: When true, \texttt{GridLayout} allocates default margins based on child visual characteristics.
                \item \textbf{static GridLayout.Spec spec(int start, float weight)}: Equivalent to \texttt{spec(start, 1, weight)}.
                \item \textbf{static GridLayout.Spec spec(int start)}: Returns a \texttt{Spec} where:
                    \begin{itemize}
                        \item \texttt{span = [start, start + 1]}
                        \item To leave the start index undefined, use \texttt{UNDEFINED}.
                    \end{itemize}
                \item \textbf{static GridLayout.Spec spec(int start, int size, GridLayout.Alignment alignment, float weight)}: Returns a \texttt{Spec} where:
                    \begin{itemize}
                        \item \texttt{span = [start, start + size]}
                        \item \texttt{alignment = alignment}
                        \item \texttt{weight = weight}
                        \item To leave the start index undefined, use \texttt{UNDEFINED}.
                    \end{itemize}
                \item \textbf{static GridLayout.Spec spec(int start, GridLayout.Alignment alignment, float weight)}: Equivalent to \texttt{spec(start, 1, alignment, weight)}.
                \item \textbf{static GridLayout.Spec spec(int start, int size, GridLayout.Alignment alignment)}: Equivalent to \texttt{spec(start, size, alignment, 0f)}.
                \item \textbf{static GridLayout.Spec spec(int start, GridLayout.Alignment alignment)}: Returns a \texttt{Spec} where:
                    \begin{itemize}
                        \item \texttt{span = [start, start + 1]}
                        \item \texttt{alignment = alignment}
                        \item To leave the start index undefined, use \texttt{UNDEFINED}.
                    \end{itemize}
                \item \textbf{static GridLayout.Spec spec(int start, int size, float weight)}: Equivalent to \texttt{spec(start, size, default\_alignment, weight)}, where \texttt{default\_alignment} is defined in \texttt{GridLayout.LayoutParams}.
                \item \textbf{static GridLayout.Spec spec(int start, int size)}: Returns a \texttt{Spec} where:
                    \begin{itemize}
                        \item \texttt{span = [start, start + size]}
                        \item To leave the start index undefined, use \texttt{UNDEFINED}.
                    \end{itemize}
            \end{itemize}

        \item \textbf{Protected methods}
            \begin{itemize}
                \item \textbf{boolean checkLayoutParams(ViewGroup.LayoutParams p)}: Determines whether the supplied layout parameters are valid for this layout.
                \item \textbf{GridLayout.LayoutParams generateDefaultLayoutParams()}: Returns a set of default layout parameters used by this \texttt{GridLayout}.
                \item \textbf{GridLayout.LayoutParams generateLayoutParams(ViewGroup.LayoutParams lp)}: Returns a safe set of layout parameters based on the supplied layout parameters.
                \item \textbf{void onLayout(boolean changed, int left, int top, int right, int bottom)}: Called during layout when this view should assign a size and position to each of its children.
                \item \textbf{void onMeasure(int widthSpec, int heightSpec)}: Measures the view and its content to determine the measured width and height.
            \end{itemize}

        \item \textbf{Fields}
            \begin{itemize}
                \item \textbf{public static final GridLayout.Alignment BASELINE}: Indicates that a view should be aligned with the baselines of the other views in its cell group.
                \item \textbf{public static final GridLayout.Alignment BOTTOM}: Indicates that a view should be aligned with the bottom edges of the other views in its cell group.
                \item \textbf{public static final GridLayout.Alignment CENTER}: Indicates that a view should be centered with the other views in its cell group.
                \item \textbf{public static final GridLayout.Alignment END}: Indicates that a view should be aligned with the end edges of the other views in its cell group.
                \item \textbf{public static final GridLayout.Alignment FILL}: Indicates that a view should expand to fill the boundaries of its cell group.
                \item \textbf{public static final GridLayout.Alignment LEFT}: Indicates that a view should be aligned with the left edges of the other views in its cell group.
                \item \textbf{public static final GridLayout.Alignment RIGHT}: Indicates that a view should be aligned with the right edges of the other views in its cell group.
                \item \textbf{public static final GridLayout.Alignment START}: Indicates that a view should be aligned with the start edges of the other views in its cell group.
                \item \textbf{public static final GridLayout.Alignment TOP}: Indicates that a view should be aligned with the top edges of the other views in its cell group.
            \end{itemize}

        \item \textbf{Constants}
            \begin{itemize}
                \item \textbf{int ALIGN\_BOUNDS}: Constant representing an \texttt{alignmentMode}. Child bounds are aligned within their grid cells.
                \item \textbf{int ALIGN\_MARGINS}: Constant representing an \texttt{alignmentMode}. Child margins are aligned within their grid cells.
                \item \textbf{int HORIZONTAL}: Constant representing a horizontal orientation for the layout.
                \item \textbf{int UNDEFINED}: Constant used to indicate that a value is undefined.
                \item \textbf{int VERTICAL}: Constant representing a vertical orientation for the layout.
            \end{itemize}

    \end{itemize}

    \pagebreak 
    \subsubsection{GridLayout.LayoutParams}
    \begin{itemize}
        \item \textbf{Hierarchy}: 
            \begin{center}
                java.lang.Object $\to $	android.view.ViewGroup.LayoutParams $\to $	android.view.ViewGroup.MarginLayoutParams $\to$	android.widget.GridLayout.LayoutParams
            \end{center}
        \item \textbf{Include}:
            \bigbreak \noindent 
            \begin{javacode}
                android.widget.GridLayout.LayoutParams
            \end{javacode}
        \item \textbf{Constructors}:
            \bigbreak \noindent 
            \begin{javacode}
                LayoutParams()
                LayoutParams(Context context, AttributeSet attrs)
                LayoutParams(ViewGroup.LayoutParams params)
                LayoutParams(ViewGroup.MarginLayoutParams params)
                LayoutParams(GridLayout.LayoutParams source)
                LayoutParams(GridLayout.Spec rowSpec, GridLayout.Spec columnSpec)
            \end{javacode}
            \bigbreak \noindent 
            \textbf{Note:} Values not defined in the attribute set take the default values defined in LayoutParams.
        \item \textbf{Public methods}
            \begin{itemize}
                \item \textbf{boolean equals(Object o)}: Indicates whether some other object is "equal to" this one.
                \item \textbf{int hashCode()}: Returns a hash code value for the object.
                \item \textbf{void setGravity(int gravity)}: Describes how the child views are positioned.
            \end{itemize}
        \item \textbf{Protected methods}
            \begin{itemize}
                \item \textbf{void setBaseAttributes(TypedArray attributes, int widthAttr, int heightAttr)}: Extracts the layout parameters from the supplied attributes.
            \end{itemize}
        \item \textbf{Fields}
            \begin{itemize}
                \item \textbf{public GridLayout.Spec columnSpec}: The spec that defines the horizontal characteristics of the cell group described by these layout parameters.
                \item \textbf{public GridLayout.Spec rowSpec}: The spec that defines the vertical characteristics of the cell group described by these layout parameters.
            \end{itemize}

    \end{itemize}

    \pagebreak 
    \subsubsection{TableLayout}
    \begin{itemize}
        \item \textbf{Hierarchy}:
            \begin{center}
                java.lang.Object $\to $	android.view.View $\to $	android.view.ViewGroup $\to $	android.widget.LinearLayout $\to $	android.widget.TableLayout
            \end{center}
        \item \textbf{Include}
            \bigbreak \noindent 
            \begin{javacode}
                android.widget.TableLayout
            \end{javacode}
        \item \textbf{Constructors}
            \bigbreak \noindent 
            \begin{javacode}
                TableLayout(Context context)
                TableLayout(Context context, AttributeSet attrs)
            \end{javacode}
        \item \textbf{Public methods}
            \begin{itemize}
                \item \textbf{void addView(View child, int index)}: Adds a child view at the specified index.
                \item \textbf{void addView(View child, ViewGroup.LayoutParams params)}: Adds a child view with the specified layout parameters.
                \item \textbf{void addView(View child)}: Adds a child view to the layout.
                \item \textbf{void addView(View child, int index, ViewGroup.LayoutParams params)}: Adds a child view at the specified index with the given layout parameters.
                \item \textbf{TableLayout.LayoutParams generateLayoutParams(AttributeSet attrs)}: Returns a new set of layout parameters based on the supplied attribute set.
                \item \textbf{CharSequence getAccessibilityClassName()}: Returns the class name of this object to be used for accessibility purposes.
                \item \textbf{boolean isColumnCollapsed(int columnIndex)}: Returns the collapsed state of the specified column.
                \item \textbf{boolean isColumnShrinkable(int columnIndex)}: Returns whether the specified column is shrinkable.
                \item \textbf{boolean isColumnStretchable(int columnIndex)}: Returns whether the specified column is stretchable.
                \item \textbf{boolean isShrinkAllColumns()}: Indicates whether all columns in the table are shrinkable.
                \item \textbf{boolean isStretchAllColumns()}: Indicates whether all columns in the table are stretchable.
                \item \textbf{void requestLayout()}: Call this when something has changed that invalidates the layout of this view.
                \item \textbf{void setColumnCollapsed(int columnIndex, boolean isCollapsed)}: Collapses or restores the specified column.
                \item \textbf{void setColumnShrinkable(int columnIndex, boolean isShrinkable)}: Sets whether the specified column can shrink if necessary.
                \item \textbf{void setColumnStretchable(int columnIndex, boolean isStretchable)}: Sets whether the specified column can stretch to fill available space.
                \item \textbf{void setOnHierarchyChangeListener(ViewGroup.OnHierarchyChangeListener listener)}: Registers a listener to be notified when a child view is added to or removed from this layout.
                \item \textbf{void setShrinkAllColumns(boolean shrinkAllColumns)}: Convenience method to mark all columns as shrinkable.
                \item \textbf{void setStretchAllColumns(boolean stretchAllColumns)}: Convenience method to mark all columns as stretchable.
            \end{itemize}

        \item \textbf{Protected methods}
            \begin{itemize}
                \item \textbf{boolean checkLayoutParams(ViewGroup.LayoutParams p)}: Determines whether the supplied layout parameters are valid for this layout.
                \item \textbf{LinearLayout.LayoutParams generateDefaultLayoutParams()}: Returns a set of default layout parameters with a width of \texttt{ViewGroup.LayoutParams.MATCH\_PARENT} and a height of \texttt{ViewGroup.LayoutParams.WRAP\_CONTENT}.
                \item \textbf{LinearLayout.LayoutParams generateLayoutParams(ViewGroup.LayoutParams p)}: Returns a safe set of layout parameters based on the supplied layout parameters.
                \item \textbf{void onLayout(boolean changed, int l, int t, int r, int b)}: Called during layout when this view should assign a size and position to each of its children.
                \item \textbf{void onMeasure(int widthMeasureSpec, int heightMeasureSpec)}: Measures the view and its content to determine the measured width and height.
            \end{itemize}

    \end{itemize}

    \pagebreak 
    \subsubsection{TableLayout.LayoutParams}
    \begin{itemize}
        \item \textbf{Hierarchy} 
            \begin{center}
                java.lang.Object $\to $	android.view.ViewGroup.LayoutParams $\to $	android.view.ViewGroup.MarginLayoutParams $\to $	android.widget.LinearLayout.LayoutParams $\to $	android.widget.TableLayout.LayoutParams
            \end{center}
        \item \textbf{Include}
            \bigbreak \noindent 
            \begin{javacode}
            android.widget.TableLayout.LayoutParams
            \end{javacode}
        \item \textbf{Constructors}
            \bigbreak \noindent 
            \begin{javacode}
                LayoutParams()
                LayoutParams(Context c, AttributeSet attrs)
                LayoutParams(ViewGroup.LayoutParams p)
                LayoutParams(ViewGroup.MarginLayoutParams source)
                LayoutParams(int w, int h)
                LayoutParams(int w, int h, float initWeight)
            \end{javacode}
        \item \textbf{Protected methods}
            \begin{itemize}
                \item \textbf{void setBaseAttributes(TypedArray a, int widthAttr, int heightAttr)}: Fixes the row's width to ViewGroup.LayoutParams.MATCH\_PARENT; the row's height is fixed to ViewGroup.LayoutParams.WRAP\_CONTENT if no layout height is specified.
            \end{itemize}
    \end{itemize}

    \pagebreak 
    \subsubsection{FrameLayout}
    \begin{itemize}
        \item \textbf{Hierarchy} 
            \begin{center}
                java.lang.Object $\to$	android.view.View $\to$	android.view.ViewGroup $\to $	android.widget.FrameLayout
            \end{center}
        \item \textbf{Include}
            \bigbreak \noindent 
            \begin{javacode}
                android.widget.FrameLayout
            \end{javacode}
        \item \textbf{Constructors}
            \bigbreak \noindent 
            \begin{javacode}
                FrameLayout(Context context)
                FrameLayout(Context context, AttributeSet attrs)
                FrameLayout(Context context, AttributeSet attrs, int defStyleAttr, int defStyleRes)
                FrameLayout(Context context, AttributeSet attrs, int defStyleAttr)
            \end{javacode}
        \item \textbf{Public methods}
            \begin{itemize}
                \item \textbf{FrameLayout.LayoutParams generateLayoutParams(AttributeSet attrs)}: Returns a new set of layout parameters based on the supplied attribute set.
                \item \textbf{CharSequence getAccessibilityClassName()}: Returns the class name of this object to be used for accessibility purposes.
                \item \textbf{boolean getConsiderGoneChildrenWhenMeasuring()}: \textit{(Deprecated in API level 15)} — Previously determined whether to include \texttt{GONE} children in measurement. Replaced by \texttt{getMeasureAllChildren()} for naming consistency.
                \item \textbf{boolean getMeasureAllChildren()}: Determines whether all children, or only those in the \texttt{VISIBLE} or \texttt{INVISIBLE} state, are considered during measurement.
                \item \textbf{void setForegroundGravity(int foregroundGravity)}: Describes how the foreground drawable is positioned within the layout.
                \item \textbf{void setMeasureAllChildren(boolean measureAll)}: Sets whether all children, or only visible ones, should be considered when measuring the layout.
                \item \textbf{boolean shouldDelayChildPressedState()}: Returns true if the pressed state should be delayed for children or descendants of this \texttt{ViewGroup}.
            \end{itemize}
        \item \textbf{Protected methods}
            \begin{itemize}
                \item \textbf{boolean checkLayoutParams(ViewGroup.LayoutParams p)}: Determines whether the supplied layout parameters are valid for this layout.
                \item \textbf{FrameLayout.LayoutParams generateDefaultLayoutParams()}: Returns a set of default layout parameters with both width and height set to \texttt{ViewGroup.LayoutParams.MATCH\_PARENT}.
                \item \textbf{ViewGroup.LayoutParams generateLayoutParams(ViewGroup.LayoutParams lp)}: Returns a safe set of layout parameters based on the supplied layout parameters.
                \item \textbf{void onLayout(boolean changed, int left, int top, int right, int bottom)}: Called during layout when this view should assign a size and position to each of its children.
                \item \textbf{void onMeasure(int widthMeasureSpec, int heightMeasureSpec)}: Measures the view and its content to determine the measured width and height.
            \end{itemize}

        
    \end{itemize}

    \pagebreak 
    \subsubsection{FrameLayout.LayoutParams}
    \begin{itemize}
        \item \textbf{Hierarchy}        
            \begin{center}
                java.lang.Object $\to$	android.view.ViewGroup.LayoutParams $\to$	android.view.ViewGroup.MarginLayoutParams $\to$	android.widget.FrameLayout.LayoutParams
            \end{center}
        \item \textbf{Include}
            \bigbreak \noindent 
            \begin{javacode}
                android.widget.FrameLayout.LayoutParams
            \end{javacode}
        \item \textbf{Constructors}
            \bigbreak \noindent 
            \begin{javacode}
                LayoutParams(Context c, AttributeSet attrs)
                LayoutParams(ViewGroup.LayoutParams source)
                LayoutParams(ViewGroup.MarginLayoutParams source)
                LayoutParams(FrameLayout.LayoutParams source)
                LayoutParams(int width, int height)
                LayoutParams(int width, int height, int gravity)
            \end{javacode}
        \item \textbf{Fields}
            \begin{itemize}
                \item \textbf{public int gravity}: The gravity to apply with the View to which these layout parameters are associated.
            \end{itemize}
        \item \textbf{Constants}
            \begin{itemize}
                \item \textbf{int UNSPECIFIED\_GRAVITY}: Value for gravity indicating that a gravity has not been explicitly specified.
            \end{itemize}
    \end{itemize}

    \pagebreak 
    \subsubsection{ListView}
    \begin{itemize}
        \item \textbf{Hierarchy}:       
            \begin{center}
                java.lang.Object $\to$	android.view.View $\to$	android.view.ViewGroup $\to$	android.widget.AdapterView<android.widget.ListAdapter> $\to$	android.widget.AbsListView $\to$	android.widget.ListView
            \end{center}
        \item \textbf{Include}
            \bigbreak \noindent 
            \begin{javacode}
                android.widget.ListView
            \end{javacode}
        \item \textbf{Constructors}
            \bigbreak \noindent 
            \begin{javacode}
                ListView(Context context)
                ListView(Context context, AttributeSet attrs)
                ListView(Context context, AttributeSet attrs, int defStyleAttr)
                ListView(Context context, AttributeSet attrs, int defStyleAttr, int defStyleRes)
            \end{javacode}
        \item \textbf{Public methods}
            \begin{itemize}
                \item \textbf{void addFooterView(View v)}: Adds a fixed view to appear at the bottom of the list.
                \item \textbf{void addFooterView(View v, Object data, boolean isSelectable)}: Adds a fixed view to appear at the bottom of the list with optional data and selectable state.
                \item \textbf{void addHeaderView(View v, Object data, boolean isSelectable)}: Adds a fixed view to appear at the top of the list with optional data and selectable state.
                \item \textbf{void addHeaderView(View v)}: Adds a fixed view to appear at the top of the list.
                \item \textbf{boolean areFooterDividersEnabled()}: Returns whether footer dividers are currently enabled.
                \item \textbf{boolean areHeaderDividersEnabled()}: Returns whether header dividers are currently enabled.
                \item \textbf{boolean dispatchKeyEvent(KeyEvent event)}: Dispatches a key event to the next view on the focus path.
                \item \textbf{CharSequence getAccessibilityClassName()}: Returns the accessibility class name. A \texttt{TYPE\_VIEW\_SCROLLED} event should be sent whenever a scroll happens, even if position and child count remain unchanged.
                \item \textbf{ListAdapter getAdapter()}: Returns the adapter currently in use by this \texttt{ListView}.
                \item \textbf{long[] getCheckItemIds()}: \textit{(Deprecated in API 15)} — Use \texttt{AbsListView.getCheckedItemIds()} instead. Returns the IDs of the currently checked items.
                \item \textbf{Drawable getDivider()}: Returns the drawable that is drawn between each list item.
                \item \textbf{int getDividerHeight()}: Returns the height of the divider between list items.
                \item \textbf{int getFooterViewsCount()}: Returns the number of fixed footer views currently added.
                \item \textbf{int getHeaderViewsCount()}: Returns the number of fixed header views currently added.
                \item \textbf{boolean getItemsCanFocus()}: Returns whether list items can contain focusable elements.
                \item \textbf{int getMaxScrollAmount()}: Returns the maximum amount the list can scroll in response to arrow events.
                \item \textbf{Drawable getOverscrollFooter()}: Returns the drawable drawn below all list content when overscrolling.
                \item \textbf{Drawable getOverscrollHeader()}: Returns the drawable drawn above all list content when overscrolling.
                \item \textbf{boolean isOpaque()}: Indicates whether this view is opaque.
                \item \textbf{void onInitializeAccessibilityNodeInfoForItem(View view, int position, AccessibilityNodeInfo info)}: Initializes accessibility node info for a specific list item.
                \item \textbf{boolean onKeyDown(int keyCode, KeyEvent event)}: Default implementation handles key presses such as \texttt{KEYCODE\_DPAD\_CENTER} or \texttt{KEYCODE\_ENTER} to trigger selection.
                \item \textbf{boolean onKeyMultiple(int keyCode, int repeatCount, KeyEvent event)}: Default implementation always returns false (does not handle multiple key events).
                \item \textbf{boolean onKeyUp(int keyCode, KeyEvent event)}: Default implementation handles key releases such as \texttt{KEYCODE\_DPAD\_CENTER}, \texttt{KEYCODE\_ENTER}, or \texttt{KEYCODE\_SPACE} to perform clicks.
                \item \textbf{boolean removeFooterView(View v)}: Removes a previously added footer view.
                \item \textbf{boolean removeHeaderView(View v)}: Removes a previously added header view.
                \item \textbf{boolean requestChildRectangleOnScreen(View child, Rect rect, boolean immediate)}: Called when a child requests that a specific rectangle within it be visible on the screen.
                \item \textbf{void setAdapter(ListAdapter adapter)}: Sets the adapter that provides the data and views for this \texttt{ListView}.
                \item \textbf{void setCacheColorHint(int color)}: When set to a nonzero value, indicates that the list is drawn on top of a solid, opaque background of that color.
                \item \textbf{void setDivider(Drawable divider)}: Sets the drawable that will be drawn between list items.
                \item \textbf{void setDividerHeight(int height)}: Sets the height of the divider drawn between list items.
                \item \textbf{void setFooterDividersEnabled(boolean footerDividersEnabled)}: Enables or disables drawing of dividers for footer views.
                \item \textbf{void setHeaderDividersEnabled(boolean headerDividersEnabled)}: Enables or disables drawing of dividers for header views.
                \item \textbf{void setItemsCanFocus(boolean itemsCanFocus)}: Indicates whether views created by the adapter can contain focusable items.
                \item \textbf{void setOverscrollFooter(Drawable footer)}: Sets the drawable to be drawn below all list content during overscroll.
                \item \textbf{void setOverscrollHeader(Drawable header)}: Sets the drawable to be drawn above all list content during overscroll.
                \item \textbf{void setRemoteViewsAdapter(Intent intent)}: Sets up this \texttt{ListView} to use a remote views adapter connected via a \texttt{RemoteViewsService}.
                \item \textbf{void setSelection(int position)}: Sets the currently selected item in the list.
                \item \textbf{void setSelectionAfterHeaderView()}: Sets the selection to the first list item following the header views.
                \item \textbf{void smoothScrollByOffset(int offset)}: Smoothly scrolls the list by the specified adapter position offset.
                \item \textbf{void smoothScrollToPosition(int position)}: Smoothly scrolls to the specified adapter position.
            \end{itemize}

        \item \textbf{Protected methods}
            \begin{itemize}
                \item \textbf{boolean canAnimate()}: Indicates whether the view group can animate its children after the first layout pass.
                \item \textbf{void dispatchDraw(Canvas canvas)}: Called by the system’s \texttt{draw()} method to render all child views within this layout.
                \item \textbf{boolean drawChild(Canvas canvas, View child, long drawingTime)}: Draws a single child view of this \texttt{ViewGroup} onto the provided \texttt{Canvas}.
                \item \textbf{void layoutChildren()}: Abstract method that subclasses must override to define how their child views are positioned and sized.
                \item \textbf{void onDetachedFromWindow()}: Called when the view is detached from its window, typically used to clean up resources or listeners.
                \item \textbf{void onFinishInflate()}: Called after a view and all its children have been inflated from XML to perform any final initialization.
                \item \textbf{void onFocusChanged(boolean gainFocus, int direction, Rect previouslyFocusedRect)}: Invoked when the view’s focus state changes, providing the direction and previously focused rectangle.
                \item \textbf{void onMeasure(int widthMeasureSpec, int heightMeasureSpec)}: Measures the view and its children to determine the overall measured width and height.
                \item \textbf{void onSizeChanged(int w, int h, int oldw, int oldh)}: Called during layout when the view’s size changes, allowing for recalculation of layout-dependent properties.
            \end{itemize}


    \end{itemize}

    \pagebreak 
    \subsubsection{TextView}
    \begin{itemize}
        \item \textbf{Hierarchy} 
            \begin{center}
                java.lang.Object $\to$	android.view.View $\to$	android.widget.TextView
            \end{center}
        \item \textbf{Include}
            \bigbreak \noindent 
            \begin{javacode}
                android.widget.TextView
            \end{javacode}
        \item \textbf{Constructors}
            \bigbreak \noindent 
            \begin{javacode}
                TextView(Context context)
                TextView(Context context, AttributeSet attrs)
                TextView(Context context, AttributeSet attrs, int defStyleAttr)
                TextView(Context context, AttributeSet attrs, int defStyleAttr, int defStyleRes) 
            \end{javacode}
        \item \textbf{Public methods}
            \begin{itemize}
                \item \textbf{void addExtraDataToAccessibilityNodeInfo(AccessibilityNodeInfo info, String extraDataKey, Bundle arguments)}: Adds extra data to an \texttt{AccessibilityNodeInfo} based on an explicit request for the additional data.
                \item \textbf{void addTextChangedListener(TextWatcher watcher)}: Adds a \texttt{TextWatcher} whose methods are called whenever this \texttt{TextView}'s text changes.
                \item \textbf{final void append(CharSequence text)}: Appends text to the display buffer, upgrading to \texttt{BufferType.EDITABLE} if needed.
                \item \textbf{void append(CharSequence text, int start, int end)}: Appends a slice of text to the display buffer, upgrading to \texttt{BufferType.EDITABLE} if needed.
                \item \textbf{void autofill(AutofillValue value)}: Automatically fills this view’s content with the provided value.
                \item \textbf{void beginBatchEdit()}: Begins a batch edit session.
                \item \textbf{boolean bringPointIntoView(int offset)}: Moves the character offset into view if needed.
                \item \textbf{boolean bringPointIntoView(int offset, boolean requestRectWithoutFocus)}: Moves the insertion position at the given offset into the visible area.
                \item \textbf{void cancelLongPress()}: Cancels a pending long press.
                \item \textbf{void clearComposingText()}:\ Uses \texttt{BaseInputConnection.removeComposingSpans()} to clear IME composing state.
                \item \textbf{void computeScroll()}: Requests the child to update \texttt{mScrollX} and \texttt{mScrollY} if necessary.
                \item \textbf{void debug(int depth)}: Outputs debug information with the given depth.
                \item \textbf{boolean didTouchFocusSelect()}:\ During a touch gesture, returns true iff initial touch moved focus to this view and changed selection.
                \item \textbf{void drawableHotspotChanged(float x, float y)}: Propagates view hotspot changes to drawables/children.
                \item \textbf{void endBatchEdit()}: Ends a batch edit session.
                \item \textbf{boolean extractText(ExtractedTextRequest request, ExtractedText outText)}: Extracts a portion of editable content into \texttt{outText}.
                \item \textbf{void findViewsWithText(ArrayList<View> outViews, CharSequence searched, int flags)}: Finds views containing the given text.
                \item \textbf{CharSequence getAccessibilityClassName()}: Returns the accessibility class name.
                \item \textbf{final int getAutoLinkMask()}: Gets the autolink mask.
                \item \textbf{int getAutoSizeMaxTextSize()}: Returns max auto-size text size.
                \item \textbf{int getAutoSizeMinTextSize()}: Returns min auto-size text size.
                \item \textbf{int getAutoSizeStepGranularity()}: Returns auto-size step granularity.
                \item \textbf{int[] getAutoSizeTextAvailableSizes()}: Returns available auto-size text sizes.
                \item \textbf{int getAutoSizeTextType()}: Returns the auto-size text type.
                \item \textbf{String[] getAutofillHints()}: Gets autofill hints for \texttt{AutofillService}.
                \item \textbf{int getAutofillType()}: Describes the autofill type for this view.
                \item \textbf{AutofillValue getAutofillValue()}: Returns current text as an \texttt{AutofillValue}.
                \item \textbf{int getBaseline()}: Returns the text baseline offset from the top.
                \item \textbf{int getBreakStrategy()}: Gets the paragraph line-break strategy.
                \item \textbf{int getCompoundDrawablePadding()}: Returns padding between compound drawables and text.
                \item \textbf{BlendMode getCompoundDrawableTintBlendMode()}: Returns the tint blend mode for compound drawables.
                \item \textbf{ColorStateList getCompoundDrawableTintList()}: Returns the tint list for compound drawables.
                \item \textbf{PorterDuff.Mode getCompoundDrawableTintMode()}: Returns the Porter-Duff tint mode for compound drawables.
                \item \textbf{Drawable[] getCompoundDrawables()}: Returns left, top, right, bottom drawables.
                \item \textbf{Drawable[] getCompoundDrawablesRelative()}: Returns start, top, end, bottom drawables.
                \item \textbf{int getCompoundPaddingBottom()}: Returns bottom padding plus drawable space.
                \item \textbf{int getCompoundPaddingEnd()}: Returns end padding plus drawable space.
                \item \textbf{int getCompoundPaddingLeft()}: Returns left padding plus drawable space.
                \item \textbf{int getCompoundPaddingRight()}: Returns right padding plus drawable space.
                \item \textbf{int getCompoundPaddingStart()}: Returns start padding plus drawable space.
                \item \textbf{int getCompoundPaddingTop()}: Returns top padding plus drawable space.
                \item \textbf{final int getCurrentHintTextColor()}: Returns current hint text color.
                \item \textbf{final int getCurrentTextColor()}: Returns current text color.
                \item \textbf{ActionMode.Callback getCustomInsertionActionModeCallback()}: Gets the custom insertion \texttt{ActionMode} callback.
                \item \textbf{ActionMode.Callback getCustomSelectionActionModeCallback()}: Gets the custom selection \texttt{ActionMode} callback.
                \item \textbf{Editable getEditableText()}: Returns the text as an \texttt{Editable}.
                \item \textbf{TextUtils.TruncateAt getEllipsize()}: Returns where long text is ellipsized.
                \item \textbf{CharSequence getError()}: Returns the current error message or \texttt{null}.
                \item \textbf{int getExtendedPaddingBottom()}: Returns extended bottom padding.
                \item \textbf{int getExtendedPaddingTop()}: Returns extended top padding.
                \item \textbf{InputFilter[] getFilters()}: Returns the list of input filters.
                \item \textbf{int getFirstBaselineToTopHeight()}: Distance from first baseline to top.
                \item \textbf{void getFocusedRect(Rect r)}: Fills \texttt{r} with the focus search rectangle.
                \item \textbf{int getFocusedSearchResultHighlightColor()}: Gets focused search result highlight color.
                \item \textbf{int getFocusedSearchResultIndex()}: Gets focused search result index.
                \item \textbf{String getFontFeatureSettings()}: Returns font feature settings.
                \item \textbf{String getFontVariationSettings()}: Returns font variation settings.
                \item \textbf{boolean getFreezesText()}: Whether full text is saved in icicles.
                \item \textbf{int getGravity()}: Returns horizontal/vertical text gravity.
                \item \textbf{int getHighlightColor()}: Returns selection highlight color.
                \item \textbf{Highlights getHighlights()}: Returns highlights.
                \item \textbf{CharSequence getHint()}: Returns the hint text.
                \item \textbf{final ColorStateList getHintTextColors()}: Returns hint text colors.
                \item \textbf{int getHyphenationFrequency()}: Gets automatic hyphenation frequency.
                \item \textbf{int getImeActionId()}: Gets IME action ID set via \texttt{setImeActionLabel}.
                \item \textbf{CharSequence getImeActionLabel()}: Gets IME action label.
                \item \textbf{LocaleList getImeHintLocales()}: Gets IME hint locales.
                \item \textbf{int getImeOptions()}: Gets IME options.
                \item \textbf{boolean getIncludeFontPadding()}: Whether extra ascent/descent padding is included.
                \item \textbf{Bundle getInputExtras(boolean create)}: Retrieves/creates input extras bundle.
                \item \textbf{int getInputType()}: Gets the editable content type.
                \item \textbf{int getJustificationMode()}: Gets text justification mode.
                \item \textbf{final KeyListener getKeyListener()}: Gets the current \texttt{KeyListener}.
                \item \textbf{int getLastBaselineToBottomHeight()}: Distance from last baseline to bottom.
                \item \textbf{final Layout getLayout()}: Gets the current text \texttt{Layout}.
                \item \textbf{float getLetterSpacing()}: Gets letter spacing (em).
                \item \textbf{int getLineBounds(int line, Rect bounds)}: Returns baseline for a line; optionally fills \texttt{bounds}.
                \item \textbf{int getLineBreakStyle()}: Gets line-break style.
                \item \textbf{int getLineBreakWordStyle()}: Gets line-break word style.
                \item \textbf{int getLineCount()}: Returns line count or 0 if layout not built.
                \item \textbf{int getLineHeight()}: Returns line height in pixels.
                \item \textbf{float getLineSpacingExtra()}: Returns extra line spacing.
                \item \textbf{float getLineSpacingMultiplier()}: Returns line spacing multiplier.
                \item \textbf{final ColorStateList getLinkTextColors()}: Returns link text colors.
                \item \textbf{final boolean getLinksClickable()}: Whether \texttt{LinkMovementMethod} is auto-set for links.
                \item \textbf{int getMarqueeRepeatLimit()}: Returns marquee repeat count.
                \item \textbf{int getMaxEms()}: Returns max width in ems, or -1.
                \item \textbf{int getMaxHeight()}: Returns max height in px, or -1.
                \item \textbf{int getMaxLines()}: Returns max lines, or -1.
                \item \textbf{int getMaxWidth()}: Returns max width in px, or -1.
                \item \textbf{int getMinEms()}: Returns min width in ems, or -1.
                \item \textbf{int getMinHeight()}: Returns min height in px, or -1.
                \item \textbf{int getMinLines()}: Returns min lines, or -1.
                \item \textbf{int getMinWidth()}: Returns min width in px, or -1.
                \item \textbf{Paint.FontMetrics getMinimumFontMetrics()}: Returns minimum font metrics used for line spacing.
                \item \textbf{final MovementMethod getMovementMethod()}: Gets the \texttt{MovementMethod}.
                \item \textbf{int getOffsetForPosition(float x, float y)}: Returns closest character offset to the given position.
                \item \textbf{TextPaint getPaint()}: Gets the \texttt{TextPaint}.
                \item \textbf{int getPaintFlags()}: Gets the current paint flags.
                \item \textbf{String getPrivateImeOptions()}: Gets private IME options.
                \item \textbf{int getSearchResultHighlightColor()}: Gets search result highlight color.
                \item \textbf{int[] getSearchResultHighlights()}: Gets current search result ranges.
                \item \textbf{int getSelectionEnd()}: Convenience for \texttt{Selection.getSelectionEnd}.
                \item \textbf{int getSelectionStart()}: Convenience for \texttt{Selection.getSelectionStart}.
                \item \textbf{int getShadowColor()}: Gets shadow color.
                \item \textbf{float getShadowDx()}: Gets shadow X offset.
                \item \textbf{float getShadowDy()}: Gets shadow Y offset.
                \item \textbf{float getShadowRadius()}: Gets shadow blur radius.
                \item \textbf{boolean getShiftDrawingOffsetForStartOverhang()}: True if shifting x-offset for start overhang.
                \item \textbf{final boolean getShowSoftInputOnFocus()}: Whether soft input shows on focus.
                \item \textbf{CharSequence getText()}: Returns the displayed text.
                \item \textbf{TextClassifier getTextClassifier()}: Returns the \texttt{TextClassifier}.
                \item \textbf{final ColorStateList getTextColors()}: Returns text colors by state.
                \item \textbf{Drawable getTextCursorDrawable()}: Returns the cursor drawable.
                \item \textbf{TextDirectionHeuristic getTextDirectionHeuristic()}: Returns resolved text direction heuristic.
                \item \textbf{Locale getTextLocale()}: Returns primary text \texttt{Locale}.
                \item \textbf{LocaleList getTextLocales()}: Returns text \texttt{LocaleList}.
                \item \textbf{PrecomputedText.Params getTextMetricsParams()}: Returns precomputed-text layout params.
                \item \textbf{float getTextScaleX()}: Returns horizontal text scale factor.
                \item \textbf{Drawable getTextSelectHandle()}: Returns selection handle drawable.
                \item \textbf{Drawable getTextSelectHandleLeft()}: Returns left selection handle drawable.
                \item \textbf{Drawable getTextSelectHandleRight()}: Returns right selection handle drawable.
                \item \textbf{float getTextSize()}: Returns text size (sp units).
                \item \textbf{int getTextSizeUnit()}: Returns the defined text size unit.
                \item \textbf{int getTotalPaddingBottom()}: Returns total bottom padding.
                \item \textbf{int getTotalPaddingEnd()}: Returns total end padding.
                \item \textbf{int getTotalPaddingLeft()}: Returns total left padding.
                \item \textbf{int getTotalPaddingRight()}: Returns total right padding.
                \item \textbf{int getTotalPaddingStart()}: Returns total start padding.
                \item \textbf{int getTotalPaddingTop()}: Returns total top padding.
                \item \textbf{final TransformationMethod getTransformationMethod()}: Returns the current text transformation method.
                \item \textbf{Typeface getTypeface()}: Returns the current \texttt{Typeface}.
                \item \textbf{URLSpan[] getUrls()}: Returns \texttt{URLSpan}s attached to the text.
                \item \textbf{boolean getUseBoundsForWidth()}: True if bounding box width is used for line breaking/drawing.
                \item \textbf{boolean hasOverlappingRendering()}: Whether the view has overlapping rendering.
                \item \textbf{boolean hasSelection()}: True iff there is a nonzero-length selection.
                \item \textbf{void invalidateDrawable(Drawable drawable)}: Invalidates the given drawable.
                \item \textbf{boolean isAllCaps()}: Whether ALL CAPS transformation is applied.
                \item \textbf{boolean isAutoHandwritingEnabled()}: Whether automatic handwriting initiation is allowed.
                \item \textbf{boolean isCursorVisible()}: Whether the cursor is visible.
                \item \textbf{boolean isElegantTextHeight()}: Gets elegant height metrics flag.
                \item \textbf{boolean isFallbackLineSpacing()}: Whether fallback font ascent/descent is respected.
                \item \textbf{final boolean isHorizontallyScrollable()}: Whether text may be wider than the view.
                \item \textbf{boolean isInputMethodTarget()}: Whether this view is the current input method target.
                \item \textbf{boolean isLocalePreferredLineHeightForMinimumUsed()}: True if locale-preferred line height is used for minimum line height.
                \item \textbf{boolean isSingleLine()}: Whether text is constrained to a single scrolling line.
                \item \textbf{boolean isSuggestionsEnabled()}: Whether suggestions are enabled.
                \item \textbf{boolean isTextSelectable()}: Returns \texttt{textIsSelectable} state.
                \item \textbf{void jumpDrawablesToCurrentState()}: Calls \texttt{jumpToCurrentState()} on associated drawables.
                \item \textbf{int length()}: Returns the text length in characters.
                \item \textbf{boolean moveCursorToVisibleOffset()}: Moves cursor to a visible offset if needed.
                \item \textbf{void onBeginBatchEdit()}: Called when a batch edit begins.
                \item \textbf{boolean onCheckIsTextEditor()}: Whether this view is a text editor.
                \item \textbf{void onCommitCompletion(CompletionInfo text)}: Called on IME completion.
                \item \textbf{void onCommitCorrection(CorrectionInfo info)}: Called on IME auto-correction.
                \item \textbf{InputConnection onCreateInputConnection(EditorInfo outAttrs)}: Creates an \texttt{InputConnection}.
                \item \textbf{void onCreateViewTranslationRequest(int[] supportedFormats, Consumer<ViewTranslationRequest> requestsCollector)}: Collects view translation requests.
                \item \textbf{boolean onDragEvent(DragEvent event)}: Handles drag events.
                \item \textbf{void onEditorAction(int actionCode)}: Called for \texttt{performEditorAction()}.
                \item \textbf{void onEndBatchEdit()}: Called when a batch edit ends.
                \item \textbf{boolean onGenericMotionEvent(MotionEvent event)}: Handles generic motion events.
                \item \textbf{boolean onKeyDown(int keyCode, KeyEvent event)}: Default handling for key down (e.g., DPAD\_CENTER/ENTER).
                \item \textbf{boolean onKeyMultiple(int keyCode, int repeatCount, KeyEvent event)}: Default returns false for multiple key events.
                \item \textbf{boolean onKeyPreIme(int keyCode, KeyEvent event)}: Handles a key event before IME processes it.
                \item \textbf{boolean onKeyShortcut(int keyCode, KeyEvent event)}: Called when key shortcut isn’t handled.
                \item \textbf{boolean onKeyUp(int keyCode, KeyEvent event)}: Default handling for key up (e.g., DPAD\_CENTER/ENTER/SPACE).
                \item \textbf{boolean onPreDraw()}: Called when the view tree is about to be drawn.
                \item \textbf{boolean onPrivateIMECommand(String action, Bundle data)}: Handles private IME commands.
                \item \textbf{ContentInfo onReceiveContent(ContentInfo payload)}: Default content reception.
                \item \textbf{PointerIcon onResolvePointerIcon(MotionEvent event, int pointerIndex)}: Resolves pointer icon for the event.
                \item \textbf{void onRestoreInstanceState(Parcelable state)}: Restores internal state from \texttt{Parcelable}.
                \item \textbf{void onRtlPropertiesChanged(int layoutDirection)}: Called when RTL-related properties change.
                \item \textbf{Parcelable onSaveInstanceState()}: Saves internal state to a \texttt{Parcelable}.
                \item \textbf{void onScreenStateChanged(int screenState)}: Called when the screen state changes.
                \item \textbf{boolean onTextContextMenuItem(int id)}: Handles a text context menu selection.
                \item \textbf{boolean onTouchEvent(MotionEvent event)}: Handles pointer events.
                \item \textbf{boolean onTrackballEvent(MotionEvent event)}: Handles trackball events.
                \item \textbf{void onVisibilityAggregated(boolean isVisible)}: Called when user-visibility may change.
                \item \textbf{void onWindowFocusChanged(boolean hasWindowFocus)}: Called when the containing window’s focus changes.
                \item \textbf{boolean performLongClick()}: Invokes \texttt{OnLongClickListener}, if defined.
                \item \textbf{void removeTextChangedListener(TextWatcher watcher)}: Removes a \texttt{TextWatcher}.
                \item \textbf{void sendAccessibilityEventUnchecked(AccessibilityEvent event)}: Sends an accessibility event without checking if accessibility is enabled.
                \item \textbf{void setAllCaps(boolean allCaps)}: Transforms input to ALL CAPS display.
                \item \textbf{final void setAutoLinkMask(int mask)}: Sets the autolink mask.
                \item \textbf{void setAutoSizeTextTypeUniformWithConfiguration(int autoSizeMinTextSize, int autoSizeMaxTextSize, int autoSizeStepGranularity, int unit)}: Configures uniform auto-size text.
                \item \textbf{void setAutoSizeTextTypeUniformWithPresetSizes(int[] presetSizes, int unit)}: Sets preset sizes for auto-size text.
                \item \textbf{void setAutoSizeTextTypeWithDefaults(int autoSizeTextType)}: Enables default auto-size configuration.
                \item \textbf{void setBreakStrategy(int breakStrategy)}: Sets paragraph break strategy.
                \item \textbf{void setCompoundDrawablePadding(int pad)}: Sets padding between drawables and text.
                \item \textbf{void setCompoundDrawableTintBlendMode(BlendMode blendMode)}: Sets blend mode for compound drawable tints.
                \item \textbf{void setCompoundDrawableTintList(ColorStateList tint)}: Applies tint to compound drawables.
                \item \textbf{void setCompoundDrawableTintMode(PorterDuff.Mode tintMode)}: Sets Porter-Duff blend mode for drawable tints.
                \item \textbf{void setCompoundDrawables(Drawable left, Drawable top, Drawable right, Drawable bottom)}: Sets left/top/right/bottom drawables.
                \item \textbf{void setCompoundDrawablesRelative(Drawable start, Drawable top, Drawable end, Drawable bottom)}: Sets start/top/end/bottom drawables.
                \item \textbf{void setCompoundDrawablesRelativeWithIntrinsicBounds(Drawable start, Drawable top, Drawable end, Drawable bottom)}: Sets start/top/end/bottom drawables with intrinsic bounds.
                \item \textbf{void setCompoundDrawablesRelativeWithIntrinsicBounds(int start, int top, int end, int bottom)}: Same as above with resource IDs.
                \item \textbf{void setCompoundDrawablesWithIntrinsicBounds(Drawable left, Drawable top, Drawable right, Drawable bottom)}: Sets left/top/right/bottom drawables with intrinsic bounds.
                \item \textbf{void setCompoundDrawablesWithIntrinsicBounds(int left, int top, int right, int bottom)}: Same as above with resource IDs.
                \item \textbf{void setCursorVisible(boolean visible)}: Shows/hides the cursor.
                \item \textbf{void setCustomInsertionActionModeCallback(ActionMode.Callback actionModeCallback)}: Sets custom insertion \texttt{ActionMode} callback.
                \item \textbf{void setCustomSelectionActionModeCallback(ActionMode.Callback actionModeCallback)}: Sets custom selection \texttt{ActionMode} callback.
                \item \textbf{final void setEditableFactory(Editable.Factory factory)}: Sets the \texttt{Editable.Factory}.
                \item \textbf{void setElegantTextHeight(boolean elegant)}: Sets elegant height metrics flag.
                \item \textbf{void setEllipsize(TextUtils.TruncateAt where)}: Enables ellipsizing of long words.
                \item \textbf{void setEms(int ems)}: Sets exact width in ems.
                \item \textbf{void setEnabled(boolean enabled)}: Enables/disables the view.
                \item \textbf{void setError(CharSequence error)}: Shows an error icon and message popup.
                \item \textbf{void setError(CharSequence error, Drawable icon)}: Shows a custom error icon and message popup.
                \item \textbf{void setExtractedText(ExtractedText text)}: Applies extracted text to the view.
                \item \textbf{void setFallbackLineSpacing(boolean enabled)}: Respects fallback font ascent/descent when enabled.
                \item \textbf{void setFilters(InputFilter[] filters)}: Sets input filters for editable content.
                \item \textbf{void setFirstBaselineToTopHeight(int firstBaselineToTopHeight)}: Adjusts top padding so first baseline is at the given distance from top.
                \item \textbf{void setFocusedSearchResultHighlightColor(int color)}: Sets focused search result highlight color.
                \item \textbf{void setFocusedSearchResultIndex(int index)}: Sets focused search result index.
                \item \textbf{void setFontFeatureSettings(String fontFeatureSettings)}: Sets font feature settings.
                \item \textbf{boolean setFontVariationSettings(String fontVariationSettings)}: Sets TrueType/OpenType variation settings.
                \item \textbf{void setFreezesText(boolean freezesText)}: Controls saving full text on instance state.
                \item \textbf{void setGravity(int gravity)}: Sets horizontal text alignment and vertical gravity.
                \item \textbf{void setHeight(int pixels)}: Sets exact height in pixels.
                \item \textbf{void setHighlightColor(int color)}: Sets selection highlight color.
                \item \textbf{void setHighlights(Highlights highlights)}: Sets highlights.
                \item \textbf{final void setHint(CharSequence hint)}: Sets hint text.
                \item \textbf{final void setHint(int resid)}: Sets hint text from a resource.
                \item \textbf{final void setHintTextColor(ColorStateList colors)}: Sets hint text colors.
                \item \textbf{final void setHintTextColor(int color)}: Sets hint text color for all states.
                \item \textbf{void setHorizontallyScrolling(boolean whether)}: Allows text to exceed view width.
                \item \textbf{void setHyphenationFrequency(int hyphenationFrequency)}: Sets hyphenation frequency.
                \item \textbf{void setImeActionLabel(CharSequence label, int actionId)}: Sets custom IME action label and ID.
                \item \textbf{void setImeHintLocales(LocaleList hintLocales)}: Sets IME hint locales.
                \item \textbf{void setImeOptions(int imeOptions)}: Sets editor type/IME options.
                \item \textbf{void setIncludeFontPadding(boolean includepad)}: Toggles extra ascent/descent font padding.
                \item \textbf{void setInputExtras(int xmlResId)}: Sets extra input data bundle from XML.
                \item \textbf{void setInputType(int type)}: Sets the input content type.
                \item \textbf{void setJustificationMode(int justificationMode)}: Sets text justification mode.
                \item \textbf{void setKeyListener(KeyListener input)}: Sets the \texttt{KeyListener}.
                \item \textbf{void setLastBaselineToBottomHeight(int lastBaselineToBottomHeight)}: Adjusts bottom padding so last baseline is at the given distance from bottom.
                \item \textbf{void setLetterSpacing(float letterSpacing)}: Sets letter spacing (em).
                \item \textbf{void setLineBreakStyle(int lineBreakStyle)}: Sets line-break style.
                \item \textbf{void setLineBreakWordStyle(int lineBreakWordStyle)}: Sets line-break word style.
                \item \textbf{void setLineHeight(int unit, float lineHeight)}: Sets explicit line height with unit.
                \item \textbf{void setLineHeight(int lineHeight)}: Sets explicit line height (px).
                \item \textbf{void setLineSpacing(float add, float mult)}: Sets line spacing extra and multiplier.
                \item \textbf{void setLines(int lines)}: Sets exact line count.
                \item \textbf{final void setLinkTextColor(ColorStateList colors)}: Sets link text colors.
                \item \textbf{final void setLinkTextColor(int color)}: Sets link text color.
                \item \textbf{final void setLinksClickable(boolean whether)}: Controls auto-setting of \texttt{LinkMovementMethod}.
                \item \textbf{void setLocalePreferredLineHeightForMinimumUsed(boolean flag)}: Uses locale-preferred line height for minimum line height.
                \item \textbf{void setMarqueeRepeatLimit(int marqueeLimit)}: Sets marquee repeat count.
                \item \textbf{void setMaxEms(int maxEms)}: Sets maximum width in ems.
                \item \textbf{void setMaxHeight(int maxPixels)}: Sets maximum height in px.
                \item \textbf{void setMaxLines(int maxLines)}: Sets maximum number of lines.
                \item \textbf{void setMaxWidth(int maxPixels)}: Sets maximum width in px.
                \item \textbf{void setMinEms(int minEms)}: Sets minimum width in ems.
                \item \textbf{void setMinHeight(int minPixels)}: Sets minimum height in px.
                \item \textbf{void setMinLines(int minLines)}: Sets minimum number of lines.
                \item \textbf{void setMinWidth(int minPixels)}: Sets minimum width in px.
                \item \textbf{void setMinimumFontMetrics(Paint.FontMetrics minimumFontMetrics)}: Sets minimum font metrics for line spacing.
                \item \textbf{final void setMovementMethod(MovementMethod movement)}: Sets the \texttt{MovementMethod}.
                \item \textbf{void setOnEditorActionListener(TextView.OnEditorActionListener l)}: Sets a listener for editor actions.
                \item \textbf{void setPadding(int left, int top, int right, int bottom)}: Sets absolute padding.
                \item \textbf{void setPaddingRelative(int start, int top, int end, int bottom)}: Sets relative padding.
                \item \textbf{void setPaintFlags(int flags)}: Sets paint flags and reflows text if changed.
                \item \textbf{void setPrivateImeOptions(String type)}: Sets private IME options string.
                \item \textbf{void setRawInputType(int type)}: Directly sets the content type integer.
                \item \textbf{void setScroller(Scroller s)}: Sets the \texttt{Scroller} for scrolling animation.
                \item \textbf{void setSearchResultHighlightColor(int color)}: Sets search result highlight color.
                \item \textbf{void setSearchResultHighlights(int... ranges)}: Sets search result ranges (flattened).
                \item \textbf{void setSelectAllOnFocus(boolean selectAllOnFocus)}: Selects all text when gaining focus.
                \item \textbf{void setSelected(boolean selected)}: Changes selection state of this view.
                \item \textbf{void setShadowLayer(float radius, float dx, float dy, int color)}: Applies a text shadow.
                \item \textbf{void setShiftDrawingOffsetForStartOverhang(boolean shiftDrawingOffsetForStartOverhang)}: Enables shifting x offset to show start overhang.
                \item \textbf{final void setShowSoftInputOnFocus(boolean show)}: Controls soft input visibility on focus.
                \item \textbf{void setSingleLine(boolean singleLine)}: Toggles single-line properties.
                \item \textbf{void setSingleLine()}: Sets single-line properties.
                \item \textbf{final void setSpannableFactory(Spannable.Factory factory)}: Sets the \texttt{Spannable.Factory}.
                \item \textbf{final void setText(int resid)}: Sets text from a string resource.
                \item \textbf{final void setText(CharSequence text)}: Sets the displayed text.
                \item \textbf{void setText(CharSequence text, TextView.BufferType type)}: Sets text and buffer type.
                \item \textbf{final void setText(int resid, TextView.BufferType type)}: Sets text from a resource with buffer type.
                \item \textbf{final void setText(char[] text, int start, int len)}: Displays a slice of a char array.
                \item \textbf{void setTextAppearance(Context context, int resId)}: \textit{Deprecated API 23} — use \texttt{setTextAppearance(int)}.
                \item \textbf{void setTextAppearance(int resId)}: Sets the text appearance from a style resource.
                \item \textbf{void setTextClassifier(TextClassifier textClassifier)}: Sets the \texttt{TextClassifier}.
                \item \textbf{void setTextColor(int color)}: Sets text color for all states.
                \item \textbf{void setTextColor(ColorStateList colors)}: Sets text colors.
                \item \textbf{void setTextCursorDrawable(Drawable textCursorDrawable)}: Sets cursor drawable.
                \item \textbf{void setTextCursorDrawable(int textCursorDrawable)}: Sets cursor drawable by resource.
                \item \textbf{void setTextIsSelectable(boolean selectable)}: Toggles text selectability.
                \item \textbf{final void setTextKeepState(CharSequence text)}: Sets text while retaining cursor position.
                \item \textbf{final void setTextKeepState(CharSequence text, TextView.BufferType type)}: Sets text and buffer type while retaining cursor position.
                \item \textbf{void setTextLocale(Locale locale)}: Sets text \texttt{Locale} to a single-locale list.
                \item \textbf{void setTextLocales(LocaleList locales)}: Sets text \texttt{LocaleList}.
                \item \textbf{void setTextMetricsParams(PrecomputedText.Params params)}: Applies text layout parameters.
                \item \textbf{void setTextScaleX(float size)}: Sets horizontal text scale.
                \item \textbf{void setTextSelectHandle(int textSelectHandle)}: Sets selection handle drawable (resource).
                \item \textbf{void setTextSelectHandle(Drawable textSelectHandle)}: Sets selection handle drawable.
                \item \textbf{void setTextSelectHandleLeft(int textSelectHandleLeft)}: Sets left selection handle (resource).
                \item \textbf{void setTextSelectHandleLeft(Drawable textSelectHandleLeft)}: Sets left selection handle.
                \item \textbf{void setTextSelectHandleRight(Drawable textSelectHandleRight)}: Sets right selection handle.
                \item \textbf{void setTextSelectHandleRight(int textSelectHandleRight)}: Sets right selection handle (resource).
                \item \textbf{void setTextSize(int unit, float size)}: Sets text size with unit.
                \item \textbf{void setTextSize(float size)}: Sets text size in scaled pixels.
                \item \textbf{final void setTransformationMethod(TransformationMethod method)}: Sets text transformation method.
                \item \textbf{void setTypeface(Typeface tf)}: Sets the typeface.
                \item \textbf{void setTypeface(Typeface tf, int style)}: Sets typeface and style (enables fake bold/italic if needed).
                \item \textbf{void setUseBoundsForWidth(boolean useBoundsForWidth)}: Uses bounding-box width for line breaking/drawing.
                \item \textbf{void setWidth(int pixels)}: Sets exact width in pixels.
                \item \textbf{boolean showContextMenu()}: Shows the context menu.
                \item \textbf{boolean showContextMenu(float x, float y)}: Shows the context menu anchored at the given coordinates.
            \end{itemize}

        \item \textbf{Protected methods}
            \begin{itemize}
                \item \textbf{int computeHorizontalScrollRange()}: Computes the horizontal range represented by the horizontal scrollbar.
                \item \textbf{int computeVerticalScrollExtent()}: Computes the vertical extent of the scrollbar thumb within the total vertical range.
                \item \textbf{int computeVerticalScrollRange()}: Computes the vertical range represented by the vertical scrollbar.
                \item \textbf{void drawableStateChanged()}: Called when the view state changes in a way that affects any displayed drawables.
                \item \textbf{int getBottomPaddingOffset()}: Returns the amount by which to extend the bottom fading region.
                \item \textbf{boolean getDefaultEditable()}: Indicates whether the view has a default \texttt{KeyListener} even if not requested via XML.
                \item \textbf{MovementMethod getDefaultMovementMethod()}: Returns the default \texttt{MovementMethod} for this view.
                \item \textbf{float getLeftFadingEdgeStrength()}: Returns the intensity of the left faded edge.
                \item \textbf{int getLeftPaddingOffset()}: Returns the amount to extend the left fading region.
                \item \textbf{float getRightFadingEdgeStrength()}: Returns the intensity of the right faded edge.
                \item \textbf{int getRightPaddingOffset()}: Returns the amount to extend the right fading region.
                \item \textbf{int getTopPaddingOffset()}: Returns the amount to extend the top fading region.
                \item \textbf{boolean isPaddingOffsetRequired()}: True if the view draws inside padding and supports fading edge padding offsets.
                \item \textbf{void onAttachedToWindow()}: Called when the view is attached to a window.
                \item \textbf{void onConfigurationChanged(Configuration newConfig)}: Called when the device/app configuration changes.
                \item \textbf{void onCreateContextMenu(ContextMenu menu)}: Allows the view to add items to its context menu.
                \item \textbf{int[] onCreateDrawableState(int extraSpace)}: Generates a new drawable state array for this view.
                \item \textbf{void onDraw(Canvas canvas)}: Performs custom drawing for this view.
                \item \textbf{void onFocusChanged(boolean focused, int direction, Rect previouslyFocusedRect)}: Called when the view’s focus state changes.
                \item \textbf{void onLayout(boolean changed, int left, int top, int right, int bottom)}: Assigns size and position to child views.
                \item \textbf{void onMeasure(int widthMeasureSpec, int heightMeasureSpec)}: Measures the view and its content to determine width and height.
                \item \textbf{void onScrollChanged(int horiz, int vert, int oldHoriz, int oldVert)}: Called when the view scrolls its own contents.
                \item \textbf{void onSelectionChanged(int selStart, int selEnd)}: Called when the text selection changes.
                \item \textbf{void onTextChanged(CharSequence text, int start, int lengthBefore, int lengthAfter)}: Called when the text content changes.
                \item \textbf{void onVisibilityChanged(View changedView, int visibility)}: Called when visibility of this view or an ancestor changes.
                \item \textbf{boolean setFrame(int l, int t, int r, int b)}: Sets the view’s frame; returns true if it changed.
                \item \textbf{boolean verifyDrawable(Drawable who)}: Returns true for any drawable that this view is displaying.
            \end{itemize}

        \item \textbf{Constants}
            \begin{itemize}
                \item \textbf{int AUTO\_SIZE\_TEXT\_TYPE\_NONE}: The TextView does not auto-size text (default).
                \item \textbf{int AUTO\_SIZE\_TEXT\_TYPE\_UNIFORM}: The TextView scales text size both horizontally and vertically to fit within the container.
                \item \textbf{int FOCUSED\_SEARCH\_RESULT\_INDEX\_NONE}: A special index used for setFocusedSearchResultIndex(int) and getFocusedSearchResultIndex() inidicating there is no focused search result.
            \end{itemize}

    \end{itemize}

    \pagebreak 
    \subsubsection{EditText}
    \begin{itemize}
        \item \textbf{Hierarchy} 
            \begin{center}
                java.lang.Object $\to$	android.view.View $\to$	android.widget.TextView $\to$	android.widget.EditText
            \end{center}
        \item \textbf{Include}
            \bigbreak \noindent 
            \begin{javacode}
                android.widget.EditText
            \end{javacode}
        \item \textbf{Constructors}
            \bigbreak \noindent 
            \begin{javacode}
                EditText(Context context)
                EditText(Context context, AttributeSet attrs)
                EditText(Context context, AttributeSet attrs, int defStyleAttr)
                EditText(Context context, AttributeSet attrs, int defStyleAttr, int defStyleRes)
            \end{javacode}
        \item \textbf{Public methods}
            \begin{itemize}
                \item \textbf{void extendSelection(int index)}: Convenience method for \texttt{Selection.extendSelection}.
                \item \textbf{CharSequence getAccessibilityClassName()}: Returns the class name of this object to be used for accessibility purposes.
                \item \textbf{boolean getFreezesText()}: Returns whether this \texttt{TextView} includes its entire text contents in frozen icicles.
                \item \textbf{Editable getText()}: Returns the text that the \texttt{TextView} is displaying.
                \item \textbf{boolean isStyleShortcutEnabled()}: Returns true if style shortcuts are enabled, otherwise false.
                \item \textbf{boolean onKeyShortcut(int keyCode, KeyEvent event)}: Called on the focused view when a key shortcut event is not handled.
                \item \textbf{boolean onTextContextMenuItem(int id)}: Called when a context menu option for the text view is selected.
                \item \textbf{void selectAll()}: Convenience method for \texttt{Selection.selectAll}.
                \item \textbf{void setEllipsize(TextUtils.TruncateAt ellipsis)}: Specifies how overflowing text should be ellipsized instead of wrapped.
                \item \textbf{void setSelection(int index)}: Convenience method for \texttt{Selection.setSelection(Spannable, int)}.
                \item \textbf{void setSelection(int start, int stop)}: Convenience method for \texttt{Selection.setSelection(Spannable, int, int)}.
                \item \textbf{void setStyleShortcutsEnabled(boolean enabled)}: Enables style shortcuts such as \texttt{Ctrl+B} for bold.
                \item \textbf{void setText(CharSequence text, TextView.BufferType type)}: Sets the text to be displayed and the \texttt{TextView.BufferType}.
            \end{itemize}

        \item \textbf{Protected methods}
            \begin{itemize}
                \item \textbf{boolean getDefaultEditable()}: Subclasses override this to specify that they have a KeyListener by default even if not specifically called for in the XML options.
                \item \textbf{MovementMethod getDefaultMovementMethod()}: Subclasses override this to specify a default movement method.
            \end{itemize}

    \end{itemize}

    \pagebreak 
    \subsubsection{Button}
    \begin{itemize}
        \item \textbf{Hierarchy} 
            \begin{center}
                java.lang.Object $\to$	android.view.View $\to$	android.widget.TextView $\to$	android.widget.Button
            \end{center}
        \item \textbf{Include}
            \bigbreak \noindent 
            \begin{javacode}
                android.widget.Button
            \end{javacode}
        \item \textbf{Constructors}
            \bigbreak \noindent 
            \begin{javacode}
                Button(Context context)
                Button(Context context, AttributeSet attrs)
                Button(Context context, AttributeSet attrs, int defStyleAttr)
                Button(Context context, AttributeSet attrs, int defStyleAttr, int defStyleRes)
            \end{javacode}
        \item \textbf{Public methods}
            \begin{itemize}
                \item \textbf{CharSequence getAccessibilityClassName()}: Return the class name of this object to be used for accessibility purposes.
                \item \textbf{PointerIcon onResolvePointerIcon(MotionEvent event, int pointerIndex)}: Resolve the pointer icon that should be used for specified pointer in the motion event.
            \end{itemize}

    \end{itemize}

    \pagebreak 
    \subsubsection{ImageView}
    \begin{itemize}
         \item \textbf{Hierarchy} 
             \begin{center}
                 java.lang.Object $\to$	android.view.View $\to$	android.widget.ImageView
             \end{center}
        \item \textbf{Include}
            \bigbreak \noindent 
            \begin{javacode}
                android.widget.ImageView
            \end{javacode}
        \item \textbf{Constructors}
            \bigbreak \noindent 
            \begin{javacode}
                ImageView(Context context)
                ImageView(Context context, AttributeSet attrs)
                ImageView(Context context, AttributeSet attrs, int defStyleAttr)
                ImageView(Context context, AttributeSet attrs, int defStyleAttr, int defStyleRes)
            \end{javacode}
        \item \textbf{Public methods}
            \begin{itemize}
                \item \textbf{void animateTransform(Matrix matrix)}: Applies a temporary transformation matrix to the view's drawable when it is drawn.
                \item \textbf{final void clearColorFilter()}: Removes the image's \texttt{ColorFilter}.
                \item \textbf{void drawableHotspotChanged(float x, float y)}: Called whenever the view hotspot changes and needs to be propagated to drawables or child views managed by the view.
                \item \textbf{CharSequence getAccessibilityClassName()}: Returns the class name of this object to be used for accessibility purposes.
                \item \textbf{boolean getAdjustViewBounds()}: Returns true if the ImageView is adjusting its bounds to preserve the aspect ratio of its drawable.
                \item \textbf{int getBaseline()}: Returns the offset of the widget's text baseline from the widget's top boundary.
                \item \textbf{boolean getBaselineAlignBottom()}: Checks whether this view's baseline is considered the bottom of the view.
                \item \textbf{ColorFilter getColorFilter()}: Returns the active color filter for this ImageView.
                \item \textbf{boolean getCropToPadding()}: Returns whether this ImageView crops to its padding.
                \item \textbf{Drawable getDrawable()}: Gets the current drawable, or null if none has been assigned.
                \item \textbf{int getImageAlpha()}: Returns the alpha value applied to the drawable of this ImageView.
                \item \textbf{Matrix getImageMatrix()}: Returns the view's transformation matrix, if any.
                \item \textbf{BlendMode getImageTintBlendMode()}: Gets the blending mode used to apply the tint to the image drawable.
                \item \textbf{ColorStateList getImageTintList()}: Returns the current color tint list used for the image drawable.
                \item \textbf{PorterDuff.Mode getImageTintMode()}: Gets the blending mode used to apply the tint to the image drawable.
                \item \textbf{int getMaxHeight()}: Returns the maximum height of this view.
                \item \textbf{int getMaxWidth()}: Returns the maximum width of this view.
                \item \textbf{ImageView.ScaleType getScaleType()}: Returns the current scale type used to resize or move the image.
                \item \textbf{boolean hasOverlappingRendering()}: Returns whether this view has overlapping content.
                \item \textbf{void invalidateDrawable(Drawable dr)}: Invalidates the specified drawable.
                \item \textbf{boolean isOpaque()}: Indicates whether this view is opaque.
                \item \textbf{void jumpDrawablesToCurrentState()}: Calls \texttt{Drawable.jumpToCurrentState()} on all associated drawables.
                \item \textbf{int[] onCreateDrawableState(int extraSpace)}: Generates the new drawable state for this view.
                \item \textbf{void onRtlPropertiesChanged(int layoutDirection)}: Called when any RTL property (layout direction or text alignment) changes.
                \item \textbf{void onVisibilityAggregated(boolean isVisible)}: Called when the visibility of this view or one of its ancestors changes.
                \item \textbf{void setAdjustViewBounds(boolean adjustViewBounds)}: When true, adjusts the bounds to preserve the drawable's aspect ratio.
                \item \textbf{void setAlpha(int alpha)}: \textit{Deprecated in API 16.} Use \texttt{setImageAlpha(int)} instead.
                \item \textbf{void setBaseline(int baseline)}: Sets the offset of the widget's text baseline from the widget's top boundary.
                \item \textbf{void setBaselineAlignBottom(boolean aligned)}: Sets whether the baseline is considered the bottom of the view.
                \item \textbf{final void setColorFilter(int color, PorterDuff.Mode mode)}: Sets a tint color and mode for the image.
                \item \textbf{void setColorFilter(ColorFilter cf)}: Applies a custom color filter to the image.
                \item \textbf{final void setColorFilter(int color)}: Sets a tint color for the image.
                \item \textbf{void setCropToPadding(boolean cropToPadding)}: Sets whether this ImageView will crop its content to padding.
                \item \textbf{void setImageAlpha(int alpha)}: Sets the alpha transparency applied to the image.
                \item \textbf{void setImageBitmap(Bitmap bm)}: Sets a bitmap as the content of the ImageView.
                \item \textbf{void setImageDrawable(Drawable drawable)}: Sets a drawable as the content of the ImageView.
                \item \textbf{void setImageIcon(Icon icon)}: Sets an icon as the content of the ImageView.
                \item \textbf{void setImageLevel(int level)}: Sets the level for a \texttt{LevelListDrawable}.
                \item \textbf{void setImageMatrix(Matrix matrix)}: Applies a transformation matrix to the drawable.
                \item \textbf{void setImageResource(int resId)}: Sets a drawable resource as the content of the ImageView.
                \item \textbf{void setImageState(int[] state, boolean merge)}: Sets the drawable state for a \texttt{StateListDrawable}.
                \item \textbf{void setImageTintBlendMode(BlendMode blendMode)}: Sets the blending mode for applying the tint.
                \item \textbf{void setImageTintList(ColorStateList tint)}: Applies a tint list to the image drawable.
                \item \textbf{void setImageTintMode(PorterDuff.Mode tintMode)}: Sets the blending mode for applying the tint.
                \item \textbf{void setImageURI(Uri uri)}: Sets the content of this ImageView to the image located at the specified URI.
                \item \textbf{void setMaxHeight(int maxHeight)}: Sets a maximum height for the view.
                \item \textbf{void setMaxWidth(int maxWidth)}: Sets a maximum width for the view.
                \item \textbf{void setScaleType(ImageView.ScaleType scaleType)}: Defines how the image should be resized or moved to fit the view bounds.
                \item \textbf{void setSelected(boolean selected)}: Changes the selection state of this view.
                \item \textbf{void setVisibility(int visibility)}: Sets the visibility state of this view.
            \end{itemize}

        \item \textbf{Protected methods}
            \begin{itemize}
                \item \textbf{void drawableStateChanged()}: Called whenever the state of the view changes in a way that affects the state of its drawables.
                \item \textbf{void onAttachedToWindow()}: Called when the view is attached to a window.
                \item \textbf{void onDetachedFromWindow()}: Called when the view is detached from a window.
                \item \textbf{void onDraw(Canvas canvas)}: Implement this method to perform custom drawing for the view.
                \item \textbf{void onMeasure(int widthMeasureSpec, int heightMeasureSpec)}: Measures the view and its content to determine the measured width and height.
                \item \textbf{boolean setFrame(int l, int t, int r, int b)}: Assigns size and position to the view within its parent layout.
                \item \textbf{boolean verifyDrawable(Drawable dr)}: Returns true if the view is displaying the given drawable; subclasses should override when managing their own drawables.
            \end{itemize}

    \end{itemize}

    \pagebreak 
    \subsubsection{CompoundButton}
    \begin{itemize}
         \item \textbf{Hierarchy} 
            \begin{center}
                java.lang.Object $\to$	android.view.View $\to$	android.widget.TextView $\to$	android.widget.Button $\to$	android.widget.CompoundButton
            \end{center}
        \item \textbf{Include}
            \bigbreak \noindent 
            \begin{javacode}
                android.widget.CompoundButton
            \end{javacode}
        \item \textbf{Constructors}
            \bigbreak \noindent 
            \begin{javacode}
                CompoundButton(Context context)
                CompoundButton(Context context, AttributeSet attrs)
                CompoundButton(Context context, AttributeSet attrs, int defStyleAttr)
                CompoundButton(Context context, AttributeSet attrs, int defStyleAttr, int defStyleRes)
            \end{javacode}
        \item \textbf{Public methods}
            \begin{itemize}
                \item \textbf{void autofill(AutofillValue value)}: Automatically fills the content of this view with the given autofill value.
                \item \textbf{void drawableHotspotChanged(float x, float y)}: Called when the view hotspot changes and must be propagated to drawables or child views.
                \item \textbf{CharSequence getAccessibilityClassName()}: Returns the class name of this object for accessibility purposes.
                \item \textbf{int getAutofillType()}: Describes the autofill type of this view, allowing an \texttt{AutofillService} to create an appropriate \texttt{AutofillValue}.
                \item \textbf{AutofillValue getAutofillValue()}: Returns the current text of this \texttt{TextView} for autofill purposes.
                \item \textbf{Drawable getButtonDrawable()}: Returns the button drawable associated with this compound button.
                \item \textbf{BlendMode getButtonTintBlendMode()}: Returns the blending mode used to apply the tint to the button drawable.
                \item \textbf{ColorStateList getButtonTintList()}: Returns the tint list applied to the button drawable.
                \item \textbf{PorterDuff.Mode getButtonTintMode()}: Returns the blending mode used to apply the tint to the button drawable.
                \item \textbf{int getCompoundPaddingLeft()}: Returns the left padding of the view, including space for the left drawable if present.
                \item \textbf{int getCompoundPaddingRight()}: Returns the right padding of the view, including space for the right drawable if present.
                \item \textbf{boolean isChecked()}: Returns the checked state of this button.
                \item \textbf{void jumpDrawablesToCurrentState()}: Calls \texttt{Drawable.jumpToCurrentState()} on all drawable objects associated with this view.
                \item \textbf{void onRestoreInstanceState(Parcelable state)}: Restores the internal state of the view from a previously saved state.
                \item \textbf{Parcelable onSaveInstanceState()}: Saves the internal state of the view for later restoration.
                \item \textbf{boolean performClick()}: Calls this view’s \texttt{OnClickListener}, if one is defined.
                \item \textbf{void setButtonDrawable(int resId)}: Sets a drawable as the compound button image using its resource identifier.
                \item \textbf{void setButtonDrawable(Drawable drawable)}: Sets a drawable as the compound button image.
                \item \textbf{void setButtonIcon(Icon icon)}: Sets the button of this compound button to the specified icon.
                \item \textbf{void setButtonTintBlendMode(BlendMode tintMode)}: Sets the blending mode used to apply the tint specified by \texttt{setButtonTintList()}.
                \item \textbf{void setButtonTintList(ColorStateList tint)}: Applies a color tint to the button drawable.
                \item \textbf{void setButtonTintMode(PorterDuff.Mode tintMode)}: Specifies the blending mode used to apply the tint specified by \texttt{setButtonTintList()}.
                \item \textbf{void setChecked(boolean checked)}: Changes the checked state of this button.
                \item \textbf{void setOnCheckedChangeListener(CompoundButton.OnCheckedChangeListener listener)}: Registers a callback to be invoked when the checked state changes.
                \item \textbf{void setStateDescription(CharSequence stateDescription)}: Called when the view or subclass sets the state description for accessibility.
                \item \textbf{void toggle()}: Toggles the checked state of the button to the opposite of its current state.
            \end{itemize}
        \item \textbf{Protected methods}
            \begin{itemize}
                \item \textbf{void drawableStateChanged()}: Called whenever the state of the view changes in a way that affects the state of its drawables.
                \item \textbf{int[] onCreateDrawableState(int extraSpace)}: Generates and returns the new drawable state array for this view, allocating extra space if needed.
                \item \textbf{void onDraw(Canvas canvas)}: Implement this method to perform custom drawing for the view.
                \item \textbf{boolean verifyDrawable(Drawable who)}: Returns true if the specified drawable is being displayed by this view; subclasses should override when managing their own drawables.
            \end{itemize}

    \end{itemize}

    \pagebreak 
    \subsubsection{CheckBox}
    \begin{itemize}
        \item \textbf{Hierarchy} 
            \begin{center}
                java.lang.Object $\to$	android.view.View $\to$	android.widget.TextView $\to$	android.widget.Button $\to$	android.widget.CompoundButton $\to$	android.widget.CheckBox
            \end{center}
        \item \textbf{Include}
            \bigbreak \noindent 
            \begin{javacode}
                android.widget.CheckBox
            \end{javacode}
        \item \textbf{Constructors}
            \bigbreak \noindent 
            \begin{javacode}
                CheckBox(Context context)
                CheckBox(Context context, AttributeSet attrs)
                CheckBox(Context context, AttributeSet attrs, int defStyleAttr)
                CheckBox(Context context, AttributeSet attrs, int defStyleAttr, int defStyleRes)
            \end{javacode}
        \item \textbf{Public methods}
            \begin{itemize}
                \item \textbf{CharSequence getAccessibilityClassName()}: Return the class name of this object to be used for accessibility purposes.
            \end{itemize}
    \end{itemize}

    \pagebreak 
    \subsubsection{RadioButton}
    \begin{itemize}
        \item \textbf{Hierarchy} 
            \begin{center}
                java.lang.Object $\to$	android.view.View $\to$	android.widget.TextView $\to$	android.widget.Button $\to$	android.widget.CompoundButton $\to$	android.widget.RadioButton
            \end{center}
        \item \textbf{Include}
            \bigbreak \noindent 
            \begin{javacode}
                android.widget.RadioButton
            \end{javacode}
        \item \textbf{Constructors}
            \bigbreak \noindent 
            \begin{javacode}
                Public constructors
                RadioButton(Context context)
                RadioButton(Context context, AttributeSet attrs)
                RadioButton(Context context, AttributeSet attrs, int defStyleAttr)
                RadioButton(Context context, AttributeSet attrs, int defStyleAttr, int defStyleRes)
            \end{javacode}
        \item \textbf{Public methods}
            \begin{itemize}
                \item \textbf{CharSequence getAccessibilityClassName()}: Return the class name of this object to be used for accessibility purposes.
                \item \textbf{void onInitializeAccessibilityNodeInfo(AccessibilityNodeInfo info)}: Initializes an AccessibilityNodeInfo with information about this view.
                \item \textbf{void toggle()}: Change the checked state of the view to the inverse of its current state
                    \bigbreak \noindent 
                    If the radio button is already checked, this method will not toggle the radio button.
            \end{itemize}

    \end{itemize}

    \pagebreak 
    \subsubsection{Spinner}
    \begin{itemize}
        \item \textbf{Hierarchy} 
            \bigbreak \noindent 
            \begin{center}
                java.lang.Object $\to$	android.view.View $\to$	android.view.ViewGroup $\to$	android.widget.AdapterView<android.widget.SpinnerAdapter> $\to$	android.widget.AbsSpinner $\to$	android.widget.Spinner
            \end{center}
        \item \textbf{Include}
            \bigbreak \noindent 
            \begin{javacode}
                android.widget.Spinner
            \end{javacode}
        \item \textbf{Constructors}
            \bigbreak \noindent 
            \begin{javacode}
                Spinner(Context context)
                Spinner(Context context, AttributeSet attrs)
                Spinner(Context context, AttributeSet attrs, int defStyleAttr)
                Spinner(Context context, AttributeSet attrs, int defStyleAttr, int mode)
                Spinner(Context context, AttributeSet attrs, int defStyleAttr, int defStyleRes, int mode)
                Spinner(Context context, AttributeSet attrs, int defStyleAttr, int defStyleRes, int mode, Resources.Theme popupTheme)
                Spinner(Context context, int mode)
            \end{javacode}
        \item \textbf{Public methods}
            \begin{itemize}
                \item \textbf{CharSequence getAccessibilityClassName()}: Returns the class name of this object to be used for accessibility purposes.
                \item \textbf{int getBaseline()}: Returns the offset of the widget’s text baseline from the widget’s top boundary.
                \item \textbf{int getDropDownHorizontalOffset()}: Returns the configured horizontal offset in pixels for the spinner’s popup window of choices.
                \item \textbf{int getDropDownVerticalOffset()}: Returns the configured vertical offset in pixels for the spinner’s popup window of choices.
                \item \textbf{int getDropDownWidth()}: Returns the configured width of the spinner’s popup window of choices in pixels.
                \item \textbf{int getGravity()}: Describes how the selected item view is positioned within the spinner.
                \item \textbf{Drawable getPopupBackground()}: Returns the background drawable for the spinner’s popup window of choices.
                \item \textbf{Context getPopupContext()}: Returns the context used to inflate the spinner’s popup window.
                \item \textbf{CharSequence getPrompt()}: Returns the prompt text displayed when the spinner dialog is shown.
                \item \textbf{void onClick(DialogInterface dialog, int which)}: Invoked when a button in the dialog is clicked.
                \item \textbf{PointerIcon onResolvePointerIcon(MotionEvent event, int pointerIndex)}: Resolves the pointer icon that should be used for the specified pointer in the motion event.
                \item \textbf{void onRestoreInstanceState(Parcelable state)}: Restores the internal state of the spinner from a previously saved state.
                \item \textbf{Parcelable onSaveInstanceState()}: Saves the spinner’s current state for later restoration.
                \item \textbf{boolean onTouchEvent(MotionEvent event)}: Handles touch input events for the spinner.
                \item \textbf{boolean performClick()}: Calls this spinner’s \texttt{OnClickListener}, if one is defined.
                \item \textbf{void setAdapter(SpinnerAdapter adapter)}: Sets the \texttt{SpinnerAdapter} that provides the data backing this spinner.
                \item \textbf{void setDropDownHorizontalOffset(int pixels)}: Sets a horizontal offset in pixels for the spinner’s popup window of choices.
                \item \textbf{void setDropDownVerticalOffset(int pixels)}: Sets a vertical offset in pixels for the spinner’s popup window of choices.
                \item \textbf{void setDropDownWidth(int pixels)}: Sets the width of the spinner’s popup window of choices in pixels.
                \item \textbf{void setEnabled(boolean enabled)}: Sets the enabled state of this spinner.
                \item \textbf{void setGravity(int gravity)}: Defines how the selected item view is positioned within the spinner.
                \item \textbf{void setOnItemClickListener(AdapterView.OnItemClickListener l)}: No-op; spinners do not support item click events.
                \item \textbf{void setPopupBackgroundDrawable(Drawable background)}: Sets the background drawable for the spinner’s popup window of choices.
                \item \textbf{void setPopupBackgroundResource(int resId)}: Sets the background resource for the spinner’s popup window of choices.
                \item \textbf{void setPrompt(CharSequence prompt)}: Sets the prompt text to display when the dialog is shown.
                \item \textbf{void setPromptId(int promptId)}: Sets the prompt text to display when the dialog is shown using a resource ID.
            \end{itemize}

        \item \textbf{Protected methods}
            \begin{itemize}
                \item \textbf{void onDetachedFromWindow()}: This is called when the view is detached from a window.
                \item \textbf{void onLayout(boolean changed, int l, int t, int r, int b)}: Called from layout when this view should assign a size and position to each of its children.
                \item \textbf{void onMeasure(int widthMeasureSpec, int heightMeasureSpec)}: Measure the view and its content to determine the measured width and the measured height.
            \end{itemize}
        \item \textbf{Constants}
            \begin{itemize}
                \item \textbf{int MODE\_DIALOG}: Use a dialog window for selecting spinner options.
                \item \textbf{int MODE\_DROPDOWN}: Use a dropdown anchored to the Spinner for selecting spinner options.
            \end{itemize}

    \end{itemize}

    \pagebreak 
    \subsubsection{Progessbar}
    \begin{itemize}
        \item \textbf{Hierarchy} 
            \begin{center}
                java.lang.Object $\to$	android.view.View $\to$	android.widget.ProgressBar
            \end{center}
        \item \textbf{Include}
            \bigbreak \noindent 
            \begin{javacode}
                android.widget.ProgressBar
            \end{javacode}
        \item \textbf{Constructors}
            \bigbreak \noindent 
            \begin{javacode}
                ProgressBar(Context context)
                ProgressBar(Context context, AttributeSet attrs)
                ProgressBar(Context context, AttributeSet attrs, int defStyleAttr)
                ProgressBar(Context context, AttributeSet attrs, int defStyleAttr, int defStyleRes)
            \end{javacode}
        \item \textbf{Public methods}
            \begin{itemize}
                \item \textbf{void drawableHotspotChanged(float x, float y)}: Called whenever the view hotspot changes and must be propagated to drawables or child views.
                \item \textbf{CharSequence getAccessibilityClassName()}: Returns the class name of this object for accessibility purposes.
                \item \textbf{Drawable getCurrentDrawable()}: Returns the drawable currently used to draw the progress bar.
                \item \textbf{Drawable getIndeterminateDrawable()}: Returns the drawable used to draw the progress bar in indeterminate mode.
                \item \textbf{BlendMode getIndeterminateTintBlendMode()}: Returns the blending mode used to apply the tint to the indeterminate drawable, if specified.
                \item \textbf{ColorStateList getIndeterminateTintList()}: Returns the color tint list used for the indeterminate drawable.
                \item \textbf{PorterDuff.Mode getIndeterminateTintMode()}: Returns the blending mode used to apply the tint to the indeterminate drawable.
                \item \textbf{Interpolator getInterpolator()}: Gets the acceleration curve type for the indeterminate animation.
                \item \textbf{int getMax()}: Returns the upper limit of this progress bar's range.
                \item \textbf{int getMaxHeight()}: Returns the maximum height of the progress bar, in pixels.
                \item \textbf{int getMaxWidth()}: Returns the maximum width of the progress bar, in pixels.
                \item \textbf{int getMin()}: Returns the lower limit of this progress bar's range.
                \item \textbf{int getMinHeight()}: Returns the minimum height of the progress bar, in pixels.
                \item \textbf{int getMinWidth()}: Returns the minimum width of the progress bar, in pixels.
                \item \textbf{int getProgress()}: Returns the current progress level of the progress bar.
                \item \textbf{BlendMode getProgressBackgroundTintBlendMode()}: Returns the blending mode used to apply the tint to the progress background, if specified.
                \item \textbf{ColorStateList getProgressBackgroundTintList()}: Returns the tint list applied to the progress background, if specified.
                \item \textbf{PorterDuff.Mode getProgressBackgroundTintMode()}: Returns the blending mode used to apply the tint to the progress background.
                \item \textbf{Drawable getProgressDrawable()}: Returns the drawable used to draw the progress bar in progress mode.
                \item \textbf{BlendMode getProgressTintBlendMode()}: Returns the blending mode used to apply the tint to the progress drawable, if specified.
                \item \textbf{ColorStateList getProgressTintList()}: Returns the color tint list applied to the progress drawable.
                \item \textbf{PorterDuff.Mode getProgressTintMode()}: Returns the blending mode used to apply the tint to the progress drawable.
                \item \textbf{int getSecondaryProgress()}: Returns the current level of secondary progress.
                \item \textbf{BlendMode getSecondaryProgressTintBlendMode()}: Returns the blending mode used to apply the tint to the secondary progress drawable.
                \item \textbf{ColorStateList getSecondaryProgressTintList()}: Returns the color tint list applied to the secondary progress drawable.
                \item \textbf{PorterDuff.Mode getSecondaryProgressTintMode()}: Returns the blending mode used to apply the tint to the secondary progress drawable.
                \item \textbf{final void incrementProgressBy(int diff)}: Increases the primary progress by the specified amount.
                \item \textbf{final void incrementSecondaryProgressBy(int diff)}: Increases the secondary progress by the specified amount.
                \item \textbf{void invalidateDrawable(Drawable dr)}: Invalidates the specified drawable, forcing a redraw.
                \item \textbf{boolean isAnimating()}: Returns whether the progress bar is currently animating.
                \item \textbf{boolean isIndeterminate()}: Indicates whether this progress bar is in indeterminate mode.
                \item \textbf{void jumpDrawablesToCurrentState()}: Calls \texttt{Drawable.jumpToCurrentState()} on all associated drawables.
                \item \textbf{void onRestoreInstanceState(Parcelable state)}: Restores the internal state of the progress bar from a previously saved state.
                \item \textbf{Parcelable onSaveInstanceState()}: Saves the internal state of the progress bar for later restoration.
                \item \textbf{void onVisibilityAggregated(boolean isVisible)}: Called when the visibility of this view or its ancestors changes.
                \item \textbf{void postInvalidate()}: Schedules a redraw for the next event loop cycle.
                \item \textbf{void setIndeterminate(boolean indeterminate)}: Changes whether the progress bar is in indeterminate mode.
                \item \textbf{void setIndeterminateDrawable(Drawable d)}: Defines the drawable used to draw the progress bar in indeterminate mode.
                \item \textbf{void setIndeterminateDrawableTiled(Drawable d)}: Defines a tileable drawable used to draw the indeterminate progress bar.
                \item \textbf{void setIndeterminateTintBlendMode(BlendMode blendMode)}: Specifies the blending mode for applying the indeterminate tint.
                \item \textbf{void setIndeterminateTintList(ColorStateList tint)}: Applies a color tint to the indeterminate drawable.
                \item \textbf{void setIndeterminateTintMode(PorterDuff.Mode tintMode)}: Specifies the blending mode for applying the indeterminate tint.
                \item \textbf{void setInterpolator(Interpolator interpolator)}: Sets the acceleration curve for the indeterminate animation.
                \item \textbf{void setInterpolator(Context context, int resID)}: Sets the interpolator resource for the indeterminate animation.
                \item \textbf{void setMax(int max)}: Sets the upper range of the progress bar.
                \item \textbf{void setMaxHeight(int maxHeight)}: Sets the maximum height the progress bar can have.
                \item \textbf{void setMaxWidth(int maxWidth)}: Sets the maximum width the progress bar can have.
                \item \textbf{void setMin(int min)}: Sets the lower range of the progress bar.
                \item \textbf{void setMinHeight(int minHeight)}: Sets the minimum height the progress bar can have.
                \item \textbf{void setMinWidth(int minWidth)}: Sets the minimum width the progress bar can have.
                \item \textbf{void setProgress(int progress)}: Sets the current progress value.
                \item \textbf{void setProgress(int progress, boolean animate)}: Sets the current progress value, optionally animating the transition.
                \item \textbf{void setProgressBackgroundTintBlendMode(BlendMode blendMode)}: Specifies the blending mode for the progress background tint.
                \item \textbf{void setProgressBackgroundTintList(ColorStateList tint)}: Applies a tint to the progress background.
                \item \textbf{void setProgressBackgroundTintMode(PorterDuff.Mode tintMode)}: Specifies the blending mode for the progress background tint.
                \item \textbf{void setProgressDrawable(Drawable d)}: Defines the drawable used to draw the progress bar in progress mode.
                \item \textbf{void setProgressDrawableTiled(Drawable d)}: Defines a tileable drawable for the progress bar in progress mode.
                \item \textbf{void setProgressTintBlendMode(BlendMode blendMode)}: Specifies the blending mode for the progress indicator tint.
                \item \textbf{void setProgressTintList(ColorStateList tint)}: Applies a tint to the progress indicator.
                \item \textbf{void setProgressTintMode(PorterDuff.Mode tintMode)}: Specifies the blending mode for the progress indicator tint.
                \item \textbf{void setSecondaryProgress(int secondaryProgress)}: Sets the current secondary progress value.
                \item \textbf{void setSecondaryProgressTintBlendMode(BlendMode blendMode)}: Specifies the blending mode for the secondary progress tint.
                \item \textbf{void setSecondaryProgressTintList(ColorStateList tint)}: Applies a tint to the secondary progress indicator.
                \item \textbf{void setSecondaryProgressTintMode(PorterDuff.Mode tintMode)}: Specifies the blending mode for the secondary progress tint.
                \item \textbf{void setStateDescription(CharSequence stateDescription)}: Called when an instance or subclass sets the state description.
            \end{itemize}
        \item \textbf{Protected methods}
            \begin{itemize}
                \item \textbf{void drawableStateChanged()}: Called whenever the state of the view changes in a way that affects the state of its drawables.
                \item \textbf{void onAttachedToWindow()}: Called when the view is attached to a window.
                \item \textbf{void onDetachedFromWindow()}: Called when the view is detached from a window.
                \item \textbf{void onDraw(Canvas canvas)}: Implement this method to perform custom drawing for the view.
                \item \textbf{void onMeasure(int widthMeasureSpec, int heightMeasureSpec)}: Measures the view and its content to determine the measured width and height.
                \item \textbf{void onSizeChanged(int w, int h, int oldw, int oldh)}: Called during layout when the size of this view has changed.
                \item \textbf{boolean verifyDrawable(Drawable who)}: Returns true if the specified drawable is being displayed by this view; subclasses should override this when managing their own drawables.
            \end{itemize}

    \end{itemize}

    \pagebreak 
    \subsubsection{AbsSeekBar}
    \begin{itemize}
         \item \textbf{Hierarchy} 
            \begin{center}
                java.lang.Object $\to$	android.view.View $\to$	android.widget.ProgressBar $\to$	android.widget.AbsSeekBar
            \end{center}
        \item \textbf{Include}
            \bigbreak \noindent 
            \begin{javacode}
            android.widget.AbsSeekBar
            \end{javacode}
        \item \textbf{Constructors}
            \bigbreak \noindent 
            \begin{javacode}
                AbsSeekBar(Context context)
                AbsSeekBar(Context context, AttributeSet attrs)
                AbsSeekBar(Context context, AttributeSet attrs, int defStyleAttr)
                AbsSeekBar(Context context, AttributeSet attrs, int defStyleAttr, int defStyleRes)
            \end{javacode}
        \item \textbf{Public methods}
            \begin{itemize}
                \item \textbf{void drawableHotspotChanged(float x, float y)}: Called whenever the view hotspot changes and needs to be propagated to drawables or child views managed by the view.
                \item \textbf{CharSequence getAccessibilityClassName()}: Returns the class name of this object to be used for accessibility purposes.
                \item \textbf{int getKeyProgressIncrement()}: Returns the amount by which the progress changes when the user presses an arrow key.
                \item \textbf{boolean getSplitTrack()}: Returns whether the track is split by the thumb.
                \item \textbf{Drawable getThumb()}: Returns the drawable representing the scroll thumb — the component the user can drag to indicate progress.
                \item \textbf{int getThumbOffset()}: Returns the amount by which the thumb extends beyond the track.
                \item \textbf{BlendMode getThumbTintBlendMode()}: Returns the blending mode used to apply the tint to the thumb drawable, if specified.
                \item \textbf{ColorStateList getThumbTintList()}: Returns the tint color list applied to the thumb drawable, if specified.
                \item \textbf{PorterDuff.Mode getThumbTintMode()}: Returns the blending mode used to apply the tint to the thumb drawable, if specified.
                \item \textbf{Drawable getTickMark()}: Returns the drawable used as the tick mark for each progress position.
                \item \textbf{BlendMode getTickMarkTintBlendMode()}: Returns the blending mode used to apply the tint to the tick mark drawable, if specified.
                \item \textbf{ColorStateList getTickMarkTintList()}: Returns the tint color list applied to the tick mark drawable, if specified.
                \item \textbf{PorterDuff.Mode getTickMarkTintMode()}: Returns the blending mode used to apply the tint to the tick mark drawable, if specified.
                \item \textbf{void jumpDrawablesToCurrentState()}: Immediately updates all drawables associated with this view to their current state.
                \item \textbf{boolean onKeyDown(int keyCode, KeyEvent event)}: Handles key press events such as DPAD center or enter when the view is enabled and clickable.
                \item \textbf{void onRtlPropertiesChanged(int layoutDirection)}: Called when any RTL (right-to-left) layout property or alignment has changed.
                \item \textbf{boolean onTouchEvent(MotionEvent event)}: Handles touch or pointer events for user interaction.
                \item \textbf{void setKeyProgressIncrement(int increment)}: Sets the amount by which progress changes when the user presses arrow keys.
                \item \textbf{void setMax(int max)}: Sets the maximum value of the progress range.
                \item \textbf{void setMin(int min)}: Sets the minimum value of the progress range.
                \item \textbf{void setSplitTrack(boolean splitTrack)}: Specifies whether the track should be visually split by the thumb.
                \item \textbf{void setSystemGestureExclusionRects(List<Rect> rects)}: Defines regions within the view where system gestures should not be intercepted.
                \item \textbf{void setThumb(Drawable thumb)}: Sets the drawable used as the thumb in the progress meter.
                \item \textbf{void setThumbOffset(int thumbOffset)}: Sets the offset allowing the thumb to extend beyond the track.
                \item \textbf{void setThumbTintBlendMode(BlendMode blendMode)}: Defines the blending mode used when applying tint to the thumb drawable.
                \item \textbf{void setThumbTintList(ColorStateList tint)}: Applies a tint color list to the thumb drawable.
                \item \textbf{void setThumbTintMode(PorterDuff.Mode tintMode)}: Specifies the blending mode used with the thumb tint.
                \item \textbf{void setTickMark(Drawable tickMark)}: Sets the drawable used as a tick mark at each progress position.
                \item \textbf{void setTickMarkTintBlendMode(BlendMode blendMode)}: Specifies the blending mode used to apply tint to the tick mark drawable.
                \item \textbf{void setTickMarkTintList(ColorStateList tint)}: Applies a tint color list to the tick mark drawable.
                \item \textbf{void setTickMarkTintMode(PorterDuff.Mode tintMode)}: Specifies the blending mode used with the tick mark tint.
            \end{itemize}
        \item \textbf{Protected methods}
            \begin{itemize}
                \item \textbf{void drawableStateChanged()}: Called whenever the state of the view changes in a way that affects the state of its drawables.
                \item \textbf{void onDraw(Canvas canvas)}: Implement this method to perform custom drawing operations for the view.
                \item \textbf{void onMeasure(int widthMeasureSpec, int heightMeasureSpec)}: Measures the view and its content to determine the measured width and height.
                \item \textbf{void onSizeChanged(int w, int h, int oldw, int oldh)}: Called during layout when the size of the view changes, providing both new and old dimensions.
                \item \textbf{boolean verifyDrawable(Drawable who)}: Returns true if the specified drawable is managed and displayed by this view; subclasses should override when handling custom drawables.
            \end{itemize}


    \end{itemize}

    \pagebreak 
    \subsubsection{SeekBar}
    \begin{itemize}
        \item \textbf{Hierarchy} 
            \begin{center}
                java.lang.Object $\to$	android.view.View $\to$	android.widget.ProgressBar $\to$	android.widget.AbsSeekBar $\to$	android.widget.SeekBar
            \end{center}
        \item \textbf{Include}
            \bigbreak \noindent 
            \begin{javacode}
            android.widget.SeekBar
            \end{javacode}
        \item \textbf{Constructors}
            \bigbreak \noindent 
            \begin{javacode}
                SeekBar(Context context)
                SeekBar(Context context, AttributeSet attrs)
                SeekBar(Context context, AttributeSet attrs, int defStyleAttr)
                SeekBar(Context context, AttributeSet attrs, int defStyleAttr, int defStyleRes)
            \end{javacode}
        \item \textbf{Public methods}
            \begin{itemize}
                \item \textbf{CharSequence getAccessibilityClassName()}: Return the class name of this object to be used for accessibility purposes.
                \item \textbf{void setOnSeekBarChangeListener(SeekBar.OnSeekBarChangeListener l)}: Sets a listener to receive notifications of changes to the SeekBar's progress level.
            \end{itemize}

    \end{itemize}

    \pagebreak 
    \subsubsection{Drawable}
    \begin{itemize}
        \item \textbf{Hierarchy} 
            \begin{center}
                java.lang.Object $\to$	android.graphics.drawable.Drawable
            \end{center}
        \item \textbf{Include}
            \bigbreak \noindent 
            \begin{javacode}
                android.graphics.drawable.Drawable
            \end{javacode}
        \item \textbf{Constructors}
            \bigbreak \noindent 
            \begin{javacode}
                Drawable()
            \end{javacode}
        \item \textbf{Public methods}
            \begin{itemize}
                \item \textbf{void applyTheme(Resources.Theme t)}: Applies the specified theme to this \texttt{Drawable} and its children.
                \item \textbf{boolean canApplyTheme()}: Returns true if this \texttt{Drawable} can apply a theme.
                \item \textbf{void clearColorFilter()}: Removes any color filter currently applied to the drawable.
                \item \textbf{final Rect copyBounds()}: Returns a copy of the drawable’s bounds in a new \texttt{Rect} object.
                \item \textbf{final void copyBounds(Rect bounds)}: Copies the drawable’s bounds into the specified \texttt{Rect}.
                \item \textbf{static Drawable createFromPath(String pathName)}: Creates a drawable from the specified file path name.
                \item \textbf{static Drawable createFromResourceStream(Resources res, TypedValue value, InputStream is, String srcName, BitmapFactory.Options opts)}: \textit{Deprecated in API 28.} Creates a drawable from an input stream using the specified options.
                \item \textbf{static Drawable createFromResourceStream(Resources res, TypedValue value, InputStream is, String srcName)}: Creates a drawable from an input stream using the given resources and density information.
                \item \textbf{static Drawable createFromStream(InputStream is, String srcName)}: Creates a drawable from the specified input stream.
                \item \textbf{static Drawable createFromXml(Resources r, XmlPullParser parser)}: Creates a drawable from an XML document.
                \item \textbf{static Drawable createFromXml(Resources r, XmlPullParser parser, Resources.Theme theme)}: Creates a drawable from an XML document using the specified theme.
                \item \textbf{static Drawable createFromXmlInner(Resources r, XmlPullParser parser, AttributeSet attrs, Resources.Theme theme)}: Creates a drawable from within an XML document using an optional theme.
                \item \textbf{static Drawable createFromXmlInner(Resources r, XmlPullParser parser, AttributeSet attrs)}: Creates a drawable from within an XML document.
                \item \textbf{abstract void draw(Canvas canvas)}: Draws the drawable within its bounds, respecting alpha and color filters.
                \item \textbf{int getAlpha()}: Returns the current alpha value for the drawable.
                \item \textbf{final Rect getBounds()}: Returns the drawable’s bounding rectangle.
                \item \textbf{Drawable.Callback getCallback()}: Returns the current callback attached to this drawable.
                \item \textbf{int getChangingConfigurations()}: Returns a mask of configuration parameters that may change and require the drawable to be re-created.
                \item \textbf{ColorFilter getColorFilter()}: Returns the current color filter, or null if none is set.
                \item \textbf{Drawable.ConstantState getConstantState()}: Returns the constant state of this drawable, allowing shared state across instances.
                \item \textbf{Drawable getCurrent()}: Returns the current drawable in use, if the drawable supports multiple states.
                \item \textbf{Rect getDirtyBounds()}: Returns the drawable’s dirty bounds rectangle.
                \item \textbf{void getHotspotBounds(Rect outRect)}: Populates the provided \texttt{Rect} with the hotspot bounds.
                \item \textbf{int getIntrinsicHeight()}: Returns the drawable’s intrinsic (default) height.
                \item \textbf{int getIntrinsicWidth()}: Returns the drawable’s intrinsic (default) width.
                \item \textbf{int getLayoutDirection()}: Returns the resolved layout direction for this drawable.
                \item \textbf{final int getLevel()}: Returns the current drawable level.
                \item \textbf{int getMinimumHeight()}: Returns the minimum height suggested by this drawable.
                \item \textbf{int getMinimumWidth()}: Returns the minimum width suggested by this drawable.
                \item \textbf{abstract int getOpacity()}: \textit{Deprecated in API 29.} Returns the drawable’s opacity mode.
                \item \textbf{Insets getOpticalInsets()}: Returns the optical insets for alignment during layout.
                \item \textbf{void getOutline(Outline outline)}: Populates the given \texttt{Outline} with the drawable’s shape for rendering effects such as shadows.
                \item \textbf{boolean getPadding(Rect padding)}: Fills the given \texttt{Rect} with content insets suggested by the drawable.
                \item \textbf{int[] getState()}: Returns the current state of the drawable as an array of state attributes.
                \item \textbf{Region getTransparentRegion()}: Returns a region representing areas of complete transparency.
                \item \textbf{boolean hasFocusStateSpecified()}: Returns true if the drawable explicitly specifies a focused state.
                \item \textbf{void inflate(Resources r, XmlPullParser parser, AttributeSet attrs, Resources.Theme theme)}: Inflates this drawable from XML, optionally styled by a theme.
                \item \textbf{void inflate(Resources r, XmlPullParser parser, AttributeSet attrs)}: Inflates this drawable from XML.
                \item \textbf{void invalidateSelf()}: Requests a redraw of the drawable using its current callback.
                \item \textbf{boolean isAutoMirrored()}: Returns whether the drawable automatically mirrors its image in RTL layouts.
                \item \textbf{boolean isFilterBitmap()}: Returns whether this drawable is filtering bitmaps when scaled or rotated.
                \item \textbf{boolean isProjected()}: Returns true if the drawable is projected.
                \item \textbf{boolean isStateful()}: Returns whether this drawable changes appearance based on its state.
                \item \textbf{final boolean isVisible()}: Returns true if the drawable is currently visible.
                \item \textbf{void jumpToCurrentState()}: Skips any active state transition animations and jumps directly to the current state.
                \item \textbf{Drawable mutate()}: Returns a mutable instance of this drawable that can be modified independently.
                \item \textbf{boolean onLayoutDirectionChanged(int layoutDirection)}: Called when the drawable’s layout direction changes.
                \item \textbf{static int resolveOpacity(int op1, int op2)}: Resolves and returns an appropriate opacity value from two input opacities.
                \item \textbf{void scheduleSelf(Runnable what, long when)}: Schedules the drawable to execute the given runnable at a specified time.
                \item \textbf{abstract void setAlpha(int alpha)}: Sets the drawable’s alpha value for transparency.
                \item \textbf{void setAutoMirrored(boolean mirrored)}: Sets whether this drawable automatically mirrors when layout direction is RTL.
                \item \textbf{void setBounds(int left, int top, int right, int bottom)}: Defines the bounding rectangle of the drawable.
                \item \textbf{void setBounds(Rect bounds)}: Sets the drawable’s bounds using the provided rectangle.
                \item \textbf{final void setCallback(Drawable.Callback cb)}: Binds a callback to the drawable for invalidation and scheduling.
                \item \textbf{void setChangingConfigurations(int configs)}: Specifies which configuration changes require the drawable to be recreated.
                \item \textbf{void setColorFilter(int color, PorterDuff.Mode mode)}: \textit{Deprecated in API 29.} Use \texttt{setColorFilter(ColorFilter)} instead.
                \item \textbf{abstract void setColorFilter(ColorFilter colorFilter)}: Sets an optional color filter to modify how the drawable’s pixels are rendered.
                \item \textbf{void setDither(boolean dither)}: \textit{Deprecated in API 23.} This property is ignored.
                \item \textbf{void setFilterBitmap(boolean filter)}: Enables or disables bilinear filtering for scaled or rotated bitmaps.
                \item \textbf{void setHotspot(float x, float y)}: Specifies the hotspot’s location within the drawable.
                \item \textbf{void setHotspotBounds(int left, int top, int right, int bottom)}: Defines the bounds for the hotspot within the drawable.
                \item \textbf{final boolean setLayoutDirection(int layoutDirection)}: Sets the layout direction for the drawable (LTR or RTL).
                \item \textbf{final boolean setLevel(int level)}: Sets the drawable’s current level, used by certain drawable types for animation or progress indication.
                \item \textbf{boolean setState(int[] stateSet)}: Sets the drawable’s current state using the given array of state attributes.
                \item \textbf{void setTint(int tintColor)}: Applies a single color tint to the drawable.
                \item \textbf{void setTintBlendMode(BlendMode blendMode)}: Specifies the blending mode used to apply the tint color.
                \item \textbf{void setTintList(ColorStateList tint)}: Applies a tint color list to the drawable for different states.
                \item \textbf{void setTintMode(PorterDuff.Mode tintMode)}: Specifies the blending mode used to apply the tint list.
                \item \textbf{boolean setVisible(boolean visible, boolean restart)}: Sets the drawable’s visibility, optionally restarting animations.
                \item \textbf{void unscheduleSelf(Runnable what)}: Cancels any scheduled runnables associated with this drawable.
            \end{itemize}

        \item \textbf{Protected methods}
            \begin{itemize}
                \item \textbf{void onBoundsChange(Rect bounds)}: Override this in your subclass to change appearance if you vary based on the bounds.
                \item \textbf{boolean onLevelChange(int level)}: Override this in your subclass to change appearance if you vary based on level.
                \item \textbf{boolean onStateChange(int[] state)}: Override this in your subclass to change appearance if you recognize the specified state.
            \end{itemize}
    \end{itemize}

    \pagebreak 
    \subsubsection{GradientDrawable}
    \begin{itemize}
        \item \textbf{Hierarchy} 
            \begin{center}
                java.lang.Object $\to$	android.graphics.drawable.Drawable $\to$	android.graphics.drawable.GradientDrawable
            \end{center}
        \item \textbf{Include}
            \bigbreak \noindent 
            \begin{javacode}
                android.graphics.drawable.GradientDrawable
            \end{javacode}
        \item \textbf{Constructors}
            \bigbreak \noindent 
            \begin{javacode}
            GradientDrawable()
            GradientDrawable(GradientDrawable.Orientation orientation, int[] colors)
        \end{javacode}
    \item \textbf{Public methods}
        \begin{itemize}
            \item \textbf{void applyTheme(Resources.Theme t)}: Applies the specified theme to this \texttt{Drawable} and its children.
            \item \textbf{boolean canApplyTheme()}: Returns true if this \texttt{Drawable} can apply a theme.
            \item \textbf{void draw(Canvas canvas)}: Draws the shape within its bounds, respecting alpha and color filter effects.
            \item \textbf{int getAlpha()}: Returns the current alpha value of the drawable.
            \item \textbf{int getChangingConfigurations()}: Returns a mask of configuration parameters that can change, requiring the drawable to be recreated.
            \item \textbf{ColorStateList getColor()}: Returns the color state list used to fill the shape, or \texttt{null} if it uses a gradient or no fill color.
            \item \textbf{ColorFilter getColorFilter()}: Returns the currently applied color filter, or \texttt{null} if none.
            \item \textbf{int[] getColors()}: Returns the colors used for the gradient fill, or \texttt{null} if not applicable.
            \item \textbf{Drawable.ConstantState getConstantState()}: Returns the constant state shared by this drawable.
            \item \textbf{float[] getCornerRadii()}: Returns the corner radii for all four corners.
            \item \textbf{float getCornerRadius()}: Returns the uniform corner radius set with \texttt{setCornerRadius(float)}.
            \item \textbf{float getGradientCenterX()}: Returns the X position of the gradient center as a fraction of the width.
            \item \textbf{float getGradientCenterY()}: Returns the Y position of the gradient center as a fraction of the height.
            \item \textbf{float getGradientRadius()}: Returns the gradient’s radius in pixels.
            \item \textbf{int getGradientType()}: Returns the gradient type: \texttt{LINEAR\_GRADIENT}, \texttt{RADIAL\_GRADIENT}, or \texttt{SWEEP\_GRADIENT}.
            \item \textbf{int getInnerRadius()}: Returns the inner radius of the ring shape in pixels.
            \item \textbf{float getInnerRadiusRatio()}: Returns the inner radius of the ring as a ratio of the ring’s width.
            \item \textbf{int getIntrinsicHeight()}: Returns the drawable’s intrinsic height.
            \item \textbf{int getIntrinsicWidth()}: Returns the drawable’s intrinsic width.
            \item \textbf{int getOpacity()}: \textit{Deprecated.} No longer used for optimization.
            \item \textbf{Insets getOpticalInsets()}: Returns the suggested layout insets for alignment operations.
            \item \textbf{GradientDrawable.Orientation getOrientation()}: Returns the orientation of the gradient.
            \item \textbf{void getOutline(Outline outline)}: Populates the given \texttt{Outline} with the drawable’s shape outline.
            \item \textbf{boolean getPadding(Rect padding)}: Returns padding in the given \texttt{Rect}, as suggested by the drawable.
            \item \textbf{int getShape()}: Returns the shape type (\texttt{LINE}, \texttt{OVAL}, \texttt{RECTANGLE}, or \texttt{RING}).
            \item \textbf{int getThickness()}: Returns the ring’s thickness in pixels.
            \item \textbf{float getThicknessRatio()}: Returns the ring’s thickness as a ratio of its width.
            \item \textbf{boolean getUseLevel()}: Returns whether the drawable’s level property is used to scale the gradient.
            \item \textbf{boolean hasFocusStateSpecified()}: Returns true if the drawable explicitly defines a focused state.
            \item \textbf{void inflate(Resources r, XmlPullParser parser, AttributeSet attrs, Resources.Theme theme)}: Inflates the drawable from XML, optionally using a theme.
            \item \textbf{boolean isStateful()}: Returns true if the drawable changes appearance based on state.
            \item \textbf{Drawable mutate()}: Returns a mutable instance of this drawable that can be modified independently.
            \item \textbf{void setAlpha(int alpha)}: Sets the transparency level of the drawable.
            \item \textbf{void setColor(ColorStateList colorStateList)}: Changes the drawable to use a solid color state list instead of a gradient.
            \item \textbf{void setColor(int argb)}: Changes the drawable to use a single solid color.
            \item \textbf{void setColorFilter(ColorFilter colorFilter)}: Applies a color filter to the drawable.
            \item \textbf{void setColors(int[] colors, float[] offsets)}: Defines multiple colors and their relative positions in the gradient.
            \item \textbf{void setColors(int[] colors)}: Sets multiple colors for the gradient fill.
            \item \textbf{void setCornerRadii(float[] radii)}: Sets custom radii for each corner of the shape.
            \item \textbf{void setCornerRadius(float radius)}: Sets a uniform radius for all corners.
            \item \textbf{void setDither(boolean dither)}: \textit{Deprecated.} Ignored property.
            \item \textbf{void setGradientCenter(float x, float y)}: Sets the center of the gradient as a fraction of width and height.
            \item \textbf{void setGradientRadius(float gradientRadius)}: Sets the gradient’s radius in pixels.
            \item \textbf{void setGradientType(int gradient)}: Defines the gradient type (\texttt{LINEAR}, \texttt{RADIAL}, or \texttt{SWEEP}).
            \item \textbf{void setInnerRadius(int innerRadius)}: Sets the inner radius of a ring shape.
            \item \textbf{void setInnerRadiusRatio(float innerRadiusRatio)}: Defines the ring’s inner radius as a ratio of its width.
            \item \textbf{void setOrientation(GradientDrawable.Orientation orientation)}: Sets the direction of the gradient.
            \item \textbf{void setPadding(int left, int top, int right, int bottom)}: Defines the padding of the shape.
            \item \textbf{void setShape(int shape)}: Sets the shape type (\texttt{LINE}, \texttt{OVAL}, \texttt{RECTANGLE}, \texttt{RING}).
            \item \textbf{void setSize(int width, int height)}: Defines the overall size of the shape.
            \item \textbf{void setStroke(int width, ColorStateList colorStateList)}: Sets stroke width and color using a color state list.
            \item \textbf{void setStroke(int width, ColorStateList colorStateList, float dashWidth, float dashGap)}: Sets stroke width, color, and dash pattern.
            \item \textbf{void setStroke(int width, int color, float dashWidth, float dashGap)}: Sets stroke width, solid color, and dash pattern.
            \item \textbf{void setStroke(int width, int color)}: Sets stroke width and solid color.
            \item \textbf{void setThickness(int thickness)}: Sets the thickness of the ring shape.
            \item \textbf{void setThicknessRatio(float thicknessRatio)}: Sets the ring thickness as a ratio of its width.
            \item \textbf{void setTintBlendMode(BlendMode blendMode)}: Specifies how tint color should be blended with the drawable.
            \item \textbf{void setTintList(ColorStateList tint)}: Applies a tint color list for different drawable states.
            \item \textbf{void setUseLevel(boolean useLevel)}: Configures whether the drawable’s level affects the gradient scaling.
        \end{itemize}
        \item \textbf{Protected methods}
            \begin{itemize}
                \item \textbf{void onBoundsChange(Rect r)}: Override this in your subclass to change appearance if you vary based on the bounds.
                \item \textbf{boolean onLevelChange(int level)}: Override this in your subclass to change appearance if you vary based on level.
                \item \textbf{boolean onStateChange(int[] stateSet)}: Override this in your subclass to change appearance if you recognize the specified state.
            \end{itemize}
        \item \textbf{Constants}
            \begin{itemize}
                \item \textbf{int ARC}: Shape is an arc.
                \item \textbf{int LINE}: Shape is a line
                \item \textbf{int LINEAR\_GRADIENT}: Gradient is linear (default.)
                \item \textbf{int OVAL}: Shape is an ellipse
                \item \textbf{int RADIAL\_GRADIENT}: Gradient is circular.
                \item \textbf{int RECTANGLE}: Shape is a rectangle, possibly with rounded corners
                \item \textbf{int RING}: Shape is a ring.
                \item \textbf{int SWEEP\_GRADIENT}: Gradient is a sweep.
            \end{itemize}
    \end{itemize}


    \pagebreak 
    \subsection{Styling widgets with java}
    \subsubsection{View}
    \begin{itemize}
        % --- Styling and Appearance ---
        \item \textbf{void setBackgroundColor(int color)}: Fills the view’s background with a solid color.
        \item \textbf{void setBackground(Drawable background)}: Sets a Drawable as the background.
        \item \textbf{Drawable getBackground()}: Returns the current background drawable.
        \item \textbf{void setForeground(Drawable foreground)}: Draws a Drawable on top of the view’s content.
        \item \textbf{void setPadding(int left, int top, int right, int bottom)}: Sets the padding inside the view.
        \item \textbf{int getPaddingLeft() / getPaddingTop() / getPaddingRight() / getPaddingBottom()}: Returns padding values.
        \item \textbf{void setElevation(float elevation)}: Adds shadow depth to the view for visual layering.
        \item \textbf{void setClipToOutline(boolean clipToOutline)}: Clips the view’s drawing to its outline (e.g., rounded corners).
        \item \textbf{void setOutlineProvider(ViewOutlineProvider provider)}: Defines the outline shape for shadows and clipping.
        \item \textbf{void setBackgroundTintList(ColorStateList tint)}: Applies tint coloring to the background.
        \item \textbf{void setBackgroundTintMode(PorterDuff.Mode mode)}: Defines how the background tint blends with the original color.
        \item \textbf{void setForegroundTintList(ColorStateList tint)}: Applies tint coloring to the foreground.
        \item \textbf{void setOutlineSpotShadowColor(int color)}: Sets the color of the view’s spot shadow.
        \item \textbf{void setOutlineAmbientShadowColor(int color)}: Sets the color of the view’s ambient shadow.
        \item \textbf{void setRotationX(float rotationX) / setRotationY(float rotationY)}: Rotates the view around the X or Y axis.
        \item \textbf{void setCameraDistance(float distance)}: Adjusts the 3D perspective depth for rotation effects.
    \end{itemize}

    \pagebreak 
    \subsubsection{TextView}
    \begin{itemize}
        \item \textbf{Color, size, typeface}
            \begin{itemize}
                \item \textbf{void setTextColor(int color)}: Sets the text color.
                \item \textbf{void setTextColor(ColorStateList colors)}: Sets text colors for different states.
                \item \textbf{void setHighlightColor(int color)}: Sets selection highlight color.
                \item \textbf{void setLinkTextColor(int color)}: Sets link color.
                \item \textbf{void setLinkTextColor(ColorStateList colors)}: Sets link colors for states.
                \item \textbf{void setTextSize(float size)}: Sets text size in scaled pixels.
                \item \textbf{void setTextSize(int unit, float size)}: Sets text size with unit.
                \item \textbf{void setTextScaleX(float size)}: Horizontal text scale.
                \item \textbf{void setTypeface(Typeface tf)}: Sets typeface.
                \item \textbf{void setTypeface(Typeface tf, int style)}: Sets typeface and style.
                \item \textbf{void setAllCaps(boolean allCaps)}: Transforms input to ALL CAPS display.
                \item \textbf{void setTextAppearance(int resId)}: Applies a text appearance style.
                \item \textbf{void setTextAppearance(Context context, int resId)}: \textit{Deprecated} in API 23.
            \end{itemize}

        \item \textbf{Auto-size text}
            \begin{itemize}
                \item \textbf{void setAutoSizeTextTypeWithDefaults(int autoSizeTextType)}: Enables default auto-size.
                \item \textbf{void setAutoSizeTextTypeUniformWithConfiguration(int min, int max, int step, int unit)}: Uniform auto-size config.
                \item \textbf{void setAutoSizeTextTypeUniformWithPresetSizes(int[] presetSizes, int unit)}: Uniform auto-size with presets.
            \end{itemize}

        \item \textbf{Typography, wrapping, justification}
            \begin{itemize}
                \item \textbf{void setLetterSpacing(float letterSpacing)}: Sets letter spacing.
                \item \textbf{void setLineSpacing(float add, float mult)}: Extra and multiplier.
                \item \textbf{void setLineHeight(int lineHeight)}: Explicit line height (px).
                \item \textbf{void setLineHeight(int unit, float lineHeight)}: Explicit line height with unit.
                \item \textbf{void setEllipsize(TextUtils.TruncateAt where)}: Ellipsize strategy.
                \item \textbf{void setBreakStrategy(int breakStrategy)}: Paragraph line-break strategy.
                \item \textbf{void setLineBreakStyle(int lineBreakStyle)}: Line-break style.
                \item \textbf{void setLineBreakWordStyle(int lineBreakWordStyle)}: Word-break style.
                \item \textbf{void setHyphenationFrequency(int hyphenationFrequency)}: Hyphenation setting.
                \item \textbf{void setJustificationMode(int justificationMode)}: Text justification.
                \item \textbf{void setFallbackLineSpacing(boolean enabled)}: Respect fallback font metrics.
                \item \textbf{void setIncludeFontPadding(boolean includepad)}: Include extra ascent/descent padding.
                \item \textbf{void setFirstBaselineToTopHeight(int px)}: Align first baseline to top padding.
                \item \textbf{void setLastBaselineToBottomHeight(int px)}: Align last baseline to bottom padding.
                \item \textbf{void setLocalePreferredLineHeightForMinimumUsed(boolean flag)}: Locale-preferred min line height.
            \end{itemize}

        \item \textbf{Compound drawables and tints}
            \begin{itemize}
                \item \textbf{void setCompoundDrawables(Drawable left, Drawable top, Drawable right, Drawable bottom)}: L/T/R/B drawables.
                \item \textbf{void setCompoundDrawablesWithIntrinsicBounds(Drawable left, Drawable top, Drawable right, Drawable bottom)}: With intrinsic bounds.
                \item \textbf{void setCompoundDrawablesWithIntrinsicBounds(int left, int top, int right, int bottom)}: By resource IDs.
                \item \textbf{void setCompoundDrawablesRelative(Drawable start, Drawable top, Drawable end, Drawable bottom)}: Start/End variants.
                \item \textbf{void setCompoundDrawablesRelativeWithIntrinsicBounds(Drawable start, Drawable top, Drawable end, Drawable bottom)}: With intrinsic bounds.
                \item \textbf{void setCompoundDrawablesRelativeWithIntrinsicBounds(int start, int top, int end, int bottom)}: By resource IDs.
                \item \textbf{void setCompoundDrawablePadding(int pad)}: Space between text and drawables.
                \item \textbf{void setCompoundDrawableTintList(ColorStateList tint)}: Drawable tint list.
                \item \textbf{void setCompoundDrawableTintMode(PorterDuff.Mode mode)}: Porter-Duff tint mode.
                \item \textbf{void setCompoundDrawableTintBlendMode(BlendMode blendMode)}: BlendMode tinting.
            \end{itemize}

        \item \textbf{Hint \& cursor/selection visuals}
            \begin{itemize}
                \item \textbf{void setHint(CharSequence hint)}: Hint text.
                \item \textbf{void setHint(int resid)}: Hint from resource.
                \item \textbf{void setHintTextColor(int color)}: Hint color.
                \item \textbf{void setHintTextColor(ColorStateList colors)}: Hint color states.
                \item \textbf{void setTextCursorDrawable(Drawable d)}: Cursor drawable.
                \item \textbf{void setTextCursorDrawable(int resId)}: Cursor drawable by resource.
                \item \textbf{void setTextSelectHandle(int resId)}: Selection handle (resource).
                \item \textbf{void setTextSelectHandle(Drawable d)}: Selection handle.
                \item \textbf{void setTextSelectHandleLeft(int resId)}: Left handle (resource).
                \item \textbf{void setTextSelectHandleLeft(Drawable d)}: Left handle.
                \item \textbf{void setTextSelectHandleRight(int resId)}: Right handle (resource).
                \item \textbf{void setTextSelectHandleRight(Drawable d)}: Right handle.
            \end{itemize}

        \item \textbf{Shadows, transforms, and paint flags}
            \begin{itemize}
                \item \textbf{void setShadowLayer(float radius, float dx, float dy, int color)}: Text shadow.
                \item \textbf{final void setTransformationMethod(TransformationMethod method)}: Visual text transformation (e.g., password).
                \item \textbf{void setPaintFlags(int flags)}: Underline/strike-through, etc.
                \item \textbf{void setElegantTextHeight(boolean elegant)}: Use elegant height metrics.
                \item \textbf{void setFontFeatureSettings(String settings)}: OpenType features.
                \item \textbf{boolean setFontVariationSettings(String settings)}: Font variations (axes).
            \end{itemize}

        \item \textbf{Alignment / layout-affecting (often part of style guides)}
            \begin{itemize}
                \item \textbf{void setGravity(int gravity)}: Horizontal/vertical alignment within the view.
                \item \textbf{void setEms(int ems)}: Exact width in ems (typographic sizing).
                \item \textbf{void setLines(int lines)}: Exact number of lines (tight control of layout look).
            \end{itemize}

        \item \textbf{Styling-related getters (useful to read current style)}
            \begin{itemize}
                \item \textbf{int getCurrentTextColor()}, \textbf{ColorStateList getTextColors()}, \textbf{int getHighlightColor()}, \textbf{ColorStateList getHintTextColors()}, \textbf{ColorStateList getLinkTextColors()}
                \item \textbf{float getTextSize()}, \textbf{Typeface getTypeface()}, \textbf{float getLetterSpacing()}, \textbf{int getLineHeight()}
                \item \textbf{TextUtils.TruncateAt getEllipsize()}, \textbf{int getBreakStrategy()}, \textbf{int getHyphenationFrequency()}, \textbf{int getJustificationMode()}
                \item \textbf{String getFontFeatureSettings()}, \textbf{String getFontVariationSettings()}
                \item \textbf{Drawable[] getCompoundDrawables()}, \textbf{Drawable[] getCompoundDrawablesRelative()}, \textbf{int getCompoundDrawablePadding()}, \textbf{ColorStateList getCompoundDrawableTintList()}, \textbf{PorterDuff.Mode getCompoundDrawableTintMode()}, \textbf{BlendMode getCompoundDrawableTintBlendMode()}
            \end{itemize}

        \item \textbf{Advanced styling hooks (from the \emph{other} list)}
            \begin{itemize}
                \item \textbf{void drawableStateChanged()}: React to state changes that affect drawables/tints.
                \item \textbf{int[] onCreateDrawableState(int extraSpace)}: Customize drawable state for styling.
                \item \textbf{void onDraw(Canvas canvas)}: Custom rendering of text/effects.
            \end{itemize}
    \end{itemize}

    \pagebreak 
    \subsubsection{EditText (Use TextView methods)}
    \begin{itemize}
        \item \textbf{boolean isStyleShortcutEnabled()}: Returns true if style shortcuts (e.g., \texttt{Ctrl+B} for bold) are enabled.
        \item \textbf{void setStyleShortcutsEnabled(boolean enabled)}: Enables or disables style shortcuts such as \texttt{Ctrl+B}, \texttt{Ctrl+I}, etc.
        \item \textbf{void setEllipsize(TextUtils.TruncateAt ellipsis)}: Specifies how overflowing text should be ellipsized (e.g., at the end or middle) instead of wrapped.
        \item \textbf{void setText(CharSequence text, TextView.BufferType type)}: Sets the text and determines how it’s stored (e.g., as plain, styled, or editable text), affecting styling.
    \end{itemize}


    \pagebreak 
    \subsubsection{Button (Use TextView methods)}

    \pagebreak 
    \subsubsection{ListView}
    \begin{itemize}
        \item \textbf{Drawable getDivider()}: Returns the drawable that is drawn between each list item.
        \item \textbf{int getDividerHeight()}: Returns the height of the divider between list items.
        \item \textbf{Drawable getOverscrollFooter()}: Returns the drawable drawn below all list content during overscroll.
        \item \textbf{Drawable getOverscrollHeader()}: Returns the drawable drawn above all list content during overscroll.
        \item \textbf{boolean areFooterDividersEnabled()}: Returns whether footer dividers are currently enabled.
        \item \textbf{boolean areHeaderDividersEnabled()}: Returns whether header dividers are currently enabled.
        \item \textbf{void setCacheColorHint(int color)}: Sets a hint color indicating the solid background behind the list, improving appearance on transparent backgrounds.
        \item \textbf{void setDivider(Drawable divider)}: Sets the drawable that will be drawn between list items.
        \item \textbf{void setDividerHeight(int height)}: Sets the height of the divider drawn between list items.
        \item \textbf{void setFooterDividersEnabled(boolean footerDividersEnabled)}: Enables or disables drawing dividers for footer views.
        \item \textbf{void setHeaderDividersEnabled(boolean headerDividersEnabled)}: Enables or disables drawing dividers for header views.
        \item \textbf{void setOverscrollFooter(Drawable footer)}: Sets the drawable to be drawn below all list content during overscroll.
        \item \textbf{void setOverscrollHeader(Drawable header)}: Sets the drawable to be drawn above all list content during overscroll.
        \item \textbf{void dispatchDraw(Canvas canvas)}: (Protected) Called to draw all child views — can be overridden for custom list-item rendering effects.
        \item \textbf{boolean drawChild(Canvas canvas, View child, long drawingTime)}: (Protected) Draws a single list item onto the canvas; used to customize per-item drawing style.
    \end{itemize}


    \pagebreak 
    \subsubsection{ImageView}
    \begin{itemize}
        \item \textbf{void animateTransform(Matrix matrix)}: Applies a temporary transformation matrix for animation or visual effects.
        \item \textbf{final void clearColorFilter()}: Removes any color filter or tint applied to the image.
        \item \textbf{ColorFilter getColorFilter()}: Returns the active color filter used for tinting or blending.
        \item \textbf{int getImageAlpha()}: Returns the current alpha (transparency) level of the image.
        \item \textbf{Matrix getImageMatrix()}: Returns the current transformation matrix applied to the image.
        \item \textbf{BlendMode getImageTintBlendMode()}: Returns the blending mode used for applying tints.
        \item \textbf{ColorStateList getImageTintList()}: Returns the color tint list applied to the image drawable.
        \item \textbf{PorterDuff.Mode getImageTintMode()}: Returns the blending mode used for applying the tint.
        \item \textbf{ImageView.ScaleType getScaleType()}: Returns the scale type that defines how the image fits within the view bounds.
        \item \textbf{void setAlpha(int alpha)}: \textit{Deprecated in API 16.} Sets the overall opacity of the view (use \texttt{setImageAlpha(int)} instead).
        \item \textbf{final void setColorFilter(int color, PorterDuff.Mode mode)}: Applies a tint color and blending mode to the image.
        \item \textbf{void setColorFilter(ColorFilter cf)}: Applies a custom color filter to modify image appearance.
        \item \textbf{final void setColorFilter(int color)}: Applies a tint color to the image using the default blending mode.
        \item \textbf{void setCropToPadding(boolean cropToPadding)}: Determines whether the image is cropped to the view’s padding.
        \item \textbf{void setImageAlpha(int alpha)}: Sets the transparency level for the image drawable.
        \item \textbf{void setImageBitmap(Bitmap bm)}: Sets a bitmap as the content of the ImageView.
        \item \textbf{void setImageDrawable(Drawable drawable)}: Sets a drawable as the image content.
        \item \textbf{void setImageIcon(Icon icon)}: Sets an icon as the content.
        \item \textbf{void setImageMatrix(Matrix matrix)}: Applies a transformation matrix to the image drawable.
        \item \textbf{void setImageResource(int resId)}: Sets an image resource by its resource ID.
        \item \textbf{void setImageTintBlendMode(BlendMode blendMode)}: Sets how the tint blends with the image.
        \item \textbf{void setImageTintList(ColorStateList tint)}: Applies a color tint list to the image drawable.
        \item \textbf{void setImageTintMode(PorterDuff.Mode tintMode)}: Defines the blending mode for the tint.
        \item \textbf{void setScaleType(ImageView.ScaleType scaleType)}: Defines how the image should be scaled or cropped to fit within the view.
        \item \textbf{void onDraw(Canvas canvas)}: (Protected) Allows custom drawing of the image—useful for advanced visual effects.
    \end{itemize}

    \pagebreak 
    \subsubsection{CompoundButton}
    \begin{itemize}
        \item \textbf{Drawable getButtonDrawable()}: Returns the drawable used as the button image (e.g., checkmark or radio indicator).
        \item \textbf{BlendMode getButtonTintBlendMode()}: Returns the blending mode used to apply the tint to the button drawable.
        \item \textbf{ColorStateList getButtonTintList()}: Returns the color tint list applied to the button drawable.
        \item \textbf{PorterDuff.Mode getButtonTintMode()}: Returns the blending mode used for tinting.
        \item \textbf{void setButtonDrawable(int resId)}: Sets a drawable resource as the button image.
        \item \textbf{void setButtonDrawable(Drawable drawable)}: Sets a drawable object as the button image.
        \item \textbf{void setButtonIcon(Icon icon)}: Sets an icon as the visual button image.
        \item \textbf{void setButtonTintBlendMode(BlendMode tintMode)}: Defines how the tint color blends with the drawable.
        \item \textbf{void setButtonTintList(ColorStateList tint)}: Applies a tint list (different colors for different states) to the button image.
        \item \textbf{void setButtonTintMode(PorterDuff.Mode tintMode)}: Specifies the tint blending mode.
        \item \textbf{void drawableStateChanged()}: (Protected) Called when the state of the view changes in a way that affects drawable appearance (e.g., pressed, focused, checked).
        \item \textbf{int[] onCreateDrawableState(int extraSpace)}: (Protected) Generates the drawable state array — affects how state-based drawables (like selectors) are drawn.
        \item \textbf{void onDraw(Canvas canvas)}: (Protected) Used for custom drawing — allows overriding default appearance.
        \item \textbf{boolean verifyDrawable(Drawable who)}: (Protected) Ensures the drawable being displayed belongs to this view; relevant for custom visual drawables.
    \end{itemize}

    \pagebreak 
    \subsubsection{CheckBox (Use Button and CompoundButton styles)}

    \pagebreak 
    \subsubsection{RadioButton (Use button and CompoundButton styles)}

    \pagebreak 
    \subsubsection{Spinner}
    \begin{itemize}
        \item \textbf{int getDropDownHorizontalOffset()}: Returns the horizontal offset in pixels for positioning the dropdown popup.
        \item \textbf{int getDropDownVerticalOffset()}: Returns the vertical offset in pixels for positioning the dropdown popup.
        \item \textbf{int getDropDownWidth()}: Returns the configured width of the dropdown popup window.
        \item \textbf{int getGravity()}: Returns how the selected item view is positioned within the spinner (e.g., left, center, right).
        \item \textbf{Drawable getPopupBackground()}: Returns the background drawable used for the spinner’s dropdown popup.
        \item \textbf{void setDropDownHorizontalOffset(int pixels)}: Sets the horizontal offset in pixels for the spinner’s dropdown popup.
        \item \textbf{void setDropDownVerticalOffset(int pixels)}: Sets the vertical offset in pixels for the spinner’s dropdown popup.
        \item \textbf{void setDropDownWidth(int pixels)}: Sets the width in pixels for the spinner’s dropdown popup window.
        \item \textbf{void setGravity(int gravity)}: Defines how the selected item is positioned within the spinner (e.g., \texttt{Gravity.CENTER}).
        \item \textbf{void setPopupBackgroundDrawable(Drawable background)}: Sets a drawable as the background for the dropdown popup.
        \item \textbf{void setPopupBackgroundResource(int resId)}: Sets a background resource for the dropdown popup.
        \item \textbf{CharSequence getPrompt()}: Returns the prompt text displayed when the dialog version of the spinner is shown.
        \item \textbf{void setPrompt(CharSequence prompt)}: Sets custom prompt text for the spinner dialog.
        \item \textbf{void setPromptId(int promptId)}: Sets the prompt text by its string resource ID.
        \item \textbf{int getBaseline()}: Returns the text baseline offset (useful when aligning the spinner with other text views visually).
        \item \textbf{void onLayout(boolean changed, int l, int t, int r, int b)}: (Protected) Handles positioning of spinner items — can affect visual alignment.
        \item \textbf{void onMeasure(int widthMeasureSpec, int heightMeasureSpec)}: (Protected) Determines size and layout — affects how the spinner and dropdown appear.
    \end{itemize}
    \textbf{Styles from AdapterView}
    \begin{itemize}
        \item \textbf{void setSelection(int position)}: Visually highlights the selected item.
        \item \textbf{void setEmptyView(View emptyView)}: Specifies a view to display when the adapter is empty.
    \end{itemize}



    \pagebreak 
    \subsubsection{ProgessBar}
    \begin{itemize}
        \item \textbf{void drawableHotspotChanged(float x, float y)}: Called when the view hotspot changes and must be propagated to drawables or child views.
        \item \textbf{Drawable getCurrentDrawable()}: Returns the drawable currently used to draw the progress bar.
        \item \textbf{Drawable getIndeterminateDrawable()}: Returns the drawable used to draw the progress bar in indeterminate mode.
        \item \textbf{BlendMode getIndeterminateTintBlendMode()}: Returns the blending mode used to apply the tint to the indeterminate drawable.
        \item \textbf{ColorStateList getIndeterminateTintList()}: Returns the color tint list applied to the indeterminate drawable.
        \item \textbf{PorterDuff.Mode getIndeterminateTintMode()}: Returns the blending mode used to apply the tint to the indeterminate drawable.
        \item \textbf{Interpolator getInterpolator()}: Gets the acceleration curve for the indeterminate animation.
        \item \textbf{BlendMode getProgressBackgroundTintBlendMode()}: Returns the blending mode used to apply the tint to the progress background.
        \item \textbf{ColorStateList getProgressBackgroundTintList()}: Returns the tint list applied to the progress background.
        \item \textbf{PorterDuff.Mode getProgressBackgroundTintMode()}: Returns the blending mode used to apply the tint to the progress background.
        \item \textbf{Drawable getProgressDrawable()}: Returns the drawable used to draw the progress bar in progress mode.
        \item \textbf{BlendMode getProgressTintBlendMode()}: Returns the blending mode used to apply the tint to the progress drawable.
        \item \textbf{ColorStateList getProgressTintList()}: Returns the color tint list applied to the progress drawable.
        \item \textbf{PorterDuff.Mode getProgressTintMode()}: Returns the blending mode used to apply the tint to the progress drawable.
        \item \textbf{BlendMode getSecondaryProgressTintBlendMode()}: Returns the blending mode used to apply the tint to the secondary progress drawable.
        \item \textbf{ColorStateList getSecondaryProgressTintList()}: Returns the color tint list applied to the secondary progress drawable.
        \item \textbf{PorterDuff.Mode getSecondaryProgressTintMode()}: Returns the blending mode used to apply the tint to the secondary progress drawable.
        \item \textbf{void invalidateDrawable(Drawable dr)}: Invalidates the specified drawable, forcing a redraw.
        \item \textbf{void jumpDrawablesToCurrentState()}: Jumps all drawables to their current state instantly, skipping animations.
        \item \textbf{void setIndeterminateDrawable(Drawable d)}: Defines the drawable used to draw the progress bar in indeterminate mode.
        \item \textbf{void setIndeterminateDrawableTiled(Drawable d)}: Defines a tileable drawable for the indeterminate mode.
        \item \textbf{void setIndeterminateTintBlendMode(BlendMode blendMode)}: Specifies the blending mode for applying the indeterminate tint.
        \item \textbf{void setIndeterminateTintList(ColorStateList tint)}: Applies a color tint to the indeterminate drawable.
        \item \textbf{void setIndeterminateTintMode(PorterDuff.Mode tintMode)}: Specifies the blending mode for the indeterminate tint.
        \item \textbf{void setInterpolator(Interpolator interpolator)}: Sets the acceleration curve for the indeterminate animation.
        \item \textbf{void setProgressBackgroundTintBlendMode(BlendMode blendMode)}: Specifies the blending mode for the progress background tint.
        \item \textbf{void setProgressBackgroundTintList(ColorStateList tint)}: Applies a tint to the progress background.
        \item \textbf{void setProgressBackgroundTintMode(PorterDuff.Mode tintMode)}: Specifies the blending mode for the progress background tint.
        \item \textbf{void setProgressDrawable(Drawable d)}: Defines the drawable used to draw the progress bar in progress mode.
        \item \textbf{void setProgressDrawableTiled(Drawable d)}: Defines a tileable drawable for the progress bar in progress mode.
        \item \textbf{void setProgressTintBlendMode(BlendMode blendMode)}: Specifies the blending mode for the progress indicator tint.
        \item \textbf{void setProgressTintList(ColorStateList tint)}: Applies a tint to the progress indicator.
        \item \textbf{void setProgressTintMode(PorterDuff.Mode tintMode)}: Specifies the blending mode for the progress indicator tint.
        \item \textbf{void setSecondaryProgressTintBlendMode(BlendMode blendMode)}: Specifies the blending mode for the secondary progress tint.
        \item \textbf{void setSecondaryProgressTintList(ColorStateList tint)}: Applies a tint to the secondary progress indicator.
        \item \textbf{void setSecondaryProgressTintMode(PorterDuff.Mode tintMode)}: Specifies the blending mode for the secondary progress tint.
        \item \textbf{void drawableStateChanged()}: Called whenever the state of the view changes in a way that affects drawable appearance.
        \item \textbf{void onDraw(Canvas canvas)}: Used to perform custom drawing for the progress bar.
        \item \textbf{boolean verifyDrawable(Drawable who)}: Returns true if the specified drawable is managed and displayed by this view.
    \end{itemize}

    \pagebreak 
    \subsubsection{AbsSeekBar}
    \begin{itemize}
        \item \textbf{void drawableHotspotChanged(float x, float y)}: Called whenever the view hotspot changes and must be propagated to drawables or child views.
        \item \textbf{boolean getSplitTrack()}: Returns whether the track is visually split by the thumb.
        \item \textbf{Drawable getThumb()}: Returns the drawable representing the thumb — the draggable component indicating progress.
        \item \textbf{int getThumbOffset()}: Returns the amount by which the thumb extends beyond the track.
        \item \textbf{BlendMode getThumbTintBlendMode()}: Returns the blending mode used to apply the tint to the thumb drawable.
        \item \textbf{ColorStateList getThumbTintList()}: Returns the tint color list applied to the thumb drawable.
        \item \textbf{PorterDuff.Mode getThumbTintMode()}: Returns the blending mode used to apply the tint to the thumb drawable.
        \item \textbf{Drawable getTickMark()}: Returns the drawable used as the tick mark for each progress position.
        \item \textbf{BlendMode getTickMarkTintBlendMode()}: Returns the blending mode used to apply tint to the tick mark drawable.
        \item \textbf{ColorStateList getTickMarkTintList()}: Returns the tint color list applied to the tick mark drawable.
        \item \textbf{PorterDuff.Mode getTickMarkTintMode()}: Returns the blending mode used to apply tint to the tick mark drawable.
        \item \textbf{void jumpDrawablesToCurrentState()}: Immediately updates all drawables associated with this view to their current state.
        \item \textbf{void setSplitTrack(boolean splitTrack)}: Specifies whether the track should be visually split by the thumb.
        \item \textbf{void setThumb(Drawable thumb)}: Sets the drawable used as the thumb in the progress meter.
        \item \textbf{void setThumbOffset(int thumbOffset)}: Sets the offset allowing the thumb to extend beyond the track visually.
        \item \textbf{void setThumbTintBlendMode(BlendMode blendMode)}: Defines the blending mode used when applying tint to the thumb drawable.
        \item \textbf{void setThumbTintList(ColorStateList tint)}: Applies a tint color list to the thumb drawable.
        \item \textbf{void setThumbTintMode(PorterDuff.Mode tintMode)}: Specifies the blending mode used with the thumb tint.
        \item \textbf{void setTickMark(Drawable tickMark)}: Sets the drawable used as tick marks at each progress position.
        \item \textbf{void setTickMarkTintBlendMode(BlendMode blendMode)}: Specifies the blending mode used to apply tint to the tick mark drawable.
        \item \textbf{void setTickMarkTintList(ColorStateList tint)}: Applies a tint color list to the tick mark drawable.
        \item \textbf{void setTickMarkTintMode(PorterDuff.Mode tintMode)}: Specifies the blending mode used with the tick mark tint.
        \item \textbf{void drawableStateChanged()}: Called whenever the state of the view changes in a way that affects drawable appearance.
        \item \textbf{void onDraw(Canvas canvas)}: Implement this method to perform custom drawing operations for the view.
        \item \textbf{boolean verifyDrawable(Drawable who)}: Returns true if the specified drawable is being displayed by this view; subclasses override when managing custom drawables.
    \end{itemize}



    \pagebreak 
    \subsubsection{SeekBar (Use styles from ProgessBar and AbsSeekBar)}


    \pagebreak 
    \subsection{Used API}
    \subsubsection{View}

    \pagebreak 
    \subsubsection{ViewGroup}

    \pagebreak 
    \subsubsection{Context}

    \pagebreak 
    \subsubsection{ConstraintLayout}

    \pagebreak 
    \subsubsection{ConstraintLayout.LayoutParams}

    \pagebreak 
    \subsubsection{RelativeLayout}

    \pagebreak 
    \subsubsection{RelativeLayout.LayoutParams}

    \pagebreak 
    \subsubsection{LinearLayout}

    \pagebreak 
    \subsubsection{LinearLayout.LayoutParams}

    \pagebreak 
    \subsubsection{GridLayout}

    \pagebreak 
    \subsubsection{GridLayout.LayoutParams}

    \pagebreak 
    \subsubsection{TableLayout}

    \pagebreak 
    \subsubsection{TableLayout.LayoutParams}

    \pagebreak 
    \subsubsection{FrameLayout}

    \pagebreak 
    \subsubsection{FrameLayout.LayoutParams}

    \pagebreak 
    \subsubsection{ListView}

    \pagebreak 
    \subsubsection{TextView}

    \pagebreak 
    \subsubsection{EditText}

    \pagebreak 
    \subsubsection{Button}

    \pagebreak 
    \subsubsection{ImageView}

    \pagebreak 
    \subsubsection{CompoundButton}

    \pagebreak 
    \subsubsection{CheckBox}

    \pagebreak 
    \subsubsection{RadioButton}

    \pagebreak 
    \subsubsection{Spinner}

    \pagebreak 
    \subsubsection{Progessbar}

    \pagebreak 
    \subsubsection{AbsSeekBar}

    \pagebreak 
    \subsubsection{SeekBar}

    \pagebreak 
    \subsubsection{Drawable}

    \pagebreak 
    \subsubsection{GradientDrawable}












\end{document}
