\documentclass{report}

\input{~/dev/latex/template/preamble.tex}
\input{~/dev/latex/template/macros.tex}

\title{\Huge{}}
\author{\huge{Nathan Warner}}
\date{\huge{}}
\fancyhf{}
\rhead{}
\fancyhead[R]{\itshape Warner} % Left header: Section name
\fancyhead[L]{\itshape\leftmark}  % Right header: Page number
\cfoot{\thepage}
\renewcommand{\headrulewidth}{0pt} % Optional: Removes the header line
%\pagestyle{fancy}
%\fancyhf{}
%\lhead{Warner \thepage}
%\rhead{}
% \lhead{\leftmark}
%\cfoot{\thepage}
%\setborder
% \usepackage[default]{sourcecodepro}
% \usepackage[T1]{fontenc}

% Change the title
\hypersetup{
    pdftitle={Command Options}
}

\begin{document}
    % \maketitle
        \begin{titlepage}
       \begin{center}
           \vspace*{1cm}
    
           \textbf{Command Options}
    
           \vspace{0.5cm}
            
                
           \vspace{1.5cm}
    
           \textbf{Nathan Warner}
    
           \vfill
                
                
           \vspace{0.8cm}
         
           \includegraphics[width=0.4\textwidth]{~/niu/seal.png}
                
           Computer Science \\
           Northern Illinois University\\
           United States\\
           
                
       \end{center}
    \end{titlepage}
    \tableofcontents
    \pagebreak 
    \unsect{Command Options}
    \begin{itemize}
        \item man 
            \begin{itemize}
                \item -k is used to search the short descriptions and manual page names for the keyword specified as an argument.
                \item -S specify a section number
            \end{itemize}
        \item chmod 
            \begin{itemize}
                \item \textbf{-R} recursive
            \end{itemize}
        \item ssh
            \begin{itemize}
                \item -l login-name
                \item -X enable X11 forwarding
            \end{itemize}
        \item SCP
            \begin{itemize}
                \item -r recursive
                \item -C  Enables compression
                \item -l Limit bandwidth
            \end{itemize}
        \item set
            \begin{itemize}
                \item -v print shell input lines as they are read
                \item -x displays expanded commands and its arguments
            \end{itemize}
        \item awk
            \begin{itemize}
                \item -f: Reads the AWK program source from the file program-file, instead of from the first command line argument
                \item -F: Change the field separator
            \end{itemize}
        \item sed
            \begin{itemize}
                \item -n silent
                \item -e script (inline script ie in the command)
                \item -f script-file
                \item -i allow changes to input
            \end{itemize}
        \item who
            \begin{itemize}
                \item -u list users logged in 
            \end{itemize}
        \item date
            \begin{itemize}
                \item \%d day number
                \item \%m month number
                \item \%y year
                \item \%A day name
                \item \%B month name
                \item \%Z time zone
                \item \%p (am or pm)
                \item \%M minute
                \item \%H hour
                \item \%S seconds
            \end{itemize}
        \item echo
            \begin{itemize}
                \item -n no trailing newline
                \item -e enable escape sequneces
            \end{itemize}
        \item ls
            \begin{itemize}
                \item -l long list
                \item -a all
                \item -R recursive
            \end{itemize}
        \item wc
            \begin{itemize}
                \item -l lines
                \item -w words
                \item -c bytes (same as -m)
                \item -m chars
            \end{itemize}
        \item read 
            \begin{itemize}
                \item -p prompt
                \item -t timeout
                \item -s silent
            \end{itemize}
        \item history
            \begin{itemize}
                \item -c search for specific text
            \end{itemize}


    \end{itemize}
    
\end{document}
