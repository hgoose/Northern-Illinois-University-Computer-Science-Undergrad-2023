\documentclass{report}

\input{~/dev/latex/template/preamble.tex}
\input{~/dev/latex/template/macros.tex}

\title{\Huge{}}
\author{\huge{Nathan Warner}}
\date{\huge{}}
\fancyhf{}
\rhead{}
\fancyhead[R]{\itshape Warner} % Left header: Section name
\fancyhead[L]{\itshape\leftmark}  % Right header: Page number
\cfoot{\thepage}
\renewcommand{\headrulewidth}{0pt} % Optional: Removes the header line
%\pagestyle{fancy}
%\fancyhf{}
%\lhead{Warner \thepage}
%\rhead{}
% \lhead{\leftmark}
%\cfoot{\thepage}
%\setborder
% \usepackage[default]{sourcecodepro}
% \usepackage[T1]{fontenc}

% Change the title
\hypersetup{
    pdftitle={Quiz Answers}
}

\begin{document}
    % \maketitle
        \begin{titlepage}
       \begin{center}
           \vspace*{1cm}
    
           \textbf{Quiz Answers} \\
           CS-330 Unix and Network Programming
    
           \vspace{0.5cm}
            
                
           \vspace{1.5cm}
    
           \textbf{Nathan Warner}
    
           \vfill
                
                
           \vspace{0.8cm}
         
           \includegraphics[width=0.4\textwidth]{~/niu/seal.png}
                
           Computer Science \\
           Northern Illinois University\\
           March 1, 2024 \\
           United States\\
           
                
       \end{center}
    \end{titlepage}
    \tableofcontents
    \pagebreak 
    \unsect{Quiz 6: Awk}
    \begin{itemize}
        \item Which function in awk is used to divide a string into pieces separated by the field seperator and store the pieces in an array ?
            \begin{itemize}
                \item split
            \end{itemize}
        \item What is the meaning of the \$0 variable in awk ?
            \begin{itemize}
                \item it holds the entire record 
            \end{itemize}
        \item awk reads input lines automatically
            \begin{itemize}
                \item true
            \end{itemize}
        \item Variables in awk are initialized how ?
            \begin{itemize}
                \item to 0 or "" depending on context when first used
            \end{itemize}
        \item Every awk program must use the BEGIN and END patterns.
            \begin{itemize}
                \item false
            \end{itemize}
        \item Which command line option allows to specifiy an awk program in a file ?  
            \begin{itemize}
                \item \textbf{-f}
            \end{itemize}
        \item awk allows strings as array index.
            \begin{itemize}
                \item true
            \end{itemize}
        \item awk can use which character as field seperator ?
            \begin{itemize}
                \item any character
            \end{itemize}
        \item In awk, strings printed with the "\%20s" printf directive will always be left justified.
            \begin{itemize}
                \item false
            \end{itemize}
        \item What is the string concatenation operator in awk ?
            \begin{itemize}
                \item the space character
            \end{itemize}
    \end{itemize}

    \pagebreak 
    \unsect{Quiz 7: Sed}
    \begin{itemize}
        \item In sed, the "i" command adds lines before the address
            \begin{itemize}
                \item True
            \end{itemize}
        \item In sed, when using a range address, the lines specified need not be consecutive.
            \begin{itemize}
                \item False
            \end{itemize}
        \item In sed, the dollar sign (\$) can be used as a single line address. What is its meaning?
            \begin{itemize}
                \item the last line of input
            \end{itemize}
        \item If the exclamation mark "!" is used after an sed address, then it will only search for obsolete lines.
            \begin{itemize}
                \item False
            \end{itemize}
        \item Which character is used to delimit the search and replacement components of the sed "s" command?
            \begin{itemize}
                \item any, as long as all 3 are the same
            \end{itemize}
        \item If sed is invoked as "sed -n" then it sorts its input numerically.
            \begin{itemize}
                \item False
            \end{itemize}
        \item The sed editor can be called from a shell script
            \begin{itemize}
                \item True
            \end{itemize}
        \item Which of the following utilities can be used to systematically process files?
            \begin{itemize}
                \item all of the above
            \end{itemize}
        \item Which of the following is NOT a valid address type for sed?
            \begin{itemize}
                \item sublet address
            \end{itemize}
        \item If no address is specified for the sed command then the command is applied to every input line.
            \begin{itemize}
                \item true
            \end{itemize}
    \end{itemize}

    \pagebreak 
    \unsect{Quiz 9: Systems Programming}
    \begin{itemize}
        \item The C library function perror translate an error code into an understandable error message.
            \begin{itemize}
                \item True
            \end{itemize}
        \item In C++, C strings are handled the same way as instances of the standard string class.
            \begin{itemize}
                \item False
            \end{itemize}
        \item C++ can call functions from the standard C library. 
            \begin{itemize}
                \item true
            \end{itemize}
        \item In C++ what is the correct include to use the strlen or strcpy library functions ?
            \begin{itemize}
                \item #include <cstring>
            \end{itemize}
        \item The C regular expression library uses 2 functions: regcomp and regexec. Which statement is true about these functions?
            \begin{itemize}
                \item regexec runs the search that was prepared by regcomp
            \end{itemize}
        \item For the strcpy C library function, the size of the destination array must be long enough to contain the source string including the terminating null character.
            \begin{itemize}
                \item true
            \end{itemize}
        \item A C++ program cannot access environment variables.
            \begin{itemize}
                \item false
            \end{itemize}
        \item C library function exit terminates a process and allows to set the return status. A status of 0 indicates failure.
            \begin{itemize}
                \item false
            \end{itemize}
        \item The C structure dirent does not contain the file name.
            \begin{itemize}
                \item false
            \end{itemize}
        \item Which C library functions enable directory I/O ?
            \begin{itemize}
                \item opendir, readdir 
            \end{itemize}
    \end{itemize}

    When it lands on the spring it has the same energy as above $\frac{1}{2}mv_{i}^{2}  = 79 J$. That will all go into spring potential energy
    \begin{align*}
        79J = \frac{1}{2}kl^{2} \\
        \implies x = 0.4m
    .\end{align*}

    \pagebreak 
    \unsect{Quiz 10}
    \begin{itemize}
        \item A system call is linked to an executable and becomes logically part of the executable.
            \begin{itemize}
                \item False
            \end{itemize}
        \item C++ uses which of the following to identify a  file in low-level IO using system calls ?
            \begin{itemize}
                \item File descriptor number
            \end{itemize}
        \item The read and write system calls use C++ strings to handle its data.
            \begin{itemize}
                \item False
            \end{itemize}
        \item open, close, create, read and write are system calls to achieve low-level I/O.
            \begin{itemize}
                \item False
            \end{itemize}
        \item Which of the following is not a valid flag to for the open system call ?
            \begin{itemize}
                \item O\_READ
            \end{itemize}
        \item Which of to following is a system call that can be used to create a file ?
            \begin{itemize}
                \item Open
            \end{itemize}
        \item A system call uses a special C++ syntax for invocation.
            \begin{itemize}
                \item False
            \end{itemize}
        \item Which system call is used to remove a file ?
            \begin{itemize}
                \item unlink
            \end{itemize}
        \item The dup system call is used to claim standard I/O from inside a program.
            \begin{itemize}
                \item True
            \end{itemize}
        \item System calls typically set the errno variable and return -1 if they encounter and error.
            \begin{itemize}
                \item True
            \end{itemize}





       
    \end{itemize}



















    \end{itemize}
   
\end{document}
