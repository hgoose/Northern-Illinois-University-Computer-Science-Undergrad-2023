\documentclass{report}

\input{~/dev/latex/template/preamble.tex}
\input{~/dev/latex/template/macros.tex}

\title{\Huge{}}
\author{\huge{Nathan Warner}}
\date{\huge{}}
\pagestyle{fancy}
\fancyhf{}
\lhead{Warner \thepage}
\rhead{}
% \lhead{\leftmark}
\cfoot{\thepage}
%\setborder
% \usepackage[default]{sourcecodepro}
% \usepackage[T1]{fontenc}

\begin{document}
    % \maketitle
        \begin{titlepage}
       \begin{center}
           \vspace*{1cm}
    
           \textbf{Discrete Structures} \\ 
           Graph Theory
    
           \vspace{0.5cm}
            
                
           \vspace{1.5cm}
    
           \textbf{Nathan Warner}
    
           \vfill
                
                
           \vspace{0.8cm}
         
           \includegraphics[width=0.4\textwidth]{~/niu/seal.png}
                
           Computer Science \\
           Northern Illinois University\\
           August 31, 2023 \\
           United States\\
           
                
       \end{center}
    \end{titlepage}
    \tableofcontents
    \pagebreak \bigbreak \noindent
    \section{\LARGE Graphs}
    \smallbreak \noindent
    \begin{definition}
    \textbf{ A \textbf{graph} $G$ consists of two finite sets: a nonempty set $V(G)$ of vertices and a set $E(G)$ of edges, where each edge is associated with a set consisting of either one or two vertices called its endpoints. Formally, a \textbf{graph} is defined as an ordered pair $G = (V,E)$, where $V$ is the set of vertices and $E$ is the set of edges
        \begin{align*}
            G = (V,E) \\
            V = \{v_{1},v_{2},v_{3},...,v_{n}\} \\
            E = \{e_{1}, e_{2}, e_{3}, ..., e_{m}\}
        .\end{align*}
} 
    \end{definition}
    \bigbreak \noindent 
    \begin{minipage}{0.47\textwidth}
        \incfig{graph1}
    \end{minipage}
    \begin{minipage}{0.47\textwidth}
    \begin{align*}
        V = \{v_{1},v_{2}, v_{3}, v_{4}\} \\
        E = \{e_{1}, e_{2},e_{3},e_{4},e_{5}\}
    .\end{align*}
    We can also represent the edges by only stating the vertices which connect the edges
        \begin{center}
        \begin{tabular}{|l|c|}
        \hline
        Edges & Endpoints \\
        	\hline
        $e_{1}$& $\{v_{1},v_{2}\} $   \\
        	\hline
        $e_{2}$ & $\{v_{1}, v_{3}\} $ \\
        \hline
        $e_{3}$ & $\{v_{2}, v_{3}\} $ \\
        \hline 
        $e_{4}$ & $\{v_{3}, v_{4}\} $ \\
        \hline
        $e_{5}$ & $\{v_{4}\} $ \\
        \hline
        \end{tabular}
    \end{center}
    \end{minipage}

    \pagebreak \bigbreak \noindent 
    \section{\LARGE Subgraphs}
    \smallbreak \noindent
    \begin{definition}
    \textbf{ Graph $H$ is said to be a \textbf{subgraph} of a graph $H$ iff every vertex in $H$ is also a vertex in $G$, every edge in $H$ is also an edge in $G$, and every edge in $H$ has the same endpoints as it has in $G$.} 
    \end{definition}
    
\end{document}
