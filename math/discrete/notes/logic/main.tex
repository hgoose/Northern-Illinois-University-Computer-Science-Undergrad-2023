\documentclass{report}

\input{~/dev/latex/template/preamble.tex}
\input{~/dev/latex/template/macros.tex}

\title{\Huge{}}
\author{\huge{Nathan Warner}}
\date{\huge{}}
\pagestyle{fancy}
\fancyhf{}
\lhead{Warner \thepage}
\rhead{}
% \lhead{\leftmark}
\cfoot{\thepage}
\setborder
% \usepackage[default]{sourcecodepro}
% \usepackage[T1]{fontenc}

\begin{document}
    % \maketitle
        \begin{titlepage}
       \begin{center}
           \vspace*{1cm}
    
           \textbf{Discrete Structures} \\
           Logic
    
           \vspace{0.5cm}
            
                
           \vspace{1.5cm}
    
           A Document By: \\
           \textbf{Nathan Warner}
    
           \vfill
                
                
           \vspace{0.8cm}
         
           \includegraphics[width=0.4\textwidth]{~/niu/seal.png}
                
           Computer Science \\
           Northern Illinois University\\
           August 11, 2023 \\
           United States\\
           
                
       \end{center}
    \end{titlepage}
    \tableofcontents
    \pagebreak \bigbreak \noindent
    \section{Statements}
    \bigbreak \noindent 
    \begin{mdframed}
        \textbf{Definition:}
        A statement (or proposition) is a sentence that is either true or false (but not both)
    \end{mdframed}
    \bigbreak \noindent 
    \begin{mdframed}
      \textbf{Example: for the following, state whether it is a statement, or not a statement}
      \bigbreak \noindent 
      \textbf{A.) "I think it will rain tomorrow" is a statement.}
      \bigbreak \noindent 
      \textbf{B.) 3 - x = 12 }
      \bigbreak \noindent 
      \textbf{C.) 2 + 2 =3 }
      \bigbreak \noindent 
      \textbf{Solutions:}
      \bigbreak \noindent 
      \textbf{A.) The sentence is not a statement because there is a chance it will rain, or not.}
      \bigbreak \noindent 
      \textbf{B.) Though the bellow sentence is a mathematical expression, however it is not a statement because it is not either true or false. Depending on what x is, the sentence is either true or false, but right now it is neither.}
      \bigbreak \noindent 
      \textbf{C.) Even though the bellow expression is false, but it is a statement because it is either true or false, but not both, and in this case it is false. Therefore "2 + 2 = 3" is a statement.}
    \end{mdframed}

    \bigbreak \noindent \bigbreak \noindent 
    \section{Compound Statements}
    \textbf{Components of a statement}
    \begin{itemize}
      \item \textbf{$p$}: Represents predicate
      \item \textbf{$q$}: Represents conclusion
    \end{itemize}
    \bigbreak \noindent 
    \textbf{Logical Connectives}
    \begin{itemize}
      \item \textbf{$\wedge$}: Represents \textbf{and}
      \item \textbf{$\land$}: Represents \textbf{or}
      \item \textbf{$\neg$ or $\sim$}: Represents \textbf{negation}
    \end{itemize}
    \bigbreak \noindent 
    By utilizing logical connectives, we can create compound statements
    \pagebreak \bigbreak \noindent 
    \begin{mdframed}
      \textbf{Example: For each sentence, choose the correct compound statement}
      \bigbreak \noindent 
      \textbf{A.) The sentence "It is not hot or it is sunny" in symbols is:}
      \bigbreak \noindent 
      \textbf{B.) The expression, "$3 \leq a$" in writing is:}
      \bigbreak \noindent 
      \textbf{C.) If p: a week has seven days, q: there are 20 hours in a day, and r: there are 60 minutes in an hour, then $\sim$p $\wedge$ $\sim$r is:}
      \bigbreak \noindent 
      \textbf{Solutions:}
      \bigbreak \noindent 
      \textbf{A.) $\sim p \lor q$}
      \bigbreak \noindent 
      \textbf{B.) $3 > a$  or $3 = a $}
      \bigbreak \noindent 
      \textbf{C.) A week doesn't have 7 days and there are not 60 minutes in an hour. }
    \end{mdframed}

    \bigbreak \noindent \bigbreak \noindent 
    \section{Truth Tables}
    \bigbreak \noindent 
    Here is a simple example of a truth table for logical and:
    \bigbreak \noindent 
    \begin{center}
        \begin{tabular}{|c|c|c|}
            \hline
            $P$ & $Q$ & $P \land Q$ \\
            \hline
            T & T & T \\
            T & F & F \\
            F & T & F \\
            F & F & F \\
            \hline
        \end{tabular}
    \end{center}
    \bigbreak \noindent 
    Here is a simple example of a truth table for logical or (inclusive):
    \bigbreak \noindent 
    \begin{center}
        \begin{tabular}{|c|c|c|}
            \hline
            $P$ & $Q$ & $P \lor Q$ \\
            \hline
            T & T & T \\
            T & F & T \\
            F & T & T \\
            F & F & F \\
            \hline
        \end{tabular}
    \end{center}
    \bigbreak \noindent 
    Here is a simple example of a truth table for logical or (exclusive):
    \bigbreak \noindent 
    \begin{center}
        \begin{tabular}{|c|c|c|}
            \hline
            $P$ & $Q$ & $P \oplus Q$ \\
            \hline
            T & T & F \\
            T & F & T \\
            F & T & T \\
            F & F & F \\
            \hline
        \end{tabular}
    \end{center}
    \bigbreak \noindent 
    Here is a simple example of a truth table for logical not (negation):
    \begin{center}
        \begin{tabular}{|c|c|}
            \hline
            $P$ & $\lnot P$ \\
            \hline
            T & F \\
            F & T \\
            \hline
        \end{tabular}
    \end{center}
    \pagebreak \bigbreak \noindent 
    \begin{mdframed}
      \textbf{Example: Construct a truth table for $(p\land q) \lor  \neg r$}
      \bigbreak \noindent 
      \textit{Figure:}
      \bigbreak \noindent 
      \begin{center}
          \begin{tabular}{|c|c|c|c|c|c|}
            \hline
            \(p\) & \(q\) & \(r\) & \((p \land q)\) & \(\neg r\) & \((p \land q) \lor \neg r\) \\
            \hline
            T & T & T & T & F & T \\
            \hline 
            T & T & F & T & T & T \\
            \hline
            T & F &T & F &F&F \\
            \hline
            T&F&F&F&T&T \\
            \hline
            F&T&T&F&F&F \\
            \hline 
            F&T&F&F&T&T \\
            \hline
            F&F&T&F&T&F \\
            \hline
            F&F&F&F&F&T \\
            \hline
        \end{tabular}
      \end{center}
    \end{mdframed}

    \bigbreak \noindent \bigbreak \noindent 
    \section{Logical Equivalence}
    \bigbreak \noindent 
    \begin{mdframed}
        \textbf{Definition}:
        Statements $p$ and $q$ are said to be logically equivalent if they have the same truth value in every model. \\
        \textbf{Notation:} the notation for logical equivalence is $\equiv $
        % in other words 
    \end{mdframed}
    \bigbreak \noindent 
    Say we have statements $p$ and $q$, and we want to show that $p \land q$, and $q \land p$ are logically equivalent, to do this, we must first construct a  truth table:
    \begin{center}
        \begin{tabular}{|c|c|c|c|}
        \hline
        $p$ & $q$ & $p\land q$ & $q \land p $\\
        \hline
        T&T&T&T  \\
        \hline
        T&F&F&F \\
        \hline
        F&T&F&F \\
        \hline   
        F&F&F&F \\
        \hline 
        \end{tabular}
    \end{center}
    \bigbreak \noindent 
    So we can see that the columns $p\land q$, $q\land p$ have the same truth values, therefore they are said to be \textbf{logically equivalent}

    \pagebreak \bigbreak \noindent 
    \section{Tautologies and Contradictions}
    \bigbreak \noindent 
    \begin{mdframed}
        \textbf{Definition:}
          A \textbf{Tautology} is a Statement that is always true, a assertion that is true in every possible interpretation \\
          A \textbf{Contradiction} is a statement that is always false.
    \end{mdframed}
    \bigbreak \noindent 
    Consider the following compound statement
    \begin{align*}
        p \lor \neg p
    .\end{align*}
    \bigbreak \noindent 
    Because this statement can never be false, we say it is a \textbf{Tautology}
    \bigbreak \noindent 
    \textbf{Contradiction}
    \bigbreak \noindent 
    Consider the statement:
    \begin{align*}
         p \land \neg p
    .\end{align*}
    \bigbreak \noindent 
    Because this statement can never be true, we say it is a \textbf{Contradiction}

    \bigbreak \noindent \bigbreak \noindent 
    \section{De Morgan's Laws}
    \bigbreak \noindent 
    De Morgan's Laws are:
    \begin{itemize}
        \item $\neg(p\land q) = \neg p \lor \neg q$
        \item $\neg(p\lor q) = \neg p \land \neg q$
    \end{itemize}
    \bigbreak \noindent 
    Consider the statement:
    \begin{align*}
         0 < x \leq 3
    .\end{align*}
    \bigbreak \noindent 
    To use De Morgan's Law, which states:
    \begin{align*}
        \neg(p \land q) = \neg p \lor \neg q
    .\end{align*}
    We can rewrite the statement as:
    \begin{align*}
        0 \geq x\ or\ x > 3
    .\end{align*}
    
    \pagebreak \bigbreak \noindent 
    \section{Logical Equivalence Laws}
    \bigbreak \noindent 
    \begin{center}
        \begin{array}{|l|l|l|}
            \hline \text { Commutative laws: } & \mathrm{p} \wedge \mathrm{q} \equiv \mathrm{q} \wedge \mathrm{p} & p \vee q \equiv q \vee p \\
            \hline \text { Associative laws: } & (\mathrm{p} \wedge \mathrm{q}) \wedge \mathrm{r} \equiv \mathrm{p} \wedge(\mathrm{q} \wedge \mathrm{r}) & (p \vee q) \vee r \equiv p \vee(q \vee r) \\
            \hline \text { Distributive laws: } & \mathrm{p} \wedge(\mathrm{q} \vee \mathrm{r}) \equiv(\mathrm{p} \wedge \mathrm{q}) \vee(\mathrm{p} \wedge \mathrm{r}) & p \vee(q \wedge r) \equiv(p \vee q) \wedge(p \vee r) \\
            \hline \text { Identity laws: } & \mathrm{p} \wedge \mathbf{t} \equiv \mathrm{p} & p \vee \mathbf{c} \equiv p \\
            \hline \text { Negation laws: } & \mathrm{p} \vee \sim \mathrm{p} \equiv \mathbf{t} & p \wedge \sim p \equiv \mathbf{c} \\
            \hline \text { Double negative law: } & \sim(\sim \mathrm{p}) \equiv \mathrm{p} & \\
            \hline \text { Idempotent laws: } & \mathrm{p} \wedge \mathrm{p} \equiv \mathrm{p} & p \vee p \equiv p \\
            \hline \text { Universal bound laws: } & \mathrm{p} \vee \mathrm{t} \equiv \mathbf{t} & p \wedge \mathbf{c} \equiv \mathbf{c} \\
            \hline \text { DeMorgan's laws: } & \sim(\mathrm{p} \wedge \mathrm{q}) \equiv \sim \mathrm{p} \vee \sim \mathrm{q} & \sim(p \vee q) \equiv \sim p \wedge \sim q \\
            \hline \text { Absorption laws: } & \mathrm{p} \vee(\mathrm{p} \wedge \mathrm{q}) \equiv \mathrm{p} & p \wedge(p \vee q) \equiv p \\
            \hline \text { Negation of $t$ and $c$ } & $\neg t = c $ & $\neg c = t$  \\
            \hline
        \end{array}
    \end{center}

    \bigbreak \noindent \bigbreak \noindent 
    \section{Conditional Statements}
    \bigbreak \noindent 
    \begin{mdframed}
        \textbf{Definition:}
        A \textbf{conditional statement} is a statement that can be written in the form “If P then Q,” where P and Q are sentences. \\
        \textbf{Syntax:} if \textit{statement} then \textit{statement}
    \end{mdframed}
    \bigbreak \noindent 
    Consider the statment
    \begin{align*}
        p \rightarrow q
    .\end{align*}
    This statement, read "if p then q", can be described with the following truth table:
    \begin{center}
        \begin{tabular}{|l|c|c|}
        \hline
        p & q & $p \rightarrow q $\\
        	\hline
        T &T  & T  \\
        	\hline
        T&F & F\\
        \hline
        F&T & T\\
        \hline
        F&F & T\\
        \hline
        \end{tabular}
    \end{center}
    \bigbreak \noindent 
    \nt{To get a truth value of "true" in $p \rightarrow q $, either $p $ and $q $ both need to be true, or both need to be false, or q needs to be true}

    \pagebreak \bigbreak \noindent 
    \section{Negation of Conditional Statements}
    \bigbreak \noindent 
    \begin{mdframed}
        \textbf{Definition:}
       The \textbf{Negation of conditional statement} is logically equivalent to a conjunction of the antecedent and the negation of the consequent.
       \begin{align*}
           Negation:\ p \rightarrow q \equiv p \land \neg q 
       .\end{align*}
       \bigbreak \noindent 
       The \textbf{Contrapositive} of a conditional statement is a combination of the converse and the inverse
       \begin{align*}
           p \rightarrow q \equiv \neg q \rightarrow \neg p
       .\end{align*}


    \end{mdframed}
    \bigbreak \noindent 
    Consider the following condition
    \begin{align*}
       p \rightarrow q 
    .\end{align*}
    \bigbreak \noindent 
    Which we know is logically equivalent to:
    \begin{align*}
        p \rightarrow q \equiv \neg p \lor q
    .\end{align*}
    \bigbreak \noindent 
    By use of De Morgan's Law, which states that:
    \begin{align*}
        \neg(p\lor q) \equiv \neg p \land \neg q
    .\end{align*}
    \bigbreak \noindent 
    We can negate $p \rightarrow q$, so:
    \begin{align*}
        \neg(p \lor q) \equiv \neg(\neg p) \land \neg q
    .\end{align*}
    \bigbreak \noindent 
    Consider the following Conditional Statement
    \begin{center}
        If my dad is at home then he cant pick me up \\
        p $\rightarrow$q
    \end{center}
    \bigbreak \noindent 
    Then the negation would be:
    \begin{align*}
        &p \land \neg q \\
        &\equiv \text{my dad is at home and he can pick me up}
    .\end{align*}


    \pagebreak \bigbreak \noindent 
    \section{Converse and Inverse}
    \bigbreak \noindent 
    \begin{mdframed}
        \textbf{Definition:}
       The \textbf{Converse} of a conditional statement is created when the hypothesis and conclusion are reversed \\
       The \textbf{Inverse} of a conditional statement is when both the hypothesis and conclusion are negated
    \end{mdframed}
    \bigbreak \noindent 
    Consider the statement
    \begin{align*}
        p \rightarrow q
    .\end{align*}
    \bigbreak \noindent 
    Then the \textbf{Converse} would be:
    \begin{align*}
        q \rightarrow p
    .\end{align*}
    \bigbreak \noindent 
    And the \textbf{Inverse} would be:
    \begin{align*}
        \neg p \rightarrow \neg q
    .\end{align*}

    \bigbreak \noindent \bigbreak \noindent 
    \section{Biconditional Statements}
    \bigbreak \noindent 
    \begin{mdframed}
        \textbf{Definition:}
       A \textbf{Biconditional Statement} is a true statement that combines a hypothesis and conclusion with the words 'if and only if' instead of the words 'if' and 'then'
    \end{mdframed}
    \bigbreak \noindent 
    Say we have the following  statement
    \begin{align*}
        q \iff p
    .\end{align*}
    Then the truth table would be:
    \begin{center}
        \begin{tabular}{|l|c|c|}
        \hline
        p & q  & q \iff p \\
        	\hline
        T&T&T   \\
        	\hline
        F&T&F \\
        \hline
        T&F&F \\ 
        \hline
        F&F&T \\
        \hline

        \end{tabular}
    \end{center}
    \bigbreak \noindent 
    \nt{Similar to conditional statements, in the truth table, $p \iff q$ is true if both p and q have the same value. So false false will be true}
    \bigbreak \noindent 




    






    
\end{document}
