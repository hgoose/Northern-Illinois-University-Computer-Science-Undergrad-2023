\documentclass{report}

\input{~/dev/latex/template/preamble.tex}
\input{~/dev/latex/template/macros.tex}

\title{\Huge{}}
\author{\huge{Nathan Warner}}
\date{\huge{}}
\pagestyle{fancy}
\fancyhf{}
\lhead{Warner \thepage}
\rhead{}
% \lhead{\leftmark}
\cfoot{\thepage}
\setborder
% \usepackage[default]{sourcecodepro}
% \usepackage[T1]{fontenc}

\begin{document}
    % \maketitle
        \begin{titlepage}
       \begin{center}
           \vspace*{1cm}
    
           \textbf{Discrete Structures} \\
           \vspace{0.5cm}
           Number Theory
            
                
           \vspace{1.5cm}
            A Document By:  \\
           \textbf{Nathan Warner}
    
           \vfill
                
                
           \vspace{0.8cm}
         
           \includegraphics[width=0.4\textwidth]{~/niu/seal.png}
                
           Computer Science \\
           Northern Illinois University\\
           August 16, 2023 \\
           United States\\
           
                
       \end{center}
    \end{titlepage}
    \tableofcontents
    \pagebreak \bigbreak \noindent
    \vspace{2in} \\
    \begin{Huge}
        \textbf{Number \bigbreak \noindent  Theory}
    \end{Huge}
    \smallbreak \noindent
    \line(1,0){490}
    \section{\LARGE Introduction}
    \bigbreak \noindent 
    \textbf{Definition 1.} Number theory is the branch of mathematics that deals with the properties and relationships of numbers, especially the positive integers.

    \bigbreak \noindent \bigbreak \noindent 
    \section{\LARGE Parity}
    \bigbreak \noindent 
    \textbf{Definition 2.} The \textbf{Parity} of an integer is its attribute of being even or odd
    \bigbreak \noindent 
    \textbf{Even integers} are in the form
    \bigbreak \noindent 
    \begin{align*}
        2k, \quad \text{where $k $ is an integer}
    .\end{align*}
    \bigbreak \noindent 
    \textbf{Odd Integers} are in the form
    \bigbreak \noindent 
    \begin{align*}
        2k+1,\quad \text{where k is an integer}
    .\end{align*}
    \bigbreak \noindent 
    \textbf{Proposition 1.} Zero is an even integer. Let's consider the integer \textit{zero}. By the common and standard definition, zero is considered an even integer. This is based on the property mentioned above: an even integer can be written in the form 2k, where k is an integer. When you plug in k = 0, you get 2 * 0 = 0, which is an even integer. In this sense, zero fits the definition of an even number. 
    If you were to apply the definition of an odd integer, which is in the form 2k + 1, and plug in k = 0, you would indeed get 2 * 0 + 1 = 1. This would suggest that zero is odd. However, this is not the definition that is traditionally used.
    \bigbreak \noindent 
    Suppose we have the integer $-101$, how might we show that this is an odd integer? Well, let's see if we can derive an equation from the rule $2k+1,\ \text{where k is an integer}$
    \begin{align*}
        2(-51) + 1 = -101
    .\end{align*}
    \bigbreak \noindent 
    Thus, -101 is an odd integer
    \bigbreak \noindent 
    Now let's suppose we have the equation:
    \begin{align*}
        6a+8b = 1, \quad \text{where a and b are integers}
    .\end{align*}
    \bigbreak \noindent 
    What result might this yield? Even or Odd?
    \bigbreak \noindent 
    To further examine this, we can use algebra. So:
    \begin{align*}
        2(3a+4b) = 1 \\ 
    .\end{align*}
    \bigbreak \noindent 
    Since we know that $a $ and $b $ are integers, then we also know that $3a+4b$ must also an integer. Thus, our equation is in the form $2k+1$, where k is an integer. This means that this equation must produce an odd integer
    \bigbreak \noindent 
    Now suppose we have the equation
    \begin{align*}
        4a^{2}b
    .\end{align*}
    \bigbreak \noindent 
    Might this result in an even, or odd integer. To find out we can again utilize some algebra. So:
    \begin{align*}
        2(2a^{2}b), \quad \text{where $a$ and $b$ are integers}
    .\end{align*}
    \bigbreak \noindent 
    And since we know $a$ and $b $ are both integers, the product of $2a^{2}b$ must also be an integer. Thus, our equation is in the form $2k$, where $k $ is an integer and this equation must yield an even integer.


    \bigbreak \noindent \bigbreak \noindent 
    \section{\LARGE Divisibility}
    \bigbreak \noindent 
    \textbf{Definition 1.} Suppose we have $n,d \in \mathbb{Z}$ where $d\ne 0$. Then $n|d \iff \exists k \in \mathbb{Z}\ |\ n = dk$. Where $n|d$ is read "$n$ divides $d$". If this theorem holds for any arbitrary integers $n,d$, then we say "$n$ divides $d$"
    \bigbreak \noindent 
    Let's suppose we have
    \begin{align*}
        \frac{6}{18}
    .\end{align*}
    \bigbreak \noindent 
    \textbf{Divisibility Rules.}
    \begin{enumerate}
        \item Divisible by 1: All integers are divisible by 1
        \item Divisible by 2: If the last digit of the integer is even.
        \item Divisible by 3: If the sum of the digit’s numbers are divisible by 3.
        \item Divisible by 4: If the last 2 digits are divisible by 4
        \item Divisible by 5: If the last digit is either 0 or 5.
        \item Divisible by 6: If it is divisible by both 2 and 3. (For divisibility by 2 and 3, check rule 2 and 3)
        \item Divisible by 7:  If you double the last digit and subtract it from the rest of the number and the answer is either:
        \begin{itemize}
            \item 0
            \item divisible by 7
        \end{itemize}
        \item Divisible by 8:  If the last three digits are divisible by 8.
        \item Divisible by 9: If the sum of the digits are divisible by 9
        \item Divisible by 10: If the number ends in 0.
        \item Divisible by 11:  Add and subtract digits in an alternating pattern (add first, subtract second, add third, etc). Then the answer must be either:
            \begin{itemize}
                \item 0 
                \item Divisible by 11
            \end{itemize}
        \item Divisible by 12:  If the number is both divisible by 3 and 4. (check divisibility rules for 3 and 4)
    \end{enumerate}

    \bigbreak \noindent \bigbreak \noindent 
    \section{\LARGE Prime Numbers}
    \bigbreak \noindent 
    \textbf{Definition 1.}  \textbf{Prime Numbers} are integers greater than 1 whose only factors are 1 and the number itself.
   \bigbreak \noindent 
   \textbf{Definition 2.}    \textbf{Composite Numbers} are integers numbers greater than 1 and not prime
   \bigbreak \noindent 
   \textbf{Definition 3.} \textbf{Fundamental theorem of arithmetic (Prime factorization theorem)} states that Any integer greater than 1 is either a prime number, or can be written as a unique product of prime numbers (ignoring the order).

   \bigbreak \noindent \bigbreak \noindent 
   \section{\LARGE Prime factorization}
   \bigbreak \noindent 
   \textbf{Definition 1.} Prime factorization is finding which prime numbers multiply together to make the original number.
   \bigbreak \noindent 
   Suppose we have the composite number $20$. Which prime numbers multiply to make 20? 
   \begin{align*}
       2 \cdot  2 \cdot  5
   .\end{align*}
   So we say that the prime factors of 20 are $2 \cdot 2 \cdot 5 $
   \bigbreak \noindent 
   To easily find the prime factorization of any composite number, we must follow these steps:
   \begin{enumerate}
       \item Find the smallest prime number that divides into our composite number.
        \item Repeat steps until you are left with 1
   \end{enumerate}
   For example, say we want to find the prime factorization of 50
   \begin{align*}
       2 | 50 = 25 \\
       5 | 25 = 5 \\
       5 | 5  = 1
   .\end{align*}
   So in this case our prime factors are $2\cdot 5\cdot 5 $
   \bigbreak \noindent 
   \textbf{Definition 2.} The \textbf{Unique factorization theorem} States that there's only one set of possible prime factors that can create a composite number

   \pagebreak \bigbreak \noindent 
   \section{\LARGE GCD and LCM}
   \bigbreak \noindent 
   \textbf{Definition 1.} The greatest common divisor (GCD) of two nonzero integers $a $ and $b $ is the greatest positive integer $d $ such that $d $ is a divisor of both $a $ and $b $
   \bigbreak \noindent 
   Suppose we have $a=6$ $b=10$. If we list out the divisors of both numbers:
   \begin{align*}
       6:\ 1,2,3,6 \\
       10:\ 1,2,5,10
   .\end{align*}
   \bigbreak \noindent 
   Then we can clearly see that the GCD is 2. Thus, $gcd(10,6) = 2 $

   \bigbreak \noindent 
   \textbf{Definition 2.}  A \textbf{Multiple} of a number is a number that is the product of a given number and some other natural number
   \bigbreak \noindent 
   \textbf{Definition 3.} The \textbf{Least Common Multiple (LCD)} is the smallest multiple that two or more numbers have in common
   \bigbreak \noindent 
   Suppose we want to find
   \begin{align*}
       LCD(10,4)
   .\end{align*}
   \bigbreak \noindent 
   So we can list the multiples of both numbers and then find the smallest multiple that is common between both numbers. So:
   \begin{align*}
       4:\ 4,8,12,16,20,24,... \\
       10:\ 10,20,30,40,...
   .\end{align*}
   \bigbreak \noindent 
   So we can see that:
   \begin{align*}
       LCD(10,4) = 20
   .\end{align*}















\end{document}
