\documentclass{report}

\input{~/dev/latex/template/preamble.tex}
\input{~/dev/latex/template/macros.tex}

\title{\Huge{}}
\author{\huge{Nathan Warner}}
\date{\huge{}}
\pagestyle{fancy}
\fancyhf{}
\lhead{Warner \thepage}
\rhead{}
% \lhead{\leftmark}
\cfoot{\thepage}
% \setborder
% \usepackage[default]{sourcecodepro}
% \usepackage[T1]{fontenc}

\begin{document}
    % \maketitle
        \begin{titlepage}
       \begin{center}
           \vspace*{1cm}
    
           \textbf{Discrete Structures} \\
           Relations
    
           \vspace{0.5cm}
            
                
           \vspace{1.5cm}
    
           \textbf{Nathan Warner}
    
           \vfill
                
                
           \vspace{0.8cm}
         
           \includegraphics[width=0.4\textwidth]{~/niu/seal.png}
                
           Computer Science\\
           Northern Illinois University\\
           August 23, 2023\\
           United States\\
           
                
       \end{center}
    \end{titlepage}
    \tableofcontents
    \pagebreak \bigbreak \noindent
    \vspace{2in} \\
    \begin{Huge}
       \textbf{Relations} 
    \end{Huge}
    \bigbreak \noindent 
    \line(1,0){490}
    \bigbreak \noindent 
    
    \section{\LARGE The language of relations}
    \smallbreak \noindent
    \begin{definition}
    \textbf{A \textbf{relation} is the relationship between two or more set of values} 
    \end{definition}
    \bigbreak \noindent 
    Suppose we have two sets $A$ and $B$, and $A \subseteq B$, then $A$ and $B$ are said to be \textbf{related}, because there is some attribute that binds them together. We can make a similar argument for $x>y$ for some $x$ and $y$. This is the general idea behind relations.
    \bigbreak \noindent 
    Suppose we have the sets:
    \begin{align*}
        A = \{1,2,3\} \quad B = \{2,3,4\}
    .\end{align*}
    \bigbreak \noindent 
    And we can create a relationship by saying: \textit{x is related to y $\iff$ $x<y$}, which is longhand for $x\ R\ y$. 
    \begin{align*}
        &1\ \mathrel{R}\ 2\ \checkmark \\
        &1\ \mathrel{R}\ 3\ \checkmark \\
        &1\ \mathrel{R}\ 4\ \checkmark \\
        &2\ \not \mathrel{R}\ 2 \\
        &2\ \mathrel{R}\ 3\ \checkmark\\
        &2\ \mathrel{R}\ 4\ \checkmark\\
        &3\ \not \mathrel{R}\ 2 \\
        &3\ \not \mathrel{R}\ 3 \\
        &3\ \mathrel{R}\ 4\ \checkmark\\
    .\end{align*}
    \bigbreak \noindent 
    And we can write it in terms of \textbf{ordered pairs}
    \begin{align*}
        \{(1,2),(1,3),(1,4),(2,3),(2,4),(3,4)\}
    .\end{align*}

    \pagebreak \bigbreak \noindent 
    \section{\LARGE Relations on sets}
    \bigbreak \noindent 
    Let $a$ and $b$ be sets. A relation $\mathrel{R}$ from $A$ to $B$ is a subset of $A \times B$. Given an ordered pair $(x,y)$ in $A\times B$, $x$ is related to $y$ by $\mathrel{R}$, written $x\ \mathrel{R}\ y$, if, and only if, $(x,y)$ is in $\mathrel{R}$. The set $A$ is called the domain of $\mathrel{R}$ and the set $B$ is called its co-domain
    \bigbreak \noindent 
    Suppose we have the sets:
    \begin{align*}
        A = \{1,2,3\} \quad B = \{1,3,5,6\}
    .\end{align*}

    \bigbreak \noindent 
    Suppose the relation $S$ means $x < y$, then we have:
    \begin{align*}
        x\ S\ y = \{(1,3),(1,5),(1,6),(2,3),(2,5),(2,6),(3,5),(3,6)\} 
    .\end{align*}

    \bigbreak \noindent 
    \nt{Notice here we are using $S$ to denote our relation, deduce that use of $\mathrel{R}$ is not strictly enforced, we can use any letter.}

    \pagebreak \bigbreak \noindent 
    \section{\LARGE Inverse of Relations}
    \bigbreak \noindent 
    \smallbreak \noindent
    \begin{definition}
        \textbf{Let $\mathrel{R}$ be a relation from $A$ to $B$. Define the inverse relation $\mathrel{R}^{-1}$ from $B$ to $A$ as follows:
            \begin{align*}
                \mathrel{R}^{-1} = \{(y,x) \in B \times A\ |\ (x,y) \in \mathrel{R}\}
            .\end{align*}
        } 
    \end{definition}
    \bigbreak \noindent 
    Suppose we have the sets:
    \begin{align*}
        A = \{2,3,4\} \quad B = \{5,6,8\} \\
        Where:\ x\ \mathrel{R}\ y \iff x|y
    .\end{align*}
    \bigbreak \noindent 
    Then we have:
    \begin{align*}
        \mathrel{R} = \{(2,6),(2,8),(3,6),(4,8)\} \\
        \mathrel{R}^{-1} = \{(6,2),(8,2),(6,3),(8,4)\}
    .\end{align*}

    \pagebreak \bigbreak \noindent 
    \section{\LARGE Reflexivity, Symmetry, and Transitivity}
    \smallbreak \noindent
    \begin{definition}
    \textbf{A binary relation \( R \) on a set \( A \) is said to be \textbf{reflexive} if every element is related to itself. Formally, a relation \( R \) is reflexive if for every \( a \in A \), the pair \( (a, a) \) is in \( R \).} 
      \begin{align*}
        \forall a \in A,\ (a,a) \in \mathrel{R}
      .\end{align*}
    \end{definition}
    \smallbreak \noindent
    \begin{definition}
    \textbf{A binary relation \( R \) on a set \( A \) is said to be \textbf{symmetric} if the relation holds in both directions between any two elements that are related. Formally, \( R \) is symmetric if for every \( (a, b) \in R \), \( (b, a) \) is also in \( R \).} 
      \begin{align*}
        \forall (a, b) \in R, (b, a) \in R
      .\end{align*}
    \end{definition}
    \smallbreak \noindent
    \begin{definition}
    \textbf{A binary relation \( R \) on a set \( A \) is said to be \textbf{transitive} if the existence of a relation from one element to a second, and from the second element to a third, implies the existence of a relation from the first element to the third. Formally, \( R \) is transitive if for every \( (a, b) \in R \) and \( (b, c) \in R \), \( (a, c) \) is also in \( R \).} 
      \begin{align*}
        \forall (a, b) \in R, (b, c) \in R \Rightarrow (a, c) \in R
      .\end{align*}
    \end{definition}

    \pagebreak \bigbreak \noindent 
    \section{\LARGE Properties of equality and less than}
    \bigbreak \noindent 
    \subsection{Equality Relation}
    \begin{enumerate}
        \item \textbf{Reflexive}: For all \(a\), \(a = a\).
        \item \textbf{Symmetric}: If \(a = b\), then \(b = a\).
        \item \textbf{Transitive}: If \(a = b\) and \(b = c\), then \(a = c\).
    \end{enumerate}
    \subsection{Less Than Relation}
    \begin{enumerate}
        \item \textbf{Irreflexive}: For all \(a\), it is not the case that \(a < a\).
        \item \textbf{Asymmetric}: If \(a < b\), then it is not the case that \(b < a\).
        \item \textbf{Transitive}: If \(a < b\) and \(b < c\), then \(a < c\).
    \end{enumerate}

    \pagebreak \bigbreak \noindent 
    \section{\LARGE Equivalence Relation}
    \smallbreak \noindent
    \begin{definition}
        \textbf{$\mathrel{R}$ is an \textbf{equivalence} relation iff $\mathrel{R}$ is:}  
        \begin{enumerate}
            \item Reflexive
            \item Symmetric
            \item Transitive
        \end{enumerate}
    \end{definition}

    \pagebreak \bigbreak \noindent 
    \section{\LARGE Equivalence Classes}
    \smallbreak \noindent
    \begin{definition}
       Let $A$ be a set and $\mathrel{R}$ be an equivalence relation on $A$. For each element in $A$, the \textbf{equivalence class} of $a$, denoted [a] and called the \textbf{class of a} for short, is the set of all elements $x$ in $A$ such that $x$ is related to $a$ by $\mathrel{R}$
        \begin{align*}
            [a] = \{x \in A\ |\ x\ \mathrel{R}\ a\}
        .\end{align*}
    \end{definition}
    \bigbreak \noindent 
    Suppose we have:
    \begin{align*}
        &A = \{0,1,2,3,4\} \\ &\mathrel{R} = \{(0,0),(0,4),(1,1),(1,3),(2,2),(3,1),(3,3),(4,0),(4,4)\}
    .\end{align*}
    \bigbreak \noindent 
    Then an example of a few \textbf{equivalence classes} would be:
    \begin{align*}
        [0] = \{0,4\} \\
        [1] = \{1,3\} \\
        etc...
    .\end{align*}




\end{document}
