\documentclass{report}

\input{~/dev/latex/template/preamble.tex}
\input{~/dev/latex/template/macros.tex}

\title{\Huge{}}
\author{\huge{Nathan Warner}}
\date{\huge{}}
\pagestyle{fancy}
\fancyhf{}
\lhead{Warner \thepage}
\rhead{}
% \lhead{\leftmark}
\cfoot{\thepage}
% \setborder
% \usepackage[default]{sourcecodepro}
% \usepackage[T1]{fontenc}

\begin{document}
    % \maketitle
        \begin{titlepage}
       \begin{center}
           \vspace*{1cm}
    
           \textbf{Discrete Structures} \\
           Functions
    
           \vspace{0.5cm}
            
                
           \vspace{1.5cm}
    
           \textbf{Nathan Warner}
    
           \vfill
                
                
           \vspace{0.8cm}
         
           \includegraphics[width=0.4\textwidth]{~/niu/seal.png}
                
           Computer Science \\
           Northern Illinois University\\
           August 22, 2023 \\
           United States\\
           
                
       \end{center}
    \end{titlepage}
    \tableofcontents
    \pagebreak \bigbreak \noindent 
    \vspace{2in} \\
    \begin{Huge}
        \textbf{Functions}
    \end{Huge}
    \bigbreak \noindent 
    \line(1,0){490}
    \bigbreak \noindent 
    
    \bigbreak \noindent \bigbreak \noindent 
    \section*{\LARGE Preface}
    \bigbreak \noindent 
    Much of the information covered in this chapter has been purposely omitted, as most of this chapter is trivial for people with a background in algebra.
    
    \pagebreak \bigbreak \noindent 
    \section{\LARGE Vocabulary}
      \bigbreak \noindent 
            \begin{itemize}
        \item A function $f:\ A \rightarrow B$ is \textbf{One-to-One (injective)} is injective if every element of $A$ has a unique image in $B$
        \item A function $f:\ A \rightarrow B$ is \textbf{Onto (surjective)}  is surjective if every element of $B$ is the image of at least one element of $A$.
        \item A function $f:\ A \rightarrow B$ is \textbf{Bijective} if it is both injective and surjective.
        \item The \textbf{Inverse} of a function reverses the direction of the original function. A function $f:\ A \rightarrow B$ has an inverse $f^{-1}:\ B \rightarrow A $ iff
          \begin{itemize}
            \item $f$ is bijective (both injective and surjective).
            \item $\forall\ a \in A,\ b \in B$, $f(a) = b \iff f^{-1}(b) = a $
          \end{itemize}
          \bigbreak \noindent 
          \nt{$\mathcal{D}$ and $\mathcal{R}$ flip for the inverse function}

      \end{itemize}



      \pagebreak \bigbreak \noindent 
      \section{\LARGE Notation}
      \bigbreak \noindent 
            \begin{itemize}
        \item  \textbf{Domain of a function}: Denoted $\mathcal{D}$ or $\mathcal{D}(f)$
        \item \textbf{Range of a function}: Denoted $\mathcal{R}$ or $\mathcal{R}(f)$
          Consider we have some function with $\mathcal{D}(f) = \mathbb{R}$ and $\mathcal{R}(f) = (2,\infty)$, then we can say 
          \begin{align*}
            f:\ \mathbb{R} \rightarrow (2,\infty):\ x \mapsto f(x) \\
            Or:\ f(x) \in (2,\infty),\ \forall\ x \in \mathbb{R}
          .\end{align*}
        \item \textbf{Functional Notation (Set-Builder)}
          \begin{align*}
            f:\ A \rightarrow B: x \mapsto f(x) \\
          .\end{align*}
            Where $A \rightarrow B $ is used to indicate the domain and codomain of the function, and  $x\mapsto f(x)$ is used to indicate how individual elements are mapped under the function. 
          \begin{align*}
              Ex:\ f:\ \mathbb{R} \rightarrow \mathbb{R}:\ x \mapsto x^{2}-6
          .\end{align*}
        \item \textbf{Exclude elements in functional notation}
          \begin{align*}
            f:\ \mathbb{R} \setminus \{2\} \mapsto \mathbb{R}:\ x \mapsto \frac{x+3}{x-2}
          .\end{align*}
        \item \textbf{Injective (one-to-one)}: 
          \begin{align*}
            \forall\ x_{1},\ x_{2} \in A, (f(x_{1}) = f(x_{2} \implies x_{1} = x_{2}))
          .\end{align*}
        \item \textbf{Subjective}
          \begin{align*}
            f:\ X \rightarrow Y\ onto\ \iff \forall y \in Y,\ \exists\ x \in X\ |\ f(x) = y
          .\end{align*}
      \end{itemize}




    
\end{document}
