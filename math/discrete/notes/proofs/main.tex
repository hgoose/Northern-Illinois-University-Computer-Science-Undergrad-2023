\documentclass{report}

\input{~/dev/latex/template/preamble.tex}
\input{~/dev/latex/template/macros.tex}

\title{\Huge{}}
\author{\huge{Nathan Warner}}
\date{\huge{}}
\pagestyle{fancy}
\fancyhf{}
\lhead{Warner \thepage}
\rhead{}
% \lhead{\leftmark}
\cfoot{\thepage}
% \setborder
% \usepackage[default]{sourcecodepro}
% \usepackage[T1]{fontenc}
\usepackage{lipsum}

\begin{document}
    % \maketitle
        \begin{titlepage}
       \begin{center}
           \vspace*{1cm}
    
           \textbf{Discrete Structures} \\
           Introduction to Proofs
    
           \vspace{0.5cm}
            
                
           \vspace{1.5cm}
            A Document By: \\ 
           \textbf{Nathan Warner}
    
           \vfill
                
                
           \vspace{0.8cm}
         
           \includegraphics[width=0.4\textwidth]{~/niu/seal.png}
                
           Computer Science \\
           Northern Illinois University\\
           August 16, 2023 \\
           United States\\
           
                
       \end{center}
    \end{titlepage}
    \tableofcontents
    \pagebreak \bigbreak \noindent
    \vspace{2in} \\
    \begin{Huge}
        \textbf{Proofs}
    \end{Huge}
    \bigbreak \noindent 
    \line(1,0){490}
    \bigbreak \noindent 
    \section{\LARGE Terminology}
    \bigbreak \noindent 
    \begin{itemize}
        \item \textbf{Conjecture}: A mathematical statement that has not yet been rigorously proved but is being proposed as being true.
        \item \textbf{Theorem}: Is a statement that can be shown to be true, or has been shown to be true.
        \item \textbf{Axioms (or Postulates)}: Is a statement that is taken to be true, to serve as a premise or starting point for further reasoning and arguments.
        \item \textbf{Lemma}: Is a less important theorem that is helpful in the proof of theorems.
        \item \textbf{Corollary}: Is a theorem that can be established directly from a theorem that has been proven.
    \end{itemize}
    \bigbreak \noindent \bigbreak \noindent 
    \section{\LARGE Direct Proof}
    \bigbreak \noindent 
    \begin{definition}
        \textbf{Definition.} A \textbf{direct proof} is a way of showing the truth or falsehood of a given statement by a straightforward combination of established facts, usually axioms, existing lemmas and theorems, without making any further assumptions.
    \end{definition}
    \bigbreak \noindent 
    Let's say we have the statement: \textit{If $n$ is odd number than $n^{2}$ is an odd number}
    \pf{Proof}{
    Let's assume that $n$ is an odd number, which means that it can be expressed as $n = 2k + 1$ for some integer $k$. This is because odd numbers are of the form $2k + 1$ where $k$ is an integer.
    \bigbreak \noindent 
    Now, let's square $n$:
    \begin{align*}
    n^2 &= (2k + 1)^2 \\
    &= 4k^2 + 4k + 1 \\
    &= 2(2k^2 + 2k) + 1
    \end{align*}
    \bigbreak \noindent 
    As we can see from the expression $2(2k^2 + 2k) + 1$, the squared value $n^2$ is expressed as an even number ($2$ times an integer) plus $1$. Since an odd number can always be represented as $2k + 1$, where $k$ is an integer, the expression $2(2k^2 + 2k) + 1$ follows the same pattern and is also an odd number.
    \bigbreak \noindent 
    Thus, we have shown that if $n$ is an odd number, then $n^2$ is indeed an odd number.
}
    \pagebreak \bigbreak \noindent 
    Now let's say we have the statement: \textit{If $n$ is even then $(-1)^{n} =1 $}
    \bigbreak \noindent 
    \pf{Proof}{
    Let's assume that $n$ is an even number, which means that it can be expressed as $n = 2k$ for some integer $k$. This is because even numbers are of the form $2k$ where $k$ is an integer.
    
    \bigbreak \noindent 
    Now, let's consider $(-1)^{2k}$:
    \begin{align*}
    (-1)^{2k} &= ((-1)^2)^k \\
    &= 1^k \\
    &= 1
    \end{align*}
    \ep
\bigbreak \noindent 
    Since any non-negative integer exponent of $1$ is always $1$, the expression $(-1)^{2k}$ simplifies to $1$.
    \bigbreak \noindent 
    Therefore, we have shown that if $n$ is an even number, then $(-1)^2 = 1$ holds true.
    \bigbreak \noindent 
    This completes the proof.
    \bigbreak \noindent 
}
    \bigbreak \noindent 
    For the next example, let's consider the following statement: \textit{if $a|b$ and $a|c$, then $a|(b+c),\ \quad a,b,c \in \mathbb{Z}$}
    \pf{Proof}{
Assume that $a|b$ and $a|c$. This means there exist integers $r$ and $t$ such that:
\begin{align*}
    b &= a \cdot r, \quad \text{(by definition of divisibility)} \\
    c &= a \cdot t. \quad \text{(by definition of divisibility)}
\end{align*}

We want to show that $a|(b+c)$. This means there exists an integer $s$ such that:
\begin{align*}
    b+c &= a \cdot s. \quad \text{(by definition of divisibility)}
\end{align*}

Adding the equations for $b$ and $c$, we get:
\begin{align*}
    b+c &= a \cdot r + a \cdot t \\
    &= a \cdot (r+t).
\end{align*}

Since $r$ and $t$ are integers, $r+t$ is also an integer. Therefore, we have shown that $b+c = a \cdot (r+t)$, which implies $a|(b+c)$. Thus, we have proved the statement.

\ep}

    \pagebreak \bigbreak \noindent 
    \section{\LARGE Proofs by Contrapositive}
    \bigbreak \noindent 
    Recall contrapostive, if $p \rightarrow q $, then the contrapostive is $\neg q \rightarrow \neg p$. Recall that these two statements are \textit{logically equivalent}
    \bigbreak \noindent 
\begin{definition}
    \textbf{Definition.} In mathematics, proof by contrapositive, or proof by contraposition, is a rule of inference used in proofs, where one infers a conditional statement from its contrapositive. In other words, the conclusion "if $A$, then $B$" is inferred by constructing a proof of the claim "if not $B$, then not $A$" instead. More often than not, this approach is preferred if the contrapositive is easier to prove than the original conditional statement itself.
\end{definition}
    \bigbreak \noindent 
    Consider the statement: \textit{$n \in \mathbb{Z},\ if\ n^{2}\ \text{is odd, then $n$ is odd}$}
    \bigbreak \noindent 
    First, let's try to prove this directly. To show that this approach is futile.
    \bigbreak \noindent 
    \pf{Proof}{
    Suppose $n^2$ is odd. Then, we can express it as $n^2 = 2k + 1$, where $k$ is an integer.
    \begin{align*}
        n^2 &= 2k + 1, \quad k \in \mathbb{Z}.
    \end{align*}

    Our goal is to prove that $n$ is also odd, implying that $n$ can be written as $n = 2k + 1$, where $k$ is an integer. Let's attempt to find a direct expression for $n$:
    \begin{align*}
        n &= \sqrt{2k + 1}.
    \end{align*}

    However, this doesn't provide any information about the parity of $n$. Therefore, a direct proof is not yielding the desired result. In such cases, we often resort to a proof by contrapositive, which can be more effective in establishing the statement.
}
    \bigbreak \noindent 
    Before we begin our proof by contrapositive, let's clarify what the contrapositive is for our statement: 
    \begin{center}
        Statement: If $n^{2}$ is odd, then $n$ is odd. \\
        Contrapositive: if $n$ is even, then $n^{2}$ is even
    \end{center}
    \bigbreak \noindent 
    \pf{Proof}{
    Suppose $n$ is even. Then, we can express it as $n = 2k$, where $k$ is an integer.
    \begin{align*}
        n  =2k, \quad k \in \mathbb{Z}
    .\end{align*}

    We want to show that $n^{2}$ is also even, implying that $n^{2} = 2k+1$, where $k$ is an integer. If we square both sides of our statement $n = 2k+1$
    \begin{align*}
        n^{2} (2k)^{2} \\
        n^{2} = 4k^{2} \\
        n^{2} = 2(2k^{2})
    .\end{align*}

    Since we know that if $k$ is an integer, then $k^{2}$ must also be an integer, we have shown that the parity of $n^{2}$ is indeed even if $n$ is even.
    \bigbreak \noindent 

    Therefore, by proving the contrapositive statement, we have established the original statement: If $n^2$ is odd, then $n$ is odd.
    \bigbreak \noindent 

    \ep
    }

    \pagebreak \bigbreak \noindent 
    Let's consider another example: \textit{$\forall$ positive real numbers, $n\cdot m > 100$, then $n>10$ or $m>10$}
    \bigbreak \noindent 
    So we have:
    \begin{center}
        Statement:  $\forall$ positive real numbers, if $n\cdot m > 100$, then $n>10$ or $m>10$ \\
        Contrapostive: $\forall$ positive real numbers, if $n \leq 10$ and $m \leq 10 $ then $n\cdot m \leq 100$
    \end{center}
    \bigbreak \noindent 
    \pf{Proof}{
    So suppose $n \leq 10$ and $m \leq 10$, we want to show that $nm \leq 100$.
    \bigbreak \noindent 
    
    If:
    \begin{align*}
      &n \leq 10 \\
      &nm \leq 10m\ \quad \text{(Multiplying both sides by m)}
    .\end{align*}
    \bigbreak \noindent 

    And:
    \begin{align*}
        &m \leq 10 \\
        &10m \leq 100\quad \text{(Multiplying both sides by 10)}
    .\end{align*}

    Thus, it follows that:
    \begin{align*}
       nm \leq 100 
    .\end{align*}
    \bigbreak \noindent 

    Therefore, we have shown that if $n \leq 10$ and $m \leq 10$, then $nm$ must be $ \leq 100$
    \bigbreak \noindent 
    \ep
    }

    \pagebreak \bigbreak \noindent 
    \section{\LARGE Proof by Contradiction} 
    \label{proofbycontradiction}
    % \textbf{Definition 1.} \textbf{Proof by Contradiction} is a form of proof that establishes the truth or the validity of a proposition, by showing that assuming the proposition to be false leads to a contradiction
    % \dfn{
    \smallbreak \noindent
    \begin{definition}
        \textbf{Proof by Contradiction} is a form of proof that establishes the truth or the validity of a proposition, by showing that assuming the proposition to be false leads to a contradiction
    \end{definition}

    \bigbreak \noindent 
    \begin{remark}
       There are infinitely many primes
    \end{remark}
    \bigbreak \noindent 
    \begin{proof}
       To prove by contradiction, let's assume that there exists a \textit{finite} number of primes  
       \bigbreak \noindent 

       If we denote the primes 
       \begin{align*}
           p_{1}, p_{2}, p_{3},p_{4},...,p_{n}
       .\end{align*}
       \bigbreak \noindent 

       Now suppose we let some integer $m$ be the product of these primes. Then we add one to this product

       \begin{align*}
           m = p_{1} \cdot p_{2} \cdot p_{3} \cdot p_{4} \cdot ... \cdot p_{n} + 1
       .\end{align*}
       \bigbreak \noindent 

       By the fundamental theorem of arithmetic, this new integer $m$ must either be prime or composite. Let's explore both possibilities
       \bigbreak \noindent 

       Prime:
       \bigbreak \noindent 

       If $m$ were to be prime, this means that we have created a new prime number. In this case, since our assumption is that there is a finite number of primes it would imply that our assumption is false, and there are indeed not a finite number of primes.
       \bigbreak \noindent 

       Composite:
       \bigbreak \noindent 

       If $m $ were to be composite then the prime factors of $m $ would need to be able to divide $m$, however, since we added one to $m$, we know that these prime factors will not be divisors of $m$. Thus, $m$ cannot be composite 

       \bigbreak \noindent 

       Thus, since $m$ cannot be composite, by the fundamental theorem of arithmetic, $m$ must be prime. This imply that there are infinitely many primes
       \bigbreak \noindent 
       \ep
    \end{proof}


    \pagebreak \bigbreak \noindent 
\begin{remark}
    $\sqrt{2}$ is irrational.
\end{remark}
\bigbreak \noindent 
\begin{proof}
   For the sake of contradiction, let's assume that $\sqrt{2}$ is \textit{rational}. If we assume that $\sqrt{2}$ is rational, then it can be expressed as:
   \begin{align*}
       \frac{a}{b}, \quad a, b \in \mathbb{Z}, \quad b \neq 0 \text{ and } \text{GCF}(a, b) = 1.
   \end{align*}
   \bigbreak \noindent 

   Lemma 1: An even integer multiplied by an even integer yields an even integer.
   \begin{align*}
       Lemma\ 1:\ \text{Show $(2k)^{2}$ is even for $k\in \mathbb{Z}$} \\
       (2k)^{2} \\
       = 4k^{2} \\
       = 2(2k^{2})
   .\end{align*}
   \bigbreak \noindent 
    
   Since $k \in \mathbb{Z}$, then $2k^{2}$ must be an integer. Which means we have the form $2k$. 
   \begin{align*}
       &\sqrt{2} = \frac{a}{b} \quad \text{(by definition of rational numbers)}\\
       &2 = \left(\frac{a}{b}\right)^{2} \quad \text{(squaring both sides of the equation, maintaining equality)} \\
       &2 = \frac{a^{2}}{b^{2}} \quad \text{(exponentiation property of fractions)}\\
       &2b^{2} = a^{2} \quad \text{(cross multiplication property)} \\
   .\end{align*}
   \bigbreak \noindent 

   If $2b^{2}$ is an even integer (by definition of an even integer), then $a^{2}$ must also be an integer. Thus, $a$ and $b$ must also be even integers:
   \begin{align*}
        &\therefore a = 2k \text{ and } b = 2l \text{ for some integers } k, l.
   \end{align*}
   \bigbreak \noindent 

   Since $\frac{2k}{2l}$ has a GCF of 2, this implies that our statement: \textit{$\sqrt{2}$ is rational} is false, which demonstrates a contradiction. Therefore, $\sqrt{2}$ must be irrational.
   \bigbreak \noindent 
   \ep
\end{proof}

    \pagebreak \bigbreak \noindent 
    \section{\LARGE Proof by Exhaustion (Proof by cases)}
    \smallbreak \noindent
    \begin{definition}
        \textbf{Proof by Exhaustion} the proof that something is true by showing that it is true for each and every case that could possibly be considered.
    \end{definition}
    \bigbreak \noindent 
    \begin{remark}
        $(n+1)^{3} \geq 3^{n}$, for $n \in \mathbb{N}$ and $n \leq 4 $
    \end{remark}
    \bigbreak \noindent 
    \begin{proof}
       To show that $(n+1)^{3} \geq 3^{n}$, for $n \in \mathbb{N}$ and $n \leq 4 $, we must show that this is true for all possible cases of $n$ 
       \begin{align*}
           \text{for $n$ $\in$ } (1,2,3,4)
       .\end{align*}
       \bigbreak \noindent 

       Case 1: $n = 1$
    \begin{align*}
        (1+1)^3 = 8 \geq 3^1 \quad \text{because } 8 \geq 3.
    \end{align*}

    Case 2: $n = 2$
    \begin{align*}
        (2+1)^3 = 27 \geq 3^2 \quad \text{because } 27 \geq 9.
    \end{align*}

    Case 3: $n = 3$
    \begin{align*}
        (3+1)^3 = 64 \geq 3^3 \quad \text{because } 64 \geq 27.
    \end{align*}

    Case 4: $n = 4$
    \begin{align*}
        (4+1)^3 = 125 \geq 3^4 \quad \text{because } 125 \geq 81.
    \end{align*}
       \bigbreak \noindent 

       Thus, we have shown that  $(n+1)^{3} \geq 3^{n}$, for $n \in \mathbb{N}$ and $n \leq 4 $ for every possible case of $n$
       \bigbreak \noindent 
       \ep
    \end{proof}
    \pagebreak \bigbreak \noindent 
    \begin{remark}
    For any positive integer \(x\) that is a perfect cube (\(x = n^3\) for some positive integer \(n\)), one of the following conditions holds:
    \begin{enumerate}
        \item \(x\) is a multiple of 9 (\(x = 9k\) for some positive integer \(k\)).
        \item \(x\) is one less than a multiple of 9 (\(x = 9k - 1\) for some positive integer \(k\)).
        \item \(x\) is one more than a multiple of 9 (\(x = 9k + 1\) for some positive integer \(k\)).
    \end{enumerate}
\end{remark}
\bigbreak \noindent 
\begin{proof}
    To show that this statement holds \(\forall x\ |\ x = n^{3},\ n \in \mathbb{Z}^{+}\) we will show that all three cases lead to a true statement. 
    \bigbreak \noindent 

    First, consider any positive integer \(n\). Since every integer can be written in the form \(9p\), \(9p-1\), or \(9p+1\) for some integer \(p\) (because the remainder when dividing by 9 must be 0, 1, or -1), we will prove the three cases.
    \bigbreak \noindent 

    Case 1 \(n=9p\):
    \begin{align*}
        (9p)^{3}  &= 729p^{3} & \text{Note: } 729p^3 = 9(81p^3) \text{ is a multiple of 9}
    \end{align*}
    \bigbreak \noindent 

    Case 2 \(n=9p-1\):
    \begin{align*}
        (9p-1)^{3}  &= 729p^{3}-243p^{2}+27p-1 & \text{Note: } 729p^3-243p^{2}+27p = 9(81p^3-27p^2+3p)
    \end{align*}
    Therefore, \((9p-1)^{3} = 9(81p^3-27p^2+3p)-1\) is one less than a multiple of 9.
    \bigbreak \noindent  

    Case 3 \(n=9p+1\):
    \begin{align*}
        (9p+1)^{3}  &= 729p^{3}+243p^{2}+27p+1 & \text{Note: } 729p^3+243p^{2}+27p = 9(81p^3+27p^2+3p)
    \end{align*}
    Therefore, \((9p+1)^{3} = 9(81p^3+27p^2+3p)+1\) is one more than a multiple of 9.
    \bigbreak \noindent 

    Therefore, it is apparent that for any positive integer \(x\) that is a perfect cube (\(x = n^3\) for some positive integer \(n\)), one of the following conditions holds:
    \begin{enumerate}
        \item \(x\) is a multiple of 9 (\(x = 9k\) for some positive integer \(k\)).
        \item \(x\) is one less than a multiple of 9 (\(x = 9k - 1\) for some positive integer \(k\)).
        \item \(x\) is one more than a multiple of 9 (\(x = 9k + 1\) for some positive integer \(k\)).
    \end{enumerate}
    \ep
\end{proof}

    \pagebreak \bigbreak \noindent    
    \section{\LARGE Proof by Existence}
    \bigbreak \noindent 
    \smallbreak \noindent
    \begin{definition}
    \textbf{A proof by existence} is a proof that establishes the existence of an element with a certain desired property. 
\end{definition}
\bigbreak \noindent 
\begin{remark}
    There exists a prime number \(p\) such that both \(p+2\) and \(p+6\) are prime numbers.
\end{remark}
\bigbreak \noindent 
\begin{proof}
    We can show that, \(\exists\ p \in \mathbb{P}\ |\ p+2, p+6 \in \mathbb{P}\), by use of numerical methods. First, let's define a subset of \(\mathbb{P}\), denoted as \(S\), as follows:
    \begin{align*}
        S = \{2,3,5,7,11,13\}.
    \end{align*}
    \bigbreak \noindent 

    By iterating through \(S\), we observe that when \(p=5\), we have \(p+2 = 7\) and \(p+6 = 11\), where both \(7\) and \(11\) are elements of \(S\) and, by definition, prime numbers.
    \bigbreak \noindent 

    Thus, we have shown that there exists a prime number \(p\) such that both \(p+2\) and \(p+6\) are prime numbers.
    \bigbreak \noindent 
    \ep
\end{proof}

    \bigbreak \noindent 
    \begin{remark}
       $\forall x \in \mathbb{Z}\ 6x = 2k,\ k\in \mathbb{Z} \rightarrow x = 2l,\ l \in \mathbb{Z}$ 
    \end{remark}
    \bigbreak \noindent 
    \begin{proof}
       We can show that $\forall x \in \mathbb{Z}\ 6x = 2k,\ k\in \mathbb{Z} \rightarrow x = 2l,\ l \in \mathbb{Z}$ finding a number $x$ such that if $x$ is not even when $6x$ is even.
       \bigbreak \noindent 

       We can start by defining a few test cases, $S$

       \begin{align*}
           S = \{-3,-2,-1,0,1,2,3,4,5\}
       .\end{align*}
       \bigbreak \noindent 

       Iterating through $S$ we can see that when $x=5$, $x$ is not even when $6x$ is even
       \begin{align*}
           6(5) = 30,\quad where\ 30 = 2k\ k \in \mathbb{Z}
           x = 5,\quad where\ 5 = 2k+1\ k \in \mathbb{Z} \\
       .\end{align*}
       \bigbreak \noindent 

       Thus, we have show that for all integers $x$, if $6x$ is even then $x$ may not be even. Therefore, the remark $\forall x \in \mathbb{Z}\ 6x = 2k,\ k\in \mathbb{Z} \rightarrow x = 2l,\ l \in \mathbb{Z}$ is not a true statement for all integers $x$.
       \bigbreak \noindent 
       \ep
    \end{proof}

    \pagebreak \bigbreak \noindent 
    \begin{prop}
        $\exists\ x,y \in \mathbb{\bar{Q}}\ |\ x^{y} \in \mathbb{Q}$ 
    \end{prop}
    \bigbreak \noindent 
    \begin{proof}
        To show that $\exists\ x,y \in \mathbb{\bar{Q}}\ |\ x^{y} \in \mathbb{Q}$, let's assume that $\sqrt{2}$ is irrational, then denote $x=\sqrt{2}$ and $y=\sqrt{2}$. If we want to show that $x^{y}$ is a rational number when $x$ and $y $ are irrational, then we must compute:
        \begin{align*}
            (\sqrt{2})^{\sqrt{2}}  \\
        .\end{align*}
        \bigbreak \noindent 

        Since this number is clearly still irrational, lets consider when $x=(\sqrt{2})^{\sqrt{2}}$ and when $y=\sqrt{2}$. Then:
        \begin{align*}
            x^{y} &= ((\sqrt{2})^{\sqrt{2}})^{\sqrt{2}} & \text{(Original expression)} \\
            &= ((\sqrt{2})^{2^{\frac{1}{2}}})^{2^{\frac{1}{2}}} & \text{(Simplify } \sqrt{2} \text{ to } 2^{\frac{1}{2}}) \\
            &= (\sqrt{2})^{4^{\frac{1}{2}}} & \text{(Power of a power rule, } (a^{b})^{c} = a^{bc}) \\
            &= \sqrt{2}^{\sqrt{4}} & \text{(Simplify } 4^{\frac{1}{2}} \text{ to } \sqrt{4}) \\
            &= \sqrt{2}^{2} & \text{(Simplify } \sqrt{4} \text{ to } 2) \\
            &= 2 & \text{(Power rule, } (\sqrt{2})^{2} = 2)
        \end{align*}
        \bigbreak \noindent 

        \bigbreak \noindent 

        Thus, when $x = \sqrt{2}^{\sqrt{2}}$ and $y=\sqrt{2}$, $x^{y}=2$, where 2 is rational. 
        \bigbreak \noindent 

        Therefore, we have shown that when $x=(\sqrt{2})^{\sqrt{2}}$ and $y=\sqrt{2}$, where $x$ and $y$ are irrational, then $x^{y}$ is rational. Proving the proposition $\exists\ x,y \in \mathbb{\bar{Q}}\ |\ x^{y} \in \mathbb{Q}$ non-constructively
        \bigbreak \noindent 
        \ep


    \end{proof}
    
    \pagebreak \bigbreak \noindent 
    \section{\LARGE Proof by Uniqueness}
    \bigbreak \noindent 
    \smallbreak \noindent
    \begin{definition}
     To \textbf{Prove by Uniqueness} is to show that some element has some desired property, and there is only one instance.
     \begin{enumerate}
         \item Show that there exists an $x$ with some desired property
        \item Show that $y\ne x$, then $y $ does not have a desired property
     \end{enumerate}
    \end{definition}

    % \bigbreak \noindent 
    % \begin{prop}
    %    $a,b \in \mathbb{R}, a \ne 0 \rightarrow\ \exists!\ r\ |\ a \cdot r + b =0$ 
    % \end{prop}
    % \bigbreak \noindent 
    % \begin{proof}
    %    To show if $a,b \in \mathbb{R}$, where $a\ne 0$ then there exists only one unique $r$  such that $a\cdot r + b = 0$, we must first isolate $r$:
    %    \begin{align*}
    %        &ar + b = 0\ \quad \text{(Original)}\\
    %        &ar = -b\ \quad \text{(Subtraction property of equality)} \\
    %        &r  = \frac{-b}{a}\ \quad \text{(Division property of equality)} 
    %    .\end{align*}
    %
    %    \bigbreak \noindent 
    %
    %    Since $r$ \textit{must} be unique, then we must show that some other number $s$, where $as + b = 0 $ does not yield a new equation.
    %    \begin{align*}
    %        as + b = 0 \quad ar+b = 0 \\
    %        as +b = ar+b \\
    %        as = ar \\
    %        s = r \\
    %    .\end{align*}
    %
    %    \bigbreak \noindent 
    %    Thus, since $s=r$, it must hold that for some number $r$, $ar+b  = 0$. Where $r$ is unique.
    % \end{proof}

    \bigbreak \noindent 
    \begin{prop}
Given $a, b \in \mathbb{R}$ and $a \ne 0$, there exists a unique $r \in \mathbb{R}$ such that $a \cdot r + b = 0$.
\end{prop}

    \begin{proof}
        To prove existence, we will show that there is at least one solution $r$ that satisfies $a \cdot r + b = 0$. Assuming $a\ne0$, We can solve for $r$ by isolating it in the equation:


    \begin{align*}
    a \cdot r + b &= 0 \quad &\text{(Original equation)} \\
    a \cdot r &= -b \quad &\text{(Subtracting $b$ from both sides)} \\
    r &= \frac{-b}{a} \quad &\text{(Dividing both sides by $a$)}
    \end{align*}
    \bigbreak \noindent 

    Since $a \ne 0$, the division is well-defined. Therefore, there exists at least one solution $r = \frac{-b}{a}$ that satisfies the equation.
    \bigbreak \noindent 

    To prove uniqueness, we will assume that there is another solution $s \ne r$ such that $a \cdot s + b = 0$. By substituting the values of $r$ and $s$ from the original equation, we get:
    \begin{align*}
        as + b = 0 \quad &\quad ar + b = 0\\
    a \cdot s + b &= a \cdot r + b \\
    a \cdot s &= a \cdot r \\
    s &= r
    \end{align*}
    \bigbreak \noindent 

    Here, we reach a contradiction as we initially assumed that $s \ne r$, but through the proof, we derived that $s = r$. Therefore, the solution $r$ is unique.
    \bigbreak \noindent 

    In conclusion, given $a, b \in \mathbb{R}$ with $a \ne 0$, there exists a unique solution $r = \frac{-b}{a}$ such that $a \cdot r + b = 0$.
    \end{proof}

    \pagebreak \bigbreak \noindent 
    \section{\LARGE Proof by Induction}
    \bigbreak \noindent 
    \smallbreak \noindent
    \begin{definition}
    To \textbf{Prove by Induction} is to prove that for every n, if the statement holds for n, then it holds for n + 1 
    \begin{enumerate}
        \item Basis Step: Show that $P(a)$ is true
        \item Inductive Step: Show that for all integers $k \geq a$, if $P(k)$ is true then $P(k+1)$ is true
    \end{enumerate}
    \end{definition}

    \bigbreak \noindent 
    \begin{prop}
       $1+2+3+...+n = \frac{n(n+1)}{2}$, for $n \in \mathbb{Z}$, n \geq 1 
    \end{prop}
    \bigbreak \noindent 
    \begin{proof}
       To show that $\summation{n}{i=1}i \  = \frac{n(n+1)}{2}$, for $n \in \mathbb{Z}, n \geq 1$, we must first show that the basis step, when $n=1$  that $\summation{1}{i=1}\ i = \frac{1(1+1)}{2} \ $
       \begin{align*}
           1 &= \frac{1(2)}{2} \\
           1 &= \frac{2}{2} \\
           1 &= 1
       .\end{align*}

       \bigbreak \noindent 

       Assume $n=k$ holds (inductive hypothesis): 
       \begin{align*}
           1+2+3+...+k = \frac{k(k+1)}{2} \\
       .\end{align*}
       \bigbreak \noindent 

       Show $n=k+1$ holds:

       \begin{align*}
           1+2+3+...+k+k+1 = \frac{k+1(k+1+1)}{2} \\
           = 1+2+3+...+k+k+1 = \frac{k+1(k+2)}{2} \\
       .\end{align*}
       \bigbreak \noindent 

       By the inductive hypothesis, if $1+2+3+...+k = \frac{k(k+1)}{2}$, then it holds that:
       \begin{align*}
           \frac{k(k+1)}{2} + k + 1 = \frac{k+1(k+2)}{2} \\
           \frac{k(k+1)}{2} + k + 1 = \frac{k^{2}+3k+2}{2} \\
            \frac{k(k+1)}{2} + \frac{2(k+1)}{2} = \frac{k^{2}+3k+2}{2} \\
            \frac{k^{2}+k+2k+1}{2} = \frac{k^{2}+3k+2}{2} \\
            \therefore \frac{k^{2}+3k+1}{2} = \frac{k^{2}+3k+2}{2} \\
       .\end{align*}
       \bigbreak \noindent 
       \ep

    \end{proof}

    \pagebreak \bigbreak \noindent  
    \begin{prop}
       $3$ is a factor of $4^{n} + 2$ 
    \end{prop}
    \bigbreak \noindent 
    \begin{proof}
        To show that 3 is a factor of $4^{n} + 2$, we must show that $4^{n} + 2 = 3r,\ r \in \mathbb{Z}$ for $P_{1},\ P_{k},\ and\ P_{k+1}$
        \bigbreak \noindent 

        Anchor: $P_{1} = 4^{1} + 2 = 6$, since $6 = 3(2)$, $P_{1}$ holds
        \bigbreak \noindent 

        Inductive hypothesis: Assume $P_{k}$ holds, this implies that $4^{k} + 2 = 3r$, for some integer $r$
        \bigbreak \noindent 

        Inductive step: Prove $P_{k+1}$ holds

        \begin{align*}
            &P_{k+1} = 4^{k+1} + 2 \\
            &= 4\cdot 4k+2\ \quad \text{(by product of powers property, $x^{n+m} = x^{n} \cdot x^{m}$)} \\
            &=(3+1)\cdot 4^{k}  +2  \\
            &=3\cdot 4^{k} +4^{k}  + 2
        .\end{align*}
        \bigbreak \noindent 

        By the inductive hypothesis, $4^{k} + 2 = 3r$. Thus, it follows that:

        \begin{align*}
            3\cdot 4^{k} + 3r \\
            = 3(4^{k} + r)
        .\end{align*}
        \bigbreak \noindent 

        Therefore, we have shown that $P_{k+1}$ has a factor of 3. Thus proving this statement by induction.
        \bigbreak \noindent \ep


    \end{proof}
    
    
    
        
    



    
    
    


    
    
    
    
    
    
    
    
    



    
    

    

    










    
    
\end{document}
