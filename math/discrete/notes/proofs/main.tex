\documentclass{report}

\input{~/dev/latex/template/preamble.tex}
\input{~/dev/latex/template/macros.tex}

\title{\Huge{}}
\author{\huge{Nathan Warner}}
\date{\huge{}}
\pagestyle{fancy}
\fancyhf{}
\lhead{Warner \thepage}
\rhead{}
% \lhead{\leftmark}
\cfoot{\thepage}
\setborder
% \usepackage[default]{sourcecodepro}
% \usepackage[T1]{fontenc}

\begin{document}
    % \maketitle
        \begin{titlepage}
       \begin{center}
           \vspace*{1cm}
    
           \textbf{Discrete Structures} \\
           Introduction to Proofs
    
           \vspace{0.5cm}
            
                
           \vspace{1.5cm}
            A Document By: \\ 
           \textbf{Nathan Warner}
    
           \vfill
                
                
           \vspace{0.8cm}
         
           \includegraphics[width=0.4\textwidth]{~/niu/seal.png}
                
           Computer Science \\
           Northern Illinois University\\
           August 16, 2023 \\
           United States\\
           
                
       \end{center}
    \end{titlepage}
    \tableofcontents
    \pagebreak \bigbreak \noindent
    \vspace{2in} \\
    \begin{Huge}
        \textbf{Proofs}
    \end{Huge}
    \bigbreak \noindent 
    \line(1,0){490}
    \bigbreak \noindent 
    \section{\LARGE Terminology}
    \bigbreak \noindent 
    \begin{itemize}
        \item \textbf{Conjecture}: A mathematical statement that has not yet been rigorously proved but is being proposed as being true.
        \item \textbf{Theorem}: Is a statement that can be shown to be true, or has been shown to be true.
        \item \textbf{Axioms (or Postulates)}: Is a statement that is taken to be true, to serve as a premise or starting point for further reasoning and arguments.
        \item \textbf{Lemma}: Is a less important theorem that is helpful in the proof of theorems.
        \item \textbf{Corollary}: Is a theorem that can be established directly from a theorem that has been proven.
    \end{itemize}
    \bigbreak \noindent \bigbreak \noindent 
    \section{\LARGE Direct Proof}
    \bigbreak \noindent 
    \textbf{Definition.} A \textbf{direct proof} is a way of showing the truth or falsehood of a given statement by a straightforward combination of established facts, usually axioms, existing lemmas and theorems, without making any further assumptions.
    \bigbreak \noindent 
    Let's say we have the statement: \textit{If $n$ is odd number than $n^{2}$ is an odd number}
    \pf{Proof}{
    Let's assume that $n$ is an odd number, which means that it can be expressed as $n = 2k + 1$ for some integer $k$. This is because odd numbers are of the form $2k + 1$ where $k$ is an integer.
    \bigbreak \noindent 
    Now, let's square $n$:
    \begin{align*}
    n^2 &= (2k + 1)^2 \\
    &= 4k^2 + 4k + 1 \\
    &= 2(2k^2 + 2k) + 1
    \end{align*}
    \bigbreak \noindent 
    As we can see from the expression $2(2k^2 + 2k) + 1$, the squared value $n^2$ is expressed as an even number ($2$ times an integer) plus $1$. Since an odd number can always be represented as $2k + 1$, where $k$ is an integer, the expression $2(2k^2 + 2k) + 1$ follows the same pattern and is also an odd number.
    \bigbreak \noindent 
    Thus, we have shown that if $n$ is an odd number, then $n^2$ is indeed an odd number.
}
    \pagebreak \bigbreak \noindent 
    Now let's say we have the statement: \textit{If $n$ is even then $(-1)^{n} =1 $}
    \bigbreak \noindent 
    \pf{Proof}{
    Let's assume that $n$ is an even number, which means that it can be expressed as $n = 2k$ for some integer $k$. This is because even numbers are of the form $2k$ where $k$ is an integer.
    
    \bigbreak \noindent 
    Now, let's consider $(-1)^{2k}$:
    \begin{align*}
    (-1)^{2k} &= ((-1)^2)^k \\
    &= 1^k \\
    &= 1
    \end{align*}
    \ep
\bigbreak \noindent 
    Since any non-negative integer exponent of $1$ is always $1$, the expression $(-1)^{2k}$ simplifies to $1$.
    \bigbreak \noindent 
    Therefore, we have shown that if $n$ is an even number, then $(-1)^2 = 1$ holds true.
    \bigbreak \noindent 
    This completes the proof.
    \bigbreak \noindent 
}
    \bigbreak \noindent 
    For the next example, let's consider the following statement: \textit{if $a|b$ and $a|c$, then $a|(b+c),\ \quad a,b,c \in \mathbb{Z}$}
    \pf{Proof}{
Assume that $a|b$ and $a|c$. This means there exist integers $r$ and $t$ such that:
\begin{align*}
    b &= a \cdot r, \quad \text{(by definition of divisibility)} \\
    c &= a \cdot t. \quad \text{(by definition of divisibility)}
\end{align*}

We want to show that $a|(b+c)$. This means there exists an integer $s$ such that:
\begin{align*}
    b+c &= a \cdot s. \quad \text{(by definition of divisibility)}
\end{align*}

Adding the equations for $b$ and $c$, we get:
\begin{align*}
    b+c &= a \cdot r + a \cdot t \\
    &= a \cdot (r+t).
\end{align*}

Since $r$ and $t$ are integers, $r+t$ is also an integer. Therefore, we have shown that $b+c = a \cdot (r+t)$, which implies $a|(b+c)$. Thus, we have proved the statement.

\ep}

    \pagebreak \bigbreak \noindent 
    \section{\LARGE Proofs by Contrapositive}
    \bigbreak \noindent 
    Recall contrapostive, if $p \rightarrow q $, then the contrapostive is $\neg q \rightarrow \neg p$. Recall that these two statements are \textit{logically equivalent}
    \bigbreak \noindent 
    \textbf{Definition.} In mathematics, proof by contrapositive, or proof by contraposition, is a rule of inference used in proofs, where one infers a conditional statement from its contrapositive. In other words, the conclusion "if $A$, then $B$" is inferred by constructing a proof of the claim "if not $B$, then not $A$" instead. More often than not, this approach is preferred if the contrapositive is easier to prove than the original conditional statement itself.
    \bigbreak \noindent 
    Consider the statement: \textit{$n \in \mathbb{Z},\ if\ n^{2}\ \text{is odd, then $n$ is odd}$}
    \bigbreak \noindent 
    First, let's try to prove this directly. To show that this approach is futile.
    \bigbreak \noindent 
    \pf{Proof}{
    Suppose $n^2$ is odd. Then, we can express it as $n^2 = 2k + 1$, where $k$ is an integer.
    \begin{align*}
        n^2 &= 2k + 1, \quad k \in \mathbb{Z}.
    \end{align*}

    Our goal is to prove that $n$ is also odd, implying that $n$ can be written as $n = 2k + 1$, where $k$ is an integer. Let's attempt to find a direct expression for $n$:
    \begin{align*}
        n &= \sqrt{2k + 1}.
    \end{align*}

    However, this doesn't provide any information about the parity of $n$. Therefore, a direct proof is not yielding the desired result. In such cases, we often resort to a proof by contrapositive, which can be more effective in establishing the statement.
}
    \bigbreak \noindent 
    Before we begin our proof by contrapositive, let's clarify what the contrapositive is for our statement: 
    \begin{center}
        Statement: If $n^{2}$ is odd, then $n$ is odd. \\
        Contrapositive: if $n$ is even, then $n^{2}$ is even
    \end{center}
    \bigbreak \noindent 
    \pf{Proof}{
    Suppose $n$ is even. Then, we can express it as $n = 2k$, where $k$ is an integer.
    \begin{align*}
        n  =2k, \quad k \in \mathbb{Z}
    .\end{align*}

    We want to show that $n^{2}$ is also even, implying that $n^{2} = 2k+1$, where $k$ is an integer. If we square both sides of our statement $n = 2k+1$
    \begin{align*}
        n^{2} (2k)^{2} \\
        n^{2} = 4k^{2} \\
        n^{2} = 2(2k^{2})
    .\end{align*}

    Since we know that if $k$ is an integer, then $k^{2}$ must also be an integer, we have shown that the parity of $n^{2}$ is indeed even if $n$ is even.
    \bigbreak \noindent 

    Therefore, by proving the contrapositive statement, we have established the original statement: If $n^2$ is odd, then $n$ is odd.
    \bigbreak \noindent 

    \ep
    }

    \pagebreak \bigbreak \noindent 
    Let's consider another example: \textit{$\forall$ positive real numbers, $n\cdot m > 100$, then $n>10$ or $m>10$}
    \bigbreak \noindent 
    So we have:
    \begin{center}
        Statement:  $\forall$ positive real numbers, if $n\cdot m > 100$, then $n>10$ or $m>10$ \\
        Contrapostive: $\forall$ positive real numbers, if $n \leq 10$ and $m \leq 10 $ then $n\cdot m \leq 100$
    \end{center}
    \bigbreak \noindent 
    \pf{Proof}{
    So suppose $n \leq 10$ and $m \leq 10$, we want to show that $nm \leq 100$.
    \bigbreak \noindent 
    
    If:
    \begin{align*}
      &n \leq 10 \\
      &nm \leq 10m\ \quad \text{(Multiplying both sides by m)}
    .\end{align*}
    \bigbreak \noindent 

    And:
    \begin{align*}
        &m \leq 10 \\
        &10m \leq 100\quad \text{(Multiplying both sides by 10)}
    .\end{align*}

    Thus, it follows that:
    \begin{align*}
       nm \leq 100 
    .\end{align*}
    \bigbreak \noindent 

    Therefore, we have shown that if $n \leq 10$ and $m \leq 10$, then $nm$ must be $ \leq 100$
    \bigbreak \noindent 
    \ep
    }





    
    
\end{document}
