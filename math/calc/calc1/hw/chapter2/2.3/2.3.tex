\documentclass{report}

\input{~/dev/latex/template/preamble.tex}
\input{~/dev/latex/template/macros.tex}

\title{\Huge{2.3 Hw Solutions}}
\author{\huge{Nathan Warner}}
\date{\huge{Jan 26, 2023}}

\begin{document}
    \maketitle
    \begin{Large}
       \noindent \textbf{Question 1:} 
    \end{Large}
    
    \bigbreak \noindent \bigbreak \noindent 
    \textbf{a. We can solve this by using laws 1 and 3)}

    \bigbreak \noindent 
    \pf{Solution}{}
    \begin{align*}
         1 + 4 \cdot \left(-5\right) \\
         = -19
    .\end{align*}

    \bigbreak \noindent 
    \textbf{b.) We can solve this by using law no. 6} 
    
    \begin{align*}
        -5^3 \\ 
        = -125
    .\end{align*}

    \bigbreak \noindent 
    \textbf{c.) Law no. 11}

    \begin{align*}
        \sqrt{1} \\ 
        = 1 
    .\end{align*}

    \bigbreak \noindent 
    \textbf{d.) Laws 3 and 5}

    \begin{align*}
        \frac{5 \left(1\right)}{-5} \\ 
        = -1
    .\end{align*}

    \bigbreak \noindent 
    \textbf{e.) Law no. 5}

    \begin{align*}
        \frac{5}{0} \\
        = DNE
    .\end{align*}

    \bigbreak \noindent 
    \textbf{f.) Laws 4 and 5}

    \begin{align*}
        \frac{5 \left(0\right)}{1} \\ 
        = 0
    .\end{align*}

    \pagebreak
    \begin{Large}
       \noindent \textbf{Question 2:} 
    \end{Large}

    \bigbreak \noindent \bigbreak \noindent 
    \pf{Solution}{}
    \noindent Plug in 12 for x:

    \begin{align*}
        8 - \frac{1}{3} \left(12\right) \\
        =4
    .\end{align*}

    \bigbreak \noindent \bigbreak \noindent 
    \begin{Large}
       \textbf{Question 3:} 
    \end{Large}

    \bigbreak \noindent 
    \pf{Solution}{}
    \noindent If we plug in -3 into the denomonator, we get 0, so we must factor
     
    \begin{align*}
        \frac{x \left(x+3\right)}{ \left(x-7\right) \left(x+3\right)}
    .\end{align*}

    \noindent Cancel out common factors:
    \begin{align*}
        \frac{x}{x-7}
    .\end{align*}
    \noindent Plug -3 into new equation:

    \begin{align*}
       \frac{-3}{-3-7} \\
       = \frac{3}{10}
    .\end{align*}

    \bigbreak \noindent \bigbreak \noindent 
    \begin{Large}
       \textbf{Question 4:} 
    \end{Large}

    \bigbreak \noindent 
    \pf{Solution}{}
    
    \noindent
    2 is not in the domain and the numerator cannot be factored, So DNE

    \bigbreak \noindent \bigbreak \noindent 
    \begin{Large}
       \textbf{Question 5:} 
    \end{Large}

    \bigbreak \noindent 
    \pf{Solution}{}
    \noindent If we plug in -4 into the denomonator, we get 0, so we must factor the equation
    and simplify, we can factor the denomonator by using sum of squares.

    \bigbreak \noindent 
    \textbf{Sum of Squares:}
    \begin{align*}
        \left(a^3+b^3\right) = \left(a+b\right) \left(a^2-ab+b^2\right)
    .\end{align*}

    \bigbreak \noindent 
    \textbf{So the denomonator turns into:}
    \begin{align*}
        \left(u+4\right) \left(u^2-4u+16\right)
    .\end{align*}

    \bigbreak \noindent 
    \textbf{With this, we get the equation:}
    \begin{align*}
        \frac{u+4}{ \left(u+4\right) \left(u^2-4u+16\right)}
    .\end{align*}

    \bigbreak \noindent 
    \textbf{now we can cancel out (u+4) and get the new equation:} 

    \begin{align*}
        \frac{1}{u^2-4u+16}
    .\end{align*}

    \bigbreak \noindent 
    \textbf{with this equation, we can plug in -4 and get our limit:}

    \bigbreak \noindent 
    \textbf{So:}

    \begin{align*}
        \frac{1}{ \left(-4\right)^2-4 \left(-4\right)+16} \\ 
        = \frac{1}{48}
    .\end{align*}

    \bigbreak \noindent \bigbreak \noindent 
    \begin{Large}
       \textbf{Question 6:} 
    \end{Large}

    \bigbreak \noindent 
    \pf{Solution}{}
    \noindent We can see that if we plug in 0 to the denomonator, we get 0, Therefore it is
    not in the domain and we must factor and simplify.

    \bigbreak \noindent 
    If we distribute out the first portion of the numerator, we get
    \begin{align*}
       h^2-16h+64 - 64
    .\end{align*}
    
    \bigbreak \noindent 
    So alltogether we have:
    \begin{align*}
        \frac{h^2-16h}{h}
    .\end{align*}

    \bigbreak \noindent 
    Which can be further simplifed to:
    \begin{align*}
       \frac{h \left(h-16\right)}{h} 
    .\end{align*}
    \bigbreak \noindent 
    Furthermore, we can cancel out the common factor \textit{h} and we are left with:
    \begin{align*}
       h-16 
    .\end{align*}
    
    \bigbreak \noindent 
    Now we can plug in 0 to this new equation and we are just left with our limit which is \textbf{-16}

    \pagebreak
    \begin{Large}
       \noindent \textbf{Question 7:} 
    \end{Large}
    
    \bigbreak \noindent 
    \pf{Solution}{}
    
    \noindent First mulitply the numerator and denomonator by the conjugate.

    \begin{align*}
        \frac{2-x}{\sqrt{x+2} -2} \cdot \frac{\sqrt{x+2}+2}{\sqrt{x+2}+2}
    .\end{align*}

    \bigbreak \noindent 
    This gives us:
    \begin{align*}
        \frac{ \left(2-x\right) \left(\sqrt{x+2}+2\right)}{x+2-4}
    .\end{align*}

    \bigbreak \noindent 
    The denomonator can be rewriten as:
    \begin{align*}
        \frac{ \left(2-x\right) \left(\sqrt{x+2}+2\right)}{x-2}
    .\end{align*}

    \bigbreak \noindent 
    Now we want to cancel out the common terms of x-2, but first we need to
    \textbf{algebriacly rewrite 2-x as x-2} 

    \begin{align*}
        2-x \\ 
        = -1 \left(-2\right) - x \\
        \text{turn 2 into $-1 \cdot -2$} \\
        = -1 \left(-2\right) - \left(x\right) \\
        \text{factor -1 out of x}
        = -1 \left(-2 + x\right) \\
        \text{Factor -1 out of -1(-2)-(x)}
        = -1 \left(x-2\right) \\
        \text{reorder term}
    .\end{align*}

    \bigbreak \noindent 
    Now we have 
    \begin{align*}
        \frac{-1 \left(x-2\right) \left(\sqrt{x+2}+2\right)}{x-2}
    .\end{align*}

    \bigbreak \noindent 
    Cancel out the common factor x-2

    \begin{align*}
        -1 \left(\sqrt{x+2}+2\right)
    .\end{align*}

    \bigbreak \noindent 
    distribute the -1

    \begin{align*}
        -\sqrt{x+2} -2
    .\end{align*}

    \bigbreak \noindent 
    Now we can plug 2 into this equation and get:

    \begin{align*}
        -\sqrt{4} -2 \\ 
        = -4
    .\end{align*}

    \pagebreak
    \begin{Large}
       \noindent \textbf{Question 8:} 
    \end{Large}

    \bigbreak \noindent 
    \pf{Solution}{}
    First we must simplify the numerator, so first multiply to get a common denomonator on both 
    sides.

    \begin{align*}
        \frac{1}{ \left(x+4\right)^2} \cdot \frac{x^2}{x^2}
    .\end{align*}

    \bigbreak \noindent 
    And:
    \begin{align*}
        \frac{1}{x^2} \cdot \frac{ \left(x+h\right)^2}{ \left(x+h\right)^2}
    .\end{align*}

    \bigbreak \noindent 
    After this we get:

    \begin{align*}
        \frac{x^2}{x^2 \left(x+h\right)^2} - \frac{ \left(x+h\right)^2}{x^2 \left(x+h\right)^2}
    .\end{align*}

    \bigbreak \noindent 
    Now that they have common denomonators, we can subtract them and get:

    \begin{align*}
        \frac{x^2 - \left(x+h\right)^2}{x^2 \left(x+h\right)^2}
    .\end{align*}

    \bigbreak \noindent 
    Now we use the difference of squares formula to simplify the numerator
    \begin{align*}
        a^2-b^2 = \left(a+b\right) \left(a-b\right)
    .\end{align*}

    \bigbreak \noindent 
    Where $a = x^2$ and\ $b = (x+h)^2$

    \bigbreak \noindent 
    \textbf{So:}

    \begin{align*}
        \left(x+x+h\right) \left(x-\left(x+h\right)\right)
    .\end{align*}

    \bigbreak \noindent 
    simplify:
    \begin{align*}
        \left(2x+h\right) \left(x-x-h\right) \\
        = \left(2x+h\right) -h  \\
        = - \left(2x+h\right) h
    .\end{align*}

    \bigbreak \noindent 
    now that the numerator is simplifed our equation looks like:

    \begin{align*}
        - \frac{\frac{ \left(2x+h\right)h}{x^2 \left(x+h\right)^2}}{h}
    .\end{align*}

    \bigbreak \noindent 
    Now we can multiply by the reciprical, which cancels out the h on top and leaves us with:
    \begin{align*}
        -\frac{ \left(2x+h\right)}{x^2 \left(x+h\right)^2}
    .\end{align*}

    \bigbreak \noindent 
    and now if we plug in 0 for h we get 
    \begin{align*}
        - \frac{2}{x^3}
    .\end{align*}

    \bigbreak \noindent \bigbreak \noindent 
    \begin{Large}
       \textbf{Question 11:} 
    \end{Large}

    \bigbreak \noindent 
    \pf{Solution}{}
    \noindent  

\end{document}

