\documentclass{report}

\input{~/dev/latex/template/preamble.tex}
\input{~/dev/latex/template/macros.tex}

\title{\Huge{Chapter 5: Quiz 4}}
\author{\huge{Nathan Warner}}
\date{\huge{May 3 2023}}
\pagestyle{fancy}
\fancyhf{}
\lhead{Warner \thepage}
\rhead{QUIZ SOLUTIONS}
% \lhead{\leftmark}
\cfoot{\thepage}
% \usepackage[default]{sourcecodepro}
% \usepackage[T1]{fontenc}

\pgfpagesdeclarelayout{boxed}
{
  \edef\pgfpageoptionborder{0pt}
}
{
  \pgfpagesphysicalpageoptions
  {%
    logical pages=1,%
  }
  \pgfpageslogicalpageoptions{1}
  {
    border code=\pgfsetlinewidth{1.5pt}\pgfstroke,%
    border shrink=\pgfpageoptionborder,%
    resized width=.95\pgfphysicalwidth,%
    resized height=.95\pgfphysicalheight,%
    center=\pgfpoint{.5\pgfphysicalwidth}{.5\pgfphysicalheight}%
  }%
}

\pgfpagesuselayout{boxed}

\begin{document}
    \maketitle
    \bigbreak \noindent \bigbreak \noindent
    \begin{mdframed}
        \textbf{1.) Use midpoints with the given value of $n$ to approximate the integral}
        \begin{align*}
            \int_{0}^{4}\ (x-1)^{2}\ dx,\ \ n=4
        .\end{align*}
    \end{mdframed}
    \bigbreak \noindent \bigbreak \noindent
    Compute $\Delta x $:

    \bigbreak \noindent \bigbreak \noindent
    If:
    \begin{align*}
        \Delta x = \frac{b-a}{n}
    .\end{align*}
    Then:
\begin{align*}
        \Delta x = \frac{4-0}{4} \\
        = 1
    .\end{align*}
    \bigbreak \noindent \bigbreak \noindent
    Now that we have $\Delta x$, we can construct a numberline with right endpoints:
    \begin{figure}[ht]
        \centering
        \incfig{nline}
        \label{fig:nline}
    \end{figure}

    \bigbreak \noindent \bigbreak \noindent
    From here if we divide $\Delta x$ by 2 and add this number to each point we can construct the number line for our midpoints:
    \begin{figure}[ht]
        \centering
        \incfig{nline2}
        \label{fig:nline2}
    \end{figure}
        
    \bigbreak \noindent 
    Now by the Riemann sum, which states:
    \begin{align*}
        M_{4} = \summation{n}{i=1}\ \Delta x f(x_{i})\ 
    .\end{align*}
    \bigbreak \noindent 
    We have:
    \begin{align*}
        1\bigg(f\bigg(\frac{1}{2}\bigg) + f\bigg(\frac{3}{2}\bigg) + f\bigg(\frac{5}{2}\bigg) + f\bigg(\frac{7}{2}\bigg)\bigg) \\
        = \bigg(\bigg(\frac{1}{2}-1\bigg)^{2} + \bigg(\frac{3}{2} -1\bigg)^{2} + \bigg(\frac{5}{2}-1\bigg)^{2} + \bigg(\frac{7}{2}-1\bigg)^{2}\bigg) \\ 
        = \bigg(\frac{1}{4} + \frac{1}{4} +\frac{9}{4} + \frac{25}{4}\bigg) \\
        =\frac{36}{4} \\
        \boxed{=9}
    .\end{align*}

    \pagebreak \bigbreak \noindent
    \begin{mdframed}
        \textbf{2.) Given that}
        \begin{align*}
            \int_{0}^{\pi}\ \sin^{2}{x}\ dx, = \frac{\pi}{2}
        .\end{align*}
        Find:
        \begin{align*}
            \int_{0}^{\pi}\ (x+\sin^{2}{x})\ dx
        .\end{align*}
    \end{mdframed}
    \bigbreak \noindent \bigbreak \noindent
    Using the property of integrals, which states:
    \begin{align*}
        \int_{a}^{b}\ \big[f(x) + g(x)\big]\ dx = \int_{a}^{b}\ f(x)\ dx + \int_{a}^{b}\ g(x)\ dx
    .\end{align*}

    \bigbreak \noindent \bigbreak \noindent
    We can write the equation as:
    \begin{align*}
        \int_{0}^{\pi}\ x\ dx + \int_{0}^{\pi}\ (\sin^{2}{x})\ dx \\
        = \int_{0}^{\pi}\ x\ dx + \frac{\pi}{2}
    .\end{align*}

    \bigbreak \noindent \bigbreak \noindent
    And if we use the fundemental theorem of calculus to evaluate $\int_{0}^{\pi}\ x\ dx $, we get:
    \begin{align*}
        \frac{1}{2}x^{2}\bigg]_{0}^{\pi} \\
        = \bigg(\frac{1}{2}(\pi )^{2}\bigg) - \bigg(\frac{1}{2}(0)^{2}\bigg) \\
        = \frac{\pi^{2}}{2}
    .\end{align*}

    \bigbreak \noindent \bigbreak \noindent
    Therefore:
    \begin{align*}
      \int_{0}^{\pi}\ x\ dx + \int_{0}^{\pi}\ (\sin^{2}{x})\ dx \\
      \boxed{\frac{\pi^{2}}{2}+\frac{\pi}{2}}
    .\end{align*}

    \pagebreak \bigbreak \noindent
    \begin{mdframed}
      \textbf{3.) Given}
      \begin{align*}
        \int_{1}^{4}\ f(t)\ dt = -5\ and\ \int_{1}^{2}\ 2f(t)\ dt = -1
      .\end{align*}
      \bigbreak \noindent 
      Use the properties of integrals to compute
      \begin{align*}
        \int_{2}^{4}\ f(t)\ dt
      .\end{align*}
    \end{mdframed}

    \bigbreak \noindent \bigbreak \noindent
    To start, we can utilize the property:
    \begin{align*}
      \int_{a}^{b}\ cf(x)\ dx = c \cdot \int_{a}^{b}\ f(x)\ dx
    .\end{align*}
    \bigbreak \noindent \bigbreak \noindent
    To see that 
    \begin{align*}
      \int_{1}^{2}\ 2f(t)\ dt =  -1 \\
      = 2(-1) \\
      = -2
    .\end{align*}
    \bigbreak \noindent \bigbreak \noindent
    Using the property, which states:
    \begin{align*}
      \int_{a}^{c}\ f(x)\ dx = \int_{a}^{b}\ f(x)\ dx + \int_{b}^{c}\ f(x)\ dx
    .\end{align*}

    \bigbreak \noindent \bigbreak \noindent
    We can deduce that $a=1$ $b=2$ and $c=4$, from this logic we can see that our first integral is $\int_{a}^{c}$, and our second integral is $\int_{a}^{b}$, and we are asked to find $\int_{b}^{c}$
    \bigbreak \noindent \bigbreak \noindent
    Therefore:
    \begin{align*}
      -5 = -2 + \int_{b}^{c}f(t)\ dt
    .\end{align*}
    \bigbreak \noindent \bigbreak \noindent
    And if we let $\int_{b}^{c} f(t)\ dt= x$, and solve for x:
    \begin{align*}
      -5 = -2 + x \\
      \boxed{x = -3}
    .\end{align*}

    \pagebreak \bigbreak \noindent
    \begin{mdframed}
      \textbf{4.) Use part one of the fundemental theorem of calculus to find the derivative of:}
      \begin{align*}
        h(x) = \int_{t}^{3}\ \frac{1}{1+x^{2}}\ dx
      .\end{align*}
    \end{mdframed}

    \bigbreak \noindent \bigbreak \noindent
    To start, we must use the property:
    \begin{align*}
      \int_{a}^{b}\ f(x)\ dx = -\int_{b}^{a}\ f(x)\ dx
    .\end{align*}
    \bigbreak \noindent 
    To flip the limits of integration such that the upper limit is a funtion of x, so:
    \begin{align*}
      -\int_{3}^{t}\ \frac{1}{1+x^{2}}\ dx
    .\end{align*}

    \bigbreak \noindent \bigbreak \noindent
    From here we can use part one of the fundemental theorem of calculus to find the derivative, so:
    \begin{align*}
      h^{\prime}(x) = \frac{d}{dt} -\int_{3}^{t}\ \frac{1}{1+x^{2}}\ dx \\
      \boxed{= -\frac{1}{1+t^{2}}}
    .\end{align*}

    \pagebreak \bigbreak \noindent
    \begin{mdframed}
      \textbf{5.) Use Part 2 of the Fundamental Theorem of Calculus to evaluate the integral}
      \begin{align*}
        \int_{-1}^{1}\ (1-x^{2})\ dx
      .\end{align*}
    \end{mdframed}

    \bigbreak \noindent \bigbreak \noindent
    If first we find the indefinite integral:
    \begin{align*}
      \int (1-x^{2})\ dx \\
      = 1x-\frac{1}{3}x^{3}
    .\end{align*}

    \bigbreak \noindent \bigbreak \noindent
    We can then use the fundemental theorem of calculus to evalute the integral

    \bigbreak \noindent \bigbreak \noindent
    So:
    \begin{align*}
      1x-\frac{1}{3}x^{3}\bigg]_{-1}^{1} \\
      = \bigg(1(1)-\frac{1}{3}\bigg(1\bigg)^{3}\bigg) - \bigg(1(-1) - \frac{1}{3}\bigg(-1\bigg)^{3}\bigg) \\
      = \bigg(1-\frac{1}{3}\bigg) - \bigg(-1 +\frac{1}{3}\bigg) \\
      = \frac{2}{3} - \bigg(-\frac{2}{3}\bigg) \\
      \boxed{= \frac{4}{3}}
    .\end{align*}






   
\end{document}
