\documentclass{report}

\input{~/dev/latex/template/preamble.tex}
\input{~/dev/latex/template/macros.tex}
\graphicspath{{./}}

\title{\Huge{Quiz 1 }}
\author{\huge{Nathan Warner}}
\date{\huge{Jan 26, 2023}}

\pgfpagesdeclarelayout{boxed}
{
  \edef\pgfpageoptionborder{0pt}
}
{
  \pgfpagesphysicalpageoptions
  {%
    logical pages=1,%
  }
  \pgfpageslogicalpageoptions{1}
  {
    border code=\pgfsetlinewidth{1.5pt}\pgfstroke,%
    border shrink=\pgfpageoptionborder,%
    resized width=.95\pgfphysicalwidth,%
    resized height=.95\pgfphysicalheight,%
    center=\pgfpoint{.5\pgfphysicalwidth}{.5\pgfphysicalheight}%
  }%
}

\pgfpagesuselayout{boxed}

\begin{document}
    \maketitle

    \bigbreak \noindent \bigbreak \noindent 
   
    \bigbreak \noindent 
    \begin{Large}
       \textbf{Question 1:} 
    \end{Large}
    
    \bigbreak \noindent 
        The graph gives the position s(t) of an object moving along a line at time t, over a 2.5
        second interval. Find the average velocity of the object over the following intervals.

    \bigbreak \noindent 
    \textbf{a.) [0.5, 2.5]}

    \bigbreak \noindent 
    \pf{Solution}{}
    \noindent if Average Velocity = $ \frac{\text{change in height}}{\text{change in time}}$ 
    \bigbreak \noindent 
    Then at the interval \textbf{[0.5, 2.5]} we would have an average velocity of:

    \begin{align*}
        \frac{150-46}{2.5-0.5} \\
        =52
    .\end{align*}

    \bigbreak \noindent 
    \textbf{b.) [0.5, 2]}
   
    \bigbreak \noindent 
    \pf{Solution}{}
    \begin{align*}
        \frac{136-46}{2-0.5} \\
        =60
    .\end{align*}

    \bigbreak \noindent 
    \textbf{c.) [0.5, 1.5]} 
    \bigbreak \noindent 
    \pf{Solution}{}
    \begin{align*}
        \frac{114-46}{1.5-0.5} \\
        = 68
    .\end{align*}

    \bigbreak \noindent 
    \textbf{d.) [0.5, 1]}
    \bigbreak \noindent 
    \pf{Solution}{}
    \begin{align*}
        \frac{84-46}{1-0.5} \\ 
        = 76
    .\end{align*}

    \pagebreak
    \begin{Large}
       \noindent \textbf{Question 2:} 
    \end{Large}

    \bigbreak \noindent 
    Evaluate each of the following limits. No work to be shown.

    \bigbreak \noindent 
    \textbf{a.) $\lim\limits_{x \to 4+}{g \left(x\right)}$ = 2} 

    \bigbreak \noindent \bigbreak \noindent  
    \textbf{b.) $\lim\limits_{x \to 4-}{g \left(x\right)}$ = 0}

    \bigbreak \noindent \bigbreak \noindent 
    \textbf{c.) $\lim\limits_{x \to 2}{g \left(x\right)}$ = DNE}

    \bigbreak \noindent \bigbreak \noindent 
    \textbf{d.) $\lim\limits_{x \to 6}{g \left(x\right)}$ = 1}

    \bigbreak \noindent \bigbreak \noindent \bigbreak \noindent  
    \begin{Large}
       \textbf{Question 3:} 
    \end{Large}

    \bigbreak \noindent 
    Evaluate the limit. Show your work. Use limit notation when necessary.
    
    \begin{align*}
        \lim\limits_{x \to 9}{ \frac{9-x}{3-\sqrt{x}}}
    .\end{align*}

    \bigbreak \noindent 
    \pf{Solution}{}

    \noindent If we plug 9 into the denomonator, we get an output of 0. Therefore 9 is not in the 
    domain of this function and we must use \textbf{Direct Substitution Property.}

    \bigbreak \noindent 
    If we multiply the numerator and denomonator by $3 + \sqrt{x}$, we get:
    \begin{align*}
        \lim\limits_{x \to 9}{ \frac{ \left(9-x\right) \left(3+\sqrt{x}\right)}{ \left(3-\sqrt{x}\right) \left(3+\sqrt{x}\right)}}
    .\end{align*}

    \bigbreak \noindent 
    If we simplify the denomonator we get:
    \begin{align*}
        \lim\limits_{x \to 9}{ \frac{ \left(9-x\right) \left(3+\sqrt{x}\right)}{9-x}}
    .\end{align*}

    \bigbreak \noindent 
    Now we can cancel out the common term (9-x) and we are left with:
    \begin{align*}
        \lim\limits_{x \to 9}{3+\sqrt{x}}
    .\end{align*}

    \bigbreak \noindent 
    Now we just plug 9 into this new equation and output our answer
    \begin{align*}
        3+\sqrt{9} \\
        = 6
    .\end{align*}

    \bigbreak \noindent 
    Therefore:
    \begin{align*}
        \lim\limits_{x \to 9 }{ \frac{9-x}{3-\sqrt{x}}} = 6
    .\end{align*}
\end{document}
