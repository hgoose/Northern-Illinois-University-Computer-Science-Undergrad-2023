\documentclass{report}

\input{~/dev/latex/template/preamble.tex}
\input{~/dev/latex/template/macros.tex}

\title{\Huge{}}
\author{\huge{Nathan Warner}}
\date{\huge{}}
\pagestyle{fancy}
\fancyhf{}
\lhead{Warner \thepage}
\rhead{}
% \lhead{\leftmark}
\cfoot{\thepage}
%\setborder
% \usepackage[default]{sourcecodepro}
% \usepackage[T1]{fontenc}

\begin{document}
    % \maketitle
        \begin{titlepage}
       \begin{center}
           \vspace*{1cm}
    
           \textbf{Calculus 2} \\
            Problem Set 1.
    
           \vspace{0.5cm}
            
                
           \vspace{1.5cm}
    
           \textbf{Nathan Warner}
    
           \vfill
                
                
           \vspace{0.8cm}
         
           \includegraphics[width=0.4\textwidth]{~/niu/seal.png}
                
           Computer Science \\
           Northern Illinois University\\
           August 27, 2023 \\
           United States\\
           
                
       \end{center}
    \end{titlepage}
    % \tableofcontents
    \pagebreak \bigbreak \noindent
   \textbf{1.a} Evaluate $g(0),\ g(1),\ g(2),\ g(3),\ g(6)$ 
   \bigbreak \noindent 

   $g(0)$
   \begin{align*}
       &g(0) = \int_{0}^{0}\ f(t) \ dt \\
       &= 0\ \quad \text{(By integral with the same bounds equals zero)} 
   .\end{align*}
   \bigbreak \noindent 

   \textbf{$g(1)$}
   \begin{align*}
      &g(1) = \int_{0}^{1}\ f(t)\ dt \\ 
       &= lw  \\
       &= 1\cdot 2 \\
       &= 2
   .\end{align*}
   \bigbreak \noindent 

   $g(2) $
   \begin{align*}
       &g(2) = \int_{0}^{2}\ ft\ dt \\
       &= \bigg(lw\bigg)+\bigg(\frac{1}{2}bh\bigg) \\
       &=\bigg(2\cdot 2\bigg) + \bigg(\frac{1}{2}\bigg(1\bigg)\bigg(2\bigg)\bigg) \\
       &=4+1 \\
       &=5
   .\end{align*}
   \bigbreak \noindent 

   $g(3)$
   \begin{align*}
       &g(3) = \int_{0}^{3}\ f(t)\ dt \\
       &= g(2) + \frac{1}{2}bh \\
       &=5 + \frac{1}{2}(1)(4) \\
       &=5+2 \\
       &=7
   .\end{align*}
   \bigbreak \noindent 

   $g(6)$
   \begin{align*}
       &g(6) = \int_{0}^{6}\ ft\ dt \\
       &= g(3) - \bigg(\int_{3}^{6}\ ft\ dt\bigg) \\
       &= 7 - \bigg(\int_{3}^{5}\ f(t)\ dt\ + \int_{5}^{6}\ f(t)\ dt\bigg) \\
       &= 7 - \bigg(\frac{1}{2}(2)(2) + 1\cdot 2\bigg) \\
        &= 7- \bigg(2 + 2\bigg) \\
        &= 7- 4 \\
        &= 3
   .\end{align*}

   \pagebreak \bigbreak \noindent 
   \textbf{1.b} $g$ has a maximum value at $g(3)$

   \bigbreak \noindent 
   \textbf{1.c} $g$ is increasing on the interval $(0,3) $

   \bigbreak \noindent 
   \textbf{1.d} Rough sketch of $g$:
   \bigbreak \noindent 

   \textit{Figure:}
    \begin{figure}[ht]
        \centering
        \incfig{roughsketch}
        \label{fig:roughsketch}
    \end{figure}
    \bigbreak \noindent 
    2. Use part 1 of the Fundamental Theorem of Calculus to find the derivative of the functions
    \bigbreak \noindent 
        \begin{remark}
       Part 1: $\int_{a}^{x}\ f(t)\ dt $ = $f(x),\ a \leq x \leq b $ 
    \end{remark}


    \bigbreak \noindent 
    \textbf{2.a} 
    \bigbreak \noindent 
    
    \begin{align*}
        If:\ g(x) = \int_{3}^{x}\ \sqrt{9-t^{2}}\ dt \\
        Then:\ g^{\prime}(x) = \frac{d}{dx}\int_{3}^{x}\ \sqrt{9-t^{2}}\ dt \\
        =\sqrt{9-x^{2}}
    .\end{align*}

    \bigbreak \noindent 
    \textbf{2.b}
    \begin{align*}
        If:\ y = \int_{0}^{\ln{x}}\ e^{t}\ dt \\
        Then:\ y^{\prime} = \frac{d}{dx} \int_{0}^{\ln{x}}\ e^{t}\ dt \\
        = e^{\ln{x}} \cdot \frac{d}{dx}\ln{x} \\
        = e^{\ln{x}} \cdot \frac{1}{x} \\
        = \frac{e^{\ln{x}}}{x}
    .\end{align*}

    \pagebreak \bigbreak \noindent 
    3. Use part 2 of the Fundamental Theorem of Calculus to evaluate the integrals.

    \bigbreak \noindent 
    \textbf{3.a} $\int_{1}^{4}\ \frac{2-x^{\frac{1}{2}}}{x^{2}} \ dx $
    \begin{align*}
        &\int_{1}^{4}\ \frac{2-x^{\frac{1}{2}}}{x^{2}} \ dx  \\
        &= \int_{1}^{4}\ \frac{2}{x^{2}}-\frac{x^{\frac{1}{2}}}{x^{2}}\ dx \\
        &= \int_{1}^{4}\ 2x^{-2}-x^{-\frac{3}{2}}\ dx \\
        &= -2x^{-1} + 2x^{-\frac{1}{2}} \bigg]^{4}_1 \\
        &= \bigg(-2(4)^{-1}+2(4)^{-\frac{1}{2}}\bigg) - \bigg(-2(1)^{-1}+2(1)^{-\frac{1}{2}}\bigg) \\
        &= \bigg(-\frac{2}{4} + \frac{2}{\sqrt{4}}\bigg) - \cancelto{0}{\bigg(-2 + 2\bigg)} \\
        &= -\frac{1}{2}+1 \\
        &=\frac{1}{2}
    .\end{align*}

    \bigbreak \noindent 
    \textbf{3.b}
    \bigbreak \noindent 
    \scalebox{0.7}{
        \begin{minipage}{0.47\textwidth}
        \unitcircle
        \end{minipage}
    }
    \begin{minipage}{0.7\textwidth}
    \begin{align*}
        &\int_{0}^{\pi}\ (\sin{x}-3\sqrt{x})\ dx  \\
        &=\int_{0}^{\pi}\ (\sin{x}-3x^{\frac{1}{2}})\ dx \\
        &=-\cos{x}-2x^{\frac{3}{2}}\bigg]^{\pi}_0 \\
        &= \bigg(-\cos{\pi}-2(\pi)^{\frac{3}{2}}\bigg) - \bigg(-\cos{(0)}-2(0)^{\frac{3}{2}}\bigg) \\
        &= \bigg(-(-1)-2\pi^{\frac{1}{2}}\bigg) - \bigg(-1\bigg) \\
        &= 1-2\pi^{\frac{3}{2}} + 1 \\
        &= 2 - 2\pi^{\frac{3}{2}}
    .\end{align*}
    \end{minipage}

    \pagebreak \bigbreak \noindent 
    4. Evaluate the following integrals
    \bigbreak \noindent 
    \textbf{4.a}
    \begin{align*}
        &\int_{}^{}\ x^{\frac{1}{2}}(x^{2}+5x+2)\ dx \\
        &=\int x^{\frac{1}{2}+2}+5x^{1+\frac{1}{2}}+2x^{\frac{1}{2}}\ dx \\
        &=\int x^{\frac{5}{2}}+5x^{\frac{3}{2}}+2x^{\frac{1}{2}}\ dx \\
        &= \frac{2}{7}x^{\frac{7}{2}} + 2x^{\frac{5}{2}}+\frac{4}{3}x^{\frac{3}{2}} + C
    .\end{align*}

    \bigbreak \noindent 
    \textbf{4.b}
    \begin{align*}
        &\int_{1}^{2}\ \bigg(\frac{1}{x^{2}} -\frac{1}{x^{3}}\bigg)\ dx \\
        &=\int_{1}^{2}\ x^{-2}-x^{-3}\ dx \\
        &=-x^{-1}+\frac{1}{2}x^{-2}\bigg]^{2}_1 \\
        &=\bigg(-(2)^{-1}+\frac{1}{2}(2)^{-2}\bigg) - \bigg(-(1)^{-1}+\frac{1}{2}(1)^{-2}\bigg) \\
        &= \bigg(-\frac{1}{2}+\frac{1}{8}\bigg) - \bigg(-1 + \frac{1}{2}\bigg) \\
        &=-\frac{3}{8} - \bigg(-\frac{1}{2}\bigg) \\
        &= -\frac{3}{8} + \frac{1}{2} \\
        &=\frac{1}{8}
    .\end{align*}

    \bigbreak \noindent 
    \textbf{4.c}
    \bigbreak \noindent 
    \scalebox{0.7}{
        \begin{minipage}{0.47\textwidth}
            \unitcircle
        \end{minipage}
    }
    \begin{minipage}{0.7\textwidth}
    \begin{align*}
        &\int_{0}^{\pi}\ (2\sin{x}-3\sec^{2}{x})\ dx \\
        &=-2\cos{x}-3\tan{x}\bigg]^{\pi}_0 \\
        &= \bigg(-2\cos{\pi}-3\tan{\pi}\bigg) - \bigg(-2\cos{0}-3\tan{0}\bigg) \\
        &= \bigg(-2(-1)-3(0)\bigg) - \bigg(-2(1)-3(0)\bigg) \\
        &= 2 - (-2) \\
        &= 4
    .\end{align*}
    \end{minipage}

    \pagebreak \bigbreak \noindent 
    \textbf{4.d}
    \bigbreak \noindent 
    \scalebox{.8}{
        \begin{minipage}{0.47\textwidth}
        \unitcircle
        \end{minipage}
    }
    \begin{minipage}{0.7\textwidth}
    \begin{align*}
        &\int_{0}^{\frac{\pi}{4}}\ \sec{x}\tan{x}\ dx \\
        &= \sec{x}\bigg]^{\frac{\pi}{4}}_0 \\
        &= \sec{\frac{\pi}{4}} - \sec{0} \\
        &= \frac{1}{\frac{\sqrt{2}}{2}} - 1 \\
        &= \frac{2\sqrt{2}}{2} - 1 \\
        &= \sqrt{2} -1
    .\end{align*}
    \end{minipage}

    \bigbreak \noindent 
    5. Find displacement and distance traveled of $t^{2}-3t-18$, $0 \leq t \leq 6 $
    \bigbreak \noindent 
    \begin{remark}
        Displacement: $\int_{a}^{b}\ v(t)\ dt$ and 
        Distance Traveled: $\int_{a}^{b}\ \abs{v(t)}\ dt $
    \end{remark}
    \bigbreak \noindent 

    Displacement:
    \begin{align*}
        &\int_{0}^{6}\ t^{2}-3t-18\ dt \\
        &= \frac{1}{3}t^{3}-\frac{3}{2}t^{2}-18t\bigg]^{6}_0 \\
        &= \bigg(\frac{1}{3}(6)^{3} - \frac{3}{2}(6)^{2}-18(6)\bigg) - \cancelto{0}{\bigg(\frac{1}{3}(0)^{3} -\frac{3}{2}(0)^{2}-18(0)\bigg)} \\
        &= 72 - 54 - 108 \\
        &= -90
    .\end{align*}
    \bigbreak \noindent 

    Distance Traveled:
    \begin{align*}
        \int_{0}^{6}\ \abs{t^{2}-3t-18}\ dt
    .\end{align*}
    \bigbreak \noindent 
    
    Find where the function turns negative
    \begin{align*}
        t^{2} - 3t-18 = 0 \\
        (t+3)(t-6) \\
        t = \cancelto{}{-3}, 6
    .\end{align*}
    \bigbreak \noindent 
    
    Rewrite Piecewise
       \begin{equation}
        v(t)=
            \begin{cases}
                t^{2}-3t-18& \text{if } t > 6 \\
                 -(t^{2}-3t-18)& \text{if } t \leq 6 
            \end{cases}
        \end{equation}

        \pagebreak \bigbreak \noindent 
        Thus:
        \begin{align*}
            &\int_{0}^{6}\ -(t^{2}-3t-18)\ dt \\
            &=\int_{0}^{6}\ -t^{2}+3t+18\ dt \\
            &= -\frac{1}{3}t^{3}+\frac{3}{2}t^{2}+18t\bigg]^{6}_0 \\
            &= \bigg(-\frac{1}{3}(6)^{3}+\frac{3}{2}(6)^{2}+18(6)\bigg) - \cancelto{0}{\bigg(-\frac{1}{3}(0)^{3}+\frac{3}{2}(0)^{2}+18(0)\bigg)} \\
            &= 90
        .\end{align*}


    



    
    

    
    


\end{document}
