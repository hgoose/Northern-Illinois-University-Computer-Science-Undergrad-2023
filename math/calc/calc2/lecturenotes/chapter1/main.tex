\documentclass{report}

\input{~/dev/latex/template/preamble.tex}
\input{~/dev/latex/template/macros.tex}

\title{\Huge{}}
\author{\huge{Nathan Warner}}
\date{\huge{}}
\pagestyle{fancy}
\fancyhf{}
\lhead{Warner \thepage}
\rhead{}
% \lhead{\leftmark}
\cfoot{\thepage}
%\setborder
% \usepackage[default]{sourcecodepro}
% \usepackage[T1]{fontenc}

\begin{document}
    % \maketitle
        \begin{titlepage}
       \begin{center}
           \vspace*{1cm}
    
           \textbf{Calculus II} \\
           Chapter 1: Integration
    
           \vspace{0.5cm}
            
                
           \vspace{1.5cm}
    
           \textbf{Nathan Warner}
    
           \vfill
                
                
           \vspace{0.8cm}
         
           \includegraphics[width=0.4\textwidth]{~/niu/seal.png}
                
           Computer Science \\
           Northern Illinois University\\
           August 28, 2023 \\
           United States\\
           
                
       \end{center}
    \end{titlepage}
    \tableofcontents
    \pagebreak \bigbreak \noindent
        \begin{center}
        \section*{\Huge Preface}
    \end{center}
    \bigbreak \noindent 
    \line(1,0){490}
    \bigbreak \noindent 
    Much of this chapter has been omitted, as it is mostly review and discussed in other documents. Topics of discussion in chapter I include:
    \begin{itemize}
        \item Approximating Areas
        \item The Definite Integral
        \item The Fundamental Theorem of Calculus
        \item Integration Formulas and the Net Change Theorem
        \item Substitution
        \item Integrals Involving Exponential and Logarithmic Functions
    \end{itemize}
    \bigbreak \noindent 
    This document will \textbf{only} contain notes for section \textit{1.7: Integrals Resulting in Inverse Trigonometric Functions}
    \pagebreak \bigbreak \noindent 
    \vspace{2in} \\
    \begin{Huge}
      \textbf{Chapter I: \\ Integration}  
    \end{Huge}
    \bigbreak \noindent 
    \line(1,0){490}
    \bigbreak \noindent 
    \section*{\LARGE 1.7: Integrals Resulting in Inverse Trigonometric Functions}
    \bigbreak \noindent 
    Recall that trigonometric functions are not one-to-one unless the domains are restricted. When working with inverses of trigonometric functions, we always need to be careful to take these restrictions into account. Also in Derivatives, we developed formulas for derivatives of inverse trigonometric functions. The formulas developed there give rise directly to integration formulas involving inverse trigonometric functions.
    \bigbreak \noindent 
    The following integration formulas yield inverse trigonometric functions. 
    \begin{enumerate}
        \item \begin{align*}
                \int \frac{du}{\sqrt{a^{2}-u^{2}}} = \sin^{-1}{\frac{u}{\abs{a}}} + C
        .\end{align*}
    \item \begin{align*}
        \int \frac{du}{a^{2}+u^{2}} = \frac{1}{a}\tan^{-1}{\frac{u}{a}} + C
    .\end{align*}
    \item \begin{align*}
            \int \frac{du}{u\sqrt{u^{2}-a^{2}}} = \frac{1}{\abs{a}}\sec^{-1}{\frac{\abs{u}}{a}} + C
    .\end{align*}
    \end{enumerate}
    If we Assume  $a>0$, then it is acceptable to drop the absolute value bars.
    \pagebreak \bigbreak \noindent 
    The following example can be found in exercise 429 of chapter 1.7. 
    \begin{examp}
        \begin{align*}
            \int_{0}^{\frac{1}{2}}\ \frac{\tan{(\sin^{-1}{t})}}{\sqrt{1-t^{2}}}\ dt \\
        .\end{align*}
    \end{examp}
    \begin{prop}
       $\tan{(\sin^{-1}{x})} = \frac{x}{\sqrt{1-x^{2}}}$ 
    \end{prop}
    \bigbreak \noindent 
    \begin{proof}
       Show that $\tan{(\sin^{-1}{x})} = \frac{x}{\sqrt{1-x^{2}}}$ 
       \begin{align*}
           y = \sin^{-1}{x} \\
           \sin{y} = x \\
           \sin{y} = \frac{x}{1} \\
           Where\ \sin{x} = \frac{opp}{hyp}
       .\end{align*}
       \bigbreak \noindent 

       Now if we draw a right triangle
       \bigbreak \noindent 
       \begin{minipage}{0.47\textwidth}
           \incfig{triangle}
       \end{minipage}
       \begin{minipage}{0.47\textwidth}
        Then we can solve for the unknown side $c$:
        \begin{align*}
            &x^{2} + a^{2} = 1^{2} \\
            &a^{2} = 1-x^{2} \\
            &a = \sqrt{1-x^{2}}
        .\end{align*}
        \bigbreak \noindent 
        Thus:
        \begin{align*}
            &\tan{x} = \frac{opp}{adj} \\
            &\tan{y} = \frac{x}{\sqrt{1-x^{2}}} \\
            &=\implies \tan{(\sin^{-1}{x})} = \frac{x}{\sqrt{1-x^{2}}}
        .\end{align*}
       \end{minipage}
       \bigbreak \noindent 
       \end{proof}
       Using this, we can compute the integral.
       \bigbreak \noindent 
      \begin{align*}
           &\int_{0}^{\frac{1}{2}}\ \frac{\tan{(\sin^{-1}{t})}}{\sqrt{1-t^{2}}}\ dt \\
           &=\int_{0}^{\frac{1}{2}}\ \frac{\frac{t}{\sqrt{1-t^{2}}}}{\sqrt{1-t^{2}}}\ dt \\
           &=\int_{0}^{\frac{1}{2}}\ \frac{t}{\sqrt{1-t^{2}}\sqrt{1-t^{2}}}\ dt \\
           &=\int_{0}^{\frac{1}{2}}\ \frac{t}{1-t^{2}}\ dt 
       .\end{align*}
       \bigbreak \noindent 
       \begin{minipage}[t]{0.47\textwidth}
       \begin{align*}
           \text{Let $u=1-t^{2}$} \\
           du = -2t\ dt \\
           -\frac{1}{2}du = t\ dt \\
           u(a) = 1-(0)^{2} = 1 \\
           u(b) =  1-\bigg(\frac{1}{2}\bigg)^{2} = \frac{3}{4}
       .\end{align*}
       \end{minipage}
       \begin{minipage}[t]{0.47\textwidth}
        \begin{align*}
          &=-\frac{1}{2}\int_{1}^{\frac{3}{4}}\ u^{-1}\ du \\
           &=-\frac{1}{2}\bigg[\ln{\abs{u}}\bigg]^{\frac{3}{4}}_{1} \\
           &= -\frac{1}{2}\bigg(\ln{\bigg|\frac{3}{4}\bigg|}-\ln{\abs{1}}\bigg) \\ 
           &= -\frac{1}{2}\ln{\bigg|\frac{3}{4}\bigg|}
       .\end{align*}
       \end{minipage}

    

    

    \bigbreak \noindent 
    \subsection{Integrals Resulting in Other Inverse Trigonometric Functions}
    \bigbreak \noindent 
    There are six inverse trigonometric functions. However, only three integration formulas are noted in the rule on integration formulas resulting in inverse trigonometric functions because the remaining three are negative versions of the ones we use. The only difference is whether the integrand is positive or negative. Rather than memorizing three more formulas, if the integrand is negative, simply factor out −1 and evaluate the integral using one of the formulas already provided. 

    \pagebreak \bigbreak \noindent 
    \section{\LARGE Additional Information from chapter I}
    \subsection{The Mean Value Theorem for Integrals}
    \bigbreak \noindent 
    The Mean Value Theorem for Integrals states that a continuous function on a closed interval takes on its average value at some point in that interval.
    \smallbreak \noindent
    \begin{definition} 
        \textbf{If $f(x)$ is continuous over an interval $[a,b]$, then there is at least one point  $c \in [a,b] $ such that
            \begin{align*}
                f(c) = \frac{1}{b-a}\ \int_{a}^{b}\ f(x)\ dx
            .\end{align*}
        This formula can also be stated as:
        \begin{align*}
            \int_{a}^{b}\ f(x)\ dx = f(c)(b-a)
        .\end{align*}
        } 

    \end{definition}
    \bigbreak \noindent \bigbreak \noindent 
    \subsection{Integration Formulas Involving Logarithmic Functions}
    \bigbreak \noindent 
    \begin{enumerate}
        \item         
    \begin{align*}
        \int \log_{a}{x} = \frac{x}{\ln{a}}(\ln{x}-1) + C
    .\end{align*}
    \item 
        \begin{align*}
            \int \ln{x}\ dx = x\ln{x} -x + C
        .\end{align*}
    \end{enumerate}






    
    
\end{document}
