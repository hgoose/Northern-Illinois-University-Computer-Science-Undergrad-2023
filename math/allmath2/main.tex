\documentclass{report}

\input{~/dev/latex/template/preamble.tex}
\input{~/dev/latex/template/macros.tex}

\title{\Huge{}}
\author{\huge{Nathan Warner}}
\date{\huge{}}
\pagestyle{fancy}
\fancyhf{}
\lhead{Warner \thepage}
\rhead{}
% \lhead{\leftmark}
\cfoot{\thepage}
%\setborder
% \usepackage[default]{sourcecodepro}
% \usepackage[T1]{fontenc}

\begin{document}
    % \maketitle
        \begin{titlepage}
       \begin{center}
           \vspace*{1cm}
    
           \textbf{Comprehensive Compendium:} \\
            Calculus II
    
           \vspace{0.5cm}
            
                
           \vspace{1.5cm}
    
           \textbf{Nathan Warner}
    
           \vfill
                
                
           \vspace{0.8cm}
         
           \includegraphics[width=0.4\textwidth]{~/niu/seal.png}
                
           Computer Science \\
           Northern Illinois University\\
           August 28,2023 \\
           United States\\
           
                
       \end{center}
    \end{titlepage}
    \tableofcontents
    \pagebreak \bigbreak \noindent
    \section{\LARGE Calc II}
    \bigbreak \noindent 
    \subsection{Integrals resulting in inverse trig functions}
    \bigbreak \noindent 
    Recall that trigonometric functions are not one-to-one unless the domains are restricted. When working with inverses of trigonometric functions, we always need to be careful to take these restrictions into account.
        \bigbreak \noindent 
    The following integration formulas yield inverse trigonometric functions. Assume  $a>0$
    \begin{enumerate}
        \item \begin{align*}
                \int \frac{du}{\sqrt{a^{2}-u^{2}}} = \sin^{-1}{\frac{u}{\abs{a}}} + C
        .\end{align*}
    \item \begin{align*}
        \int \frac{du}{a^{2}+u^{2}} = \frac{1}{a}\tan^{-1}{\frac{u}{a}} + C
    .\end{align*}
    \item \begin{align*}
            \int \frac{du}{u\sqrt{u^{2}-a^{2}}} = \frac{1}{\abs{a}}\sec^{-1}{\frac{\abs{u}}{a}} + C
    .\end{align*}
    \end{enumerate}
        \bigbreak \noindent 
    There are six inverse trigonometric functions. However, only three integration formulas are noted in the rule on integration formulas resulting in inverse trigonometric functions because the remaining three are negative versions of the ones we use. The only difference is whether the integrand is positive or negative. Rather than memorizing three more formulas, if the integrand is negative, simply factor out −1 and evaluate the integral using one of the formulas already provided. 


    
\end{document}
