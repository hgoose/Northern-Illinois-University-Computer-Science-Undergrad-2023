\documentclass[12pt]{article}

%
%Margin - 1 inch on all sides
%
\usepackage[letterpaper]{geometry}
\usepackage{times}
\geometry{top=1.0in, bottom=1.0in, left=1.0in, right=1.0in}

%
%Doublespacing
%
\usepackage{setspace}
\doublespacing

%
%Rotating tables (e.g. sideways when too long)
%
\usepackage{rotating}


%
%Fancy-header package to modify header/page numbering (insert last name)
%
\usepackage{fancyhdr}
\pagestyle{fancy}
\lhead{} 
\chead{} 
\rhead{Warner \thepage} 
\lfoot{} 
\cfoot{} 
\rfoot{} 
\renewcommand{\headrulewidth}{0pt} 
\renewcommand{\footrulewidth}{0pt} 
%To make sure we actually have header 0.5in away from top edge
%12pt is one-sixth of an inch. Subtract this from 0.5in to get headsep value
\setlength\headsep{0.333in}

%
%Works cited environment
%(to start, use \begin{workscited...}, each entry preceded by \bibent)
% - from Ryan Alcock's MLA style file
%
\newcommand{\bibent}{\noindent \hangindent 40pt}
\newenvironment{workscited}{\newpage \begin{center} Works Cited \end{center}}{\newpage }


%
%Begin document
%
\begin{document}
\begin{flushleft}

%%%%First page name, class, etc
Nathan Warner\\
Professor Goad\\
Math 128\\
25 July 2023
 \\


%%%%Title
\begin{center}
    Midterm Reflection
\end{center}


%%%%Changes paragraph indentation to 0.5in
\setlength{\parindent}{0.5in}
%%%%Begin body of paper here
1.) So far I am having an enjoyable experience in this class. I have learned a ton of sublime information, and overall having a fun time.

2.) My life outside of class is overall good, I am transferring to NIU in the coming fall which I am very excited about. I've got a couple of weeks to prepare before I move into my dorm.

3.) My overall grade so far is 94.2% (A). If this was my final grade I would be happy with it, although it could be a bit higher if I didn't miss some points on the homework assignments. It does not surprise me that there is a high correlation between the grade a student has at midterm and the final grade in the course, I have noticed that after midterms I usually have a good grip on how to maintain or improve my grade in the coming units due to learning how exams are structured and study/learning techniques for the material.

4.) A positive moment I've had so far was seeing my unit 1 exam grade. I am very happy in seeing that I received an A. I was excited to see this as I am aiming for an A in the course, so that exam grade is a major step in that direction. I was a bit nervous about my grade because I neglected Simpson's paradox during my studying, so I wasn't very confident with my answer to the last question.

5.) One weakness that I have addressed in the following chapters is the actions I took while completing chapters 1-4, for these chapters, I waited until I had all of my notes completed before I started the review process. For the following chapters (5-8), I started the review process after each chapter, which definitely made study a lot easier.

6.) Specific:

             - What: achieving an overall grade of at least 95% 

              - When: by the end of the course.

              - Why: To demonstrate that I have properly learned all material for this course.

    To do this I must achieve 100% on all future assignments, and 95%s on both upcoming exams. Furthermore, I must take diligent notes and properly review them to ensure that I have a strong understanding of the material come test time. This goal will keep me confident about future math courses and set me on the correct path to academic success. For this goal, I will spend at least 32 hours per week on course material.

7.) One of my strengths is persistence. This strength allows me to continue on even when things may get hard, I have great determination to not skip over material. One of my past math professors stated that the difference between A students and B students is that A students will not accept not understanding a certain concept, they will continue with it until it is properly learned and understood, and this is something that I have integrated into my life. Furthermore, I recently read the book Atomic Habits, by James Clear, where he discusses the science behind making good habits and the road to mastery. In the book, one of the many things he discusses is the difference between masters and the people that are "just okay", he explains that the masters make of habit of showing up even when things get hard, and the power of consistency.

8.) I believe that I demonstrate these attributes, in terms of Zest, I believe that my curiosity and interest in mathematical subjects allow me to properly be engaged whilst learning new material. Furthermore, as discussed in the above question, my persistence allows for continual incremental gains in the knowledge obtained in my academic path, which will eventually lead to my career path. The possibilities of what I will surely become if I continue to hammer down on my studies provide continual motivation. 

9.) The advice I would give my past self would be to use the more efficient study techniques discussed in question 5.

10.) I believe that practice tests are extremely helpful in studying for exams, and I'm also happy to see when professors implement them. I also really enjoy the interactive assignments. I am a massive fan of Micheal Sullivan so Im always happen to see him make an appearance in the assignment videos. Overall I think that this course is structured extremely well and would like to see nothing changed.


\newpage



%%%%Works cited
\begin{workscited}

\bibent


\end{workscited}

\end{flushleft}
\end{document}
