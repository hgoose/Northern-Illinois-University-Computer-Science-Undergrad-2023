\documentclass{report}

\input{~/dev/latex/template/preamble.tex}
\input{~/dev/latex/template/macros.tex}

\title{\Huge{}}
\author{\huge{Nathan Warner}}
\date{\huge{}}
\pagestyle{fancy}
\fancyhf{}
\lhead{Warner \thepage}
\rhead{}
% \lhead{\leftmark}
\cfoot{\thepage}
%\setborder
% \usepackage[default]{sourcecodepro}
% \usepackage[T1]{fontenc}

\begin{document}
    % \maketitle
        \begin{titlepage}
       \begin{center}
           \vspace*{1cm}
    
           \textbf{Bash Mastery} \\
           The complete guide to BASH shell scripting
    
           \vspace{0.5cm}
            
                
           \vspace{1.5cm}
    
           \textbf{Nathan Warner}
    
           \vfill
                
                
           \vspace{0.8cm}
         
           \includegraphics[width=0.4\textwidth]{}
                
           Computer Science \\
           Northern Illinois University\\
           August 3, 2023 \\
           United States\\
           
                
       \end{center}
    \end{titlepage}
    \tableofcontents
    \pagebreak \bigbreak \noindent
    \section{\LARGE Setting up scripts}
    \bigbreak \noindent 
    In order to write shell scripts, we must use the file extension .sh. When we begin the script, we must include a \textit{shebang}, which looks something like:
    \bigbreak \noindent 
    \begin{minted}{bash}
#!/usr/bin/bash
    \end{minted}
    \bigbreak \noindent 
    However, this could depend on the users system. To locate which file path to use, we can use the command:
    \begin{minted}{bash}
which bash
    \end{minted}
    \bigbreak \noindent 
    This will provide the path to use for the shebang.
    \bigbreak \noindent 
    The anatomy of a shell script can be described with the following parts:
    \bigbreak \noindent 
    \begin{enumerate}
        \item Shebang 
        \item Commands
        \item Exit statement (0=successful, 1-255=unsuccessful)
    \end{enumerate}
    \bigbreak \noindent 
    Example:
    \bigbreak \noindent 
    \begin{mdframed}[style=purplebox]
    \begin{minted}{bash}
#!/usr/bin/bash

echo "Hello World!"
exit 0
    \end{minted}
    \bigbreak \noindent 
    \end{mdframed}
    \bigbreak \noindent 
    \nt{The recommended file permissions for scripts is 744 (chmod 744 \textit{filename})}

    \bigbreak \noindent 
    \subsection{Adding scripts to PATH}
    \bigbreak \noindent 
    To add scripts to your PATH for BASH, we can open up our .bashrc, and add at the bottom of the file:
    \bigbreak \noindent 
    \begin{mdframed}[style=purplebox]
    \begin{minted}{bash}
export PATH="$PATH:$HOME/dirlocation"
    \end{minted}
    \end{mdframed}
    \bigbreak \noindent 
    This will append some directory to the end of our PATH variable. To do the same for the \textit{fish} shell, in the fish config file, we can add.
    \bigbreak \noindent 
    \begin{mdframed}[style=purplebox]
    \begin{minted}{bash}
set -gx PATH $PATH $HOME/somedirectory
    \end{minted}
    \end{mdframed}

    \pagebreak \bigbreak \noindent 
    \section{\LARGE Variables and Shell Expansions}
    \bigbreak \noindent 
    \subsection{User-Defined variables and parameter expansion}
    \bigbreak \noindent 
    \smallbreak \noindent
    \begin{definition}
        A \textbf{parameter} is any entity that stores values. In bash, we have three types:
        \begin{enumerate}
            \item Variables
            \item Positional Parameters
            \item Special Parameters
        \end{enumerate}
    \end{definition}
    \bigbreak \noindent 
    To define variables in our script, we can do:
    \begin{mdframed}[style=purplebox]
    \begin{minted}{bash}
identifier=value # NO WHITESPACE
name="nate" # Example
declare -i a=1 # Integer variable

# Parameter Expansion (Reference variables)
echo "Hello, ${name}!" 
    \end{minted}
    \bigbreak \noindent 
    \end{mdframed}

    \bigbreak \noindent 
    \subsection{Shell variables}
    \bigbreak \noindent 
    Shell variables are builtin variables that we can access but don't need to define ourself, some common shell variables are:
    \begin{itemize}
        \item PATH
        \item HOME 
        \item USER 
        \item HOSTNAME
        \item HOSTTYPE
    \end{itemize}
    \bigbreak \noindent 
    \subsection{Positional Parameters}
    \bigbreak \noindent 
    Positional parameters are variables that hold the command-line arguments to a script or function. They are denoted by numbers.
    \begin{enumerate}
        \item \$0 (Contains the name of the script)
        \item \$1, \$2 ... \$n (The first, second, third, etc. arguments to the script or function.)
        \item \$\# (The number of arguments passed to the script or function.)
        \item \$@ (All the arguments. When quoted ("\$@"), it treats each argument as a separate word. Useful for loops, more on this later)
        \item \$* (All the arguments. When quoted ("\$*"), it treats all arguments as a single word. Useful for loops, more on this later)
    \end{enumerate}

    \pagebreak \bigbreak \noindent 
    \subsection{Special Parameters}
    \bigbreak \noindent 
    These are variables that provide special functionality or information about the script or command's execution:
    \bigbreak \noindent 
    \begin{itemize}
        \item \$?: The exit status of the last executed command. \(0\) usually indicates success, and a non-zero value indicates an error.
        \item \$\$: The process ID (PID) of the currently executing script or shell instance.
        \item \$!: The process ID (PID) of the last backgrounded command.
        \item \$-: The current options set for the shell. For instance, if you used \texttt{set -x} for debugging, \texttt{x} would be part of the value.
        \item \$\_: The last argument of the previous command. Also sometimes used to get the last path argument to the \texttt{cd} command.
    \end{itemize}

    \bigbreak \noindent 
    \subsection{Parameter Expansion Tricks}
    \bigbreak \noindent 
    Default Values:
    \begin{itemize}
        \item \$\{parameter:-word\}: If parameter is unset or null, this expansion will return word. Otherwise, it returns the value of parameter.
        \item \$\{parameter:=word\}: If parameter is unset or null, it will be set to word.
    \end{itemize}
    \bigbreak \noindent 
    String Length:
    \begin{itemize}
        \item \${\#parameter}: Returns the length of the value of the parameter.
    \end{itemize}
    \bigbreak \noindent 
    Substring Expansion:
    \begin{itemize}
        \item \$\{parameter:offset:length\}: Extracts a substring from \$parameter starting at offset (0-indexed) and of length length.
    \end{itemize}
    \bigbreak \noindent 
    String Removal (Pattern Matching):
    \begin{itemize}
        \item \$\{parameter\#pattern\}: Removes the shortest match of pattern from the beginning of \$parameter.
        \item \$\{parameter\#\#pattern\}: Removes the longest match of pattern from the beginning of \$parameter.
        \item \$\{parameter\%pattern\}: Removes the shortest match of pattern from the end of \$parameter.
        \item \$\{parameter\%\%pattern\}: Removes the longest match of pattern from the end of \$parameter. 
    \end{itemize}
    \bigbreak \noindent 
    String Replacement:
    \begin{itemize}
        \item \$\{parameter/pattern/string\}: Replaces the first match of pattern with string in \$parameter.
        \item \$\{parameter//pattern/string\}: Replaces all matches of pattern with string in \$parameter
    \end{itemize}
    \bigbreak \noindent 
    Variable Indirection:
    \begin{itemize}
        \$\{!parameter\}: Treats the value of parameter as the name of another variable, and fetches the value of that variable.
    \end{itemize}
    \bigbreak \noindent 
    Case Modification:
    \begin{itemize}
        \item \$\{parameter$\land$\}: Capitalizes the first letter of the value.
        \item \$\{parameter$\land\land$\}: Capitalizes all letters of the value.
        \item \$\{parameter,\}: Converts the first letter to lowercase.
        \item \$\{parameter,,\}: Converts all letters to lowercase.
    \end{itemize}

    \pagebreak \bigbreak \noindent 
    \subsection{Command Substitution}
    \bigbreak \noindent 
    \begin{concept}
		 Command substitution can be used to:
	\end{concept}
    \begin{itemize}
        \item Save the output of commands in variables
        \item Use the output of one command \textit{inside} another command
    \end{itemize}
    \bigbreak \noindent 
    The syntax for this is:
    \begin{mdframed}[style=purplebox]
    \begin{minted}{bash}
$(command)
# Example...
time=$(date +%H:%M:%S)
echo "Hello, the current time is ${time}"
    \end{minted}
    \bigbreak \noindent
    \end{mdframed}

    \bigbreak \noindent 
    \subsection{Arithmetic Expansion}
    \bigbreak \noindent 
    The syntax for \textit{Arithmetic Expansion} is:
    \begin{mdframed}[style=purplebox]
    \begin{minted}{bash}
$((expression))
# Example...
echo $((1+1)) # 2
# When dealing with arithmetic expansion, we do not need a $ to reference variables
x=1
y=1
echo $(( x + y ))
    \end{minted}
    \bigbreak \noindent
    \end{mdframed}
    
    \bigbreak \noindent 
    \subsection{Dealing with floating point numbers}
    \bigbreak \noindent 
    To be able to do floating point arithmetic in our scripts, we need to use the \textbf{bc} command.
    \bigbreak \noindent 
    Example:
    \begin{mdframed}[style=purplebox]
    \begin{minted}{bash}
echo "scale=2; 5/2" | bc # 2.50
# scale sets the precision of the output
    \end{minted}
    \bigbreak \noindent
    \end{mdframed}
    \bigbreak \noindent 

    \pagebreak \bigbreak \noindent 
    \subsection{Tilde Expansion}
    \bigbreak \noindent 
    I'm sure you're already familiar with using tilde to jump to your home directory, but we can also use $\sim$- to jump between our current directory, and our home directory

    \bigbreak \noindent 
    \subsection{Brace Expansion}
    \bigbreak \noindent 
    We have two types of brace expansions:
    \begin{itemize}
        \item String lists
        \item Range lists
    \end{itemize}
    \bigbreak \noindent 
    Here is examples of what we can do with brace expansion:
    \bigbreak \noindent 
    \begin{mdframed}[style=purplebox]
    \begin{minted}{bash}
echo {jan,feb,march} # jan feb march NO WHITESPACE IN BRACES
echo {1..5} # 1 2 3 4 5
echo {1..10..2} # 1 3 5 7 9
echo {a..e} # a b c d e
echo {a,b}{1,2,3} # (Cartesian product...) a1 a2 a3 b1 b2 b3
# Useful for commands...
mkdir dir_{1..3}.txt
touch file_{1..5}.txt
    \end{minted}
    \bigbreak \noindent
    \end{mdframed}

    \pagebreak \bigbreak \noindent 
    \section{\LARGE How Bash Processes Commands}
    \bigbreak \noindent 
    Bash uses a 5 step process to interpret a command
    \bigbreak \noindent 
    \begin{enumerate}
        \item \textbf{Tokenisation:} A token is a sequence of characters that is considered as a single unit by the shell. The shell determines were a token starts and ends with the following special (meta) characters
            \begin{itemize}
                \item |
                \item \&
                \item ;
                \item ()
                \item < >
                \item Space, tab, newline
            \end{itemize}
            The shell then determines whether these tokens are words, or operators. A \textbf{word} is a token that does not contain an unquoted metacharacter. \textbf{Operators} are tokens that contain at least one unquoted metacharacter. This makes quoting a key concept in how the shell operates
        \item \textbf{Command identification:} The shell breaks the command up into either simple, or compound commands. \textbf{Simple commands} are just a bunch of individual words, and each simple command is terminated by a control operator. \textbf{Compound commands} provide bash with its programming constructs, such as if statements, for loops, while loops, etc...
        \item \textbf{Expansions:}
        \item \textbf{Quote removal:} We add quotes to control how the command is interpreted, so this step will simply remove all those supportive quotes.
        \item \textbf{Redirection:}
    \end{enumerate}
    After these 5 steps are completed, bash will then execute the command that is left over.

    \bigbreak \noindent 
    \subsection{Quoting}
    \bigbreak \noindent 
    \begin{concept}
		 Quoting is about \textbf{removing special meanings}. There are three types of quoting:
	\end{concept}
    \begin{itemize}
        \item \textbf{Backslash (\textbackslash):} This removes special meaning from next character
        \item \textbf{Single Quotes:} Removes special meaning from all characters inside
        \item \textbf{Double Quotes:} Removes special meaning from all inside except dollar signs (\$) and backticks (\ \`\ )
    \end{itemize}

    \pagebreak \bigbreak \noindent 
    \subsection{Step 1. Tokenisation}
    \bigbreak \noindent 
    We learned earlier that whether or not a token is interpreted as a word or an operator, depends on if there are any \textbf{unquoted metacharacters}. In bash we have two types of operators:
    \bigbreak \noindent 
    \begin{minipage}[t]{0.5\textwidth}
    \textbf{Control operators:}
    \begin{itemize}
        \item Newline: command separator, similar to the semicolon (;).
        \item | used to send the output of one command as the input to another command.
        \item || Used to execute the command following it only if the command preceding it fails 
        \item \& This is used to execute a command in the background
        \item \&\& Used to execute the command following it only if the command preceding it succeeds 
        \item ; Acts as a command separator.
        \item ;; Used in the context of a case statement in shell scripting. It signifies the end of an option within a case block.
        \item ;\& Also used in a case statement. After executing the associated block for a matched pattern, the control will flow to the block of the next pattern without testing.
        \item ;;\& Another operator used in a case statement. The control will test the next pattern after executing the block for the matched pattern.
        \item |\& This is shorthand for 2>&1 |. It redirects both standard output (stdout) and standard error (stderr) of the command before the pipe to the command after the pipe.
        \item ( used to group commands and execute them in a subshell. 
        \item ) used to group commands and execute them in a subshell. 
    \end{itemize}
   \end{minipage}
   \begin{minipage}[t]{0.5\textwidth}
       \textbf{Redirection Operators}
       \begin{itemize}
           \item < Redirects input for a command from a file rather than from the keyboard.
            \item > Redirects the output of a command to a file
            \item << Provides multiple lines of input to a command 
            \item >> Similar to >, but instead of overwriting the file, it appends to the file.
            \item <\& Duplicates one input file descriptor to another, allowing for more advanced redirections.
            \item >\& Duplicates one output file descriptor to another.
            \item >| (Clobber): This is used in conjunction with the noclobber option in Bash (set -o noclobber). It allows you to forcefully overwrite a file when output redirection is used, even if noclobber is set.
            \item <<- Similar to <<, but it allows leading tabs (not spaces) to be ignored, 
            \item <>  Opens a file in read-write mode for a command. 
       \end{itemize}
   \end{minipage}

   \pagebreak \bigbreak \noindent 
   \subsection{Step 2. Command Identification (Simple commands)}
   \bigbreak \noindent 
   As we stated earlier, we have two types of commands, simple, and complex. Let's have a look at a simple command.
   \bigbreak \noindent 
   \begin{center}
       $\underbrace{\text{echo}}_{\parbox{2cm}{\centering \text{Command}\\ \text{name}}} \underbrace{1\ 2\ 3}_{\parbox{2cm}{individual arguments}}$
   \end{center}
   \bigbreak \noindent 
   All simple commands are terminated by a control operator, which we have discussed earlier. In this case, it is a newline
   \bigbreak \noindent 
   Thus, "echo" is identified as the command, and the rest is identified as the commands arguments, since there are no \textbf{control operators}

   \bigbreak \noindent 
  \subsection{Step 2. Command Identification (Simple commands)}
  \bigbreak \noindent 
  \begin{concept}
		 Compound commands are essentially bash's programming constructs. Each compound command begins with a \textbf{reserved word} and is terminated by a \textbf{reserved word}.
	\end{concept}

  \bigbreak \noindent 
  \subsection{Step 3. Expansions}
  \bigbreak \noindent 
  The shell goes through 4 stages of expansion.
  \begin{itemize}
      \item \textbf{Stage 1:} Brace expansion
        \item \textbf{Stage 2:}
            \begin{itemize}
                \item Parameter expansion
                \item Arithmetic expansion
                \item Command substation
                \item Tilde expansion
            \end{itemize}
        \item \textbf{Stage 3:} Word splitting
        \item \textbf{Stage 4:} Globbing
  \end{itemize}
  \bigbreak \noindent 
  An important thing to know is that expansions in earlier stages, happen before expansions in later stages. This means if we had the code:
  \begin{mdframed}[style=purplebox]
  \begin{minted}{bash}
x=10
echo {1..$x}
  \end{minted}
  \bigbreak \noindent
  \end{mdframed}
  \bigbreak \noindent 
  We would \textbf{not} get the expected results, this is because the brace expansion will take place \textbf{before} the parameter expansion.
  \bigbreak \noindent 
  Another important thing to know is that items in \textbf{stage 2} have the same precedence, thus, they will be preformed left to right. Similar to how multiplication and division works in the rules of PEMDAS.

  \pagebreak \bigbreak \noindent 
  \subsection{Word splitting}
  \bigbreak \noindent 
  \begin{concept}
     \textbf{Word splitting}  is a process the shell performs to split the result of some unquoted expansions into separate words. Word splitting can have some very significant effects on how your commands are interpreted
  \end{concept}
  \bigbreak \noindent 
  Word splitting is only performed on the results of unquoted 
  \begin{itemize}
      \item Parameter expansions
        \item Command substitutions
        \item Arithmetic expansions
  \end{itemize}
  \bigbreak \noindent 
  The characters used to split words are governed by the IFS (Internal Field Separator) variable.
  \begin{itemize}
      \item Space, tab, and newline
  \end{itemize}
  \bigbreak \noindent 
  Suppose we have:
  \begin{mdframed}[style=purplebox]
  \begin{minted}{bash}
numbers="1 2 3 4 5"
touch ${numbers}
  \end{minted}
  \bigbreak \noindent
  \end{mdframed}
  \bigbreak \noindent 
  We will get 5 \textbf{different} files, labeled 1-5. This is because the parameter expansion was \textbf{unquoted}. Instead, we can do:
  \begin{mdframed}[style=purplebox]
  \begin{minted}{bash}
numbers="1 2 3 4 5"
touch "${numbers}"
  \end{minted}
  \bigbreak \noindent
  \end{mdframed}
  \bigbreak \noindent 
  This will prevent word splitting, and treat "1 2 3 4 5" as a single word. Creating just \textbf{one} file.
  \bigbreak \noindent 
  So we shall follow one simple rule. If we want the output of a:
  \begin{itemize}
      \item Parameter expansion 
        \item Command substation
        \item Arithmetic expansion
  \end{itemize}
  \bigbreak \noindent 
  To be considered as a \textbf{single word}, we must \textbf{wrap that expansion in double quotes!}

  \pagebreak \bigbreak \noindent 
  \subsection{Globbing}
  \bigbreak \noindent 




  




    

    





    
\end{document}
