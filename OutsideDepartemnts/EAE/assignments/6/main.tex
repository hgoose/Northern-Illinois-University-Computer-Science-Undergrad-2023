\documentclass{report}

\input{~/dev/latex/template/preamble.tex}
\input{~/dev/latex/template/macros.tex}

\title{\Huge{}}
\author{\huge{Nathan Warner}}
\date{\huge{}}
\pagestyle{fancy}
\fancyhf{}
\lhead{Warner \thepage}
\rhead{}
% \lhead{\leftmark}
\cfoot{\thepage}
%\setborder
% \usepackage[default]{sourcecodepro}
% \usepackage[T1]{fontenc}

\begin{document}
    % \maketitle
    %     \begin{titlepage}
    %    \begin{center}
    %        \vspace*{1cm}
    % 
    %        \textbf{EAE 103} \\
    %        Assignment 5
    % 
    %        \vspace{0.5cm}
    %         
    %             
    %        \vspace{1.5cm}
    % 
    %        \textbf{Nathan Warner}
    % 
    %        \vfill
    %             
    %             
    %        \vspace{0.8cm}
    %      
    %        \includegraphics[width=0.4\textwidth]{}
    %             
    %        Computer Science \\
    %        Northern Illinois University\\
    %        September 25, 2023 \\
    %        United States\\
    %        
    %             
    %    \end{center}
    % \end{titlepage}
    % \tableofcontents
    \bigbreak \noindent 
    \textbf{1a. Regions in the asteroid belt contain few or no asteroids. Asteroids at these distances
would experience a resonance with Jupiter and the Sun and would be pulled out of these zones.
What are these asteroid-free zones called?}
\bigbreak \noindent 
    \pf{Answer}{
        \textbf{Kirkwood gaps}
    }

    \bigbreak \noindent 
    \textbf{1b. What is the name of the largest asteroid (it is also a dwarf planet)?} 
    \bigbreak \noindent 
    \pf{Answer}{
        \textbf{Ceres} 
    }

    \bigbreak \noindent 
    \textbf{1c. What are the names of asteroids trapped in Jupiter’s orbit at Jupiter’s Lagrange points (60o
    ahead and behind Jupiter)?}
    \bigbreak \noindent 
    \pf{Answer}{
    \textbf{Trojan asteroids} 
    }

    \bigbreak \noindent 
    \textbf{1d. What is the name of asteroids that cross the orbit of Mars, but not Earth?}
    \bigbreak \noindent 
    \pf{Answer}{
        \textbf{Amor asteroids.} 
    }

    \bigbreak \noindent 
    \textbf{1e. What name is given to Near Earth Asteroids (NEAs) and comets that cross the Earth’s orbit
and can be deadly?}
    \bigbreak \noindent 
    \pf{Answer}{
        \textbf{ Potentially Hazardous Asteroids (PHAs).}
    }

    \bigbreak \noindent 
    \textbf{2a. What type asteroid is the most common in the inner asteroid belt? }
    \bigbreak \noindent 
    \pf{Answer}{
        \textbf{S-type (silicaceous) } 
    }

    \bigbreak \noindent 
    \textbf{2b. What type is most common in the outer asteroid belt?}
    \bigbreak \noindent 
    \pf{Answer}{
        \textbf{C-type (carbonaceous) asteroid.}
    }

    \bigbreak \noindent 
    \textbf{2c.  Which type of asteroid appears to be metal-rich, occurs throughout the asteroid belt, and
    might be mined for these metals in the future?}
    \bigbreak \noindent 
    \pf{Answer}{
        M-type (metallic)
    }

    \pagebreak \bigbreak \noindent 
    \textbf{3a. Approximately how many asteroids have been discovered as of March, 2020 (check
    credible Web sites to get the latest numbers since the textbook may be slightly out of date)?}
    \bigbreak \noindent 
    \pf{Answer}{
        \begin{align*}
            \approx  1,308,871
        .\end{align*}
    source: Nasa
    
    }

    \bigbreak \noindent 
    \textbf{3b. Science fiction movies and TV shows often show spacecraft having to dodge asteroids.
    What is wrong with this depiction?}
    \bigbreak \noindent 
    \pf{Answer}{
        Simply put, the asteroid belt is very large. The average distance between asteroids in the belt is hundreds of thousands of kilometers.
    }

    \bigbreak \noindent 
    \textbf{4.}
    \begin{itemize}
        \item \textbf{Meteor:}   streak or burst of light in the sky as a meteoroid burns
        \item \textbf{Meteoroid:} a small planetary body before entry into Earth’s atmosphere
        \item \textbf{Meteorite:} solid material from a meteoroid that reaches Earth’s surface 
    \end{itemize}

    \bigbreak \noindent 
    \textbf{5a. List the 3 main categories of meteorites}
    \bigbreak \noindent 
    \pf{Answer}{
        \bigbreak \noindent 
        \begin{itemize}
            \item Stony meteorites (or Chondrites)
            \item Iron meteorites
            \item Stony-iron meteorites
        \end{itemize}
    }

    \bigbreak \noindent 
    \textbf{5b. What part of the solar system is probably the source for most meteorites? }
    \bigbreak \noindent 
    \pf{Answer}{
        \textbf{asteroid belt between Mars and Jupiter.}
    }

    \bigbreak \noindent 
    \textbf{5c. What type of meteorite is the most primitive, and contains organic molecules?}
    \bigbreak \noindent 
    \pf{Answer}{
        \textbf{carbonaceous chondrite} 
    }

    \bigbreak \noindent 
    \textbf{5d. Why do iron meteorites comprise only about 6\% of falls, but 50\% of finds?}
    \bigbreak \noindent 
    \pf{Answer}{
        \textbf{because they are more resistant to weathering and erosion compared to stony meteorites.}
    }

    \bigbreak \noindent 
    \textbf{6a. During a meteor shower, meteors appear to come from one distinct area in the sky.
What is this called?}
    \bigbreak \noindent 
    \pf{Answer}{
        \textbf{The radiant}
    }

    \pagebreak \bigbreak \noindent 
    \textbf{6b. What are Meteors called when they occur without a meteor shower?}
    \bigbreak \noindent 
    \pf{Answer}{
    
        \textbf{sporadic meteors}
    }

    \bigbreak \noindent 
    \textbf{6c. Meteor showers occur when Earth’s orbit intersects the orbit of what other type of
planetary body?}
    \bigbreak \noindent 
    \pf{Answer}{
    \textbf{comet} 
    }

    \bigbreak \noindent 
    \textbf{7a. Most comets have 2 types of tails. What are the names of these tails?}
    \bigbreak \noindent 
    \pf{Answer}{
        \bigbreak \noindent 
        \begin{itemize}
            \item Ion tail (or Type I tail)
            \item Dust tail (or Type II tail)
        \end{itemize}
    }
    
    \bigbreak \noindent 
    \textbf{7b.  From what part of the solar system do most comets originate?}
    \bigbreak \noindent 
    \pf{Answer}{
        \textbf{Oort Cloud and the Kuiper Belt.}
    
    }

    \bigbreak \noindent 
    \textbf{7c.  Is Halley’s comet a long- or short-period comet? }
    \bigbreak \noindent 
    \pf{Answer}{
        \textbf{short-period comet.}

    }

    \bigbreak \noindent 
    \textbf{7d. What causes a comet’s nuclei to break up?}
    \bigbreak \noindent 
    \pf{Answer}{
   \textbf{A comet's nuclei can break up due to the intense heat and gravitational forces as they approach the Sun.} 
    }
    
    \bigbreak \noindent 
    \textbf{7e. A large impact from an asteroid or comet hit the northern Yucatán Peninsula of Mexico
approximately 65 million years ago, causing widespread extinctions (including the eventual end
of the dinosaurs). What is the name of this impact feature (named after the Mexican village
near the impact site)?}
    \bigbreak \noindent 
    \pf{Answer}{
        \textbf{Chicxulub crater}

    }

    \bigbreak \noindent 
    \textbf{8a. What is the approximate density (in g/cm3
) of comet nuclei?}
    \bigbreak \noindent 
    \pf{Answer}{
        \begin{align*}
            0.5 g/cm^{3}.
        .\end{align*}
    
    }

    \pagebreak \bigbreak \noindent 
    \textbf{8b.  What does this density tell us about the structure of comet nucle}
    \bigbreak \noindent 
    \pf{Answer}{
    This relatively low density suggests that comet nuclei have a porous structure.  
    }
    
    \bigbreak \noindent 
    \textbf{8c. List at least two chemical compounds identified in comet nuclei. }
    \bigbreak \noindent 
    \pf{Answer}{
        \bigbreak \noindent 
        \begin{itemize}
            \item $H_{2}O$
            \item $CO_{2}$
        \end{itemize}
    }

    \begin{align*}
        (x^{2}+36) = (x+6)(x-6)
    .\end{align*}











\end{document}
