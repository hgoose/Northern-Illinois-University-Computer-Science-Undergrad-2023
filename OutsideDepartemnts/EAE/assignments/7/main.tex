\documentclass{report}

\input{~/dev/latex/template/preamble.tex}
\input{~/dev/latex/template/macros.tex}

\title{\Huge{}}
\author{\huge{Nathan Warner}}
\date{\huge{}}
\pagestyle{fancy}
\fancyhf{}
\lhead{Warner \thepage}
\rhead{}
% \lhead{\leftmark}
\cfoot{\thepage}
%\setborder
% \usepackage[default]{sourcecodepro}
% \usepackage[T1]{fontenc}

\begin{document}
    % \maketitle
    %     \begin{titlepage}
    %    \begin{center}
    %        \vspace*{1cm}
    % 
    %        \textbf{EAE 103} \\
    %        Assignment 5
    % 
    %        \vspace{0.5cm}
    %         
    %             
    %        \vspace{1.5cm}
    % 
    %        \textbf{Nathan Warner}
    % 
    %        \vfill
    %             
    %             
    %        \vspace{0.8cm}
    %      
    %        \includegraphics[width=0.4\textwidth]{}
    %             
    %        Computer Science \\
    %        Northern Illinois University\\
    %        September 25, 2023 \\
    %        United States\\
    %        
    %             
    %    \end{center}
    % \end{titlepage}
    % \tableofcontents
    \bigbreak \noindent 
    \textbf{1a. From the readings, why is it difficult to drop a probe like Galileo into Jupiter’s atmosphere?}
    \bigbreak \noindent 
    \pf{Answer}{
        It is difficult to drop a probe like Galileo into Jupiter's atmosphere because of reasons such as:
        \bigbreak \noindent 
        \begin{itemize}
            \item Harsh Atmospheric Conditions, high amount of pressure
            \item Radiation 
            \item Short lifespan
        \end{itemize}

    }

    \bigbreak \noindent 
    \textbf{1b. How did engineers solve this problem?}
    \bigbreak \noindent 
    \pf{Answer}{
        Engineers found ways to protect against radiation, and ensured the probe could quickly transmit as much data as possible before its inevitable end in the hostile environment.
    }

    \bigbreak \noindent 
    \textbf{2a. From the reading: (a) Explain why visual observation of the gas giants is not sufficient to
    determine their rotation periods.}
    \bigbreak \noindent 
    \pf{Answer}{
        Gas giants like Jupiter and Saturn do not rotate as solid bodies. Different "latitudinal bands" can rotate at different rates. Also, Gas giants have thick atmospheres with volatile cloud patterns, storms, and winds. These features can move at different velocities and directions, which can be misleading when trying to determine a planet's rotation period based based on these features 
    }

    \bigbreak \noindent 
    \textbf{2b. What evidence was used to deduce the rotation correct periods?}
    \pf{Answer}{
         Jupiter and Saturn have strong magnetic fields. These magnetic fields have periodic variations as the planet rotates. By studying the regular fluctuations in the magnetic field, scientists can determine a more accurate rotation period for the planet's interior, which is believed to be more representative of the planet's true rotation period than surface observations alone.
    }

    \bigbreak \noindent 
    \textbf{3. At the pressures in Jupiter’s interior, what two forms does Hydrogen take?}
    \bigbreak \noindent 
    \pf{Answer}{
        \bigbreak \noindent 
        \begin{itemize}
            \item Molecular Hydrogen
            \item Metallic Hydrogen
        \end{itemize}
    
    }

    \bigbreak \noindent 
    \textbf{4. Jupiter has the strongest magnetic field and largest magnetosphere of all the planets. What hazards does this pose to spacecraft?}
    \bigbreak \noindent 
    \pf{Answer}{
        \bigbreak \noindent 
        \begin{itemize}
            \item Intense Radiation
            \item Charged Particle Impact
            \item Communication Interference
            \item Navigation Disruptions
        \end{itemize}
    
    }

    \bigbreak \noindent 
    \textbf{5. List at least 4 main gases in Jupiter and Saturn’s atmosphere}
    \bigbreak \noindent 
    \pf{Answer}{
        \bigbreak \noindent 
        \begin{itemize}
            \item Hydrogen
            \item Helium
            \item Methane
            \item Ammonia
        \end{itemize}
    
    }

    \bigbreak \noindent 
    \textbf{6. Suppose a small comet was on a path to collide with Earth. Its density is \(0.5 \, \text{g/cm}^3\) (or \(500 \, \text{kg/m}^3\)). Earth's radius is \SI{6378}{km} and Earth's density is about \(5500 \, \text{kg/m}^3\) (or \(5.5 \, \text{g/cm}^3\)). At what height would it disintegrate (i.e., what is its Roche limit)?}
    \bigbreak \noindent 
    \pf{Answer}{



    \[ d \approx 6378 \times \left( 2 \times \frac{5500}{500} \right)^{1/3} \]
    \[ \Rightarrow d \approx 6378 \times (11)^{1/3} \]
    \[ \Rightarrow d \approx 14176.7 \text{ km} \]

    This is the distance from the center of Earth. The height \( h \) above Earth's surface is:

    \[ h = d - R \]
    \[ \Rightarrow h \approx 14176.7 - 6378 \]
    \[ \Rightarrow h \approx 7798.7 \text{ km} \]

    Thus, the comet would begin to disintegrate at a height of approximately \( 7798.7 \text{ km} \) above Earth's surface.
    }

    \bigbreak \noindent 
    \textbf{7. Saturn’s moon Titan is extremely cold, but it shares similarities with Earth. List two similarities and two differences with Earth.}
    \bigbreak \noindent 
    \pf{Answer}{
        \textbf{Similarities}
        \bigbreak \noindent 
        \begin{itemize}
            \item Atmosphere
            \item Liquid on Surface
        \end{itemize}
        \bigbreak \noindent 
        \textbf{Differences}
        \bigbreak \noindent 
        \begin{itemize}
            \item Temperature
            \item Atmospheric Composition
        \end{itemize}
    
    }

    \pagebreak \bigbreak \noindent 
    \textbf{8. As a practice for the quiz please match the following moons with the correct characteristics:}
    \bigbreak \noindent 
    \pf{Answer}{
        \bigbreak \noindent 
        \begin{itemize}
            \item Only moon with an atmosphere: Titan
            \item Old surface showing no geological activity: Callisto
            \item Cracked icy surface with subsurface ocean: Europa
            \item Active volcanoes: Io
            \item Erupting jets of water into space: Enceladus
            \item Giant crater on one side: Mimas
            \item Old surface with large canyon or chasm: Tethys
            \item Largest moon in the solar system: Ganymede
        \end{itemize}
    
    }
    









\end{document}
