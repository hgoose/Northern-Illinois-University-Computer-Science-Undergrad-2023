\documentclass{report}

\input{~/dev/latex/template/preamble.tex}
\input{~/dev/latex/template/macros.tex}

\title{\Huge{}}
\author{\huge{Nathan Warner}}
\date{\huge{}}
\pagestyle{fancy}
\fancyhf{}
\lhead{Warner \thepage}
\rhead{}
% \lhead{\leftmark}
\cfoot{\thepage}
%\setborder
% \usepackage[default]{sourcecodepro}
% \usepackage[T1]{fontenc}

\begin{document}
    % \maketitle
        \begin{titlepage}
       \begin{center}
           \vspace*{1cm}
    
           \textbf{Assignment 1} \\
           EAE 103
    
           \vspace{0.5cm}
            
                
           \vspace{1.5cm}
    
           \textbf{Nathan Warner}
    
           \vfill
                
                
           \vspace{0.8cm}
         
           \includegraphics[width=0.4\textwidth]{~/niu/seal.png}
                
           Computer Science \\
           Northern Illinois University\\
           August 30, 2023 \\
           United States\\
           
                
       \end{center}
    \end{titlepage}
    % \tableofcontents
    \pagebreak \bigbreak \noindent
    \textbf{1.a}
    \bigbreak \noindent 
    \begin{center}
        \includegraphics[scale=0.5]{ ~/Pictures/planets.png }
    \end{center}

    \bigbreak \noindent 
    \textbf{1.b} Mars

    \bigbreak \noindent 
    \textbf{1.c} Jupiter, Saturn

    \bigbreak \noindent 
    \textbf{2.}
    \bigbreak \noindent 
    \textit{Table 7.1:}
    \begin{center}
        
        \scalebox{0.8}{
        
    \begin{tabular}{|c|c|c|c|c|c|}
        \hline
        Name & Distance from Sun (AU) & Revolution Period (y) & Diameter (km) & Mass ($10^{23}$ kg) & Density (g/cm$^3$) \\
        \hline
        Mercury & 0.39 & 0.24 & 4,878 & 3.3 & 5.4 \\
        Venus & 0.72 & 0.62 & 12,120 & 48.7 & 5.2 \\
        Earth & 1.00 & 1.00 & 12,756 & 59.8 & 5.5 \\
        Mars & 1.52 & 1.88 & 6,787 & 6.4 & 3.9 \\
        Jupiter & 5.20 & 11.86 & 142,984 & 18,991 & 1.3 \\
        Saturn & 9.54 & 29.46 & 120,536 & 5686 & 0.7 \\
        Uranus & 19.18 & 84.07 & 51,118 & 866 & 1.3 \\
        Neptune & 30.06 & 164.82 & 49,660 & 1030 & 1.6 \\
        \hline
    \end{tabular}
        }
    \end{center}


    \bigbreak \noindent 
    \textbf{2.a} Jupiter is a distance of:
    \begin{align*}
        5.2\ AU \cdot \frac{93E6\ Million\ Miles}{1\ AU} \cdot \frac{1\ Foot}{1E6\ Million Miles} \\
        = \frac{5.2 \cdot 93E6}{1E6} \\ 
        = 483.6\ ft
    .\end{align*}

    \bigbreak \noindent 
    \textbf{2.b} Saturn  is a distance of:
    \begin{align*}
        9.54\ AU \cdot \frac{93E6\ Million\ Miles}{1\ AU} \cdot \frac{1\ Foot}{1E6\ Million Miles} \\
        = \frac{9.54 \cdot 93E6}{1E6} \\ 
        = 887.22\ ft
    .\end{align*}

    \bigbreak \noindent 
    \textbf{2.c} Uranus is a distance of:
    \begin{align*}
        19.18\ AU \cdot \frac{93E6\ Million\ Miles}{1\ AU} \cdot \frac{1\ Foot}{1E6\ Million Miles} \\
        = \frac{19.18 \cdot 93E6}{1E6} \\ 
        = 1783.74\ ft
    .\end{align*}

    \pagebreak \bigbreak \noindent 
    \textbf{3.a} The Moon's orbital path is tilted approximately 5 degrees relative to the Earth's orbital plane, known as the ecliptic. Eclipses are only possible when the Moon intersects this ecliptic plane during either a New Moon or a Full Moon.

    \bigbreak \noindent 
    \textbf{3.b} A node is the point at which the moon intersects the earths ecliptic
    
    \bigbreak \noindent 
    \textbf{4.a}   New moon, waxing crescent, first quarter, waxing gibbous, full moon, waning gibbous, third quarter, waning- crescent.
    \bigbreak \noindent 
    \textbf{4.b} This cycle takes 29.5 days and is called the \textbf{synodic period}





    
\end{document}
