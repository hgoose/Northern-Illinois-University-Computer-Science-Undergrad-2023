\documentclass{report}

\input{~/dev/latex/template/preamble.tex}
\input{~/dev/latex/template/macros.tex}

\title{\Huge{}}
\author{\huge{Nathan Warner}}
\date{\huge{}}
\pagestyle{fancy}
\fancyhf{}
\lhead{Warner \thepage}
\rhead{}
% \lhead{\leftmark}
\cfoot{\thepage}
%\setborder
% \usepackage[default]{sourcecodepro}
% \usepackage[T1]{fontenc}

\begin{document}
    % \maketitle
        \begin{titlepage}
       \begin{center}
           \vspace*{1cm}
    
           \textbf{Assignment 3}
    
           \vspace{0.5cm}
            
                
           \vspace{1.5cm}
    
           \textbf{Nathan Warner}
    
           \vfill
                
                
           \vspace{0.8cm}
         
           \includegraphics[width=0.4\textwidth]{~/niu/seal.png}
                
           Computer Science \\
           Northern Illinois University\\
           September 14, 2023 \\
           United States\\
           
                
       \end{center}
    \end{titlepage}
    \tableofcontents
    \pagebreak \bigbreak \noindent
    \textbf{Question 1.} What is the name for the outward flow of charged particles from the Sun, flowing to Earth and beyond?
    \bigbreak \noindent 
    \textbf{Answer.} Solar wind

    \bigbreak \noindent 
    \textbf{Question 2.} What is the most violent type of eruption from the surface of the Sun? These events pose a major hazard to astronauts, spacecraft electronics, and can disrupt communications on Earth.
    \bigbreak \noindent 
    \textbf{Answer.} Solar flare

    \bigbreak \noindent 
    \textbf{Question 3.} Sunspots appear dark because they are cooler regions on the Sun's surface compared to the surrounding areas. The period of the sunspot cycle is approximately 11 years. The darkest part of a sunspot is called the "umbra.". The Maunder Minimum is a period of extremely low sunspot activity that occurred between the years 1645 and 1715.

    \bigbreak \noindent 
    \textbf{Question 4.} Convection currents (rising and sinking gas) occur just below the Sun’s photosphere.  What feature do we see in the Sun’s photosphere that provides evidence for this?
    \bigbreak \noindent 
    \textbf{Answer.} The feature in the Sun's photosphere that provides evidence for convection currents just below it is known as "granulation." Granules on the Sun's surface show that convection currents are happening below. These bright spots are where hot gas rises, and the darker areas are where it cools and sinks back down.

    \bigbreak \noindent 
    \textbf{Question 5.} Nuclear fusion, in which hydrogen atoms combine to form helium atoms, releasing tremendous energy, occurs in what part of the Sun? 
    \bigbreak \noindent 
    \textbf{Answer.} Nuclear fusion occurs in the Sun's core. This is where hydrogen atoms combine to form helium, releasing a huge amount of energy in the process.

    \bigbreak \noindent 
    \textbf{Question 6.} What illustration does he use to show how loose particles may have been clumped (accreted) together from a Supernova (exploding star) shock wave?
    \bigbreak \noindent 
    \textbf{Answer.} He shows that gravity draws together the dust which clump together to form  asteroids, and after some time, rocky planets. He provides an example of a snow plow, which gathers snow together, clumping together the snow to create snow mounds.

    \bigbreak \noindent 
    \textbf{Question 7.} What is the first stage in the solar nebula theory origin of the solar system and What types of materials are formed near the Sun in this first stage?
    \bigbreak \noindent 
    \textbf{Answer.} The first stage in the solar nebula theory for the origin of the solar system is the collapse of a giant molecular cloud, also known as a solar nebula

    \bigbreak \noindent 
    \textbf{Question 8.} Earth’s average temperature is about 58.6oF. Convert this to:  (a) oC (Celcius) and (b) K (Kelvins).
    \bigbreak \noindent 
    \textbf{Answer.} 
    \bigbreak \noindent 

    To convert from F to C: 
    \begin{align*}
        ^\circ\text{C} = \frac{( 58.6 - 32 ) \times 5}{9} \\
        ^\circ\text{C} = \frac{26.6 \times 5}{9} \\
        ^\circ\text{C} = \frac{133}{9} \\
        ^\circ\text{C} = 14.78 \\
    .\end{align*}

    To convert from C to K:
    \begin{align*}
        \text{K} = 14.78 + 273.15 \\
        \text{K} = 287.93
    .\end{align*}

    \pagebreak \bigbreak \noindent 
    \textbf{Question 9.} The ozone layer is unique to Earth in the planets of our solar system.  (a) What is the approximate height of the ozone layer above Earth’s surface and in which layer of the atmosphere does the ozone layer exist?   (b) Why is the ozone layer important for survival of life on Earth. 
    \bigbreak \noindent 
    \textbf{Answer.} The ozone layer is about 15 to 35 miles. The layer in which it lies is the stratosphere. \\
    It's important because it blocks most of the Sun's harmful UV rays, keeping us and other living things safe from health problems like skin cancer.

    \bigbreak \noindent 
    \textbf{Question 10.} List an example of a sedimentary rock, igneous rock, and metamorphic rock.
    \bigbreak \noindent 
    \textbf{Answer.} Sedimentary Rock: Limestone \\
        Igneous Rock: Granite \\
        Metamorphic Rock: Marble






\end{document}
