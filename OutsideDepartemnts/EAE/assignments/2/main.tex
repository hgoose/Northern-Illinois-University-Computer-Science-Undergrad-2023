\documentclass{report}

\input{~/dev/latex/template/preamble.tex}
\input{~/dev/latex/template/macros.tex}

\title{\Huge{}}
\author{\huge{Nathan Warner}}
\date{\huge{}}
\pagestyle{fancy}
\fancyhf{}
\lhead{Warner \thepage}
\rhead{}
% \lhead{\leftmark}
\cfoot{\thepage}
%\setborder
% \usepackage[default]{sourcecodepro}
% \usepackage[T1]{fontenc}

\begin{document}
    % \maketitle
        \begin{titlepage}
       \begin{center}
           \vspace*{1cm}
    
           \textbf{Assignment 2}
    
           \vspace{0.5cm}
            
                
           \vspace{1.5cm}
    
           \textbf{Nathan Warner}
    
           \vfill
                
                
           \vspace{0.8cm}
         
           \includegraphics[width=0.4\textwidth]{~/niu/seal.png}
                
           Computer Science \\
           Northern Illinois University\\
           September 7, 2023 \\
           United States\\
           
                
       \end{center}
    \end{titlepage}
    \tableofcontents
    \pagebreak \bigbreak \noindent
    \begin{mdframed}
        \textbf{1. Suppose a new planet was discovered that lies at an average distance of 50 AU from the sun. What is its orbital period?}
    \end{mdframed}
    \bigbreak \noindent 

    By Kepler's third law, orbital period is computed by $p^{2} = a^{3}$, where $p$ is the period of planet orbit in years, and $a$ is the average distance of planet from Sun in AU
    \bigbreak \noindent 

    Thus, 
    \begin{align*}
        p^{2} = 50^{3} \\
        p^{2} = 125e3 \\
        p = 353.55\ \text{years}
    .\end{align*}

    \bigbreak \noindent 
    \begin{mdframed}
        \textbf{2. Suppose your weight on Earth is \(150 \, \text{lbs}\ (668 \, \text{Newtons})\) and your mass is \(68 \, \text{kg}\). The acceleration due to gravity on Mars is \(3.7 \, \text{m/s}^2\). What is your weight on Mars in Newtons and pounds? 
        Note that \(1 \, \text{Newton (1 N)}\) is \(1 \, \text{kg m} / \text{s}^2\). Also, \(4.4 \, \text{N} = 1 \, \text{lb}\) of force.
}
    \end{mdframed}
    \bigbreak \noindent 

    By Newton's second law of motion, which states: $f = ma$ (force = mass $\times$ acceleration)
    \begin{align*}
        F = ma = (68)(3.7) \\
        = 351.6\ N \\
    .\end{align*}
    \begin{align*}
    \text{Weight on Mars in lbs} &= \frac{251.6 \, \text{N}}{4.4 \, \text{N/lb}} \\
    &\approx 57.18 \, \text{lbs}
    \end{align*}

    \bigbreak \noindent 
    \begin{mdframed}
        \textbf{3. Why is the term “weightless” extremely misleading when referring to astronauts and spacecraft in orbit around the Earth?}
    \end{mdframed}
    \bigbreak \noindent 
    
    Weight is the force exerted on an object due to gravity. Since gravity is still acting on the astronauts, they technically still "weigh" something—they are just not experiencing a normal force against them like we do when standing on Earth's surface.

    \pagebreak 
    \begin{mdframed}
        \textbf{4. Write the Universal Law of Gravitation and define each term in the equation. (b) Is gravitational force attractive or repulsive? (c) List one use for the Universal Law of Gravitation.}
    \end{mdframed}
    \bigbreak \noindent 
    According to Sir Issac Newton, \textit{Newton's law of Gravitation} is sated as follows:
    \begin{align*}
        F = G\frac{m_{1}m_{2}}{r^{2}}
    .\end{align*}
    \bigbreak \noindent 
    Where:
    \begin{itemize}
        \item  F = attractive force between two masses 
        \item  G = Universal Gravitational Constant = 6.67 x 10-11 m3/(s2 kg)
        \item  $m_{1}$ = larger mass
        \item  $m_{2}$ = smaller mass
        \item  r = distance between the center of masses
    \end{itemize}
    \bigbreak \noindent 
    Newton described gravational force as being \textit{attractive}. One use for this equation can be finding the acceleration due to gravity at Earth's surface.

    \bigbreak \noindent 
    \begin{mdframed}
        \textbf{5.  In the video of Sunita Williams touring the International Space Station (ISS) what
happens to bone density and muscle mass in the human body during prolonged periods in
space? (b) Why is the exercise bar (simulating weight lifting) not anchored to the space shuttle?}
    \end{mdframed}
    \bigbreak \noindent 
    The conditions of microgravity mean that the skeletal and muscular systems are not exposed to the usual loads and stresses that they experience on Earth, leading to atrophy. The exercise bar is not anchored to the space shuttle because The reason is Newton's Third Law. If an astronaut were to lift a weight that was anchored to the spacecraft, the action of lifting would produce an equal and opposite reaction force, which could potentially shift the orientation of the space station.

    \bigbreak \noindent 
    \begin{mdframed}
        \textbf{6. In the video of the Space Shuttle launch STS 122 what happens to the solid rocket
boosters at the end of the video?}
    \end{mdframed}
    \bigbreak \noindent 
    The solid rocket booster separated from the rest of the rocket. The Space Shuttle used two Solid Rocket Boosters to provide additional thrust during the initial minutes of the flight. These boosters then disconnect from the rest of the rocket because they are no longer needed.


    
\end{document}
