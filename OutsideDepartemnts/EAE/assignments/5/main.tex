\documentclass{report}

\input{~/dev/latex/template/preamble.tex}
\input{~/dev/latex/template/macros.tex}

\title{\Huge{}}
\author{\huge{Nathan Warner}}
\date{\huge{}}
\pagestyle{fancy}
\fancyhf{}
\lhead{Warner \thepage}
\rhead{}
% \lhead{\leftmark}
\cfoot{\thepage}
%\setborder
% \usepackage[default]{sourcecodepro}
% \usepackage[T1]{fontenc}

\begin{document}
    % \maketitle
    %     \begin{titlepage}
    %    \begin{center}
    %        \vspace*{1cm}
    % 
    %        \textbf{EAE 103} \\
    %        Assignment 5
    % 
    %        \vspace{0.5cm}
    %         
    %             
    %        \vspace{1.5cm}
    % 
    %        \textbf{Nathan Warner}
    % 
    %        \vfill
    %             
    %             
    %        \vspace{0.8cm}
    %      
    %        \includegraphics[width=0.4\textwidth]{}
    %             
    %        Computer Science \\
    %        Northern Illinois University\\
    %        September 25, 2023 \\
    %        United States\\
    %        
    %             
    %    \end{center}
    % \end{titlepage}
    % \tableofcontents
    % \pagebreak \bigbreak \noindent
    \bigbreak \noindent 
    \textbf{1. Large atmospheric pressure can be difficult to imagine. Atmospheric pressure on the
surface of Venus is 90 bars; Earth’s atmospheric pressure at the surface is about 1 bar. How
deep in the ocean would you need to dive to experience a pressure of 90 bars? Express your
answer in both meters (m) and feet. Use the following conversions in your calculation: Every
10.2 m of seawater increases pressure by 1 bar; 1 m = 3.28 ft.}
\bigbreak \noindent 
    \pf{Solution}{
        \begin{align*}
            \text{Pressure needed from water alone}:\ 90-1 = 89\ bars
        .\end{align*} 
    \bigbreak \noindent 
    \begin{remark}
       For every 10.2 meters of seawater, the pressure increases by 1 bar.  
    \end{remark}
    \bigbreak \noindent 
    Thus:
    \begin{align*}
        10.2 \times 89 = 907.8\ m
    .\end{align*}
    \bigbreak \noindent 
    Converting to feet:
    \begin{align*}
        &907.8 \cdot 3.28\ ft/m \\
        &=2978\ ft
    .\end{align*}
    
    }
    \bigbreak \noindent 
    \textbf{2a. Explain why it’s difficult to learn about Venus from Earth-based observation alone?}
    \bigbreak \noindent 
    \pf{Solution}{
        Reasons include:
        \begin{itemize}
            \item \textbf{Thick Atmosphere}
            \item \textbf{Brightness (due to thick atmosphere)}
            \item \textbf{Distance}
            \item \textbf{Day length}: A day on Venus is longer than its year
        \end{itemize}
    }

    \bigbreak \noindent 
    \textbf{2b. What method did the Magellan spacecraft use to map Venus’ surface? }
    \bigbreak \noindent 
    \pf{Answer}{
        The Magellan spacecraft used radar to map the surface of Venus. 
    }
    \bigbreak \noindent 
    \textbf{2c. List and describe two volcanic features on Venus’ surface (labeled diagrams are welcome}
too).
    \pf{Solution}{
        \bigbreak \noindent 
   \begin{enumerate}
       \item They are very large, some of them are 100km across.
        \item The Shield volcanoes have calderas, which collapse craters due to magma withdrawal
   \end{enumerate} 
    }

    \pagebreak \bigbreak \noindent 
    \textbf{3a. List two ways Venus is similar to Earth and two ways Venus is different from Earth.}
    \bigbreak \noindent 
    \pf{Solution}{
   \bigbreak \noindent 
   \begin{enumerate}
       \item The diameter is almost identical to earth's
        \item Surface gravity is 91\% of earth's 
   \end{enumerate}
    }

    \bigbreak \noindent 
    \textbf{3b. How do we know Venus does not have mobile tectonic plates like Earth?}
    \bigbreak \noindent 
    \pf{Solution}{
        \bigbreak \noindent 
        \begin{enumerate}
            \item Venus lacks subduction zones, as evidenced by the absence of trenches.
            \item Unlike Earth, Venus doesn't have global ridge systems akin to mid-ocean ridges.
            \item Venus's surface has numerous faults and fractures. Some encircle coronae, while others in elevated regions hint at localized compression.
            \item Venus's surface has evenly distributed impact craters, indicating a uniformly aged surface. If Venus had plate tectonics, we'd see regions with varying crater densities, reflecting different ages.
        \end{enumerate}
    }

    \bigbreak \noindent 
    \textbf{3c. What evidence suggests that Venus has been “resurfaced” within the past 500 million years?} 
    \bigbreak \noindent 
    \pf{Solution}{
        \bigbreak \noindent 
        \begin{enumerate}
            \item \textbf{Crater Distribution}: Venus's craters are uniformly spread, suggesting a consistent surface age.
            \item \textbf{Lack of Ancient Surfaces}: There are no highly cratered, ancient terrains on Venus.
            \item \textbf{Volcanism}: The presence of many volcanic structures indicates extensive recent volcanic activity.
        \end{enumerate}
    }

    \bigbreak \noindent 
    \textbf{3d. How do greenhouse gases (like carbon-dioxide, water vapor and methane) affect planet surface temperature?}
    \bigbreak \noindent 
    \pf{Solution}{
        \bigbreak \noindent 
        Absorbs heat, trapping in the atmosphere
    
    }

    \pagebreak \bigbreak \noindent 
    \textbf{4a. Describe two ways Mars is similar to Earth and two ways Mars is different from Earth
today. } 
    \bigbreak \noindent 
    \pf{Solution}{
        \bigbreak \noindent 
        \begin{enumerate}
            \item \textbf{Rotation time}: 24 hours and 37 minutes
            \item \textbf{Equatorial tilt}: 25.2 degrees, leading Mars to experience seasons similarly to Earth. Presence of ice caps at the poles.
        \end{enumerate}

    }

    \bigbreak \noindent 
    \textbf{4b. How has Mars’ surface temperature changed over time?}
    \bigbreak \noindent 
    \pf{Solution}{
        \bigbreak \noindent 
        \begin{enumerate}
        \item \textbf{Early Mars}: Likely warmer with liquid water due to a thicker atmosphere.
        \item \textbf{Transition Era}: Cooling began as Mar's atmosphere thinned.
        \item \textbf{Hesperian Period}: volcanic activity occasionally warmed it.
        \item \textbf{Amazonian Period to Now}: Predominantly cold and dry with polar ice caps, indicative of its chilly climate.
        \end{enumerate}
    }

    \bigbreak \noindent 
    \textbf{4c. List two pieces of evidence that Mars’ climate has changed over time. }
    \bigbreak \noindent 
    \pf{Solution}{
        \bigbreak \noindent 
        \begin{enumerate}
            \item \textbf{Valley Networks and Ancient Lakebeds}: Geological formations resembling dried river valleys and lake basins suggest Mars once had flowing liquid water, indicating a warmer and possibly wetter climate in its past.
            \item \textbf{Polar Ice Caps}: The presence and layered structure of polar ice caps, made of water and carbon dioxide ice, hint at varying climatic conditions and atmospheric composition over time.
        \end{enumerate}
    
    }

    \bigbreak \noindent 
    \textbf{4d. How do weathering and erosion differ?}
    \bigbreak \noindent 
    \pf{Solution}{
        \bigbreak \noindent 
        Weather is the breakdown of rocks in place, there is both physical and chemical weathering. Erosion is the movement of weathered rock debris.

    
    
    }

    \pagebreak \bigbreak \noindent 
    \textbf{5a. During which period in Mars’ history was most of the atmosphere lost, leading Mars to become a cold dry planet? }
    \bigbreak \noindent 
    \pf{Solution}{
        \bigbreak \noindent 
        Most of Mars' atmosphere was lost during the Transition Era
    
    }

    \bigbreak \noindent 
    \textbf{5b. What process probably destroyed most of the atmosphere?}
    \bigbreak \noindent 
    \pf{Solution}{
        \bigbreak \noindent 
        The process that probably destroyed most of Mars' atmosphere is the \textbf{solar wind stripping.}

    
    }

    \bigbreak \noindent 
    \textbf{5c. Where is the Oxygen on Mars today?}
    \bigbreak \noindent 
    \pf{Solution}{
        \bigbreak \noindent 
        \begin{enumerate}
            \item Surface materials
            \item Atmosphere
            \item Water ice
        \end{enumerate}
    
    }

    \bigbreak \noindent 
    \textbf{5d.  How do you know where the Oxygen is on Mars today?} 
    \bigbreak \noindent 
    \pf{Solution}{
        \bigbreak \noindent 
        Our understanding of where the oxygen is on Mars comes from:
        \begin{itemize}
            \item Mars Orbiters and Probes
            \item Rovers and Landers
            \item Atmospheric Studies
        \end{itemize}
    
    }

    \bigbreak \noindent 
    \textbf{6a. From the readings, why do most scientists now think the possibility of life on the
    surface of Mars is negligible? }
    \bigbreak \noindent 
    \pf{Solution}{
        \bigbreak \noindent 
        \begin{itemize}
            \item Extreme cold
            \item Thin atmosphere
            \item High radiation
        \end{itemize}
    
    }

    \pagebreak \bigbreak \noindent 
    \textbf{6b. In your own words explain how Venus, Earth and Mars illustrate divergent planetary evolution?}
    \bigbreak \noindent 
    \pf{Solution}{
        \bigbreak \noindent 
        \begin{enumerate}
            \item Venus: Similar in size to Earth, Venus developed a thick CO₂-rich atmosphere, leading to extreme greenhouse heating. It lacks significant water and a stabilizing magnetic field, resulting in a scorching, high-pressure environment.
            \item Earth: Perfectly positioned from the Sun and with abundant water, Earth developed a balanced atmosphere and climate, supported by a strong magnetic field,  shaped its atmosphere.
            \item Mars: Once warmer with liquid water, its small size likely led to a cooled core and a diminished magnetic field. This allowed solar winds to strip its atmosphere, turning Mars into a cold, thin-aired desert.
        \end{enumerate}
    
    }




    
    
\end{document}
