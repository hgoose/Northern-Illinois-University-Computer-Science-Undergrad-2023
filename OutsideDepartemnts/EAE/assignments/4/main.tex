\documentclass{report}

\input{~/dev/latex/template/preamble.tex}
\input{~/dev/latex/template/macros.tex}

\title{\Huge{}}
\author{\huge{Nathan Warner}}
\date{\huge{}}
\pagestyle{fancy}
\fancyhf{}
\lhead{Warner \thepage}
\rhead{}
% \lhead{\leftmark}
\cfoot{\thepage}
%\setborder
% \usepackage[default]{sourcecodepro}
% \usepackage[T1]{fontenc}

\begin{document}
    % \maketitle
        \begin{titlepage}
       \begin{center}
           \vspace*{1cm}
    
           \textbf{Assignment 4}
    
           \vspace{0.5cm}
            
                
           \vspace{1.5cm}
    
           \textbf{Nathan Warner}
    
           \vfill
                
                
           \vspace{0.8cm}
         
           \includegraphics[width=0.4\textwidth]{~/niu/seal.png}
                
           Computer Science \\
           Northern Illinois University\\
           September 18, 2023 \\
           United States\\
           
                
       \end{center}
    \end{titlepage}
    \tableofcontents
    \pagebreak \bigbreak \noindent
    \textbf{1.} List at least 3 features of a complex crater – you can also draw these with proper labels.
    \bigbreak \noindent 
    \textbf{Answer.} 
    \begin{itemize}
        \item More than 20km in diameter
        \item Has a central peak
        \item Contains lava flows
    \end{itemize}

    \bigbreak \noindent 
    \textbf{2.a} Compute the age of a basalt rock sample from the Moon containing 6 grams of Uranium-235 (parent isotope) and 42 grams of Lead-207 (daughter isotope). This rock is from a complex crater flooded with lava. The half-life of Uranium-235 is 704 million years
    \bigbreak \noindent 
    \textbf{Answer.} 
    \bigbreak \noindent 
    
    \begin{enumerate}
        \item  total radio-isotope mass: 
            \begin{align*}
                 6g+42g = 48g
            .\end{align*}
        \item  decay sequence:
            \begin{itemize}
                \item One half-lives: 24g U-235, 24g L-207 
                \item Two half-lives: 12g U-235, 36g L-207 
                \item Three half-lives: 6g U-235, 42g L-207
            \end{itemize}
    \end{enumerate}
    Thus, the rock has undergone three half lifes, and the age is given by:
    \begin{align*}
        704 \cdot 3 \\
        =2.112\ \text{Billion Years}
    .\end{align*}

    \bigbreak \noindent 
    \textbf{2.b} What are the three assumptions you made in this calculation?
    \bigbreak \noindent 
    \textbf{Answer.}
    \begin{enumerate}
        \item Closed system, no parent or daughter lost or added
        \item No initial Lead-207 present when the rock was formed
        \item Consistent half-lives over time
    \end{enumerate}

    \pagebreak \bigbreak \noindent 
    \textbf{3. } Define the following: (a) lunar breccia, (b) lunar regolith, (c) jumbled (also called lineated, weird or broken) terrain, (d) relative dating of rocks, (e) multi-ringed impact basin
    \bigbreak \noindent 
    \textbf{Answer.}
    \begin{itemize}
        \item Lunar breccia: A rock on the Moon made of mixed fragments fused together, often by meteor impacts.
        \item Lunar regolith: The Moon's surface layer of loose dust and broken rock, formed by meteor impacts and radiation.
        \item Jumbeld: Chaotic landscape, most often caused by impacts or tectonic activity.
        \item Relative dating of rocks: A method to figure out the age sequence of rock layers without knowing their actual ages.
        \item Multi-Ringed Impact Basin: A large, circular crater on celestial bodies like the Moon, characterized by multiple rings of hills or mountains.
    \end{itemize}

    \bigbreak \noindent 
    \textbf{4.} What process forms a sinuous rille on the moon? (b) What process forms a nonsinuous rille on the moon (also called a graben)?
    \bigbreak \noindent 
    \textbf{Answer (4.a)}: Basalt lava channels or tubes that collapsed 
    \bigbreak \noindent 
    \textbf{Answer (4.b)}: When the lunar Mare basalts sag and compression 

    \bigbreak \noindent 
    \textbf{5.} Mercury looks superficially like the Moon -- a cratered surface with no atmosphere. Appearances can be deceiving, however. List at least two major differences between the Moon and Mercury.
    \bigbreak \noindent 
    \textbf{Answer.}
    \begin{itemize}
        \item Magnetic Field: The moon has no magnetic field, but mercury does.
        \item Composition and Density: Mercury is much denser than the Moon. Mercury has a large iron core. 
    \end{itemize}

    \bigbreak \noindent 
    \textbf{6.a:} Note three specific examples of experiments or activities the Apollo 16 crew performed that greatly expanded upon activities carried out during Apollo 11, or were new experiments. 
    \bigbreak \noindent 
    \textbf{Answer.}
    \begin{itemize}
        \item Lunar Roving Vehicle Usage
        \item Ultraviolet Camera/Spectrograph: This experiment was designed to take pictures of the Earth's upper sky and its magnetic field, along with capturing images of the gas between stars and the glowing rings that appear around galaxies.
        \item Extensive Geological Exploration and Sampling: The astronauts collected samples from large rocks to soil particles and conducted experiments to measure the local magnetic field and the resistance of the soil to compaction.
    \end{itemize}

    \bigbreak \noindent 
    \textbf{6.b:} The Apollo 16 moon landing was almost cancelled while in orbit around the Moon. What was the problem that almost cancelled the landing?
    \bigbreak \noindent 
    \textbf{Answer.} The problem was with the backup control system of the main engine in the command module. The system showed uncontrolled oscillations, which raised concerns about the engine's reliability.





\end{document}
