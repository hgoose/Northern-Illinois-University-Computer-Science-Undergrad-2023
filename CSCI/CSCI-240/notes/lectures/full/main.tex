\documentclass{report}

\input{~/dev/latex/template/preamble.tex}
\input{~/dev/latex/template/macros.tex}

\title{\Huge{}}
\author{\huge{Nathan Warner}}
\date{\huge{}}
\pagestyle{fancy}
\fancyhf{}
\lhead{Warner \thepage}
\rhead{}
% \lhead{\leftmark}
\cfoot{\thepage}
% \setborder
% \usepackage[default]{sourcecodepro}
% \usepackage[T1]{fontenc}

\begin{document}
    % \maketitle
        \begin{titlepage}
       \begin{center}
           \vspace*{1cm}
    
           \textbf{NIU CS240} \\
           Computer Programming In CPP
    
           \vspace{0.5cm}
            
                
           \vspace{1.5cm}
             
           \textbf{Nathan Warner}
    
           \vfill
                
                
           \vspace{0.8cm}
         
           \includegraphics[width=0.4\textwidth]{~/niu/seal.png}
                
           Computer Science \\
           Northern Illinois University\\
           August 28, 2023
           United States\\
           
                
       \end{center}
    \end{titlepage}
    \tableofcontent 
    \pagebreak \bigbreak \noindent 
    \vspace{2in} \\
    \begin{Huge}
       \textbf{Computer Programming \\
       In CPP} 
    \end{Huge}
    \bigbreak \noindent 
    \line(1,0){490}
    
    \bigbreak \noindent \bigbreak \noindent 
    \section{Basic Language Features}
    \bigbreak \noindent 
    \subsection{Data}
    \begin{itemize}
        \item There are several data types (numbers, characters, etc)
        \item individual data items must be declared and named - this is known as creating a variable
        \item values that are put into variables can come from
            \begin{itemize}
                \item program instructions 
                \item user input
                \item files
            \end{itemize}
        \item program instructions can alter these values
        \item original or newly computer values can go to
            \begin{itemize}
                \item screen
                \item printer
                \item disk
            \end{itemize}
    \end{itemize}
    \bigbreak \noindent 
    \textbf{Instructions:}
    \begin{itemize}
        \item for data input (from keyboard, disk)
        \item for data output (to screen, printer, disk)
        \item computation of new values
        \item program control (decisions, repetition)
        \item modularization (putting a sequence of instructions into a package called a function)
    \end{itemize}

    \bigbreak \noindent \bigbreak \noindent 
    \subsection{The Language}
    \bigbreak \noindent 
    The C++ language is made up of 
    \begin{itemize}
        \item keywords/reserved words (if, while, int, etc.)
        \item symbols: \{ \} =  |  <=  ! [ ]  *  \&  (and more)
        \item programmer-defined names for variables and functions
    \end{itemize}
    \bigbreak \noindent 
    These programmer-defined names:
    \begin{itemize}
        \item  1 - 31 chars long; use letters, digits, \_ (underscore)
        \item start with a letter or \_
        \item are case-sensitive: \textit{Num} is \textbf{different} than \textit{num}
        \item should be meaningful: \textit{studentCount} is better than s or sc
    \end{itemize}

    \pagebreak \bigbreak \noindent 
    \section{\LARGE Primitive Data Types}
    \bigbreak \noindent 

    \subsection{Data Types}
    \bigbreak \noindent 
    Each data item has a type and a name chosen by the programmer. The type determines the range of possible values it can hold as well as the operations that can be used on it. For example, you can add a number to a numeric data type, but you cannot add a number to someone's name. (What would "Joe" + 1 mean?)
    \bigbreak \noindent 
    \textit{Figure:}
    \begin{center}
    \begin{tabular}{|c|c|}
        \hline
        Type & Keyword \\
        \hline
        Boolean & bool \\
        Character & char \\
        String & string \\
        Integer & int \\
        Floating point & float \\
        Double floating point & double \\
        Valueless & void \\
        % Wide character & wchar\_t \\
        \hline
    \end{tabular}
        \end{center}
        \bigbreak \noindent 
        \nt{Floating point numbers have a limit of 6 significant figures and doubles have a limit of 12 characters.}
        \bigbreak \noindent 

        \bigbreak \noindent \bigbreak \noindent 
        \subsection{Integers}
        \bigbreak \noindent 
        \begin{itemize}
            \item no decimal point, comma, or \$ 
            \item leading + - allowed
            \item range (on our system): $\pm$ 2 billion
            \item int constants are written as: $1234$ $-3$ $43$
        \end{itemize}

        \bigbreak \noindent 
        \subsection{Integer declaration}
        \bigbreak \noindent 
\line(1,0){490}
        \begin{minted}{cpp}
int x; // gives type and name; no value yet
int x,y; //declares 2 ints; no values yet
int population = 160000; //declares & sets the initial value
        \end{minted}
\line(1,0){490}
        
        \bigbreak \noindent \bigbreak \noindent 
        \nt{It is critically importannt that variables have values before they are used in a program.}
        \bigbreak \noindent 

        \bigbreak \noindent \bigbreak \noindent 
        \subsection{Float and Double}
        \bigbreak \noindent 
        These data types are also commonly referred to as \textbf{real} variables  (following mathematics usage) when it's not important which one we mean and we don't want to always have to say "float or double"
        \begin{itemize}
            \item has decimal point
            \item leading $\pm$ allowed
            \item no comma, \$
            \item range (float) $\pm$ 10 to $38^{th}$ (limited to 66 sig figs)
            \item range (double) $\pm$ 10 to $308^{th}$ (limited to 12 sig figs)
        \end{itemize}
        
        
        \bigbreak \noindent 
        \subsection{float and double declaration}
        \bigbreak \noindent 
\line(1,0){490}
        \begin{minted}{cpp}
float  pi = 3.1416;   //declares, names, sets init value

double x = 3.5,	      //note comma here
       y = 1245.6543; //can use > 6 digits

OR

double x   = 3.5;
double y   = 1245.6543;

float  big = 5.28e3;  // exponential notation: 5280
        \end{minted}
\line(1,0){490}


        \bigbreak \noindent \bigbreak \noindent 
        \subsection{Char}
        \bigbreak \noindent 
        \begin{itemize}
            \item can hold just one character
            \item char constants are written with single quotes 'a'
        \end{itemize}

        \bigbreak \noindent 
        \subsection{Char declarations}
        \bigbreak \noindent 
    \line(1,0){490}
    \begin{minted}{cpp}
    char ch;
    char choice = 'q';
    \end{minted}
    \line(1,0){490}

    \bigbreak \noindent 
    \subsection{String}
    \bigbreak \noindent 
    \begin{itemize}
        \item can hold 0 to many characters
        \item string constants are written with double quotes:  "Hello, world"
    \end{itemize}

    \bigbreak \noindent 
    \subsection{String declarations}
    \bigbreak \noindent 
    \line(1,0){490}
    \begin{minted}{cpp}
string s;
string MyName = "Amy';
    \end{minted}
    \line(1,0){490}


        \pagebreak \bigbreak \noindent 
        \section{\LARGE Arithemetic Operators}
        \bigbreak \noindent 
        The arithmetic operators are:
        \bigbreak \noindent 
        \begin{itemize}
            \item +  addition
            \item -  subtraction or unary negation (-5)
            \item *  multiplication
            \item /  division (see special notes on division below)
            \item \%  modulus division \texttt{--} integer remainder of integer division
        \end{itemize}
        \bigbreak \noindent 
        \nt{There is no exponential operator}
        \bigbreak \noindent 
        In C++, a division with 2 int operands has an int resulting value, But with 1 or 2 float/double operands, the resulting value is a float or double. Be aware. Forgetting this can easily cause an error.
        \bigbreak \noindent \bigbreak \noindent 
        \subsection{Arithmetic Expressions}
        \textbf{Arithmetic Expressions} are formed by operands and operators. Operands are the values used, operators are the things done to the numbers. Parts of arithmetic expressions are often grouped by () for clarity or to affect the meaning of the expression:
        \bigbreak \noindent 
        \begin{center}
        % \scalebox{1.25}{
         \begin{tabular}{|c|c|c|}
                \hline
                Expression & Value & Notes \\
                \hline
                \( x + y \) & 13 & \\
                \( x \times y \) & 22 & \\
                \( x \times y + x \) & 33 & \\
                \( x - y \) & 9 & \\
                \( -x + y \) & -9 & Unary negation: ``minus \( x \)'' \\
                \( \frac{x}{y} \) & 5 & int since both ops are int \\
                \( x \% y \) & 1 & rem when \( x \) divided by \( y \) \\
                \( \frac{x}{\text{realnum}} \) & 5.5 & one op is real so result is real  \\
                \hline
            \end{tabular}
        % }
        \end{center}
        \bigbreak \noindent 
        \nt{Note: The expressions above by themselves are not valid C++ statements - they would be part of a statement. But each expression (if it is in a valid C++ statement) has a value. Also - the spaces written between operands and operators are not required by C++, but do improve legibility.}
        \bigbreak \noindent 
        More complex expressions are evaluated by C++ using Rules of Precedence (like algebra)
        \begin{enumerate}
            \item sub-expressions in ()
            \item unary -
            \item * and / - left to right
            \item + and - - left to right
        \end{enumerate}

        \pagebreak \bigbreak \noindent 
        \section{\LARGE Assignment Statement/Operation}
        \bigbreak \noindent 
        The symbol "=" is a \textbf{command}. It means: "evaluate the \textbf{expression} on the right and store this \textbf{value} in the \textbf{variable} on the left"
        \bigbreak \noindent 
        \textbf{General Syntax:}
        \bigbreak \noindent 
        \line(1,0){490}
        \begin{minted}{cpp}
Identifier = Expression;
        \end{minted}
        \line(1,0){490}
        \bigbreak \noindent 
        \textbf{Examples:}
        \bigbreak \noindent 
        \line(1,0){490}
        \begin{minted}{cpp}
x = y * 3;
x = x + y;
        \end{minted}
        \line(1,0){490}
        \bigbreak \noindent 
        \nt{Notice that there is always exactly one variable on the left to receive the value of the expression on the right.         This is \textbf{NOT} like algebra; it is \textbf{NOT} an equation. All variables in expressions on the right must have defined values or you get random results. This is an example of what I meant before when I said variables must have values before they are used.}
        \bigbreak \noindent \bigbreak \noindent 
        \subsection{Multiple assignments}
        You can do multiple assignments on one line:
        \bigbreak \noindent 
        \line(1,0){490}
        \begin{minted}{cpp}
x = y = z = 3;  //all get value 3
        \end{minted}
        \line(1,0){490}

        \pagebreak \bigbreak \noindent 
        \subsection{Naming Variables}
        \bigbreak \noindent 
        Use meaningful names that explain themselves to reader of the program. (You, me, the maintainer of the program after you go to your next job...)





















    % \pagebreak \bigbreak \noindent 
    % \section{Assignment Statements; Control Structures; Symbolic Constants; Formatting Output}
    % \pagebreak \bigbreak \noidennt
    %
    % \section{Cascading Ifs; Conditional Expressions; Compound Conditions; Data Type \texttt{bool}}
    %
    %
    % \pagebreak \bigbreak \noidennt
    % \section{Functions}
    % \pagebreak \bigbreak \noidennt
    %
    % \section{Function Summary Sheet}
    % \pagebreak \bigbreak \noidennt
    %
    % \section{Character Functions}
    % \pagebreak \bigbreak \noidennt
    %
    % \section{Arrays}
    % \pagebreak \bigbreak \noidennt
    %
    % \section{Arrays and Functions}
    % \pagebreak \bigbreak \noidennt
    %
    % \section{References and Call By Reference; Input and Output}
    % \pagebreak \bigbreak \noidennt
    %
    % \section{Object Oriented Programming}
    % \pagebreak \bigbreak \noidennt
    %
    % \section{C++ Strings}
    % \pagebreak \bigbreak \noidennt
    %
    % \section{Structures}
    % \pagebreak \bigbreak \noindent 

    
\end{document}
