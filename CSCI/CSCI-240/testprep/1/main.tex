\documentclass{report}

\input{~/dev/latex/template/preamble.tex}
\input{~/dev/latex/template/macros.tex}

\title{\Huge{}}
\author{\huge{Nathan Warner}}
\date{\huge{}}
\pagestyle{fancy}
\fancyhf{}
\lhead{Warner \thepage}
\rhead{}
% \lhead{\leftmark}
\cfoot{\thepage}
%\setborder
% \usepackage[default]{sourcecodepro}
% \usepackage[T1]{fontenc}

\begin{document}
    % \maketitle
        \begin{titlepage}
       \begin{center}
           \vspace*{1cm}
    
           \textbf{CSCI-240} \\
           Exam I: Test Prep
    
           \vspace{0.5cm}
            
                
           \vspace{1.5cm}
    
           \textbf{Nathan Warner}
    
           \vfill
                
                
           \vspace{0.8cm}
         
           \includegraphics[width=0.4\textwidth]{}
                
           Computer Science \\
           Northern Illinois University\\
           October 1, 2023 \\
           United States\\
           
                
       \end{center}
    \end{titlepage}
    \tableofcontents
    \pagebreak \bigbreak \noindent
    \section{\LARGE Quiz I}
    \bigbreak \noindent 
    \begin{enumerate}
    \item (2 points) Most lines in a C++ program end with a
    \begin{itemize}
        \item ; (semi-colon)
    \end{itemize}
    
    \item (2 points) \texttt{main()} marks the beginning of a C++ program. What C++ reserved word precedes it?
    \begin{itemize}
        \item \texttt{int}
    \end{itemize}
    
    \item (2 points) What is the correct way to declare an integer variable named "score"?
    \begin{itemize}
        \item \texttt{int score;}
    \end{itemize}
    
    \item (2 points) Given the following declaration:
    \begin{verbatim}
    double x;
    \end{verbatim}
    What is the value of x?
    \begin{itemize}
        \item Garbage
    \end{itemize}
    
    \item (2 points) According to the lecture notes, the two main conceptual components of a program are \underline{\hspace{2cm}} and \underline{\hspace{2cm}}.
        \begin{itemize}
            \item Data + instructions
        \end{itemize}
    
    \item (2 points) Given the following:
    \begin{verbatim}
    double price = 30.00;
    double tax = 1.80;
    double sum;

    sum = price + tax;
    \end{verbatim}
    Explain in detail what the last line does in terms of variables, calculations, and assignment of values.
    \begin{itemize}
        \item Assigns the sum of the two double variables \textit{price} and \textit{tax} to the variable \textit{sum}. The computation would be $30.00 + 1.80 = 31.8$
    \end{itemize}
    
    \item (2 points) When you write an illegal C++ statement and try to compile and run the program, you will get a
    \begin{itemize}
        \item compile error
    \end{itemize}
    
    \item (2 points) What is the value (in C++) of the expression \(4 / 2 \times 2\)?
        \begin{itemize}
            \item $4$
        \end{itemize}
    
    \item (2 points) What is the value (in C++) of the expression \(3/2\)?
        \begin{itemize}
            \item 1
        \end{itemize}
    
    \item (2 points) Which data type has the largest range?
    \begin{itemize}
        \item \texttt{double}
    \end{itemize}
    
    \item (2 points) About how many decimal places of accuracy does a \texttt{float} have?
    \begin{itemize}
        \item 6
    \end{itemize}
    
    \item (2 points) Modify (rewrite) the following instruction so that the subtraction is evaluated first:
    \begin{verbatim}
    i = a * b / (c - d);
    \end{verbatim}
\end{enumerate}

    \pagebreak \bigbreak \noindent 
    \section{\LARGE Quiz II}
    \bigbreak \noindent 
    \begin{enumerate}
    \item (2 points) What instruction will display data on the screen from a C++ program?
    \begin{itemize}
        \item \texttt{cout}
    \end{itemize}
    
    \item (2 points) About how many decimal places of accuracy does a \texttt{float} have?
    \begin{itemize}
        \item 6
    \end{itemize}
    
    \item (2 points) The formula for converting a Fahrenheit temperature to Centigrade is \(5/9(F - 32)\). What is wrong with writing it in C++ as
    \begin{verbatim}
    C = 5/9 * (F - 32);
    \end{verbatim}
    assuming that C and F are both declared as doubles, and F has a valid value.
    \begin{itemize}
        \item Integer division
    \end{itemize}
    
    \item (2 points) What is the value of the expression \(25 \% 3\)?
    \begin{itemize}
        \item 1
    \end{itemize}
    
    \item (2 points) Explain in detail what the following instruction does (assuming \texttt{i} is declared as \texttt{int}):
    \begin{verbatim}
    cin >> i;
    \end{verbatim}
    \bigbreak \noindent 
    \textbf{Answer:} The instruction \texttt{cin >> i;} is used for input in C++. Specifically, it waits for the user to enter a value from the keyboard. Once the user enters a value and presses the Enter key, the entered value is stored in the variable \texttt{i}. It's important to note that since \texttt{i} is declared as an \texttt{int}, the user is expected to enter an integer value. If the user enters a non-integer value, the behavior might be unpredictable or result in an error.

    
    \item (2 points) Suppose you have two integer variables (named \texttt{num} and \texttt{sum}) with valid values. Write a single \texttt{cout} instruction to display them as follows:
    \begin{verbatim}
    num is __
    sum is __
    \end{verbatim}
    the underscore characters will show the actual values in \texttt{num} and \texttt{sum} - for example:
    \begin{verbatim}
    num is 4
    sum is 24
    \end{verbatim}
    \begin{verbatim}
std::cout << "num is " << num 
    << std::endl
    << "sum is " << sum;
        
    \end{verbatim}
    
    
    \item (2 points) Name two libraries that should be \texttt{\#include'd} at the top of a C++ program.
        \begin{itemize}
            \item iostream
            \item iomanip
        \end{itemize}
    
    \item (2 points) Assuming that two floating point numbers have been saved in the variables \texttt{num1} and \texttt{num2}, write a chunk of program code that will display the integer result of the sum of \texttt{num1} and \texttt{num2}. For example, if \texttt{num1} contains 2.4 and \texttt{num2} contains 2.5, then the value 4 should be displayed.
        \begin{verbatim}
int sum = num1 + num2;
cout << sum;
        \end{verbatim}
        
    
    \item (2 points) Write a chunk of program code that asks the user to enter two floating point numbers and saves the values in two \texttt{float} variables called \texttt{val1} and \texttt{val2}.
    \begin{verbatim}
float num1, num2;
cout << "Enter two floating point numbers";
cin >> num1 >> num2;
    \end{verbatim}
    \end{enumerate}

    \pagebreak \bigbreak \noindent 
    \section{\LARGE Quiz 3}
    \bigbreak \noindent 
    \begin{enumerate}
    \item (2 points) Which of the following increments x by 1?
    \begin{itemize}
        \item \(x += 1\);
    \end{itemize}
    
    \item (2 points) Select the three control structures that (along with sequence) will be studied in this course.
    \begin{itemize}
        \item decision
        \item repetition/looping
        \item branch and return/function calling
    \end{itemize}
    
    \item (2 points) Name one command that is used to implement the decision statement control structure that will be studied in this course.
        \begin{itemize}
            \item if
        \end{itemize}
    
    \item (2 points) Name the 3 C++ statements used to create a loop.
        \begin{itemize}
            \item for 
            \item while
            \item do while
        \end{itemize}
    
    \item (2 points) What will the following code display on the screen and where will it display?
    \begin{verbatim}
for (i = 0; i < 5; i++)
  cout << "\n";
  cout << i;
    \end{verbatim}
    \begin{itemize}
        \item Assuming these incompetent morons can write a valid for loop, this will display numbers 0-4, each on a separate line
    \end{itemize}
    
    \item (2 points) Write a for loop to display the first 5 multiples of 10 on one line. For example: 10 20 30 40 50
        \begin{verbatim}
for (int i = 10; i <= 50; i+=10) {
    cout << i << " ";
} 
        \end{verbatim}
        
    
    \item (2 points) Write a while loop to display the first 5 multiples of 10 on one line. For example: 10 20 30 40 50
    \begin{verbatim}
int control = 10;
while (control <= 50) {
    cout << control << " ";
    control+=10;
}
    \end{verbatim}
    
    
    \item (2 points) When is the 3rd subexpression in \texttt{for (--; --; --)} statement executed?
        \begin{itemize}
            \item at the end of each iteration 
        \end{itemize}
    
    \item (2 points) Write a decision statement to test if a number is even or not. If it is, print "even". If it is not, add 1 to it and print "it was odd, but now it's not".
        \begin{verbatim}
if (num % 2 == 0) {
    cout << "even" << endl;
} else {
    cout << "It was odd, but now its not";
    num+=1;
}
        \end{verbatim}
        \begin{itemize}
    
    \item (2 points) Why is a while loop described as "top-driven"?
            \item Because the condition is checked before each iteration begins
        \end{itemize}
    
    \item (2 points) If a read-loop is written to process an unknown number of values using the while construct, and if there is one read before the while instruction there will also be one
    \begin{itemize}
        \item at the bottom of the body of the loop
    \end{itemize}
    \end{enumerate}

    \pagebreak \bigbreak \noindent 
    \section{\LARGE Quiz 4}
    \bigbreak \noindent 
    \begin{enumerate}
    \item (2 points) The three basic flow-of-control patterns are sequence, _\_\_\_\_\_\_\_\_\_ and \_\_\_\_\_\_\_\_\_\_.
        \begin{itemize}
            \item decision
            \item repetition
        \end{itemize}
    
    \item (2 points) The three basic loop statements in C++ are while, \_\_\_\_\_\_\_\_\_\_, and \_\_\_\_\_\_\_\_\_\_.
        \begin{itemize}
            \item do while
            \item for
        \end{itemize}
    
    \item (2 points) Write a fragment of code that checks a string variable to see if it contains your first name or not, and then prints either "that's me" or "that's not me" accordingly. Assume that the string variable \texttt{aName} has a valid string in it before the test is made.
    \begin{verbatim}
if (aName == "nate") cout << "Thats me" << endl;
else cout << "Thats not me" << endl;
    \end{verbatim}

    
    \item (2 points) Write a fragment of code that accepts integers from the user until a negative number is entered. The prompt should be "Enter a number (negative to quit):" Add up the numbers and print the sum (not including the negative number). Assume and use the following declarations:
    \begin{verbatim}
int sum = 0;
int num;

cout << "Enter a positive number (negative number to quit): ";
cin >> num;

while (num >= 0) {
    sum += num;
    cout << "Enter a positive number (negative number to quit): ";
    cin >> num;
}
cout << sum;
    \end{verbatim}
    
    \item (2 points) Given the variables \texttt{top}, \texttt{left}, \texttt{right}, and \texttt{bottom} representing the coordinates of a rectangle's upper left and lower right corners, and variables \texttt{ptX} and \texttt{ptY}, representing the coordinates of a point, write a condition to test if the point is outside the rectangle. Assume that x increases to the right and that y increases to the top.
    \begin{verbatim}
if ( (ptX < left || ptX > right) || (ptY > top || ptY < bottom) ) cout << "Your outside the rectangle";
else cout << "You're inside the rectangle";
    \end{verbatim}
    
    
    \item (2 points) Write a for loop that does exactly the same thing as the following code:
    \begin{verbatim}
i = 1;
while (i < 11)
  {
  cout << "the square root of " << i << " is " << sqrt(i) << endl;
  i += 2;
}
for (i=1; i < 11; i+=2) {
    cout << "the square root of " << i << " is " << sqrt(i) << endl;
}

    \end{verbatim}
    
    \item (2 points) Create a symbolic constant named \texttt{PI} to represent the value of pi (3.14) in two different ways.
\begin{verbatim}
#define PI 3.14 
const double PI{3.14};
\end{verbatim}

    
    \item (2 points) Give two reasons why symbolic constants might be used in a C++ program.
    \begin{itemize}
        \item if we have a fixed value in multiple places in our program, this way if we need to change the value we only need to change it in one place.
        \item to give a identifier for the values, for example all 3.14s in a program can be named PI for better readability
    \end{itemize}
    
    \item (2 points) What will the following code fragment print?
    \begin{verbatim}
    for (i = 0; i < 5; i++);
        cout << i + 1;
    cout << i;
    \end{verbatim}
    \begin{itemize}
        \item 10
    \end{itemize}
    
    \item (2 points) Suppose
    \begin{verbatim}
    x = 0;
    y = 5;
    \end{verbatim}
    Is the following condition true or false?
    \begin{verbatim}
    if (x > 0 && x < 10 || y == 4)
    \end{verbatim}
    \begin{itemize}
        \item False
    \end{itemize}
    
    \item (2 points) Which of the following is a legal way to create a C-style symbolic constant?
    \begin{itemize}
        \item \#define MAX\_VAL 30
    \end{itemize}
    \end{enumerate}




    
\end{document}
