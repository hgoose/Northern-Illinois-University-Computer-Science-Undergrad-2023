\documentclass{report}

\input{~/dev/latex/template/preamble.tex}
\input{~/dev/latex/template/macros.tex}

\title{\Huge{}}
\author{\huge{Nathan Warner}}
\date{\huge{}}
\fancyhf{}
\rhead{}
\fancyhead[R]{\itshape Warner} % Left header: Section name
\fancyhead[L]{\itshape\leftmark}  % Right header: Page number
\cfoot{\thepage}
\renewcommand{\headrulewidth}{0pt} % Optional: Removes the header line
%\pagestyle{fancy}
%\fancyhf{}
%\lhead{Warner \thepage}
%\rhead{}
% \lhead{\leftmark}
%\cfoot{\thepage}
%\setborder
% \usepackage[default]{sourcecodepro}
% \usepackage[T1]{fontenc}

% Change the title
\hypersetup{
    pdftitle={Unix and Network Programming}
}

\begin{document}
    % \maketitle
        \begin{titlepage}
       \begin{center}
           \vspace*{1cm}
    
           \textbf{Unix and Network Programming} \\
           With Linux and C++
    
           \vspace{0.5cm}
            
                
           \vspace{1.5cm}
    
           \textbf{Nathan Warner}
    
           \vfill
                
                
           \vspace{0.8cm}
         
           \includegraphics[width=0.4\textwidth]{~/niu/seal.png}
                
           Computer Science \\
           Northern Illinois University\\
           February 16, 2023 \\
           United States\\
           
                
       \end{center}
    \end{titlepage}
    \tableofcontents
    \pagebreak 
    \unsect{Commands}
    \begin{itemize}
        \item \textbf{more, less, pg}: Display contents of file one page at a time
        \item \textbf{head}: Display beginning  portion of file (Default: 10 lines)
        \item \textbf{tail}: Display end portion of file 
        \item \textbf{wc}: Count file content (-l -w -c)  (lines, words, characters)
        \item \textbf{diff}: Compare two files line by line
        \item \textbf{gzip, gunzip, zcat}: compress file content (.gz files)
        \item \textbf{sort}: Sort file contents (-r -n -t -k -f) (reverse, numeric, field delimiter, field1[,field2], ignore case)
        \item \textbf{quota -v}: Disk quota
        \item \textbf{lpr}: Send files to printer, -P to specify printer (lpcsl, lpfrl, etc)
        \item \textbf{lpq}: Show print queue
        \item \textbf{lprm}: Remove job from print queue
    \end{itemize}



    \pagebreak 
    \unsect{Permissions}
    \bigbreak \noindent 
    Unix uses discretionary access control (DAC) model
    \begin{itemize}
        \item Each directory/file has owner
        \item Owner has discretion over access control details
    \end{itemize}
    With the exception of the super user

    \bigbreak \noindent 
    \subsection{Changing Permissions}
    \bigbreak \noindent 
    There are four categories regarding permissions 
    \begin{itemize}
        \item User 
        \item Group
        \item Other
        \item All
    \end{itemize}
    \bigbreak \noindent 
    To change the permissions of a file, we use the \texttt{chmod} command
    \bigbreak \noindent 
    \begin{bashcode}
    chmod -options mode file/directory
    \end{bashcode}
    \bigbreak \noindent 
    \subsubsection{Changing Permissions: Symbolic mode}
    \bigbreak \noindent 
    \begin{minipage}[]{0.47\textwidth}
        \fig{0.5}{./figures/1.png}
    \end{minipage}
    \hspace{.1in}
    \begin{minipage}[]{0.47\textwidth}
        \fig{0.5}{./figures/2.png}
    \end{minipage}

    

    \pagebreak 
    \subsubsection{Changing permissions: Octal mode}
    \fig{.8}{./figures/3.png}

    \bigbreak \noindent 
    \subsubsection{Exercise: Changing permissions}
    \bigbreak \noindent 
    Suppose we want to change the permissions of "myfile". We want
    \begin{itemize}
        \item Read, write, and execute for user 
        \item Read and execute for group
        \item Execute for other
    \end{itemize}
    \bigbreak \noindent 
    \begin{bashcode}
    chmod u=rwx, g=rx, o=x myfile
    chmod 751 myfile
    \end{bashcode}

    \bigbreak \noindent 
    \subsection{Special Permissions}
    \bigbreak \noindent 
    3 additional permissions can be set on files and directories
    \begin{itemize}
        \item Set user ID (SUID)
        \item Set group ID (SGID)
        \item Sticky bit
    \end{itemize}

    \bigbreak \noindent 
    \subsubsection{Set user ID (SUID)}
    \bigbreak \noindent 
    \begin{concept}
       SUID is used for executable files, it makes executables run with permissions of file owner, rather than invoker 
    \end{concept}
    \bigbreak \noindent 
    For example, the passwd command uses this permission. This allows user access to otherwise protected system files while changing password

    \pagebreak 
    \subsubsection{Set group ID (SGID)}
    \bigbreak \noindent 
    \begin{concept}
        \begin{itemize}
            \item \textbf{For executables}: The logic for SGID is the same as SUID, but for group owner rather than file owner
            \item \textbf{For directories}: A file created in the directory will be owned by the group owner of the directory, not the group of the user who created the file
        \end{itemize}
    \end{concept}

    \bigbreak \noindent 
    \subsubsection{Sticky Bit}
    \bigbreak \noindent 
    \begin{concept}
       \begin{itemize}
           \item \textbf{For executables}: Executable is kept in memory even after it ended
            \item \textbf{For directories}: Files can only be deleted by the user that created it 
       \end{itemize} 
    \end{concept}

    \bigbreak \noindent 
    \subsubsection{Displaying special permissions}
    \bigbreak \noindent 
    \begin{concept}
        The \texttt{ls -l}  command does not display special permission bits. However, since special permissions require execute, they mask the execute permission when displayed with \texttt{ls -l}
    \end{concept}

    \bigbreak \noindent 
    \subsubsection{Setting special permissions (octal)}
    \bigbreak \noindent 
    \fig{.8}{./figures/4.png}

    \pagebreak 
    \subsubsection{Setting special permissions (Symbolic)}
    \bigbreak \noindent 
    \fig{.8}{./figures/5.png}

    \bigbreak \noindent 
    \subsection{User mask (umask)}
    \bigbreak \noindent 
    \fig{.8}{./figures/6.png}

    \pagebreak 
    \subsubsection{Examples}
    \bigbreak \noindent 
    \fig{.8}{./figures/7.png}

    \pagebreak 
    \unsect{Network Utilitys}

    \begin{itemize}
        \item Login to another computer 
            \begin{itemize}
                \item telnet, rlogin, rsh, ssh
            \end{itemize}
        \item Copy files to another computer 
            \begin{itemize}
                \item scp
                \item ftp, sftp
            \end{itemize}
    \end{itemize}

    \bigbreak \noindent 
    \subsection{Login to another computer}
    \bigbreak \noindent 
    \begin{itemize}
        \item telnet rlogin, rsh no longer used
            \begin{itemize}
                \item Transmit username/password without encryption
            \end{itemize}
        \item ssh
            \begin{itemize}
                \item Invokes shell on remote computer securely
                \item \textbf{Used to:} Remote login and run command on remote computer
            \end{itemize}
    \end{itemize}

    \bigbreak \noindent 
    \subsection{ssh}

    \bigbreak \noindent 
    \subsubsection{Syntax}
    \begin{bashcode}
        ssh [user@]hostname [command]
    \end{bashcode}
    \bigbreak \noindent 
    This command logs in user to hostname, or if command is given, runs it on remote host 

    \bigbreak \noindent 
    \subsubsection{Common options}
    \bigbreak \noindent 
    \begin{itemize}
        \item \textbf{-l}: login-name 
        \item \textbf{-X}: enable X11 forwarding
    \end{itemize}

    \bigbreak \noindent 
    \subsubsection{Examples}
    \bigbreak \noindent 
    \fig{1}{./figures/8.png}

    \pagebreak 
    \subsection{Copy files to another computer}

    \bigbreak \noindent 
    \subsubsection{Currently in use}
    \begin{itemize}
        \item ftp
    \end{itemize}

    \bigbreak \noindent 
    \subsubsection{Secure, encrypted, part of OpenSSH}
    \bigbreak \noindent 
    \begin{itemize}
        \item \textbf{sftp}: Secure file transfer
        \item \textbf{scp}: Secure copy to remote host
    \end{itemize}

    \bigbreak \noindent 
    \subsection{ftp}
    \bigbreak \noindent 
    \subsubsection{Syntax}
    \bigbreak \noindent 
    \begin{bashcode}
        ftp hostname
    \end{bashcode}
    \bigbreak \noindent 
    This will prompt for userid and password

    \bigbreak \noindent 
    \subsubsection{Anonymous ftp}
    \bigbreak \noindent 
    \begin{itemize}
        \item \textbf{Userid}: ftp or anonymous
        \item \textbf{Password}: Your email address
    \end{itemize}

    \bigbreak \noindent 
    \subsubsection{Commands}
    \bigbreak \noindent 
    \begin{itemize}
        \item \textbf{help}
        \item \textbf{ls}
        \item \textbf{cd} 
        \item \textbf{put, get}
            \begin{itemize}
                \item copy a file from local to remote host, or vice versa 
            \end{itemize}
        \item \textbf{mput, mget}
            \begin{itemize}
                \item put/get multiple files, can use wildcards
            \end{itemize}
        \item \textbf{bye}
    \end{itemize}

    \pagebreak 
    \subsection{sftp (Secure file transfer)}

    \bigbreak \noindent 
    \subsubsection{Syntax}
    \bigbreak \noindent 
    \begin{bashcode}
    sftp user@hostname
    \end{bashcode}
    \bigbreak \noindent 
    \begin{itemize}
        \item Will prompt for password
        \item Same commands as ftp
    \end{itemize}

    \bigbreak \noindent 
    \subsection{scp}
    \bigbreak \noindent 
    \subsubsection{Syntax}
    \bigbreak \noindent 
    \begin{bashcode}
        scp source target
    \end{bashcode}
    \begin{itemize}
        \item source and target use extended form of pathname
            \bigbreak \noindent 
            \begin{bashcode}
            user@host:pathname
            \end{bashcode}
    \end{itemize}

    \bigbreak \noindent 
    \subsubsection{Common options}
    \begin{itemize}
        \item \textbf{-r}: Recursively copy entire directories
        \item \textbf{-C}: Enables compression
        \item \textbf{-l}: Limit bandwidth, specified in Kbit/s
    \end{itemize}

    \bigbreak \noindent 
    \subsubsection{Examples}
    \bigbreak \noindent 
    \begin{bashcode}
    scp screenshot.png z123456@turing.cs.niu.edu:
    scp z123456@hopper.cs.niu.edu:assign1.cc .
    \end{bashcode}



    
\end{document}
