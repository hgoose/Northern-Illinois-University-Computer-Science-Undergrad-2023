\documentclass{report}

\input{~/dev/latex/template/preamble.tex}
\input{~/dev/latex/template/macros.tex}

\title{\Huge{}}
\author{\huge{Nathan Warner}}
\date{\huge{}}
\fancyhf{}
\rhead{}
\fancyhead[R]{\itshape Warner} % Left header: Section name
\fancyhead[L]{\itshape\leftmark}  % Right header: Page number
\cfoot{\thepage}
\renewcommand{\headrulewidth}{0pt} % Optional: Removes the header line
%\pagestyle{fancy}
%\fancyhf{}
%\lhead{Warner \thepage}
%\rhead{}
% \lhead{\leftmark}
%\cfoot{\thepage}
%\setborder
% \usepackage[default]{sourcecodepro}
% \usepackage[T1]{fontenc}

% Change the title
\hypersetup{
    pdftitle={Qt Object Methods}
}

\begin{document}
    % \maketitle
        \begin{titlepage}
       \begin{center}
           \vspace*{1cm}
    
           \textbf{QT Methods and Functions}
    
           \vspace{0.5cm}
            
                
           \vspace{1.5cm}
    
           \textbf{Nathan Warner}
    
           \vfill
                
                
           \vspace{0.8cm}
         
           \includegraphics[width=0.4\textwidth]{~/niu/seal.png}
                
           Computer Science \\
           Northern Illinois University\\
           December 12, 2023 \\
           United States\\
                
       \end{center}
    \end{titlepage}
    \tableofcontents
    \pagebreak 
    \unsect{Known Includes}
    \begin{itemize}
        \item #include <QApplication>  // Manages application-wide resources and the main event loop
        \item #include <QString>       // String class for handling Unicode text
        \item #include <QWidget>       // Base class for all UI objects in Qt
        \item #include <QPushButton>   // Provides a push button widget
        \item #include <QLabel>        // Provides a text or image display widget
        \item #include <QProcess>      // Enables running external processes
        \item #include <QStringList>   // List of QString objects, often used for string manipulation
        \item #include <QPainter>      // Used for drawing graphics in widgets
        \item #include <QPoint>        // Represents x and y coordinates in a 2D space
        \item #include <QtCore>        // For all shape objects
        \item #include <QRect>         // Defines a rectangle in the plane using integer precision
        \item #include <QPolygon>      // Represents a polygon defined by a vector of points
        \item #include <QBrush>        // Used for filling shapes with solid colors, patterns, or gradients
        \item #include <QPen>          // Used for drawing lines and outlines of shapes
        \item #include <QImage>        // Represents an image; used in conjunction with QPainter
        \item #include <QColor>        // Used to define html colors
        \item #include <QGradient>     // To create gradient objects
        \item #include <QLayout>       // Used for layouts
        \item #include <QHBoxLayout>   // Used for layouts
        \item #include <QVBoxLayout>   // Used for layouts
        \item #include <QFont>         // Used for fonts and font styles
     \end{itemize}

    \pagebreak 
    \unsect{Known Objects}
    \begin{itemize}
        \item \textbf{QString}
        \item \textbf{QStringList}
        \item \textbf{QPushButton}
        \item \textbf{QLabel}
        \item \textbf{QFont}
        \item \textbf{QColor}
        \item \textbf{QLinearGradient}
        \item \textbf{QRadialGradient}
        \item \textbf{QConicalGradient}
        \item \textbf{Qpen}
        \item \textbf{QBrush}
        \item \textbf{Colors}
        \item \textbf{QPainter}
        \item \textbf{Shape Objects}
        \item \textbf{QHBoxLayout}
        \item \textbf{QVBoxLayout}
        \item \textbf{QGridLayout}
    \end{itemize}

    \pagebreak 
    \unsect{Constructors}
    \begin{itemize}
        \item \textbf{QString}:
        \begin{itemize}
            \item \texttt{QString()}
            \item \texttt{QString(const QString \&)}
            \item \texttt{QString(const char *)}
        \end{itemize}

        \item \textbf{QStringList}:
        \begin{itemize}
            \item \texttt{QStringList()}
            \item \texttt{QStringList(const QString \&)}
        \end{itemize}

        \item \textbf{QPushButton}:
        \begin{itemize}
            \item \texttt{QPushButton(QWidget *parent = nullptr)}
            \item \texttt{QPushButton(const QString \&text, QWidget *parent = nullptr)}
        \end{itemize}

        \item \textbf{QLabel}:
        \begin{itemize}
            \item \texttt{QLabel(QWidget *parent = nullptr, Qt::WindowFlags f = Qt::WindowFlags())}
            \item \texttt{QLabel(const QString \&text, QWidget *parent = nullptr, Qt::WindowFlags f = Qt::WindowFlags())}
        \end{itemize}

        \item \textbf{QFont}:
        \begin{itemize}
            \item \texttt{QFont()}
            \item \texttt{QFont(const QString \&family, int pointSize = -1, int weight = -1, bool italic = false)}
        \end{itemize}

        \item \textbf{QColor}:
        \begin{itemize}
            \item \texttt{QColor()}
            \item \texttt{QColor(int r, int g, int b, int a = 255)}
            \item \texttt{QColor(const QString \&name)}
            \item \texttt{QColor(Qt::GlobalColor color)}
        \end{itemize}

        \item \textbf{QLinearGradient}:
        \begin{itemize}
            \item \texttt{QLinearGradient()}
            \item \texttt{QLinearGradient(const QPointF \&start, const QPointF \&finalStop)}
            \item \texttt{QLinearGradient(qreal xStart, qreal yStart, qreal xFinalStop, qreal yFinalStop)}
        \end{itemize}

        \item \textbf{QRadialGradient}:
        \begin{itemize}
            \item \texttt{QRadialGradient()}
            \item \texttt{QRadialGradient(const QPointF \&center, qreal radius, const QPointF \&focalPoint)}
            \item \texttt{QRadialGradient(qreal cx, qreal cy, qreal radius, qreal fx, qreal fy)}
        \end{itemize}

        \item \textbf{QConicalGradient}:
        \begin{itemize}
            \item \texttt{QConicalGradient()}
            \item \texttt{QConicalGradient(const QPointF \&center, qreal startAngle)}
            \item \texttt{QConicalGradient(qreal cx, qreal cy, qreal startAngle)}
        \end{itemize}

        \item \textbf{QPen}:
        \begin{itemize}
            \item \texttt{QPen()}
            \item \texttt{QPen(const QColor \&color)}
            \item \texttt{QPen(Qt::PenStyle style)}
            \item \texttt{QPen(const QBrush \&brush, qreal width, Qt::PenStyle style = Qt::SolidLine, Qt::PenCapStyle cap = Qt::SquareCap, Qt::PenJoinStyle join = Qt::BevelJoin)}
        \end{itemize}

        \item \textbf{QBrush}:
        \begin{itemize}
            \item \texttt{QBrush()}
            \item \texttt{QBrush(Qt::BrushStyle style)}
            \item \texttt{QBrush(const QColor \&color, Qt::BrushStyle style = Qt::SolidPattern)}
            \item \texttt{QBrush(const QPixmap \&pixmap)}
            \item \texttt{QBrush(const QBrush \&brush)}
        \end{itemize}

        \item \textbf{QPainter}: (No default constructor; use \texttt{begin()} and \texttt{end()} methods)

        \item \textbf{Shape Objects} (e.g., QRect, QPoint):
        \begin{itemize}
            \item \texttt{QRect()}
            \item \texttt{QRect(int x, int y, int width, int height)}
            \item \texttt{QPoint()}
            \item \texttt{QPoint(int xpos, int ypos)}
        \end{itemize}

        \item \textbf{QHBoxLayout}:
        \begin{itemize}
            \item \texttt{QHBoxLayout()}
            \item \texttt{QHBoxLayout(QWidget *parent)}
        \end{itemize}

        \item \textbf{QVBoxLayout}:
        \begin{itemize}
            \item \texttt{QVBoxLayout()}
            \item \texttt{QVBoxLayout(QWidget *parent)}
        \end{itemize}

        \item \textbf{QGridLayout}:
        \begin{itemize}
            \item \texttt{QGridLayout()}
            \item \texttt{QGridLayout(QWidget *parent)}
        \end{itemize}
    \end{itemize}

    \pagebreak 
    \unsect{QWidget Methods}
    \begin{itemize}
        \item \textbf{Constructor} $\mapsto$ \texttt{QWidget(QWidget *parent = nullptr, Qt::WindowFlags f = Qt::WindowFlags())}: Initializes a new instance of QWidget with optional parent and window flags.
        \item \textbf{setGeometry} $\mapsto$ \texttt{void}: Sets the geometry of the widget.
        \item \textbf{geometry} $\mapsto$ \texttt{QRect}: Returns the widget's geometry.
        \item \textbf{move} $\mapsto$ \texttt{void}: Moves the widget to a specified position.
        \item \textbf{resize} $\mapsto$ \texttt{void}: Resizes the widget.
        \item \textbf{setFixedSize} $\mapsto$ \texttt{void}: Sets a fixed size for the widget.
        \item \textbf{setStyle} $\mapsto$ \texttt{void}: Sets the style of the widget.
        \item \textbf{setStyleSheet} $\mapsto$ \texttt{void}: Sets the style sheet used for custom widget styling.
        \item \textbf{update} $\mapsto$ \texttt{void}: Updates the widget.
        \item \textbf{repaint} $\mapsto$ \texttt{void}: Repaints the widget immediately.
        \item \textbf{show} $\mapsto$ \texttt{void}: Shows the widget.
        \item \textbf{hide} $\mapsto$ \texttt{void}: Hides the widget.
        \item \textbf{setVisible} $\mapsto$ \texttt{void}: Sets the visibility of the widget.
        \item \textbf{close} $\mapsto$ \texttt{bool}: Closes the widget.
        \item \textbf{setLayout} $\mapsto$ \texttt{void}: Sets the layout for the widget.
        \item \textbf{setFocus} $\mapsto$ \texttt{void}: Sets focus to the widget.
        \item \textbf{clearFocus} $\mapsto$ \texttt{void}: Clears focus from the widget.
        \item \textbf{setFocusPolicy} $\mapsto$ \texttt{void}: Sets the focus policy for the widget.
        \item \textbf{windowTitle} / \textbf{setWindowTitle} $\mapsto$ \texttt{QString} / \texttt{void}: Gets or sets the window title.
        \item \textbf{setProperty} $\mapsto$ \texttt{void}: Sets a property of the widget.
        \item \textbf{property} $\mapsto$ \texttt{QVariant}: Returns the value of a property.
    \end{itemize}

    \pagebreak 
    \unsect{QString Methods}
    \begin{itemize}
        \item \textbf{length()} const $\mapsto$ \texttt{int}: Returns the length of the string.
        \item \textbf{isEmpty()} const $\mapsto$ \texttt{bool}: Returns true if the string is empty.
        \item \textbf{append(const QString \&str)} $\mapsto$ \texttt{QString \&}: Appends the given string to this string.
        \item \textbf{prepend(const QString \&str)} $\mapsto$ \texttt{QString \&}: Prepends the given string to this string.
        \item \textbf{contains(const QString \&str, Qt::CaseSensitivity cs = Qt::CaseSensitive)} const $\mapsto$ \texttt{bool}: Returns true if the string contains the given substring.
        \item \textbf{indexOf(const QString \&str, int from = 0, Qt::CaseSensitivity cs = Qt::CaseSensitive)} const $\mapsto$ \texttt{int}: Returns the index position of the first occurrence of the given substring.
        \item \textbf{lastIndexOf(const QString \&str, int from = -1, Qt::CaseSensitivity cs = Qt::CaseSensitive)} const $\mapsto$ \texttt{int}: Returns the index position of the last occurrence of the given substring.
        \item \textbf{remove(int pos, int len)} $\mapsto$ \texttt{QString \&}: Removes \texttt{len} characters from the string starting at position \texttt{pos}.
        \item \textbf{replace(const QString \&before, const QString \&after, Qt::CaseSensitivity cs = Qt::CaseSensitive)} $\mapsto$ \texttt{QString \&}: Replaces occurrences of the substring \texttt{before} with the substring \texttt{after}.
        \item \textbf{split(const QString \&delimiter, Qt::SplitBehavior splitBehavior = Qt::KeepEmptyParts, Qt::CaseSensitivity cs = Qt::CaseSensitive)} const $\mapsto$ \texttt{QStringList}: Splits the string into a list of strings divided by the given delimiter.
        \item \textbf{toLower()} const $\mapsto$ \texttt{QString}: Returns a copy of the string converted to lowercase.
        \item \textbf{toUpper()} const $\mapsto$ \texttt{QString}: Returns a copy of the string converted to uppercase.
        \item \textbf{trimmed()} const $\mapsto$ \texttt{QString}: Returns a copy of the string with whitespace removed from the start and end.
        \item \textbf{left(int n)} const $\mapsto$ \texttt{QString}: Returns the leftmost \texttt{n} characters of the string.
        \item \textbf{right(int n)} const $\mapsto$ \texttt{QString}: Returns the rightmost \texttt{n} characters of the string.
        \item \textbf{mid(int position, int n = -1)} const $\mapsto$ \texttt{QString}: Returns a substring of \texttt{n} characters from the string starting at \texttt{position}.
    \end{itemize}

    \pagebreak 
    \unsect{QStringList Methods}
    \begin{itemize}
        \item \textbf{append(const QString \&str)}: Adds the given string to the end of the list.
        \item \textbf{at(int i) const} $\mapsto$ \texttt{QString}: Returns the string at the specified position in the list.
        \item \textbf{join(const QString \&separator) const} $\mapsto$ \texttt{QString}: Concatenates all the strings in the list into a single string with a specified separator.
        \item \textbf{sort(Qt::SortOrder order = Qt::AscendingOrder)}: Sorts the list in ascending or descending order.
        \item \textbf{filter(const QString \&pattern, Qt::CaseSensitivity cs = Qt::CaseSensitive) const} $\mapsto$ \texttt{QStringList}: Returns a new list containing only the strings that match a given pattern.
        \item \textbf{size() const} / \textbf{count() const} $\mapsto$ \texttt{int}: Returns the number of items in the list.
        \item \textbf{isEmpty() const} $\mapsto$ \texttt{bool}: Checks if the list is empty.
        \item \textbf{clear()}: Clears all items from the list.
        \item \textbf{removeDuplicates()} $\mapsto$ \texttt{int}: Removes duplicate strings from the list and returns the number of removed items.
        \item \textbf{contains(const QString \&str, Qt::CaseSensitivity cs = Qt::CaseSensitive) const} $\mapsto$ \texttt{bool}: Returns true if the list contains the given string.
        \item \textbf{indexOf(const QString \&str, int from = 0) const} $\mapsto$ \texttt{int}: Returns the index of the first occurrence of the string in the list, searching forward from index \texttt{from}.
        \item \textbf{replaceInStrings(const QString \&before, const QString \&after, Qt::CaseSensitivity cs = Qt::CaseSensitive)}: Replaces occurrences of a substring within all the strings of the list.
    \end{itemize}



    \pagebreak 
    \unsect{Button Methods}
    \begin{itemize}
        \item \textbf{setText(const QString \&text)} $\mapsto$ \texttt{void}: Sets the button's text to \texttt{text}.
        \item \textbf{text()} const $\mapsto$ \texttt{QString}: Returns the button's text.
        \item \textbf{setIcon(const QIcon \&icon)} $\mapsto$ \texttt{void}: Sets the icon of the button to \texttt{icon}.
        \item \textbf{icon()} const $\mapsto$ \texttt{QIcon}: Returns the button's icon.
        \item \textbf{setShortcut(const QKeySequence \&key)} $\mapsto$ \texttt{void}: Sets a shortcut key for the button with \texttt{key}.
        \item \textbf{shortcut()} const $\mapsto$ \texttt{QKeySequence}: Returns the shortcut key associated with the button.
        \item \textbf{setChecked(bool check)} $\mapsto$ \texttt{void}: Sets the check state of the button to \texttt{check} (if the button is checkable).
        \item \textbf{isChecked()} const $\mapsto$ \texttt{bool}: Returns true if the button is checked.
        \item \textbf{setFlat(bool flat)} $\mapsto$ \texttt{void}: Sets whether the button is flat to \texttt{flat}.
        \item \textbf{isFlat()} const $\mapsto$ \texttt{bool}: Returns true if the button is flat.
        \item \textbf{setMenu(QMenu *menu)} $\mapsto$ \texttt{void}: Sets the associated drop-down menu of the button to \texttt{menu}.
        \item \textbf{menu()} const $\mapsto$ \texttt{QMenu*}: Returns the associated menu of the button.
        \item \textbf{showMenu()} $\mapsto$ \texttt{void}: Displays the associated drop-down menu.
        \item \textbf{setAutoDefault(bool autoDefault)} $\mapsto$ \texttt{void}: Sets whether the button is an auto default button to \texttt{autoDefault}.
        \item \textbf{isAutoDefault()} const $\mapsto$ \texttt{bool}: Returns true if the button is an auto default button.
        \item \textbf{setDefault(bool default)} $\mapsto$ \texttt{void}: Sets whether the button is the default button to \texttt{default}.
        \item \textbf{isDefault()} const $\mapsto$ \texttt{bool}: Returns true if the button is the default button.
        \item \textbf{setCheckable(bool checkable)} $\mapsto$ \texttt{void}: Sets whether the button is checkable to \texttt{checkable}.
        \item \textbf{isCheckable()} const $\mapsto$ \texttt{bool}: Returns true if the button is checkable.
        \item \textbf{click()} $\mapsto$ \texttt{void}: Simulates a click on the button.
        \item \textbf{animateClick(int msec = 100)} $\mapsto$ \texttt{void}: Simulates an animated click on the button, with the animation lasting \texttt{msec} milliseconds.
        \item \textbf{setAutoRepeat(bool autoRepeat)} $\mapsto$ \texttt{void}: Enables or disables auto-repeat for the button to \texttt{autoRepeat}.
        \item \textbf{autoRepeat()} const $\mapsto$ \texttt{bool}: Returns true if auto-repeat is enabled for the button.
        \item \textbf{setAutoRepeatDelay(int delay)} $\mapsto$ \texttt{void}: Sets the auto-repeat delay for the button to \texttt{delay} milliseconds.
        \item \textbf{autoRepeatDelay()} const $\mapsto$ \texttt{int}: Returns the auto-repeat delay in milliseconds.
        \item \textbf{setAutoRepeatInterval(int interval)} $\mapsto$ \texttt{void}: Sets the auto-repeat interval for the button to \texttt{interval} milliseconds.
        \item \textbf{autoRepeatInterval()} const $\mapsto$ \texttt{int}: Returns the auto-repeat interval in milliseconds.
    \end{itemize}
    \bigbreak \noindent 
    \subsection{Signals}
    \begin{itemize}
        \item \textbf{clicked(bool checked)}: Emitted when the button is clicked. If the button is checkable, \texttt{checked} is true if the button is checked, otherwise false.
        \item \textbf{pressed()}: Emitted when the button is pressed down.
        \item \textbf{released()}: Emitted when the button is released.
        \item \textbf{toggled(bool checked)}: Emitted when the toggle state of the button changes. \texttt{checked} is true if the button is checked, otherwise false.
    \end{itemize}

    \pagebreak 


    \unsect{Label Methods}
    \begin{itemize}
        \item \textbf{setText(const QString \&)} $\mapsto$ \texttt{void}: Sets the label's text.
        \item \textbf{setPixmap(const QPixmap \&)} $\mapsto$ \texttt{void}: Sets the label's pixmap.
        \item \textbf{setAlignment(Qt::Alignment)} $\mapsto$ \texttt{void}: Sets the alignment of the label's content.
        \item \textbf{setWordWrap(bool)} $\mapsto$ \texttt{void}: Enables or disables word wrapping.
        \item \textbf{text() const} $\mapsto$ \texttt{QString}: Returns the label's text.
        \item \textbf{pixmap() const} $\mapsto$ \texttt{const QPixmap *}: Returns the label's pixmap.
        \item \textbf{alignment() const} $\mapsto$ \texttt{Qt::Alignment}: Returns the alignment of the label's content.
        \item \textbf{wordWrap() const} $\mapsto$ \texttt{bool}: Returns whether word wrapping is enabled.
        \item \textbf{clear()} $\mapsto$ \texttt{void}: Clears the label's contents.
        \item \textbf{setIndent(int)} $\mapsto$ \texttt{void}: Sets the indent used for the label's text or pixmap.
        \item \textbf{indent() const} $\mapsto$ \texttt{int}: Returns the indent used for the label's text or pixmap.
        \item \textbf{setMargin(int)} $\mapsto$ \texttt{void}: Sets the margin around the label's contents.
        \item \textbf{margin() const} $\mapsto$ \texttt{int}: Returns the margin around the label's contents.
        \item \textbf{setOpenExternalLinks(bool)} $\mapsto$ \texttt{void}: Sets whether the label should open external links.
        \item \textbf{openExternalLinks() const} $\mapsto$ \texttt{bool}: Returns whether the label opens external links.
    \end{itemize}

    \subsubsection{Signals}
    \begin{itemize}
        \item \textbf{linkActivated(const QString \&)}: Emitted when a link in the text is activated.
        \item \textbf{linkHovered(const QString \&)}: Emitted when the mouse hovers over a link in the text.
    \end{itemize}



    \pagebreak 
    \unsect{QFont methods}
    \bigbreak \noindent 
    \begin{itemize}
        \item \textbf{setFamily(QString \&family)} $\mapsto$ \texttt{void}: Sets the font family to \texttt{family}.
        \item \textbf{family()} const $\mapsto$ \texttt{QString}: Returns the family name of the font.
        \item \textbf{setPointSize(int size)} $\mapsto$ \texttt{void}: Sets the font size in points to \texttt{size}.
        \item \textbf{pointSize()} const $\mapsto$ \texttt{int}: Returns the point size of the font.
        \item \textbf{setPixelSize(int size)} $\mapsto$ \texttt{void}: Sets the font size in pixels to \texttt{size}.
        \item \textbf{pixelSize()} const $\mapsto$ \texttt{int}: Returns the pixel size of the font.
        \item \textbf{setWeight(int weight)} $\mapsto$ \texttt{void}: Sets the weight of the font to \texttt{weight}.
        \item \textbf{weight()} const $\mapsto$ \texttt{int}: Returns the weight of the font.
        \item \textbf{setBold(bool bold)} $\mapsto$ \texttt{void}: Sets the font's bold property to \texttt{bold}.
        \item \textbf{bold()} const $\mapsto$ \texttt{bool}: Returns true if the font is bold.
        \item \textbf{setItalic(bool italic)} $\mapsto$ \texttt{void}: Sets the font's italic property to \texttt{italic}.
        \item \textbf{italic()} const $\mapsto$ \texttt{bool}: Returns true if the font is italic.
        \item \textbf{setUnderline(bool underline)} $\mapsto$ \texttt{void}: Sets the font's underline property to \texttt{underline}.
        \item \textbf{underline()} const $\mapsto$ \texttt{bool}: Returns true if the font is underlined.
        \item \textbf{setOverline(bool overline)} $\mapsto$ \texttt{void}: Sets the font's overline property to \texttt{overline}.
        \item \textbf{overline()} const $\mapsto$ \texttt{bool}: Returns true if the font has an overline.
        \item \textbf{setStrikeOut(bool strikeOut)} $\mapsto$ \texttt{void}: Sets the font's strikeout property to \texttt{strikeOut}.
        \item \textbf{strikeOut()} const $\mapsto$ \texttt{bool}: Returns true if the font is struck out.
        \item \textbf{setKerning(bool enable)} $\mapsto$ \texttt{void}: Enables or disables kerning based on \texttt{enable}.
        \item \textbf{kerning()} const $\mapsto$ \texttt{bool}: Returns true if kerning is enabled.
        \item \textbf{setStyle(QFont::Style style)} $\mapsto$ \texttt{void}: Sets the style of the font to \texttt{style}.
        \item \textbf{style()} const $\mapsto$ \texttt{QFont::Style}: Returns the style of the font.
        \item \textbf{setStyleHint(QFont::StyleHint hint, QFont::StyleStrategy strategy = QFont::PreferDefault)} $\mapsto$ \texttt{void}: Sets the style hint and strategy of the font to \texttt{hint} and \texttt{strategy}.
        \item \textbf{styleHint()} const $\mapsto$ \texttt{QFont::StyleHint}: Returns the style hint of the font.
        \item \textbf{setStretch(int factor)} $\mapsto$ \texttt{void}: Sets the stretch factor of the font to \texttt{factor}.
        \item \textbf{stretch()} const $\mapsto$ \texttt{int}: Returns the stretch factor of the font.
        \item \textbf{setLetterSpacing(QFont::SpacingType type, qreal spacing)} $\mapsto$ \texttt{void}: Sets the type and amount of letter spacing.
        \item \textbf{letterSpacing()} const $\mapsto$ \texttt{qreal}: Returns the amount of letter spacing.
        \item \textbf{setWordSpacing(qreal spacing)} $\mapsto$ \texttt{void}: Sets the amount of word spacing to \texttt{spacing}.
        \item \textbf{wordSpacing()} const $\mapsto$ \texttt{qreal}: Returns the amount of word spacing.
    \end{itemize}

    \pagebreak 
    \unsect{QColor Methods}
    \begin{itemize}
        \item \textbf{setRgb(int r, int g, int b, int a = 255)}: Sets the color using RGBA values.
        \item \textbf{setRgbF(qreal r, qreal g, qreal b, qreal a = 1.0)}: Sets the color using RGBA values as floating point numbers.
        \item \textbf{setNamedColor(const QString \&name)}: Sets the color using a color name.
        \item \textbf{setHsl(int h, int s, int l, int a = 255)}: Sets the color using HSL values.
        \item \textbf{setHslF(qreal h, qreal s, qreal l, qreal a = 1.0)}: Sets the color using HSL values as floating point numbers.
        \item \textbf{red() const} $\mapsto$ \texttt{int}: Returns the red component of the color.
        \item \textbf{green() const} $\mapsto$ \texttt{int}: Returns the green component of the color.
        \item \textbf{blue() const} $\mapsto$ \texttt{int}: Returns the blue component of the color.
        \item \textbf{alpha() const} $\mapsto$ \texttt{int}: Returns the alpha (transparency) component of the color.
        \item \textbf{hue() const} $\mapsto$ \texttt{int}: Returns the hue component of the color.
        \item \textbf{saturation() const} $\mapsto$ \texttt{int}: Returns the saturation component of the color.
        \item \textbf{lightness() const} $\mapsto$ \texttt{int}: Returns the lightness component of the color.
        \item \textbf{darker(int factor = 200) const} $\mapsto$ \texttt{QColor}: Returns a darker color.
        \item \textbf{lighter(int factor = 150) const} $\mapsto$ \texttt{QColor}: Returns a lighter color.
        \item \textbf{isValid() const} $\mapsto$ \texttt{bool}: Returns true if the color is valid.
        \item \textbf{name(QColor::NameFormat format = QColor::HexRgb) const} $\mapsto$ \texttt{QString}: Returns the name of the color.
        \item \textbf{toRgb() const} $\mapsto$ \texttt{QColor}: Converts the color to an RGB color.
        \item \textbf{toHsl() const} $\mapsto$ \texttt{QColor}: Converts the color to an HSL color.
    \end{itemize}

    \pagebreak 
    \unsect{QGradient Methods}

    \bigbreak \noindent 
    \subsection{QGradient Class Methods}
    \begin{itemize}
        \item \textbf{setColorAt(qreal position, const QColor \&color)}: Sets the color at the specified position in the gradient.
        \item \textbf{setSpread(QGradient::Spread spread)}: Sets the spread method for the gradient.
        \item \textbf{spread() const} $\mapsto$ \texttt{QGradient::Spread}: Returns the current spread method.
        \item \textbf{setCoordinateMode(QGradient::CoordinateMode mode)}: Sets the coordinate mode of the gradient.
        \item \textbf{coordinateMode() const} $\mapsto$ \texttt{QGradient::CoordinateMode}: Returns the coordinate mode.
        \item \textbf{stops() const} $\mapsto$ \texttt{QList<QPair<qreal, QColor>>}: Returns the gradient stops as a list of pairs of positions and colors.
    \end{itemize}

    \bigbreak \noindent 
    \subsection{QLinearGradient Class Methods}
    \begin{itemize}
        \item \textbf{setStart(const QPointF \&start)}: Sets the start point of the linear gradient.
        \item \textbf{setFinalStop(const QPointF \&stop)}: Sets the final stop point of the linear gradient.
        \item \textbf{start() const} $\mapsto$ \texttt{QPointF}: Returns the start point.
        \item \textbf{finalStop() const} $\mapsto$ \texttt{QPointF}: Returns the final stop point.
    \end{itemize}

    \bigbreak \noindent 
    \subsection{QRadialGradient Class Methods}
    \begin{itemize}
        \item \textbf{setCenter(const QPointF \&center)}: Sets the center of the radial gradient.
        \item \textbf{setRadius(qreal radius)}: Sets the radius of the radial gradient.
        \item \textbf{setFocalPoint(const QPointF \&focalPoint)}: Sets the focal point of the radial gradient.
        \item \textbf{center() const} $\mapsto$ \texttt{QPointF}: Returns the center point.
        \item \textbf{radius() const} $\mapsto$ \texttt{qreal}: Returns the radius.
        \item \textbf{focalPoint() const} $\mapsto$ \texttt{QPointF}: Returns the focal point.
    \end{itemize}

    \bigbreak \noindent 
    \subsection{QConicalGradient Class Methods}
    \begin{itemize}
        \item \textbf{setCenter(const QPointF \&center)}: Sets the center of the conical gradient.
        \item \textbf{setAngle(qreal angle)}: Sets the start angle of the conical gradient.
        \item \textbf{center() const} $\mapsto$ \texttt{QPointF}: Returns the center point.
        \item \textbf{angle() const} $\mapsto$ \texttt{qreal}: Returns the start angle.
    \end{itemize}

    \pagebreak 
    \unsect{QPen Methods}
    \begin{itemize}
        \item \textbf{setColor(const QColor \&color)} $\mapsto$ \texttt{void}: Sets the color of the pen to \texttt{color}.
        \item \textbf{color()} const $\mapsto$ \texttt{QColor}: Returns the color of the pen.
        \item \textbf{setWidth(int width)} $\mapsto$ \texttt{void}: Sets the width of the pen to \texttt{width}.
        \item \textbf{width()} const $\mapsto$ \texttt{int}: Returns the width of the pen.
        \item \textbf{setBrush(const QBrush \&brush)} $\mapsto$ \texttt{void}: Sets the brush of the pen to \texttt{brush}.
        \item \textbf{brush()} const $\mapsto$ \texttt{QBrush}: Returns the brush of the pen.
        \item \textbf{setStyle(Qt::PenStyle style)} $\mapsto$ \texttt{void}: Sets the style of the pen to \texttt{style}.
        \item \textbf{style()} const $\mapsto$ \texttt{Qt::PenStyle}: Returns the style of the pen.
        \item \textbf{setCapStyle(Qt::PenCapStyle capStyle)} $\mapsto$ \texttt{void}: Sets the cap style of the pen to \texttt{capStyle}.
        \item \textbf{capStyle()} const $\mapsto$ \texttt{Qt::PenCapStyle}: Returns the cap style of the pen.
        \item \textbf{setJoinStyle(Qt::PenJoinStyle joinStyle)} $\mapsto$ \texttt{void}: Sets the join style of the pen to \texttt{joinStyle}.
        \item \textbf{joinStyle()} const $\mapsto$ \texttt{Qt::PenJoinStyle}: Returns the join style of the pen.
    \end{itemize}

    \pagebreak 
    \unsect{QBrush Methods}
    \begin{itemize}
        \item \textbf{setColor(const QColor \&color)} $\mapsto$ \texttt{void}: Sets the color of the brush to \texttt{color}.
        \item \textbf{color()} const $\mapsto$ \texttt{QColor}: Returns the color of the brush.
        \item \textbf{setStyle(Qt::BrushStyle style)} $\mapsto$ \texttt{void}: Sets the style of the brush to \texttt{style}.
        \item \textbf{style()} const $\mapsto$ \texttt{Qt::BrushStyle}: Returns the style of the brush.
        \item \textbf{setTexture(const QPixmap \&pixmap)} $\mapsto$ \texttt{void}: Sets the texture of the brush to the pixmap \texttt{pixmap}.
        \item \textbf{texture()} const $\mapsto$ \texttt{QPixmap}: Returns the pixmap used as the texture of the brush.
        \item \textbf{setTextureImage(const QImage \&image)} $\mapsto$ \texttt{void}: Sets the texture of the brush to the image \texttt{image}.
        \item \textbf{textureImage()} const $\mapsto$ \texttt{QImage}: Returns the image used as the texture of the brush.
        \item \textbf{setMatrix(const QMatrix \&matrix)} $\mapsto$ \texttt{void}: Sets the transformation matrix of the brush to \texttt{matrix}.
        \item \textbf{matrix()} const $\mapsto$ \texttt{QMatrix}: Returns the transformation matrix of the brush.
    \end{itemize}

    \pagebreak 
    \unsect{Colors for pen and brush}
    \begin{itemize}
        \item \textbf{Qt::black}: Represents the color black.
        \item \textbf{Qt::white}: Represents the color white.
        \item \textbf{Qt::red}: Represents the color red.
        \item \textbf{Qt::green}: Represents the color green.
        \item \textbf{Qt::blue}: Represents the color blue.
        \item \textbf{Qt::cyan}: Represents the color cyan (a mix of green and blue).
        \item \textbf{Qt::magenta}: Represents the color magenta (a mix of red and blue).
        \item \textbf{Qt::yellow}: Represents the color yellow.
        \item \textbf{Qt::darkRed}: Represents a dark shade of red.
        \item \textbf{Qt::darkGreen}: Represents a dark shade of green.
        \item \textbf{Qt::darkBlue}: Represents a dark shade of blue.
        \item \textbf{Qt::darkCyan}: Represents a dark shade of cyan.
        \item \textbf{Qt::darkMagenta}: Represents a dark shade of magenta.
        \item \textbf{Qt::darkYellow}: Represents a dark shade of yellow.
        \item \textbf{Qt::gray}: Represents the color gray.
        \item \textbf{Qt::darkGray}: Represents a dark shade of gray.
        \item \textbf{Qt::lightGray}: Represents a light shade of gray.
        \item \textbf{Qt::transparent}: Represents a transparent color.
    \end{itemize}

    \pagebreak 
    \unsect{QPainter Methods}
    \begin{itemize}
        \item \textbf{begin(QPaintDevice *device)} $\mapsto$ \texttt{bool}: Initializes the painter for the given paint device.
        \item \textbf{end()} $\mapsto$ \texttt{bool}: Ends the painting process and releases any resources used for painting.
        \item \textbf{pen() const} $\mapsto$ \texttt{QPen}: Returns the currently used pen.
        \item \textbf{brush() const} $\mapsto$ \texttt{QBrush}: Returns the currently used brush.
        \item \textbf{setPen(const QPen \&pen)}: Sets the pen to be used for drawing lines and outlines.
        \item \textbf{setBrush(const QBrush \&brush)}: Sets the brush to be used for filling shapes.
        \item \textbf{setFont(const QFont \&font)}: Sets the font to be used for drawing text.
        \item \textbf{font() const} $\mapsto$ \texttt{QFont}: Returns the font currently set for the QPainter.
        \item \textbf{save()}: Saves the current state of the painter.
        \item \textbf{restore()}: Restores the painter to the state saved by the most recent call to save().
        \item \textbf{setTransform(const QTransform \&transform, bool combine = false)}: Sets the transformation matrix for the painter.
        \item \textbf{transform() const} $\mapsto$ \texttt{QTransform}: Returns the current transformation matrix.
        \item \textbf{setRenderHint(QPainter::RenderHint hint, bool on = true)}: Sets a render hint to improve drawing quality.
        \item \textbf{drawLine(const QPoint \&p1, const QPoint \&p2)}: Draws a line between the points p1 and p2.
        \item \textbf{drawRect(const QRect \&rect)}: Draws a rectangle with the top-left corner and size specified by rect.
        \item \textbf{drawEllipse(const QRect \&rect)}: Draws an ellipse inside the specified rectangular area.
        \item \textbf{drawText(const QPoint \&point, const QString \&text)}: Draws the given text at the specified point.
        \item \textbf{drawPixmap(const QPoint \&point, const QPixmap \&pixmap)}: Draws a pixmap at the given point.
        \item \textbf{drawImage(const QPoint \&point, const QImage \&image)}: Draws an image at the specified point.
        \item \textbf{fillRect(const QRect \&rect, const QBrush \&brush)}: Fills the given rectangle with the specified brush.
        \item \textbf{translate(const QPointF \&offset)}: Translates the coordinate system by the given offset.
        \item \textbf{rotate(qreal angle)}: Rotates the coordinate system by the specified angle.
        \item \textbf{scale(qreal sx, qreal sy)}: Scales the coordinate system by the factors sx and sy.
    \end{itemize}

    \pagebreak 
    \unsect{Shape Objects <QtCore>}
    \begin{itemize}
        \item \textbf{QPoint}: Represents a point in 2D space with integer precision.
        \item \textbf{QPointF}: Represents a point in 2D space with floating-point precision.
        \item \textbf{QSize}: Defines the size of a 2D object using integer precision.
        \item \textbf{QSizeF}: Defines the size of a 2D object using floating-point precision.
        \item \textbf{QRect}: Represents a rectangle in 2D space with integer precision.
        \item \textbf{QRectF}: Represents a rectangle in 2D space with floating-point precision.
        \item \textbf{QLine}: Represents a line in 2D space with integer precision.
        \item \textbf{QLineF}: Represents a line in 2D space with floating-point precision.
        \item \textbf{QPolygon}: Represents a polygon defined by a vector of points with integer precision.
        \item \textbf{QPolygonF}: Represents a polygon defined by a vector of points with floating-point precision.
        \item \textbf{QRegion}: Represents a region in a plane, which can be non-contiguous.
        \item \textbf{QPath}: Represents a path, potentially containing lines, curves, and subpaths.
        \item \textbf{QPainterPath}: Represents a path that can be drawn with QPainter, including moveto, lineto, and curveto operations.
    \end{itemize}

    \pagebreak 
    \unsect{Layout Methods}

    \bigbreak \noindent 
    \subsection{QHBoxLayout Methods}
    \begin{itemize}
        \item \textbf{addStretch(int stretch = 0)}: Adds a stretchable space to the layout.
        \item \textbf{addWidget(QWidget *widget, int stretch = 0, Qt::Alignment alignment = 0)}: Adds a widget to the layout.
        \item \textbf{insertStretch(int index, int stretch = 0)}: Inserts a stretchable space at the specified index in the layout.
        \item \textbf{insertWidget(int index, QWidget *widget, int stretch = 0, Qt::Alignment alignment = 0)}: Inserts a widget at the specified index in the layout.
    \end{itemize}

    \bigbreak \noindent 
    \section{QVBoxLayout Methods}
    \begin{itemize}
        \item \textbf{addStretch(int stretch = 0)}: Adds a stretchable space to the layout.
        \item \textbf{addWidget(QWidget *widget, int stretch = 0, Qt::Alignment alignment = 0)}: Adds a widget to the layout.
        \item \textbf{insertStretch(int index, int stretch = 0)}: Inserts a stretchable space at the specified index in the layout.
        \item \textbf{insertWidget(int index, QWidget *widget, int stretch = 0, Qt::Alignment alignment = 0)}: Inserts a widget at the specified index in the layout.
    \end{itemize}

    \bigbreak \noindent 
    \subsection{QGridLayout Methods}
    \begin{itemize}
        \item \textbf{addWidget(QWidget *widget, int row, int column, Qt::Alignment alignment = 0)}: Adds a widget to the layout at the specified row and column.
        \item \textbf{addWidget(QWidget *widget, int row, int column, int rowSpan, int columnSpan, Qt::Alignment alignment = 0)}: Adds a widget to the layout, spanning multiple rows and/or columns.
        \item \textbf{setRowStretch(int row, int stretch)}: Sets the stretch factor for the specified row.
        \item \textbf{setColumnStretch(int column, int stretch)}: Sets the stretch factor for the specified column.
        \item \textbf{setRowMinimumHeight(int row, int minSize)}: Sets the minimum height for the specified row.
        \item \textbf{setColumnMinimumWidth(int column, int minSize)}: Sets the minimum width for the specified column.
    \end{itemize}

    \pagebreak 
    \unsect{Other QObjects derived objects}

    \bigbreak \noindent 
    \subsection{Widget Classes}
    \begin{itemize}
        \item \textbf{QWidget}: Base class for all UI objects.
        \item \textbf{QMainWindow}: Main application window class.
        \item \textbf{QDialog}: Base class for dialog windows.
        \item \textbf{QFrame}: Frame widget class.
        \item \textbf{QLineEdit}: Single-line text input widget.
        \item \textbf{QTextEdit}: Rich text editing widget.
        \item \textbf{QListView}, \textbf{QTableView}, \textbf{QTreeView}: For displaying data in list, table, and tree formats.
        \item \textbf{QComboBox}: Combines a button with a dropdown list.
    \end{itemize}

    \bigbreak \noindent 
    \subsection{Layout Classes}
    \begin{itemize}
        \item \textbf{QLayout}: Base class for layouts.
        \item \textbf{QHBoxLayout}, \textbf{QVBoxLayout}: Horizontal and vertical box layouts.
        \item \textbf{QGridLayout}: Grid layout class.
        \item \textbf{QStackedLayout}: Stacking widgets layout.
    \end{itemize}

    \bigbreak \noindent 
    \subsection{Event and IO Classes}
    \begin{itemize}
        \item \textbf{QTimer}: Timer class.
        \item \textbf{QEventLoop}: Event loop manager.
        \item \textbf{QIODevice}: Base class for IO devices.
        \item \textbf{QFile}, \textbf{QDataStream}, \textbf{QTextStream}: File and data stream classes.
    \end{itemize}

    \bigbreak \noindent 
    \subsection{Networking Classes}
    \begin{itemize}
        \item \textbf{QNetworkAccessManager}: Network operations class.
        \item \textbf{QTcpSocket}, \textbf{QUdpSocket}: TCP and UDP socket classes.
        \item \textbf{QNetworkRequest}, \textbf{QNetworkReply}: Network request and response classes.
    \end{itemize}

    \bigbreak \noindent 
    \subsection{Model/View Classes}
    \begin{itemize}
        \item \textbf{QAbstractItemModel}, \textbf{QStandardItemModel}: Base classes for item models.
        \item \textbf{QAbstractListModel}, \textbf{QAbstractTableModel}: List and table model classes.
    \end{itemize}

    \bigbreak \noindent
    \subsection{Graphics View Framework}
    \begin{itemize}
        \item \textbf{QGraphicsItem}, \textbf{QGraphicsScene}, \textbf{QGraphicsView}: Classes for 2D graphics.
    \end{itemize}

    \bigbreak \noindent 
    \subsection{Multimedia Classes}
    \begin{itemize}
        \item \textbf{QMediaPlayer}, \textbf{QAudioOutput}: Multimedia handling classes.
    \end{itemize}

    \bigbreak \noindent 
    \subsection{Utility Classes}
    \begin{itemize}
        \item \textbf{QProcess}: External program execution class.
        \item \textbf{QThread}: Threading support class.
    \end{itemize}

    \bigbreak \noindent 
    \subsection{Other Core Classes}
    \begin{itemize}
        \item \textbf{QApplication}: Manages application-wide resources.
        \item \textbf{QObject}: Base class for many Qt classes.
    \end{itemize}

    \pagebreak 
    \unsect{Other QWidget derived objects}
    \begin{itemize}
        \item \textbf{QMainWindow}: Main window class, providing a framework for building an application's main user interface.
        \item \textbf{QDialog}: Base class for dialog windows, used for creating modal or modeless dialogs.
        \item \textbf{QFrame}: Frame class, used to provide a frame and a background for other widgets.
        \item \textbf{QLabel}: Label widget, used for displaying text or images.
        \item \textbf{QPushButton}: Button widget, commonly used for receiving user inputs like clicks.
        \item \textbf{QLineEdit}: Single-line text editing widget, allowing user input of text strings.
        \item \textbf{QTextEdit}: Rich text editing widget, capable of displaying and editing formatted text.
        \item \textbf{QComboBox}: Combines a line edit for editing and a dropdown list for selecting text items.
        \item \textbf{QCheckBox}: Checkbox widget, providing an option that can be checked or unchecked.
        \item \textbf{QRadioButton}: Radio button widget, used for selecting one of a set of options.
        \item \textbf{QSlider}: Slider widget, used for selecting a value from a range.
        \item \textbf{QSpinBox}: Spin box widget, allowing selection of a value from a range of values.
        \item \textbf{QTabWidget}: Tab widget, used for stacking multiple widgets and allowing navigation between them via tabs.
        \item \textbf{QListWidget}, \textbf{QTreeWidget}, \textbf{QTableWidget}: High-level item-based widgets for displaying lists, trees, and tables of data.
        \item \textbf{QGroupBox}: Group box widget, used to group collections of widgets.
        \item \textbf{QToolBar}: Toolbar widget, providing a set of tool buttons.
        \item \textbf{QStatusBar}: Status bar widget, used for displaying status information.
        \item \textbf{QProgressBar}: Progress bar widget, for giving feedback about the progress of a task.
        \item \textbf{QGraphicsView}: Widget for displaying contents of a QGraphicsScene.
    \end{itemize}





    
\end{document}
