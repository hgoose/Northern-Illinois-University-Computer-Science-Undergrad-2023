\documentclass{report}

\input{~/dev/latex/template/preamble.tex}
\input{~/dev/latex/template/macros.tex}

\title{\Huge{Chapter 3 Test Prep}}
\author{\huge{Nathan Warner}}
\date{\huge{March 20, 2023}}
\pagestyle{fancy}
\fancyhf{}
\rhead{TEST PREP}
\lhead{\leftmark}
\cfoot{\thepage}
% \usepackage[default]{sourcecodepro}
% \usepackage[T1]{fontenc}

\pgfpagesdeclarelayout{boxed}
{
  \edef\pgfpageoptionborder{0pt}
}
{
  \pgfpagesphysicalpageoptions
  {%
    logical pages=1,%
  }
  \pgfpageslogicalpageoptions{1}
  {
    border code=\pgfsetlinewidth{1.5pt}\pgfstroke,%
    border shrink=\pgfpageoptionborder,%
    resized width=.95\pgfphysicalwidth,%
    resized height=.95\pgfphysicalheight,%
    center=\pgfpoint{.5\pgfphysicalwidth}{.5\pgfphysicalheight}%
  }%
}

\pgfpagesuselayout{boxed}

\begin{document}
    \maketitle
    \pagebreak \bigbreak \noindent
    \begin{Large}
        \begin{mdframed}
            \begin{center}
                \textbf{3.1}
            \end{center}
        \end{mdframed}
    \end{Large}
    \begin{Large}
        \begin{center}
            \textbf{Differential Rule}
        \end{center}
    \end{Large}
    \line(1,0){490}
    
    \bigbreak \noindent 
    \begin{mdframed}
        \begin{itemize}
           \item $\frac{d}{dx}e^{x} = e^{x}$ 
            \item $\frac{d}{dx}c \cdot f(x) = c \cdot \frac{d}{dx}f(x)$ 
            \item $m_{tangent} \cdot m_{normal} = -1 $
                \begin{itemize}
                    \item if $m_{tangent} = 8$
                    \item then  $m_{normal} = -\frac{1}{8}$
                \end{itemize}
            \item $v(t) = f^{\prime}(x)$
            \item $a(t) = f^{\prime\prime}(x)$ 
        \end{itemize}
    \end{mdframed}

    \bigbreak \noindent
    \begin{Large}
        \begin{mdframed}
            \begin{center}
                \textbf{3.2}
            \end{center}
        \end{mdframed}
    \end{Large}
    \begin{Large}
        \begin{center}
            \textbf{Product and quotient rule}
        \end{center}
    \end{Large}
    \line(1,0){490}

    \bigbreak \noindent 
    \begin{mdframed}
        \begin{itemize}
            \item $ \frac{d}{dx}[f(x) \cdot g(x)] = f(x) \frac{d}{dx}[g(x)] + g(x) \frac{d}{dx}[f(x)]$
            \item $ \frac{d}{dx}\bigg[ \frac{f(x)}{g(x)}\bigg] = \frac{g(x) \frac{d}{dx}[f(x)] - f(x) \frac{d}{dx}[g(x)]}{[g(x)]^2}$ 
        \end{itemize}
    \end{mdframed}

    \bigbreak \noindent 
    \begin{Large}
        \begin{mdframed}
            \begin{center}
                \textbf{3.3}
            \end{center}
        \end{mdframed}
    \end{Large}
    \begin{Large}
        \begin{center}
            \textbf{Derivatives of trig functions}
        \end{center}
    \end{Large}
    \line(1,0){490}
    
    \bigbreak \noindent 
    \begin{mdframed}
        \textbf{Pythag Identities}
        \begin{itemize}
            \item $\sin^{2}{\theta} = 1-\cos^{2}{\theta }$ 
            \item $\cos^{2}{\theta } = 1 - \sin^{2}{\theta}$
            \item $\sin^{2}{\theta} + \cos^{2}{\theta } =1$
        \end{itemize}
    \end{mdframed}
    \bigbreak \noindent 
    \begin{mdframed}
        \textbf{Limit Defs}
        \begin{itemize}
            \item $ \lim_{\theta \to 0}{\frac{\sin{\theta}}{\theta } = 1}$
            \item $\lim_{\theta  \to 0}{\frac{\cos{\theta } -1}{\theta }} = 0 $
        \end{itemize}
    \end{mdframed}

    \bigbreak \noindent 
    \begin{mdframed}
        \textbf{Deriviatives of trig functions:}
          \begin{itemize}
            \item $ \frac{d}{dx}( \sin{x}) = \cos{x}$
            \item $ \frac{d}{dx}( \cos{x}) = - \sin{x}$
            \item $ \frac{d}{dx}( \tan{x}) = \sec^2{x}$
            \item $ \frac{d}{dx}( \csc{x}) =-\csc{x}\cot{x}$
            \item $ \frac{d}{dx}( \sec{x}) =\sec{x}\tan{x}$
            \item $ \frac{d}{dx}( \cot{x}) =-\csc^2{x}$
          \end{itemize}
    \end{mdframed}

    \bigbreak \noindent 
    \begin{Large}
        \begin{mdframed}
            \begin{center}
                \textbf{3.4}
            \end{center}
        \end{mdframed}
    \end{Large}
    \begin{Large}
        \begin{center}
            \textbf{chain rule}
        \end{center}
    \end{Large}
    \line(1,0){490}
    
    \bigbreak \noindent 
    \begin{mdframed}
        \begin{itemize}
           \item \textbf{Know the chain rule}
           \item $\frac{d}{dx}a^{x} = a^{x}\cdot \ln{a}$ 
        \end{itemize}
    \end{mdframed}

    \bigbreak \noindent 
    \begin{Large}
        \begin{mdframed}
            \begin{center}
                \textbf{3.5}
            \end{center}
        \end{mdframed}
    \end{Large}
    \begin{Large}
        \begin{center}
            \textbf{Implicit Differentation/derivitives of inverse trig functions}
        \end{center}
    \end{Large}
    \line(1,0){490}
    
   \bigbreak \noindent  
   \begin{mdframed}
       \begin{itemize}
           \item know how to use implicit Differentation
       \end{itemize}
   \end{mdframed}

   \pagebreak \bigbreak \noindent
   \begin{mdframed}
       \begin{itemize}
          \item $\frac{d}{dx}(\sin^{-1}{x}) = \frac{1}{\sqrt{1-x^{2}}}$ 
          \item $\frac{d}{dx}(\cos^{-1}{x}) = -\frac{1}{\sqrt{1-x^{2}}}$ 
          \item $\frac{d}{dx}(\tan^{-1}{x}) = \frac{1}{1+x^{2}}$ 
          \item $\frac{d}{dx}(\csc^{-1}{x}) = -\frac{1}{x\sqrt{x^{2}-1}}$ 
          \item $\frac{d}{dx}(\sec^{-1}{x}) = \frac{1}{x\sqrt{x^{2}-1}}$ 
          \item $\frac{d}{dx}(\cot^{-1}{x}) = -\frac{1}{1+x^{2}}$ 
        \end{itemize}
    \end{mdframed}

    \bigbreak \noindent 
    \begin{Large}
        \begin{mdframed}
            \begin{center}
                \textbf{3.6}
            \end{center}
        \end{mdframed}
    \end{Large}
    \begin{Large}
        \begin{center}
            \textbf{Deriviatives of log functions}
        \end{center}
    \end{Large}
    \line(1,0){490}
    
    \bigbreak \noindent 
    \begin{mdframed}
        \item $\frac{d}{dx}\ln{x} = \frac{1}{x}$ 
        \item $\frac{d}{dx}\log_a{x} = \frac{1}{x\ln{a}}$
    \end{mdframed}

    \bigbreak \noindent 
    \begin{mdframed}
        \textbf{Logarithmic Differentation}
        \begin{enumerate}
            \item Take ln of both sides
            \item Differentiate implicitly with respect to x
            \item solve for $y^{\prime}$
        \end{enumerate}
    \end{mdframed}
    
    \bigbreak \noindent 
    \begin{Large}
        \begin{mdframed}
            \begin{center}
                \textbf{3.7}
            \end{center}
        \end{mdframed}
    \end{Large}
    \begin{Large}
        \begin{center}
            \textbf{Rates of change in natural and social sciences}
        \end{center}
    \end{Large}
    \line(1,0){490}

    \bigbreak \noindent 
    \begin{mdframed}
        \begin{itemize}
            Know how to solve these problems  
        \end{itemize}
    \end{mdframed}

    \bigbreak \noindent 
    \begin{Large}
        \begin{mdframed}
            \begin{center}
                \textbf{3.8}
            \end{center}
        \end{mdframed}
    \end{Large}
    \begin{Large}
        \begin{center}
            \textbf{Exponential Growth and decay}
        \end{center}
    \end{Large}
    \line(1,0){490}
    
   \bigbreak \noindent  
   \begin{mdframed}
       \begin{itemize}
           \item $y=Ce^{kt}$
                \begin{itemize}
                    \item y\ =\ population 
                    \item C\ =\ initial value
                    \item k\ =\ relative growth rate
                    \item
                \end{itemize}
       \end{itemize}

   \end{mdframed}

   \bigbreak \noindent 
   \begin{mdframed}
       \textbf{Newton's law of coolig}
        \begin{itemize}
            \item $T(t) = t_s + Ce^{kt}$ 
                \begin{itemize}
                    \item $C = t_0 - t_s$
                \end{itemize}
        \end{itemize}
   \end{mdframed}

   \bigbreak \noindent 
   \begin{Large}
       \begin{mdframed}
           \begin{center}
               \textbf{3.9}
           \end{center}
       \end{mdframed}
   \end{Large}
   \begin{Large}
       \begin{center}
           \textbf{Related rates}
       \end{center}
   \end{Large}
   \line(1,0){490}

   \bigbreak \noindent 
   \begin{mdframed}
       \begin{itemize}
           \item Know how to solve these problems
       \end{itemize}
   \end{mdframed}

   \bigbreak \noindent 
   \begin{Large}
       \begin{mdframed}
           \begin{center}
               \textbf{3.10}
           \end{center}
       \end{mdframed}
   \end{Large}
   \begin{Large}
       \begin{center}
           \textbf{Linear Approx and Differentials}
       \end{center}
   \end{Large}
   \line(1,0){490}
   
   \bigbreak \noindent 
   \begin{mdframed}
       \begin{itemize}
           \item $L(x) = f(a) - f^{\prime}(a)(x-a)$
       \end{itemize}
   \end{mdframed}
   
    

    
    
    
    
    
\end{document}
