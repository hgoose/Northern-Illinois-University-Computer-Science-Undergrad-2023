\documentclass{report}

\input{~/dev/latex/template/preamble.tex}
\input{~/dev/latex/template/macros.tex}

\title{\Huge{2.7 Hw Solutions}}
\author{\huge{Nathan Warner}}
\date{\huge{}}

\pgfpagesdeclarelayout{boxed}
{
  \edef\pgfpageoptionborder{0pt}
}
{
  \pgfpagesphysicalpageoptions
  {%
    logical pages=1,%
  }
  \pgfpageslogicalpageoptions{1}
  {
    border code=\pgfsetlinewidth{1.5pt}\pgfstroke,%
    border shrink=\pgfpageoptionborder,%
    resized width=.95\pgfphysicalwidth,%
    resized height=.95\pgfphysicalheight,%
    center=\pgfpoint{.5\pgfphysicalwidth}{.5\pgfphysicalheight}%
  }%
}

\pgfpagesuselayout{boxed}

\begin{document}
    \maketitle
    \begin{Large}
        \noindent \textbf{Question 1:}
    \end{Large}
    \bigbreak \noindent 
    \pf{Solution}{}
    \bigbreak \noindent 
    \textbf{a.)}
    \begin{center}
        If:
    \end{center}
    \begin{align*}
        m_{tan} = \lim\limits_{x \to -3}{ \frac{f(x) - f(a)}{x - a}}
    .\end{align*}
    \begin{center}
        And \textbf{a = -3, f(a) = -18, then:}
    \end{center}
    \begin{align*}
        m_{tan} = \lim\limits_{x \to -3}{ \frac{x^2+9x- \left(-18\right)}{x - \left(-3\right)}} \\ 
        = \lim\limits_{x \to -3}{ \frac{x^2 +9x+18}{x + 3}}
    .\end{align*}
    \begin{center}
        numerator factors into:
    \end{center}
    \begin{align*}
        \lim\limits_{x \to -3}{ \frac{ \left(x+3\right) \left(x+6\right)}{x+3}}
    .\end{align*}
    \begin{center}
        Cancel out common factor:
    \end{center}
    \begin{align*}
        \lim\limits_{x  \to -3}{x+6} 
    .\end{align*}
    \begin{center}
        Plug in -3 for x
    \end{center}
    \begin{align*}
        m_{tan} = -3 + 6 \\ 
        = 3
    .\end{align*}

    \bigbreak \noindent \bigbreak \noindent 
    \textbf{b.)} 
    \bigbreak \noindent 
    \noindent If:
    \begin{align*}
        m_{tan} = \lim\limits_{h \to 0}{ \frac{f(a + h) - f(a)}{h}}
    .\end{align*}
    \bigbreak \noindent And a = -3 and f(a) = -18, and we plug in (a+h) for x:
    \begin{align*}
        m_{tan} = \lim\limits_{h \to 0}{ \frac{ \left(-3 + h\right)^2 + 9 \left(-3 + h\right) - \left(-18\right)}{h}} \\ 
    .\end{align*}
    \bigbreak \noindent 
    And we distribute out the terms:
    \begin{align*}
        m_{tan} = \lim\limits_{h \to 0}{ \frac{h^2-6h+9-27+9h-18}{h}} \\
        = \frac{ h^2 + 3h}{h} \\ 
        = \frac{h \left(h+3\right)}{h} \\ 
        = h + 3
    .\end{align*}
    \bigbreak \noindent 
    Now if we plug in zero: 
    \begin{align*}
        m_{tan} = 0 + 3 \\
        = 3
    .\end{align*}

    \bigbreak \noindent \bigbreak \noindent 
    \textbf{c.)} 
    The equation of the tangent line if:
    \begin{align*}
        y - y_1 = m \left(x - x_1\right) \\ 
    .\end{align*}
    \bigbreak \noindent 
    Then:
    \begin{align*}
         y - \left(-18\right) = 3 \left(x - \left(-3\right)\right) \\
         y + 18 = 3 \left(x+3\right) \\ 
         y + 18 = 3x + 9 \\ 
         y = 3x - 9
    .\end{align*}


    \bigbreak \noindent \bigbreak \noindent \bigbreak \noindent 
    \begin{Large}
        \textbf{Question 2:}
    \end{Large}
    \bigbreak \noindent 
    \pf{Solution}{}
    \bigbreak \noindent 
    We know that:
    \begin{align*}
        m_{tan} = \lim\limits_{x \to a}{\frac{f(x) - f(a)}{x -a}}
    .\end{align*}
    \bigbreak \noindent 
    And a = 6 and f(a) = 19:
    \begin{align*}
        \lim\limits_{x \to 6}{ \frac{2x^2-9x-1-19}{x-19}} \\
        = \lim\limits_{x \to 6}{ \frac{2x^2 -9x -18}{x - 19}}
    .\end{align*}
    \bigbreak \noindent 
    factor using the x method:
    \begin{align*}
        \lim\limits_{x \to 6}{ \frac{ \left(2x+3\right) \left(x-6\right)}{x-6}} \\ 
        = \lim\limits_{x \to 6}{2x+3} \\ 
        = 2 \left(6\right) + 3 \\ 
        =15
    .\end{align*}
    \bigbreak \noindent 
    Plug $m_{tan} = 15$ into \textbf{\textit{Point slope form equation}} to get equation of tangent line:

    \begin{align*}
        y - 19 = 15 \left(x-6\right) \\ 
        y - 19 = 15x - 90 \\ 
        y = 15x - 71
    .\end{align*}

    \bigbreak \noindent \bigbreak \noindent \bigbreak \noindent 
    \begin{Large}
        \textbf{Question 3:}
    \end{Large}
    \bigbreak \noindent 
    \pf{Solution}{}
    \bigbreak \noindent 
    \textbf{a.)}
    \bigbreak \noindent If:
    \begin{align*}
        m_{tan} = \lim\limits_{h \to 0}{ \frac{f(a+h) - f(a)}{h}}
    .\end{align*}
    \bigbreak \noindent Then:
    \begin{align*}
        m_{tan} = \lim\limits_{h \to 0}{ \frac{4 + 5\left(a+h\right)^2 - 2 \left(a+h\right)^3 - \left(4+5a^2-2a^3\right)}{h}}
    .\end{align*}
    \bigbreak \noindent     
    \bigbreak \noindent 
    Distribute -1 to each term in $4+5a^2-2a^3$
    \begin{align*}
        = -4 -5a^2+2a^3
    .\end{align*}
    \bigbreak \noindent 
    Foil out $-2 \left(a+h\right)^3$
    \begin{align*}
        = -2a^3-2h^3-6a^2h-6ah^2
    .\end{align*}
    \bigbreak \noindent 
    Foil out $5 \left(a+h\right)^2$
    \begin{align*}
        = 5a^2+5h^2+10ah
    .\end{align*}
    \bigbreak \noindent 
    And we also have the 4 in the beginning, so combine like terms

    \begin{align*}
        -2h^3-6a^2h-6ah^2+5h^2+10ah
    .\end{align*}
    \bigbreak \noindent 
    Add to equation:
    \begin{align*}
        \lim\limits_{h \to 0}{ \frac{-2h^3-6a^2h-6ah^2+5h^2+10ah}{h}}
    .\end{align*}
    \bigbreak \noindent 
    factor out a \textbf{\textit{h}}, and cancel out common term \textbf{\textit{h}}
    \begin{align*}
        \lim\limits_{h \to 0}{ \frac{h \left(-2h^2-6a^2-6ah+5h+10a\right)}{h}} \\
        =-2h^2-6a^2-6ah+5h+10a 
    .\end{align*}
    \bigbreak \noindent 
    Plug in zero for each h
    \begin{align*}
        -2 \left(0\right)^2-6a^2-6a \left(0\right) + 5 \left(0\right)+10a \\
        = -6a^2+10a
    .\end{align*}
    \bigbreak \noindent \bigbreak \noindent 
    \textbf{b.)}
    Plug in 1 for x,
    \begin{align*}
        m = -6 \left(1\right)^2 + 10 \left(1\right) \\ 
        = 4
    .\end{align*}
    \bigbreak \noindent 
    Plug into point slope form equation 
    \begin{align*}
        y - 7 = 4 \left(x - 1\right) \\ 
        y = 4x+3
    .\end{align*}

    \bigbreak \noindent \bigbreak \noindent \bigbreak \noindent 
    \begin{Large}
       \textbf{Question 4:}
    \end{Large}
    \bigbreak \noindent 
    \pf{Solution}{}
    \bigbreak \noindent 
    \textbf{Part b.)} 
    \begin{align*}
        16t^2 = 36 \\ 
         t^2 = \frac{36}{16} \\ 
         t = \frac{ \sqrt{36}}{ \sqrt{16}} \\
         t = \frac{6}{4} \\ 
         t = \frac{3}{2} \\
         t = 1.5 s 
    .\end{align*}
    \bigbreak \noindent 
    \textbf{Part d.)}
    \begin{align*}
        Formula = v_{inst} = \lim\limits_{h \to 0}{ \frac{f(a+h) - f(a)}{h}} \\ 
        = \lim\limits_{h \to 0}{ \frac{16(1.5+h)^2 - 16(1.5)^2}{h}} \\
        = \lim\limits_{h \to 0}{ \frac{16(h^2 + 3h + 2.25) - 36}{h}} \\ 
        = \lim\limits_{h \to 0}{ \frac{16h^2 + 48h + 36 - 36}{h}} \\ 
        = \lim\limits_{d \to 0}{ \frac{16h^2+48h}{h}} \\
        = \lim\limits_{h \to 0}{ \frac{h(16h + 48)}{h}} \\ 
        = \lim\limits_{h \to 0}{16h + 48} \\ 
        = 16(0) + 48 \\ 
        = 48
    .\end{align*}

    \bigbreak \noindent \bigbreak \noindent \bigbreak \noindent 
    \begin{Large}
        \textbf{Question 5:}
    \end{Large}
    \bigbreak \noindent 
    \pf{Solution}{}
    \bigbreak \noindent 
    \textbf{Part a.)}
    \begin{align*}
        t = a
    .\end{align*}
    \begin{align*}
        v_{inst} = \lim\limits_{t \to a}{ \frac{ \frac{6}{t^2} - \frac{6}{a^2}}{t-a}} \\
    .\end{align*}
    \bigbreak \noindent 
    \textit{Multiply by lcd $a^2t^2$}
    \begin{align*}
        \lim\limits_{t \to a}{ \frac{ \frac{6a^2t^2}{t^2} - \frac{6a^2t^2}{a^2}}{a^2t^2(t-a)}} \\ 
        = \lim\limits_{t \to a}{ \frac{6a^2 - 6t^2}{a^2t^2(t-a)}} \\ 
        \lim\limits_{t \to a}{ \frac{6(a^2-t^2)}{a^2t^2(t - a)}} \\ 
        = \lim\limits_{t \to a}{ \frac{6(a-t)(a+t)}{a^2t^2(t-a )}} \\ 
        = \lim\limits_{t \to a}{ \frac{-6(t-a)(t+a)}{a^2t^2(t-a)}} \\ 
        = \lim\limits_{t \to a}{ \frac{-6(t+a)}{a^2t^2}} \\ 
    .\end{align*}
    \bigbreak \noindent 
    \textit{Plug in a for t}
    \begin{align*}
        \frac{-6(a+a)}{a^2a^2} \\ 
        = \frac{-6a-6a}{a^4} \\ 
        = \frac{-12a}{a^4} \\ 
        = \frac{-12}{a^3}
    .\end{align*}
    \bigbreak \noindent 
    We can use this equation to get parts b-d, but instead here is work if we didnt have the equation above (:
    \bigbreak \noindent 
    \textbf{Part b.)}
    \begin{align*}
        t=1  
    .\end{align*}
    \begin{align*}
        v_{inst} = \lim\limits_{t \to 1}{ \frac{f(t) - f(a)}{t - a}} \\
        = \lim\limits_{t \to 1}{ \frac{ \frac{6}{t^2} - \frac{6}{(-1)^2}}{t-1}} \\ 
        = \lim\limits_{t \to 1}{ \frac{ \frac{6}{t^2} - 6}{t - 1}}
    .\end{align*}
    \bigbreak \noindent 
    \textit{Clear out fraction in numerator by Multiplying by lcd}

    \begin{align*}
        \lim\limits_{t \to 1}{ \frac{ (\frac{6}{t^2} \cdot \frac{t^2}{1}) - (\frac{6}{1} \cdot \frac{t^2}{1})}{t^2(t-1)}} \\
        = \lim\limits_{t \to 1}{ \frac{6-6t^2}{t^2(t-1)}} \\
        = \lim\limits_{t \to 1}{ \frac{-6(t^2-1)}{t^2(t-1)}} \\
        = \lim\limits_{t \to 1}{ \frac{-6(t-1)(t+1)}{t^2(t-1)}} \\ 
        = \lim\limits_{t \to 1}{-6(t+1)} \\ 
        = -6(1+1) \\ 
        = -12
    .\end{align*}

    \bigbreak \noindent 
    \textbf{c.)} 
    \begin{align*}
        t= 2
    .\end{align*}
    \bigbreak \noindent 
    \begin{align*}
        \lim\limits_{t \to 2 }{ \frac{f(t)- f(a)}{t -1}} \\
        \lim\limits_{t \to 2}{ \frac{ \frac{6}{t^2} - \frac{6}{4}}{t-2}} \\ 
        \lim\limits_{t \to 2}{ \frac{ \frac{6}{t^2} - \frac{3}{2}}{t-2}} \\
    .\end{align*}
    \bigbreak \noindent 
    \textit{Multiply by lcd of $2t^2$} 
    \bigbreak \noindent 
    \begin{align*}
        \lim\limits_{t \to 2}{ \frac{12 - 3t^2}{2t^2(t-2)}} \\ 
        = \lim\limits_{t \to 2}{ \frac{-3(t^2-4)}{2t^2(t-2)}} \\ 
        = \lim\limits_{t \to 2}{ \frac{-3(t-2)(t+2)}{2t^2(t-2)}} \\ 
        = \lim\limits_{t \to 2}{ \frac{-3(t+2)}{2t^2}} \\
        \frac{-2(2+2)}{2(2)^2} \\ 
        = -1.5
    .\end{align*}
    

    \bigbreak \noindent \bigbreak \noindent \bigbreak \noindent 
    \begin{Large}
        \textbf{Question 6:}
    \end{Large}
    \bigbreak \noindent 
    \pf{Solution}{}
    \bigbreak \noindent 
    \textbf{Part a.)}
    (10,400), (60,750)
    \begin{align*}
        m_{pq} = \frac{750-400}{60-10} \\
        = \frac{350}{50} \\ 
        = 7
    .\end{align*}
    \bigbreak \noindent 
    \textbf{Part c.)}
    \begin{align*}
        \frac{200-600}{40-0} \\ 
        = \frac{-400}{40} \\ 
        = -10
    .\end{align*}
    \bigbreak \noindent 
    \textbf{d.)}
    \begin{align*}
        f\prime (50)
    .\end{align*}
    \bigbreak \noindent 
    So drawn at tangent line at point (50,f(50)), and calculate the slope, we can see we have another point on the tangent line
    at (60, f(60))
    \begin{align*}
        \frac{(60,f(60)) - (50,f(50))}{60-50} \\ 
        = \frac{600-400}{60-50} \\ 
        = 20
    .\end{align*}


    \bigbreak \noindent \bigbreak \noindent \bigbreak \noindent 
    \begin{Large}
        \textbf{Question 7:}
    \end{Large}
    \bigbreak \noindent 
    \pf{Solution}{}
    \bigbreak \noindent 
    \textit{Formula:}
    \begin{align*}
        f\prime (a) = \lim\limits_{x \to a}{ \frac{f(x) - f(a)}{x-a}}
    .\end{align*}
    \bigbreak \noindent 
    \textit{If a = 7}
    \begin{align*}
        \lim\limits_{x \to 7}{ \frac{ \sqrt{6x+7} - ( \sqrt{6(7)+7})}{x-7}} \\
        \lim\limits_{x \to 7}{ \frac{ \sqrt{6x+7} - 7}{x-7}}
    .\end{align*}
    \bigbreak \noindent 
    \textit{Multiply by the conjugate:}
    \begin{align*}
        \lim\limits_{x \to 7}{ \frac{ \sqrt{6x+7} -7}{x-7}} \cdot \frac{ \sqrt{6x+7+7}}{ \sqrt{6x+7+7}} \\
        = \lim\limits_{x \to 7}{ \frac{6x-42}{(x-7) ( \sqrt{6x+7} + 7)}} \\
        = \lim\limits_{x \to 7}{ \frac{6(x-7)}{(x-7) ( \sqrt{6x+7} + 7)}} \\
        = \lim\limits_{x \to 7}{ \frac{6}{ \sqrt{6x+7}+7}}
    .\end{align*}
    \bigbreak \noindent 
    \textit{Plug in 7 for x:}
    \begin{align*}
        \frac{6}{ \sqrt{6(7)+7}+7} \\
        = \frac{3}{7}
    .\end{align*}


    \bigbreak \noindent \bigbreak \noindent \bigbreak \noindent 
    \begin{Large}
        \textbf{Question 8:}
    \end{Large}
    \bigbreak \noindent 
    \pf{Solution}{}
    \bigbreak \noindent 
    \begin{align*}
        f\prime(a) = \lim\limits_{h \to 0}{ \frac{f(a+h) - f(a)}{h}}
    .\end{align*}
    \bigbreak \noindent 
    \textit{So:}
    \begin{align*}
        \lim\limits_{h \to 0}{ \frac{2(a+h)^2-3(a+h)+3-[2a^2-3a+3]}{h}} \\ 
        = \lim\limits_{h \to 0}{\frac{2a^2+2h^2+4ah+3-2a^2+3a-3-3a-3h}{h}} \\
        = \lim\limits_{h \to 0}{\frac{2h^2-3h+4ah}{h}} \\ 
        = \lim\limits_{h \to 0}{\frac{h(h-3+4a)}{h}} \\
        = \lim\limits_{h \to 0}{h+4a-3} \\ 
        = 0+4a-3 \\
        = 4a-3
    .\end{align*}

    \bigbreak \noindent \bigbreak \noindent \bigbreak \noindent 
    \begin{Large}
        \textbf{Question 9:}
    \end{Large}
    \bigbreak \noindent 
    \pf{Solution}{}
    \bigbreak \noindent 
    \textit{Review from on HW}

    \bigbreak \noindent \bigbreak \noindent \bigbreak \noindent 
    \begin{Large}
        \textbf{Question 10:}
    \end{Large}
    \bigbreak \noindent 
    \pf{Solution}{}
    \bigbreak \noindent 
    \textit{Formula:}
    \begin{align*}
        m_{tan} = \lim\limits_{x \to a}{ \frac{f(x) -f(a)}{x-a}}
    .\end{align*}
    \bigbreak \noindent 
    \textit{Therefore:}
    \begin{align*}
        \lim\limits_{x \to 8}{ \frac{2x-5 - [2(8)-5]}{x-8}} \\
        \lim\limits_{x \to 8}{ \frac{2x-5 -11}{x-8}} \\
        \lim\limits_{x \to 8}{ \frac{2x-16}{x-8}} \\
        \lim\limits_{x \to 8}{ \frac{2(x-8)}{x-8}} \\
        = 2
    .\end{align*}
    \bigbreak \noindent 
    Since we have no more x value to plug 8 into, our m is just \textbf{\textit{2}}
    

\end{document}
