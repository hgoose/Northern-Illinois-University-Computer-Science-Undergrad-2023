\documentclass{report}

\input{~/dev/latex/template/preamble.tex}
\input{~/dev/latex/template/macros.tex}

\title{\Huge{}}
\author{\huge{Nathan Warner}}
\date{\huge{}}
\pagestyle{fancy}
\fancyhf{}
\lhead{Warner \thepage}
\rhead{}
% \lhead{\leftmark}
\cfoot{\thepage}
\setborder
% \usepackage[default]{sourcecodepro}
% \usepackage[T1]{fontenc}

\begin{document}
    % \maketitle
        \begin{titlepage}
       \begin{center}
           \vspace*{1cm}
    
           \textbf{NIU CS240} \\
           Computer Programming In CPP
    
           \vspace{0.5cm}
            
                
           \vspace{1.5cm}
             
           \textbf{Nathan Warner}
    
           \vfill
                
                
           \vspace{0.8cm}
         
           \includegraphics[width=0.4\textwidth]{~/niu/seal.png}
                
           Computer Science \\
           Northern Illinois University\\
           August 28, 2023
           United States\\
           
                
       \end{center}
    \end{titlepage}
    \tableofcontent 
    \pagebreak \bigbreak \noindent 
    \vspace{2in} \\
    \begin{Huge}
       \textbf{Computer Programming \\
       In CPP} 
    \end{Huge}
    \bigbreak \noindent 
    \line(1,0){490}
    
    \bigbreak \noindent \bigbreak \noindent 
    \section{Lecture 1}
    \bigbreak \noindent 
    \subsection{Data}
    \begin{itemize}
        \item There are several data types (numbers, characters, etc)
        \item individual data items must be declared and named - this is known as creating a variable
        \item values that are put into variables can come from
            \begin{itemize}
                \item program instructions 
                \item user input
                \item files
            \end{itemize}
        \item program instructions can alter these values
        \item original or newly computer values can go to
            \begin{itemize}
                \item screen
                \item printer
                \item disk
            \end{itemize}
    \end{itemize}
    \bigbreak \noindent 
    \textbf{Instructions:}
    \begin{itemize}
        \item for data input (from keyboard, disk)
        \item for data output (to screen, printer, disk)
        \item computation of new values
        \item program control (decisions, repetition)
        \item modularization (putting a sequence of instructions into a package called a function)
    \end{itemize}

    \bigbreak \noindent \bigbreak \noindent 
    \subsection{The Language}
    \bigbreak \noindent 
    The C++ language is made up of 
    \begin{itemize}
        \item keywords/reserved words (if, while, int, etc.)
        \item symbols: \{ \} =  |  <=  ! [ ]  *  \&  (and more)
        \item programmer-defined names for variables and functions
    \end{itemize}
    \bigbreak \noindent 
    These programmer-defined names:
    \begin{itemize}
        \item  1 - 31 chars long; use letters, digits, \_ (underscore)
        \item start with a letter or \_
        \item are case-sensitive: \textit{Num} is \textbf{different} than \textit{num}
        \item should be meaningful: \textit{studentCount} is better than s or sc
    \end{itemize}

    \bigbreak \noindent \bigbreak \noindent 
    \subsection{Data Types}
    \bigbreak \noindent 
    Each data item has a type and a name chosen by the programmer. The type determines the range of possible values it can hold as well as the operations that can be used on it. For example, you can add a number to a numeric data type, but you cannot add a number to someone's name. (What would "Joe" + 1 mean?)
    \bigbreak \noindent 
    \textit{Figure:}
    \begin{center}
    \begin{tabular}{|c|c|}
        \hline
        Type & Keyword \\
        \hline
        Boolean & bool \\
        Character & char \\
        Integer & int \\
        Floating point & float \\
        Double floating point & double \\
        Valueless & void \\
        Wide character & wchar\_t \\
        \hline
    \end{tabular}
        \end{center}
        \bigbreak \noindent 
        \nt{Floating point numbers have a limit of 6 significant figures and doubles have a limit of 12 characters.}
        \bigbreak \noindent 

        \pagebreak \bigbreak \noindent 
        \subsection{Arithemetic operators}
        \bigbreak \noindent 
        The arithmetic operators are:
        \bigbreak \noindent 
        \begin{itemize}
            \item +  addition
            \item -  subtraction or unary negation (-5)
            \item *  multiplication
            \item /  division (see special notes on division below)
            \item \%  modulus division \texttt{--} integer remainder of integer division
        \end{itemize}
        \bigbreak \noindent 
        \nt{There is no exponential operator}




    \pagebreak \bigbreak \noindent 
    \section{Assignment Statements; Control Structures; Symbolic Constants; Formatting Output}
    \pagebreak \bigbreak \noidennt

    \section{Cascading Ifs; Conditional Expressions; Compound Conditions; Data Type \texttt{bool}}


    \pagebreak \bigbreak \noidennt
    \section{Functions}
    \pagebreak \bigbreak \noidennt

    \section{Function Summary Sheet}
    \pagebreak \bigbreak \noidennt

    \section{Character Functions}
    \pagebreak \bigbreak \noidennt

    \section{Arrays}
    \pagebreak \bigbreak \noidennt

    \section{Arrays and Functions}
    \pagebreak \bigbreak \noidennt

    \section{References and Call By Reference; Input and Output}
    \pagebreak \bigbreak \noidennt

    \section{Object Oriented Programming}
    \pagebreak \bigbreak \noidennt

    \section{C++ Strings}
    \pagebreak \bigbreak \noidennt

    \section{Structures}
    \pagebreak \bigbreak \noindent 

    
\end{document}
