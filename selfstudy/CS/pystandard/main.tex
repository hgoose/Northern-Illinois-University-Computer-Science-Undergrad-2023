\documentclass{report}

\input{~/dev/latex/template/preamble.tex}
\input{~/dev/latex/template/macros.tex}

\title{\Huge{}}
\author{\huge{Nathan Warner}}
\date{\huge{}}
\pagestyle{fancy}
\fancyhf{}
\lhead{Warner \thepage}
\rhead{}
% \lhead{\leftmark}
\cfoot{\thepage}
%\setborder
% \usepackage[default]{sourcecodepro}
% \usepackage[T1]{fontenc}

\begin{document}
    % \maketitle
        \begin{titlepage}
       \begin{center}
           \vspace*{1cm}
    
           \textbf{The Python Standard Library} \\
           Methods \& Functions
    
           \vspace{0.5cm}
            
                
           \vspace{1.5cm}
    
           \textbf{Nathan Warner}
    
           \vfill
                
                
           \vspace{0.8cm}
         
           \includegraphics[width=0.4\textwidth]{~/niu/seal.png}
                
           Computer Science \\
           Northern Illinois University\\
           November 9, 2023
           United States\\
           
                
       \end{center}
    \end{titlepage}
    \tableofcontents
    \pagebreak \bigbreak \noindent
    \section{\LARGE String Methods}
    \bigbreak \noindent 
    
      \begin{center}
        \textbf{Pertains to CHANGING upper/lowercase (6)}
      \end{center}
      \bigbreak \noindent 
      \textbf{\textit{\underline{Upper Case:}}}
      \begin{itemize}
        \item[\ding{43}] \textbf{capitalize()}	Converts the first character to upper case $\longrightarrow$ \textbf{\textit{\underline{[Paramaters: Null] [Return: String]}}}
        \item[\ding{43}] \textbf{title()}		Converts the first character of each word to upper case $\longrightarrow$ \textbf{\textit{\underline{[Null] [String]}}}
        \item[\ding{43}] \textbf{upper()}	Converts a string into upper case $\longrightarrow$ \textbf{\textit{\underline{[Null] [String]}}}
      \end{itemize}

      \bigbreak \noindent 
      \textbf{\textit{\underline{Lower Case:}}}
      \begin{itemize}
        \item[\ding{43}] \textbf{casefold()}	Converts string into lower case $\longrightarrow$ \textbf{\textit{\underline{[Null] [String]}}}
        \item[\ding{43}] \textbf{lower()}	Converts a string into lower case $\longrightarrow$ \textbf{\textit{\underline{[Null] [String]}}}
      \end{itemize}

      \bigbreak \noindent 
      \textbf{\textit{\underline{Both:}}}
      \begin{itemize}
        \item[\ding{43}] \textbf{swapcase()}	Swaps cases, lower case becomes upper case and vice versa $\longrightarrow$ \textbf{\textit{\underline{[Null] [String]}}}
      \end{itemize}
    

    \bigbreak \noindent 
    
      \begin{center}
        \textbf{"Is" Methods/Returns Bool (14)}
      \end{center}
      \bigbreak \noindent 
      \begin{itemize}
        \item[\ding{43}] \textbf{isupper()} Returns True if all characters in the string are upper case $\longrightarrow$ \textbf{\textit{\underline{[Null] [bool]}}}
        \item[\ding{43}] \textbf{islower()}	Returns True if all characters in the string are lower case $\longrightarrow$ \textbf{\textit{\underline{[Null] [bool]}}}
        \item[\ding{43}] \textbf{isalnum()}	Returns True if all characters in the string are alphanumeric $\longrightarrow$ \textbf{\textit{\underline{[Null] [bool]}}}
        \item[\ding{43}] \textbf{isalpha()}	Returns True if all characters in the string are in the alphabet $\longrightarrow$ \textbf{\textit{\underline{[Null] [bool]}}}
        \item[\ding{43}] \textbf{isascii()}	Returns True if all characters in the string are ascii characters $\longrightarrow$ \textbf{\textit{\underline{[Null] [bool]}}}
        \item[\ding{43}] \textbf{isdecimal()}	Returns True if all characters in the string are decimals $\longrightarrow$ \textbf{\textit{\underline{[Null] [bool]}}}
        \item[\ding{43}] \textbf{isdigit()}	Returns True if all characters in the string are digits $\longrightarrow$ \textbf{\textit{\underline{[Null] [bool]}}}
        \item[\ding{43}] \textbf{isidentifier()}	Returns True if the string is an identifier $\longrightarrow$ \textbf{\textit{\underline{[Null] [bool]}}}
        \item[\ding{43}] \textbf{isnumeric()}	Returns True if all characters in the string are numeric $\longrightarrow$ \textbf{\textit{\underline{[Null] [bool]}}}
        \item[\ding{43}] \textbf{isprintable()}	Returns True if all characters in the string are printable $\longrightarrow$ \textbf{\textit{\underline{[Null] [bool]}}}
        \item[\ding{43}] \textbf{isspace()}	Returns True if all characters in the string are whitespaces  $\longrightarrow$ \textbf{\textit{\underline{[Null] [bool]}}}
        \item[\ding{43}] \textbf{istitle()}	Returns True if the string follows the rules of a title $\longrightarrow$ \textbf{\textit{\underline{[Null] [bool]}}}
        \item[\ding{43}] \textbf{endswith()}	Returns true if the string ends with the specified value $\longrightarrow$ \textbf{\textit{\underline{[r:Value, o:Start, o:End] [bool]}}}
        \item[\ding{43}] \textbf{startswith()}	Returns true if the string starts with the specified value $\longrightarrow$ \textbf{\textit{\underline{[r:Value, o:Start, o:End] [bool]}}}
      \end{itemize}
    

    \pagebreak \bigbreak \noindent
    
      \begin{center}
        \textbf{Searching (4)}
      \end{center}
      \begin{itemize}
        \item[\ding{43}] \textbf{find()}	Searches the string for a specified value and returns the position of where it was found 
          \smallbreak
          \textbf{\textit{\underline{[r:Value, o:Start, o:End] [Int: Pos of first occurence] [Int: -1 if not found]}}}
          \smallbreak
        \item[\ding{43}] \textbf{rfind()}	Searches the string for a specified value and returns the last position of where it was found 
          \smallbreak \noindent 
          \textbf{\textit{\underline{[r:Value, o:Start, o:End] [Int: Pos of last occurence] [Int: -1 if not found]}}}
          \smallbreak
        \item[\ding{43}] \textbf{index()}	Searches the string for a specified value and returns the position of where it was found 
          \smallbreak
          \textbf{\textit{\underline{[r:Value, o:Start, o:End] [Int: Pos of first occurence] [Throws error if not found]}}}
          \smallbreak
        \item[\ding{43}] \textbf{rindex()}	Searches the string for a specified value and returns the last position of where it was found 
          \smallbreak \noindent 
          \textbf{\textit{\underline{[r:Value, o:Start, o:End] [Int: Pos of last occurence] [Throws error if not found]}}}
        \smallbreak
      \item[\ding{43}] \textbf{count()}	Returns the number of times a specified value occurs in a string
        \smallbreak
        \textbf{\textit{\underline{[r:Value, o:Start, o:End] [Int: Num of occurences]}}}

      \end{itemize}
    

    \bigbreak \noindent 
    
      \begin{center}
        \textbf{Mutate String (11)}
      \end{center}
      \begin{itemize}
        \item[\ding{43}] \textbf{replace()} Returns a string where a specified value is replaced with a specified value
          \smallbreak
          \textbf{\textit{\underline{[r:oldvalue, r:newvalue, o:count] [String]}}}
          \smallbreak
        \item[\ding{43}] \textbf{center()} Returns a centered string, default character is " "
          \smallbreak
          \textbf{\textit{\underline{[r:length, o:character] [string]}}}
          \smallbreak
        \item[\ding{43}] \textbf{strip()}	Returns a trimmed version of the string 
          \smallbreak
          \textbf{\textit{\underline{[o:Character]}}}
        \item[\ding{43}] \textbf{lstrip()} Returns a left trim version of the string, default character is " "
          \smallbreak
          \textbf{\textit{\underline{[o:Character] [String]}}}
          \smallbreak
        \item[\ding{43}] \textbf{rstrip()} Returns a right trim version of the string, default character is " "
          \smallbreak
          \textbf{\textit{\underline{[o:Character] [String]}}}
          \smallbreak
        \item[\ding{43}] \textbf{rjust()}	Returns a right justified version of the string, defualt character is " "
          \smallbreak
          \textbf{\textit{\underline{[r:Length, o:Character] [String]}}}
          \smallbreak
        \item[\ding{43}] \textbf{ljust()}	Returns a left justified version of the string, default character is " "
          \smallbreak
          \textbf{\textit{\underline{[r:Length, o:Character] [String]}}}
          \smallbreak
        \item[\ding{43}] \textbf{zfill()}	Fills the string with a specified number of 0 values at the beginning
          \smallbreak
          \textbf{\textit{\underline{[r:length] [String]}}}
          \smallbreak
        \item[\ding{43}] \textbf{maketrans()}    Returns a translation table to be used in translations
          \smallbreak
          \textbf{\textit{\underline{[r:String, r:String, o:String[characters to remove]] [Dict]}}}
          \smallbreak
        \item[\ding{43}] \textbf{translate()}    Returns a translated string
          \smallbreak
          \textbf{\textit{\underline{[r:Table[dict]] [String]}}}
          \smallbreak
        \item[\ding{43}] \textbf{encode()}    Returns an encoded version of the string, If no encoding is specified, UTF-8 will be used.
          \smallbreak
          \textbf{\textit{\underline{[o:Encoding, o:Errors[See Web]]}}}
      \end{itemize}
    

    \bigbreak \noindent 
    
      \begin{center}
        \textbf{Returns a List/Tuple (5)}
      \end{center}
      \bigbreak \noindent 
      \textbf{\textit{\underline{Returns List:}}}
      \begin{itemize}
        \item[\ding{43}] \textbf{split()}	Splits the string at the specified separator, and returns a list
          \smallbreak
          \textbf{\textit{\underline{[o:Separator, o:Maxsplit] [List]}}}
          \smallbreak
        \item[\ding{43}] \textbf{rsplit()} Splits the string at the specified separator, and returns a list
          \smallbreak
          \textbf{\textit{\underline{[o:Separator, o:Maxsplit] [List]}}}
          \smallbreak
        \item[\ding{43}] \textbf{splitlines()}	Splits the string at line breaks and returns a list
          \smallbreak
          \textbf{\textit{\underline{[o:keeplinebreaks=Bool] [List]}}}
      \end{itemize}
      \bigbreak \noindent 
      \textbf{\textit{\underline{Returns Tuple:}}}
      \begin{itemize}
        \item[\ding{43}] \textbf{partition()}	Returns a tuple where the string is parted into three parts
          \smallbreak
          \textbf{\textit{\underline{[r:Value] [Tuple]}}}
          \smallbreak
        \item[\ding{43}] \textbf{rpartition()}	Returns a tuple where the string is parted into three parts
          \smallbreak
          \textbf{\textit{\underline{[r:Value] [Tuple]}}}
      \end{itemize}
    

    \bigbreak \noindent 
    
      \begin{center}
        \textbf{Format String (2)}
      \end{center}
      \begin{itemize}
        \item[\ding{43}] \textbf{format()}    Formats specified values in a string
          \smallbreak
          \textbf{\textit{\underline{[r:values[comma seperated list] [String]]}}}
          \smallbreak
        \item[\ding{43}] \textbf{format\_map()} Formats specified values in a string
          \smallbreak
          \textbf{\textit{\underline{[r:Dict], [String]}}}
          \smallbreak
        \item[\ding{43}] \textbf{expandtabs()}    Sets the tab size of the string
          \smallbreak
          \textbf{\textit{\underline{[r:Tabsize[int]] [String]}}}
      \end{itemize}
    

    \bigbreak \noindent 
    
      \begin{center}
        \textbf{Takes iterable and turns into string (1)} 
      \end{center}
      \begin{itemize}
        \item[\ding{43}] \textbf{join()} Converts the elements of an iterable into a string, A string must be specified as the separator \textbf{before the method call}
          \smallbreak
          \textbf{\textit{\underline{[r:Iterable[Must be strings]] [String]}}}
      \end{itemize}
      \bigbreak \noindent 
      \nt{When using a dictionary as an iterable, the returned values are the keys, not the values.}
    

    \pagebreak \bigbreak \noindent 
    \section{\LARGE List Methods}
    \bigbreak \noindent 
    
      \begin{center}
         \textbf{Add To List (2)} 
      \end{center}
      \begin{itemize}
        \item[\ding{43}] \textbf{append()}	Adds an element at the end of the list
          \smallbreak
          \textbf{\textit{[r:Element] [Mutates List]}}
          \smallbreak
        \item[\ding{43}] \textbf{insert()}	Adds an element at the specified position
          \smallbreak
          \textbf{\textit{[r:Pos, r:Element] [Mutates List]}}
          \smallbreak
        \item[\ding{43}] \textbf{extend()}	Add the elements of a list (or any iterable), to the end of the current list
          \smallbreak
          \textbf{\textit{[r:Iterable] [Mutates List]}}
      \end{itemize}
    

    \bigbreak \noindent 
    
      \begin{center}
        \textbf{Remove From list (2)}
      \end{center}
      \begin{itemize}
        \item[\ding{43}] \textbf{remove()}	Removes the first item with the specified value
          \smallbreak
          \textbf{\textit{[r:Element] [Mutates List]}}
          \smallbreak
        \item[\ding{43}] \textbf{pop()}		Removes the element at the specified position
          \smallbreak
          \textbf{\textit{[r:Pos] [Mutates List]}}
          \smallbreak
        \item[\ding{43}] \textbf{clear()}		Removes all the elements from the list
          \smallbreak
          \textbf{\textit{[Null] [Mutates List]}}
      \end{itemize}
    
    
    \pagebreak \bigbreak \noindent
    
      \begin{center}
        \textbf{Search (2)}
      \end{center}
      \begin{itemize}
        \item[\ding{43}] \textbf{index()}		Returns the index of the first element with the specified value
          \smallbreak
          \textbf{\textit{[r:Element] [Int]}}
          \smallbreak
        \item[\ding{43}] \textbf{count()}		Returns the number of elements with the specified value	
          \smallbreak
          \textbf{\textit{[r:Value] [Int]}}
      \end{itemize}
    

    
      \begin{center}
        \textbf{Other (3)}
      \end{center}
      \begin{itemize}
        \item[\ding{43}] \textbf{copy()}		Returns a copy of the list
          \smallbreak
          \textbf{\textit{[Null] [List]}}
          \smallbreak
        \item[\ding{43}] \textbf{reverse()}	Reverses the order of the list
          \smallbreak
          \textbf{\textit{[Null] [Mutates List]}}
          \smallbreak
        \item[\ding{43}] \textbf{sort()}		Sorts the list
          \smallbreak
          \textbf{\textit{[o:reverse=Bool, o:key=Func] [Mutates List]}}
      \end{itemize}
    

    
      \begin{center}
        \textbf{Bonus - Tuple Methods (2)}
      \end{center}
      \begin{itemize}
        \item[\ding{43}] \textbf{count()}	Returns the number of times a specified value occurs in a tuple
          \smallbreak
          \textbf{\textit{[r:Value] [Int]}}
          \smallbreak
        \item[\ding{43}] \textbf{index()}	Searches the tuple for a specified value and returns the position of where it was found
          \smallbreak
          \textbf{\textit{[r:Element] [Int]}}
      \end{itemize}
    

    \pagebreak \bigbreak \noindent 
    \section{\LARGE Dict Methods}
    \bigbreak \noindent 
     
        \begin{center}
            \textbf{Fetch (7)}
        \end{center}
    \begin{itemize}
        \item[\ding{43}] \textbf{fromkeys()}	Returns a dictionary with the specified keys and value
            \smallbreak
            \textbf{\textit{[r:Keys, o:Value] [Dict]}}
            \smallbreak
            \textbf{\textit{Keys: 	Required. An iterable specifying the keys of the new dictionary}}
            \smallbreak
            \textbf{\textit{Value:	Optional. The value for all keys. Default value is None}}
            \smallbreak
        \item[\ding{43}] \textbf{get()}		Returns the value of the specified key
            \smallbreak
            \textbf{\textit{[r:Keyname, o:Value] [Type of retrieved item]}}
            \smallbreak
            \textbf{\textit{Keyname: 	Required. The keyname of the item you want to return the value from}}
            \smallbreak
            \textbf{\textit{Value: Optional. A value to return if the specified key does not exist. Default value None}}
            \smallbreak
        \item[\ding{43}] \textbf{items()}		Returns a list containing a tuple for each key value pair
            \smallbreak
            \textbf{\textit{[Null] [View Object: Tuple]}}
            \smallbreak \noindent
        \item[\ding{43}] \textbf{keys()}		Returns a list containing the dictionary's keys
            \smallbreak \noindent
            \textbf{\textit{[Null] [View Object]}}
            \smallbreak
        \item[\ding{43}] \textbf{values()}	Returns a list of all the values in the dictionary
            \smallbreak \noindent
            \textbf{\textit{[Null] [View Object]}}
            \smallbreak \noindent
        \item[\ding{43}] \textbf{setdefault()}	Returns the value of the specified key. If the key does not exist: insert the key, with the specified value
            \smallbreak \noindent
            \textbf{\textit{[r:Keyname, r:Value] [Value/Mutates Dict]}}
            \smallbreak \noindent
        \item[\ding{43}] \textbf{Update} Insert an item to the dictionary
            \smallbreak \noindent
            \textbf{\textit{[r:Dict] [Mutates Dict]}}
    \end{itemize}
    

    \bigbreak \noindent 
    
        \begin{center}
            \textbf{Remove Items (3)}
        \end{center}
        \begin{itemize}
            \item[\ding{43}] \textbf{pop()}		Removes the element with the specified key
                \smallbreak \noindent
                \textbf{\textit{[r:Keyname, o:ReturnValueIfDNE] [Mutates Dict/Specified Return Value]}}
                \smallbreak \noindent
            \item[\ding{43}] \textbf{popitem()}	Removes the last inserted key-value pair
                \smallbreak \noindent
                \textbf{\textit{[Null] [Mutates list and returns tuple of removed item]}}
                \smallbreak \noindent
            \item[\ding{43}] \textbf{clear()}		Removes all the elements from the dictionary
                \smallbreak \noindent
                \textbf{\textit{[Null], [Mutates Dict]}}
        \end{itemize}
    

    \bigbreak \noindent 
    
        \begin{center}
           \textbf{Other (1)} 
        \end{center}
    \begin{itemize}
        \item[\ding{43}] \textbf{copy()} returns a copy of the specified dictionary.
            \smallbreak \noindent
            \textbf{\textit{[Null] [Dict]}}
    \end{itemize}
    

    \pagebreak \bigbreak \noindent 
    \section{\LARGE Set Methods}
    \bigbreak \noindent 
    
        \begin{center}
            \textbf{Add to set (3)}
        \end{center}
        \begin{itemize}
            \item[\ding{43}] \textbf{add()}	Adds an element to the set
                \smallbreak
                \textbf{\textit{[r:Element] [Mutates Set]}}
                \smallbreak \noindent
            \item[\ding{43}] \textbf{update()}	Update the set with another set, or any other iterable
                \smallbreak \noindent
                \textbf{\textit{[Set] [Mutates Set]}}
                \smallbreak \noindent
            \item[\ding{43}] \textbf{symmetric\_difference\_update()}	inserts the symmetric differences from this set and another
              \smallbreak \noindent
              \textbf{\textit{[r:Set] [Mutates Set]}}
        \end{itemize}
    

    \bigbreak \noindent 
    
        \begin{center}
            \textbf{Remove from set (6)}
        \end{center}
        \begin{itemize}
            \item[\ding{43}] \textbf{pop()}	Removes an element from the set
                \smallbreak \noindent
                \textbf{\textit{[Null] [Removed Element]}}
                \smallbreak \noindent
            \item[\ding{43}] \textbf{remove()}	Removes the specified element
                \smallbreak \noindent
                \textbf{\textit{[r:Item] [Mutates Set]}}
                \smallbreak \noindent
            \item[\ding{43}] \textbf{clear()}	Removes all the elements from the set
                \smallbreak \noindent
                \textbf{\textit{[Null] [Mutates Set]}}
                \smallbreak \noindent
            \item[\ding{43}] \textbf{discard()}	Remove the specified item
                \smallbreak \noindent
                \textbf{\textit{[r:Value] [Mutates Set]}}
                \smallbreak \noindent
            \item[\ding{43}] \textbf{difference\_update()}	Removes the items in this set that are also included in another, specified set
                \smallbreak \noindent
                \textbf{\textit{[r:Set] [Mutates Set]}}
                \smallbreak \noindent
              \item[\ding{43}] \textbf{intersection\_update()}	Removes the items in this set that are not present in other, specified set(s)
                \smallbreak \noindent
                \textbf{\textit{[r:Set] [Mutates Set]}}
        \end{itemize}
        \bigbreak \noindent 
        \nt{the discard() method is different from the remove() method, because the remove() method will raise an error if the specified item does not exist, and the discard() method will not.}
    

    \pagebreak \bigbreak \noindent
    
      \begin{center}
        \textbf{\textit{is} methods [Returns Bool] (3)}
      \end{center}
      \begin{itemize}
        \item[\ding{43}] \textbf{isdisjoint()}	Returns whether two sets have a intersection or not
          \smallbreak \noindent
          \textbf{\textit{[r:Set] [Bool]}}
          \smallbreak \noindent
        \item[\ding{43}] \textbf{issubset()}	Returns whether another set contains this set or not
          \smallbreak \noindent
          \textbf{\textit{[r:Set] [Bool]}}
          \smallbreak \noindent
        \item[\ding{43}] \textbf{issuperset()}	Returns whether this set contains another set or not
          \smallbreak \noindent
          \textbf{\textit{[r:Set] [Bool]}}
      \end{itemize}
    

    \bigbreak \noindent 
    
      \begin{center}
        \textbf{Returns a Set (5)}
      \end{center}
      \begin{itemize}
        \item[\ding{43}] \textbf{difference()}	Returns a set containing the difference between two or more sets
          \smallbreak \noindent
          \textbf{\textit{[r:Set] [Set]}}
          \smallbreak \noindent
        \item[\ding{43}] \textbf{intersection()}	Returns a set, that is the intersection of two or more sets
          \smallbreak \noindent
          \textbf{\textit{[r:Set] [Set]}}
          \smallbreak \noindent
        \item[\ding{43}] \textbf{symmetric\_difference()}	Returns a set with the symmetric differences of two sets
          \smallbreak \noindent
          \textbf{\textit{[r:Set] [Set]}}
        \item[\ding{43}] \textbf{union()}	Return a set containing the union of sets
          \smallbreak \noindent
          \textbf{\textit{[r:Set] [Set]}}
          \smallbreak \noindent
        \item[\ding{43}] \textbf{copy()}	Returns a copy of the set
          \smallbreak \noindent
          \textbf{\textit{[Null] [Set]}}
          \smallbreak \noindent
      \end{itemize}
    



    \pagebreak \bigbreak \noindent 
    \section{\LARGE Builtin Functions}
    \bigbreak \noindent 
    
      \begin{center}
        \textbf{Types (15)}
      \end{center}
      \begin{itemize}
        \item[\ding{43}] \textbf{type()	Returns the type of an object}
          \smallbreak \noindent
          Parameters: (obj, bases, dict)
          \smallbreak \noindent
          r:Object $\rightarrow$ Required. If only one parameter is specified, the type() function returns the type of this object
          \smallbreak \noindent
          o:Bases $\rightarrow$ Optional. Specifies the base classes
          \smallbreak \noindent
          o:Dict $\rightarrow$ Optional. Specifies the namespace with the definition for the class
          \smallbreak \noindent
        \item[\ding{43}] \textbf{int() Returns an integer number}
          \smallbreak \noindent
          Parameters: (Value, Base)
          \smallbreak \noindent
          r:Value $\rightarrow$ 	A number or a string that can be converted into an integer number
          \smallbreak \noindent
          o:Base $\rightarrow$ base	A number representing the number format. Default value: 10
          \smallbreak \noindent
        \item[\ding{43}] \textbf{str()	Returns a string object}
          \smallbreak \noindent
          Parameters: (obj, encoding=encoding, errors=errors)
          \smallbreak \noindent
          r:Object $\rightarrow$ 	Any object. Specifies the object to convert into a string
          \smallbreak \noindent
          o:Encoding $\rightarrow$ 	Any object. Specifies the object to convert into a string
          \smallbreak \noindent
          o:Errors	Specifies what to do if the decoding fails
          \smallbreak \noindent
        \item[\ding{43}] \textbf{float() Returns a floating point number}
          \smallbreak \noindent
          Parameters: (Value)
          \smallbreak \noindent
          r:Value $\rightarrow$ A number or a string that can be converted into a floating point number
          \smallbreak \noindent
        \item[\ding{43}] \textbf{bool()	Returns the boolean value of the specified object}
          \smallbreak \noindent
          Parameters: (Obj)
          \smallbreak \noindent
          r:Object $\rightarrow$ Any object, like String, List, Number etc.
          \smallbreak \noindent
          Note: The object will always return True, unless: \\
          The object is empty, like [], (), {} \\
          The object is False \\
          The object is 0 \\
          The object is None
          \smallbreak \noindent
        \item[\ding{43}] \textbf{complex() Returns a complex number}
          \smallbreak \noindent
          Parameters: (Real, Imaginary)
          \smallbreak \noindent
          r:Real $\rightarrow$ Required. A number representing the real part of the complex number. Default 0. The real number can also be a String, like this '3+5j', when this is the case, the second parameter should be omitted.
          \smallbreak \noindent
          o:Imaginary $\rightarrow$ Optional. A number representing the imaginary part of the complex number. Default 0.
          \smallbreak \noindent
        \item[\ding{43}] \textbf{bytes() Returns a bytes object}
          \smallbreak \noindent
          Parameters: (x, encoding, error)
          \smallbreak \noindent
          r:X $\rightarrow$ A source to use when creating the bytes object. \\
          If it is an integer, an empty bytes object of the specified size will be created. \\
          If it is a String, make sure you specify the encoding of the source.
          \smallbreak \noindent
          o:Encoding $\rightarrow$ 	The encoding of the string
          \smallbreak \noindent
          o:Error $\rightarrow$ Specifies what to do if the encoding fails
          \smallbreak \noindent
        \item[\ding{43}] \textbf{bytearray() Returns an array of bytes}
          \smallbreak \noindent
          Parameters: (x, encoding, error)
          \smallbreak \noindent
          r:X $\rightarrow$ A source to use when creating the bytes object. \\
          If it is an integer, an empty bytes object of the specified size will be created. \\
          If it is a String, make sure you specify the encoding of the source.
          \smallbreak \noindent
          o:Encoding $\rightarrow$ 	The encoding of the string
          \smallbreak \noindent
          o:Error $\rightarrow$ Specifies what to do if the encoding fails
          \smallbreak \noindent
        \item[\ding{43}] \textbf{bin()	Returns the binary version of a number}
          \smallbreak \noindent
          Parameters: (n)
          \smallbreak \noindent
          r:N $\rightarrow$ An integer
          \smallbreak \noindent
          Note: The result will always start with the prefix 0b.
          \smallbreak \noindent
        \item[\ding{43}] \textbf{list()	Returns a list}
          \smallbreak \noindent
          Parameters: (iterable)
          \smallbreak \noindent
          o:Iterable $\rightarrow$ A sequence, collection or an iterator object
          \smallbreak \noindent
        \item[\ding{43}] \textbf{tuple()	Returns a tuple}
          \smallbreak \noindent
          Parameters: (iterable)
          \smallbreak \noindent
          o:Iterable $\rightarrow$ A sequence, collection or an iterator object
          \smallbreak \noindent
        \item[\ding{43}] \textbf{dict()	Returns a dictionary (Array)}
          \smallbreak \noindent
          Parameters: (kwargs)
          \smallbreak \noindent
          o:Kwargs $\rightarrow$ As many keyword arguments you like, separated by comma: key = value, key = value ...
          \smallbreak \noindent
        \item[\ding{43}] \textbf{set()Returns a new set object}
          \smallbreak \noindent
          Parameters: (iterable)
          \smallbreak \noindent
          r:Optional $\rightarrow$ A sequence, collection or an iterator object
          \smallbreak \noindent
        \item[\ding{43}] \textbf{frozenset() Returns a frozenset object}
          \smallbreak \noindent
          Parameters: (iterable)
          \smallbreak \noindent
          o:Iterable $\rightarrow$ An iterable object, like list, set, tuple etc.
          \smallbreak \noindent
      \end{itemize}
    

    \bigbreak \noindent 
    
      \begin{center}
        \textbf{Works on Iteratables (6)}
      \end{center}
      \begin{itemize}
        \item[\ding{43}] \textbf{all()	Returns True if all items in an iterable object are true}
          \smallbreak \noindent
          Parameters: (iterable)
          \smallbreak \noindent
          r:Iterable $\rightarrow$ 	An iterable object (list, tuple, dictionary)
          \smallbreak \noindent
        \item[\ding{43}] \textbf{any()	Returns True if any item in an iterable object is true}
          \smallbreak \noindent
          Parameters: (iterable)
          \smallbreak \noindent
          r:Iterable $\rightarrow$ An iterable object (list, tuple, dictionary)
          \smallbreak \noindent
        \item[\ding{43}] \textbf{filter()	Use a filter function to exclude items in an iterable object}
          \smallbreak \noindent
          Parameters: (function, iterable)
          \smallbreak \noindent
          r:Function $\rightarrow$ A Function to be run for each item in the iterable
          \smallbreak \noindent
          r:Iterable $\rightarrow$ A Function to be run for each item in the iterable
          \smallbreak \noindent
        \item[\ding{43}] \textbf{max()	Returns the largest item in an iterable}
          \smallbreak \noindent
          Parameters: (iterable)
          \smallbreak \noindent
          r:Iterable $\rightarrow$ An iterable, with one or more items to compare
          \smallbreak \noindent
        \item[\ding{43}] \textbf{min()	Returns the smallest item in an iterable}
          \smallbreak \noindent
          Parameters: (iterable)
          \smallbreak \noindent
          r:Iterable $\rightarrow$ An iterable, with one or more items to compare
          \smallbreak \noindent
        \item[\ding{43}] \textbf{next()	Returns the next item in an iterable}
          \smallbreak \noindent
          Parameters: (iterable, defualt)
          \smallbreak \noindent
          r:Iterable $\rightarrow$ An iterable object.
          \smallbreak \noindent
          o:Defualt $\rightarrow$ An default value to return if the iterable has reached to its end.
          \smallbreak \noindent
        \item[\ding{43}] \textbf{sorted()	Returns a sorted list}
          \smallbreak \noindent
          Parameters: (iterable, key=func, reverse=bool)
          \smallbreak \noindent
          r:Iterable $\rightarrow$ sorted()	Returns a sorted list
          \smallbreak \noindent
          o:Key $\rightarrow$ A Function to execute to decide the order. Default is None
          \smallbreak \noindent
          o:Reverse $\rightarrow$ A Boolean. False will sort ascending, True will sort descending. Default is False
          \smallbreak \noindent
          Note: str's sort alphabetically, numbers sort numerically \\
          You cannot sort a list that contains both str and numerical values
          \smallbreak \noindent
        \item[\ding{43}] \textbf{enumerate()	Takes a collection (e.g. a tuple) and returns it as an enumerate object}
          \smallbreak \noindent
          Parameters: (iterable, start)
          \smallbreak \noindent
          r:Iterable $\rightarrow$ an iterable
          \smallbreak \noindent
          o:Start $\rightarrow$ enumerate()	Takes a collection (e.g. a tuple) and returns it as an enumerate object
      \end{itemize}
    

    \bigbreak \noindent 
    
      \begin{center}
        \textbf{Pertains to Iterators (5)}
      \end{center}
      \begin{itemize}
        \item[\ding{43}] \textbf{iter()	Returns an iterator object}
          \smallbreak \noindent
          Parameters: (object, sentienel)
          \smallbreak \noindent
          r:Object $\rightarrow$ An iterable object
          \smallbreak \noindent
          o:Sentienel $\rightarrow$ If the object is a callable object the iteration will stop when the returned value is the same as the sentinel
          \smallbreak \noindent
        \item[\ding{43}] \textbf{map()	Returns the specified iterator with the specified function applied to each item}
          \smallbreak \noindent
          Parameters: (function, iterable)
          \smallbreak \noindent
          r:Function $\rightarrow$ The function to execute for each item
          \smallbreak \noindent
          r:Iterable $\rightarrow$ A sequence, collection or an iterator object. You can send as many iterables as you like, just make sure the function has one parameter for each iterable.
          \smallbreak \noindent
        \item[\ding{43}] \textbf{reversed()	Returns a reversed iterator}
          \smallbreak \noindent
          Parameters: (sequence)
          \smallbreak \noindent
          r:Sequence $\rightarrow$ any iterable object
          \smallbreak \noindent
        \item[\ding{43}] \textbf{sum()	Sums the items of an iterator}
          \smallbreak \noindent
          Parameters: (iterable, start)
          \smallbreak \noindent
          r:Iterable $\rightarrow$ the sequence to sum
          \smallbreak \noindent
          o:Start $\rightarrow$ A value that is added to the return value
          \smallbreak \noindent
        \item[\ding{43}] \textbf{zip()	Returns an iterator, from two or more iterators}
          \smallbreak \noindent
          Parameters: (iterator1, iterator2, iterator3....) [n iterators]
          \smallbreak \noindent
          r:iterators $\rightarrow$ 	Iterator objects that will be joined together
          \smallbreak \noindent
      \end{itemize}
    

    \bigbreak \noindent 
    
      \begin{center}
        \textbf{Pertains to numbers (9)}
      \end{center}
      \begin{itemize}
        \item[\ding{43}] \textbf{abs()	Returns the absolute value of a number}
          \smallbreak \noindent
          Parameters: (n)
          \smallbreak \noindent
          r:N $\rightarrow$ a number
          \smallbreak \noindent
        \item[\ding{43}] \textbf{round()	Rounds a numbers}
          \smallbreak \noindent
          Parameters: (number, digits)
          \smallbreak \noindent
          r:Number $\rightarrow$ A number to be rounded
          \smallbreak \noindent
          o:Digits $\rightarrow$ The number of decimals to use when rounding the number, default is 0
          \smallbreak \noindent
        \item[\ding{43}] \textbf{hex()	Converts a number into a hexadecimal value}
          \smallbreak \noindent
          Parameters: (Number) 
          \smallbreak \noindent
          r:Number $\rightarrow$ an Integer
          \smallbreak \noindent
          Note: The returned string always starts with the prefix 0x.
          \smallbreak \noindent
        \item[\ding{43}] \textbf{oct()	Converts a number into an octal}
          \smallbreak \noindent
          Parameters: (number)
          \smallbreak \noindent
          r:Number $\rightarrow$ an Integer
          \smallbreak \noindent
          Note: Octal strings in Python are prefixed with 0o.
          \smallbreak \noindent
        \item[\ding{43}] \textbf{range()	Returns a sequence of numbers, starting from 0 and increments by 1 (by default)}
          \smallbreak \noindent
          Parameters: (start, stop, step)
          \smallbreak \noindent
          o:Start $\rightarrow$ An integer number specifying at which position to start. Default is 0
          \smallbreak \noindent
          r:Stop $\rightarrow$ An integer number specifying at which position to stop (not included).
          \smallbreak \noindent
          o:Step $\rightarrow$ An integer number specifying the incrementation. Default is 1
          \smallbreak \noindent
        \item[\ding{43}] \textbf{pow()	Returns the value of x to the power of y}
          \smallbreak \noindent
          Parameters: (x, y, z)
          \smallbreak \noindent
          r:X $\rightarrow$ the base
          \smallbreak \noindent
          r:Y $\rightarrow$ the exponent
          \smallbreak \noindent
          o:Z $\rightarrow$ the modulus 
          \smallbreak \noindent
        \item[\ding{43}] \textbf{divmod()	Returns the quotient and the remainder when argument1 is divided by argument2}
          \smallbreak \noindent
          Parameters: (dividend, divisor)
          \smallbreak \noindent
          r:Dividend $\rightarrow$ The number you want to divide
          \smallbreak \noindent
          r:Divisor $\rightarrow$ The number you want to divide with
          \smallbreak \noindent
        \item[\ding{43}] \textbf{ord()	Convert an integer representing the Unicode of the specified character}
          \smallbreak \noindent
          Parameters: (character)
          \smallbreak \noindent
          r:Character $\rightarrow$ any character
          \smallbreak \noindent
        \item[\ding{43}] \textbf{chr()	Returns a character from the specified Unicode code.}
          \smallbreak \noindent
          Parameters: (number)
          \smallbreak \noindent
          r:Number $\rightarrow$ 	An integer representing a valid Unicode code point
          \smallbreak \noindent
      \end{itemize}
    

    \bigbreak \noindent 
    
      \begin{center}
        \textbf{Returns Bool (4)}
      \end{center}
      \begin{itemize}
        \item[\ding{43}] \textbf{callable()	Returns True if the specified object is callable, otherwise False}
          \smallbreak \noindent
          Parameters: (object)
          \smallbreak \noindent
          r:Object $\rightarrow$ 	The object you want to test if it is callable or not.
          \smallbreak \noindent
        \item[\ding{43}] \textbf{hasattr() Returns True if the specified object has the specified attribute, property/method}
          \smallbreak \noindent
          Parameters: (object, attribute)
          \smallbreak \noindent
          r:Object $\rightarrow$ An object
          \smallbreak \noindent
          r:Attribute $\rightarrow$	The name of the attribute you want to check if exists
          \smallbreak \noindent
        \item[\ding{43}] \textbf{isinstance()	Returns True if a specified object is an instance of a specified object}
          \smallbreak \noindent
          Parameters: (object, type)
          \smallbreak \noindent
          r:Object $\rightarrow$ An object 
          \smallbreak \noindent
          r:Type $\rightarrow$ 	A type or a class, or a tuple of types and/or classes
          \smallbreak \noindent
        \item[\ding{43}] \textbf{issubclass()	Returns True if a specified class is a subclass of a specified object}
          \smallbreak \noindent
          Parameters: (object, subclass)
          \smallbreak \noindent
          r:Object $\rightarrow$ An object
          \smallbreak \noindent
          r:Subclass $\rightarrow$ 	A class object, or a tuple of class objects
          \smallbreak \noindent
      \end{itemize}
    

    \bigbreak \noindent 
    
      \begin{center}
        \textbf{Returns info (10)}
      \end{center}
      \begin{itemize}
        \item[\ding{43}] \textbf{dir()	Returns a list of the specified object's properties and methods}
          \smallbreak \noindent
          Parameters: (object)
          \smallbreak \noindent
          r:Object $\rightarrow$ 	The object you want to see the valid attributes of
          \smallbreak \noindent
        \item[\ding{43}] \textbf{help()	Executes the built-in help system}
          \smallbreak \noindent
          Parameters: (object)
          \smallbreak \noindent
          r:Object $\rightarrow$ Object you wish to recieve info on example: help(list)
          \smallbreak \noindent
        \item[\ding{43}] \textbf{id()	Returns the id of an object}
          \smallbreak \noindent
          Parameters: (object)
          \smallbreak \noindent
          r:Object $\rightarrow$ 	The object you want to see the valid attributes of
          \smallbreak \noindent
          Note: The id is the object's memory address, and will be different for each time you run the program. (except for some object that has a constant unique id, like integers from -5 to 256)
          \smallbreak \noindent
        \item[\ding{43}] \textbf{hash()	Returns the hash value of a specified object}
          \smallbreak \noindent
          Parameters: (object)
          \smallbreak \noindent
          r:Object $\rightarrow$ object you want to get the hash value of
          \smallbreak \noindent
        \item[\ding{43}] \textbf{len()	Returns the length of an object}
          \smallbreak \noindent
          Parameters: (object)
          \smallbreak \noindent
          r:Object $\rightarrow$ An object. Must be a sequence or a collection
          \smallbreak \noindent
        \item[\ding{43}] \textbf{memoryview()	Returns a memory view object}
          \smallbreak \noindent
          Parameters: (object)
          \smallbreak \noindent
          r:Object $\rightarrow$ 	A Bytes object or a Bytearray object.
          \smallbreak \noindent
        \item[\ding{43}] \textbf{getattr()	Returns the value of the specified attribute (property or method)}
          \smallbreak \noindent
          Parameters: (object, attribute, default)
          \smallbreak \noindent
          r:Object $\rightarrow$ An object
          \smallbreak \noindent
          r:Attribute $\rightarrow$ The name of the attribute you want to get the value from
          \smallbreak \noindent
          o:Default $\rightarrow$ The value to return if the attribute does not exist
          \smallbreak \noindent
        \item[\ding{43}] \textbf{repr()	Returns a readable version of an object}
          \smallbreak \noindent
          Parameters: (object)
          \smallbreak \noindent
          r:Object $\rightarrow$ returns the canonical string representation of an object
          \smallbreak \noindent
        \item[\ding{43}] \textbf{property()	Gets, sets, deletes a property}
          \smallbreak \noindent
          Parameters: (fget, fset, fdel, doc)
        \item[\ding{43}] \textbf{locals()	Returns an updated dictionary of the current local symbol table}
          \smallbreak \noindent
          Parameters: (none)
          \smallbreak \noindent
        \item[\ding{43}] \textbf{globals()	Returns the current global symbol table as a dictionary}
          \smallbreak \noindent
          Parameters: (none)
          \smallbreak \noindent
      \end{itemize}
    
    
    \bigbreak \noindent 
    
      \begin{center}
        \textbf{Works with objects (4)}
      \end{center}
      \begin{itemize}
        \item[\ding{43}] \textbf{ascii()	Returns a readable version of an object. Replaces none-ascii characters with escape character}
          \smallbreak \noindent
          Parameters: (object)
          \smallbreak \noindent
          r:Object $\rightarrow$ An object, like String, List, Tuple, Dictionary etc.
          \smallbreak \noindent
        \item[\ding{43}] \textbf{vars()	Returns the \_\_dict\_\_ property of an object}
          \smallbreak \noindent
          Parameters: (object)
          \smallbreak \noindent
          o:Object $\rightarrow$ Any object with a \_\_dict\_\_attribute
          \smallbreak \noindent
          Note: calling the vars() function without parameters will return a dictionary containing the local symbol table. \\
          The \_\_dict\_\_ attribute is a dictionary containing the object's changeable attributes. 
          \smallbreak \noindent
        \item[\ding{43}] \textbf{setattr()	Sets an attribute (property/method) of an object}
          \smallbreak \noindent
          Parameters: (object, attribute, value)
          \smallbreak \noindent
          r:Object $\rightarrow$ An object
          \smallbreak \noindent
          r:Attribute $\rightarrow$ The name of the attribute you want to set
          \smallbreak \noindent
          r:Value $\rightarrow$ The value you want to give the specified attribute
          \smallbreak \noindent
        \item[\ding{43}] \textbf{delattr()	Deletes the specified attribute (property or method) from the specified object}
          \smallbreak \noindent
          Parameters: (object, attribute)
          \smallbreak \noindent
          r:Object $\rightarrow$ An object 
          \smallbreak \noindent
          r:Attribute $\rightarrow$ The name of the attribute you want to remove
          \smallbreak \noindent
      \end{itemize}
    

    \bigbreak \noindent 
    
      \begin{center}
        \textbf{All others that return an object (5)}
      \end{center}
      \begin{itemize}
        \item[\ding{43}] \textbf{super()	Returns an object that represents the parent class}
          \smallbreak \noindent
          Parameters: (None)
          \smallbreak \noindent
        \item[\ding{43}] \textbf{slice()	Returns a slice object}
          \smallbreak \noindent
          Parameters: (Start, end, step)
          \smallbreak \noindent
          o:Start $\rightarrow$ An integer number specifying at which position to start the slicing. Default is 0
          \smallbreak \noindent
          r:Stop $\rightarrow$ 	An integer number specifying at which position to end the slicing
          \smallbreak \noindent
          o:Step $\rightarrow$ An integer number specifying the step of the slicing. Default is 1
          \smallbreak \noindent
        \item[\ding{43}] \textbf{open()	Opens a file and returns a file object}
          \smallbreak \noindent
          Parameters: (file, mode)
          \smallbreak \noindent
          r:File $\rightarrow$ 	The path and name of the file
          \smallbreak \noindent
          o:Mode $\rightarrow$ A string, define which mode you want to open the file in:
          \smallbreak \noindent
          Modes:  \\
          "r" - Read - Default value. Opens a file for reading, error if the file does not exist \\
          "a" - Append - Opens a file for appending, creates the file if it does not exist \\
          "w" - Write - Opens a file for writing, creates the file if it does not exist \\
          "x" - Create - Creates the specified file, returns an error if the file exist \\
          In addition you can specify if the file should be handled as binary or text mode \\
          "t" - Text - Default value. Text mode \\
          "b" - Binary - Binary mode (e.g. images)
          \smallbreak \noindent
        \item[\ding{43}] \textbf{object()	Returns a new object}
          \smallbreak \noindent
          Parameters: (none)
          \smallbreak \noindent
      \end{itemize}
    

    \bigbreak \noindent 
    
      \begin{center}
        \textbf{Other (8)}
      \end{center}
      \begin{itemize}
        \item[\ding{43}] \textbf{classmethod()	Converts a method into a class method}
          \smallbreak \noindent
          Parameters: (function)
          \smallbreak \noindent
          r:Function $\rightarrow$ returns a class method for the given function.
          \smallbreak \noindent
        \item[\ding{43}] \textbf{staticmethod()	Converts a method into a static method}
          \smallbreak \noindent
          Parameters: (function)
          \smallbreak \noindent
          r:Function $\rightarrow$ The function to convert to static method
          \smallbreak \noindent
        \item[\ding{43}] \textbf{eval()	Evaluates and executes an expression}
          \smallbreak \noindent
          Parameters: (expression, globals, locals)
          \smallbreak \noindent
          r:Expression $\rightarrow$ that will be evaluated as Python code
          \smallbreak \noindent
          o:Globals $\rightarrow$ A dictionary containing global parameters
          \smallbreak \noindent
          o:Locals $\rightarrow$ A dictionary containing local parameters
          \smallbreak \noindent
        \item[\ding{43}] \textbf{exec()	Executes the specified code (or object)}
          \smallbreak \noindent
          Parameters: (object, globals, locals)
          \smallbreak \noindent
          r:Object $\rightarrow$ 	A String, or a code object
          \smallbreak \noindent
          o:Globals $\rightarrow$ A dictionary containing global parameters
          \smallbreak \noindent
          o:Locals $\rightarrow$ A dictionary containing local parameters
          \smallbreak \noindent
          Note: The exec() function accepts large blocks of code, unlike the eval() function which only accepts a single expression
          \smallbreak \noindent
        \item[\ding{43}] \textbf{format()	Formats a specified value}
          \smallbreak \noindent
          Parameters: (value, format)
          \smallbreak \noindent
          r:Value $\rightarrow$ A value of any format
          \smallbreak \noindent
          o:Format $\rightarrow$ The format you want to format the value into.
          \smallbreak \noindent
        \item[\ding{43}] \textbf{input()	Allowing user input}
          \smallbreak \noindent
          Parameters: (Prompt)
          \smallbreak \noindent
          r:Prompt $\rightarrow$ 	A String, representing a default message before the input.
          \smallbreak \noindent
        \item[\ding{43}] \textbf{print()	Prints to the standard output device}
          \smallbreak \noindent
          Parameters: (object(s), sep=\textit{seperator}, end=\textit{end}, file=\textit{file}, flush=\textit{flush})
          \smallbreak \noindent
          r:Objects $\rightarrow$ 	Any object, and as many as you like. Will be converted to string before printed
          \smallbreak \noindent
          o:Sep $\rightarrow$ Specify how to separate the objects, if there is more than one. Default is ' '
          \smallbreak \noindent
          o:End $\rightarrow$ Specify what to print at the end. Default is '\\n' (line feed)
          \smallbreak \noindent
          o:File $\rightarrow$ An object with a write method. Default is sys.stdout
          \smallbreak \noindent
          o:Flush $\rightarrow$ A Boolean, specifying if the output is flushed (True) or buffered (False). Default is False
          \smallbreak \noindent
      \end{itemize}
    




\end{document}
