\documentclass{report}

\input{~/dev/latex/template/preamble.tex}
\input{~/dev/latex/template/macros.tex}

\title{\Huge{}}
\author{\huge{Nathan Warner}}
\date{\huge{}}
\fancyhf{}
\rhead{}
\fancyhead[R]{\itshape Warner} % Left header: Section name
\fancyhead[L]{\itshape\leftmark}  % Right header: Page number
\cfoot{\thepage}
\renewcommand{\headrulewidth}{0pt} % Optional: Removes the header line
%\pagestyle{fancy}
%\fancyhf{}
%\lhead{Warner \thepage}
%\rhead{}
% \lhead{\leftmark}
%\cfoot{\thepage}
%\setborder
% \usepackage[default]{sourcecodepro}
% \usepackage[T1]{fontenc}

% Change the title
\hypersetup{
    pdftitle={Linux Commands}
}

\begin{document}
    % \maketitle
        \begin{titlepage}
       \begin{center}
           \vspace*{1cm}
    
           \textbf{Linux Commands} \\
           A Useful Document
    
           \vspace{0.5cm}
            
                
           \vspace{1.5cm}
    
           \textbf{Nathan Warner}
    
           \vfill
                
                
           \vspace{0.8cm}
         
           \includegraphics[width=0.4\textwidth]{~/niu/seal.png}
                
           Computer Science \\
           Northern Illinois University\\
           United States\\
           December 10, 2023
                
       \end{center}
    \end{titlepage}
    \tableofcontents
    \pagebreak \bigbreak \noindent 
    \unsect{Find and Replace For a Directory}
    \bigbreak \noindent 
    Suppose we have some codebase, and we decide that we want to change every occurrence of the word `foo` with the word `bar`. 
    \begin{bashcode}
        find . -type f -exec grep -l 'foo' {} \; -exec sed -i 's/foo/bar/g' {} + 
    \end{bashcode}
    \bigbreak \noindent 
    Where
    \bigbreak \noindent 
    \begin{itemize}
        \item find . -type f: Find all files (-type f) in the current directory (.) and its subdirectories.
        \item -exec grep -l 'foo' {} \;: For each file found, use grep to search for 'foo'. The -l option makes grep only list the names of files with matching lines.
        \item -exec sed -i 's/foo/bar/g' {} +: For each file where 'foo' is found, use sed to replace 'foo' with 'bar'. The -i option edits files in place (i.e., it saves the changes to the file). The s/foo/bar/g is the substitution command: s for substitute, foo is the search term, bar is the replacement term, and g is for global replacement (replace all occurrences in the file).
    \end{itemize}

    \bigbreak \noindent 
    \subsection{Excluding Directory's}
    \bigbreak \noindent 
    \subsubsection{Single Directory}
    \bigbreak \noindent 
    \begin{bashcode}
    find . -type d -path ./excluded_dir -prune -o -type f -exec grep -l 'foo' {} \; -exec sed -i 's/foo/bar/g' {} +
    \end{bashcode}
    \bigbreak \noindent 
    Where
    \bigbreak \noindent 
    \begin{itemize}
        \item -type d -path ./excluded\_dir -prune: This part of the command checks if the current file is a directory (-type d) and matches the path ./excluded\_dir. If it does, -prune is used to exclude it from further processing.
        \item -o: This is the logical OR operator. It separates the directory exclusion part from the rest of the command.
    \item The rest of the command (-type f -exec grep -l 'onedark' {} \; -exec sed -i 's/onedark/onedark-fk/g' {} +) remains the same
    \end{itemize}
    \pagebreak 
    \subsubsection{Multiple Directory's}
    \bigbreak \noindent 
    \begin{bashcode}
    find . -type d \( -path ./excluded_dir1 -o -path ./excluded_dir2 \) -prune -o -type f -exec grep -l 'onedark' {} \; -exec sed -i 's/onedark/onedark-fk/g' {} +
    \end{bashcode}




    \pagebreak 
    \unsect{Changing File Names for each file in sub-directory's with a specific name}
    \bigbreak \noindent 
    Suppose we have some codebase and we want to change every filename that is `foo.cpp` to `bar.cpp`
    \bigbreak \noindent 
    \begin{bashcode}
        find . -type f -name foo.cpp -exec sh -c 'mv "${0}" "$(dirname "${0}")/bar.cpp" ' {} \;
    \end{bashcode}
    \bigbreak \noindent 
    Where
    \bigbreak \noindent
    \begin{itemize}
        \item find . -type f -name 'foo.cpp': This finds all files named foo.cpp in the current directory and its subdirectories.
        \item -exec sh -c 'mv "\$0" "\$(dirname "\$0")/bar.cpp"' {} \textbackslash ;: For each file found, this executes a shell command. The {} is replaced by the path of each found file.
            \begin{itemize}
                \item mv "\$0": Moves the found file.
                \item "\$(dirname "\$0")/bar.cpp": This constructs the new file path. dirname "\$0" gives the directory of the found file, and /bar.cpp appends the new file name to this path.
            \end{itemize}
    \end{itemize}
    
\end{document}
