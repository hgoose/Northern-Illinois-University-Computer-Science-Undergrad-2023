\documentclass{report}

\input{~/dev/latex/template/preamble.tex}
\input{~/dev/latex/template/macros.tex}

\title{\Huge{}}
\author{\huge{Nathan Warner}}
\date{\huge{}}
\pagestyle{fancy}
\fancyhf{}
\lhead{Warner \thepage}
\rhead{}
% \lhead{\leftmark}
\cfoot{\thepage}
%\setborder
% \usepackage[default]{sourcecodepro}
% \usepackage[T1]{fontenc}

\begin{document}
    % \maketitle
        \begin{titlepage}
       \begin{center}
           \vspace*{1cm}
    
           \textbf{Hypertext Markup Language} \\
           HTML 
    
           \vspace{0.5cm}
            
                
           \vspace{1.5cm}
    
           \textbf{Nathan Warner}
    
           \vfill
                
                
           \vspace{0.8cm}
         
           \includegraphics[width=0.4\textwidth]{~/niu/seal.png}
                
           Computer Science \\
           Northern Illinois University\\
           September 19, 2023 \\
           United States\\
           
                
       \end{center}
    \end{titlepage}
    \tableofcontents
    \pagebreak \bigbreak \noindent
    \section{\LARGE Boilerplate template}
    \bigbreak \noindent 
    \sepline
    \begin{minted}[linenos]{html}
<!DOCTYPE html>
<html lang="en">
<head>
    <meta charset="UTF-8">
    <title>This is document title</title>
    <link rel="stylesheet" href="style.css">
</head>
<body>
    <h1>This is a heading</h1>
    <p>Document content goes here.....</p>
</body>
</html>  
    \end{minted}
    \sepline
    \bigbreak \noindent 
    \section{\LARGE Tags}
    \bigbreak \noindent 
    \begin{itemize}
        \item <!DOCTYPE...> This tag defines the document type and HTML version.
        \item <html> This tag encloses the complete HTML document and mainly comprises of document header which is represented by <head>...</head> and document body which is represented by <body>...</body> tags.
        \item <head> This tag represents the document's header which can keep other HTML tags like <title>, <link> etc.
        \item <title> The <title> tag is used inside the <head> tag to mention the document title.
        \item <body> This tag represents the document's body which keeps other HTML tags like <h1>, <div>, <p> etc.
        \item <h1> This tag represents the heading
    \end{itemize}

    \bigbreak \noindent 
    \section{\LARGE Line break}
    \sepline
    \begin{minted}[linenos]{html}
<br /> 
    \end{minted}
    \sepline
    \bigbreak \noindent 
    \section{\LARGE centering content}
    \bigbreak \noindent 
    \sepline
    \begin{minted}[linenos]{html}
<center>
Some text
</center>
    \end{minted}
    \sepline

    \pagebreak \bigbreak \noindent 
    \section{\LARGE Horizontal lines}
    \bigbreak \noindent 
    \sepline
    \begin{minted}[linenos]{html}
<hr />
    \end{minted}
    \sepline

    \bigbreak \noindent 
    \section{\LARGE Preserve Formatting}
    \bigbreak \noindent 
\sepline
\begin{minted}[linenos]{html}
<pre>
some 
    text
</pre>
\end{minted}
\sepline

    \bigbreak \noindent 
    \section{\LARGE Nonbreaking spaces}
    \bigbreak \noindent 
    \sepline
    \begin{minted}[linenos]{html}
<p>An example of this technique appears in the movie "12&nbsp;Angry&nbsp;Men."</p>
    \end{minted}
    \sepline

    \pagebreak \bigbreak \noindent 
    \section{\LARGE Elements}
    \bigbreak \noindent 
    \begin{itemize}
        \item <p> This is paragraph content. </p>
        \item <h1> This is heading content. </h1>
        \item <div> This is division content. </div>
        \item <br />
    \end{itemize}

    \bigbreak \noindent 
    \section{\LARGE Attributes}
    \bigbreak \noindent 
    An attribute is used to define the characteristics of an HTML element and is placed inside the
element's opening tag. All attributes are made up of two parts: a name and a value:
\bigbreak \noindent 
\sepline
\begin{minted}[linenos]{html}
<p align="left">This is left aligned</p>
\end{minted}
\sepline

\bigbreak \noindent 
    \subsection{Core attributes}
    \bigbreak \noindent 
    The four core attributes that can be used on the majority of HTML elements (although not all)
    are:
    \begin{itemize}
        \item Id: The id attribute of an HTML tag can be used to uniquely identify any element within an HTML page. 
        \item Title: The behavior of this attribute will depend upon the element that carries it, although it is often displayed as a tooltip when cursor comes over the element or while the element is loading.
        \item Class: The class attribute is used to associate an element with a style sheet, and specifies the class of element. 
        \item Style: The style attribute allows you to specify Cascading Style Sheet (CSS) rules within the element.
        \item Dir: The dir attribute allows you to indicate to the browser about the direction in which the text should flow. 
            \begin{itemize}
                \item ltr: Left to right (the default value)
                \item rtl: Right to left (for languages such as Hebrew or Arabic that are read right to left)
            \end{itemize}
        \item align: Horizontally aligns tags
            \begin{itemize}
                \item right
                \item left
                \item center
            \end{itemize}
    \item background: Places a background image behind an element (Used only in body tag, and tables )
        \begin{itemize}
            \item URL
        \end{itemize}
    \end{itemize}

    \bigbreak \noindent 
    \section{\LARGE Setting background image for webpage}
    \bigbreak \noindent 
    To set the background image for the webpage, we can add the \textit{background} attribute for the \textit{body} tag:
    \bigbreak \noindent 
    \sepline
    \begin{minted}[linenos]{html}
<body background="image/linktoimage"> </body>
    \end{minted}
    \sepline

    \bigbreak \noindent 
    \section{\LARGE Formatting}
    \bigbreak \noindent 
    \begin{itemize}
        \item Bold: <br> </br>
        \item Italic: <i> </i>
        \item Underline: <u> </u>
        \item Strike out: <strike> </strike>
        \item Monospaced font: <tt> </tt>
        \item Superscript text: <sup> </sup>
        \item Subscript text: <sub> </sub>
        \item Inserted text: <ins> </ins> 
        \item Deleted text: <del> </del>
        \item Larger text: <big> </big>
        \item Smaller Text: <small> </small>
        \item Emphasize text: <em> </em>
        \item Marked Text: <mark> </mark>, this element will display text with yellow ink.
        \item Strong text: <strong> </strong>
        \item Abbreviation: <abbr title="Abhishek">Abhy</abbr>
        \item Acronym: <acronym>XHTML</acronym>
        \item Quoting Text: <blockquote>...</blockquote>
        \item Short Quotations: <q> </q>
        \item Text Citations: <cite> </cite>
        \item Computer Code: <code> </code>
        \item Keyboard Text: <kbd> </kbd>
        \item Programming Variables: <var> </var>
        \item Address Text: <address> </address>
    \end{itemize}

    \pagebreak \bigbreak \noindent 
    \section{\LARGE Div and span}
    \bigbreak \noindent 
    The <div> and <span> elements allow you to group together several elements to create sections or subsections of a page.
    \bigbreak \noindent 
    \begin{itemize}
        \item Div: <div> </div>: Used for larger groupings... Perhaps multiple tags
        \item Span: The <span> element, on the other hand, can be used to group inline elements only. So, if you have a part of a sentence or paragraph which you want to group together
    \end{itemize}
    \bigbreak \noindent 
    \textbf{Span:}
    \bigbreak \noindent 
    \sepline
    \begin{minted}[linenos]{html}
<p> This is <span style="color:blue"> some </span> text </p>
    \end{minted}
    \sepline

    \bigbreak \noindent 
    \section{\LARGE Meta tags}
    \bigbreak \noindent 
    You can add metadata to your web pages by placing <meta> tags inside the header of the
    document which is represented by <head> and </head> tags. A meta tag can have
    following attributes in addition to core attributes:
    \begin{itemize}
        \item charset: Specifies the character encoding for the HTML document
        \item name: Name for the property. Can be anything. Examples include, keywords, description, author, revised, generator etc.
        \item http-equiv: Used for http response message headers. For example, http-equiv can be used to refresh the page or to set a cookie. Values include content-type, expires, refresh and set-cookie.
        \item content: Specifies the property's value.
    \end{itemize}
    \bigbreak \noindent 
    \sepline
    \begin{minted}[linenos]{html}
<meta http-equiv="refresh" content="30"> <!--- Refresh the page every 30 seconds ---!>
    \end{minted}
    \sepline

    \bigbreak \noindent 
    \section{\LARGE Page Redirection}
    \bigbreak \noindent 
    You can use <meta> tag to redirect your page to any other webpage. You can also specify a
duration if you want to redirect the page after a certain number of seconds.
\bigbreak \noindent 
\sepline
\begin{minted}[linenos]{html}
<head>
<title>Meta Tags Example</title>
<meta name="keywords" content="HTML, Meta Tags, Metadata" />
<meta name="description" content="Learning about Meta Tags." />
<meta name="revised" content="Tutorialspoint, 3/7/2014" />
<meta http-equiv="refresh" content="5; url=http://www.tutorialspoint.com" />
</head>
\end{minted}
\sepline

    \pagebreak \bigbreak \noindent 
    \section{\LARGE Lists}
    \bigbreak \noindent 
    \subsection{Unordered}
    \bigbreak \noindent 
    \sepline
    \begin{minted}[linenos]{html}
<ul>
  <li>Coffee</li>
  <li>Tea</li>
  <li>Milk</li>
</ul>
    \end{minted}
    \sepline

    \bigbreak \noindent 
    \subsection{Ordered}
    \bigbreak \noindent 
    \sepline
    \begin{minted}[linenos]{html}
<ol>
  <li>Coffee</li>
  <li>Tea</li>
  <li>Milk</li>
</ol>
    \end{minted}
    \sepline
    \bigbreak \noindent 
    \subsection{Description Lists}
    \bigbreak \noindent 
    \sepline
    \begin{minted}[linenos]{html}
<dl>
  <dt>Coffee</dt>
  <dd>- black hot drink</dd>
  <dt>Milk</dt>
  <dd>- white cold drink</dd>
</dl>
    \end{minted}
    \sepline

    \pagebreak \bigbreak \noindent 
    \section{\LARGE Links}
    \bigbreak \noindent 
    \sepline
    \begin{minted}[linenos]{html}
<a href="url">link text</a>
    \end{minted}
    \sepline
    \bigbreak \noindent 
    \subsection{Attributes}
    \bigbreak \noindent 
    The target attribute can have one of the following values:
    \begin{itemize}
        \item \_self - Default. Opens the document in the same window/tab as it was clicked
        \item \_blank - Opens the document in a new window or tab
        \item \_parent - Opens the document in the parent frame
        \item \_top - Opens the document in the full body of the window
    \end{itemize}
    \bigbreak \noindent 
    \sepline
    \begin{minted}[linenos]{html}
<a href="https://www.w3schools.com/" target="_blank">Visit W3Schools!</a>
    \end{minted}
    \sepline

    \bigbreak \noindent 
    \section{\LARGE Images }
    \bigbreak \noindent 
    \sepline
    \begin{minted}[linenos]{html}
<img src="pic_trulli.jpg" alt="Italian Trulli">
<img src="img_girl.jpg" alt="Girl in a jacket" style="width:500px;height:600px;">
    \end{minted}
    \sepline

    \bigbreak \noindent 
    \section{\LARGE Select}
    \bigbreak \noindent 
    The select element is used to create drop-down lists
    \bigbreak \noindent 
    \sepline
    \begin{minted}[linenos]{html}
<label for="some name">Some text: </label>
<select name="somename" id="someid">
    <option value="someidentifier">option1</option>
    <option>option2</option>
    <option>option3</option>
    <option>option4</option>
</select>
    \end{minted}
    \sepline

    \pagebreak \bigbreak \noindent 
    \section{\LARGE Input}
    \bigbreak \noindent 
    The <input> HTML element is used to create interactive controls for web-based forms in order to accept data from the user
    \bigbreak \noindent 
    \sepline
    \begin{minted}[linenos]{html}
<label for="mylabel"> Some label text </label>
<input type=""> </input>
    \end{minted}
    \sepline
    \bigbreak \noindent 
    Available \textbf{type} values:
    \begin{itemize}
        \item Text
        \item Button
        \item Checkbox
        \item Color
        \item Date
        \item Email 
        \item File
        \item Hidden
        \item Image
        \item Month
        \item Number
        \item Password
        \item Range
        \item Reset 
        \item Search
        \item Submit
        \item Tel
        \item Url
    \end{itemize}
    \bigbreak \noindent 
    We can also provide the attributes \textbf{minlength} and \textbf{maxlength}.

    \pagebreak \bigbreak \noindent 
    \section{\LARGE Forms}
    \bigbreak \noindent 
    the <form> element represents a section of a document that contains form controls, such as text fields, checkboxes, radio buttons, submit buttons, and more. The <form> element can be used to collect user input for further processing, such as submitting to a server for registration, data retrieval, or any other type of client-server interaction.

    \bigbreak \noindent 
    \subsection{Attributes}
    \bigbreak \noindent 
    \begin{itemize}
        \item \textbf{action}: Specifies the URL to which the form data should be sent when the form is submitted.
        \item \textbf{method}: Defines the HTTP method to be used when sending the form data. Common methods are \texttt{GET} (appends form data to the URL in name/value pairs) and \texttt{POST} (sends the form data in the body of the request).
        \item \textbf{enctype}: Specifies how the form data should be encoded when submitting it to the server. It's used when the form contains \texttt{<input type="file">} elements. Common values include:
        \begin{itemize}
            \item \texttt{application/x-www-form-urlencoded} (default): All characters are encoded before being sent.
            \item \texttt{multipart/form-data}: No characters are encoded, necessary for file uploads.
            \item \texttt{text/plain}: Spaces are converted to + symbols, but no other characters are encoded.
        \end{itemize}
        \item \textbf{target}: Specifies where to display the response after submitting the form, e.g., \texttt{\_blank} (a new tab/window), \texttt{\_self} (the same tab/window, default), \texttt{\_parent}, \texttt{\_top}, or the name of a frame.
        \item \textbf{autocomplete}: Specifies whether the browser should enable autocomplete for the form. It can have values "on" or "off".
        \item \textbf{novalidate}: This Boolean attribute indicates that the form shouldn't be validated when submitted. It's useful when you have custom validation scripts.
    \end{itemize}


    \bigbreak \noindent 
    \subsection{Form Controls}
    \bigbreak \noindent 
    \begin{itemize}
        \item Text fields (<input type="text">)
        \item Radio buttons (<input type="radio">)
        \item Checkboxes (<input type="checkbox">)
        \item Password fields (<input type="password">)
        \item Submit buttons (<input type="submit"> or <button type="submit">)
        \item Dropdown lists (<select>)
        \item Textarea (<textarea>)
        \item File upload (<input type="file">)
    \end{itemize}

    \pagebreak \bigbreak \noindent 
    Example:
    \bigbreak \noindent 
    \line(1,0){490}
    \begin{minted}[linenos]{css}
<form action="/submit-data" method="POST">
    <label for="username">Username:</label>
    <input type="text" id="username" name="username" required>
    
    <label for="password">Password:</label>
    <input type="password" id="password" name="password" required>
    
    <input type="submit" value="Login">
</form>
    \end{minted}
    \line(1,0){490}














    
\end{document}
