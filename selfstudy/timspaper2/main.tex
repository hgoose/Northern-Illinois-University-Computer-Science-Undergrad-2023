\documentclass[12pt]{article}

%
%Margin - 1 inch on all sides
%
\usepackage[letterpaper]{geometry}
% \usepackage{times}
\geometry{top=1.0in, bottom=1.0in, left=1.0in, right=1.0in}

%
%Doublespacing
%
\usepackage{setspace}
\doublespacing

%
%Rotating tables (e.g. sideways when too long)
%
\usepackage{rotating}


%
%Fancy-header package to modify header/page numbering (insert last name)
%
\usepackage{fancyhdr}
\pagestyle{fancy}
\lhead{} 
\chead{} 
\rhead{Konsoer \thepage} 
\lfoot{} 
\cfoot{} 
\rfoot{} 
\renewcommand{\headrulewidth}{0pt} 
\renewcommand{\footrulewidth}{0pt} 
%To make sure we actually have header 0.5in away from top edge
%12pt is one-sixth of an inch. Subtract this from 0.5in to get headsep value
\setlength\headsep{0.333in}

%
%Works cited environment
%(to start, use \begin{workscited...}, each entry preceded by \bibent)
% - from Ryan Alcock's MLA style file
%
\newcommand{\bibent}{\noindent \hangindent 40pt}
\newenvironment{workscited}{\newpage \begin{center} Works Cited \end{center}}{\newpage }


%
%Begin document
%
\begin{document}
\begin{flushleft}

%%%%First page name, class, etc
Tim Konsoer\\
\textit{Professor ...}\\
\textit{Class name and number}\\
April 2, 2024\\


%%%%Title
\begin{center}
Analysis and Redesign Regarding Job Satisfaction in University Administration
\end{center}


%%%%Changes paragraph indentation to 0.5in
\setlength{\parindent}{0.5in}
%%%%Begin body of paper here
In this paper, we seek to look at the motivational challenges that are faced by Cathal, who is a registry assistant at a UK university, who experiences a decrease in enthusiasm because of inadequate training, a lack of support, and heavy unfulfilled job responsibilities. Using motivation theories and evidence-based strategies, and by examining the gap between Cathal's expectations and his actual job experience, this paper hopes to find a potential redesign for the registry position.
\bigbreak \noindent 
\hspace{\parindent} We begin by examining concepts from \textit{Maslow's hierarchy of needs}. Using concepts from Maslow's hierarchy of needs,  we see that while his basic needs for security and stability were met, which we see from his decent salary and benefits, the university does not quite fulfill all his needs. In the article titled \textit{Correctional Officer Turnover: Of Maslow's Needs Hierarchy and Herzberg's Motivation Theory} by Ikwukananne I. Udechuk, it is said, "job satisfaction and other variables such as organizational commitment were the most widely studied variables linked to voluntary turnover" (Udechukwu, 2009). Furthermore, Cathal's initial enthusiasm was decreased because of a lack of recognition and opportunities for personal growth, aligning with findings from the article that these elements are extremely important in keeping employees. Despite his efforts to contribute to the team and improve processes, his attempts were often not taken seriously, which led to Cathal beingundervalued and questioning his place within the organization.
\bigbreak \noindent 
\hspace{\parindent} Expanding on this, Cathal's struggle for self-actualization, along with the pursuit of rising to his full potential, was hindered by obvious subpar training and the less-than-ideal attitude of his superiors regarding his suggestions for potential improvement. This situation shows the larger issue of employee turnover, where "turnover is expensive monetarily and costly in many other ways" (Udechukwu, 2009), not just for the individual but for the entire organization. Cathal's case shows how not addressing the and many factors that contribute to job satisfaction can lead to a talented employee's quitting, which shows the importance of building and maintaining an environment that caters to the employee's needs.
\bigbreak \noindent 
\hspace{\parindent} We further explore Cathals expericences with concepts of Herzberg's two-factor theory, which defines two sets of factors that greatly influence employee satisfaction and motivation. The two factors are hygiene and motivators. Hygiene factors, such as salary, job security, working conditions, and company policies can lead to dissatisfaction in the workplace if not properly established. The second factor is Motivators, such as recognition, advancement opportunities, and the nature of the work itself, which directly contribute to job satisfaction and motivate employees to achieve higher performance.
\bigbreak \noindent 
\hspace{\parindent} Tying these concepts back to Cathal, we see his initial enthusiasm for his role at the university's main registry office shows a crucial aspect of job design and employee motivation that seems to have potentially been overlooked, specifically the significance of effective training. Herzberg's two-factor theory, which we know distinguishes between hygiene factors and motivators, provides a useful way to approach Cathal's situation. We first look at potential hygiene factors, such as Cathal's salary and job security, which were evident and in place, although these alone may have been insufficient in providing job satisfaction or motivating him to do well in his role. The lack of effective training and meaningful engagement with his work represents a failure by his superiors to address Herzberg's motivators, such as achievement, recognition, and the work itself. "Herzberg's two-factor theory provides an insight into key elements that encourage employees. It can be used to understand the motivations of individual employees that can in turn be utilized to craft a holistic employee motivation plan. By boosting the motivation levels of smaller teams, an encouraging and positive work environment can be created, thus improving the likelihood of project success." (Iyer, Yuvika. "Herzberg's Two-Factor Theory in Project Management"). This oversight led to Cathal's dissatisfaction, as not having these motivators very likely prevented him from finding satisfaction in his work.
\bigbreak \noindent 
\hspace{\parindent} Building on the analysis of Cathal's experiences through Herzberg's two-factor theory, we now examine the Self-Determination Theory, which offers another perspective that may help explain Cathals diminishing motivation and job satisfaction. SDT explains that motivation and psychological health heavily rely on the fulfillment of three crucial needs. They are autonomy, competence, and relatedness. Cathal's experiences in his position demonstrate a shortfall in these areas. Despite his initial readiness and capability, the lack of structured training and support from the university's administration left him feeling incompetent, which undermined his initial motivation. This scenario is a classic example of how factors such as poor training and lack of support, can decrease the sense of competence, which is a crucial element for maintaining motivation. "People need to gain mastery of tasks and learn different skills. When people feel that they have the skills needed for success, they are more likely to take actions that will help them achieve their goals." (Cherry 2022). Furthermore, Cathal was not provided with meaningful tasks during the initial months, on top of blatant disregard for his suggestions for improvement. This could have otherwise enhanced his sense of control and contribution to the organization.
\bigbreak \noindent 
\hspace{\parindent} Another concept that can help us understand Cathal's experience is the idea of Self-Determination Theory. SDT explains that motivation and psychological health heavily rely on the fulfillment of three crucial needs, which are autonomy, competence, and relatedness. Cathal's experiences in his position shows a lack in these areas. Although his initial readiness and capability was there, the lack of structured training and support from the university's administration left him feeling incompetent, which caused his decrease in initial motivation. This scenario is an example of how poor training and lack of support can decrease the sense of competence, which is a crucial element for maintaining motivation. "People need to gain mastery of tasks and learn different skills. When people feel that they have the skills needed for success, they are more likely to take actions that will help them achieve their goals." (Cherry 2022). Furthermore, Cathal was not provided with meaningful tasks during the initial months, on top of blatant disregard for his suggestions for improvement. This could have otherwise enhanced his sense of control and contribution to the organization.
\bigbreak \noindent 
\hspace{\parindent} Lastly, we look at the concepts regarding the Social Cognitive Theory (SCT). This theory, explains how people learn and change their behavior by observing others, believing in their abilities, and interacting with their environment. This theory offers a deeper understanding of the connection between people's actions, their social environment, and behavioral outcomes. This framework is relevant to Cathal's situation, showing specifically how his environment and lack of social support significantly impacted his behavior and motivation. The theory's concept of "reciprocal determinism" provides evidence that Cathal's lack of ability to properly perform his role was not only a result of his personal limitations but also because of the restrictive and highly unsupportive environment that failed to protect his skills and acknowledge his contributions.
\bigbreak \noindent 
\hspace{\parindent} On top of this, SCT's emphasis on observational learning provides a look into Cathal's struggles with his training and role satisfaction. The lack of role models within the workplace very likely hindered his observational learning, which is a key component for acquiring and changing behaviors. Furthermore, the lack of encouragement and feedback from his superiors could have majorly impacted Cathal's self-efficacy, which as a result diminished his belief in his ability to succeed in his role. These concepts show the importance of a supportive environment and positive role models in creating the development of new behaviors and the maintenance of motivation and self-efficacy, which in turn influence an individual's capability to achieve goals.
\bigbreak \noindent 
\hspace{\parindent} Through experimenting with different task allocation methods, it is possible to address the issue of specialization. By aligning tasks with staff members' biggest skill sets and areas of expertise, the registry would ensure that each team member is performing the roles that maximize their strengths and increase overall efficiency. This approach not only improves productivity but could also increase job satisfaction among the staff members because it would allow them to excel in their areas of highest competence.
\bigbreak \noindent 
\hspace{\parindent} Looking into the effects of these changes is critical in understanding their impact on productivity and employee satisfaction. Furthermore, regular feedback sessions and performance reviews would provide both the managers and staff with deep insight into the effectiveness of the changes. This cycle of observation, experimentation, and analysis makes sure that the registry can fix and refine its processes in the long term.
\bigbreak \noindent 
\hspace{\parindent} Moreover, Innovative, cost-effective initiatives that require low financial investment but provide large motivational benefits should also be looked at. Establishing a peer recognition program within the registry can create a sense of appreciation and acknowledgment, which addresses the need for social belonging and esteem highlighted in motivational theories, which we saw had a substantial impact on Cathal's overall experience. Similar to this, creating opportunities for cross-training among the staff would highly enhance the versatility and resilience throughout the team. This would ensure that the registry can maintain high levels of service even when faced with unexpected challenges.
\bigbreak \noindent 
\hspace{\parindent} In conclusion, redesigning the registry position and increasing staff motivation requires a careful approach that uses the scientific management of tasks and a basic understanding of what helps with motivation among the staff members. Through observing and analyzing the current practices, experimenting with new approaches to specialization and task management, and leveraging existing supports and innovative ideas, the registry can create a much more efficient, motivated, and satisfied workplace. 






%%%%Works cited
\begin{workscited}

\bibent 
Udechukwu, Ikwukananne. Correctional Officer Turnover: Of Maslow’s Needs Hierarchy and Herzberg’s Motivation Theory, web-p-ebscohost-com.dcu.idm.oclc.org/ehost/pdfviewer/pdfviewer?vid=0&sid=31822bca-a7b5-43f3-b711-693a93fa2f91\%40redis. 

\bibent
Iyer, Yuvika. “Herzberg’s Two-Factor Theory in Project Management: Wrike.” Blog Wrike, 4 Oct. 2023, www.wrike.com/blog/what-is-herzbergs-two-factor-theory/.

\bibent 
Kendra Cherry, MSEd. “How Does Self-Determination Theory Explain Motivation?” Verywell Mind, Verywell Mind, 8 Nov. 2022, www.verywellmind.com/what-is-self-determination-theory-2795387. 

\bibent
“Behavioral Change Models.” The Social Cognitive Theory, sphweb.bumc.bu.edu/otlt/mph-modules/sb/behavioralchangetheories/behavioralchangetheories5.html. Accessed 4 Apr. 2024. 

\bibent
Jameslopresti. “Scientific Management Theory: Definition, History, Principles, Goals.” Villanova University, 28 June 2022, www.villanovau.com/articles/leadership/scientific-management-theory-explained/#:~:text=What\%20is\%20Scientific\%20Management\%20Theory,observation\%2C\%20experimentation\%2C\%20and\%20analysis.
\end{workscited}

\end{flushleft}
\end{document}
