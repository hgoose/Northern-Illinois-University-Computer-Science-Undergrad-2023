\documentclass{report}

\input{~/dev/latex/template/preamble.tex}
\input{~/dev/latex/template/macros.tex}

\title{\Huge{}}
\author{\huge{Nathan Warner}}
\date{\huge{}}
\pagestyle{fancy}
\fancyhf{}
\lhead{Warner \thepage}
\rhead{}
% \lhead{\leftmark}
\cfoot{\thepage}
%\setborder
% \usepackage[default]{sourcecodepro}
% \usepackage[T1]{fontenc}

\begin{document}
    % \maketitle
        \begin{titlepage}
       \begin{center}
           \vspace*{1cm}
    
           \textbf{Vectors in Mathematics}
    
           \vspace{0.5cm}
            
                
           \vspace{1.5cm}
    
           \textbf{Nathan Warner}
    
           \vfill
                
                
           \vspace{0.8cm}
         
           \includegraphics[width=0.4\textwidth]{~/niu/seal.png}
                
           Computer Science \\
           Northern Illinois University\\
           November 8, 2023 \\
           United States\\
           
                
       \end{center}
    \end{titlepage}
    \tableofcontents
    \pagebreak \bigbreak \noindent
    \section{\LARGE What is a vector}
    \bigbreak \noindent 
        A \textbf{vector} is two pieces  of information. 
        \begin{enumerate}
            \item Length 
            \item Direction (Magnitude)
        \end{enumerate}

    \bigbreak \noindent 
    \subsection{Vector notation}
    \bigbreak \noindent 
    The notation for vectors is simply a variable name, with an arrow over top.
    \begin{align*}
        \vec{v}
    .\end{align*}
    We can also specify the \textbf{components} of a vector
    \begin{align*}
        \vect{v} = [x,y] \text{ or } 
        \begin{bmatrix}
            x \\ y
        \end{bmatrix}
    .\end{align*}
    \bigbreak \noindent 

    \bigbreak \noindent 
    \subsection{Length of a vector}
    \bigbreak \noindent 
    Furthermore, the length of the vector would be denoted 
    \begin{align*}
        || \vec{v} || 
    .\end{align*}
    \bigbreak \noindent 
    We can find the length by observation, if we have the x and y denominations, then we can use Pythagorean's theorem to find the length, or hypotenuse. Thus, the length of a vector would be 
    \begin{align*}
        || \vec{v} || = \sqrt{x^{2} + y^{2}}
    .\end{align*}
    \bigbreak \noindent 
    \nt{The name of a specific vector does \textbf{not} have to be $v$}

    \pagebreak \bigbreak \noindent 
    \subsection{Vector addition}
    \bigbreak \noindent 
    Suppose we have two vectors $\vec{v} = \begin{bmatrix} 3 \\ 2 \end{bmatrix}$ and $\vec{u} = \begin{bmatrix} 1 \\ 4 \end{bmatrix}$. Then 
    \begin{align*}
        &\vec{v}  + \vec{u} = \begin{bmatrix} 3 \\ 2 \end{bmatrix} +  \begin{bmatrix} 1 \\ 4 \end{bmatrix} \\
        &= \begin{bmatrix} 3 + 1 \\ 2 + 4 \end{bmatrix} \\
        &= \begin{bmatrix} 4 \\ 6 \end{bmatrix}
    .\end{align*}
    \bigbreak \noindent 
    Let's take a look at this graphically...
    \bigbreak \noindent 
\begin{figure}[ht]
    \centering
    \incfig{vectors}
    \label{fig:vectors}
\end{figure}

    \bigbreak \noindent 
    \subsection{Multiplying a vector by a scalar}
    \bigbreak \noindent 
    Suppose we have the vector $ \vec{v} = \begin{bmatrix} 1 \\ 2 \end{bmatrix} $. Then
    \begin{align*}
        &2 \vec{v} = 2 \begin{bmatrix} 1 \\ 2 \end{bmatrix} \\
        &= \begin{bmatrix} 2 \cdot 1 \\ 2 \cdot 2 \end{bmatrix} \\
        &= \begin{bmatrix} 2 \\ 4 \end{bmatrix}
    .\end{align*}
    \bigbreak \noindent 
    So you can imagine we just double the length of the vector

    \pagebreak \bigbreak \noindent 
    \subsection{Vector Subtraction}
    \bigbreak \noindent 
    Suppose we have the vectors $ \vec{v} = \begin{bmatrix} x_{1} \\ y_{1}  \end{bmatrix} $ and $ \vec{u} = \begin{bmatrix} x_{2} \\ y_{2} \end{bmatrix} $. Then how might we compute $ \vec{v} - \vec{u}$\ \ ? 
    \begin{align*}
        &\vec{v} - \vec{u}  \\
        =&\vec{v}  + (-\vec{u})
    .\end{align*}
    \bigbreak \noindent 
    Or just simply
    \begin{align*}
        \vec{v} - \vec{u} = \begin{bmatrix} x_{1} - x_{2} \\ y_{1} - y_{2} \end{bmatrix}
    .\end{align*}

    \bigbreak \noindent 
    \subsection{Vectors in 3 dimensions}
    \bigbreak \noindent 
    With a three dimension vector, instead of having $\vec{v} = \begin{bmatrix} x \\ y \end{bmatrix} $, we will have $ \vec{v} = \begin{bmatrix} x \\ y  \\ z\end{bmatrix}$. Suppose we have the vector $ \vec{v} = \begin{bmatrix} 1 \\ 4 \\ 3 \end{bmatrix}$. Then graphically we would have
    \bigbreak \noindent 
    \begin{figure}[ht]
        \centering
        \incfig{vectors6}
        \label{fig:vectors6}
    \end{figure}

    \bigbreak \noindent 
    \subsection{Length of a vector in three dimensions}
    \bigbreak \noindent 
    By examining the above figure, we notice that to find $||\vec{v}||$, we need to find the hypotenuse of two separate triangles. Thus, we can generalize the length of a three dimensional vector with
    \begin{align*}
        &|| \vec{v} || = \sqrt{(\sqrt{x^{2} + y^{2}})^{2} + z^{2}} \\
        &=\sqrt{x^{2} + y^{2} + z^{2}}
    .\end{align*}

    \pagebreak \bigbreak \noindent 
    \subsection{Definition of $\mathbb{R}^{n}$}
    \bigbreak \noindent 
    \begin{definition}
       $\mathbb{R}^{n}$ is the set of all $n-tuples$ of real numbers  
    \end{definition}
    \bigbreak \noindent 
    For example, a vector $\vec{v}$ in two dimensions has two components $\begin{bmatrix} x \\ y\end{bmatrix}$. Thus, we say that $\vec{v}$ is a $2-tuple$. Similarly, for any vector $\vec{v} = [v_{1}, v_{2}, v_{3},...,v_{n}]$, we say it is a $n-tuple$. Thus we can declare:
    \begin{align*}
        &\vec{v} = [v_{1}, v_{2}]\ \vec{v} \in \mathbb{R}^{2} \\
        &\vec{u} = [u_{1}, u_{2},u_{3}]\ \vec{u} \in \mathbb{R}^{3} \\
        &\vec{w} = [w_{1}, w_{2},w_{3}, w_{n}]\ \vec{w} \in \mathbb{R}^{n} \\
    .\end{align*}
    \bigbreak \noindent 
    This may seem intuitive, if we recall the definition for the set of all $(x,y)$ pairs on the Cartesian plane, we have
    \begin{align*}
        &\mathbb{R} \times \mathbb{R} = \{(x,y) \mid x \in \mathbb{R}^{2}\} \\
        &\text{or }\mathbb{R} \times \mathbb{R} = \{(x,y) \mid x \in \mathbb{R}, y \in \mathbb{R}\}
    .\end{align*}
    \bigbreak \noindent 
    So for a three dimensional plane
    \begin{align*}
        \mathbb{R} \times \mathbb{R} \times \mathbb{R} = \{(x,y,z) \mid (x,y,z) \in \mathbb{R}^{3}\}
    .\end{align*}

    \pagebreak \bigbreak \noindent 
    \subsection{Algebraic Properties of Vectors }
    \bigbreak \noindent 
    \begin{enumerate}[label=(\roman*)]
        \item $\vec{u}  + \vec{v}  = \vec{v} + \vec{u}$ 
        \item 
    \end{enumerate}
    

    
    
\end{document}
