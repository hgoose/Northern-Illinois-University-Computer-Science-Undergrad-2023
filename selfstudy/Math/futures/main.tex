\documentclass{report}

\input{~/dev/latex/template/preamble.tex}
\input{~/dev/latex/template/macros.tex}

\title{\Huge{}}
\author{\huge{Nathan Warner}}
\date{\huge{}}
\fancyhf{}
\rhead{}
\fancyhead[R]{\itshape Warner} % Left header: Section name
\fancyhead[L]{\itshape\leftmark}  % Right header: Page number
\cfoot{\thepage}
\renewcommand{\headrulewidth}{0pt} % Optional: Removes the header line
%\pagestyle{fancy}
%\fancyhf{}
%\lhead{Warner \thepage}
%\rhead{}
% \lhead{\leftmark}
%\cfoot{\thepage}
%\setborder
% \usepackage[default]{sourcecodepro}
% \usepackage[T1]{fontenc}

% Change the title
\hypersetup{
    pdftitle={Futures}
}

\begin{document}
    % \maketitle
        \begin{titlepage}
       \begin{center}
           \vspace*{1cm}
    
           \textbf{Cyptocurrency Futures}
    
           \vspace{0.5cm}
            
                
           \vspace{1.5cm}
    
           \textbf{Nathan Warner}
    
           \vfill
                
                
           \vspace{0.8cm}
         
           \includegraphics[width=0.4\textwidth]{~/niu/seal.png}
                
           Computer Science \\
           Northern Illinois University\\
           United States\\
           
                
       \end{center}
    \end{titlepage}
    \tableofcontents
    \pagebreak 
    \unsect{What is futures trading?}
    \bigbreak \noindent 
    \begin{concept}
        Futures trading involves a contractual agreement to buy or sell a particular quantity of a commodity, financial instrument, or other tradable item at a predetermined price at a specified time in the future. These contracts are standardized and traded on futures exchanges.
        \bigbreak \noindent 
        In simple terms, imagine you're a farmer who grows wheat, and you want to lock in a price for your wheat that you'll harvest in a few months. You can enter a futures contract with a buyer who agrees to purchase your wheat at a set price when it's harvested. This way, you're protected if prices fall before you harvest, but you also agree to sell at this price even if market prices rise. Similarly, a bakery that needs wheat might use futures to lock in a price for wheat they'll buy in the future, protecting them against price rises but committing to the agreed price even if the market price falls. This mechanism allows both producers and consumers to hedge against price volatility.
    \end{concept}

    \bigbreak \noindent 
    \subsection{Types of futures trading}
    \bigbreak \noindent 
    \begin{itemize}
        \item \textbf{Commodity Futures:} This is one of the oldest forms of futures trading, involving physical commodities like agricultural products (wheat, corn, soybeans), metals (gold, silver, copper), and energy products (crude oil, natural gas). Traders can speculate on price movements or hedge against price volatility of these commodities.
        \item \textbf{Financial Futures:} These involve financial instruments as underlying assets. Examples include:
            \begin{itemize}
                \item \textbf{Interest Rate Futures:} Contracts based on the future movements of interest rates. Examples include Treasury bond futures, Treasury bill futures, and Eurodollar futures.
                \item \textbf{Currency Futures:} These contracts are agreements to exchange one currency for another at a future date at a predetermined price. They are used to hedge against or speculate on changes in currency exchange rates.
                \item \textbf{Stock Index Futures:} These futures contracts are based on the future value of a stock index, like the S&P 500 or Dow Jones Industrial Average. Traders use these to speculate on the overall direction of the stock market or hedge investment portfolios.
            \end{itemize}
        \item \textbf{Cryptocurrency Futures:} As digital currencies have grown in popularity, futures markets for cryptocurrencies like Bitcoin and Ethereum have emerged. These allow traders to speculate on the future price movements of cryptocurrencies without owning the actual digital assets.
        \item \textbf{Real Estate Futures:} Although not as common, there are futures contracts based on real estate price indices. These can be used by investors to hedge against or speculate on changes in real estate markets.
        \item \textbf{Weather Futures:} These are financial derivatives that allow businesses to hedge against or speculate on weather-related risks, such as temperature fluctuations, rainfall, or snowfall.
    \end{itemize}
    \bigbreak \noindent 
    \nt{This document will focus on \textit{cryptocurrency futures}}

    \bigbreak \noindent 
    \subsection{Standard Futures vs USDT-M Perp Futures vs Coin-M Perp Futures}

    \bigbreak \noindent 
    \subsubsection{Standard Futures}
    \bigbreak \noindent 
    \begin{itemize}
        \item \textbf{Definition:} Standard futures are traditional futures contracts found in cryptocurrency trading that involve agreeing to buy or sell a specific amount of a cryptocurrency at a predetermined price at a specified time in the future.
        \item \textbf{Settlement:} These contracts can be either physically settled (where actual cryptocurrency changes hands) or cash-settled (where the difference in price is exchanged in fiat currency or another form of value).
        \item \textbf{Expiration:} They typically have a set expiration date upon which the contract must be settled.
    \end{itemize}

    \bigbreak \noindent 
    \subsubsection{USDT-Margined Perpetual Futures (USDT-M Perp Futures)}
    \bigbreak \noindent 
    \begin{itemize}
        \item \textbf{Definition:} These are a type of futures contract that does not have an expiration date, meaning they can be held indefinitely (perpetual).
        \item \textbf{Collateral:} These contracts are margined and settled in USDT (Tether), a stablecoin pegged to the US dollar. This means the profits, losses, margin calls, and settlements are handled in USDT.
        \item \textbf{Price Mechanism:} The price of these contracts often tracks the underlying asset very closely. To maintain this alignment, mechanisms like funding rates are used, where payments are exchanged between long and short positions depending on market conditions.
    \end{itemize}

    \bigbreak \noindent 
    \subsubsection{Coin-Margined Perpetual Futures (Coin-M Perp Futures)}
    \bigbreak \noindent 
    \begin{itemize}
        \item \textbf{Definition:} Like USDT-Margined Perpetual Futures, these are also perpetual and do not have an expiration date.
        \item \textbf{Collateral:} The key difference is that these contracts use the cryptocurrency itself as collateral. For example, if you are trading Bitcoin futures, the margin and settlement could be handled in Bitcoin.
        \item \textbf{Settlement Currency:} Settlements, profits, and losses are transacted in the cryptocurrency used as collateral, making them potentially more volatile due to the price movements of the cryptocurrency itself.
        \item \textbf{Price Alignment:} These also use mechanisms like funding rates to keep the futures price aligned with the spot price of the cryptocurrency.
    \end{itemize}

    \bigbreak \noindent 
    \unsect{Futures contract/trading pair}
    \bigbreak \noindent 
    For example, When you see "BTCUSDT" on a cryptocurrency futures exchange, it denotes a specific type of futures contract or trading pair involving Bitcoin (BTC) and Tether (USDT). 
    \bigbreak \noindent 
    \begin{itemize}
        \item \textbf{BTC (Bitcoin):} This represents Bitcoin, the first and most widely known cryptocurrency. In the context of a trading pair or futures contract, it's the primary asset that's being traded or speculated upon.
        \item \textbf{USDT (Tether):} USDT is a stablecoin, which is a type of cryptocurrency designed to maintain a stable value over time. Tether aims to be pegged to the US dollar, with 1 USDT theoretically equal in value to 1 USD. Stablecoins like USDT are often used in cryptocurrency trading to provide a stable base or quote currency against the often volatile prices of cryptocurrencies like Bitcoin.
        \item \textbf{BTCUSDT Futures Contract:} When combined in a trading pair like BTCUSDT on a futures exchange, it implies that the futures contract is for speculating on the future price of Bitcoin (BTC) against the value of Tether (USDT). Participants in this market are essentially betting on what the price of Bitcoin will be in USDT terms at a specified future date when the contract expires.
        \item \textbf{Purpose and Use:} Traders use BTCUSDT futures contracts for various purposes, including hedging against future price movements of Bitcoin, speculating on the price direction of Bitcoin, and managing portfolio exposure without the need to hold actual Bitcoins. The use of USDT as the quote currency allows traders to manage their risk and returns in a currency whose value is more stable than other cryptocurrencies, making it easier to calculate profits, losses, and margins.
        \item \textbf{Settlement:} Depending on the specific terms of the futures contract, settlement can be either physical (actual delivery of Bitcoin in exchange for USDT) or cash-settled (the net difference in value is exchanged in USDT). Most cryptocurrency futures contracts, especially those involving stablecoins like USDT, are cash-settled.
    \end{itemize}


    \pagebreak 
    \unsect{Acquiring Positions}
    \bigbreak \noindent 
    \subsection{Going Long}
    Suppose we go long on the BTCUSDT pair. Placing a "long" position on BTCUSDT in the context of a futures exchange means you are entering a contract to buy Bitcoin at a future date at a price specified now, with the expectation that the price of Bitcoin (in terms of Tether, USDT) will increase
    \bigbreak \noindent 
    \begin{itemize}
        \item \textbf{Opening the Position:} When you decide to go long on BTCUSDT, you are essentially betting that the price of Bitcoin will rise relative to USDT. You enter a futures contract at the current market price, which is the agreed price at which you will buy Bitcoin in the future. This does not require you to own Bitcoin at the time you open the position.
        \item \textbf{Margin Requirement:} To open a long position, you'll need to post a margin, which is a fraction of the total value of the contract. This margin acts as collateral to cover potential losses. Futures trading is leveraged, meaning you can control a large contract value with a relatively small amount of capital, amplifying both potential gains and losses.
        \item \textbf{Price Movement:} After you open your position, one of two things can happen regarding the price of Bitcoin:
            \begin{itemize}
                \item If the price of Bitcoin increases (against USDT) as you predicted, your long position will profit. The profit is generally the difference between the price at which you entered the contract and the higher price of Bitcoin at the time of settlement or when you decide to close the position early.
                \item If the price of Bitcoin decreases, your position will incur a loss, proportional to the decline in price from your entry point.
            \end{itemize}
        \item \textbf{Maintenance Margin:} If your position moves into a loss that exceeds the initial margin, you may be subject to a margin call, requiring you to deposit additional funds to maintain your position. Failure to meet the margin call can lead to the forced liquidation of your position by the exchange.
        \item \textbf{Settlement or Closing the Position:} Your long position on BTCUSDT can be closed in two main ways:
            \begin{itemize}
                \item \textbf{Manual Closure:} You can choose to close your position before the contract expires by taking an opposite action in the market (selling the contract you initially bought). If you’re in profit, this locks in your gains; if you’re at a loss, it stops further losses.
                \item \textbf{Expiration:} If held to expiration, the contract will settle based on the terms of the futures contract. For cash-settled contracts, the difference between the entry price and the settlement price will be credited to or debited from your account in USDT.
            \end{itemize}
    \end{itemize}
    \bigbreak \noindent 
    \subsection{Going Short}
    Again, suppose we are trading the BTCUSDT pair. Shorting on BTCUSDT in the context of a cryptocurrency futures exchange means you are entering into a contract to sell Bitcoin at a future date at a price specified now, with the expectation that the price of Bitcoin (in terms of Tether, USDT) will decrease. Here’s how shorting works in the futures market:
    \begin{itemize}
        \item \textbf{Opening the Position:} To short BTCUSDT, you sell a futures contract at the current market price, which is the agreed price at which you commit to selling Bitcoin in the future. This position profits from a decline in Bitcoin's price relative to USDT. Importantly, you do not need to own Bitcoin when you initiate a short position.
        \item \textbf{Margin Requirement:} Similar to going long, shorting requires posting a margin, a fraction of the total contract value, serving as collateral. Futures are leveraged, allowing you to control a significant position with a smaller amount of capital, which amplifies potential gains and losses.
        \item \textbf{Price Movement and Profit/Loss:} After opening your short position, the following outcomes are possible:
            \begin{itemize}
                \item If the price of Bitcoin decreases as anticipated, your short position will be profitable. The profit is generally the difference between the price at which you entered the short contract and the lower price of Bitcoin when you close or settle the position.
                \item If the price of Bitcoin increases, contrary to your expectation, your position will incur a loss, proportional to the rise in price from your entry point.
            \end{itemize}
    \end{itemize}

    \bigbreak \noindent 
    \unsect{Characteristics of Cryptocurrency Future Contracts} 
    \bigbreak \noindent 
    \begin{itemize}
        \item \textbf{Standardization:} Like traditional futures, cryptocurrency futures contracts are standardized in terms of contract size, expiration date, and settlement methods, facilitating ease of trade on the exchange.
        \item \textbf{Settlement:} Contracts can be settled either through physical delivery (less common in cryptocurrency futures) or cash settlement. Most cryptocurrency futures are cash-settled, meaning the difference between the purchase price and the selling price is exchanged in fiat currency or stablecoins like USDT (Tether).
        \item \textbf{Leverage:} Cryptocurrency futures trading is typically leveraged, allowing traders to gain a large exposure to the underlying digital currency with a relatively small amount of capital (known as the "margin").
    \end{itemize}
    \bigbreak \noindent 
    \subsection{Contract sizes}
    \bigbreak \noindent 
    The contract size in cryptocurrency futures denotes the amount of cryptocurrency that each contract represents. This standardization is crucial for understanding the scale of one’s investment and potential exposure to market movements.
    \bigbreak \noindent 
    \begin{itemize}
        \item \textbf{Determining Exposure:} The contract size determines how much of the cryptocurrency's price movement will impact the trader’s position. A larger contract size means greater exposure to price changes and, consequently, higher potential profits or losses.
        \item \textbf{Margin Requirements:} Since cryptocurrency futures are leveraged, the contract size also influences the margin requirement — the amount of capital a trader must deposit to open a position. Exchanges specify margin requirements as a percentage of the contract's total value. Smaller contract sizes can lower the barrier to entry for individual traders by requiring less capital upfront.
        \item \textbf{Flexibility and Accessibility:} Exchanges often offer contracts with various sizes to cater to different types of traders. For instance, besides standard contracts, there might be "mini" or "micro" contracts available with smaller sizes, making it easier for individual investors to participate in the market.
    \end{itemize}
    \bigbreak \noindent 
    \subsubsection{Examples}
    \bigbreak \noindent 
    \begin{itemize}
        \item A Bitcoin futures contract might have a contract size of 1 BTC, meaning each contract represents the future delivery of 1 Bitcoin.
        \item A smaller contract, such as a micro Bitcoin futures contract, might have a contract size of 0.1 BTC, offering lower margin requirements and less capital exposure.
    \end{itemize}

    \pagebreak 
    \unsect{Point increase and ticks}
    \bigbreak \noindent 
    In cryptocurrency futures trading, just as in traditional futures markets, the term "tick" is often used to describe the minimum point increase or decrease in the price of a futures contract. However, the general concept of a "point increase" remains applicable and is understood in the context of how much the price of a cryptocurrency futures contract has risen, measured in the contract's quote currency (e.g., USDT, USD).
    \bigbreak \noindent 
    \subsection{Tick size}
    \begin{itemize}
        \item \textbf{Definition:} A tick is the smallest possible price change on the upside or downside. Each futures contract, whether it's for cryptocurrencies or other assets, has a specified "tick size" that determines the minimum increment by which the price can move.
        \item \textbf{Example:} If the tick size of a BTCUSDT futures contract is 0.5 USDT, it means the price of the contract can move up or down in increments of 0.5 USDT. A move from 50,000 USDT to 50,000.5 USDT would represent a one-tick increase.
    \end{itemize}
    \bigbreak \noindent 
    \subsection{Point increase}
    \begin{itemize}
        \item \textbf{Understanding Point Increase:} In the context of cryptocurrency futures, a "point" typically refers to a whole number movement in the price of the futures contract, as opposed to fractional movements (ticks). The actual value of a "point" in terms of the underlying cryptocurrency (e.g., Bitcoin, Ethereum) can vary depending on the contract size and the currency in which the contract is denominated.
        \item \textbf{Significance:} A point increase in a cryptocurrency futures contract indicates a rise in the contract's price, which can be beneficial for traders holding long positions. For example, in a BTCUSDT futures contract, if the price of Bitcoin increases by 100 USDT, that would be a 100-point increase, benefiting traders who speculated on the price going up.
    \end{itemize}

    \pagebreak 
    \unsect{The Order Book}
    \bigbreak \noindent 
    An order book in a futures exchange is a real-time, continuously updated list that records all the buy and sell orders for futures contracts. Each order listed in the book displays the number of contracts being offered and the price level at which the buyer or seller is willing to trade. The order book is a crucial tool for traders and investors, providing deep insights into the market's current demand and supply dynamics, liquidity, and potential price movements
    \bigbreak \noindent 
    \subsection{Components}
    \bigbreak \noindent 
    \begin{itemize}
        \item \textbf{Bid:} The bid side of the order book lists all the buy orders at various price levels, with the highest bid price at the top. This price represents the highest price that buyers are currently willing to pay for a futures contract.
        \item \textbf{Ask (Offer):} The ask side lists all the sell orders, with the lowest ask price at the top. This is the lowest price at which sellers are willing to sell a futures contract.
        \item \textbf{Price Levels:} Each price level shows the total number of contracts buyers are willing to buy or sell at that specific price.
        \item \textbf{Depth of Market:} The order book shows the depth of market for a futures contract by displaying the number of buy and sell orders at different price levels. This depth can indicate the liquidity of the market; a deeper market (with more orders) typically suggests higher liquidity.
    \end{itemize}

    \pagebreak 
    \unsect{Mark Price vs Index Price}
    \bigbreak \noindent 
    In cryptocurrency futures trading, the terms "mark price" and "index price" refer to two different but related pricing mechanisms used to manage and value futures contracts
    \bigbreak \noindent 
    \subsection{Index Price}
    \bigbreak \noindent 
    The index price is a benchmark price that reflects the real-time spot price of a cryptocurrency. It's derived from an average price of the cryptocurrency across multiple spot exchanges. The purpose of the index price is to provide a fair, transparent, and manipulation-resistant reference price for futures contracts. Here are some key points about the index price:
    \begin{itemize}
        \item \textbf{Aggregation:} It is usually calculated as a weighted average of the spot prices from several major exchanges.
        \item \textbf{Purpose:} The index price is used to determine the settlement prices of futures contracts and to ensure that the futures prices are aligned with the underlying market.
        \item \textbf{No Trading:} You cannot trade directly at the index price; instead, it serves as a reference for the trading of futures contracts.
    \end{itemize}

    \bigbreak \noindent 
    \subsection{Mark Price}
    \bigbreak \noindent 
    The mark price is the price at which a cryptocurrency futures contract is currently valued for trading purposes on an exchange. It plays a crucial role in the management of leveraged positions and is used primarily to prevent unnecessary liquidations due to market manipulation or illiquidity. Here are some important aspects of the mark price:
    \begin{itemize}
        \item \textbf{Calculation:} The mark price is often calculated by taking the index price and adjusting it with the funding rate or interest rate differentials between the futures prices and the spot prices.
        \item \textbf{Function:} Its main function is to trigger liquidations and to calculate unrealized profit and loss on open positions. This helps ensure that the futures market remains closely tied to the actual market conditions reflected in the index price.
        \item \textbf{Safety Mechanism:} By using the mark price instead of the last traded price for margin calculations and liquidations, exchanges protect traders from price spikes or dumps that might occur in less liquid trading environments.
    \end{itemize}

    \pagebreak 
    \unsect{Technical Analysis}
    \bigbreak \noindent 
    \subsection{Impulse and Pullback}
    \bigbreak \noindent 
    \subsubsection{Impulse}
    \bigbreak \noindent 
    An impulse refers to a strong, directional movement in the price of an asset, aligning with the prevailing trend. In a bullish trend, an impulse would be a sharp move upwards, whereas in a bearish trend, it would be a significant move downwards. Here are some characteristics of impulses:
    \begin{itemize}
        \item \textbf{Strong Movement:} Impulses are characterized by rapid and significant price movements that clearly indicate the direction of the trend.
        \item \textbf{Volume Support:} These moves are often supported by high trading volumes, suggesting strong market participation and commitment to the trend direction.
        \item \textbf{Psychological Impact:} Impulsive moves can reinforce the existing sentiment about the asset, either bullish or bearish, as they reflect a consensus among traders about the value direction.
    \end{itemize}
    \bigbreak \noindent 
    \subsubsection{Pullback}
    \bigbreak \noindent 
    \begin{itemize}
        \item \textbf{Correction:} Pullbacks allow the market to "breathe" by letting prices correct from an overextended move, helping to mitigate the risks of entering at peak levels.
        \item \textbf{Entry Opportunities:} For traders looking to enter a trend, pullbacks often provide safer entry points as they can buy into the trend at a lower price in an uptrend or sell at a higher price in a downtrend.
        \item \textbf{Volume Observation:} Typically, pullbacks occur on lower volumes compared to impulses, suggesting less conviction in the counter-trend movement.
    \end{itemize}

    % \bigbreak \noindent 
    % \nt{We usually consider the advance up to the new higher high to be the impulse, and the correction of this impulse to be the pullback. Small movements up and down during the pullback are not labeled impulse or pullback.}

    \pagebreak 
    \subsection{Trending Markets}
    \bigbreak \noindent 
    \subsubsection{Uptrend}
    \bigbreak \noindent 
    An uptrend in the context of financial markets, such as forex, stocks, or commodities, refers to a situation where the price of an asset shows a general pattern of increasing over time. It's essentially a period during which buyers are in control, pushing prices higher. Here are the key characteristics and implications of an uptrend:
    \begin{itemize}
        \item \textbf{Higher Highs and Higher Lows:} The most identifiable feature of an uptrend is a series of higher highs (peaks) and higher lows (troughs). This means each peak in the price action is higher than the previous peak, and each trough is higher than the previous trough, indicating sustained buying interest.
        \item \textbf{Trend Lines:} An uptrend can often be visually represented by drawing a trend line that connects the lows in the price action. This trend line acts as a support level that can indicate potential buying opportunities when the price touches or comes close to it.
        \item \textbf{Moving Averages:} In an uptrend, shorter-term moving averages (like the 50-day moving average) typically stay above longer-term moving averages (like the 200-day moving average), and the price is often above these moving averages. When these moving averages cross, it can signal a strengthening of the trend.
    \end{itemize}
    \bigbreak \noindent 
    \textbf{Implications}
    \begin{itemize}
        \item \textbf{Market Sentiment:} An uptrend signifies positive sentiment toward the asset, suggesting that the market participants believe in its potential for growth or are optimistic about its future.
        \item \textbf{Investment Strategy:} Traders and investors might see an uptrend as a favorable time to buy into the market, anticipating further price increases. However, it's crucial to consider the uptrend's maturity; entering a trade early in an uptrend can be more rewarding than entering during a late stage, where the risk of a trend reversal might be higher.
        \item \textbf{Risk Management:} While an uptrend can present opportunities for profit, it's important for traders to employ risk management strategies. Setting stop-loss orders and taking profits at predetermined levels can help protect against sudden reversals.
    \end{itemize}
    \bigbreak \noindent 
    \fig{.4}{./figures/Uptrend.png}

    \bigbreak \noindent 
    \fig{.6}{./figures/upternd-market.png}

    \bigbreak \noindent 
    \nt{As long as the next trough does not surpase (go lower) than the previous trough, the market remains in an uptrend. The path to the new peak is considered the impulse. After each impulsive move there will be pullbacks, these pullbacks are the paths to the new troughs. Hence, as long as the new pullback does not surpase the previous, the market is still in an uptrend.}

    \pagebreak 
    \subsection{Downtrend}
    \bigbreak \noindent 
    A downtrend in financial markets is characterized by a series of lower highs and lower lows in the price of an asset, reflecting a sustained decrease in its value over time. This pattern indicates that the demand for the asset is weakening relative to its supply, leading to a decline in price. Here’s a detailed breakdown of the concept of a downtrend:
    \bigbreak \noindent 
    \subsubsection{Identification of a Downtrend}
    \bigbreak \noindent 
    \begin{itemize}
        \item \textbf{Lower Highs and Lower Lows:} The most fundamental characteristic of a downtrend is that each successive peak (high) and trough (low) in price is lower than the previous ones. This pattern can be visually identified on a chart and is indicative of ongoing selling pressure.
        \item \textbf{Trendlines:} A downtrend can often be confirmed by drawing a descending trendline that connects the lower highs. This trendline acts as resistance; the price struggles to break above this line and often bounces off it, continuing the downtrend.
        \item \textbf{Moving Averages:} Technical analysts might use moving averages to identify and confirm downtrends. For instance, when a shorter-term moving average (like the 50-day moving average) crosses below a longer-term moving average (like the 200-day moving average), it can signal the beginning or continuation of a downtrend.
    \end{itemize}
    \bigbreak \noindent 
    \subsubsection{Psychological and Market Dynamics}
    \begin{itemize}
        \item \textbf{Selling Pressure:} In a downtrend, increased selling pressure is often driven by negative sentiment towards the asset, which could be due to fundamental reasons (like deteriorating company performance or economic conditions) or technical reasons (like breaking key support levels).
        \item \textbf{Fear and Capitulation:} As prices continue to fall, the market sentiment often shifts towards fear, which can lead to capitulation—where investors give up any previous hopes of recovery and sell their positions, further driving down prices.
    \end{itemize}
    \pagebreak 
    \subsubsection{Figures}
    \bigbreak \noindent     
    \fig{.5}{./figures/price-downtrend.png}
    \bigbreak \noindent 
    \fig{.3}{./figures/EURUSD-4-hour-chart-56a22ddb3df78cf77272e828.jpg}


    \pagebreak 
    \subsection{Support and Resistance}
    \bigbreak \noindent 
    Support and resistance are fundamental concepts in technical analysis used by traders to identify potential price levels on a chart where an asset's price will tend to stop and reverse. These levels are determined by the past interactions between buyers and sellers and are often where an influx of buying or selling activity can be expected.
    \bigbreak \noindent 
    \subsubsection{Support}
    \bigbreak \noindent 
    Support refers to a price level where a downtrend can be expected to pause due to a concentration of demand (buying pressure). As the price of an asset drops, demand for the asset increases, thus forming the support line. This happens because:
    \begin{itemize}
        \item \textbf{Perceived Value:} Investors perceive the asset as more attractive or undervalued at lower prices, leading to increased purchases.
        \item \textbf{Psychological Effect:} The support level also often represents a psychological barrier where traders expect prices to bounce back.
        \item \textbf{Historical Interaction:} Traders look at historical price levels where an asset has previously ceased to drop further, expecting the pattern to repeat.
    \end{itemize}
    \bigbreak \noindent 
    When the price approaches a support level but does not break below it, the support is considered \textbf{strong}. If the price does break below the support level, it might go on to establish a new lower support level.

    \bigbreak \noindent 
    \subsubsection{Resistance}
    \bigbreak \noindent 
    Resistance is the opposite of support, representing a price level where a price uptrend is likely to pause or stop due to a concentration of supply (selling pressure). As the price of an asset rises, the inclination to sell it increases, thus forming the resistance line. This occurs because:
    \begin{itemize}
        \item \textbf{Profit Taking:} Higher prices entice traders to sell to capture profits, increasing the supply over demand.
        \item \textbf{Psychological Barrier:} Like support, resistance levels can also represent psychological price points that the market finds too expensive or overvalued.
        \item \textbf{Historical Price Levels:} Resistance is often established by looking at price levels where an asset has previously stopped rising.
    \end{itemize}
    \bigbreak \noindent 
    If a price reaches a resistance level but fails to break through it, the resistance is deemed \textbf{strong}. However, if the price does break through the resistance, it could climb until it finds a new upper resistance level.
    \bigbreak \noindent 
    \subsubsection{Trading with Support and Resistance}
    \begin{itemize}
        \item \textbf{Entry and Exit Points:} Support and resistance levels help traders determine strategic entry and exit points based on the expectation that prices will move between these levels until they don't. Buying near support and selling near resistance can optimize risk/reward.
        \item \textbf{Breakouts and Breakdowns:} A price moving past a support or resistance level might indicate a strong move in the direction of the breakout. Traders might use this as a signal to enter a trade anticipating a new trend.
        \item \textbf{Trend Confirmation:} Sustained movement above a resistance level or below a support level might indicate a change in the broader market trend.
        \item \textbf{Stop-Loss Positions:} Support and resistance levels are often used to set stop-loss orders to manage risk. A stop-loss might be placed just below a support level on long trades or just above a resistance level on short trades.
    \end{itemize}

    \bigbreak \noindent 
    \fig{.5}{./figures/Support and Resistance.png}












    \bigbreak \noindent 
    \subsection{Consolidating}
    \bigbreak \noindent 
    A consolidating market, often referred to as a consolidation phase, is characterized by an asset's price moving within a relatively stable range without a clear trend in either the upward or downward direction. This phase indicates a period where the supply and demand for an asset are roughly equal, leading to a balance in price movements. Consolidation is seen as a significant phase in the market because it often precedes a breakout into a new trend.
    \bigbreak \noindent 
    \subsubsection{Characteristics}
    \bigbreak \noindent 
    \begin{itemize}
        \item \textbf{Range-bound Movement:} The price oscillates between defined upper resistance and lower support levels, moving horizontally on a chart, which indicates that neither the buyers nor the sellers are in control.
        \item \textbf{Volume:} Often, trading volume diminishes during consolidation periods as traders are unsure of the market's direction and may hold off on making large bets.
        \item \textbf{Duration:} Consolidation can occur over various time frames, from relatively short periods (like days or weeks) to much longer durations (months or even years), depending on the asset and market conditions.
    \end{itemize}




    
\end{document}
