\documentclass{report}

\input{~/dev/latex/template/preamble.tex}
\input{~/dev/latex/template/macros.tex}

\title{\Huge{}}
\author{\huge{Nathan Warner}}
\date{\huge{}}
\fancyhf{}
\rhead{}
\fancyhead[R]{\itshape Warner} % Left header: Section name
\fancyhead[L]{\itshape\leftmark}  % Right header: Page number
\cfoot{\thepage}
\renewcommand{\headrulewidth}{0pt} % Optional: Removes the header line
%\pagestyle{fancy}
%\fancyhf{}
%\lhead{Warner \thepage}
%\rhead{}
% \lhead{\leftmark}
%\cfoot{\thepage}
%\setborder
% \usepackage[default]{sourcecodepro}
% \usepackage[T1]{fontenc}

% Change the title
\hypersetup{
    pdftitle={Futures}
}

\begin{document}
    % \maketitle
        \begin{titlepage}
       \begin{center}
           \vspace*{1cm}
    
           \textbf{Cyptocurrency Futures}
    
           \vspace{0.5cm}
            
                
           \vspace{1.5cm}
    
           \textbf{Nathan Warner}
    
           \vfill
                
                
           \vspace{0.8cm}
         
           \includegraphics[width=0.4\textwidth]{~/niu/seal.png}
                
           Computer Science \\
           Northern Illinois University\\
           United States\\
           
                
       \end{center}
    \end{titlepage}
    \tableofcontents
    \pagebreak 
    \unsect{What is futures trading?}
    \bigbreak \noindent 
    \begin{concept}
        Futures trading involves a contractual agreement to buy or sell a particular quantity of a commodity, financial instrument, or other tradable item at a predetermined price at a specified time in the future. These contracts are standardized and traded on futures exchanges.
        \bigbreak \noindent 
        In simple terms, imagine you're a farmer who grows wheat, and you want to lock in a price for your wheat that you'll harvest in a few months. You can enter a futures contract with a buyer who agrees to purchase your wheat at a set price when it's harvested. This way, you're protected if prices fall before you harvest, but you also agree to sell at this price even if market prices rise. Similarly, a bakery that needs wheat might use futures to lock in a price for wheat they'll buy in the future, protecting them against price rises but committing to the agreed price even if the market price falls. This mechanism allows both producers and consumers to hedge against price volatility.
    \end{concept}

    \bigbreak \noindent 
    \subsection{Types of futures trading}
    \bigbreak \noindent 
    \begin{itemize}
        \item \textbf{Commodity Futures:} This is one of the oldest forms of futures trading, involving physical commodities like agricultural products (wheat, corn, soybeans), metals (gold, silver, copper), and energy products (crude oil, natural gas). Traders can speculate on price movements or hedge against price volatility of these commodities.
        \item \textbf{Financial Futures:} These involve financial instruments as underlying assets. Examples include:
            \begin{itemize}
                \item \textbf{Interest Rate Futures:} Contracts based on the future movements of interest rates. Examples include Treasury bond futures, Treasury bill futures, and Eurodollar futures.
                \item \textbf{Currency Futures:} These contracts are agreements to exchange one currency for another at a future date at a predetermined price. They are used to hedge against or speculate on changes in currency exchange rates.
                \item \textbf{Stock Index Futures:} These futures contracts are based on the future value of a stock index, like the S\&P 500 or Dow Jones Industrial Average. Traders use these to speculate on the overall direction of the stock market or hedge investment portfolios.
            \end{itemize}
        \item \textbf{Cryptocurrency Futures:} As digital currencies have grown in popularity, futures markets for cryptocurrencies like Bitcoin and Ethereum have emerged. These allow traders to speculate on the future price movements of cryptocurrencies without owning the actual digital assets.
        \item \textbf{Real Estate Futures:} Although not as common, there are futures contracts based on real estate price indices. These can be used by investors to hedge against or speculate on changes in real estate markets.
        \item \textbf{Weather Futures:} These are financial derivatives that allow businesses to hedge against or speculate on weather-related risks, such as temperature fluctuations, rainfall, or snowfall.
    \end{itemize}
    \bigbreak \noindent 
    \nt{This document will focus on \textit{cryptocurrency futures}}

    \bigbreak \noindent 
    \subsection{Standard Futures vs USDT-M Perp Futures vs Coin-M Perp Futures}

    \bigbreak \noindent 
    \subsubsection{Standard Futures}
    \bigbreak \noindent 
    \begin{itemize}
        \item \textbf{Definition:} Standard futures are traditional futures contracts found in cryptocurrency trading that involve agreeing to buy or sell a specific amount of a cryptocurrency at a predetermined price at a specified time in the future.
        \item \textbf{Settlement:} These contracts can be either physically settled (where actual cryptocurrency changes hands) or cash-settled (where the difference in price is exchanged in fiat currency or another form of value).
        \item \textbf{Expiration:} They typically have a set expiration date upon which the contract must be settled.
    \end{itemize}

    \bigbreak \noindent 
    \subsubsection{USDT-Margined Perpetual Futures (USDT-M Perp Futures)}
    \bigbreak \noindent 
    \begin{itemize}
        \item \textbf{Definition:} These are a type of futures contract that does not have an expiration date, meaning they can be held indefinitely (perpetual).
        \item \textbf{Collateral:} These contracts are margined and settled in USDT (Tether), a stablecoin pegged to the US dollar. This means the profits, losses, margin calls, and settlements are handled in USDT.
        \item \textbf{Price Mechanism:} The price of these contracts often tracks the underlying asset very closely. To maintain this alignment, mechanisms like funding rates are used, where payments are exchanged between long and short positions depending on market conditions.
    \end{itemize}

    \bigbreak \noindent 
    \subsubsection{Coin-Margined Perpetual Futures (Coin-M Perp Futures)}
    \bigbreak \noindent 
    \begin{itemize}
        \item \textbf{Definition:} Like USDT-Margined Perpetual Futures, these are also perpetual and do not have an expiration date.
        \item \textbf{Collateral:} The key difference is that these contracts use the cryptocurrency itself as collateral. For example, if you are trading Bitcoin futures, the margin and settlement could be handled in Bitcoin.
        \item \textbf{Settlement Currency:} Settlements, profits, and losses are transacted in the cryptocurrency used as collateral, making them potentially more volatile due to the price movements of the cryptocurrency itself.
        \item \textbf{Price Alignment:} These also use mechanisms like funding rates to keep the futures price aligned with the spot price of the cryptocurrency.
    \end{itemize}

    \bigbreak \noindent 
    \unsect{Futures contract/trading pair}
    \bigbreak \noindent 
    For example, When you see "BTCUSDT" on a cryptocurrency futures exchange, it denotes a specific type of futures contract or trading pair involving Bitcoin (BTC) and Tether (USDT). 
    \bigbreak \noindent 
    \begin{itemize}
        \item \textbf{BTC (Bitcoin):} This represents Bitcoin, the first and most widely known cryptocurrency. In the context of a trading pair or futures contract, it's the primary asset that's being traded or speculated upon.
        \item \textbf{USDT (Tether):} USDT is a stablecoin, which is a type of cryptocurrency designed to maintain a stable value over time. Tether aims to be pegged to the US dollar, with 1 USDT theoretically equal in value to 1 USD. Stablecoins like USDT are often used in cryptocurrency trading to provide a stable base or quote currency against the often volatile prices of cryptocurrencies like Bitcoin.
        \item \textbf{BTCUSDT Futures Contract:} When combined in a trading pair like BTCUSDT on a futures exchange, it implies that the futures contract is for speculating on the future price of Bitcoin (BTC) against the value of Tether (USDT). Participants in this market are essentially betting on what the price of Bitcoin will be in USDT terms at a specified future date when the contract expires.
        \item \textbf{Purpose and Use:} Traders use BTCUSDT futures contracts for various purposes, including hedging against future price movements of Bitcoin, speculating on the price direction of Bitcoin, and managing portfolio exposure without the need to hold actual Bitcoins. The use of USDT as the quote currency allows traders to manage their risk and returns in a currency whose value is more stable than other cryptocurrencies, making it easier to calculate profits, losses, and margins.
        \item \textbf{Settlement:} Depending on the specific terms of the futures contract, settlement can be either physical (actual delivery of Bitcoin in exchange for USDT) or cash-settled (the net difference in value is exchanged in USDT). Most cryptocurrency futures contracts, especially those involving stablecoins like USDT, are cash-settled.
    \end{itemize}


    \pagebreak 
    \unsect{Acquiring Positions}
    \bigbreak \noindent 
    \subsection{Going Long}
    Suppose we go long on the BTCUSDT pair. Placing a "long" position on BTCUSDT in the context of a futures exchange means you are entering a contract to buy Bitcoin at a future date at a price specified now, with the expectation that the price of Bitcoin (in terms of Tether, USDT) will increase
    \bigbreak \noindent 
    \begin{itemize}
        \item \textbf{Opening the Position:} When you decide to go long on BTCUSDT, you are essentially betting that the price of Bitcoin will rise relative to USDT. You enter a futures contract at the current market price, which is the agreed price at which you will buy Bitcoin in the future. This does not require you to own Bitcoin at the time you open the position.
        \item \textbf{Margin Requirement:} To open a long position, you'll need to post a margin, which is a fraction of the total value of the contract. This margin acts as collateral to cover potential losses. Futures trading is leveraged, meaning you can control a large contract value with a relatively small amount of capital, amplifying both potential gains and losses.
        \item \textbf{Price Movement:} After you open your position, one of two things can happen regarding the price of Bitcoin:
            \begin{itemize}
                \item If the price of Bitcoin increases (against USDT) as you predicted, your long position will profit. The profit is generally the difference between the price at which you entered the contract and the higher price of Bitcoin at the time of settlement or when you decide to close the position early.
                \item If the price of Bitcoin decreases, your position will incur a loss, proportional to the decline in price from your entry point.
            \end{itemize}
        \item \textbf{Maintenance Margin:} If your position moves into a loss that exceeds the initial margin, you may be subject to a margin call, requiring you to deposit additional funds to maintain your position. Failure to meet the margin call can lead to the forced liquidation of your position by the exchange.
        \item \textbf{Settlement or Closing the Position:} Your long position on BTCUSDT can be closed in two main ways:
            \begin{itemize}
                \item \textbf{Manual Closure:} You can choose to close your position before the contract expires by taking an opposite action in the market (selling the contract you initially bought). If you’re in profit, this locks in your gains; if you’re at a loss, it stops further losses.
                \item \textbf{Expiration:} If held to expiration, the contract will settle based on the terms of the futures contract. For cash-settled contracts, the difference between the entry price and the settlement price will be credited to or debited from your account in USDT.
            \end{itemize}
    \end{itemize}
    \bigbreak \noindent 
    \subsection{Going Short}
    Again, suppose we are trading the BTCUSDT pair. Shorting on BTCUSDT in the context of a cryptocurrency futures exchange means you are entering into a contract to sell Bitcoin at a future date at a price specified now, with the expectation that the price of Bitcoin (in terms of Tether, USDT) will decrease. Here’s how shorting works in the futures market:
    \begin{itemize}
        \item \textbf{Opening the Position:} To short BTCUSDT, you sell a futures contract at the current market price, which is the agreed price at which you commit to selling Bitcoin in the future. This position profits from a decline in Bitcoin's price relative to USDT. Importantly, you do not need to own Bitcoin when you initiate a short position.
        \item \textbf{Margin Requirement:} Similar to going long, shorting requires posting a margin, a fraction of the total contract value, serving as collateral. Futures are leveraged, allowing you to control a significant position with a smaller amount of capital, which amplifies potential gains and losses.
        \item \textbf{Price Movement and Profit/Loss:} After opening your short position, the following outcomes are possible:
            \begin{itemize}
                \item If the price of Bitcoin decreases as anticipated, your short position will be profitable. The profit is generally the difference between the price at which you entered the short contract and the lower price of Bitcoin when you close or settle the position.
                \item If the price of Bitcoin increases, contrary to your expectation, your position will incur a loss, proportional to the rise in price from your entry point.
            \end{itemize}
    \end{itemize}

    \bigbreak \noindent 
    \unsect{Characteristics of Cryptocurrency Future Contracts} 
    \bigbreak \noindent 
    \begin{itemize}
        \item \textbf{Standardization:} Like traditional futures, cryptocurrency futures contracts are standardized in terms of contract size, expiration date, and settlement methods, facilitating ease of trade on the exchange.
        \item \textbf{Settlement:} Contracts can be settled either through physical delivery (less common in cryptocurrency futures) or cash settlement. Most cryptocurrency futures are cash-settled, meaning the difference between the purchase price and the selling price is exchanged in fiat currency or stablecoins like USDT (Tether).
        \item \textbf{Leverage:} Cryptocurrency futures trading is typically leveraged, allowing traders to gain a large exposure to the underlying digital currency with a relatively small amount of capital (known as the "margin").
    \end{itemize}
    \bigbreak \noindent 
    \subsection{Contract sizes}
    \bigbreak \noindent 
    The contract size in cryptocurrency futures denotes the amount of cryptocurrency that each contract represents. This standardization is crucial for understanding the scale of one’s investment and potential exposure to market movements.
    \bigbreak \noindent 
    \begin{itemize}
        \item \textbf{Determining Exposure:} The contract size determines how much of the cryptocurrency's price movement will impact the trader’s position. A larger contract size means greater exposure to price changes and, consequently, higher potential profits or losses.
        \item \textbf{Margin Requirements:} Since cryptocurrency futures are leveraged, the contract size also influences the margin requirement — the amount of capital a trader must deposit to open a position. Exchanges specify margin requirements as a percentage of the contract's total value. Smaller contract sizes can lower the barrier to entry for individual traders by requiring less capital upfront.
        \item \textbf{Flexibility and Accessibility:} Exchanges often offer contracts with various sizes to cater to different types of traders. For instance, besides standard contracts, there might be "mini" or "micro" contracts available with smaller sizes, making it easier for individual investors to participate in the market.
    \end{itemize}
    \bigbreak \noindent 
    \subsubsection{Examples}
    \bigbreak \noindent 
    \begin{itemize}
        \item A Bitcoin futures contract might have a contract size of 1 BTC, meaning each contract represents the future delivery of 1 Bitcoin.
        \item A smaller contract, such as a micro Bitcoin futures contract, might have a contract size of 0.1 BTC, offering lower margin requirements and less capital exposure.
    \end{itemize}

    \pagebreak 
    \unsect{Point increase and ticks}
    \bigbreak \noindent 
    In cryptocurrency futures trading, just as in traditional futures markets, the term "tick" is often used to describe the minimum point increase or decrease in the price of a futures contract. However, the general concept of a "point increase" remains applicable and is understood in the context of how much the price of a cryptocurrency futures contract has risen, measured in the contract's quote currency (e.g., USDT, USD).
    \bigbreak \noindent 
    \subsection{Tick size}
    \begin{itemize}
        \item \textbf{Definition:} A tick is the smallest possible price change on the upside or downside. Each futures contract, whether it's for cryptocurrencies or other assets, has a specified "tick size" that determines the minimum increment by which the price can move.
        \item \textbf{Example:} If the tick size of a BTCUSDT futures contract is 0.5 USDT, it means the price of the contract can move up or down in increments of 0.5 USDT. A move from 50,000 USDT to 50,000.5 USDT would represent a one-tick increase.
    \end{itemize}
    \bigbreak \noindent 
    \subsection{Point increase}
    \begin{itemize}
        \item \textbf{Understanding Point Increase:} In the context of cryptocurrency futures, a "point" typically refers to a whole number movement in the price of the futures contract, as opposed to fractional movements (ticks). The actual value of a "point" in terms of the underlying cryptocurrency (e.g., Bitcoin, Ethereum) can vary depending on the contract size and the currency in which the contract is denominated.
        \item \textbf{Significance:} A point increase in a cryptocurrency futures contract indicates a rise in the contract's price, which can be beneficial for traders holding long positions. For example, in a BTCUSDT futures contract, if the price of Bitcoin increases by 100 USDT, that would be a 100-point increase, benefiting traders who speculated on the price going up.
    \end{itemize}

    \pagebreak 
    \unsect{The Order Book}
    \bigbreak \noindent 
    An order book in a futures exchange is a real-time, continuously updated list that records all the buy and sell orders for futures contracts. Each order listed in the book displays the number of contracts being offered and the price level at which the buyer or seller is willing to trade. The order book is a crucial tool for traders and investors, providing deep insights into the market's current demand and supply dynamics, liquidity, and potential price movements
    \bigbreak \noindent 
    \subsection{Components}
    \bigbreak \noindent 
    \begin{itemize}
        \item \textbf{Bid:} The bid side of the order book lists all the buy orders at various price levels, with the highest bid price at the top. This price represents the highest price that buyers are currently willing to pay for a futures contract.
        \item \textbf{Ask (Offer):} The ask side lists all the sell orders, with the lowest ask price at the top. This is the lowest price at which sellers are willing to sell a futures contract.
        \item \textbf{Price Levels:} Each price level shows the total number of contracts buyers are willing to buy or sell at that specific price.
        \item \textbf{Depth of Market:} The order book shows the depth of market for a futures contract by displaying the number of buy and sell orders at different price levels. This depth can indicate the liquidity of the market; a deeper market (with more orders) typically suggests higher liquidity.
    \end{itemize}

    \pagebreak 
    \unsect{Mark Price vs Index Price}
    \bigbreak \noindent 
    In cryptocurrency futures trading, the terms "mark price" and "index price" refer to two different but related pricing mechanisms used to manage and value futures contracts
    \bigbreak \noindent 
    \subsection{Index Price}
    \bigbreak \noindent 
    The index price is a benchmark price that reflects the real-time spot price of a cryptocurrency. It's derived from an average price of the cryptocurrency across multiple spot exchanges. The purpose of the index price is to provide a fair, transparent, and manipulation-resistant reference price for futures contracts. Here are some key points about the index price:
    \begin{itemize}
        \item \textbf{Aggregation:} It is usually calculated as a weighted average of the spot prices from several major exchanges.
        \item \textbf{Purpose:} The index price is used to determine the settlement prices of futures contracts and to ensure that the futures prices are aligned with the underlying market.
        \item \textbf{No Trading:} You cannot trade directly at the index price; instead, it serves as a reference for the trading of futures contracts.
    \end{itemize}

    \bigbreak \noindent 
    \subsection{Mark Price}
    \bigbreak \noindent 
    The mark price is the price at which a cryptocurrency futures contract is currently valued for trading purposes on an exchange. It plays a crucial role in the management of leveraged positions and is used primarily to prevent unnecessary liquidations due to market manipulation or illiquidity. Here are some important aspects of the mark price:
    \begin{itemize}
        \item \textbf{Calculation:} The mark price is often calculated by taking the index price and adjusting it with the funding rate or interest rate differentials between the futures prices and the spot prices.
        \item \textbf{Function:} Its main function is to trigger liquidations and to calculate unrealized profit and loss on open positions. This helps ensure that the futures market remains closely tied to the actual market conditions reflected in the index price.
        \item \textbf{Safety Mechanism:} By using the mark price instead of the last traded price for margin calculations and liquidations, exchanges protect traders from price spikes or dumps that might occur in less liquid trading environments.
    \end{itemize}

    \pagebreak 
    \unsect{Technical Analysis}
    \bigbreak \noindent 
    \subsection{Impulse and Pullback}
    \bigbreak \noindent 
    \subsubsection{Impulse}
    \bigbreak \noindent 
    An impulse refers to a strong, directional movement in the price of an asset, aligning with the prevailing trend. In a bullish trend, an impulse would be a sharp move upwards, whereas in a bearish trend, it would be a significant move downwards. Here are some characteristics of impulses:
    \begin{itemize}
        \item \textbf{Strong Movement:} Impulses are characterized by rapid and significant price movements that clearly indicate the direction of the trend.
        \item \textbf{Volume Support:} These moves are often supported by high trading volumes, suggesting strong market participation and commitment to the trend direction.
        \item \textbf{Psychological Impact:} Impulsive moves can reinforce the existing sentiment about the asset, either bullish or bearish, as they reflect a consensus among traders about the value direction.
    \end{itemize}
    \bigbreak \noindent 
    \subsubsection{Pullback}
    \bigbreak \noindent 
    \begin{itemize}
        \item \textbf{Correction:} Pullbacks allow the market to "breathe" by letting prices correct from an overextended move, helping to mitigate the risks of entering at peak levels.
        \item \textbf{Entry Opportunities:} For traders looking to enter a trend, pullbacks often provide safer entry points as they can buy into the trend at a lower price in an uptrend or sell at a higher price in a downtrend.
        \item \textbf{Volume Observation:} Typically, pullbacks occur on lower volumes compared to impulses, suggesting less conviction in the counter-trend movement.
    \end{itemize}

    % \bigbreak \noindent 
    % \nt{We usually consider the advance up to the new higher high to be the impulse, and the correction of this impulse to be the pullback. Small movements up and down during the pullback are not labeled impulse or pullback.}

    \pagebreak 
    \subsection{Trending Markets}
    \bigbreak \noindent 
    \subsubsection{Uptrend}
    \bigbreak \noindent 
    An uptrend in the context of financial markets, such as forex, stocks, or commodities, refers to a situation where the price of an asset shows a general pattern of increasing over time. It's essentially a period during which buyers are in control, pushing prices higher. Here are the key characteristics and implications of an uptrend:
    \begin{itemize}
        \item \textbf{Higher Highs and Higher Lows:} The most identifiable feature of an uptrend is a series of higher highs (peaks) and higher lows (troughs). This means each peak in the price action is higher than the previous peak, and each trough is higher than the previous trough, indicating sustained buying interest.
        \item \textbf{Trend Lines:} An uptrend can often be visually represented by drawing a trend line that connects the lows in the price action. This trend line acts as a support level that can indicate potential buying opportunities when the price touches or comes close to it.
        \item \textbf{Moving Averages:} In an uptrend, shorter-term moving averages (like the 50-day moving average) typically stay above longer-term moving averages (like the 200-day moving average), and the price is often above these moving averages. When these moving averages cross, it can signal a strengthening of the trend.
    \end{itemize}
    \bigbreak \noindent 
    \textbf{Implications}
    \begin{itemize}
        \item \textbf{Market Sentiment:} An uptrend signifies positive sentiment toward the asset, suggesting that the market participants believe in its potential for growth or are optimistic about its future.
        \item \textbf{Investment Strategy:} Traders and investors might see an uptrend as a favorable time to buy into the market, anticipating further price increases. However, it's crucial to consider the uptrend's maturity; entering a trade early in an uptrend can be more rewarding than entering during a late stage, where the risk of a trend reversal might be higher.
        \item \textbf{Risk Management:} While an uptrend can present opportunities for profit, it's important for traders to employ risk management strategies. Setting stop-loss orders and taking profits at predetermined levels can help protect against sudden reversals.
    \end{itemize}
    \bigbreak \noindent 
    \fig{.4}{./figures/Uptrend.png}

    \bigbreak \noindent 
    \fig{.6}{./figures/upternd-market.png}

    \bigbreak \noindent 
    \nt{As long as the next trough does not surpase (go lower) than the previous trough, the market remains in an uptrend. The path to the new peak is considered the impulse. After each impulsive move there will be pullbacks, these pullbacks are the paths to the new troughs. Hence, as long as the new pullback does not surpase the previous, the market is still in an uptrend.}

    \pagebreak 
    \subsection{Downtrend}
    \bigbreak \noindent 
    A downtrend in financial markets is characterized by a series of lower highs and lower lows in the price of an asset, reflecting a sustained decrease in its value over time. This pattern indicates that the demand for the asset is weakening relative to its supply, leading to a decline in price. Here’s a detailed breakdown of the concept of a downtrend:
    \bigbreak \noindent 
    \subsubsection{Identification of a Downtrend}
    \bigbreak \noindent 
    \begin{itemize}
        \item \textbf{Lower Highs and Lower Lows:} The most fundamental characteristic of a downtrend is that each successive peak (high) and trough (low) in price is lower than the previous ones. This pattern can be visually identified on a chart and is indicative of ongoing selling pressure.
        \item \textbf{Trendlines:} A downtrend can often be confirmed by drawing a descending trendline that connects the lower highs. This trendline acts as resistance; the price struggles to break above this line and often bounces off it, continuing the downtrend.
        \item \textbf{Moving Averages:} Technical analysts might use moving averages to identify and confirm downtrends. For instance, when a shorter-term moving average (like the 50-day moving average) crosses below a longer-term moving average (like the 200-day moving average), it can signal the beginning or continuation of a downtrend.
    \end{itemize}
    \bigbreak \noindent 
    \subsubsection{Psychological and Market Dynamics}
    \begin{itemize}
        \item \textbf{Selling Pressure:} In a downtrend, increased selling pressure is often driven by negative sentiment towards the asset, which could be due to fundamental reasons (like deteriorating company performance or economic conditions) or technical reasons (like breaking key support levels).
        \item \textbf{Fear and Capitulation:} As prices continue to fall, the market sentiment often shifts towards fear, which can lead to capitulation—where investors give up any previous hopes of recovery and sell their positions, further driving down prices.
    \end{itemize}
    \pagebreak 
    \subsubsection{Figures}
    \bigbreak \noindent     
    \fig{.5}{./figures/price-downtrend.png}
    \bigbreak \noindent 
    \fig{.3}{./figures/EURUSD-4-hour-chart-56a22ddb3df78cf77272e828.jpg}


    \pagebreak 
    \subsection{Support and Resistance}
    \bigbreak \noindent 
    Support and resistance are fundamental concepts in technical analysis used by traders to identify potential price levels on a chart where an asset's price will tend to stop and reverse. These levels are determined by the past interactions between buyers and sellers and are often where an influx of buying or selling activity can be expected.
    \bigbreak \noindent 
    \subsubsection{Support}
    \bigbreak \noindent 
    Support refers to a price level where a downtrend can be expected to pause due to a concentration of demand (buying pressure). As the price of an asset drops, demand for the asset increases, thus forming the support line. This happens because:
    \begin{itemize}
        \item \textbf{Perceived Value:} Investors perceive the asset as more attractive or undervalued at lower prices, leading to increased purchases.
        \item \textbf{Psychological Effect:} The support level also often represents a psychological barrier where traders expect prices to bounce back.
        \item \textbf{Historical Interaction:} Traders look at historical price levels where an asset has previously ceased to drop further, expecting the pattern to repeat.
    \end{itemize}
    \bigbreak \noindent 
    When the price approaches a support level but does not break below it, the support is considered \textbf{strong}. If the price does break below the support level, it might go on to establish a new lower support level.

    \bigbreak \noindent 
    \subsubsection{Resistance}
    \bigbreak \noindent 
    Resistance is the opposite of support, representing a price level where a price uptrend is likely to pause or stop due to a concentration of supply (selling pressure). As the price of an asset rises, the inclination to sell it increases, thus forming the resistance line. This occurs because:
    \begin{itemize}
        \item \textbf{Profit Taking:} Higher prices entice traders to sell to capture profits, increasing the supply over demand.
        \item \textbf{Psychological Barrier:} Like support, resistance levels can also represent psychological price points that the market finds too expensive or overvalued.
        \item \textbf{Historical Price Levels:} Resistance is often established by looking at price levels where an asset has previously stopped rising.
    \end{itemize}
    \bigbreak \noindent 
    If a price reaches a resistance level but fails to break through it, the resistance is deemed \textbf{strong}. However, if the price does break through the resistance, it could climb until it finds a new upper resistance level.
    \bigbreak \noindent 
    \subsubsection{Trading with Support and Resistance}
    \begin{itemize}
        \item \textbf{Entry and Exit Points:} Support and resistance levels help traders determine strategic entry and exit points based on the expectation that prices will move between these levels until they don't. Buying near support and selling near resistance can optimize risk/reward.
        \item \textbf{Breakouts and Breakdowns:} A price moving past a support or resistance level might indicate a strong move in the direction of the breakout. Traders might use this as a signal to enter a trade anticipating a new trend.
        \item \textbf{Trend Confirmation:} Sustained movement above a resistance level or below a support level might indicate a change in the broader market trend.
        \item \textbf{Stop-Loss Positions:} Support and resistance levels are often used to set stop-loss orders to manage risk. A stop-loss might be placed just below a support level on long trades or just above a resistance level on short trades.
    \end{itemize}

    \bigbreak \noindent 
    \fig{.5}{./figures/Support and Resistance.png}

    \bigbreak \noindent 
    \subsection{The break and retest strategy (Using support and resistance)}
    \bigbreak \noindent 
    \subsubsection{Break of a Level}
    \bigbreak \noindent 
    \begin{itemize}
        \item \textbf{Breakthrough:} This strategy hinges on the price breaking through either a support or resistance level with significant volume. This break is seen as a potential signal that the market sentiment is strong enough to establish a new trend.
        \item \textbf{Confirmation:} Traders often look for additional confirmation that the break is genuine. This could be in the form of an increase in trading volume or additional technical indicators like moving averages or momentum oscillators.
    \end{itemize}
    \bigbreak \noindent 
    \subsubsection{Retest of the Broken Level}
    \bigbreak \noindent 
    \begin{itemize}
        \item \textbf{Pullback or Retest:} After breaking a key level, prices often pull back to the level it just broke through, effectively retesting it. This is where the level that was previously support may become resistance (in the case of a downward break), or resistance turns into support (in the case of an upward break).
        \item \textbf{Entry Point:} The retest phase is critical because it provides a lower-risk entry point. Traders look for signs that the price is rejecting the old level in its new role (as new support or new resistance) before entering a trade.
    \end{itemize}
    \bigbreak \noindent 
    \subsubsection{Trade Execution}
    \begin{itemize}
        \item \textbf{Entry:} Traders enter a trade based on the rejection of the retested level. For example, if the price breaks above resistance and then falls back to test this level but does not break below it, a trader might take a long position.
        \item \textbf{Stop Loss and Take Profit:} Proper risk management involves setting stop-loss orders below or above the retested level (depending on the direction of the trade) to minimize potential losses. Profit targets are often set based on previous price swings or using a risk-reward ratio.
    \end{itemize}
    \bigbreak \noindent 
    \fig{.5}{./figures/main-qimg-6a1d999c04cd99142d111231d7298155-lq.jpeg}


    \bigbreak \noindent 
    \subsection{Consolidating}
    \bigbreak \noindent 
    A consolidating market, often referred to as a consolidation phase, is characterized by an asset's price moving within a relatively stable range without a clear trend in either the upward or downward direction. This phase indicates a period where the supply and demand for an asset are roughly equal, leading to a balance in price movements. Consolidation is seen as a significant phase in the market because it often precedes a breakout into a new trend.
    \bigbreak \noindent 
    \subsubsection{Characteristics}
    \bigbreak \noindent 
    \begin{itemize}
        \item \textbf{Range-bound Movement:} The price oscillates between defined upper resistance and lower support levels, moving horizontally on a chart, which indicates that neither the buyers nor the sellers are in control.
        \item \textbf{Volume:} Often, trading volume diminishes during consolidation periods as traders are unsure of the market's direction and may hold off on making large bets.
        \item \textbf{Duration:} Consolidation can occur over various time frames, from relatively short periods (like days or weeks) to much longer durations (months or even years), depending on the asset and market conditions.
    \end{itemize}
    \bigbreak \noindent 
    \nt{This market pattern is also referred to as sideways, or choppy}
    \bigbreak \noindent 
    \fig{.3}{./figures/8.-sideways-market.png}
    \bigbreak \noindent 
    \fig{.6}{./figures/sideways-trend.png}



    \bigbreak \noindent 
    \subsection{Indicators}
    \bigbreak \noindent 
    Indicators in technical analysis are mathematical calculations based on the price, volume, or open interest of a security or contract used by traders to forecast future price movements. These indicators can provide unique insights into the bullish or bearish momentum and can help to identify potential reversal points or confirm trend direction.
    \bigbreak \noindent 
    \subsubsection{Average True Range (ATR)}
    \bigbreak \noindent 
    The Average True Range (ATR) is a technical analysis indicator that measures market volatility by decomposing the entire range of an asset for a specific period
    \bigbreak \noindent 
    The atr works by summing the change in pips for the last 14 candles, and then divides by 14 to get the \textbf{average} price momevent (in terms of $\Delta$ pip) over those 14 candles
    \bigbreak \noindent 
    \fig{.4}{./figures/atr.png}
    \bigbreak \noindent 
    We see here that the ATR is 285.90. Thus, this is the average movement
    \bigbreak \noindent 
    To use this knowledge to place trades, we simply look at the previous pullbacks wick, and position our stop loss 1 atr past that wick. For take profits, we may position 2xATR

    \bigbreak \noindent 
    \subsection{Indicators: Moving averages}
    \bigbreak \noindent 
     A moving average (MA) calculates the average price of a futures contract over a specific number of past periods. The most typical periods used are 20, 50, 100, and 200 days, but these can be adjusted based on the trader's strategy and the time frame they are analyzing.
     \bigbreak \noindent 
     \subsubsection{Types of Moving Averages}
     \bigbreak \noindent 
     \begin{itemize}
         \item \textbf{Simple Moving Average (SMA):} This is the average price over a specific period of time. For example, a 50-day SMA is the sum of the last 50 closing prices divided by 50.
         \item \textbf{Exponential Moving Average (EMA):} This type gives more weight to recent prices in an attempt to make it more responsive to new information. It reacts faster to price changes than the SMA.
     \end{itemize}
     \bigbreak \noindent 
     \subsubsection{Applications}
     \begin{itemize}
         \item \textbf{Trend Identification:} If the price of the futures contract is above a certain moving average, it suggests an uptrend, and if it's below, a downtrend. For example, many traders view a futures contract trading above its 200-day moving average as in a long-term uptrend.
         \item \textbf{Support and Resistance Levels:} Moving averages can act as levels of support in a downtrend (where price might stop falling and reverse) or resistance in an uptrend (where price might stop rising and reverse).
     \end{itemize}
     \bigbreak \noindent 
     \fig{.2}{./figures/ma.jpg}

     \bigbreak \noindent 
     \subsubsection{Choose the Right Moving Average}
     \bigbreak \noindent 
      Shorter lengths (like 10 or 20 periods) react faster to price changes, providing tighter stops. Longer lengths (like 50 or 100 periods) give more room for the trade to breathe and are less responsive to minor price fluctuations.
      \bigbreak \noindent 
      The most common lengths are 20, 50, and 200.

     \bigbreak \noindent 
     \subsubsection{Defining trends with moving averages}
     \bigbreak \noindent 
     \begin{concept}
        If price is trading above the moving average, this signifies an uptrend. Conversely, if price is trading below the moving averge, this signifies a downtrend. 
     \end{concept}
     \bigbreak \noindent 
     \subsubsection{Identify areas of value with moving averages}
     \bigbreak \noindent 
     We can use moving averages along side the concepts of support and resistance to identify potential areas of value.
     \bigbreak \noindent 
     \fig{.4}{./figures/grade4-exponential-moving-averages-break2.png}
     \bigbreak \noindent 
     \fig{.4}{./figures/Moving-Average-Support-and-Resistance.png}
     \bigbreak \noindent 
     \subsubsection{Moving averages as a trailing stop}
     \bigbreak \noindent 
     A trailing stop, also called a trailing stop-loss, is a type of market order that sets a stop-loss at a specific percentage below an asset's market price, rather than on a single value. The stop-loss then trails behind the stock as its price moves.
     \bigbreak \noindent 
     Using a moving average as a trailing stop is a dynamic way to manage trades and lock in profits while providing a mechanism to allow a position to continue to benefit from favorable price movements. Here's how to effectively use a moving average as a trailing stop in trading:
     \bigbreak \noindent 
     As the moving average moves, adjust the stop loss accordingly:
     \begin{itemize}
         \item \textbf{If the market is trending upwards in a long position}, the moving average will rise, and the trailing stop should be moved up to stay a fixed distance below the moving average.
         \item \textbf{If the market is trending downwards in a short position}, the moving average will fall, and the trailing stop should be moved down to maintain a fixed distance above the moving average.
     \end{itemize}

     \bigbreak \noindent 
     \subsection{Indicators: Relative Strength Index (RSI)}
     \bigbreak \noindent 
     The Relative Strength Index (RSI) is a popular momentum oscillator used in technical analysis to measure the speed and change of price movements of a security.
     \bigbreak \noindent 
     RSI is commonly used to identify overbought or oversold conditions in a market, making it a valuable tool for traders looking to gauge potential price reversals.
     \bigbreak \noindent 
     Many people use this indicator as a way to forsee potential reversals, we infact it is best used to see when price has been moving in a certain direction for a long period of time.
     \bigbreak \noindent 
     \subsubsection{How RSI is Calculated}
     \bigbreak \noindent 
     The RSI is calculated using the following steps:
     \begin{itemize}
         \item \textbf{Average Gain and Loss:} First, the average gains and average losses over a specific period are calculated. This period is typically 14 days, but it can be adjusted according to the trader's needs.
         \item \textbf{RS Calculation:} The Relative Strength (RS) is then calculated by dividing the average gain by the average loss.
         \item \textbf{RSI Formula:} The RSI value is derived using the formula:
             \begin{align*}
                 RSI = 100 - \left(\frac{100}{1+RS}\right)
             .\end{align*}
     \end{itemize}
     This formula produces a value that ranges between 0 and 100.
     \bigbreak \noindent 
     \subsubsection{Interpreting RSI}
     \bigbreak \noindent 
     \begin{itemize}
         \item \textbf{Overbought and Oversold Levels:} RSI values over 70 typically indicate that a security is becoming overbought or overvalued and may be primed for a trend reversal or corrective pullback in price. Conversely, an RSI reading below 30 is generally seen as an indication that a security is oversold or undervalued.
         \item \textbf{Centerline Crossover:} The 50 level is often considered a midpoint that can signal the overall direction of the momentum. If the RSI is above 50, it suggests bullish momentum, while below 50 indicates bearish momentum.
     \end{itemize}
     \bigbreak \noindent 
     \fig{.6}{./figures/RSI1_602x345.png}

     \bigbreak \noindent 
     \subsubsection{Divergence}
     \bigbreak \noindent 
     If the RSI diverges from the price, it suggests that the current trend might be weakening. For example, if the price hits a new high but the RSI fails to surpass its previous high, it could indicate a loss of momentum.
     \bigbreak \noindent 
     \fig{.2}{./figures/dotdash_INV-final-Divergence-Definition-and-Uses-Apr-2021-01-41d9b314d3a645118a911367acce55a7.jpg}

     \pagebreak 
     \subsection{Candlestick patterns}
     \bigbreak \noindent 
     Candlestick patterns are used to predict future market behavior based on past price actions. Some common uses include:
     \bigbreak \noindent 
     \begin{itemize}
         \item \textbf{Trend Identification:} Patterns can suggest the continuation or reversal of a trend. For instance, a series of ascending green candlesticks suggests a strong upward trend.
         \item \textbf{Momentum Indicators:} Certain patterns indicate whether a price movement is likely to continue. For example, a "bullish engulfing" pattern suggests that buyers are gaining control and may push prices higher.
         \item \textbf{Price Reversals:} Reversal patterns indicate that the current price trend might be changing direction. A "doji," which has a small body and equal shadows, suggests indecision that could precede a reversal.
         \item \textbf{Support and Resistance Levels:} Candlestick patterns can help identify crucial price levels that are likely to cause price to bounce back (support) or fall back (resistance).
     \end{itemize}

     \bigbreak \noindent 
     \subsection{Candlestick Shadow (Wick)}
     \bigbreak \noindent 
     A shadow, or a wick, is a line found on a candle in a candlestick chart that is used to indicate where the price of a stock has fluctuated relative to the opening and closing prices.
     \bigbreak \noindent 
     \fig{.5}{./figures/shadow.png}
     \bigbreak \noindent 
     \fig{.4}{./figures/shadow2.png}

     \bigbreak \noindent 
     \subsection{Short/Long Body candles}
     \bigbreak \noindent 
     The length of the body in a candlestick provides significant insights into market sentiment and price action during the time period it represents. Understanding the implications of short and long bodies can help traders gauge the strength of the current market movements and potential future trends. Here’s how these can be interpreted:
     \bigbreak \noindent 
     \subsubsection{Short Body}
     \bigbreak \noindent 
     A candlestick with a short body, where the open and close prices are close to each other, signifies a few key aspects:
     \begin{itemize}
         \item \textbf{Indecision:} A short body indicates that there was not much difference between the opening and closing prices, suggesting indecision or equilibrium between buyers and sellers. This is often seen in periods of low volatility.
         \item \textbf{Potential Reversal:} If a short-bodied candlestick appears after a prolonged trend (either up or down), it can signal that momentum is waning and might lead to a reversal, as neither bulls nor bears could gain significant ground.
         \item \textbf{Pause in Trend:} It can also indicate a consolidation phase, where the market is taking a pause before continuing the prevailing trend.
     \end{itemize}

     \bigbreak \noindent 
     \subsubsection{Long Body}
     \bigbreak \noindent 
     A candlestick with a long body, where there is a significant difference between the open and close prices, indicates strong market activity and sentiment:
     \begin{itemize}
         \item \textbf{Strong Buying Pressure:} In a bullish candlestick (usually green or white), a long body indicates strong buying pressure. The closing price ended much higher than the opening price, showing that buyers were aggressive and in control throughout the trading session.
         \item \textbf{Strong Selling Pressure:} Conversely, in a bearish candlestick (usually red or black), a long body indicates strong selling pressure. The closing price is significantly lower than the opening price, suggesting that sellers dominated the session.
     \end{itemize}

     \bigbreak \noindent 
     \subsubsection{Importance}
     \bigbreak \noindent 
     The significance of the body length also depends on the context in which it appears:
     \begin{itemize}
         \item \textbf{Trends:} During a strong uptrend or downtrend, long bodies of the predominant color (green in uptrends, red in downtrends) confirm the strength of the trend. Frequent long bodies in one direction often indicate sustained momentum.
         \item \textbf{Market Turns:} Transitioning from long bodies to progressively shorter bodies can signal that the current trend is losing steam and might be near a turning point.
         \item \textbf{Volume and Other Indicators:} The interpretation of candlestick body lengths can be further refined by considering the trading volume associated with them and other technical indicators like moving averages or oscillators.
     \end{itemize}

     \bigbreak \noindent 
     \subsection{Candlestick patterns: 38.2\% candle}
     \bigbreak \noindent 
     \fig{1}{./figures/382.png}
     \bigbreak \noindent 
     This candlestick pattern is similar to the hammer (which we will see later). The long shadow (bottom for bullish, top for bearish), is measured using a fibonanci retracement (as seen in the figure above). 
     \bigbreak \noindent 
     Using the candlestick pattern we can identify buying/selling pressure, let's take a look at an example.
     \bigbreak \noindent 
     \fig{.5}{./figures/3822.png}
     \bigbreak \noindent 
     From the figure above, we see a clear uptrend, with a clear level of resistence that seems to want to become support. Looking at the 38.2 candle, we can deduce that buying pressure has overcome selling pressure, indicating this will infact become a level of support and buying pressure wlll continue to push the market forward.
     \bigbreak \noindent 
     Let's consider a different example
     \bigbreak \noindent 
     \fig{.8}{./figures/3823.png}
     \bigbreak \noindent 
     From this we see a 38.2 candle, indicating increased selling pressure, which is potentially indicating a possible reversal.

     \bigbreak \noindent 
     \subsection{Candlestick patterns: Engulfing}
     \bigbreak \noindent 
     \fig{.5}{./figures/eng.jpg}
     \bigbreak \noindent 
    The Engulfing candlestick pattern is a powerful tool in technical analysis, suggesting a potential reversal in the market direction. It is formed by two candles and is considered one of the stronger and more reliable candlestick reversal patterns when confirmed by other signals. Here's an in-depth look at the pattern:

    \bigbreak \noindent 
    \subsubsection{Structure}
    \bigbreak \noindent 
    The Engulfing pattern consists of two candlesticks:
    \begin{itemize}
        \item \textbf{First Candle:} This candle is smaller and is fully 'engulfed' by the next candle. Its color corresponds to the trend that is potentially ending (green/white for an uptrend, red/black for a downtrend).
        \item \textbf{Second Candle:} This candle is larger and completely covers the body of the first candle. Its color is opposite to that of the first candle, indicating the potential for a trend reversal.
    \end{itemize}

    \bigbreak \noindent 
    \subsubsection{Types}
    \begin{itemize}
        \item \textbf{Bullish Engulfing Pattern:}
            \begin{itemize}
                \item \textbf{Occurrence:} This pattern appears during a downtrend.
                \item \textbf{Structure:} The first candle is bearish (red or black), followed by a larger bullish (green or white) candle that completely engulfs the body of the first candle.
                \item \textbf{Interpretation:} It suggests that buying pressure has overwhelmed selling pressure, potentially leading to a reversal of the downtrend.
            \end{itemize}
        \item \textbf{Bearish Engulfing Pattern:}
            \begin{itemize}
                \item \textbf{Occurrence:} This pattern appears during an uptrend.
                \item \textbf{Structure:} The first candle is bullish (green or white), followed by a larger bearish (red or black) candle that completely engulfs the body of the first candle.
                \item \textbf{Interpretation:} It indicates that selling pressure has surpassed buying pressure, potentially leading to a reversal of the uptrend.
            \end{itemize}
    \end{itemize}

    \bigbreak \noindent 
    \subsubsection{Considerations}
    \bigbreak \noindent 
    \begin{itemize}
        \item \textbf{Confirmation:} Engulfing patterns are more reliable when confirmed by other technical indicators such as RSI, MACD, or volume analysis. Increased volume on the engulfing day provides further validation of the pattern.
        \item \textbf{Context:} The pattern should occur at a potential support or resistance level or after a significant move in one direction to be considered a strong signal.
        \item \textbf{Follow-through:} Traders often wait for additional confirmation following the pattern, such as a break of a trendline or more bullish or bearish candles, before entering a trade.
        \item \textbf{Stop-Loss Placement:} Typically, a stop-loss is placed just beyond the engulfing candle to protect against the possibility that the potential reversal does not materialize.
    \end{itemize}

    \bigbreak \noindent 
    \subsubsection{Example}
    \bigbreak \noindent 
    \fig{.5}{./figures/engulf2.png}



    \pagebreak 
     \subsection{Candlestick patterns: Close above / Close below}
     \bigbreak \noindent 
     \fig{.6}{./figures/cacb.png}
     \bigbreak \noindent 
     \bigbreak \noindent 
     \subsubsection{Close above}
     \bigbreak \noindent 
     \begin{itemize}
         \item \textbf{Definition:} This occurs when the closing price of a candle is higher than a specific level, which could be the high of the previous candle, a moving average, or a specific resistance level.
         \item \textbf{Significance:} It indicates bullish sentiment, suggesting that buyers are strong enough to push prices higher. A close above a specific level often leads to traders interpreting this as a signal to enter long positions or add to existing ones.
     \end{itemize}
     \bigbreak \noindent 
     \subsubsection{Close below}
     \bigbreak \noindent 
     \begin{itemize}
         \item \textbf{Definition:} This happens when the closing price of a candle is lower than a specific level, such as the low of the previous candle, a moving average, or a particular support level.
         \item \textbf{Significance:} It suggests bearish sentiment, meaning sellers are strong enough to push prices down. Traders often see a close below a critical level as a cue to enter short positions or exit long ones.
     \end{itemize}
     
     \pagebreak 
     \unsect{Selecting time frames}
     \bigbreak \noindent 
     Selecting the right chart timeframe is crucial for effective trading and technical analysis. The timeframe you choose depends on your trading style, strategy, and the specific market conditions. Here’s how to select chart timeframes and what each can imply:
     \bigbreak \noindent 
     Chart timeframes can range from ultra-short to long-term, each serving different trading styles:
     \begin{itemize}
         \item \textbf{Intraday Timeframes:} These include 1-minute, 5-minute, 15-minute, 30-minute, and 1-hour charts. They are used by day traders who make multiple trades within a single day, aiming to capture quick, short-term movements.
         \item \textbf{Daily Charts:} Each candlestick represents one day of trading. This timeframe is favored by swing traders and those who hold positions for several days to a few weeks.
         \item \textbf{Weekly Charts:} Each candlestick represents one week of trading. These are used by position traders or long-term investors who hold trades for weeks to months.
         \item \textbf{Monthly Charts:} Each candlestick on a monthly chart represents a month of price data, suitable for long-term investing and historical trend analysis.
     \end{itemize}

     \bigbreak \noindent 
     \subsection{Match Timeframe to Trading Strategy}
     \bigbreak \noindent 
     Your trading strategy should dictate the timeframe of the chart you use:
     \begin{itemize}
         \item \textbf{Scalping:} Utilizes the shortest timeframes, such as 1-minute or 5-minute charts, to make rapid trades.
         \item \textbf{Day Trading:} Often uses 15-minute to 1-hour charts to find trading opportunities within the same trading day.
         \item \textbf{Swing Trading:} Typically relies on daily charts to identify trends that play out over several days to weeks.
         \item \textbf{Position Trading:} Uses weekly or monthly charts to capture major market moves over a longer period.
     \end{itemize}

     \bigbreak \noindent 
     \subsection{Using multiple timeframes}
     \bigbreak \noindent 
     During periods of high volatility, shorter timeframes may show a lot of noise, making it difficult to discern a clear trend. Longer timeframes can help smooth out these fluctuations and provide a clearer picture of the underlying trend.
     \bigbreak \noindent 
     Many successful traders use multiple timeframes to gain a more comprehensive view of the market:
     \begin{itemize}
         \item \textbf{Top-Down Analysis:} Start with a longer timeframe to understand the broader market trend and then drill down to shorter timeframes to fine-tune entry and exit points.
         \item \textbf{Confirmation:} Use a shorter timeframe to get entry and exit signals and a longer timeframe to confirm the overall trend.
     \end{itemize}

     \pagebreak 
     \subsection{Chart Patterns}
     \bigbreak \noindent 
     Chart patterns are graphical representations of historical price movements that traders and analysts use in technical analysis to predict future price movements. They are formed by the aggregation of price action over time and provide visual clues about market psychology and potential future price direction. Here’s an overview:

     \bigbreak \noindent 
     \subsection{Chart patterns: Double bottom / double top}
     \bigbreak \noindent 
     Double bottom and double top are reversal chart patterns that signal a potential change in the trend direction. These patterns typically form over an extended period and are considered reliable indicators when confirmed with a breakout.
     \bigbreak \noindent 
     \subsubsection{Double bottom}
     \bigbreak \noindent 
     A double bottom pattern signals a potential bullish reversal from a downward trend. It consists of the following features:
     \bigbreak \noindent 
     \begin{enumerate}
         \item \textbf{Two Bottoms:} The pattern has two distinct troughs at roughly the same price level, with a peak (or recovery) between them.
         \item \textbf{Support Level:} The lows of the two troughs form a support level, which price fails to breach.
         \item \textbf{Neckline:} The peak between the two bottoms serves as the neckline or resistance level.
         \item \textbf{Breakout:} The pattern is confirmed when the price breaks above the neckline with increased volume, indicating that buyers are taking control.
         \item \textbf{Price Target:} The height of the pattern, from the support level to the neckline, is often used to estimate the potential price move.
     \end{enumerate}
     \bigbreak \noindent 
     \fig{.3}{./figures/db.jpg}
     \bigbreak \noindent 
     \fig{.5}{./figures/mane.png}

     \bigbreak \noindent 
     \subsubsection{Double top}
     \bigbreak \noindent 
     A double top pattern indicates a potential bearish reversal from an upward trend. Its key features are:
     \bigbreak \noindent 
     \begin{enumerate}
         \item \textbf{Two Tops:} The pattern includes two prominent peaks at approximately the same price level, with a trough between them.
         \item \textbf{Resistance Level:} The highs of the two peaks create a resistance level, where the price fails to break through.
         \item \textbf{Neckline:} The trough between the peaks serves as the neckline or support level.
         \item \textbf{Breakout:} The pattern is confirmed when the price breaks below the neckline with increased volume, signaling that sellers are gaining strength.
         \item \textbf{Price Target:} The distance from the neckline to the peaks is used to predict the potential downward price movement.
     \end{enumerate}
     \bigbreak \noindent 
     \fig{.3}{./figures/dp.jpeg}

     \bigbreak \noindent 
     \subsubsection{Termination zone}
     \bigbreak \noindent 
     When we start to notice a potential double bottom / double top pattern in formation, i.e we see an impulse into a potential neckline with a pullback, we place a rectangle starting at the highest body of the neckline, and ending at the highest close. The body of the peak of the second top must not be above / below this zone.
     \bigbreak \noindent 
     \fig{.5}{./figures/mane2.png}

     \bigbreak \noindent 
     \subsection{Breakout patterns: Flag patterns}
     \bigbreak \noindent 
     Flag patterns are technical analysis chart patterns that indicate a brief consolidation before the previous trend resumes. They usually follow a strong price movement, known as a "flagpole," and form when the price moves sideways in a narrow range. Flag patterns are considered continuation patterns, signaling that the previous trend is likely to continue.
     \bigbreak \noindent 
     \subsubsection{Components}
     \begin{itemize}
         \item \textbf{Flagpole:} A sharp price movement, either up or down, that forms the initial trend leading to the flag pattern.
         \item \textbf{Flag:} A small rectangular formation that represents a brief consolidation. The flag often tilts against the prevailing trend.
         \item \textbf{Breakout:} The pattern is confirmed when the price breaks out of the flag formation in the direction of the prior trend.
     \end{itemize}
     \bigbreak \noindent 
     \subsubsection{Types of Flag Patterns}
     \bigbreak \noindent 
     \begin{itemize}
         \item \textbf{Bullish Flag:} Forms after a sharp upward price move (bullish flagpole). The flag usually slopes downward or moves sideways, indicating a brief consolidation before continuing the upward trend.
         \item \textbf{Bearish Flag:} Occurs after a significant downward price movement (bearish flagpole). The flag tilts upward or moves sideways, suggesting a temporary consolidation before resuming the downtrend.
     \end{itemize}

     \bigbreak \noindent 
     \subsubsection{Characteristics}
     \begin{itemize}
         \item \textbf{Trend Continuation:} The flag pattern suggests that the prevailing trend will continue after the breakout.
         \item \textbf{Volume:} Volume typically decreases during the flag formation, reflecting reduced trading activity during consolidation. It often increases during the breakout, confirming the continuation of the trend.
         \item \textbf{Timeframe:} Flags generally form over a short period, usually from a few days to a few weeks.
     \end{itemize}

     \bigbreak \noindent 
     \subsubsection{Trading with flag patterns}
     \begin{enumerate}
         \item \textbf{Identify the Pattern:} Recognize the flag pattern with a distinct flagpole and consolidation phase.
         \item \textbf{Wait for Confirmation:} The pattern is confirmed when the price breaks out of the flag's upper or lower boundary in the direction of the initial trend.
         \item \textbf{Entry Point:} Enter the trade after the breakout is confirmed.
         \item \textbf{Stop Loss:} Set a stop loss below the flag's lower boundary (bullish flag) or above the flag's upper boundary (bearish flag) to manage risk.
         \item \textbf{Take Profit:} Measure the height of the flagpole to estimate the potential price movement after the breakout and set profit targets accordingly.
     \end{enumerate}


     \bigbreak \noindent 
     \fig{.5}{./figures/flag.png}
     \bigbreak \noindent 
     \fig{.5}{./figures/flag2.png}

     \pagebreak 
     \subsection{Breakout patterns: Acending/Descending Wedge}
     \bigbreak \noindent 
     Ascending and descending wedge patterns are technical analysis chart patterns that indicate a narrowing range of prices. They often signal potential trend reversals, but sometimes can also act as continuation patterns. Let's delve into the specifics of each:
     \bigbreak \noindent 
     \subsubsection{Acending wedge}
     \bigbreak \noindent 
     An ascending wedge, also called a rising wedge, is characterized by converging trendlines that slope upward.
     \begin{enumerate}
         \item \textbf{Shape:} The upper and lower trendlines converge, with the upper trendline ascending at a slower rate than the lower one. This creates a narrowing wedge shape that slopes upward.
         \item \textbf{Volume:} Volume generally declines as the pattern develops, indicating weakening momentum in the prevailing trend.
         \item \textbf{Interpretation:} An ascending wedge often indicates a bearish reversal. It forms during an uptrend, signaling that the trend is weakening and a breakdown may occur. If it forms during a downtrend, it usually acts as a continuation pattern.
         \item \textbf{Breakout:} The pattern is confirmed when the price breaks below the lower trendline, indicating a potential sell-off.
     \end{enumerate}
     \bigbreak \noindent 
     \fig{.4}{./figures/wedge1.png}

     \pagebreak 
     \subsubsection{Descending Wedge}
     \bigbreak \noindent 
     A descending wedge, also known as a falling wedge, has converging trendlines that slope downward.
     \begin{enumerate}
         \item \textbf{Shape:} The upper and lower trendlines converge, with the upper trendline descending at a faster rate than the lower one. This results in a narrowing wedge shape that slopes downward.
         \item \textbf{Volume:} Volume typically decreases as the pattern matures, reflecting declining interest or uncertainty.
         \item \textbf{Interpretation:} A descending wedge usually indicates a bullish reversal. It forms during a downtrend, suggesting that the trend is losing strength and a breakout to the upside might occur. If it forms during an uptrend, it usually suggests continuation.
         \item \textbf{Breakout:} The pattern is confirmed when the price breaks above the upper trendline, indicating potential upward movement.
     \end{enumerate}

     \bigbreak \noindent 
     \fig{.5}{./figures/wedg2.jpg}
     \bigbreak \noindent 
     \subsubsection{Trading Strategies}
     \bigbreak \noindent 
     \begin{enumerate}
         \item \textbf{Identify the Pattern:} Clearly identify the wedge pattern with its converging trendlines and volume characteristics.
         \item \textbf{Wait for Confirmation:} Wait for a breakout in the anticipated direction, confirmed by increased volume.
         \item \textbf{Entry Point:} Enter the trade after the breakout is confirmed.
         \item \textbf{Stop Loss:} Place a stop loss below the breakout point for a descending wedge or above the breakout point for an ascending wedge to manage risk.
         \item \textbf{Take Profit:} Measure the height of the widest part of the wedge to estimate the potential price movement after the breakout and set profit targets accordingly.
     \end{enumerate}

     \pagebreak 
     \subsection{Fibonacci}
     \bigbreak \noindent 
     \subsubsection{The Fibonacci sequence}
     \bigbreak \noindent 
        The Fibonacci sequence and the associated ratios used in technical analysis come from a mathematical sequence introduced to the Western world by Leonardo of Pisa, commonly known as Fibonacci, in his 1202 book "Liber Abaci."
        \bigbreak \noindent 
        The Fibonacci sequence is a series of numbers in which each number is the sum of the two preceding ones, starting from 0 and 1. The sequence looks like this: 0, 1, 1, 2, 3, 5, 8, 13, 21, and so on.
        \bigbreak \noindent 
        This sequence can therefore be defined recursively as
        \begin{align*}
            f(n) = f(n-2) + f(n-1)
        .\end{align*}

        \bigbreak \noindent 
        \subsubsection{Fibonacci ratios}
        \bigbreak \noindent 
        The ratios used in technical analysis are derived from the relationships between numbers in the Fibonacci sequence:
        \begin{itemize}
            \item \textbf{23.6\%:} Derived by dividing a number in the series by the number three places to its right (e.g., 8 divided by 34 equals approximately 0.236 or 23.6\%).
            \item \textbf{38.2\%:} Derived by dividing a number by the number two places to its right (e.g., 21 divided by 55 equals approximately 0.382 or 38.2\%).
            \item \textbf{50\%:} This is not an actual Fibonacci ratio, but it's often used due to its significance in Dow Theory, which suggests that prices often retrace about half of their previous movement.
            \item \textbf{61.8\%:} Derived by dividing a number by the number immediately to its right (e.g., 34 divided by 55 equals approximately 0.618 or 61.8\%).
            \item \textbf{100\%:} The ratio of a number to itself.
        \end{itemize}
        \bigbreak \noindent 
        \subsubsection{The Golden Ratio}
        \bigbreak \noindent 
        The most significant ratio in Fibonacci analysis is 61.8\%, known as the "golden ratio." This is because as the Fibonacci sequence progresses, the ratio between successive numbers approaches 0.618. This ratio is found in various natural phenomena, like the branching of trees, the arrangement of leaves, and the spirals of shells. It's also believed to have psychological significance, hence its use in technical analysis.
        \bigbreak \noindent 
        In trading, these ratios are used because many traders believe they represent natural levels where a market might reverse or find significant support/resistance.

     \bigbreak \noindent 
     \subsubsection{Fib retractements}
     \bigbreak \noindent 
     The tool is based on the Fibonacci sequence and the mathematical relationships derived from it, specifically the ratios of 23.6\%, 38.2\%, 50\%, 61.8\%, and 100\%. Here's how it works:
     \bigbreak \noindent 
     \begin{enumerate}
         \item \textbf{Identify a Swing High and Swing Low:} The first step is to find a significant peak (swing high) and trough (swing low) in a market. The tool is then drawn from the swing high to the swing low, or vice versa, depending on the direction of the trend.
         \item \textbf{Draw the Retracement Levels:} Once the swing high and low points are identified, horizontal lines are drawn at each Fibonacci level between these two points. For instance, if the price of a stock went from \$100 (swing high) to \$50 (swing low), a 61.8\% retracement would be around \$80.
         \item \textbf{Interpretation:} Traders use Fibonacci retracement levels to identify potential reversal points in the market. For example:
             \begin{itemize}
                 \item In an uptrend, retracement levels act as potential support levels where the price may bounce back up.
                 \item In a downtrend, retracement levels act as potential resistance levels where the price may face selling pressure.
             \end{itemize}
         \item \textbf{Usage in Trading Strategies:}
             \begin{itemize}
                 \item \textbf{Entry and Exit Points:} Traders often use Fibonacci levels to decide entry and exit points. For example, they might enter a trade when the price retraces to a key level and shows a reversal.
                 \item \textbf{Stop Loss and Take Profit:} These levels can also help in placing stop-loss orders to limit risk and take-profit orders to lock in gains.
                 \item \textbf{Confluence:} Fibonacci retracement levels are often used in conjunction with other technical analysis tools to confirm trade signals.
             \end{itemize}
     \end{enumerate} 
     \bigbreak \noindent 
     \subsubsection{Selecting point of significance}
     \bigbreak \noindent 
     Suppose we spot a trend reversal. For example, the market was making lower highs and lower lows, but we see that the next low fails to surpase the previous. 
     \bigbreak \noindent 
     We then draw the fibonacci retracement from the previous low to the point where the market failed to make the new low.
     \bigbreak \noindent 
     \fig{.3}{./figures/fib.jpg}

     \bigbreak \noindent 
     \subsubsection{Entering trades}
     \bigbreak \noindent 
     The 0.5 - 0.618 zone is called the \textbf{fibonanci gold zone}, this is where we enter trades.
     \bigbreak \noindent 
     \nt{As always, we need a combination of factors/confirmations to safely enter trades, we never just use one indicator}
     \bigbreak \noindent 
     \fig{.5}{./figures/fib2.png}
     \bigbreak \noindent 
















     



     




     






    
\end{document}
