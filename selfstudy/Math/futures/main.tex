\documentclass{report}

\input{~/dev/latex/template/preamble.tex}
\input{~/dev/latex/template/macros.tex}

\title{\Huge{}}
\author{\huge{Nathan Warner}}
\date{\huge{}}
\fancyhf{}
\rhead{}
\fancyhead[R]{\itshape Warner} % Left header: Section name
\fancyhead[L]{\itshape\leftmark}  % Right header: Page number
\cfoot{\thepage}
\renewcommand{\headrulewidth}{0pt} % Optional: Removes the header line
%\pagestyle{fancy}
%\fancyhf{}
%\lhead{Warner \thepage}
%\rhead{}
% \lhead{\leftmark}
%\cfoot{\thepage}
%\setborder
% \usepackage[default]{sourcecodepro}
% \usepackage[T1]{fontenc}

% Change the title
\hypersetup{
    pdftitle={Futures}
}

\begin{document}
    % \maketitle
        \begin{titlepage}
       \begin{center}
           \vspace*{1cm}
    
           \textbf{Futures}
    
           \vspace{0.5cm}
            
                
           \vspace{1.5cm}
    
           \textbf{Nathan Warner}
    
           \vfill
                
                
           \vspace{0.8cm}
         
           \includegraphics[width=0.4\textwidth]{~/niu/seal.png}
                
           Computer Science \\
           Northern Illinois University\\
           United States\\
           
                
       \end{center}
    \end{titlepage}
    \tableofcontents
    \pagebreak 
    \unsect{What is futures trading?}
    \bigbreak \noindent 
    \begin{concept}
        Futures trading involves a contractual agreement to buy or sell a particular quantity of a commodity, financial instrument, or other tradable item at a predetermined price at a specified time in the future. These contracts are standardized and traded on futures exchanges.
        \bigbreak \noindent 
        In simple terms, imagine you're a farmer who grows wheat, and you want to lock in a price for your wheat that you'll harvest in a few months. You can enter a futures contract with a buyer who agrees to purchase your wheat at a set price when it's harvested. This way, you're protected if prices fall before you harvest, but you also agree to sell at this price even if market prices rise. Similarly, a bakery that needs wheat might use futures to lock in a price for wheat they'll buy in the future, protecting them against price rises but committing to the agreed price even if the market price falls. This mechanism allows both producers and consumers to hedge against price volatility.
    \end{concept}

    \bigbreak \noindent 
    \unsect{Contract sizes}
    
\end{document}
