\documentclass{report}

\input{~/dev/latex/template/preamble.tex}
\input{~/dev/latex/template/macros.tex}

\title{\Huge{}}
\author{\huge{Nathan Warner}}
\date{\huge{}}
\fancyhf{}
\rhead{}
\fancyhead[R]{\itshape Warner} % Left header: Section name
\fancyhead[L]{\itshape\leftmark}  % Right header: Page number
\cfoot{\thepage}
\renewcommand{\headrulewidth}{0pt} % Optional: Removes the header line
\pagestyle{fancy}
\fancyhf{}
\lhead{Warner}
\rhead{}
% \lhead{\leftmark}
%\cfoot{\thepage}
%\setborder
% \usepackage[default]{sourcecodepro}
% \usepackage[T1]{fontenc}

% Change the title
\hypersetup{
    pdftitle={Forex}
}

\begin{document}
    % \maketitle
        \begin{titlepage}
       \begin{center}
           \vspace*{1cm}
    
           \textbf{Forex}
    
           \vspace{0.5cm}
            
                
           \vspace{1.5cm}
    
           \textbf{Nathan Warner}
    
           \vfill
                
                
           \vspace{0.8cm}
         
           \includegraphics[width=0.4\textwidth]{~/niu/seal.png}
                
           Computer Science \\
           Northern Illinois University\\
           United States\\
           
                
       \end{center}
    \end{titlepage}
    \tableofcontents
    \pagebreak 
    \unsect{Preface}
    \bigbreak \noindent 
    Forex, short for foreign exchange, refers to the global marketplace where currencies are traded.
    \bigbreak \noindent 
    Forex trading involves buying one currency while simultaneously selling another, with the aim of profiting from changes in their exchange rates.
    \bigbreak \noindent 
    \subsection{The Main Idea}
    \bigbreak \noindent 
    Currencies are traded in pairs, with each pair consisting of a base currency and a quote currency. For example, in the EUR/USD pair, EUR is the base currency, and USD is the quote currency. If you believe the euro will increase in value against the dollar, you would buy EUR/USD. If you think the euro will decrease in value against the dollar, you would sell EUR/USD.

    \bigbreak \noindent 
    \subsection{Example}
    \bigbreak \noindent 
    Suppose we choose to trade EUR/AUD. Assume the current price of EUR/AUD is 1.6000. This means 1 euro is equivalent to 1.60 Australian dollars.
    \bigbreak \noindent 
    you decide to buy EUR/AUD, expecting the euro to appreciate against the Australian dollar. In forex trading, buying a currency pair means you're buying the base currency (EUR) and selling the equivalent in the quote currency (AUD).
    \bigbreak \noindent 
    Say then we decide to trade 10,000 units of EUR/AUD. The value of your trade in Australian dollars is 16,000 AUD, given by
    \begin{align*}
        &10,000 \cdot 1.6000 \\
        &=16,000
    .\end{align*}
    \bigbreak \noindent 
    After you enter the trade, let's say the market moves in your favor, and the price of EUR/AUD increases to 1.6100. This price movement indicates that the euro has appreciated relative to the Australian dollar.
    \bigbreak \noindent 
    Satisfied with the profit, you decide to close your position by selling EUR/AUD at 1.6100.
    The difference between the price at which you bought EUR/AUD and the price at which you sold it is 0.0100 (1.6100 - 1.6000). Since you traded 10,000 units, your profit is 100 AUD (0.0100 * 10,000).

    \bigbreak \noindent 
    \subsection{The trade}
    \bigbreak \noindent 
     When you trade in the forex market, you're simultaneously buying one currency and selling another
     \bigbreak \noindent 
     \subsubsection{Buying a currency pair}
     \bigbreak \noindent 
     When you "buy" a currency pair, you are buying the base currency and selling an equivalent amount of the quote currency.
     \bigbreak \noindent 
     If you buy EUR/USD, you are buying euros and selling an equivalent amount of US dollars. You're essentially betting that the euro will strengthen against the dollar. If the euro's value rises relative to the dollar, you can sell the pair at a higher price than you bought it for, thus making a profit.
     \bigbreak \noindent 
     \subsubsection{Selling a Currency Pair}
     \bigbreak \noindent 
      when you "sell" a currency pair, you are selling the base currency and buying the equivalent amount of the quote currency.
      \bigbreak \noindent 
       If you sell EUR/USD, you are selling euros and buying US dollars. This position would be beneficial if the euro weakens against the dollar. If the euro's value decreases relative to the dollar, you can buy the pair back at a lower price than you sold it for, thus making a profit.

       \bigbreak \noindent 
       \subsection{Analogy}
       \bigbreak \noindent 
       Imagine you're in the US and want to buy a piece of furniture from Europe, priced at €1,000. To make this purchase, you need euros because that's the currency the seller accepts. Let's say the current exchange rate is 1 EUR = 1.20 USD (for simplicity, we'll ignore transaction fees). To buy €1,000 worth of furniture, you need to exchange \$1,200 USD into euros.
       \bigbreak \noindent 
       You're effectively "buying" euros because you need them to make your purchase. Simultaneously, you're "selling" your dollars to get the euros.
       \bigbreak \noindent 
       \subsection{Connection to forex trading}
       \bigbreak \noindent 
       When you trade a currency pair, such as EUR/USD, you're dealing with two currencies:
       \begin{itemize}
           \item \textbf{EUR} (Euro) is the Base Currency.
           \item \textbf{USD} (US Dollar) is the Quote Currency.
       \end{itemize}
       \bigbreak \noindent 
        When you "buy" EUR/USD, you're speculating that the euro will increase in value relative to the US dollar.
        \bigbreak \noindent 
     You're not physically handling currencies, but through your trading platform, you're entering into a contract that reflects this exchange. If the euro strengthens against the dollar (meaning it takes more dollars to buy the same amount of euros), your trade becomes more valuable.
     \bigbreak \noindent 
     If you decide to close your trade after the euro has appreciated, you're effectively selling euros and buying back dollars. If the exchange rate has moved in your favor, you get more dollars back than you initially sold, making a profit.


     \pagebreak 
     \unsect{Lot Sizes}
     \bigbreak \noindent 
     Here is where we discuss lot sizes in greater detail.
     \bigbreak \noindent 
     Forex is traded in specific amounts called lots. The standard lot sizes (as seen previously), are
     \begin{itemize}
         \item \textbf{Standard Lot (1.00):} 100,000 units of the base currency.
         \item \textbf{Mini Lot (0.1):} 10,000 units of the base currency.
         \item \textbf{Micro Lot (0.01):} 1,000 units of the base currency.
         \item \textbf{Nano Lot (0.001):} 100 units of the base currency (not offered by all brokers).
     \end{itemize}
     \bigbreak \noindent 
     \subsection{Shorthand Notation}
     \bigbreak \noindent 
     When traders refer to lot sizes in forex trading as 1.00 for a standard lot, 0.10 for a mini lot, and 0.01 for a micro lot, they are using a shorthand notation that represents the number of base currency units in each type of lot. This notation is commonly used in trading platforms to simplify the process of selecting the size of the trade you want to execute. 
     \bigbreak \noindent 
     \subsubsection{Standard Lot}
     \begin{itemize}
         \item \textbf{Notation:} 1.00
         \item \textbf{Units:} 100,000 units of the base currency
         \item \textbf{Explanation:} A standard lot size in forex trading represents 100,000 units of the base currency in a currency pair. If you're trading EUR/USD, for example, a 1.00 lot size would mean you are trading 100,000 euros.
     \end{itemize}
     \bigbreak \noindent 
     \subsubsection{Mini Lot}
     \bigbreak \noindent 
     \begin{itemize}
         \item \textbf{Notation:} 0.10
         \item \textbf{Units:} 10,000 units of the base currency
         \item \textbf{Explanation:} A mini lot is one-tenth the size of a standard lot and represents 10,000 units of the base currency. Using the same EUR/USD example, a 0.10 lot size trade would involve 10,000 euros.
     \end{itemize}
     \bigbreak \noindent 
     \subsubsection{Micro Lot}
     \bigbreak \noindent 
     \begin{itemize}
         \item \textbf{Notation:} 0.01
         \item \textbf{Units:} 1,000 units of the base currency
         \item \textbf{Explanation:} A micro lot is one-hundredth of a standard lot and represents 1,000 units of the base currency. In the context of EUR/USD, a 0.01 lot size corresponds to trading 1,000 euros.
     \end{itemize}
     \bigbreak \noindent 
     \subsubsection{Idea behind this notation}
     \bigbreak \noindent 
     So as you might notice we define a standard lot to be 1.00, which has 100,000 units. Then when we look at a minilot (10,000) units, we can define this to be 0.1, because this lot has 10\% the number of units as the standard lot.

     \pagebreak 
     \unsect{Pips (Percentage in Point)}
     \bigbreak \noindent 
     A pip is a standard unit of measurement for the change in value between two currencies in the forex market. Understanding pips is crucial for traders to manage trades, calculate profits or losses, and assess trading costs
     \bigbreak \noindent 
     \subsection{Formal Definition}
     \bigbreak \noindent 
     More formally, a pip Represents the smallest price move that a currency pair can make. For most currency pairs, a pip is equivalent to a one-digit movement in the fourth decimal place of the quoted price. For example, if the EUR/USD pair moves from 1.1050 to 1.1051, that's a one pip change.
     \bigbreak \noindent 
    \subsection{Exceptions}
    \bigbreak \noindent 
     For currency pairs involving the Japanese yen (JPY), a pip is typically associated with the second decimal place because the yen is much closer in value to one hundredth of other major currencies. For instance, if USD/JPY changes from 110.01 to 110.02, that is a one pip movement.
     \bigbreak \noindent 
     \subsection{Pipettes}
     \bigbreak \noindent 
     Some brokers quote currency pairs beyond the standard 4 and 2 decimal places to 5 and 3 decimal places. They call these smaller increments "pipettes" or "fractional pips." A pipette is thus one-tenth of a pip. For example, if EUR/USD moves from 1.10501 to 1.10502, that 0.00001 change is one pipette.
     \bigbreak \noindent 
     \subsection{Why care about pips}
     \bigbreak \noindent 
     Traders use pips to calculate profits and losses. By knowing the number of pips they have gained or lost on a trade and the value of each pip, traders can calculate the financial outcome of their trades.
     \bigbreak \noindent 
     The monetary value of a pip can vary depending on the size of your trade (lot size) and the currency pair you are trading. In a standard lot of 100,000 units of currency, a pip usually equals \$10 in the base currency for pairs where the USD is the quote currency. For a mini lot (10,000 units of currency), a pip is typically worth \$1, and for a micro lot (1,000 units of currency), a pip is usually worth \$0.10.
     \bigbreak \noindent 
     \subsection{Calculating Pip Value}
     \bigbreak \noindent 
     First of all, what does it mean exactly to "calculate a pip value". Simply put, to "calculate pip" value in forex trading means to determine the value of a one-pip movement in a currency pair in terms of a specific currency. This calculation is crucial for managing risk, setting stop-loss and take-profit levels, and understanding the potential profit or loss from a trade.
     \bigbreak \noindent 
     The value of a pip varies based on the currency pair being traded, the size of the trade, and the exchange rate. For pairs where the USD is the quote currency, calculating the pip value is straightforward:
     \begin{itemize}
         \item \textbf{Standard Lot:} If you're trading a standard lot (100,000 units), a pip is usually worth \$10.
         \item \textbf{Mini Lot:} For a mini lot (10,000 units), a pip is worth \$1.
         \item \textbf{Micro Lot:} For a micro lot (1,000 units), a pip is worth \$0.10.
     \end{itemize}

     \bigbreak \noindent 
     Because the pip value varies depending on the currency pair you are trading and the currency your account is denominated in, it is best to use online calculators. However, we will briefly discuss the concept of calculating a pip value.
     \bigbreak \noindent 
     \subsubsection{When USD is the base currency}
     \bigbreak \noindent 
     \begin{itemize}
         \item Pip Value = (Pip in decimal places $\times$ Trade Size) / Exchange Rate
         \item Example for a standard lot of EUR/USD, where the trade size is 100,000 units and the pip size is 0.0001:
             \begin{align*}
                 \text{Pip Value } = (0.0001 \times 100,000) / 1 = \$10
             .\end{align*}
     \end{itemize}
     \bigbreak \noindent 
     \subsection{When USD is Not the Quote Currency or You Want the Pip Value in Another Currency:}
     \begin{itemize}
         \item First, calculate the pip value in the quote currency, then convert it to USD (or another currency) using an appropriate exchange rate.
            \item Example for a standard lot of USD/JPY, where the trade size is 100,000 units and the pip size is 0.01:
        \end{itemize}
            \item Pip Value in JPY = 0.01 $\times$ 100,000 = 1,000 JPY
            \item To convert to USD, divide by the USD/JPY exchange rate (assuming it's 110.00):
            \item Pip Value in USD = 1,000 / 110.00 = \$9.09
        \begin{itemize}
     \end{itemize}

     \pagebreak 
     \unsect{Leverages}
     \bigbreak \noindent 
     Leverage in forex trading is a tool that allows traders to control a large position with a relatively small amount of capital. It's expressed as a ratio, such as 50:1, 100:1, or even 500:1. This ratio determines the amount of money you can control in the market for every dollar in your trading account. Leverage amplifies both potential profits and losses, making it a powerful but risky tool.
     \bigbreak \noindent 
     \subsection{How Leverage Works}
     The basic concept is simple. If you have \$1,000 in your trading account and use 100:1 leverage, you can control a position worth \$100,000 (\$1,000 times 100). This means you can open larger positions than your actual capital would otherwise allow.
     \bigbreak \noindent 
     When you open a leveraged position, you're essentially borrowing money from your broker to increase the size of your trade. Despite the larger position, the amount required as a deposit is relatively small, known as the "margin."
     \bigbreak \noindent 
     The margin is the amount of capital required to open and maintain a leveraged position. It's a fraction of the full value of your trade. For example, with 100:1 leverage, the margin requirement might be 1\% of the total value of the position. So, to control a \$100,000 position, you need to have \$1,000 in your account as margin.
     \bigbreak \noindent 
     Leverage increases the potential for profit and loss by the same magnitude. If the market moves in your favor, you can achieve significant profits on the capital you've actually invested. However, if the market moves against you, you can quickly incur substantial losses, potentially exceeding your initial investment.

     \bigbreak \noindent 
     \subsubsection{Example}
     \bigbreak \noindent 
     Let's say you want to trade the EUR/USD currency pair, and you have \$1,000 in your trading account. You decide to use 100:1 leverage, allowing you to control a \$100,000 position in the market.
     \begin{itemize}
         \item If the Market Moves in Your Favor: Suppose the EUR/USD price moves up by 1\% after you open your position. Without leverage, a 1\% increase on a \$1,000 investment would have yielded a \$10 profit. However, because you're controlling a \$100,000 position with your \$1,000 investment, that same 1\% move translates to a \$1,000 profit (1\% of \$100,000).
         \item If the Market Moves Against You: Conversely, if the EUR/USD price moves down by 1\%, you would lose \$1,000, which is the entire amount of your initial investment. In a highly leveraged position, even small market movements can have significant impacts on your account balance.
     \end{itemize}

     \bigbreak \noindent 
     \subsection{Risks of Using Leverage}
     \begin{itemize}
         \item \textbf{Margin Calls:} If your account balance falls below the required margin due to trading losses, your broker may issue a margin call, requiring you to deposit additional funds to maintain your open positions. If you fail to meet the margin call, the broker might close your positions to limit further losses.
         \item \textbf{Rapid Losses:} High leverage can lead to rapid losses, especially in volatile markets. It's possible to lose more than your initial investment in a short period.
     \end{itemize}

     \bigbreak \noindent 
     \subsection{Best Practices}
     \begin{itemize}
         \item \textbf{Risk Management:} Use stop-loss orders to limit potential losses. Determine in advance the maximum amount you are willing to risk on a trade.
         \item \textbf{Leverage Ratio:} Choose a leverage ratio that matches your risk tolerance. Less experienced traders should consider using lower leverage to reduce risk.
     \end{itemize}

     \pagebreak 
     \unsect{Stop/Loss}

     \pagebreak 
     \unsect{Risk management}






     \pagebreak 
     \unsect{Appendix A: Terminology}
     \bigbreak \noindent 
     Forex trading comes with its own set of jargon and terminology that traders need to understand to navigate the markets effectively. Comprised in this section is a list of terms to be familiar with.
     \bigbreak \noindent 
     \textbf{Note:} Some of these terms are discussed in more detail in above sections
     \bigbreak \noindent 
     \begin{itemize}
         \item \textbf{Liquid market}: A market allowing the buying or selling of large quantities of an asset at any time and at low transactions costs
         \item \textbf{Illiquid market}: The converse of a liquid market. These markets may have considerably large spreads between the highest available buyer and the lowest available seller.
         \item \textbf{Thin market}: Term used to describe an illiquid market
         \item \textbf{Base currency}: The first currency in a currency pair, which is used as the reference for buying or selling.
         \item \textbf{Quote currency}: The second currency in a currency pair, which shows how much of this currency is needed to buy one unit of the base currency.
         \item \textbf{Spread}: The difference between the bid (sell) and ask (buy) price of a currency pair. Tighter spreads generally mean lower trading costs.
         \item \textbf{Bull market}: A market condition in which the prices of securities are rising, encouraging buying.
         \item \textbf{Bear market}: A market condition in which the prices of securities are falling, encouraging selling.
         \item \textbf{Margin}: The amount of capital required in your account to open and maintain a position. Margin is usually expressed as a percentage of the full position. For example, if a broker requires 1\% margin to trade a standard lot of EUR/USD, you would need \$1,000 in your account to open a \$100,000 position.
         \item \textbf{Leverage}: A loan provided by the broker to the trader, allowing the trader to control a large position with a relatively small amount of invested capital. Leverage is expressed as a ratio, such as 50:1, 100:1, or 500:1. While it can amplify profits, it also increases the risk of significant losses.
         \item \textbf{Lot size:} The number of currency units you are buying or selling in a trade. Standard lot sizes include:
             \begin{itemize}
                 \item \textbf{Standard Lot:} 100,000 units of the base currency.
                 \item \textbf{Mini Lot:} 10,000 units of the base currency.
                 \item \textbf{Micro Lot:} 1,000 units of the base currency.
                 \item \textbf{Nano Lot:} 100 units of the base currency.
             \end{itemize}
            \item \textbf{Pip (Percentage in Point):}
                The smallest price move that a given exchange rate can make based on market convention. Most major currency pairs are priced to four decimal places, and a pip is one unit of the fourth decimal point, for most pairs. For example, if EUR/USD moves from 1.1050 to 1.1051, that 0.0001 USD rise in value is one pip.
            \item \textbf{Volatile market}: Volatility is an investment term that describes when a market or security experiences periods of unpredictable, and sometimes sharp, price movements.
        \item \textbf{Rollover}: A rollover is the process of keeping a position open beyond its expiry
        \item \textbf{Going long}: If you buy a currency pair expecting it to increase in value, you are “GOING LONG.”
        \item \textbf{Shorting}: If you sell a currency pair expecting it to decrease in value, you are “SHORTING.”
        \item \textbf{Swaps}: A swap in forex refers to the interest that you either earn or pay for a trade that you keep open overnight.
     \end{itemize}


      \pagebreak 
     \unsect{Appendix B: Formulas}
     \begin{itemize}
         \item \textbf{Pip calculation} 
             \begin{align*}
                  \item Pip Value = (Pip in decimal places $\times$ Trade Size) / Exchange Rate
             .\end{align*}
     \end{itemize}

     \pagebreak 
     \unsect{Appendix C: Chart patterns}






    
\end{document}
