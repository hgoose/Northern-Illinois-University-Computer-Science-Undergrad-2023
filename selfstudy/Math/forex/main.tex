\documentclass{report}

\input{~/dev/latex/template/preamble.tex}
\input{~/dev/latex/template/macros.tex}

\title{\Huge{}}
\author{\huge{Nathan Warner}}
\date{\huge{}}
\fancyhf{}
\rhead{}
\fancyhead[R]{\itshape Warner} % Left header: Section name
\fancyhead[L]{\itshape\leftmark}  % Right header: Page number
\cfoot{\thepage}
\renewcommand{\headrulewidth}{0pt} % Optional: Removes the header line
\pagestyle{fancy}
\fancyhf{}
\lhead{Warner}
\rhead{}
% \lhead{\leftmark}
%\cfoot{\thepage}
%\setborder
% \usepackage[default]{sourcecodepro}
% \usepackage[T1]{fontenc}

% Change the title
\hypersetup{
    pdftitle={Forex}
}

\begin{document}
    % \maketitle
        \begin{titlepage}
       \begin{center}
           \vspace*{1cm}
    
           \textbf{Forex}
    
           \vspace{0.5cm}
            
                
           \vspace{1.5cm}
    
           \textbf{Nathan Warner}
    
           \vfill
                
                
           \vspace{0.8cm}
         
           \includegraphics[width=0.4\textwidth]{~/niu/seal.png}
                
           Computer Science \\
           Northern Illinois University\\
           United States\\
           
                
       \end{center}
    \end{titlepage}
    \tableofcontents
    \pagebreak 
    \unsect{Preface}
    \bigbreak \noindent 
    Forex, short for foreign exchange, refers to the global marketplace where currencies are traded.
    \bigbreak \noindent 
    Forex trading involves buying one currency while simultaneously selling another, with the aim of profiting from changes in their exchange rates.
    \bigbreak \noindent 
    \subsection{The Main Idea}
    \bigbreak \noindent 
    Currencies are traded in pairs, with each pair consisting of a base currency and a quote currency. For example, in the EUR/USD pair, EUR is the base currency, and USD is the quote currency. If you believe the euro will increase in value against the dollar, you would buy EUR/USD. If you think the euro will decrease in value against the dollar, you would sell EUR/USD.

    \bigbreak \noindent 
    \subsection{Example}
    \bigbreak \noindent 
    Suppose we choose to trade EUR/AUD. Assume the current price of EUR/AUD is 1.6000. This means 1 euro is equivalent to 1.60 Australian dollars.
    \bigbreak \noindent 
    you decide to buy EUR/AUD, expecting the euro to appreciate against the Australian dollar. In forex trading, buying a currency pair means you're buying the base currency (EUR) and selling the equivalent in the quote currency (AUD).
    \bigbreak \noindent 
    Say then we decide to trade 10,000 units of EUR/AUD. The value of your trade in Australian dollars is 16,000 AUD, given by
    \begin{align*}
        &10,000 \cdot 1.6000 \\
        &=16,000
    .\end{align*}
    \bigbreak \noindent 
    After you enter the trade, let's say the market moves in your favor, and the price of EUR/AUD increases to 1.6100. This price movement indicates that the euro has appreciated relative to the Australian dollar.
    \bigbreak \noindent 
    Satisfied with the profit, you decide to close your position by selling EUR/AUD at 1.6100.
    The difference between the price at which you bought EUR/AUD and the price at which you sold it is 0.0100 (1.6100 - 1.6000). Since you traded 10,000 units, your profit is 100 AUD (0.0100 * 10,000).

    \bigbreak \noindent 
    \subsection{The trade}
    \bigbreak \noindent 
     When you trade in the forex market, you're simultaneously buying one currency and selling another
     \bigbreak \noindent 
     \subsubsection{Buying a currency pair}
     \bigbreak \noindent 
     When you "buy" a currency pair, you are buying the base currency and selling an equivalent amount of the quote currency.
     \bigbreak \noindent 
     If you buy EUR/USD, you are buying euros and selling an equivalent amount of US dollars. You're essentially betting that the euro will strengthen against the dollar. If the euro's value rises relative to the dollar, you can sell the pair at a higher price than you bought it for, thus making a profit.
     \bigbreak \noindent 
     \subsubsection{Selling a Currency Pair}
     \bigbreak \noindent 
      when you "sell" a currency pair, you are selling the base currency and buying the equivalent amount of the quote currency.
      \bigbreak \noindent 
       If you sell EUR/USD, you are selling euros and buying US dollars. This position would be beneficial if the euro weakens against the dollar. If the euro's value decreases relative to the dollar, you can buy the pair back at a lower price than you sold it for, thus making a profit.

       \bigbreak \noindent 
       \subsection{Analogy}
       \bigbreak \noindent 
       Imagine you're in the US and want to buy a piece of furniture from Europe, priced at €1,000. To make this purchase, you need euros because that's the currency the seller accepts. Let's say the current exchange rate is 1 EUR = 1.20 USD (for simplicity, we'll ignore transaction fees). To buy €1,000 worth of furniture, you need to exchange \$1,200 USD into euros.
       \bigbreak \noindent 
       You're effectively "buying" euros because you need them to make your purchase. Simultaneously, you're "selling" your dollars to get the euros.
       \bigbreak \noindent 
       \subsection{Connection to forex trading}
       \bigbreak \noindent 
       When you trade a currency pair, such as EUR/USD, you're dealing with two currencies:
       \begin{itemize}
           \item \textbf{EUR} (Euro) is the Base Currency.
           \item \textbf{USD} (US Dollar) is the Quote Currency.
       \end{itemize}
       \bigbreak \noindent 
        When you "buy" EUR/USD, you're speculating that the euro will increase in value relative to the US dollar.
        \bigbreak \noindent 
     You're not physically handling currencies, but through your trading platform, you're entering into a contract that reflects this exchange. If the euro strengthens against the dollar (meaning it takes more dollars to buy the same amount of euros), your trade becomes more valuable.
     \bigbreak \noindent 
     If you decide to close your trade after the euro has appreciated, you're effectively selling euros and buying back dollars. If the exchange rate has moved in your favor, you get more dollars back than you initially sold, making a profit.


     \pagebreak 
     \unsect{Lot Sizes}
     \bigbreak \noindent 
     Here is where we discuss lot sizes in greater detail.
     \bigbreak \noindent 
     Forex is traded in specific amounts called lots. The standard lot sizes (as seen previously), are
     \begin{itemize}
         \item \textbf{Standard Lot (1.00):} 100,000 units of the base currency.
         \item \textbf{Mini Lot (0.1):} 10,000 units of the base currency.
         \item \textbf{Micro Lot (0.01):} 1,000 units of the base currency.
         \item \textbf{Nano Lot (0.001):} 100 units of the base currency (not offered by all brokers).
     \end{itemize}
     \bigbreak \noindent 
     \subsection{Shorthand Notation}
     \bigbreak \noindent 
     When traders refer to lot sizes in forex trading as 1.00 for a standard lot, 0.10 for a mini lot, and 0.01 for a micro lot, they are using a shorthand notation that represents the number of base currency units in each type of lot. This notation is commonly used in trading platforms to simplify the process of selecting the size of the trade you want to execute. 
     \bigbreak \noindent 
     \subsubsection{Standard Lot}
     \begin{itemize}
         \item \textbf{Notation:} 1.00
         \item \textbf{Units:} 100,000 units of the base currency
         \item \textbf{Explanation:} A standard lot size in forex trading represents 100,000 units of the base currency in a currency pair. If you're trading EUR/USD, for example, a 1.00 lot size would mean you are trading 100,000 euros.
     \end{itemize}
     \bigbreak \noindent 
     \subsubsection{Mini Lot}
     \bigbreak \noindent 
     \begin{itemize}
         \item \textbf{Notation:} 0.10
         \item \textbf{Units:} 10,000 units of the base currency
         \item \textbf{Explanation:} A mini lot is one-tenth the size of a standard lot and represents 10,000 units of the base currency. Using the same EUR/USD example, a 0.10 lot size trade would involve 10,000 euros.
     \end{itemize}
     \bigbreak \noindent 
     \subsubsection{Micro Lot}
     \bigbreak \noindent 
     \begin{itemize}
         \item \textbf{Notation:} 0.01
         \item \textbf{Units:} 1,000 units of the base currency
         \item \textbf{Explanation:} A micro lot is one-hundredth of a standard lot and represents 1,000 units of the base currency. In the context of EUR/USD, a 0.01 lot size corresponds to trading 1,000 euros.
     \end{itemize}
     \bigbreak \noindent 
     \subsubsection{Idea behind this notation}
     \bigbreak \noindent 
     So as you might notice we define a standard lot to be 1.00, which has 100,000 units. Then when we look at a minilot (10,000) units, we can define this to be 0.1, because this lot has 10\% the number of units as the standard lot.

     \pagebreak 
     \unsect{Pips (Percentage in Point)}
     \bigbreak \noindent 
     A pip is a standard unit of measurement for the change in value between two currencies in the forex market. Understanding pips is crucial for traders to manage trades, calculate profits or losses, and assess trading costs
     \bigbreak \noindent 
     \subsection{Formal Definition}
     \bigbreak \noindent 
     More formally, a pip Represents the smallest price move that a currency pair can make. For most currency pairs, a pip is equivalent to a one-digit movement in the fourth decimal place of the quoted price. For example, if the EUR/USD pair moves from 1.1050 to 1.1051, that's a one pip change.
     \bigbreak \noindent 
    \subsection{Exceptions}
    \bigbreak \noindent 
     For currency pairs involving the Japanese yen (JPY), a pip is typically associated with the second decimal place because the yen is much closer in value to one hundredth of other major currencies. For instance, if USD/JPY changes from 110.01 to 110.02, that is a one pip movement.
     \bigbreak \noindent 
     \subsection{Pipettes}
     \bigbreak \noindent 
     Some brokers quote currency pairs beyond the standard 4 and 2 decimal places to 5 and 3 decimal places. They call these smaller increments "pipettes" or "fractional pips." A pipette is thus one-tenth of a pip. For example, if EUR/USD moves from 1.10501 to 1.10502, that 0.00001 change is one pipette.
     \bigbreak \noindent 
     \subsection{Why care about pips}
     \bigbreak \noindent 
     Traders use pips to calculate profits and losses. By knowing the number of pips they have gained or lost on a trade and the value of each pip, traders can calculate the financial outcome of their trades.
     \bigbreak \noindent 
     The monetary value of a pip can vary depending on the size of your trade (lot size) and the currency pair you are trading. In a standard lot of 100,000 units of currency, a pip usually equals \$10 in the base currency for pairs where the USD is the quote currency. For a mini lot (10,000 units of currency), a pip is typically worth \$1, and for a micro lot (1,000 units of currency), a pip is usually worth \$0.10.
     \bigbreak \noindent 
     \subsection{Calculating Pip Value}
     \bigbreak \noindent 
     First of all, what does it mean exactly to "calculate a pip value". Simply put, to "calculate pip" value in forex trading means to determine the value of a one-pip movement in a currency pair in terms of a specific currency. This calculation is crucial for managing risk, setting stop-loss and take-profit levels, and understanding the potential profit or loss from a trade.
     \bigbreak \noindent 
     The value of a pip varies based on the currency pair being traded, the size of the trade, and the exchange rate. For pairs where the USD is the quote currency, calculating the pip value is straightforward:
     \begin{itemize}
         \item \textbf{Standard Lot:} If you're trading a standard lot (100,000 units), a pip is usually worth \$10.
         \item \textbf{Mini Lot:} For a mini lot (10,000 units), a pip is worth \$1.
         \item \textbf{Micro Lot:} For a micro lot (1,000 units), a pip is worth \$0.10.
     \end{itemize}

     \bigbreak \noindent 
     Because the pip value varies depending on the currency pair you are trading and the currency your account is denominated in, it is best to use online calculators. However, we will briefly discuss the concept of calculating a pip value.
     \bigbreak \noindent 
     \subsubsection{When USD is the base currency}
     \bigbreak \noindent 
     \begin{itemize}
         \item Pip Value = (Pip in decimal places $\times$ Trade Size) / Exchange Rate
         \item Example for a standard lot of EUR/USD, where the trade size is 100,000 units and the pip size is 0.0001:
             \begin{align*}
                 \text{Pip Value } = (0.0001 \times 100,000) / 1 = \$10
             .\end{align*}
     \end{itemize}
     \bigbreak \noindent 
     \subsection{When USD is Not the Quote Currency or You Want the Pip Value in Another Currency:}
     \begin{itemize}
         \item First, calculate the pip value in the quote currency, then convert it to USD (or another currency) using an appropriate exchange rate.
            \item Example for a standard lot of USD/JPY, where the trade size is 100,000 units and the pip size is 0.01:
        \end{itemize}
            \item Pip Value in JPY = 0.01 $\times$ 100,000 = 1,000 JPY
            \item To convert to USD, divide by the USD/JPY exchange rate (assuming it's 110.00):
            \item Pip Value in USD = 1,000 / 110.00 = \$9.09
        \begin{itemize}
     \end{itemize}

     \pagebreak 
     \unsect{Leverages}
     \bigbreak \noindent 
     Leverage in forex trading is a tool that allows traders to control a large position with a relatively small amount of capital. It's expressed as a ratio, such as 50:1, 100:1, or even 500:1. This ratio determines the amount of money you can control in the market for every dollar in your trading account. Leverage amplifies both potential profits and losses, making it a powerful but risky tool.
     \bigbreak \noindent 
     \subsection{How Leverage Works}
     The basic concept is simple. If you have \$1,000 in your trading account and use 100:1 leverage, you can control a position worth \$100,000 (\$1,000 times 100). This means you can open larger positions than your actual capital would otherwise allow.
     \bigbreak \noindent 
     When you open a leveraged position, you're essentially borrowing money from your broker to increase the size of your trade. Despite the larger position, the amount required as a deposit is relatively small, known as the "margin."
     \bigbreak \noindent 
     The margin is the amount of capital required to open and maintain a leveraged position. It's a fraction of the full value of your trade. For example, with 100:1 leverage, the margin requirement might be 1\% of the total value of the position. So, to control a \$100,000 position, you need to have \$1,000 in your account as margin.
     \bigbreak \noindent 
     Leverage increases the potential for profit and loss by the same magnitude. If the market moves in your favor, you can achieve significant profits on the capital you've actually invested. However, if the market moves against you, you can quickly incur substantial losses, potentially exceeding your initial investment.

     \bigbreak \noindent 
     \subsubsection{Example}
     \bigbreak \noindent 
     Let's say you want to trade the EUR/USD currency pair, and you have \$1,000 in your trading account. You decide to use 100:1 leverage, allowing you to control a \$100,000 position in the market.
     \begin{itemize}
         \item If the Market Moves in Your Favor: Suppose the EUR/USD price moves up by 1\% after you open your position. Without leverage, a 1\% increase on a \$1,000 investment would have yielded a \$10 profit. However, because you're controlling a \$100,000 position with your \$1,000 investment, that same 1\% move translates to a \$1,000 profit (1\% of \$100,000).
         \item If the Market Moves Against You: Conversely, if the EUR/USD price moves down by 1\%, you would lose \$1,000, which is the entire amount of your initial investment. In a highly leveraged position, even small market movements can have significant impacts on your account balance.
     \end{itemize}

     \bigbreak \noindent 
     \subsection{Risks of Using Leverage}
     \begin{itemize}
         \item \textbf{Margin Calls:} If your account balance falls below the required margin due to trading losses, your broker may issue a margin call, requiring you to deposit additional funds to maintain your open positions. If you fail to meet the margin call, the broker might close your positions to limit further losses.
         \item \textbf{Rapid Losses:} High leverage can lead to rapid losses, especially in volatile markets. It's possible to lose more than your initial investment in a short period.
     \end{itemize}

     \bigbreak \noindent 
     \subsection{Best Practices}
     \begin{itemize}
         \item \textbf{Risk Management:} Use stop-loss orders to limit potential losses. Determine in advance the maximum amount you are willing to risk on a trade.
         \item \textbf{Leverage Ratio:} Choose a leverage ratio that matches your risk tolerance. Less experienced traders should consider using lower leverage to reduce risk.
     \end{itemize}

     \bigbreak \noindent 
     \subsection{Margin Calls and Stop-Out Levels}
     \bigbreak \noindent 
     The concepts described in this section detail what happens if your leveraged position starts to fall beyound your margin.
     \bigbreak \noindent 
     \subsubsection{Margin Calls}
     \bigbreak \noindent 
     If your account equity falls below a certain percentage of the required margin (because of trading losses), your broker will issue a margin call. This is a demand to deposit additional funds into your account to meet the minimum margin requirement. If you do not meet the margin call by depositing more funds, the broker may take action to close your positions to prevent further losses.

     \bigbreak \noindent 
     \subsubsection{Stop-Out Level}
     \bigbreak \noindent 
    If your account continues to fall and reaches the stop-out level, the broker will automatically start closing your open positions at the current market rates to prevent your account balance from going negative. This level is usually set at a certain percentage of the margin. For example, if the stop-out level is 20\%, your positions will start to be closed automatically when your account equity falls to 20\% of the required margin.
    \bigbreak \noindent 
    \subsubsection{Negative Balance Protection}
    \bigbreak \noindent 
    Many brokers offer negative balance protection, especially in jurisdictions where it's mandated by regulation. This policy ensures that traders cannot lose more money than they have deposited in their accounts. If a highly volatile market situation causes your account to go into a negative balance, the broker will absorb the loss and reset your account balance to zero.



     \pagebreak 
     \unsect{Fundemental Analysis}
     \bigbreak \noindent 
     Fundamental analysis in Forex trading involves evaluating the intrinsic value of a currency by examining related economic, financial, and other qualitative and quantitative factors. This approach aims to predict currency movements by analyzing the economic health of a country, considering various indicators and events that could influence currency supply and demand. 
     \bigbreak \noindent 
     \nt{Fundemental analysis is very useful in predicting the \textbf{direction}, while technical analysis is useful for deducing entries and exits. Fundemental trends can last anywhere from minutes, to years. Fundementals are what truly drives a markets underlying value. Fundemental trends can last anywhere from minutes, to years.}
     \bigbreak \noindent 
     \subsection{The Three Pilars of FA}
     \bigbreak \noindent 
     Each of the following fall under fundemental analysis
     \begin{itemize}
         \item \textbf{Economic Data}: (unemployment, GDP, etc...)
            \item \textbf{Central Bank Decisions}: (rate hikes, rate cuts)
            \item \textbf{Geopolitical events}: (Wars, pandemics, brexit, etc, ...)
     \end{itemize}
     \bigbreak \noindent 
     \subsubsection{Economic Data: Interest rates}
     \bigbreak \noindent 
     \begin{concept}
         An interest rate is the amount of interest due per period \footnote{The term "period" refers to the specific duration of time over which the interest is calculated and applied to the principal sum. The length of a period can vary widely depending on the agreement or the financial product in question.}, as a proportion of the amount lent, deposited, or borrowed. The total interest on an amout lent or borrowed depends on the principal sum \footnote{The principal sum, often simply referred to as the "principal," is the original amount of money that is either invested, saved, or borrowed, before any interest or earnings are added or any deductions (like fees or penalties) are made.}, the interest rate, the compounding frequency, and the length of time over which it is lent, deposited, or borrowed.
         \bigbreak \noindent 
         Interest rates set by central banks are among the most influential factors for currency value. Higher interest rates offer lenders higher returns relative to other countries, attracting foreign capital that increases the value of the home currency. Conversely, lower interest rates can decrease a currency's value.
     \end{concept}
     \bigbreak \noindent 
     \nt{There are two main aspects of interest rates, the ones that you pay to the bank, and the ones that the bank pays to you (like with a savings account).}

     \bigbreak \noindent 
     \subsubsection{Why higher interest rates positively affect currency}
     \bigbreak \noindent 
     In the context of Forex fundamental analysis, the effect of rising interest rates is multifaceted and can influence currency values in several ways. Here's how an increase in interest rates typically affects the Forex market:
     \bigbreak \noindent 
     \textbf{Attracting Foreign Investment}
     \bigbreak \noindent 
     Higher interest rates make a country's financial assets more attractive to foreign investors. For instance, if the U.S. Federal Reserve increases interest rates, yields on U.S. government bonds and other interest-bearing assets in the U.S. become more appealing. Foreign investors seeking higher returns might buy more of these assets, which requires them to purchase and hold U.S. dollars, thereby increasing demand for the USD in the Forex market. This demand can lead to an appreciation of the USD against other currencies.
     \bigbreak \noindent 
     \textbf{Increasing the Cost of Borrowing}
     \bigbreak \noindent 
     When interest rates go up, borrowing costs also rise. This can lead to a decrease in consumer spending and business investment since loans for things like homes, cars, and business expansions become more expensive. While this might slow economic growth, in the context of Forex trading, the immediate effect often focuses on the increased yields on investments in that currency, which can bolster the currency's value in the short term.
     \bigbreak \noindent 
     \textbf{Impact on Inflation}
     \bigbreak \noindent 
     Higher interest rates can help to control inflation. By making borrowing more expensive and saving more attractive, consumer spending can decrease, reducing upward pressure on prices. A currency that is seen as being strong against inflation can be more attractive to foreign investors. In the Forex market, currencies of countries with controlled inflation are often seen as more stable and potentially more valuable.
     \bigbreak \noindent 
     \textbf{Comparative Interest Rates}
     \bigbreak \noindent 
     In Forex trading, it's not just the absolute interest rate of a country that matters but how it compares to others. If the U.S. raises interest rates while other countries keep theirs steady or lower them, the relative attractiveness of U.S. assets increases even more, potentially leading to a stronger USD in Forex markets. Conversely, if other countries also raise rates in a similar fashion, the impact might be neutralized.
     \bigbreak \noindent 
     \textbf{Conclusion}
     \bigbreak \noindent 
     When considering Forex fundamental analysis, an increase in interest rates is generally seen as bullish for the currency, primarily due to the potential for increased foreign investment in higher-yielding assets. However, the overall impact on the Forex market also depends on other factors, including the economic context in which the rate increase occurs, expectations about future rate movements, and how rate changes compare to those in other countries.

     \bigbreak \noindent 
     \subsection{Economic Data: Inflation Rates}
     \bigbreak \noindent 
     \begin{concept}
         Inflation rates are a critical economic indicator in Forex fundamental analysis, reflecting the rate at which the general level of prices for goods and services is rising, and, subsequently, purchasing power is falling. Central banks attempt to limit inflation, and avoid deflation, to keep the economy running smoothly. Understanding inflation is crucial for Forex traders, as it affects currency value, interest rates, and overall economic health.
     \end{concept}

     \bigbreak \noindent 
     \subsubsection{How inflation rates affect forex markets}
     \bigbreak \noindent 
     \textbf{Interest Rate Adjustments}
     \bigbreak \noindent 
     Central banks use interest rate adjustments as a primary tool to control inflation. High inflation typically leads to higher interest rates, which can attract foreign capital to interest-bearing assets in the country, increasing demand for the currency and potentially causing it to appreciate. Conversely, low inflation or deflation may lead to lower interest rates, potentially reducing the currency's attractiveness and causing depreciation.
     \bigbreak \noindent 
     \textbf{Currency Value}
     \bigbreak \noindent 
     Inflation directly impacts currency value in the Forex market. A country with a lower inflation rate compared to others will see an appreciation in the value of its currency. A low inflation rate indicates that prices are rising more slowly than in other countries, increasing the purchasing power of this currency relative to others.
     \bigbreak \noindent 
     \textbf{Purchasing Power Parity (PPP)}
     \bigbreak \noindent 
     One of the fundamental theories in Forex trading is Purchasing Power Parity (PPP). PPP suggests that currencies will adjust in the Forex markets to offset changes in purchasing power, meaning that a country with a high inflation rate will see its currency depreciate against currencies of countries with lower inflation rates. This adjustment is because, over time, inflation reduces a currency's purchasing power domestically and internationally.
     \bigbreak \noindent 
     \subsubsection{Expectations and Speculation}
     \bigbreak \noindent 
     Forex markets not only respond to current inflation rates but also to expectations of future inflation. Traders speculate on future central bank actions (like adjusting interest rates) based on their inflation outlook. If traders believe that a central bank will raise rates to combat rising inflation, they might buy the currency in anticipation of its appreciation.
     \bigbreak \noindent 
     \subsubsection{Economic Health}
     \bigbreak \noindent 
     Moderate inflation is often a sign of a healthy economy, as it indicates growing demand. However, too high or too low inflation can signal economic problems. Forex traders monitor inflation closely, as it provides clues about economic health, consumer spending, and potential central bank actions—all of which can influence currency values.
     \bigbreak \noindent 
     \subsection{Interest and inflation rates in terms of currency pairs}
     \bigbreak \noindent 
     Suppose we are trading EUR/USD, where EUR is the base currency, and USD is the quote currency.
     \bigbreak \noindent 
     \subsection{USD Interest Rates Go Up}
     \bigbreak \noindent 
     If the U.S. Federal Reserve increases interest rates while the European Central Bank (ECB) keeps rates steady or increases them less aggressively, the interest rate differential widens in favor of the USD. Higher U.S. interest rates can attract investors looking for better returns on USD-denominated assets, increasing demand for the USD. As a result, the value of the USD may rise against the EUR, causing the EUR/USD pair to fall. This is because you now need fewer USD to buy one EUR, reflecting the increased strength of the USD relative to the EUR.
     \bigbreak \noindent 
     When we say "a stronger USD against the EUR," it means that the value of the US dollar has increased relative to the euro. In practical terms, you would need fewer USD to buy one EUR compared to before the USD strengthened.
     \bigbreak \noindent 
     Suppose one EUR is 1.3 USD. If the USD strengthens against the EUR, the exchange rate might change to, for example, 1 EUR = 1.2 USD.
     \bigbreak \noindent 
     \subsubsection{Base and Quote Currency}
     In the context of EUR/USD, the EUR is the base currency, and the USD is the quote currency. When discussing the impact of interest rates on this pair, an increase in the interest rates of the quote currency (USD) typically leads to a depreciation of the pair (meaning it costs fewer USD to buy one EUR), assuming all other factors remain constant. Conversely, if the interest rates rise in the Eurozone and remain constant or rise less in the U.S., the EUR may strengthen against the USD, causing the EUR/USD pair to appreciate (meaning it costs more USD to buy one EUR).

     \bigbreak \noindent 
     \subsubsection{Generalization}
     \bigbreak \noindent 
     \nt{When positive economic developments or policy changes occur in the country of the quote currency, it can lead to speculation that the currency pair's value will decrease, assuming all other factors remain constant. This is because the "good" news typically strengthens the quote currency relative to the base currency.}




     












     
     % \bigbreak \noindent 
     % \subsection{Economic Indicators}
     % \bigbreak \noindent 
     % \begin{concept}
     %     \textbf{Economic indicators} are statistics that provide insights into the economic performance of a country. They are used by traders to predict future currency movements. Key indicators include:
     % \end{concept}
     % \bigbreak \noindent 
     % \begin{itemize}
     %     \item \textbf{Gross Domestic Product (GDP):} The total market value of all goods and services produced in a country over a specific period. It's a primary indicator of a country's economic health.
     %     \item \textbf{Employment Indicators:} This includes unemployment rates and non-farm payrolls. High employment rates typically signify a strong economy, which can lead to a stronger currency.
     %     \item \textbf{Inflation Rates:} Measured by indicators like the Consumer Price Index (CPI) and Producer Price Index (PPI). Central banks often aim to control inflation by adjusting interest rates, which can directly affect currency values.
     %     \item \textbf{Trade Balance:} The difference between a country's exports and imports. A positive trade balance (more exports than imports) can lead to a rise in the currency's value as foreign buyers convert their currency to the exporter's currency.
     % \end{itemize}
     % \bigbreak \noindent 

     
     \bigbreak \noindent 
     \subsection{GDP}
     \bigbreak \noindent 
     \begin{concept}
         \textbf{Gross Domestic Product (GDP)} and its relationship with the foreign exchange (forex) market intertwine through the economic health of a country and its currency's value. Understanding this connection requires a breakdown into two main components: what GDP is and how it influences forex markets.
     \end{concept}
     \bigbreak \noindent 
     \subsubsection{What is GDP?}
     \bigbreak \noindent 
     Gross Domestic Product (GDP) is the total monetary or market value of all the finished goods and services produced within a country's borders in a specific time period. As a broad measure of overall domestic production, it functions as a comprehensive scorecard of a country’s economic health. Economists use GDP to gauge the economic performance of a country and to make international comparisons. GDP can be measured in three ways:
     \bigbreak \noindent 
     \begin{itemize}
         \item \textbf{Production Approach:} The total output of goods and services minus inputs or intermediate goods used during production.
         \item \textbf{Income Approach:} The total economic value generated by the production of goods and services, represented as wages, rent, interest, and profits.
         \item \textbf{Expenditure Approach:} The total value of all purchases made in a country, minus imports, over a specified period. This includes consumption, investment, government spending, and net exports.
     \end{itemize}
     \bigbreak \noindent 
     \subsubsection{Inputs}
     \bigbreak \noindent 
     Inputs refer to the resources used in the production of goods and services. These can be categorized into several types:
     \bigbreak \noindent 
     \begin{itemize}
         \item \textbf{Labor:} The human effort involved in the production process, including both physical and intellectual work.
         \item \textbf{Capital:} The tools, machinery, buildings, and technology used in production. It doesn't refer to money in this context but to physical assets that aid in producing goods and services.
         \item \textbf{Raw Materials:} The basic, unprocessed resources extracted from the earth or harvested from plants and animals that are used in the production of goods. Examples include crude oil, metals, timber, and agricultural products like wheat and cotton.
         \item \textbf{Energy:} The power used to operate machinery and equipment, light buildings, and transport goods and services. It can include electricity, natural gas, and petroleum products.
     \end{itemize}
     \bigbreak \noindent 
     \subsubsection{Intermediate Goods}
     \bigbreak \noindent 
     Intermediate goods are products that are used as inputs in the production of other goods and services. They are not finished goods; rather, they are used up or transformed in the manufacturing process. Examples of intermediate goods include:
     \begin{itemize}
         \item \textbf{Components:} Parts that are assembled or processed to create a final product. For instance, the processor in a computer or the tires on a car.
         \item \textbf{Raw Materials after Initial Processing:} These could be metals that have been refined but not yet manufactured into final products, or lumber that has been cut and treated but not yet used to build furniture or houses.
         \item \textbf{Services:} Services used by businesses in producing goods, such as cleaning, maintenance, or consulting services, can also be considered intermediate goods in the context of the production process.
     \end{itemize}

     \bigbreak \noindent 
     \subsubsection{How GDP Pertains to Forex}
     \bigbreak \noindent 
     GDP impacts the forex market in several significant ways:
     \begin{itemize}
         \item \textbf{Economic Health Indicator:} GDP is a key indicator of economic health. A country with a growing GDP is seen as having a strong economy, which attracts foreign investors looking for profitable opportunities. These investors need to buy the country’s currency to invest, which increases demand for the currency and, typically, its value.
         \item \textbf{Interest Rates and Inflation:} Central banks watch GDP closely as part of their mandate to manage inflation and set interest rates. Strong GDP growth might lead a central bank to raise interest rates to combat inflationary pressures, which can attract foreign capital due to higher returns, thus boosting the currency's value. Conversely, if GDP growth is slow, central banks might lower interest rates to stimulate growth, potentially reducing the currency's attractiveness.
         \item \textbf{Investment Flows:} High GDP growth rates can signal profitable investment opportunities in a country, drawing in foreign direct investment (FDI) and portfolio investment from abroad. These investments require the conversion of foreign currencies into the local currency, increasing demand for the local currency and potentially boosting its value.
         \item \textbf{Trade Balances:} GDP growth affects a country’s trade balance – the difference between exports and imports. A strong, growing economy can lead to increased consumption, possibly increasing imports. If the growth is not export-driven, this can lead to a trade deficit, potentially weakening the currency. Conversely, if GDP growth is driven by an increase in export activities, this could strengthen the currency.
             \item \textbf{Sentiment and Speculation:} Forex markets are also driven by sentiment and speculation. Strong or improving GDP figures can enhance investor sentiment regarding a country’s economic prospects, leading to increased demand for its currency in anticipation of future strength.
     \end{itemize}

     \bigbreak \noindent 
     \subsection{Unemployment Rates}
     \bigbreak \noindent 
     Unemployment rates are a critical economic indicator reflecting the percentage of the labor force that is jobless and actively seeking employment. Understanding how unemployment rates affect foreign exchange (forex) trading requires an appreciation of their impact on a country's economic health, consumer confidence, and central bank policies—all of which can influence currency value.
     \bigbreak \noindent 
     \subsubsection{Calculation}
     \bigbreak \noindent 
     The unemployment rate is calculated by taking the number of unemployed people, dividing it by the total labor force (the sum of employed and unemployed individuals), and then multiplying by 100 to get a percentage. 
     \bigbreak \noindent 
     \nt{It's important to note that "unemployed" doesn't just mean not having a job; it specifically refers to those who are not working but are available for work and have been actively seeking employment.}

     \bigbreak \noindent 
     \subsubsection{Connection to forex}
     \bigbreak \noindent 
     \begin{itemize}
         \item \textbf{Economic Health Indicator:} The unemployment rate is a lagging indicator of economic health. High unemployment typically signifies a weakening economy, as businesses are not hiring due to low demand or poor financial conditions. Conversely, low unemployment suggests a strong economy, with businesses hiring more due to high demand for products and services. Forex traders closely watch unemployment rates as a signal of economic strength or weakness, which can influence a currency's value.
         \item \textbf{Consumer Confidence and Spending:} High unemployment can lead to decreased consumer confidence, as jobless individuals are likely to reduce their spending. Lower consumer spending can slow economic growth, affecting the country's overall economic health and, subsequently, its currency value. A strong job market, on the other hand, can boost consumer confidence and spending, potentially leading to economic expansion and a stronger currency.
         \item \textbf{Interest Rates and Monetary Policy:} Central banks monitor unemployment rates closely as part of their mandate to maintain economic stability and manage inflation. High unemployment may prompt a central bank to lower interest rates to stimulate economic growth by encouraging borrowing and investing. Lower interest rates can make a currency less attractive to investors, as returns on investments in that currency may be lower compared to others. Conversely, improving employment conditions may lead to higher interest rates to control inflation, attracting forex traders looking for higher yields.
         \item \textbf{Inflation:} There's also a relationship between unemployment rates and inflation, often described by the Phillips curve, which suggests an inverse relationship between the two. Low unemployment can lead to wage increases as employers compete for a limited workforce, potentially leading to inflation. Central banks may respond by raising interest rates, which can strengthen the currency. High unemployment usually has the opposite effect, with lower wage pressure and subdued inflation.
     \end{itemize}
     \bigbreak \noindent 
     \subsubsection{Trading strategy based on unemployment data}
     \bigbreak \noindent 
     Forex traders use unemployment data to anticipate central bank actions, economic trends, and shifts in consumer sentiment. A worse-than-expected unemployment rate can lead to a currency weakening, while better-than-expected figures can strengthen a currency. Traders might position themselves ahead of unemployment reports to capitalize on these movements or adjust their strategies in response to unexpected data.


     \bigbreak \noindent 
     \subsubsection{Philips curve}
     \bigbreak \noindent 
     The original Phillips Curve posits that when unemployment is low, resources in the economy (particularly labor) are in high demand, which drives up wages. As wages increase, consumers have more money to spend, leading to an increase in demand for goods and services. If demand outpaces supply, it can lead to higher prices, or inflation. Conversely, high unemployment leads to less demand for labor, lower wages, and subsequently, lower consumer spending, which can reduce inflation.

     \bigbreak \noindent 
     \subsubsection{Short-run vs. Long-run Phillips Curve}
     \begin{itemize}
         \item \textbf{Short-run Phillips Curve:} In the short term, the Phillips Curve suggests that there is a trade-off between inflation and unemployment; a country could reduce unemployment at the cost of higher inflation, and vice versa. This relationship is dynamic and can shift due to various factors, such as expectations of inflation, supply shocks, or changes in potential output.
         \item \textbf{Long-run Phillips Curve:} Economists like Milton Friedman and Edmund Phelps argued that the trade-off between inflation and unemployment only holds in the short run. They suggested that in the long run, there is no trade-off between inflation and unemployment, leading to the concept of the long-run Phillips Curve being vertical at the natural rate of unemployment. This rate is determined by the structure of the labor market and is not affected by inflation. In the long run, attempts to maintain an unemployment rate below the natural rate could lead to accelerating inflation without reducing unemployment.
     \end{itemize}

     \bigbreak \noindent 
     \subsubsection{Expectations-Augmented Phillips Curve}
     \bigbreak \noindent 
  One significant development was the introduction of inflation expectations into the Phillips Curve model. People's expectations of future inflation can influence their behavior, affecting wage negotiations and price setting. If people expect higher inflation, they will seek higher wages, which can, in turn, lead to higher prices, thereby influencing the actual inflation. This modification to the Phillips Curve suggests that policymakers cannot exploit the trade-off between inflation and unemployment in the long run without causing inflation to continually accelerate.

  \bigbreak \noindent 
  \subsection{Central bank policies}
  \bigbreak \noindent 











     

     \pagebreak 
     \unsect{Technical Analysis}

     \pagebreak 
     \unsect{Trading Strategies}

     \pagebreak 
     \unsect{Trading Psychology}

     \pagebreak 
     \unsect{Money Management}

     \pagebreak 
     \unsect{Regulations and Ethics}








     \pagebreak 
     \unsect{Appendix A: Terminology}
     \bigbreak \noindent 
     Forex trading comes with its own set of jargon and terminology that traders need to understand to navigate the markets effectively. Comprised in this section is a list of terms to be familiar with.
     \bigbreak \noindent 
     \textbf{Note:} Some of these terms are discussed in more detail in above sections
     \bigbreak \noindent 
     \begin{itemize}
         \item \textbf{Liquid market}: A market allowing the buying or selling of large quantities of an asset at any time and at low transactions costs
         \item \textbf{Illiquid market}: The converse of a liquid market. These markets may have considerably large spreads between the highest available buyer and the lowest available seller.
         \item \textbf{Thin market}: Term used to describe an illiquid market
         \item \textbf{Base currency}: The first currency in a currency pair, which is used as the reference for buying or selling.
         \item \textbf{Quote currency}: The second currency in a currency pair, which shows how much of this currency is needed to buy one unit of the base currency.
         \item \textbf{Spread}: The difference between the bid (sell) and ask (buy) price of a currency pair. Tighter spreads generally mean lower trading costs.
         \item \textbf{Bull market}: A market condition in which the prices of securities are rising, encouraging buying.
         \item \textbf{Bear market}: A market condition in which the prices of securities are falling, encouraging selling.
         \item \textbf{Margin}: The amount of capital required in your account to open and maintain a position. Margin is usually expressed as a percentage of the full position. For example, if a broker requires 1\% margin to trade a standard lot of EUR/USD, you would need \$1,000 in your account to open a \$100,000 position.
         \item \textbf{Leverage}: A loan provided by the broker to the trader, allowing the trader to control a large position with a relatively small amount of invested capital. Leverage is expressed as a ratio, such as 50:1, 100:1, or 500:1. While it can amplify profits, it also increases the risk of significant losses.
         \item \textbf{Lot size:} The number of currency units you are buying or selling in a trade. Standard lot sizes include:
             \begin{itemize}
                 \item \textbf{Standard Lot:} 100,000 units of the base currency.
                 \item \textbf{Mini Lot:} 10,000 units of the base currency.
                 \item \textbf{Micro Lot:} 1,000 units of the base currency.
                 \item \textbf{Nano Lot:} 100 units of the base currency.
             \end{itemize}
            \item \textbf{Pip (Percentage in Point):}
                The smallest price move that a given exchange rate can make based on market convention. Most major currency pairs are priced to four decimal places, and a pip is one unit of the fourth decimal point, for most pairs. For example, if EUR/USD moves from 1.1050 to 1.1051, that 0.0001 USD rise in value is one pip.
            \item \textbf{Volatile market}: Volatility is an investment term that describes when a market or security experiences periods of unpredictable, and sometimes sharp, price movements.
        \item \textbf{Rollover}: A rollover is the process of keeping a position open beyond its expiry
        \item \textbf{Going long}: If you buy a currency pair expecting it to increase in value, you are “GOING LONG.”
        \item \textbf{Shorting}: If you sell a currency pair expecting it to decrease in value, you are “SHORTING.”
        \item \textbf{Swaps}: A swap in forex refers to the interest that you either earn or pay for a trade that you keep open overnight.
        \item \textbf{Interest rates}: An interest rate is the amount of interest due per period, as a proportion of the amount lent, deposited, or borrowed. The total interest on an amout lent or borrowed depends on the principal sum, the interest rate, the compounding frequency, and the length of time over which it is lent, deposited, or borrowed.
        \item \textbf{Interest rate period}: The term "period" refers to the specific duration of time over which the interest is calculated and applied to the principal sum. The length of a period can vary widely depending on the agreement or the financial product in question.
        \item \textbf{Principal sum:} The principal sum, often simply referred to as the "principal," is the original amount of money that is either invested, saved, or borrowed, before any interest or earnings are added or any deductions (like fees or penalties) are made.
     \end{itemize}



     


      \pagebreak 
     \unsect{Appendix B: Formulas}
     \begin{itemize}
         \item \textbf{Pip calculation} 
             \begin{align*}
                  \item Pip Value = (Pip in decimal places $\times$ Trade Size) / Exchange Rate
             .\end{align*}
     \end{itemize}

     \pagebreak 
     \unsect{Appendix C: Chart patterns}


     \pagebreak 
     \unsect{Appendix D: Fundemental Analysis}
     \begin{itemize}
         \item \textbf{Common interest rate periods}
             \begin{itemize}
                 \item \textbf{Daily:} Interest is calculated and applied every day. This is common in some types of savings accounts and loans.
                 \item \textbf{Monthly:} Interest is calculated and applied once a month. Many loans and savings accounts use monthly periods for interest calculations.
                 \item \textbf{Quarterly:} Interest is calculated and applied every three months. Certain investments and savings accounts might compound interest quarterly.
                 \item \textbf{Annually:} Interest is calculated and applied once a year. This is a very common period for the calculation of interest on loans, savings accounts, and investments.
             \end{itemize}
     \end{itemize}





    
\end{document}
