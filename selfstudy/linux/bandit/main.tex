\documentclass{report}

\input{~/dev/latex/template/preamble.tex}
\input{~/dev/latex/template/macros.tex}

\title{\Huge{}}
\author{\huge{Nathan Warner}}
\date{\huge{}}
\pagestyle{fancy}
\fancyhf{}
\lhead{Warner \thepage}
\rhead{}
% \lhead{\leftmark}
\cfoot{\thepage}
%\setborder
% \usepackage[default]{sourcecodepro}
% \usepackage[T1]{fontenc}

\begin{document}
    % \maketitle
        \begin{titlepage}
       \begin{center}
           \vspace*{1cm}
    
           \textbf{Over the wire solutions} \\
           Bandit
    
           \vspace{0.5cm}
            
                
           \vspace{1.5cm}
    
           \textbf{Nathan Warner}
    
           \vfill
                
                
           \vspace{0.8cm}
         
           \includegraphics[width=0.4\textwidth]{~/niu/seal.png}
                
           Computer Science \\
           Northern Illinois University\\
           November 5, 2023 \\
           United States\\
           
                
       \end{center}
    \end{titlepage}
    \tableofcontents
    \pagebreak 
    \phantomsection
    \addcontentsline{toc}{section}{Passwords}
    \section*{Passwords}
    \begin{enumerate}[start=0]
        \item bandit0
        \item NH2SXQwcBdpmTEzi3bvBHMM9H66vVXjL
        \item rRGizSaX8Mk1RTb1CNQoXTcYZWU6lgzi
        \item aBZ0W5EmUfAf7kHTQeOwd8bauFJ2lAiG
        \item 2EW7BBsr6aMMoJ2HjW067dm8EgX26xNe
        \item lrIWWI6bB37kxfiCQZqUdOIYfr6eEeqR
    \end{enumerate}

    \pagebreak 
    \phantomsection
    \addcontentsline{toc}{section}{Commands used}
    \section*{Commands used}
    \begin{itemize}
        \item \textbf{xclip -selection clipboard < filename}: used to save contents of a file into clipboard (can't use on bandit servers)
        \item \textbf{File:} The file command is used to determine the type of a file. In this game, we use it to determine if a file is human-readable (ASCII text), as apposed to DATA.  
        \item \textbf{Find:}
            \begin{itemize}
                \item Exec flag
                \item ! -executable (negation of -executable)
                \item -size 1033c
            \end{itemize}
        \item \textbf{grep}
            \begin{itemize}
                \item -e (match pattern)
                \item -v (match NOT pattern)
            \end{itemize}
        \item \textbf{Exclude "permission denied" output from grep}
            \begin{itemize}
                \item find / [expressions] 2>\&1 | grep -v "Permission denied"
            \end{itemize}
        \item \textbf{awk}
            \begin{itemize}
                \item awk '\{print $2\}'
            \end{itemize}
    \end{itemize}



    \pagebreak  \bigbreak \noindent 
    \vspace{2in} \\
    \begin{Huge}
        \textbf{Over The Wire \\ \hspace*{.4in} Solutions}
    \end{Huge}
    \bigbreak \noindent 
    \line(1,0){490}

    \bigbreak \noindent 
    \phantomsection
    \addcontentsline{toc}{section}{Level 4}
    \section*{Level 4} 
    \bigbreak \noindent 
    The password for the next level is stored in the only human-readable file in the inhere directory
    \bigbreak \noindent 
    \subsection*{The file command}
    The way we solve this level is by use of the file command. The file command will output whether a file is DATA or ASCII text. We are looking for the file that contains ASCII text
    \bigbreak \noindent 
    \begin{minted}{bash}
for i in *;do file "./$i";done
    \end{minted}
    \bigbreak \noindent
    \phantomsection
    \addcontentsline{toc}{subsection}{Looping with the "find" command}
    \subsection*{Looping with the "find" command}
    \bigbreak \noindent 
    Instead of using a simple for loop, we can instead make use of the \textit{find} command with the -exec flag. The general syntax for this  method is 
    \bigbreak \noindent 
    \begin{minted}{bash}
find [path] [expression] -exec [command] {} \;
// or
find [path] [expression] -exec [command] {} +
    \end{minted}
    \bigbreak \noindent
    Where:
    \begin{itemize}
        \item \relax [path] is the directory find starts searching from. If omitted, find uses the current directory.
        \item \relax [expression] is used to specify search criteria such as name, type, size, etc.
        \item \relax [command] is the command that find will execute on each file that matches the search criteria.
        \item \relax \{\} is a placeholder that find replaces with the current file name being processed.
        \item \textbackslash; is used to terminate the command.
    \end{itemize}
    \bigbreak \noindent 
    When using \; at the end of the -exec command, find will execute the command once for each file found. However, when you use + at the end of the -exec command, find will pass all the matched files to the command at once, rather than one by one. This is often more efficient, especially when dealing with a large number of files, because it minimizes the number of times the command is called.
    \bigbreak \noindent 
    Thus, the solution would be 
    \bigbreak \noindent 
    \begin{minted}{bash}
find . -type f -exec file {} \; | grep -e 'text'
// or
find . -type f -exec file {} + | grep -e 'text'
    \end{minted}
    \bigbreak \noindent
    We grep 'text' because file with either output the file as being data, or ASCII text, so we only want to display the file that has ASCII text.

    \bigbreak \noindent 
    \phantomsection
    \addcontentsline{toc}{section}{Level 5}
    \section*{Level 5}
    \bigbreak \noindent 
    The password for the next level is stored in a file somewhere under the inhere directory and has all of the following properties:
    \begin{itemize}
        \item human-readable
        \item 1033 bytes in size
        \item not executable
    \end{itemize}
    \bigbreak \noindent 
    The way in which we solve this problem is by use of:
    \begin{itemize}
        \item For loop
        \item Find command 
        \item file command
    \end{itemize}
    \bigbreak \noindent 
    \begin{minted}{bash}
for i in *;do find "./${i}" -type f -size 1033c ! -executable -exec file {} \; | grep -e 'ASCII' ;done
    \end{minted}
    \bigbreak \noindent

    \bigbreak \noindent 
    \phantomsection
    \addcontentsline{toc}{section}{Level 6}
    \section*{Level 6}
    \bigbreak \noindent 
    The password for the next level is stored somewhere on the server and has all of the following properties:
    \begin{itemize}
        \item owned by user bandit7
        \item owned by group bandit6
        \item 33 bytes in size
    \end{itemize}
    \bigbreak \noindent 
    \begin{minted}{bash}
find / -type f -size 33c -user bandit7 -group bandit6 2>&1 | grep -v "Permission denied"
    \end{minted}
    \bigbreak \noindent
    \nt{The reason we also redirect standard error and standard output to the grep command (btw grep -v excludes matches) is because "permission denied" matches are errors, thus we need to send them to our grep}

    \pagebreak 
    \phantomsection
    \addcontentsline{toc}{section}{Level 7}
    \section*{Level 7}
    \bigbreak \noindent 
    The password for the next level is stored in the file data.txt next to the word millionth
    \bigbreak \noindent 
    This one is pretty simple, we just grep with awk
    \bigbreak \noindent 
    \begin{minted}{bash}
grep -e 'millionth' data.txt | awk '{print $2}'
    \end{minted}
    \bigbreak \noindent

    \bigbreak \noindent 
    \phantomsection
    \addcontentsline{toc}{section}{Level 8}
    \section*{Level 8}
    \bigbreak \noindent 
    The password for the next level is stored in the file data.txt and is the only line of text that occurs only once
    \bigbreak \noindent 
    For this level we use the \textit{uniq} command with the -u flag, this will output unique lines. However, the uniq command  does not detect repeated lines unless they are adjacent.  You may want to sort the input first
    \bigbreak \noindent 
    \begin{minted}{bash}
sort data.txt | uniq -u
    \end{minted}
    \bigbreak \noindent





    
\end{document}
