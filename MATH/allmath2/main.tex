\documentclass{report}

\input{~/dev/latex/template/preamble.tex}
\input{~/dev/latex/template/macros.tex}

\title{\Huge{}}
\author{\huge{Nathan Warner}}
\date{\huge{}}
\pagestyle{fancy}
\fancyhf{}
\lhead{Warner \thepage}
\rhead{}
% \lhead{\leftmark}
\cfoot{\thepage}
%\setborder
% \usepackage[default]{sourcecodepro}
% \usepackage[T1]{fontenc}

\begin{document}
    % \maketitle
        \begin{titlepage}
       \begin{center}
           \vspace*{1cm}
    
           \textbf{Comprehensive Compendium:} \\
            Calculus II
    
           \vspace{0.5cm}
            
                
           \vspace{1.5cm}
    
           \textbf{Nathan Warner}
    
           \vfill
                
                
           \vspace{0.8cm}
         
           \includegraphics[width=0.4\textwidth]{~/niu/seal.png}
                
           Computer Science \\
           Northern Illinois University\\
           August 28,2023 \\
           United States\\
           
                
       \end{center}
    \end{titlepage}
    \tableofcontents
    \pagebreak \bigbreak \noindent
    \section{\LARGE Calc II}
    \bigbreak \noindent 

    \bigbreak \noindent 
    \subsection{Chapter 1 Key Equations}
    \bigbreak \noindent 
    \begin{itemize}
        \item \textbf{Mean Value Theorem For Integrals}: If  $f(x)$ is continuous over an interval  [a,b], then there is at least one point  $c\in[a,b]$ such that 
            \begin{align*}
                f(c) = \frac{1}{b-a}\int f(x)\ dx
            .\end{align*}
        \item \textbf{Integrals resulting in inverse trig functions}
                \begin{enumerate}
        \item \begin{align*}
                \int \frac{dx}{\sqrt{a^{2}-x^{2}}} = \sin^{-1}{\frac{x}{\abs{a}}} + C
        .\end{align*}
    \item \begin{align*}
        \int \frac{dx}{a^{2}+x^{2}} = \frac{1}{a}\tan^{-1}{\frac{x}{a}} + C
    .\end{align*}
    \item \begin{align*}
            \int \frac{dx}{x\sqrt{x^{2}-a^{2}}} = \frac{1}{\abs{a}}\sec^{-1}{\frac{\abs{x}}{a}} + C
    .\end{align*}
    \end{enumerate}
    \end{itemize}

    \pagebreak \bigbreak \noindent 
    \subsection{Chapter 2 Key Terms / Ideas}
    \bigbreak \noindent 
    \begin{itemize}
        \item \textbf{Finding limits of integration for region between two functions}: Usually, we want our limits of integration to be the points where the functions intersect
        \item A \textbf{"complex region"} between curves usually refers to an area that is not easily described by a single, continuous function over the interval of interest.
        \item \textbf{compound regions} are regions bounded by the graphs of functions that cross one another
        \item \textbf{Cross-section:} The intersection of a plane and a solid object.
        \item a \textbf{cylinder} is a three-dimensional shape that has two parallel, congruent bases connected by a curved surface. The bases are usually circles, but they can be other shapes as well
        \item The line segment connecting the centers of the two bases is called the \textbf{"axis" of the cylinder.}
        \item \textbf{Slicing method:} A method of calculating the volume of a solid that involves cutting the solid into pieces, estimating the volume of each piece, then adding these estimates to arrive at an estimate of the total volume; as the number of slices goes to infinity, this estimate becomes an integral that gives the exact value of the volume.
        \begin{enumerate}
            \item Examine the solid and determine the shape of a cross-section of the solid. It is often helpful to draw a picture if one is not provided.
            \item Determine a formula for the area of the cross-section.
            \item Integrate the area formula over the appropriate interval to get the volume.
        \end{enumerate}
        \item \textbf{Solid of revolution:} A solid generated by revolving a region in a plane around a line in that plane.
        \item \textbf{Disk method:} A special case of the slicing method used with solids of revolution when the slices are disks.
        \item A \textbf{Washer (Annuli)} is a disk with holes in the center.
        \item \textbf{Washer method:} A special case of the slicing method used with solids of revolution when the slices are washers.
        \item \textbf{Method of cylindrical shells:} A method of calculating the volume of a solid of revolution by dividing the solid into nested cylindrical shells; this method is different from the methods of disks or washers in that we integrate with respect to the opposite variable.
        % \item A \textbf{cylinder} is defined as any solid that can be generated by translating a plane region along a line perpendicular to the region, called the \textbf{axis of the cylinder}.
         \item \textbf{Arc length:} The arc length of a curve can be thought of as the distance a person would travel along the path of the curve.
        \item \textbf{Surface area:} The surface area of a solid is the total area of the outer layer of the object; for objects such as cubes or bricks, the surface area of the object is the sum of the areas of all of its faces.
        % \item \textbf{Catenary:} A curve in the shape of the function \(y = a \cosh(x/a)\) is a catenary; a cable of uniform density suspended between two supports assumes the shape of a catenary.
        % \item \textbf{Center of mass:} The point at which the total mass of the system could be concentrated without changing the moment.
        % \item \textbf{Centroid:} The centroid of a region is the geometric center of the region; laminas are often represented by regions in the plane; if the lamina has a constant density, the center of mass of the lamina depends only on the shape of the corresponding planar region; in this case, the center of mass of the lamina corresponds to the centroid of the representative region.
        % \item \textbf{Density function:} A density function describes how mass is distributed throughout an object; it can be a linear density, expressed in terms of mass per unit length; an area density, expressed in terms of mass per unit area; or a volume density, expressed in terms of mass per unit volume; weight-density is also used to describe weight (rather than mass) per unit volume.
        % \item \textbf{Doubling time:} If a quantity grows exponentially, the doubling time is the amount of time it takes the quantity to double, and is given by \(\frac{\ln 2}{k}\).
        % \item \textbf{Exponential decay:} Systems that exhibit exponential decay follow a model of the form \(y = y_0 e^{-kt}\).
        % \item \textbf{Exponential growth:} Systems that exhibit exponential growth follow a model of the form \(y = y_0 e^{kt}\).
        % \item \textbf{Frustum:} A portion of a cone; a frustum is constructed by cutting the cone with a plane parallel to the base.
        % \item \textbf{Half-life:} If a quantity decays exponentially, the half-life is the amount of time it takes the quantity to be reduced by half. It is given by \(\frac{\ln 2}{k}\).
        % \item \textbf{Hooke's Law:} This law states that the force required to compress (or elongate) a spring is proportional to the distance the spring has been compressed (or stretched) from equilibrium; in other words, \(F = kx\), where \(k\) is a constant.
        % \item \textbf{Hydrostatic pressure:} The pressure exerted by water on a submerged object.
        % \item \textbf{Lamina:} A thin sheet of material; laminas are thin enough that, for mathematical purposes, they can be treated as if they are two-dimensional.
        % \item \textbf{Moment:} If \(n\) masses are arranged on a number line, the moment of the system with respect to the origin is given by \(M = \sum_{i=1}^{n} m_i x_i\); if, instead, we consider a region in the plane, bounded above by a function \(f(x)\) over an interval \([a, b]\), then the moments of the region with respect to the \(x\)- and \(y\)-axes are given by \(M_x = \rho \int_{a}^{b} \frac{[f(x)]^2}{2} dx\) and \(M_y = \rho \int_{a}^{b} x f(x) dx\), respectively.
        % \item \textbf{Symmetry principle:} The symmetry principle states that if a region \(R\) is symmetric about a line \(l\), then the centroid of \(R\) lies on \(l\).
        % \item \textbf{Theorem of Pappus for volume:} This theorem states that the volume of a solid of revolution formed by revolving a region around an external axis is equal to the area of the region multiplied by the distance traveled by the centroid of the region.
        % \item \textbf{Work:} The amount of energy it takes to move an object; in physics, when a force is constant, work is expressed as the product of force and distance.
    \end{itemize}

    \pagebreak \bigbreak \noindent 
    \subsection{Chapter 2 Key Equations}
    \bigbreak \noindent 
    \begin{itemize}

    \item \textbf{Area between two curves, integrating on the x-axis}
    \begin{align}
        A = \int_{a}^{b} [f(x) - g(x)] \, dx
    \end{align}
    Where $f(x) \geq g(x)$
    \begin{align*}
        A = \int_{a}^{b}\ [g(x) - f(x)]\ dx
    .\end{align*}
    for $g(x) \geq f(x)$

    \item \textbf{Area between two curves, integrating on the y-axis}
    \begin{align}
        A = \int_{c}^{d} [u(y) - v(y)] \, dy
    \end{align}

    \item \textbf{Areas of compound regions}
        \begin{align*}
          \int_{a}^{b}\ \abs{f(x)-g(x)}\ dx 
        .\end{align*}
    \item \textbf{Area of complex regions}
        \begin{align*}
            \int_{a}^{b}\ f(x)\ dx + \int_{b}^{c}\ g(x)\ dx
        .\end{align*}
    \item \textbf{Slicing Method}
        \begin{align*}
            V(s) = \summation{n}{i=1}\ A(x_{i}^{*})\ \Delta x  = \int_{a}^{b}\ A(x)\ dx
        .\end{align*}
    \item \textbf{Disk Method along the x-axis}
    \begin{align}
        V = \int_{a}^{b} \pi [f(x)]^2 \, dx
    \end{align}

    \item \textbf{Disk Method along the y-axis}
    \begin{align}
        V = \int_{c}^{d} \pi [g(y)]^2 \, dy
    \end{align}

    \item \textbf{Washer Method along the x-axis}
    \begin{align}
        V = \int_{a}^{b} \pi [(f(x))^2 - (g(x))^2] \, dx
    \end{align}

    \item \textbf{Washer Method along the y-axis}
    \begin{align}
        V = \int_{c}^{d} \pi [(u(y))^2 - (v(y))^2] \, dy
    \end{align}

    \item \textbf{Radius if revolved around other line (Washer Method)}
        \begin{align*}
           If:\ x=-k\\
           Then:\ r = Function + k
        .\end{align*}
        \begin{align*}
           If:\ x=k\\
           Then:\ r = k - Function
        .\end{align*}

    \item \textbf{Method of Cylindrical Shells (x-axis)}
    \begin{align}
        V = \int_{a}^{b} 2\pi x f(x) \, dx
    \end{align}

    \item \textbf{Method of Cylindrical Shells (y-axis)}
    \begin{align}
        V = \int_{c}^{d} 2\pi y g(y) \, dy
    \end{align}

    \item \textbf{Region revolved around other line (method of cylindrical shells):}
        \begin{align*}
            If:\ x=-k \\
            Then:\ V = \int_{a}^{b}\ 2\pi (x+k)(f(x))\ dx
        .\end{align*}
        \begin{align*}
             If:\ x=k \\
            Then:\ V = \int_{a}^{b}\ 2\pi (k-x)(f(x))\ dx
        .\end{align*}
    \item \textbf{A Region of Revolution Bounded by the Graphs of Two Functions (method cylindrical shells)}
        \begin{align*}
            V = \int_{a}^{b}\ 2\pi x\left[f(x)-g(x)\right]\ dx
        .\end{align*}

    \item \textbf{Arc Length of a Function of x}
    \begin{align}
        \text{Arc Length} = \int_{a}^{b} \sqrt{1 + [f'(x)]^2} \, dx
    \end{align}

    \item \textbf{Arc Length of a Function of y}
    \begin{align}
        \text{Arc Length} = \int_{c}^{d} \sqrt{1 + [g'(y)]^2} \, dy
    \end{align}

    \item \textbf{Surface Area of a Function of x}
    \begin{align}
        \text{Surface Area} = \int_{a}^{b} 2\pi f(x) \sqrt{1 + [f'(x)]^2} \, dx
    \end{align}
    \item \textbf{Natural logarithm function}
    \begin{align}
        \ln x = \int_{1}^{x} \frac{1}{t} \, dt\ Z
    \end{align}

    \item \textbf{Exponential function}
    \begin{align}
        y = e^x, \quad \ln y = \ln(e^x) = x\ Z
    \end{align}

    % \item \textbf{Mass of a one-dimensional object}
    % \begin{align}
    %     m = \int_{a}^{b} \rho(x) \, dx
    % \end{align}
    %
    % \item \textbf{Mass of a circular object}
    % \begin{align}
    %     m = \int_{0}^{r} 2\pi x \rho(x) \, dx
    % \end{align}
    %
    % \item \textbf{Work done on an object}
    % \begin{align}
    %     W = \int_{a}^{b} F(x) \, dx
    % \end{align}
    %
    % \item \textbf{Hydrostatic force on a plate}
    % \begin{align}
    %     F = \int_{a}^{b} \rho w(x) s(x) \, dx
    % \end{align}
    %
    % \item \textbf{Mass of a lamina}
    % \begin{align}
    %     m = \rho \int_{a}^{b} f(x) \, dx
    % \end{align}
    %
    % \item \textbf{Moments of a lamina}
    % \begin{align}
    %     M_x = \rho \int_{a}^{b} \frac{[f(x)]^2}{2} \, dx, \quad M_y = \rho \int_{a}^{b} x f(x) \, dx
    % \end{align}
    %
    % \item \textbf{Center of mass of a lamina}
    % \begin{align}
    %     \bar{x} = \frac{M_y}{m},\ \ \text{and}\ \  \bar{y} = \frac{M_x}{m}
    % \end{align}

    \end{itemize}





\end{document}
