\documentclass{report}

\input{~/dev/latex/template/preamble.tex}
\input{~/dev/latex/template/macros.tex}

\title{\Huge{}}
\author{\huge{Nathan Warner}}
\date{\huge{}}
\pagestyle{fancy}
\fancyhf{}
\rhead{}
\fancyhead[R]{\itshape Warner} % Right header: Last Name
\fancyhead[L]{\itshape\leftmark}  % Right header: Section Name
\cfoot{\thepage}
\renewcommand{\headrulewidth}{0pt} % Remove the header line

% Change the title
\hypersetup{
	pdftitle={Math 2}
}


\begin{document}
    % \maketitle
        \begin{titlepage}
       \begin{center}
           \vspace*{1cm}
    
           \textbf{Comprehensive Compendium:} \\
            Calculus II
    
           \vspace{0.5cm}
            
                
           \vspace{1.5cm}
    
           \textbf{Nathan Warner}
    
           \vfill
                
                
           \vspace{0.8cm}
         
           \includegraphics[width=0.4\textwidth]{~/niu/seal.png}
                
           Computer Science \\
           Northern Illinois University\\
           August 28,2023 \\
           United States\\
           
                
       \end{center}
    \end{titlepage}
    \tableofcontents
    \pagebreak \bigbreak \noindent
    \unsect{Calculus II}
    \bigbreak \noindent 

    \bigbreak \noindent 
    \subsection{Chapter 1 Definitions and Theorems}
    \bigbreak \noindent 
    \begin{itemize}
        \item \textbf{Mean Value Theorem For Integrals}: If  $f(x)$ is continuous over an interval  [a,b], then there is at least one point  $c\in[a,b]$ such that 
            \begin{align*}
                f(c) = \frac{1}{b-a}\int f(x)\ dx
            .\end{align*}
        \item \textbf{Integrals resulting in inverse trig functions}
            \begin{enumerate}
                \item \begin{align*}
                        \int \frac{dx}{\sqrt{a^{2}-x^{2}}} = \sin^{-1}{\frac{x}{\abs{a}}} + C
                    .\end{align*}
                \item \begin{align*}
                        \int \frac{dx}{a^{2}+x^{2}} = \frac{1}{a}\tan^{-1}{\frac{x}{a}} + C
                    .\end{align*}
                \item \begin{align*}
                        \int \frac{dx}{x\sqrt{x^{2}-a^{2}}} = \frac{1}{\abs{a}}\sec^{-1}{\frac{\abs{x}}{a}} + C
                    .\end{align*}
            \end{enumerate}
    \end{itemize}

    % \pagebreak \bigbreak \noindent 
    % \subsection{Chapter 2 Key Terms / Ideas}
    % \bigbreak \noindent 
    % \begin{itemize}
    %     \item \textbf{Finding limits of integration for region between two functions}: Usually, we want our limits of integration to be the points where the functions intersect
    %     \item A \textbf{"complex region"} between curves usually refers to an area that is not easily described by a single, continuous function over the interval of interest.
    %     \item \textbf{compound regions} are regions bounded by the graphs of functions that cross one another
    %     \item \textbf{Cross-section:} The intersection of a plane and a solid object.
    %     \item a \textbf{cylinder} is a three-dimensional shape that has two parallel, congruent bases connected by a curved surface. The bases are usually circles, but they can be other shapes as well
    %     \item The line segment connecting the centers of the two bases is called the \textbf{"axis" of the cylinder.}
    %     \item \textbf{Slicing method:} A method of calculating the volume of a solid that involves cutting the solid into pieces, estimating the volume of each piece, then adding these estimates to arrive at an estimate of the total volume; as the number of slices goes to infinity, this estimate becomes an integral that gives the exact value of the volume.
    %     \begin{enumerate}
    %         \item Examine the solid and determine the shape of a cross-section of the solid. It is often helpful to draw a picture if one is not provided.
    %         \item Determine a formula for the area of the cross-section.
    %         \item Integrate the area formula over the appropriate interval to get the volume.
    %     \end{enumerate}
    %     \item \textbf{Solid of revolution:} A solid generated by revolving a region in a plane around a line in that plane.
    %     \item \textbf{Disk method:} A special case of the slicing method used with solids of revolution when the slices are disks.
    %     \item A \textbf{Washer (Annuli)} is a disk with holes in the center.
    %     \item \textbf{Washer method:} A special case of the slicing method used with solids of revolution when the slices are washers.
    %     \item \textbf{Method of cylindrical shells:} A method of calculating the volume of a solid of revolution by dividing the solid into nested cylindrical shells; this method is different from the methods of disks or washers in that we integrate with respect to the opposite variable.
    %     % \item A \textbf{cylinder} is defined as any solid that can be generated by translating a plane region along a line perpendicular to the region, called the \textbf{axis of the cylinder}.
    %      \item \textbf{Arc length:} The arc length of a curve can be thought of as the distance a person would travel along the path of the curve.
    %     \item \textbf{Surface area:} The surface area of a solid is the total area of the outer layer of the object; for objects such as cubes or bricks, the surface area of the object is the sum of the areas of all of its faces.
    %     % \item \textbf{Catenary:} A curve in the shape of the function \(y = a \cosh(x/a)\) is a catenary; a cable of uniform density suspended between two supports assumes the shape of a catenary.
    %     % \item \textbf{Center of mass:} The point at which the total mass of the system could be concentrated without changing the moment.
    %     % \item \textbf{Centroid:} The centroid of a region is the geometric center of the region; laminas are often represented by regions in the plane; if the lamina has a constant density, the center of mass of the lamina depends only on the shape of the corresponding planar region; in this case, the center of mass of the lamina corresponds to the centroid of the representative region.
    %     % \item \textbf{Density function:} A density function describes how mass is distributed throughout an object; it can be a linear density, expressed in terms of mass per unit length; an area density, expressed in terms of mass per unit area; or a volume density, expressed in terms of mass per unit volume; weight-density is also used to describe weight (rather than mass) per unit volume.
    %     % \item \textbf{Doubling time:} If a quantity grows exponentially, the doubling time is the amount of time it takes the quantity to double, and is given by \(\frac{\ln 2}{k}\).
    %     % \item \textbf{Exponential decay:} Systems that exhibit exponential decay follow a model of the form \(y = y_0 e^{-kt}\).
    %     % \item \textbf{Exponential growth:} Systems that exhibit exponential growth follow a model of the form \(y = y_0 e^{kt}\).
    %     % \item \textbf{Frustum:} A portion of a cone; a frustum is constructed by cutting the cone with a plane parallel to the base.
    %     % \item \textbf{Half-life:} If a quantity decays exponentially, the half-life is the amount of time it takes the quantity to be reduced by half. It is given by \(\frac{\ln 2}{k}\).
    %     % \item \textbf{Hooke's Law:} This law states that the force required to compress (or elongate) a spring is proportional to the distance the spring has been compressed (or stretched) from equilibrium; in other words, \(F = kx\), where \(k\) is a constant.
    %     % \item \textbf{Hydrostatic pressure:} The pressure exerted by water on a submerged object.
    %     % \item \textbf{Lamina:} A thin sheet of material; laminas are thin enough that, for mathematical purposes, they can be treated as if they are two-dimensional.
    %     % \item \textbf{Moment:} If \(n\) masses are arranged on a number line, the moment of the system with respect to the origin is given by \(M = \sum_{i=1}^{n} m_i x_i\); if, instead, we consider a region in the plane, bounded above by a function \(f(x)\) over an interval \([a, b]\), then the moments of the region with respect to the \(x\)- and \(y\)-axes are given by \(M_x = \rho \int_{a}^{b} \frac{[f(x)]^2}{2} dx\) and \(M_y = \rho \int_{a}^{b} x f(x) dx\), respectively.
    %     % \item \textbf{Symmetry principle:} The symmetry principle states that if a region \(R\) is symmetric about a line \(l\), then the centroid of \(R\) lies on \(l\).
    %     % \item \textbf{Theorem of Pappus for volume:} This theorem states that the volume of a solid of revolution formed by revolving a region around an external axis is equal to the area of the region multiplied by the distance traveled by the centroid of the region.
    %     % \item \textbf{Work:} The amount of energy it takes to move an object; in physics, when a force is constant, work is expressed as the product of force and distance.
    % \end{itemize}

    \pagebreak \bigbreak \noindent 
    \subsection{Chapter 2 Definitions and Theorems}
    \bigbreak \noindent 
    \begin{itemize}

        \item \textbf{Area between two curves, integrating on the x-axis}
            \begin{align}
                A = \int_{a}^{b} [f(x) - g(x)] \, dx
            \end{align}
            Where $f(x) \geq g(x)$
            \begin{align*}
                A = \int_{a}^{b}\ [g(x) - f(x)]\ dx
            .\end{align*}
            for $g(x) \geq f(x)$

        \item \textbf{Area between two curves, integrating on the y-axis}
            \begin{align}
                A = \int_{c}^{d} [u(y) - v(y)] \, dy
            \end{align}

        \item \textbf{Areas of compound regions}
            \begin{align*}
                \int_{a}^{b}\ \abs{f(x)-g(x)}\ dx 
            .\end{align*}
        \item \textbf{Area of complex regions}
            \begin{align*}
                \int_{a}^{b}\ f(x)\ dx + \int_{b}^{c}\ g(x)\ dx
            .\end{align*}
        \item \textbf{Slicing Method}
            \begin{align*}
                V(s) = \summation{n}{i=1}\ A(x_{i}^{*})\ \Delta x  = \int_{a}^{b}\ A(x)\ dx
            .\end{align*}
        \item \textbf{Disk Method along the x-axis}
            \begin{align}
                V = \int_{a}^{b} \pi [f(x)]^2 \, dx
            \end{align}

        \item \textbf{Disk Method along the y-axis}
            \begin{align}
                V = \int_{c}^{d} \pi [g(y)]^2 \, dy
            \end{align}

        \item \textbf{Washer Method along the x-axis}
            \begin{align}
                V = \int_{a}^{b} \pi [(f(x))^2 - (g(x))^2] \, dx
            \end{align}

        \item \textbf{Washer Method along the y-axis}
            \begin{align}
                V = \int_{c}^{d} \pi [(u(y))^2 - (v(y))^2] \, dy
            \end{align}

        \item \textbf{Radius if revolved around other line (Washer Method)}
            \begin{align*}
                If:\ x=-k\\
                Then:\ r = Function + k
            .\end{align*}
            \begin{align*}
                If:\ x=k\\
                Then:\ r = k - Function
            .\end{align*}

        \item \textbf{Method of Cylindrical Shells (x-axis)}
            \begin{align}
                V = \int_{a}^{b} 2\pi x f(x) \, dx
            \end{align}

        \item \textbf{Method of Cylindrical Shells (y-axis)}
            \begin{align}
                V = \int_{c}^{d} 2\pi y g(y) \, dy
            \end{align}

        \item \textbf{Region revolved around other line (method of cylindrical shells):}
            \begin{align*}
                If:\ x=-k \\
                Then:\ V = \int_{a}^{b}\ 2\pi (x+k)(f(x))\ dx
            .\end{align*}
            \begin{align*}
                If:\ x=k \\
                Then:\ V = \int_{a}^{b}\ 2\pi (k-x)(f(x))\ dx
            .\end{align*}
        \item \textbf{A Region of Revolution Bounded by the Graphs of Two Functions (method cylindrical shells)}
            \begin{align*}
                V = \int_{a}^{b}\ 2\pi x\left[f(x)-g(x)\right]\ dx
            .\end{align*}

        \item \textbf{Arc Length of a Function of x}
            \begin{align}
                \text{Arc Length} = \int_{a}^{b} \sqrt{1 + [f'(x)]^2} \, dx
            \end{align}

        \item \textbf{Arc Length of a Function of y}
            \begin{align}
                \text{Arc Length} = \int_{c}^{d} \sqrt{1 + [g'(y)]^2} \, dy
            \end{align}

        \item \textbf{Surface Area of a Function of x (Around x)}
            \begin{align}
                \text{Surface Area} = \int_{a}^{b} 2\pi f(x) \sqrt{1 + [f'(x)]^2} \, dx
            \end{align}

        \item \textbf{Surface Area of a Function of x (Around y)}
            \begin{align}
                \text{Surface Area} = \int_{a}^{b} 2\pi x \sqrt{1 + [f'(x)]^2} \, dx \\
                \text{Or: } \int_{a}^{b}\ 2\pi u(y)\sqrt{1+(u^{\prime}(y))^{2}}\ dy
            \end{align}

        \item \textbf{Natural logarithm function}
            \begin{align}
                \ln x = \int_{1}^{x} \frac{1}{t} \, dt\
            \end{align}

        \item \textbf{Exponential function}
            \begin{align}
                y = e^x, \quad \ln y = \ln(e^x) = x\
            \end{align}

        \item  \textbf{Logarithm Differentiation}
            \begin{align*}
                f^{\prime}(x) = f(x) \cdot \frac{d}{dx}\ln{\left(f^{\prime}(x)\right)}
            .\end{align*}
            \textbf{Note:} Use properties of logs before you differentiate whats inside the logarithm

            % \item \textbf{Mass of a one-dimensional object}
            % \begin{align}
            %     m = \int_{a}^{b} \rho(x) \, dx
            % \end{align}
            %
            % \item \textbf{Mass of a circular object}
            % \begin{align}
            %     m = \int_{0}^{r} 2\pi x \rho(x) \, dx
            % \end{align}
            %
            % \item \textbf{Work done on an object}
            % \begin{align}
            %     W = \int_{a}^{b} F(x) \, dx
            % \end{align}
            %
            % \item \textbf{Hydrostatic force on a plate}
            % \begin{align}
            %     F = \int_{a}^{b} \rho w(x) s(x) \, dx
            % \end{align}
            %
            % \item \textbf{Mass of a lamina}
            % \begin{align}
            %     m = \rho \int_{a}^{b} f(x) \, dx
            % \end{align}
            %
            % \item \textbf{Moments of a lamina}
            % \begin{align}
            %     M_x = \rho \int_{a}^{b} \frac{[f(x)]^2}{2} \, dx, \quad M_y = \rho \int_{a}^{b} x f(x) \, dx
            % \end{align}
            %
            % \item \textbf{Center of mass of a lamina}
            % \begin{align}
            %     \bar{x} = \frac{M_y}{m},\ \ \text{and}\ \  \bar{y} = \frac{M_x}{m}
            % \end{align}

    \end{itemize}


    % \pagebreak \bigbreak \noindent 
    % \subsection{Chapter 3 Key Terms}
    % \bigbreak \noindent 
    % \begin{itemize}
    %     \item \textbf{integration by parts}: a technique of integration that allows the exchange of one integral for another using the formula 
    %     \item \textbf{integration table}: a table that lists integration formulas.
    %     \item \textbf{power reduction formula}: a rule that allows an integral of a power of a trigonometric function to be exchanged for an integral involving a lower power.
    %     \item \textbf{trigonometric integral}: an integral involving powers and products of trigonometric functions.
    %     \item \textbf{trigonometric substitution}: an integration technique that converts an algebraic integral containing expressions of the form \( \sqrt{a^2 - x^2} \), \( \sqrt{a^2 + x^2} \), or \( \sqrt{x^2 - a^2} \) into a trigonometric integral.
    %     \item \textbf{partial fraction decomposition}: a technique used to break down a rational function into the sum of simple rational functions.
    %     \item \textbf{improper integral}: an integral over an infinite interval or an integral of a function containing an infinite discontinuity on the interval; an improper integral is defined in terms of a limit. The improper integral converges if this limit is a finite real number; otherwise, the improper integral diverges.
    %      % \item \textbf{absolute error}: if \( B \) is an estimate of some quantity having an actual value of \( A \), then the absolute error is given by \( |A-B| \).
    %     % \item \textbf{computer algebra system (CAS)}: technology used to perform many mathematical tasks, including integration.
    %     % \item \textbf{midpoint rule}: a rule that uses a Riemann sum of the form 
    %     % \item \textbf{numerical integration}: the variety of numerical methods used to estimate the value of a definite integral, including the midpoint rule, trapezoidal rule, and Simpson’s rule.
    %     % \item \textbf{relative error}: error as a percentage of the absolute value, given by 
    %     % \item \textbf{Simpson’s rule}: a rule that approximates \( \int_{a}^{b} f(x) \, dx \) using the integrals of a piecewise quadratic function. The approximation \( S_n \) to \( \int_{a}^{b} f(x) \, dx \) is given by 
    %     % \item \textbf{trapezoidal rule}: a rule that approximates \( \int_{a}^{b} f(x) \, dx \) using trapezoids.
    % \end{itemize}
    %
    \pagebreak \bigbreak \noindent 
    \subsection{Chapter 3 Definitions and Theorems}
    \bigbreak \noindent 
    \begin{itemize}
        \item \textbf{Integration by parts formula} 
            \begin{align*}
                \int u \, dv &= uv - \int v \, du 
            .\end{align*}
        \item \textbf{Integration by parts for definite integral}
            \begin{align*}
                \int_{a}^{b} u \, dv &= uv\big|_{a}^{b} - \int_{a}^{b} v \, du
            \end{align*}
        \item \textbf{To integrate products involving  sin(ax), sin(bx), cos(ax), and  cos(bx), use the substitutions:}
            \begin{itemize}
                \item \textbf{Sine Products}
                    \begin{align*}
                        \sin(ax) \sin(bx) &= \frac{1}{2} \cos((a-b)x) - \frac{1}{2} \cos((a+b)x)
                    \end{align*}

                \item \textbf{Sine and Cosine Products}
                    \begin{align*}
                        \sin(ax) \cos(bx) &= \frac{1}{2} \sin((a-b)x) + \frac{1}{2} \sin((a+b)x)
                    \end{align*}

                \item \textbf{Cosine Products}
                    \begin{align*}
                        \cos(ax) \cos(bx) &= \frac{1}{2} \cos((a-b)x) + \frac{1}{2} \cos((a+b)x)
                    \end{align*}

                \item \textbf{Power Reduction Formula (sine)}
                    \begin{align*}
                        &\int \sin^{n}{x}\ dx = -\frac{1}{n}\sin^{n-1}{x}\cos{x} + \frac{n-1}{n}\int \sin^{n-2}{x}\ dx \\
                        &\int_{0}^{\frac{\pi}{2}}\ \sin^{n}{x}\ dx = \frac{n-1}{n}\int_{0}^{\frac{\pi}{2}}\ \sin^{n-2}{x}\ dx
                    .\end{align*}
                \item \textbf{Power Reduction Formula (cosine)}
                    \begin{align*}
                        &\int \cos^{n}{x}\ dx = \frac{1}{n}\cos^{n-1}{x}\sin{x} + \frac{n-1}{n}\int \cos^{n-2}{x}\ dx \\
                        &\int_{0}^{\frac{\pi}{2}}\ \cos^{n}{x}\ dx = \frac{n-1}{n}\int_{0}^{\frac{\pi}{2}}\ \cos^{n-2}{x}\ dx
                    .\end{align*}
                \item \textbf{Power Reduction Formula (secant)}
                    \begin{align*}
                        \int \sec^{n}{x}\ dx &= \frac{1}{n-1}\sec^{n-1}{x}\sin{x}+\frac{n-2}{n-1}\int \sec^{n-2}{x}\ dx \\
                        \int \sec^{n}{x}\ dx &= \frac{1}{n-1}\sec^{n-2}{x}\tan{x}+\frac{n-2}{n-1}\int \sec^{n-2}{x}\ dx
                    \end{align*}

                \item \textbf{Power Reduction Formula (tangent)}
                    \begin{align*}
                        \int \tan^n x \, dx &= \frac{1}{n-1} \tan^{n-1}x - \int \tan^{n-2}x \, dx
                    \end{align*}
            \end{itemize}
        \item \textbf{Trigonometric Substitution}
            \begin{itemize}
                \item $\sqrt{a^{2} - x^{2}}$ use $x =a\sin{\theta }$ with domain restriction $\bigg[-\frac{\pi}{2},\frac{\pi}{2}\bigg] $
                \item $\sqrt{a^{2} + x^{2}}$ use $x=a\tan{\theta}$ with domain restriction $\left(-\frac{\pi}{2}, \frac{\pi}{2}\right)$
                \item $\sqrt{x^{2} - a^{2}}$ use $x =a\sec{\theta}$ with domain restriction $\bigg[0,\frac{\pi}{2}\bigg) \cup \bigg[\pi,\frac{3\pi}{2}\bigg)$ 
            \end{itemize}

        \item \textbf{Steps for fraction decomposition}
            \begin{enumerate}
                \item Ensure $deg(Q) < deg(P)$, if not, long divide
                \item Factor denominator
                \item Split up fraction into factors
                \item Multiply through to clear denominator
                \item Group terms and equalize
                \item Solve for constants
                \item Plug constants into split up fraction
                \item Compute integral
            \end{enumerate}

        \item \textbf{Solving for constants}
            Either:
            \begin{itemize}
                \item Plug in values (often the roots)
                \item Equalize 
            \end{itemize}
        \item \textbf{Cases for partial fractions}
            \begin{itemize}
                \item Non repeated linear factors
                \item Repeated linear factors
                \item Nonfactorable quadratic factors
            \end{itemize}
        \item \textbf{Midpoint rule}
            \begin{align*}
                M_{n} = \summation{n}{i=1}\ f(m_{i})\ \Delta x 
            .\end{align*}
        \item \textbf{Absolute error}
            \begin{align*}
                err = \bigg|\text{Actual} - \text{Estimated}\bigg|
            .\end{align*}
        \item \textbf{Relative error}
            \begin{align*}
                err = \bigg|\frac{\text{Actual} - \text{Estimated}}{\text{Actual}}\bigg| \cdot 100\%
            .\end{align*}
        \item \textbf{Error upper bound for midpoint rule}
            \begin{align*}
                E_{M} \leq \frac{M(b-a)^3}{24n^2}
            \end{align*}
            Where $M$ is the maximum value of the second derivative
        \item \textbf{Trapezoidal rule}
            \begin{align*}
                T_n \frac{1}{2} \Delta x \left( f(x_0) + 2f(x_1) + 2f(x_2) + \cdots + 2f(x_{n-1}) + f(x_n) \right)
            \end{align*}
        \item \textbf{Error upper bound for trapezoidal rule}
            \begin{align*}
                E_{T} \leq \frac{M(b-a)^3}{12n^2}
            \end{align*}
            Where $M$ is the maximum value of the second derivative
        \item \textbf{Simpson’s rule}
            \begin{align*}
                S_n = \frac{\Delta x}{3} \left( f(x_0) + 4f(x_1) + 2f(x_2) + 4f(x_3) + 2f(x_4) + 4f(x_5) + \cdots + 2f(x_{n-2}) + 4f(x_{n-1}) + f(x_n) \right)
            \end{align*}
        \item \textbf{Error upper bound for Simpson’s rule}
            \begin{align*}
                E_{S} \leq \frac{M(b-a)^5}{180n^4}
            \end{align*}
            Where $M$ is the maximum value of the fourth derivative
        \item \textbf{Finding $n$ with error bound functions}
            \begin{enumerate}
                \item Find $f^{\prime\prime}(x)$
                \item Find maximum values of $f^{\prime\prime}(x)$ in the interval
                \item Plug into error bound function 
                \item Set value $\leq$ desired accuracy (ex: 0.01)
                \item Solve: 
                \item If we were to truncate, we would use the ceil function $\ceil*{n}$ DO NOT FLOOR
            \end{enumerate}
        \item \textbf{Improper integrals (Infinite interval)}
            \begin{itemize}
                \item $\int_{a}^{+\infty}\ f(x)\ dx  = \lim\limits_{t \to +\infty}{\int_{a}^{t}\ f(x)\ dx}$  
                \item $\int_{-\infty}^{b}\ f(x)\ dx = \lim\limits_{t \to -\infty}{\int_{t}^{b}\ f(x)\ dx}$ 
                \item $\int_{-\infty}^{+\infty}\ f(x)\ dx = \int_{-\infty}^{0}\ f(x)\ dx + \int_{0}^{+\infty}\ f(x)\ dx$
            \end{itemize}
        \item \textbf{Improper integral (discontinuous)}
            \begin{itemize}
                \item Let $f(x)$ be continuous on $[a,b)$, then;
                    \begin{align*}
                        \int_{a}^{b}\ f(x)\ dx = \lim\limits_{t \to b^{-}}{\int_{a}^{t}\ f(x)\ dx}\
                    .\end{align*}
                \item Let $f(x)$ be continuous on $(a,b]$, then;
                    \begin{align*}
                        \int_{a}^{b}\ f(x)\ dx = \lim\limits_{t \to b^{+}}{\int_{t}^{b}\ f(x)\ dx}\
                    .\end{align*}
                    In each case, if the limit exists, then the improper integral is said to converge. If the limit does not exist, then the improper integral is said to diverge.
                \item Let $f(x)$ be continuous on $[a,b]$ except at a point $c \in (a,b)$, then;
                    \begin{align*}
                        \int_{a}^{b}\ f(x)\ dx = \int_{a}^{c}\ f(x)\ dx  +\int_{c}^{b}\ f(x)\ dx
                    .\end{align*}
                    If either integral diverges, then $\int_{a}^{b}\ f(x)\ dx $ diverges
            \end{itemize}
        \item \textbf{Comparison theorem}
            Let $f(x)$ and $g(x)$ be continuous over $[a,+\infty)$. Assume that $0 \leq f(x) \leq g(x)$ for $x \geq a$.
            \begin{itemize}
                \item If $\int_a^{+\infty} f(x) \, dx = \lim_{t \to +\infty} \int_a^t f(x) \, dx = +\infty$,  \\
                    then $\int_a^{+\infty} g(x) \, dx = \lim_{t \to +\infty} \int_a^t g(x) \, dx = +\infty$.
                \item If $\int_a^{+\infty} g(x) \, dx = \lim_{t \to +\infty} \int_a^t g(x) \, dx = L$, where $L$ is a real number,  \\
                    then $\int_a^{+\infty} f(x) \, dx = \lim_{t \to +\infty} \int_a^t f(x) \, dx = M$ for some real number $M \leq L$.
            \end{itemize}
            \pagebreak \bigbreak \noindent 
        \item \textbf{P-integrals}
            \begin{itemize}
                \item $\int_{0}^{+\infty} \frac{1}{x^{p}}\ dx =  
                    \begin{cases}
                        \frac{1}{p-1} & \text{if } p>1 \\
                        +\infty & \text{if } p \leq 1
                    \end{cases}$
                \item $\int_{0}^{1} \frac{1}{x^p}\ dx =    
                    \begin{cases}
                        \frac{1}{1-p} & \text{if } p<1 \\
                        +\infty & \text{if } p \geq 1
                    \end{cases}$
                \item $\int_{a}^{+\infty} \frac{1}{x^{p}}\ dx =  
                    \begin{cases}
                        \frac{a^{1-p}}{p-1} & \text{if } p>1 \\
                        +\infty & \text{if } p \leq 1
                    \end{cases}$
                \item $\int_{0}^{a} \frac{1}{x^p}\ dx =    
                    \begin{cases}
                        \frac{a^{1-p}}{1-p} & \text{if } p<1 \\
                        +\infty & \text{if } p \geq 1
                    \end{cases}$
            \end{itemize}
        \item \textbf{Bypass L'Hospital's Rule}
            \begin{align*}
                \ln{(\ln{(x)})},\ \ln{(x)},\ \cdots\ x^{\frac{1}{100}},\ x^{\frac{1}{3}},\ \sqrt{x},\ 1,\ x^{2},\ x^{3},\ \cdots\ e^{x},\ e^{2x},\ e^{3x},\ \cdots,\ e^{x^{2}},\ \cdots\ e^{e^{x}}
            .\end{align*}
            Essentially what it means is things on the right grow faster than things on the left. Thus, if we have say:
            \begin{align*}
                \lim\limits_{x \to \infty}{\frac{x^{2}}{e^{2x}}} 
            .\end{align*}
            We can be sure that it is zero. Because this is $x^{2}\cdot e^{-2x}$. If we take  $ \lim\limits_{x \to \infty}{x^{2}e^{-2x}}$, we get $\infty \cdot 0$. As we see by the sequence $e^{-2x}$ overrules $x^{2}$ and we can say the limit is zero.
            % \item \textbf{something to consider for limits}: Suppose we have $f:\ A \rightarrow B:\ x \mapsto f(x)$. It would not be meaninful to consider some $ \lim\limits_{x \to b+n}{f(x)}$ for $n>0 $. Thus we shall conclude that the limit is undefined. For example, the domain of arcsine is $[-1,1]$, thus any $\lim\limits_{x \to a}{f(x)}$ for $(-\infty,-1)\cup (1,\infty)$ would be undefined
        \item \textbf{Consideration for Limits}: Let \(f: A \rightarrow B\) be a function defined by \(x \mapsto f(x)\). If a point \(c\) lies outside the domain \(A\), then the expression \(\lim\limits_{x \to c} f(x)\) is not meaningful, and we classify this limit as undefined. For instance, the function arcsine has a domain of \([-1,1]\). Therefore, limits like \(\lim\limits_{x \to a} \sin^{-1}(x)\) where \(a \notin [-1,1]\) are undefined.
        \item \textbf{Why does}
            \begin{align*}
            &\lim\limits_{x \to 2}{\tan^{-1}{\frac{1}{x-2}}} 
        .\end{align*}
        \begin{minipage}[]{0.47\textwidth}
            \begin{align*}
                &=\lim\limits_{x \to 2^{-}}{\tan^{-1}{\frac{1}{x-2}}} \\
                &= \lim\limits_{x \to -\infty}{\tan^{-1}{x}} \\
                &= -\pi/2
            .\end{align*}
        \end{minipage}
        \begin{minipage}[]{0.47\textwidth}
            \begin{align*}
                &=\lim\limits_{x \to 2^{+}}{\tan^{-1}{\frac{1}{x-2}}} \\
                &=\lim\limits_{x \to +\infty}{\tan^{-1}{x}} \\
                &=\frac{\pi}{2}
            .\end{align*}
        \end{minipage}
\end{itemize}

% \pagebreak \bigbreak \noindent 
% \subsection{Chapter 5 Key Terms}
% \bigbreak \noindent 
% \begin{itemize}
% 
% \item Alternating series: 
% \[
% \text{A series of the form } \sum_{n=1}^{\infty} (-1)^{n+1} b_n \text{ or } \sum_{n=1}^{\infty} (-1)^n b_n, \text{ where } b_n \geq 0, \text{ is called an alternating series.}
% \]
% 
% \item Alternating series test: 
% \[
% \text{For an alternating series of either form, if } b_{n+1} \leq b_n \text{ for all integers } n \geq 1 \text{ and } b_n \to 0, \text{ then an alternating series converges.}
% \]
% 
% \item Arithmetic sequence: 
% \[
% \text{A sequence in which the difference between every pair of consecutive terms is the same is called an arithmetic sequence.}
% \]
% 
% \item Bounded above: 
% \[
% \text{A sequence } \{a_n\} \text{ is bounded above if there exists a constant } M \text{ such that } a_n \leq M \text{ for all positive integers } n.
% \]
% 
% \item Bounded below: 
% \[
% \text{A sequence } \{a_n\} \text{ is bounded below if there exists a constant } M \text{ such that } M \leq a_n \text{ for all positive integers } n.
% \]
% 
% \item Bounded sequence: 
% \[
% \text{A sequence } \{a_n\} \text{ is bounded if there exists a constant } M \text{ such that } |a_n| \leq M \text{ for all positive integers } n.
% \]
% 
% 
% \item Convergence of a series: 
% \[
% \text{A series converges if the sequence of partial sums for that series converges.}
% \]
% 
% \item Convergent sequence: 
% \[
% \text{A convergent sequence is a sequence } \{a_n\} \text{ for which there exists a real number } L \text{ such that } a_n \text{ is arbitrarily close to } L \text{ as long as } n \text{ is sufficiently large.}
% \]
% 
% \item Divergence of a series: 
% \[
% \text{A series diverges if the sequence of partial sums for that series diverges.}
% \]
% 
% \item Divergence test: 
% \[
% \text{If } \lim_{n \to \infty} a_n \neq 0, \text{ then the series } \sum_{n=1}^{\infty} a_n \text{ diverges.}
% \]
% 
% \item Divergent sequence: 
% \[
% \text{A sequence that is not convergent is divergent.}
% \]
% 
% \item Explicit formula: 
% \[
% \text{A sequence may be defined by an explicit formula such that } a_n = f(n).
% \]
% 
% \item Geometric sequence: 
% \[
% \text{A sequence } \{a_n\} \text{ in which the ratio } \frac{a_{n+1}}{a_n} \text{ is the same for all positive integers } n \text{ is called a geometric sequence.}
% \]
% 
% \item Geometric series: 
% \[
% \text{A geometric series is a series that can be written in the form } \sum_{n=1}^{\infty} ar^{n-1} = a + ar + ar^2 + ar^3 + \cdots.
% \]
% 
% \item Harmonic series: 
% \[
% \text{The harmonic series takes the form } \sum_{n=1}^{\infty} \frac{1}{n} = 1 + \frac{1}{2} + \frac{1}{3} + \cdots.
% \]
% 
% \item Index variable: 
% \[
% \text{The subscript used to define the terms in a sequence is called the index.}
% \]
% 
% \item Infinite series: 
% \[
% \text{An infinite series is an expression of the form } a_1 + a_2 + a_3 + \cdots = \sum_{n=1}^{\infty} a_n.
% \]
% 
% \item Integral test: 
% \[
% \text{For a series } \sum_{n=1}^{\infty} a_n \text{ with positive terms } a_n, \text{ if there exists a continuous, decreasing function } f \text{ such that } f(n) = a_n \text{ for all positive integers } n, \text{ then } \sum_{n=1}^{\infty} a_n \text{ and } \int_{1}^{\infty} f(x) \, dx \text{ either both converge or both diverge.}
% \]
% 
% \item Limit comparison test: 
% \[
% \text{Suppose } a_n, b_n \geq 0 \text{ for all } n \geq 1. \text{ If } \
% \]
% \end{itemize}

\pagebreak \bigbreak \noindent 
\subsection{Chapter 5 Definitions and Theorems}
\bigbreak \noindent 
\begin{itemize}
    \item \textbf{Sequence notation}
        \begin{align*}
            \{a_{n}\}_{n=1}^{\infty},\ \text{or simply } \{a_{n}\}
        .\end{align*}
    \item \textbf{Sequence notation (ordered list)}
        \begin{align*}
            a_{1},\ a_{2},\ a_{3},\ \cdots,\ a_{n},\ \cdots
        .\end{align*}
    \item \textbf{Arithemetic Sequence Difference}
        \begin{align*}
            d = a_{n} - a_{n-1}
        .\end{align*}
    \item \textbf{Arithmetic sequence (common difference between subsequent terms) general form}
        \begin{align*}
            &\text{Index starting at 0}:\ a_{n} = a + nd \\
            &\text{Index starting at 1}:\ a_{n} = a + (n-1)d \\
        .\end{align*}
    \item \textbf{Arithmetic sequence (common difference between subsequent terms) recursive form}
        \begin{align*}
            a_{n} = a_{n-1} + d
        .\end{align*}
    \item \textbf{Sum of arithmetic sequence}
        \begin{align*}
            &S_{n} = \frac{n}{2}\left[a + a_{n}\right] \\
            &S_{n} = \frac{n}{2}\left[2a + (n-1)d\right]
        .\end{align*}
    \item \textbf{Geometric sequence form common ratio}
        \begin{align*}
            r = \frac{a_{n}}{a_{n-1}}
        .\end{align*}
    \item \textbf{Geometric sequence general form}
        \begin{align*}
            &a_{n} = ar^{n}\ \text{(Index starting at 0)} \\
            &a_{n} = a^{n+1} \text{(index starting at 0 and a=r)} \\
            &a_{n} = ar^{n-1}\ \text{(Index starting at 1)} \\
            &a_{n} = a^{n} \text{(index starting at 1 and a=r)}
        .\end{align*}
    \item \textbf{Geometric sequence recursive form}
        \begin{align*}
            &a_{n} = ra_{n-1}
        .\end{align*}
    \item \textbf{Sum of geometric sequence (finite terms)}
        \begin{align*}
            S_{n} = \frac{a(1-r^{n})}{1-r}\ \quad r\ne 1
        .\end{align*}
    \item \textbf{Convergence / Divergence}: If 
        \begin{align*}
            \lim\limits_{n \to +\infty}{a_{n}} = L
        .\end{align*}
        We say that the sequence converges, else it diverges
    \item \textbf{Formal definition of limit of sequence}
        \begin{align*}
            \lim\limits_{n \to +\infty}{a_{n}= L} \iff \forall \varepsilon > 0, \exists N \in \mathbb{Z} \mid \abs{a_{n} - L} < \varepsilon,\ \text{if } n \geq n
        .\end{align*}
        Then we can say 
        \begin{align*}
            \lim\limits_{n \to +\infty}{a_{n} = L}\ \text{or } a_{n} \rightarrow L 
        .\end{align*}
    \item \textbf{Limit of a sequence defined by a function}:         Consider a sequence \( \{a_n\} \) such that \( a_n = f(n) \) for all \( n \geq 1 \). If there exists a real number \( L \) such that
        \[
            \lim_{{x \to \infty}} f(x) = L,
        \]
        then \( \{a_n\} \) converges and
        \[
            \lim_{{n \to \infty}} a_n = L.
        \]
    \item \textbf{Algebraic limit laws}:
        Given sequences \( \{a_n\} \) and \( \{b_n\} \) and any real number \( c \), if there exist constants \( A \) and \( B \) such that \( \lim_{{n \to \infty}} a_n = A \) and \( \lim_{{n \to \infty}} b_n = B \), then
        \begin{itemize}
            \item \( \lim_{{n \to \infty}} c = c \)
            \item \( \lim_{{n \to \infty}} c a_n = c \lim_{{n \to \infty}} a_n = cA \)
            \item \( \lim_{{n \to \infty}} (a_n \pm b_n) = \lim_{{n \to \infty}} a_n \pm \lim_{{n \to \infty}} b_n = A \pm B \)
            \item \( \lim_{{n \to \infty}} (a_n \cdot b_n) = (\lim_{{n \to \infty}} a_n) \cdot (\lim_{{n \to \infty}} b_n) = A \cdot B \)
            \item \( \lim_{{n \to \infty}} \frac{a_n}{b_n} = \frac{\lim_{{n \to \infty}} a_n}{\lim_{{n \to \infty}} b_n} = \frac{A}{B} \), provided \( B \neq 0 \) and each \( b_n \neq 0 \).
        \end{itemize}
    \item \textbf{Continuous Functions Defined on Convergent Sequences}:
        Consider a sequence \( \{a_n\} \) and suppose there exists a real number \( L \) such that the sequence \( \{a_n\} \) converges to \( L \). Suppose \( f \) is a continuous function at \( L \). Then there exists an integer \( N \) such that \( f \) is defined at all values \( a_n \) for \( n \geq N \), and the sequence \( \{f(a_n)\} \) converges to \( f(L) \).
    \item \textbf{Squeeze Theorem for Sequences}:           Consider sequences \( \{a_n\} \), \( \{b_n\} \), and \( \{c_n\} \). Suppose there exists an integer \( N \) such that
        \[ a_n \leq b_n \leq c_n \text{ for all } n \geq N. \]
        If there exists a real number \( L \) such that
        \[ \lim_{{n \to \infty}} a_n = L = \lim_{{n \to \infty}} c_n, \]
        then \( \{b_n\} \) converges and \( \lim_{{n \to \infty}} b_n = L \)
    \item \textbf{Bounded above}:           A sequence \( \{a_n\} \) is bounded above if there exists a real number \( M \) such that
        \[ a_n \leq M \]
        for all positive integers \( n \).
    \item \textbf{Bounded below}:
        A sequence \( \{a_n\} \) is bounded below if there exists a real number \( M \) such that
        \[ M \leq a_n \]
        for all positive integers \( n \).
    \item \textbf{Bounded}:
        A sequence \( \{a_n\} \) is a bounded sequence if it is bounded above and bounded below. 
    \item \textbf{Unbounded}:
        If a sequence is not bounded, it is an unbounded sequence.
    \item \textbf{If a sequence  $\{a_{n}\} $ converges, then it is bounded.}
    \item \textbf{Increasing sequence}: A sequence \( \{a_n\} \) is increasing for all \( n \geq n_0 \) if
        \[ a_n \leq a_{n+1} \text{ for all } n \geq n_0. \]
    \item \textbf{Decreasing sequence}: A sequence \( \{a_n\} \) is decreasing for all \( n \geq n_0 \) if
        \[ a_n \geq a_{n+1} \text{ for all } n \geq n_0. \]
    \item \textbf{Monotone sequence}: A sequence \( \{a_n\} \) is a \textbf{monotone sequence} for all \( n \geq n_0 \) if it is increasing for all \( n \geq n_0 \) or decreasing for all \( n \geq n_0 \)
    \item \textbf{Monotone Convergence Theorem}:         If \( \{a_n\} \) is a bounded sequence and there exists a positive integer \( n_0 \) such that \( \{a_n\} \) is monotone for all \( n \geq n_0 \), then \( \{a_n\} \) converges.
    \item \textbf{Infinite Series form:}
        \begin{align*}
            \sum_{n=1}^{\infty} a_n = a_1 + a_2 + a_3 + \cdots.
        .\end{align*}
    \item \textbf{Partial sum ($k^{th}$ partial sum)}
        \begin{align*}
            S_k = \sum_{n=1}^{k} a_n = a_1 + a_2 + a_3 + \cdots + a_k
        .\end{align*}
    \item \textbf{Convergence of infinity series notation}
        \bigbreak \noindent 
        For a series, say...
        \begin{align*}
            \summation{\infty}{n=1}\ a_{n}\ 
        .\end{align*}
        its convergence is determined by the limit of its sequence of partial sums. Specifically, if
        \begin{align*}
            \lim\limits_{n \to +\infty}{S_{n}} = S \rightarrow \summation{\infty}{n=1}\ a_{n}\ = S 
        .\end{align*}
    \item \textbf{Harmonic series}
        \begin{align*}
            \summation{\infty}{n=1}\ \frac{1}{n}  =  \frac{1}{2} + \frac{1}{3} + \frac{1}{4} + \ldots\ 
        .\end{align*}
        Which diverges to $+\infty$
    \item \textbf{Algebraic Properties of Convergent Series}
        Let $ \sum_{n=1}^{\infty} a_n$ and $\sum_{n=1}^{\infty} b_n$ be convergent series. Then the following algebraic properties hold:
        \begin{enumerate}
            \item The series 
                $\sum_{n=1}^{\infty} (a_n + b_n)$ converges and 
                \begin{align*}
                    \sum_{n=1}^{\infty} (a_n + b_n) = \sum_{n=1}^{\infty} a_n + \sum_{n=1}^{\infty} b_n. \quad \text{(Sum Rule)}
                .\end{align*}
            \item The series $\sum_{n=1}^{\infty} (a_n - b_n)$ converges and 
                \begin{align*}
                    \sum_{n=1}^{\infty} (a_n - b_n) = \sum_{n=1}^{\infty} a_n - \sum_{n=1}^{\infty} b_n. \quad \text{(Difference Rule)}
                .\end{align*}
            \item For any real number \( c \), the series $\sum_{n=1}^{\infty} c a_n$ converges and 
                \begin{align*}
                    \sum_{n=1}^{\infty} c a_n = c \sum_{n=1}^{\infty} a_n. \quad \text{(Constant Multiple Rule)}
                .\end{align*}
        \end{enumerate}

    \item \textbf{Geometric series convergence or divergence: }
        \begin{align*}
            \summation{\infty}{n=1}\ ar^{n-1} \  = \quad \quad 
            \begin{cases}
                \frac{a}{1-r} & \text{if }  \abs{r} < 1\\
                diverges & \text{if }  \abs{r} \geq 1
            \end{cases}
        .\end{align*}

    \item \textbf{Divergence test}: In the context of sequences, if $\lim_{{n \to \infty}} a_n = c \neq 0$ or the limit does not exist, then the series $\sum_{{n=1}}^{\infty} a_n$ is said to diverge. The converse is not true.
        \bigbreak \noindent 
        Because:
        \begin{align*}
            \lim_{k \to \infty} a_k = \lim_{k \to \infty} (S_k - S_{k-1}) = \lim_{k \to \infty} S_k - \lim_{k \to \infty} S_{k-1} = S - S = 0.
        .\end{align*}
    \item \textbf{Integral Test Prelude}:
        for any integer $k$, the $k$th partial sum $S_k$ satisfies
        \begin{align*}
            S_k = a_1 + a_2 + a_3 + \cdots + a_k < a_1 + \int_{1}^{k} f(x) \, dx < a_1 + \int_{1}^{\infty} f(x) \, dx.
        .\end{align*}
        and
        \begin{align*}
            S_k = a_1 + a_2 + a_3 + \cdots + a_k > \int_{1}^{k+1} f(x) \, dx.
        .\end{align*}

    \item \textbf{Intgeral test}
        Suppose  $\summation{\infty}{n=1}\ a_{n}\  $ is a series with positive terms  $a_{n}$ Suppose there exists a function  $f $
        and a positive integer  $N$ 
        such that the following three conditions are satisfied:
        \begin{enumerate}
            \item \( f \) positive, continuous, and decreasing on $[N,\infty)$
            \item \( f(n) = a_n \) for all integers \( n \geq N \), $N \in \mathbb{Z^{+}} $
        \end{enumerate}
        \begin{align*}
            \text{Then the series} \sum_{n=1}^{\infty} a_n \text{ and the improper integral} \int_{N}^{\infty} f(x) \, dx \text{ either both converge or both diverge.}
        .\end{align*}

    \item \textbf{P-series}
        $\forall p \in \mathbb{R}$, the series 
        \begin{align*}
            \summation{\infty}{n=1}\ \frac{1}{n^{P}}\ 
        .\end{align*}
        Is called a \textbf{p-series}. Furthermore, 
        \begin{align*}
            \sum_{n=1}^{\infty} \frac{1}{n^p} \begin{cases}
                \text{converges if } p>1 \\
                \text{diverges if } p \leq 1.
            \end{cases}
        .\end{align*}

    \item \textbf{P-series extended}
        \begin{align*}
            \summation{\infty}{n=2}\ \frac{1}{n\ln{(n)}^{p}}\ 
            \begin{cases}
                \text{converges if } p>1 \\
                \text{diverges if } p \leq 1.
            \end{cases}
        .\end{align*}
    \item \textbf{Remainder estimate for the integral test}
        Suppose \( \sum_{n=1}^{\infty} a_n \)
        is a convergent series with positive terms. Suppose there exists a function \( f \) and a positive integer $M$
        satisfying the following three conditions:
        \begin{enumerate}
            \item \( f \) is positive, decreasing, and continuous on $[M,\infty)$
            \item \( f(n) = a_n \) for all integers \( n \geq M \).
        \end{enumerate}
        Let \( S_N \) be the \( N \)th partial sum of \( \sum_{n=1}^{\infty} a_n \).
        For all positive integers \( N \),
        \[
            S_N + \int_{N+1}^{\infty} f(x) \, dx < \sum_{n=1}^{\infty} a_n < S_N + \int_{N}^{\infty} f(x) \, dx.
        \]
        In other words, the remainder \( R_N = \sum_{n=1}^{\infty} a_n - S_N = \sum_{n=N+1}^{\infty} a_n \)
        satisfies the following estimate:
        \[
            \int_{N+1}^{\infty} f(x) \, dx < R_N < \int_{N}^{\infty} f(x) \, dx.
        \]
        This is known as the remainder estimate 
        \bigbreak \noindent 
        To find a value of $N$ such that we are withing a desired margin of error, Since we know $R_{n} < \int_{N}^{\infty}\ f(x)\ dx $. Simply compute the improper integral and set the result < the desired error to solve for $N$
    \item \textbf{Find $a_{n}$ given the expression for the partial sum}
        \begin{align*}
            a_{n} = S_{n} - S_{n-1}
        .\end{align*}
    \item \textbf{telescoping series}: Telescoping series are a type of series where each term cancels out a part of another term, leaving only a few terms that do not cancel. When you sum the series, most of the terms collapse or "telescope," which simplifies the calculation of the sum. Here are some key points and generalizations you can note about telescoping series:
        \begin{itemize}
            \item Partial Fraction Decomposition
            \item Cancellation Pattern: In a telescoping series, look for a pattern where a term in one fraction will cancel out with a term in another fraction.
            \item Write out Terms
            \item What is left is $S_{n}$, thus the sum of the series is the $\lim\limits_{n \to \infty}{S_{n}} $
        \end{itemize}
        Try: 
        \begin{align*}
            \summation{\infty}{n=2}\ \frac{1}{n^{2}-1}\ 
        .\end{align*}
        Hint, its not only the first and last terms cancel, we also have a $\frac{\frac{1}{2}}{n}$, when $a_{n-1}$: Answer is $\frac{3}{4}$
        \pagebreak 
    \item \textbf{Comparison test for series}
        \begin{enumerate}
            \item Suppose there exists an integer \( N \) such that \( 0 \leq a_n \leq b_n \) for all \( n \geq N \). If \( \sum_{n=1}^{\infty} b_n \) converges, then \( \sum_{n=1}^{\infty} a_n \) converges. 
            \item  Suppose there exists an integer \( N \) such that \( a_n \geq b_n \geq 0 \) for all \( n \geq N \). If \( \sum_{n=1}^{\infty} b_n \) diverges, then \( \sum_{n=1}^{\infty} a_n \) diverges.
        \end{enumerate}
    \item \textbf{Limit Comparison Test}
        Let \( a_n, b_n \geq 0 \) for all \( n \geq 1 \).
        \begin{itemize}
            \item If \( \lim_{n \to \infty} \frac{a_n}{b_n} = L \neq 0 \), then \( \sum_{n=1}^{\infty} a_n \) and \( \sum_{n=1}^{\infty} b_n \) both converge or both diverge.
            \item If \( \lim_{n \to \infty} \frac{a_n}{b_n} = 0 \) and \( \sum_{n=1}^{\infty} b_n \) converges, then \( \sum_{n=1}^{\infty} a_n \) converges.
            \item If \( \lim_{n \to \infty} \frac{a_n}{b_n} = \infty \) and \( \sum_{n=1}^{\infty} b_n \) diverges, then \( \sum_{n=1}^{\infty} a_n \) diverges.
        \end{itemize}
        \textbf{Note:} Note that if $\frac{a_n}{b_n} \to 0$ and $\sum_{n=1}^{\infty} b_n$ diverges, the limit comparison test gives no information. Similarly, if $\frac{a_n}{b_n} \to \infty$ and $\sum_{n=1}^{\infty} b_n$ converges, the test also provides no information. 
        \bigbreak \noindent 
        Consider the series 
        \begin{align*}
            \summation{\infty}{n=1}\ \frac{n^{4} + 6}{n^{5} + 4}\ 
        .\end{align*}
        To find our $b_{n}$ we can only focus on the leading coefficients. Thus: 
        \begin{align*}
            b_{n} = \frac{n^{4}}{n^{5}} = \frac{1}{n}
        .\end{align*}
        So our test...
        \smallbreak \noindent
        \begin{minipage}[t]{0.47\textwidth}
            \begin{align*}
        &\lim\limits_{n \to \infty}{\frac{a_{n}}{b_{n}}} = \frac{\frac{n^{4} + 6}{n^{5} + 4}}{\frac{1}{n}} \\
        &=\lim\limits_{n \to \infty}{\frac{n(n^{4}+6)}{n^{5} + 4}} \\
        &=\lim\limits_{n \to \infty}{\frac{n^{5}+6n}{n^{5} + 4}} \\
        &=1
    .\end{align*}
\end{minipage}
\begin{minipage}[t]{0.47\textwidth}
    Since $\lim\limits_{n \to \infty}{\frac{a_{n}}{b_{n}}} \ne 0 \lor +\infty$. And $\frac{1}{n}$ diverges, we can conclude that $a_{n}$ will also diverge.
\end{minipage}

\item \textbf{Determine which series (or function) is greater}

    \begin{itemize}
        \item \textbf{Subtraction}: Given two functions $f(x) = \frac{1}{x}$ and $g(x) = \frac{x^4 + 6}{x^5 + 4}$, we want to compare them by considering the function $h(x) = f(x) - g(x)$:

            \[
                h(x) = f(x) - g(x) = \frac{1}{x} - \frac{x^4 + 6}{x^5 + 4}
            \]

            To compare these directly, it would be helpful to have a common denominator:

            \[
                h(x) = \frac{x^4 + 4 - (x^4 + 6)}{x(x^5 + 4)} = \frac{-2}{x(x^5 + 4)}
            \]

            Now, we can see that the sign of $h(x)$ depends on the sign of $x$ because the denominator $x(x^5 + 4)$ is always positive for $x \neq 0$. So:

            \begin{itemize}
                \item For $x > 0$, $h(x) < 0$, which means $f(x) < g(x)$.
                \item For $x < 0$, $h(x) > 0$, which means $f(x) > g(x)$.
            \end{itemize}
    \end{itemize}

\item \textbf{Alternating Series}
    Any series whose terms alternate between positive and negative values is called an alternating series. An alternating series can be written in the form 
    \begin{align*}
        \sum_{n=1}^{\infty} (-1)^{n+1} b_n = b_1 - b_2 + b_3 - b_4 + \cdots
    .\end{align*}
    or
    \begin{align*}
        \sum_{n=1}^{\infty} (-1)^n b_n = -b_1 + b_2 - b_3 + b_4 - \cdots
    .\end{align*}
    Where  $b_n > 0$  for all positive integers $n$.
\item \textbf{alternating series test (Leibniz criterion)}
    An alternating series of the form
    \[
        \sum_{n=1}^{\infty} (-1)^{n+1} b_n \quad \text{or} \quad \sum_{n=1}^{\infty} (-1)^n b_n
    \]
    converges if
    \begin{itemize}

        \item $0 < b_{n+1} \leq b_n\ \forall\ n \geq 1$
        \item $\lim_{n \to \infty} b_n = 0.$
    \end{itemize}
    \textbf{Note:} We remark that this theorem is true more generally as long as there exists some integer \( N \) such that \( 0 < b_{n+1} \leq b_n \) for all \( n \geq N \).
    \bigbreak \noindent 
    \textbf{Additional note:} The AST allows us to consider just the positive terms to check for these two conditions because if a series of decreasing positive terms that approach zero is alternated in sign, the alternating series will converge. This is a special property of alternating series that does not generally hold for non-alternating series.

\item \textbf{Show decreasing (For the AST)}: Consider the series
    \begin{align*}
        \summation{\infty}{n=1}\ \frac{(-1)^{n+1}}{n^{2}}\ 
    .\end{align*}
    \bigbreak \noindent 
    So you see we have $b_{n}=  \frac{1}{n^{2}}$. For the AST, we must show that this is decreasing. If $b_{n+1} = \frac{1}{(n+1)^{2}}$. Then we see
    \begin{align*}
        \frac{1}{(n+1)^{2}} < \frac{1}{n^{2}}
    .\end{align*}
    \bigbreak \noindent 
    Thus it is decreasing for $n \geq 1$ ($b_{n+1} < b_{n}$)
    \blacksquare
\item \textbf{Remainders in alternating series}
    Consider an alternating series of the form
    \[
        \sum_{n=1}^{\infty} (-1)^{n+1} b_n \quad \text{or} \quad \sum_{n=1}^{\infty} (-1)^n b_n,
    \]
    that satisfies the hypotheses of the alternating series test. Let \( S \) denote the sum of the series and \( S_N \) denote the \( N \)-th partial sum. For any integer \( N \geq 1 \), the remainder \( R_N = S - S_N \) satisfies
    \[
        \lvert R_N \rvert \leq b_{N+1}.
    \]
    This tells us that if we stop at the $N^{th}$ term, the error we are making is at most the size of the next term
\item \textbf{Absolute and conditional convergence}
    \begin{itemize}
        \item A series \(\sum_{n=1}^{\infty} a_n\) exhibits absolute convergence if \(\sum_{n=1}^{\infty} |a_n|\) converges.
        \item A series \(\sum_{n=1}^{\infty} a_n\) exhibits conditional convergence if \(\sum_{n=1}^{\infty} a_n\) converges but \(\sum_{n=1}^{\infty} |a_n|\) diverges.
        \item If $\summation{\infty}{n=1}\ \lvert a_{n} \rvert\ $ converges then $\summation{\infty}{n=1}\ a_{n}\ $ converges
    \end{itemize}
    \bigbreak \noindent 
    \textbf{Note:} if $\abs{a_{n}}$ diverges, we cannot have absolute convergence, thus we must examine to see if normal $a_{n}$ converges, in which case we would have conditional convergence
    \bigbreak \noindent 
    \textbf{Big Note:} If a series not strictly decreasing, we can still check for absolute/conditional convergence. Take $\summation{\infty}{n=1}\ \frac{\sin{n}}{3^{n} + 4}\  $ for example.
\item \textbf{Ratio test}
    Let \(\sum_{n=1}^{\infty} a_n\) be a series with nonzero terms. Let
    \begin{align*}
        \rho = \lim_{n \to \infty} \left| \frac{a_{n+1}}{a_n} \right|.
    .\end{align*}
    Then:
    \begin{enumerate}[label=\roman*.]
        \item If \(0 \leq \rho < 1\), then \(\sum_{n=1}^{\infty} a_n\) converges absolutely.
        \item If \(\rho > 1\) or \(\rho = \infty\), then \(\sum_{n=1}^{\infty} a_n\) diverges.
        \item If \(\rho = 1\), the test does not provide any information.
    \end{enumerate}
    \bigbreak \noindent 
    \textbf{Note:} The ratio test is useful for series whose terms involve factorials
\item \textbf{Root test}
    Consider the series \(\sum_{n=1}^{\infty} a_n\). Let
    \begin{align*}
        \rho = \lim_{n \to \infty} \sqrt[n]{|a_n|}.
    .\end{align*}
    \begin{enumerate}[label=\roman*.]
        \item If \(0 \leq \rho < 1\), then \(\sum_{n=1}^{\infty} a_n\) converges absolutely. 
        \item If \(\rho > 1\) or \(\rho = \infty\), then \(\sum_{n=1}^{\infty} a_n\) diverges. 
        \item If \(\rho = 1\), the test does not provide any information.
    \end{enumerate}
    \bigbreak \noindent 
    \textbf{Note:} The root test is useful for series whose terms involve exponentials
\item \textbf{Which tests require positive terms}
    \begin{itemize}
        \item The \textbf{Integral Test:} This test applies to series where the terms come from a function that is positive, continuous, and decreasing on a certain interval. The convergence or divergence of the series is determined by the convergence or divergence of the corresponding improper integral of the function.
        \item The \textbf{Remainder estimate for the integral test}
        \item The \textbf{Comparison Test:} This test compares the terms of a series to those of another series with known convergence behavior. It requires that the terms of both series be positive or non-negative.
        \item The \textbf{Limit Comparison Test:} Similar to the Comparison Test, this test involves comparing the terms of two series by taking the limit of the ratio of their terms. It requires that the terms of both series be positive.
        \item In \textbf{alternating series}, $b_{n}$ must have only positive terms
    \end{itemize}
\end{itemize}

\pagebreak \bigbreak \noindent 
\subsection{Chapter 6 Definitions and Theorems}
\begin{itemize}
    \item \textbf{Euler definition for $e$}
        \begin{align*}
                &e^{a} = \lim\limits_{n \to \infty}{\left(1+\frac{a}{n}\right)^{n}} \\
                &\frac{1}{e^{a}} = \lim\limits_{n \to \infty}{\left(1+\frac{-a}{n}\right)^{n}} \\ 
                &\frac{1}{e^{a}} = \lim\limits_{n \to \infty}{\left(\frac{n}{n+a}\right)^{n}}
            .\end{align*}
        \item \textbf{Other definition for $e $}
            \begin{align*}
                &e = \summation{\infty}{n=0}\ \frac{1}{n!}\  \\
                &e-1 = \summation{\infty}{n=1}\ \frac{1}{n!}\ \\
                &e^{x} = \summation{\infty}{n=0}\ \frac{x^{n}}{n!}\ 
            .\end{align*}
        \item \textbf{Power series}:
            A series of the form
            \begin{align*}
                \sum_{n=0}^{\infty} c_n x^n &= c_0 + c_1 x + c_2 x^2 + \cdots 
            .\end{align*}
            is a power series centered at \( x = 0 \).
            \bigbreak \noindent 
            A series of the form
            \begin{align*}
                \sum_{n=0}^{\infty} c_n (x - a)^n &= c_0 + c_1 (x - a) + c_2 (x - a)^2 + \cdots 
            .\end{align*}
            is a power series centered at \( x = a \).

        \item \textbf{Convergence of a Power Series}
            \bigbreak \noindent 
            Consider the power series \(\sum_{n=0}^{\infty} c_n (x - a)^n\). The series satisfies exactly one of the following properties:
            \begin{enumerate}[label=(\roman*)]
                \item The series converges at \( x = a \) and diverges for all \( x \neq a \).
                \item The series converges for all real numbers \( x \).
                \item There exists a real number \( R > 0 \) such that the series converges if \( |x - a| < R \) and diverges if \( |x - a| > R \). At the values \( x \) where \( |x - a| = R \), the series may converge or diverge.
            \end{enumerate}
        \item \textbf{A power series always converges at its center}
        \item \textbf{Radius of convergence}:         Consider the power series \(\sum_{n=0}^{\infty} c_n (x - a)^n\). The set of real numbers \( x \) where the series converges is the interval of convergence. If there exists a real number \( R > 0 \) such that the series converges for \( |x - a| < R \) and diverges for \( |x - a| > R \), then \( R \) is the radius of convergence. If the series converges only at \( x = a \), we say the radius of convergence is \( R = 0 \). If the series converges for all real numbers \( x \), we say the radius of convergence is \( R = \infty \) (Figure 6.2).
        \item \textbf{Finding interval of convergence and radius of convergence}
            \begin{itemize}
                \item Fact: power series is always convergent on its center
                \item Use ratio test (values of $\rho$) 
                \item Use $\rho < 1$ to find Radius of convergence
                \item Test end points of interval by plugging into original series and seeing whether the series is convergent or divergent
            \end{itemize}
        \item \textbf{If $\rho = 0$, the power series converges for all $x$}
        \item \textbf{If $\rho = \infty$}, the series diverges for all $x \neq a$ 
        \item \textbf{Combining Power Series:}
            Suppose that the two power series \(\sum_{n=0}^{\infty} c_n x^n\) and \(\sum_{n=0}^{\infty} d_n x^n\) converge to the functions \(f\) and \(g\), respectively, on a common interval \(I\).
            \begin{enumerate}[label=(\roman*)]
                \item The power series \(\sum_{n=0}^{\infty} (c_n x^n \pm d_n x^n)\) converges to \(f \pm g\) on \(I\).
                \item For any integer \(m \geq 0\) and any real number \(b\), the power series \(\sum_{n=0}^{\infty} b x^m c_n x^n\) converges to \(b x^m f(x)\) on \(I\).
                    \smallbreak \noindent
                    Eg: If we know $\summation{\infty}{n=0}\ a_{n}x_{n}\  $ has $I = (-1,1) $. Then
                    \begin{align*}
                            &\summation{\infty}{n=0}\ a_{n}3^{n}x^{n}\  \\
                        =&\summation{\infty}{n=0}\ a_{n}(3x)^{n}\  \\
                         &I=(-3,3)
                     .\end{align*}
                 \item For any integer \(m \geq 0\) and any real number \(b\), the series \(\sum_{n=0}^{\infty} c_n (b x^m)^n\) converges to \(f(b x^m)\) for all \(x\) such that \(b x^m\) is in \(I\).
             \end{enumerate}
         \item \textbf{For part I, II, and III, the interval of the combined series is the smaller interval}
         \item \textbf{Cauchy product (Multiplying power series)}:
             Suppose that the power series \(\sum_{n=0}^{\infty} c_n x^n\) and \(\sum_{n=0}^{\infty} d_n x^n\) converge to \(f\) and \(g\), respectively, on a common interval \(I\). Let
             \begin{align*}
                    &e_n = c_0 d_n + c_1 d_{n-1} + c_2 d_{n-2} + \cdots + c_{n-1} d_1 + c_n d_0  \\
                    &= \sum_{k=0}^{n} c_k d_{n-k}.
                .\end{align*}
                Then
                \[
                    \left( \sum_{n=0}^{\infty} c_n x^n \right) \left( \sum_{n=0}^{\infty} d_n x^n \right) = \sum_{n=0}^{\infty} e_n x^n
                \]
                and
                \[
                    \sum_{n=0}^{\infty} e_n x^n \text{ converges to } f(x) \cdot g(x) \text{ on } I.
                \]
                The series \(\sum_{n=0}^{\infty} e_n x^n\) is known as the Cauchy product of the series \(\sum_{n=0}^{\infty} c_n x^n\) and \(\sum_{n=0}^{\infty} d_n x^n\).
            \item \textbf{Sterling's Approximation}
                \begin{align*}
                    n! \approx \sqrt{2\pi n}\left(\frac{n}{e}\right)^{n}
                .\end{align*}
            \item \textbf{Gamma function (extension of the factorial function)}
                \begin{align*}
                     &\Gamma(z) = \int_{0}^{\infty}\ e^{-t}t^{z-1}\ dt \\
                     &\text{Thus, } n! = \Gamma(n+1)
                 .\end{align*}
             \item \textbf{Cool definition for $e^{x}$}
                 \begin{align*}
                    &f^{\prime}(x) = rf(x) \\
                    &\implies f(x) = ce^{rx}
                .\end{align*}
            \item \textbf{Term-by-Term Differentiation and Integration for Power Series.}
                Suppose that the power series $\sum_{n=0}^{\infty} c_n (x - a)^n$ converges on the interval $(a - R, a + R)$ for some $R > 0$. Let $f$ be the function defined by the series
                \[
                    f(x) = \sum_{n=0}^{\infty} c_n (x - a)^n = c_0 + c_1(x - a) + c_2(x - a)^2 + c_3(x - a)^3 + \cdots
                \]
                for $|x - a| < R$. Then $f$ is differentiable on the interval $(a - R, a + R)$ and we can find $f'$ by differentiating the series term-by-term:
                \[
                    f'(x) = \sum_{n=1}^{\infty} n c_n (x - a)^{n-1} = c_1 + 2c_2(x - a) + 3c_3(x - a)^2 + \cdots
                \]
                for $|x - a| < R$. Also, to find $\int f(x) \, dx$, we can integrate the series term-by-term. The resulting series converges on $(a - R, a + R)$, and we have
                \[
                    \int f(x) \, dx = C + \sum_{n=0}^{\infty} \frac{c_n (x - a)^{n+1}}{n+1} = C + c_0(x - a) + \frac{c_1(x - a)^2}{2} + \frac{c_2(x - a)^3}{3} + \cdots
                \]
                for $|x - a| < R$.
                \bigbreak \noindent 
                \textbf{NOTE!} when a power series is differentiated or integrated term-by-term, it says nothing about what happens at the endpoints.
            \item \textbf{Uniqueness of Power Series}:
                Let $\sum_{n=0}^{\infty} c_n (x - a)^n$ and $\sum_{n=0}^{\infty} d_n (x - a)^n$ be two convergent power series such that
                \[
                    \sum_{n=0}^{\infty} c_n (x - a)^n = \sum_{n=0}^{\infty} d_n (x - a)^n
                \]
                for all \( x \) in an open interval containing \( a \). Then \( c_n = d_n \) for all \( n \geq 0 \).
            \item \textbf{When finding the Cauchy product of two power series, we include the zero term when finding the new power series general term. For integrating and differentiating, we do not}
            \item \textbf{Taylor and Maclaurin series}:
                If $f$ has derivatives of all orders at $x=a$, then the Taylor series for the function $f$ at $a$ is
                \begin{equation}
                    \sum_{n=0}^{\infty} \frac{f^{(n)}(a)}{n!}(x-a)^n = f(a) + f'(a)(x-a) + \frac{f''(a)}{2!}(x-a)^2 + \cdots + \frac{f^{(n)}(a)}{n!}(x-a)^n + \cdots.
                \end{equation}
                The Taylor series for $f$ at $0$ is known as the Maclaurin series for $f$.
                \bigbreak \noindent 
            \item \textbf{Uniqueness of Taylor series}:  If a function $f$ has a power series at $a$ that converges to $f$ on some open interval containing $a$, then that power series is the Taylor series for $f$ at $a$.
            \item \textbf{Taylor-Macluarin Polynomials}:         
                If $f$ has $n$ derivatives at $x=a$, then the $n$th Taylor polynomial for $f$ at $a$ is
                \begin{equation}
                    p_n(x) = f(a) + f'(a)(x-a) + \frac{f''(a)}{2!}(x-a)^2 + \frac{f'''(a)}{3!}(x-a)^3 + \cdots + \frac{f^{(n)}(a)}{n!}(x-a)^n.
                \end{equation}
                The $n$th Taylor polynomial for $f$ at $0$ is known as the $n$th Maclaurin polynomial for $f$.
            \item \textbf{Taylor’s Theorem with Remainder}:
                Let $f$ be a function that can be differentiated $n+1$ times on an interval $I$ containing the real number $a$. Let $p_n$ be the $n$th Taylor polynomial of $f$ at $a$ and let
                \begin{align*}
                    R_n(x) = f(x) - p_n(x)
                \end{align*}
                be the $n$th remainder. Then for each $x$ in the interval $I$, there exists a real number $c$ between $a$ and $x$ such that
                \begin{align*}
                    R_n(x) = \frac{f^{(n+1)}(c)}{(n+1)!}(x - a)^{n+1}.
                \end{align*}
                If there exists a real number $M$ such that $\left|f^{(n+1)}(x)\right| \leq M$ for all $x \in I$, then
                \begin{align*}
                    \left|R_n(x)\right| \leq \frac{M}{(n+1)!}\left|x - a\right|^{n+1}
                \end{align*}
                $\forall\ x \in I$ 
            \item \textbf{Maclaurin Series/Polynomials for sine}:
                The Taylor series for the sine function is 
                \begin{align*}
                    \sin{(x)} = \summation{\infty}{n=0}\ (-1)^{n}\frac{x^{2n+1}}{(2n+1)!}\ = x-\frac{x^{3}}{3!} + \frac{x^{5}}{5!} - \frac{x^{7}}{7!}  + ... \quad \text{For } x\in\mathbb{R} 
                .\end{align*}
                Where $p_{n}$ obeys
                \begin{align*}
                   &p_{2m+1} = p_{2m+2} \\
                   &=x-\frac{x^{3}}{3!}+\frac{x^{5}}{5!}-\frac{x^{7}}{7!} + ... + \frac{(-1)^{m}x^{2m+1}}{(2m+1)!}
               .\end{align*}
               \textbf{Note:} When discussing specific polynomials, say $P_5 $ for example, we arnt talking about the first $5$ terms in the series above, we are talking about the polynomial \texttt{up to} degree 5. Thus it would have 3 terms
           \item \textbf{Maclaurin Series/Polynomials for cosine}: Similar to the sine function, the Maclaurin series for the cosine function is 
               \begin{align*}
                   \cos{(x)} = \summation{\infty}{n=0}\ (-1)^{n}\frac{x^{2n}}{(2n)!}\ = 1 - \frac{x^{2}}{2!} + \frac{x^{4}}{4!} -\frac{x^{6}}{6!} + ... \quad \text{For } x\in\mathbb{R} 
               .\end{align*}
               Where $p_{n} $ obeys
               \begin{align*}
                    &p_{2m} = p_{2m+1} \\
                    &=1 - \frac{x^{2}}{2!} + \frac{x^{4}}{4!} -\frac{x^{6}}{6!}     + ... + (-1)^{n}\frac{x^{2m}}{(2m)!}
                .\end{align*}
            \item \textbf{Maclaurin Series/Polynomials for $e^{x}$}: We find the Maclaurin series for the exponential function to be 
                \begin{align*}
                    e^{x} = \summation{\infty}{n=0}\ \frac{x^{n}}{n!}\  = 1 + x + \frac{x^{2}}{2!} + \frac{x^{3}}{3!} + ... \quad \text{For } x\in\mathbb{R}
                .\end{align*}
                \bigbreak \noindent 
                \textbf{Note:} this definition is described above but now we have a way of showing its truthiness
            \item \textbf{Convergence of Taylor Series}:
                Suppose that $f$ has derivatives of all orders on an interval $I$ containing $a$. Then the Taylor series
                \[
                    \sum_{n=0}^{\infty} \frac{f^{(n)}(a)}{n!}(x-a)^n
                \]
                converges to $f(x)$ for all $x$ in $I$ if and only if
                \[
                    \lim_{n \to \infty} R_n(x) = 0
                \]
                for all $x$ in $I$.
                \bigbreak \noindent 
                \textbf{Note:}  With this theorem, we can prove that a Taylor series for $f$ at $a$ converges to $f$ if we can prove that the remainder $R_n(x) \to 0$. To prove that $R_n(x) \to 0$, we typically use the bound
                \[
                    |R_n(x)| \leq \frac{M}{(n+1)!}|x - a|^{n+1}
                \]
                from Taylor’s theorem with remainder.
            \item \textbf{Using taylor series to find limits}: Consider the limit $\lim\limits_{x \to 0+}{\frac{\cos{\sqrt{x}} -1}{2x}}$. We know we have a problem if we attempt to use the \texttt{direct substitution property}. Thus, we can substitute $\cos{(\sqrt{x})}$ for its \texttt{Maclaurin series} and see what happens. We know the maclaurin series for $\cos{(x)}$ is 
                \begin{align*}
                    \cos{x} &\sim \summation{\infty}{n=0}\ (-1)^{n}\frac{x^{2n}}{(2n)!}\   = 1 - \frac{x^{2}}{2!} + \frac{x^{4}}{4!} - \frac{x^{6}}{6!} + ... \\
                    \implies \cos{\sqrt{x}} &\sim \summation{\infty}{n=0}\ (-1)^{n}\frac{(x^{\frac{1}{2}})^{2n}}{(2n)!}\   = 1 - \frac{(x^{\frac{1}{2}})^{2}}{2!} + \frac{(x^{\frac{1}{2}})^{4}}{4!} - \frac{(x^{\frac{1}{2}})^{6}}{6!} + ... \\
                                            &=\summation{\infty}{n=0}\ (-1)^{n}\frac{x^{n}}{(2n)!}\   = 1 - \frac{x}{2!} + \frac{x^{2}}{4!} - \frac{x^{3}}{6!} + ... 
                                        .\end{align*}
                                        So we have
                                        \begin{align*}
                    &\lim\limits_{x \to 0^{+}}{\frac{\left(1-\frac{x}{2!}+\frac{x^{2}}{4!}-\frac{x^{3}}{6!} + ...\right)-1}{2x}}\\
                    &=\lim\limits_{x \to 0^{+}}{\frac{\left(-\frac{x}{2!}+\frac{x^{2}}{4!}-\frac{x^{3}}{6!} + ...\right)}{2x}}\\
                    &=\lim\limits_{x \to 0^{+}}\left(-\frac{x}{2!}+\frac{x^{2}}{4!}-\frac{x^{3}}{6!} + ...\right)\cdot \frac{1}{2x}\\
                    &=\lim\limits_{x \to 0^{+}}{-\frac{1}{4}} \\
                    &=-\frac{1}{4}
                .\end{align*}
            \item \textbf{Multiplying a known Taylor series by some other function:} Consider $f(x) = x\cos{x}$. Since we know that the taylor series for $\cos{(x)} = \summation{\infty}{n=0}\ (-1)^{n}\frac{x^{2n}}{(2n)!}\ $, which converges $\forall\ x \in \mathbb{R} $. We can easily just multiply this by $x$ to get the taylor series for $f(x) = x\cos{(x)}$.  Since the Taylor series for $\cos{(x)}$ converges for all real $x$, multiplying it by  $x$ won't affect its convergence properties. The resulting series will still converge for all $x$.
                \bigbreak \noindent 
                \textbf{Note:} The product of the Taylor series and the function will be valid only where both the series converges and the function is well-defined. Probably analyze the convergence of the product series.
            \item \textbf{Multiplying a known Taylor series by some other function where convergence is affected:} Consider the example above. Although this time suppose we multiply $\cos{(x)}$ by $\frac{1}{x} $ instead of $x$. We know the resulting taylor series must not be convergent at $x=0$ because $\frac{1}{x}$ is not defined at zero. It is important to understand that we will not get this conclusion from the ratio test alone. The Ratio Test alone does not account for points where the series or its terms are not defined.
                \bigbreak \noindent 
                \textbf{TLDR:} Be mindful about the domain of the function you are multiplying the Taylor series by. Do not only rely on the ratio test to find points of convergence.
            \item \textbf{Analytic function:} 
                \begin{itemize}
                    \item An analytic function is infinitely differentiable within its domain.
                    \item An analytic function can be represented by a convergent power series (like a Taylor series) around any point in its domain. 
                    \item  The power series representing an analytic function not only exists but also converges to the function within a certain radius around the point of expansion
                \end{itemize}
            \item \textbf{Maclaurin series for $\frac{1}{1-x}$}:
                \begin{align*}
                    \frac{1}{1-x} \sim \summation{\infty}{n=0}\ x^{n} \ \quad \text{for } \abs{x} < 1
                .\end{align*}
            \item \textbf{Maclaurin series for $\ln{(1+x)}$}:
                \begin{align*}
                    \ln{(1+x)} \sim \summation{\infty}{n=1}\ (-1)^{n+1}\frac{x^{n}}{n} \ \quad \text{for }\abs{x} < 1
                .\end{align*}
            \item \textbf{Maclaurin series for $\tan^{-1}{x}$}:
                \begin{align*}
                    \tan^{-1}{x} \sim \summation{\infty}{n=0}\ (-1)^{n}\frac{x^{2n+1}}{2n+1} \ \quad \text{for } \abs{x} \leq 1
                .\end{align*}

            \item \textbf{Binomial expansion for $(1+x)^{r}$ for $r\in\mathbb{Z^{+}}$}
                \begin{align*}
                    (1+x)^{r}= \summation{r}{n=0}\ \binom{r}{n}x^{n}\ \quad r \in \mathbb{Z^{+}}
                .\end{align*}
            \item \textbf{Binomial expansion for $(1+x)^{r}$ for $r\in\mathbb{R}$}
                \begin{align*}
                    (1+x)^{r} = \summation{\infty}{n=0}\ \binom{r}{n}x^{n} = 1+rx+ \frac{r(r-1)}{2!}x^2 + \cdots + \frac{r(r-1)\cdots(r-n+1)}{n!}x^n + \cdots
                .\end{align*}
                Where 
                \begin{align*}
            &\binom{r}{n} = \frac{f^{(n)}(0)}{n!} = \frac{r(r-1)(r-2)\cdot ...\cdot (r-n+1)}{n!} 
        .\end{align*}
        \textbf{Note:} When $n=0$, $\binom{r}{0} = 1$, when $n=1$, $\binom{r}{1} = r$, when $n=2$, $\binom{r}{2} = \frac{r(r-1)}{2!} $, etc...
    \item \textbf{The binomial theorem:}
        For any real number \(r\), the Maclaurin series for \( f(x) = (1 + x)^r \) is the binomial series. It converges to \(f\) for \(|x| < 1\), and we write
        \begin{align*}
            (1 + x)^r = \sum_{n=0}^{\infty} \binom{r}{n} x^n = 1 + rx + \frac{r(r-1)}{2!}x^2 + \cdots + \frac{r(r-1) \cdots (r-n+1)}{n!}x^n + \cdots
        \end{align*}
        for \(|x| < 1\).
\end{itemize}


\pagebreak \bigbreak \noindent 
\subsection{Chapter 6 Problems to Remember}
\begin{itemize}
    \item \textbf{Problem to remember (Properties of power series):} Evaluate the infinite series by identifying it as the value of an integral of a geometric series.
        \begin{align*}
            \summation{\infty}{n=0}\ \frac{(-1)^{n}}{2n+1}\ 
        .\end{align*}
        \begin{remark}
            If we can find which geometric power series's integral (with some bounds) gives us the given series, we can then integrate the function representation to get the value of the original series. Consider the geometric power series
            \begin{align*}
                \frac{1}{1+x^{2}} = \frac{1}{1-(-x^{2})} = \summation{\infty}{n=0}\ (-x^{2})^{n}\  = \summation{\infty}{n=0}\ (-1)^{n}x^{2n}\ 
            .\end{align*}
            Suppose we then integrate the power series
            \begin{align*}
                        &\int_{}^{}\ \summation{\infty}{n=0}\ (-1)^{n}x^{2n}\ dx \\
                        &=\summation{\infty}{n=0}\ (-1)^{n}\ \int_{}^{}\ x^{2n}\ dx \\
                        &=\frac{1}{2n+1}x^{2n+1}
                    .\end{align*}
                    Now we must deduce for which bounds will the FTC give us the original series $\summation{\infty}{n=0}\ \frac{(-1)^{n}}{2n+1}\  $. We come to the conclusion
                    \begin{align*}
                        \summation{\infty}{n=0}\ (-1)^{n}\ \int_{0}^{1}\ x^{2n}\ dx = \summation{\infty}{n}\ \frac{(-1)^{n}}{2n+1}\ 
                    .\end{align*}
                    This implies we can integrate the function representation of the geometric power series we just integrated to get the value of the infinite series $\summation{\infty}{n=0}\ \frac{(-1)^{n}}{2n+1}\  $. Thus,
                    \begin{align*}
                        \int_{0}^{1}\ \frac{1}{1+x^{2}}\ dx = \frac{\pi}{4}
                    .\end{align*}
                \end{remark}
            \item \textbf{Problem to remember (Properties of power series):} Find the power series for $f(x) = \ln{x}$ centered at $x=9$ by using term-by-term integration or differentiation.
                \smallbreak \noindent
                \textit{Solution.} The goal is to find a function that resembles one we know (sum of geometric series $\frac{a}{1-x}$) such that if we integrate or differentiate we can get $\ln{x}$. Since we know the integral of $\frac{1}{x}$ is $\ln{x}$, and we can easily manipulate $\frac{1}{x}$ to be in the form $\frac{a}{1-x}$, we choose $\frac{1}{x} $ to be the function to examine. Thus, 
                \begin{align*}
                    \frac{1}{x} &= \frac{1}{9+x-9} = \frac{1}{9-(-(x-9))} = \frac{1/9}{1-\left(\frac{-(x-9)}{9}\right)} \\
                    \text{If } f(x) &= \frac{a}{1-x} \sim \summation{\infty}{n=0}\ a(x^{n})\ = a + ax + ax^{2} + ax^{3} + ... \\
                    \implies f(x) &= \frac{1/9}{1-\left(\frac{-(x-9)}{9}\right)} \sim \summation{\infty}{n=0}\ \frac{1}{9}\ \left(\frac{-(x-9)}{9}\right)^{n} = \summation{\infty}{n=0}\ \frac{(-1)^{n}(x-9)^{n}}{9^{n+1}}\ 
                .\end{align*}
                Then we can throw in some integrals
                \begin{align*}
                    &\int f(x)\ dx = \int \frac{1/9}{1-\left(\frac{-(x-9)}{9}\right)}\ dx = \int \summation{\infty}{n=0}\ \frac{(-1)^{n}(x-9)^{n}}{9^{n+1}}\ \ dx \\
                    &\ln{x} = \summation{\infty}{n=0}\ \frac{(-1)^{n}}{9^{n+1}}\  \int (x-9)^{n}\ dx \\
                    &\ln{x}  =  \summation{\infty}{n=0}\ \frac{(-1)^{n}(x-9)^{n+1}}{(n+1)9^{n+1}}\ + C 
                .\end{align*}
                If we let $x=9$
                \begin{align*}
                    &\ln{9}  =  \summation{\infty}{n=0}\ \frac{(-1)^{n}(9-9)^{n+1}}{(n+1)9^{n+1}}\ + C  \\
                    &\ln{9} = C
                .\end{align*}
                Thus, we have
                \begin{align*}
                    f(x) = \ln{9} + \summation{\infty}{n=0}\ \frac{(-1)^{n}(9-9)^{n+1}}{(n+1)9^{n+1}}\  \\
                .\end{align*}
                \textbf{Note:} the "$+C$" is initially omitted from $\ln{(x)}$ because we're considering a specific antiderivative. When you integrate the power series, you include "$+C$" to account for the general form of the antiderivative. The value of  $C$ is then determined using a specific condition to match the specific antiderivative you're interested in.
            \item \textbf{Problem to remember}: Say we want to find the power series for $7x\ln{1+x}$. We can first find the power series for $\ln{(1+x)} $
                \begin{align*}
                    \frac{d}{dx} \ln{(1+x)} = \frac{1}{1+x} = \frac{1}{1-(-x)}
                .\end{align*}
                We know the power series for  $\frac{1}{1-(-x)} $ is
                \begin{align*}
                    &\summation{\infty}{n=0}\ (-1)^{n}x^{n}\  \\
                    &\implies \int \frac{1}{1+x} = \int \summation{\infty}{n=0}\ (-1)x^{n}\   \\
                    &\ln{(1+x)} = \summation{\infty}{n=0}\ (-1)^{n}\frac{x^{n+1}}{n+1}\  + C  \\
                    &\ln{(1+x)} = \summation{\infty}{n=0}\ (-1)^{n}\frac{x^{n+1}}{n+1}\ 
                .\end{align*}
                Now that we have found the power series for $\ln{(1+x)}$, to find the power series for $7x\ln{(1+x)}$...
                \begin{align*}
                    &7x\summation{\infty}{n=0}\ (-1)^{n}\frac{x^{n+1}}{n+1}\  \\
                    &\summation{\infty}{n=0}\ 7x\left((-1)^{n}\frac{x^{n+1}}{n+1}\right)\  \\
                    &=\summation{\infty}{n=0}\ (-1)^{n}\frac{7x^{n+2}}{n+1}\ 
                .\end{align*}
                \pagebreak \bigbreak \noindent 
            \item \textbf{Problem to remember (Cumbersome taylor polynomial)}: Suppose we have some function $f$, and we would like to find the Taylor polynomial up to degree $3$. Say $f(x) = e^{2x}\cos{(x)}$. We could find each derivative up to degree 3, however, given that we know the taylor series for both $e^{2x}$ and $\cos{(x)}$.
                \begin{align*}
                    &e^{2x} = \summation{\infty}{n=0}\ \frac{(2x)^{n}}{n!}\  = 1+2x + \frac{4x^{2}}{2!} + \frac{8x^{3}}{3!} + ...\\
                    &\cos{(x)} = \summation{\infty}{n=0}\ (-1)^{n}\frac{x^{2n}}{(2n)!} = 1-\frac{x^{2}}{2!} + \frac{x^{4}}{4!} + \frac{x^{6}}{6!} + ...\ 
                .\end{align*}
                We can find $P_{3}(x)$ by multiplying  these Taylor series. Thus,
                \begin{align*}
                    (1+2x + \frac{4x^{2}}{2!} + \frac{8x^{3}}{3!} + ...)(1-\frac{x^{2}}{2!} + \frac{x^{4}}{4!} + \frac{x^{6}}{6!}+...)
                .\end{align*}
                And we can find $P_{3}$ to be 
                \begin{align*}
                    P_{3} = 1 +2x + \frac{3}{2}x^{2} +\frac{1}{3}x^{3}
                .\end{align*}
            \item \textbf{Problem to Remember (Using known Taylor series to find sum of series):} Consider the series $\summation{\infty}{n=0}\ (-1)^{n}\frac{\left(\frac{1}{25}\right)^{n-3}}{2n+1}\ $
                \bigbreak \noindent 
                We notice this resembles the Taylor series for the arctangent function $\tan^{-1}{x} = \summation{\infty}{n=0}\ (-1)^{n}\frac{x^{2n+1}}{2n+1}\ \text{ for } \abs{x} \leq 1$. Thus, we maipulate the series to better conform to the Taylor series for $\tan^{-1}{x}$.
                \begin{align*}
                   &\summation{\infty}{n=0}\ (-1)^{n}\frac{\left(\frac{1}{25}\right)^{n-3}}{2n+1}\  \\
                   &=\summation{\infty}{n=0}\ (-1)^{n}\frac{\left(\frac{1}{5}\right)^{2n-6}}{2n+1}\ \\
                   &=\summation{\infty}{n=0}\ (-1)^{n}\frac{\left(\frac{1}{5}\right)^{2n}5^{6}}{2n+1}\ \\
                   &=\summation{\infty}{n=0}\ (-1)^{n}\frac{\left(\frac{1}{5}\right)^{2n}5^{6}\left(\frac{1}{5}\right)5}{2n+1}\ \\
                   &=\summation{\infty}{n=0}\ (-1)^{n}\frac{\left(\frac{1}{5}\right)^{2n+1}5^{7}}{2n+1}\ \\
               .\end{align*}
               Thus the sum will be 
               \begin{align*}
                   5^{7}\tan^{-1}{\frac{1}{5}}
               .\end{align*}

       \end{itemize}


    \pagebreak 
    \unsect{Fundemental Physics: Classical Mechanics}
    \bigbreak \noindent 
    \subsection{Chapter 1: Units and Measurement}
    \smallbreak \noindent
    \subsubsection{Key Terms}
    \begin{itemize}
        \item \textbf{Physics}, which comes from the Greek \textit{phúsis}, meaning “nature,” is concerned with describing the interactions of energy, matter, space, and time to uncover the fundamental mechanisms that underlie every phenomenon.
        \item A \textbf{model} is a representation of something that is often too difficult (or impossible) to display directly. Although
        \item a \textbf{theory} is a testable explanation for patterns in nature supported by scientific evidence and verified multiple times by various groups of researchers.
        \item A \textbf{law} uses concise language to describe a generalized pattern in nature supported by scientific evidence and repeated experiments. Often, a law can be expressed in the form of a single mathematical equation. 
        \item \textbf{SI units} (for the French Système International d’Unités), also known as the \textbf{metric system}.
        \item The Metric system may also be referred to as the \textbf{centimeter–gram–second (cgs)} system.
        \item  \textbf{English units} (also known as the customary or imperial system). English units were historically used in nations once ruled by the British Empire and are still widely used in the United States.
        \item The English system may also be referred to as the \textbf{the foot–pound–second (fps)} system. 
        \item In any system of units, the units for some physical quantities must be defined through a measurement process. These are called the \textbf{base quantities} for that system
        \item The base quantities units are the system’s \textbf{base units}
        \item All other physical quantities can then be expressed as algebraic combinations of the base quantities. Each of these physical quantities is then known as a \textbf{derived quantity} and each unit is called a \textbf{derived unit.}
        \item A \textbf{conversion factor} is a ratio that expresses how many of one unit are equal to another unit.
        \item The \textbf{dimension} of any physical quantity expresses its dependence on the base quantities as a product of symbols (or powers of symbols) representing the base quantities.
        \item Any quantity with a dimension that can be written so that all seven powers are zero (that is, its dimension is \( L^0M^0T^0I^0\Theta^0N^0J^0 \)) is called \textbf{dimensionless} (or sometimes “of dimension 1,” because anything raised to the zero power is one). 
        \item  Physicists often call dimensionless quantities \textbf{pure numbers}.
        \item The importance of the concept of dimension arises from the fact that any mathematical equation relating physical quantities must be \textbf{dimensionally consistent}
            \begin{itemize}
                \item Every term in an expression must have the same dimensions; it does not make sense to add or subtract quantities of differing dimension (think of the old saying: “You can’t add apples and oranges”). In particular, the expressions on each side of the equality in an equation must have the same dimensions.
                \item The arguments of any of the standard mathematical functions such as trigonometric functions (such as sine and cosine), logarithms, or exponential functions that appear in the equation must be dimensionless. These functions require pure numbers as inputs and give pure numbers as outputs.
            \end{itemize}
        \item \textbf{Estimation} means using prior experience and sound physical reasoning to arrive at a rough idea of a quantity’s value.
        \item \textbf{Accuracy} is how close a measurement is to the accepted reference value for that measurement.
        \item The \textbf{precision of measurements} refers to how close the agreement is between repeated independent measurements (which are repeated under the same conditions).
        \item The precision of a measuring system is related to the \textbf{uncertainty} in the measurements whereas the accuracy is related to the \textbf{discrepancy} from the accepted reference value.
        \item \textbf{Discrepancy} (or “measurement error”) is the difference between the measured value and a given standard or expected value.
        \item   Another method of expressing uncertainty is as a percent of the measured value. 
    \end{itemize}

    \pagebreak 
    \subsubsection{Defintions and Theorems (and important things)}
    \begin{itemize}
        \item \textbf{Order of Magnitude (Method one)}
            \begin{align*}
                m = \log_{10}{l},\ l \in \mathbb{R}
            .\end{align*}
            For example
            \begin{align*}
                &\log_{}{450} \approx 3
            .\end{align*}
            Thus the order of magnitude is $10^{3}$
            \bigbreak \noindent 
            \textbf{Note:} Always round s.t $m \in \mathbb{R}$ 
        \item \textbf{Order of Magnitude (Method two)}: We begin by writing the number in scientific notation. For example, suppose we want to find the order of magnitude for $800$, we first write 
            \begin{align*}
                8 \cdot 10^{2}
            .\end{align*}
            We then check to see if the first factor is greater or less than $\sqrt{10} \approx 3$
            \begin{itemize}
                \item If it is less than $\sqrt{10} \approx  3$, we round it down to one
                \item If it is greater than $\sqrt{10} \approx  3$, we round it up to ten
            \end{itemize}
            Since 8 is greater than $\sqrt{10} \approx  3$, our number becomes $10 \cdot 10^{2} = 10^{2+1} = 10^{3}$. Thus, we have order $10^{3}$
        \item The SI unit for time is the \textbf{second}
        \item The SI unit for length is the \textbf{meter}
        \item The SI unit for mass is the \textbf{kilogram}
        \item The SI unit for weight is the \textbf{Newton}
        \item \textbf{$10^{3}$ (Megagram) is also called a \textit{metric ton}, abbreviated $t$}
        \item \textbf{Average speed}
            \begin{align*}
                \frac{\text{distance}}{\text{time}} = \frac{d}{t}
            .\end{align*}
        \item \textbf{Dimension notation}: Square brackets
            \begin{align*}
                [r] = L
            .\end{align*}
        \item \textbf{Checking for dimensional consistency}: 
            check that each term in a given equation has the same dimensions as the other terms in that equation and that the arguments of any standard mathematical functions are dimensionless.
        \item \textbf{Dimension for velocity}
            \begin{align*}
                [v] = LT^{-1}
            .\end{align*}
        \item \textbf{Dimension for acceleration}
            \begin{align*}
                [a] = LT^{-2}
            .\end{align*}
        \item \textbf{Dimensions for derivatives}
            \begin{align*}
                \left[\frac{dv}{dt}\right] = \frac{[v]}{[t]}
            .\end{align*}
        \item \textbf{Dimensions for integrals}
            \begin{align*}
                \left[ \int v \, dt \right] = [v] \cdot [t].
            .\end{align*}
        \item \textbf{Volume}
            \begin{align*}
                V = AD
            .\end{align*}
            Where $A$ is the area and $D$ is the depth
        \item \textbf{Mass}
            \begin{align*}
                M = \rho V
            .\end{align*}
            Where $\rho$ is the density and $V$ is the volume
        \item \textbf{Density (Volume density)}
            \begin{align*}
                \rho = \frac{M}{V}
            .\end{align*}
            Where $M$ is the mass and $V$ is the volume
        \item \textbf{Density (Area density)}
            \begin{align*}
                \rho = \frac{M}{A}
            .\end{align*}
            Where $M$ is the mass and $A$ is the surface area
        \item \textbf{Geometric mean of bounds}
            \begin{align*}
                (order_{1} \times order_{2})^{0.5}
            .\end{align*}
        \item \textbf{Significant figures}: 
            \begin{itemize}
                \item All non-zero numbers are Significant.
                    \begin{align*}
                        563 \rightarrow 3
                    .\end{align*}
                \item Zeros are significant if they reside between two significant figures.
                    \begin{align*}
                        5002 \rightarrow 4
                    .\end{align*}
                \item Leading zeros are never significant. 
                    \begin{align*}
                        0.0056 \rightarrow 2
                    .\end{align*}
                \item Trailing zeros without a decimal point are not significant
                    \begin{align*}
                        500 \rightarrow 1
                    .\end{align*}
                \item Trailing zeros with a decimal point are significant. The decimal point indicates that the zeros are measured and are significant. 
                    \begin{align*}
                        500.0 \rightarrow 4
                    .\end{align*}
            \end{itemize}
        \item \textbf{Percent uncertainty}: If a measurement \( A \) is expressed with uncertainty \( \delta A \), the percent uncertainty is defined as
            \[
                \text{Percent uncertainty} = \frac{\delta A}{A} \times 100\%.
            \]
            where $A$ is the average, and $\delta A$ is the margin of error
            \bigbreak \noindent 
            \textbf{Note:} The value of the PU will take the place of the margin of error, so something like $5.1\ lbs \pm 0.3\ lbs $ will become $5.1 \pm 6 \% $
        \item \textbf{We can find the margin of error ($\delta A$) by taking half of the range}
        \item \textbf{If the measurements going into the calculation have small uncertainties (a few percent or less), then the method of adding percents can be used for multiplication or division. This method states the percent uncertainty in a quantity calculated by multiplication or division is the sum of the percent uncertainties in the items used to make the calculation. For example, if a floor has a length of \(4.00 \, \text{m}\) and a width of \(3.00 \, \text{m}\), with uncertainties of \(2\%\) and \(1\%\), respectively, then the area of the floor is \(12.0 \, \text{m}^2\) and has an uncertainty of \(3\%\). (Expressed as an area, this is \(0.36 \, \text{m}^2\) [\(12.0 \, \text{m}^2 \times 0.03\)], which we round to \(0.4 \, \text{m}^2\) since the area of the floor is given to a tenth of a square meter.)}
        \item \textbf{    When combining measurements with different degrees of precision with the mathematical operations of addition, subtraction, multiplication, or division, then the number of significant digits in the final answer can be no greater than the number of significant digits in the least-precise measured value. There are two different rules, one for multiplication and division and the other for addition and subtraction. There are two different rules}
            \begin{itemize}
                \item \textbf{For multiplication and division}, the result should have the same number of significant figures as the quantity with the least number of significant figures entering into the calculation 
                \item \textbf{For addition and subtraction}, the answer can contain no more decimal places than the least-precise measurement.
            \end{itemize}
        \item \textbf{If a quantity increases n\%, that is the same as saying that it is multiplied by a factor of}
            \begin{align*}
                1 + \left(\frac{n}{100}\right)
            .\end{align*}
        \item \textbf{If a quantity decreases n\%, that is the same as saying that it is multiplied by a factor of}
            \begin{align*}
                1 - \left(\frac{n}{100}\right)
            .\end{align*}
        \item \textbf{Proportional notation}: Suppose $A$ is proportional to $B$, then we say
            \begin{align*}
               A \alpha B 
            .\end{align*}
            This means if $B$ increases by some factor, then $A$ must increase by the same factor.
            \bigbreak \noindent 
            In other words, the ratio of two values of $B$ is equal to the ratio of the corresponding values of $A$:
            \begin{align*}
                \frac{B_{2}}{B_{1}} = \frac{A_{2}}{A_{1}}
            .\end{align*}
            \bigbreak \noindent 
            Ex: Given the circumference formula for a circle
            \begin{align*}
               c =2\pi r
            .\end{align*}
            We can say
            \begin{align*}
                C \alpha r
            .\end{align*}
            If the radius doubles, the circumference also doubles
        \item \textbf{Numbers that are exact (defined) have an infinite number of significant figures} because they are not measurements with any uncertainty, but rather are defined values or counts of discrete objects. For example, considering a conversion of 93.4 beats/min to beats/hour, we would compute
            \begin{align*}
            &\frac{93.4\ beats}{1\ m} \cdot \frac{60\ m}{1\ h} \\
            &=5604\ b/h
            .\end{align*}
            Since the number 60 is an exact number (it is a defined conversion factor), we should report the answer with three significant figures (since 93.4 has three). Thus, we round to 5600. To explicitly express 5600 with three sig figs, we write as $5.60 \cdot 10^{3} $
        \item \textbf{Explicity express sig figs with scientific notation}
            \begin{align*}
                &5600 \rightarrow 2 \\
                &5.60 \cdot 10^{3} \rightarrow 3
            .\end{align*}
        \item \textbf{Forms of scientific notation} 
            \begin{align*}
                10^{10} = 1.0 \cdot 10^{10} = 1.0e+10
            .\end{align*}
        \item \textbf{Decimal places in additon and subtraction}: consider the example
            \begin{align*}
                501.258313 + 54.5235 + 350.257 = 906.038813
            .\end{align*}
            We would report the value as 
            \begin{align*}
                906.039
            .\end{align*}
            Because the least precise measurement has three decimal points
        \item unless you are working on lab-related calculations (ie in your lab reports) or the question explicitly asks you to pay attention to sig figs, \textbf{you can err on the side of including extra ones}
        \item \textbf{Weight is the measure of the force exerted on an object due to gravity. It is calculated as the mass of the object multiplied by the acceleration due to gravity.}
        \item \textbf{Newton}
            \begin{align*}
                N = kg \cdot \frac{m}{s^{2}}
            .\end{align*}
    \end{itemize}

    \pagebreak 
    \subsubsection{Fundemental Tables and figures}
    \begin{itemize}
        \item \textbf{Known ranges of length, mass, and time}
            \begin{center}
                \includegraphics[scale=.7]{./figures/physfig1.jpeg  }
            \end{center}
        \item \textbf{SI Units: Base and Derived Units}
            \bigbreak \noindent 
            \begin{tabularx}{\textwidth}{|X|X|}
                \hline
                ISQ Base Quantity & SI Base Unit \\
                \hline
                Length & meter (m) \\
                Mass & kilogram (kg) \\
                Time & second (s) \\
                Electrical current & ampere (A) \\
                Thermodynamic temperature & kelvin (K) \\
                Amount of substance & mole (mol) \\
                Luminous intensity & candela (cd) \\
                \hline
            \end{tabularx}
            \pagebreak 
        \item \textbf{Metric Prefixes}
            \begin{center}
                \begin{tabularx}{\textwidth}{|X|X|X|X|X|X|}
                    \hline
                    Prefix & Symbol & Meaning & Prefix & Symbol & Meaning \\ 
                    \hline
                    yotta- & Y & $10^{24}$ & yocto- & y & $10^{-24}$ \\
                    zetta- & Z & $10^{21}$ & zepto- & z & $10^{-21}$ \\
                    exa-   & E & $10^{18}$ & atto-  & a & $10^{-18}$ \\
                    peta-  & P & $10^{15}$ & femto- & f & $10^{-15}$ \\
                    tera-  & T & $10^{12}$ & pico-  & p & $10^{-12}$ \\
                    giga-  & G & $10^9$   & nano-  & n & $10^{-9}$  \\
                    mega-  & M & $10^6$   & micro- & $\mu$ & $10^{-6}$ \\
                    kilo-  & k & $10^3$   & milli- & m & $10^{-3}$ \\
                    hecto- & h & $10^2$   & centi- & c & $10^{-2}$ \\
                    deka-  & da & $10^1$  & deci-  & d & $10^{-1}$ \\
                    \hline
                \end{tabularx}
            \end{center}
            \bigbreak \noindent 
        \item \textbf{Base Quantity \& Symbol for Dimension}
            \begin{center}
                \begin{tabularx}{\textwidth}{|X|X|}
                    \hline
                    Base Quantity & Symbol for Dimension \\
                    \hline
                    Length & L \\
                    Mass & M \\
                    Time & T \\
                    Current & I \\
                    Thermodynamic temperature & $\Theta$ \\
                    Amount of substance & N \\
                    Luminous intensity & J \\
                    \hline
                \end{tabularx}
            \end{center}
    \end{itemize}

    \pagebreak 
    \subsubsection{Memorize conversions}
    \begin{itemize}
        \item \textbf{Newton to Pounds}: 
            \begin{align*}
                1N = 0.225lbs
            .\end{align*}
        \item \textbf{Pounds to newtons}
            \begin{align*}
                1lb = 4.448\ N
            .\end{align*}
        \item \textbf{Length/Distance }
            \begin{itemize}
                \item 1 inch (in) = 2.54 centimeters (cm)
                \item 1 centimeter (cm) = 0.393701 inches (in)
                \item 1 foot (ft) = 0.3048 meters (m)
                \item 1 meter (m) = 3.28 feet (ft)
                \item 1 mile (mi) = 1.6 kilometers (km)
                \item 1 kilometer (km) = 0.621371 miles (mi)
            \end{itemize}
    \item \textbf{Weight to mass or mass to weight}
        \begin{itemize}
            \item 1 pound = 0.453592 kilograms.
            \item 1 kilogram = 2.2 pounds.
        \end{itemize}
    \end{itemize}



    \pagebreak 
    \subsubsection{Problems to remember}
    \begin{itemize}
        \item \textbf{Restating mass}: Restate the mass  $1.93 \times 10^{13}$ using a metric prefix such that the resulting numerical value is bigger than one but less than 1000.
            \bigbreak \noindent 
            First, we must restate in terms of grams. Since $1kg = 10^{3}g$, we write
            \begin{align*}
                &1.93 \times 10^{13} \times 10^{3}g \\
                &1.93 \times 10^{16}g
            .\end{align*}
            Since $1Pg = 10^{15}g$, we can write
            \begin{align*}
               1.93 \times 10^{1}Pg 
            .\end{align*}
            Since $16-15 = 1 $
        \item \textbf{Unit conversion}: The distance from the university to home is 10 mi and it usually takes 20 min to drive this distance. Calculate the average speed in meters per second (m/s). (Note: Average speed is distance traveled divided by time of travel.)
        \bigbreak \noindent 
        \textbf{Note:} There are 1609 meters in 1 mile
        \bigbreak \noindent 
        First, we can compute the average speed with the units given $\left(\frac{miles}{minute}\right)$
        \begin{align*}
            \text{Average Speed} = \frac{\text{miles}}{\text{minute}} = \frac{10}{20} = 0.5\ mi/min
        .\end{align*}
        \bigbreak \noindent 
        Now we simply convert to m/s
        \begin{align*}
            &\frac{0.5\ \cancel{mi}}{1\ \cancel{min}} \times \frac{1\ \cancel{min}}{60\ sec} \times \frac{1609\ m}{1\ \cancel{mi}} \\
            &\approx 13 m/s
        .\end{align*}
    \item \textbf{Unit conversion}: The density of iron is  $7.86 g/cm^{3}$ under standard conditions. Convert this to $kg/m^{3}$.
        \begin{align*}
            &\frac{7.86\ g}{1\ cm^{3}} \times \left(\frac{100\ cm}{1\ m}\right)^{3} \times \frac{1\ kg}{1000\ g} \\
            &= \frac{7.86(100^{3})(1\ kg)}{1000(1\ m)} \\
            &=7.86 \cdot 10^{3} kg/m^{3}
        .\end{align*}
    \item \textbf{Proportional}: I found that if I drive my car 110 miles, I use 4 gallons ofgas. If I assume that the relationship between gas guzzled and distance driven is linearly proportional, how many gallons of gas do I use if I drive 275 miles?
        \bigbreak \noindent 
        To answer this lets find the linear equation
        \begin{align*}
            4\ gal = k(100\ mi)
        .\end{align*}
        Where $k$ is some arbitrary factor, Since we know the relationship is proportional, we can write 
        \begin{align*}
            &\frac{X}{4 gal} = \frac{k(275\ mi)}{k(110\ mi)} \\
            &X=10\ gal
        .\end{align*}

    \end{itemize}

    \pagebreak 
    \subsection{Chapter 2: Vectors}

    \smallbreak \noindent
    \subsubsection{Vocabulary}
    \begin{itemize}
        \item Many familiar physical quantities can be specified completely by giving a single number and the appropriate unit. For example, “a class period lasts 50 min” or “the gas tank in my car holds 65 L” or “the distance between two posts is 100 m.” A physical quantity that can be specified completely in this manner is called a \textbf{scalar quantity}. Scalar is a synonym of “number.” Time, mass, distance, length, volume, temperature, and energy are examples of \textbf{scalar quantities}.
        \item possible. Physical quantities specified completely by giving a number of units (magnitude) and a direction are called \textbf{vector quantities}. Examples of vector quantities include displacement, velocity, position, force, and torque.
    \end{itemize}

    \pagebreak 
    \subsubsection{Definitions and theorems (and important things)}
    \begin{itemize}
        \item Vectors (vector quantities) have a \textbf{number of units (magnitude)}, and a \textbf{direction}
        \item \textbf{Algebraic Operations with vectors}
            \begin{itemize}
                \item We can \textbf{add or subtract} two vectors
                \item we can \textbf{multiply} a vector by a scalar or by another vector
                \item We \textbf{cannot} divide by a vector. The operation of division by a vector is not defined.
            \end{itemize}
        \item \textbf{Vector Notation}: We denote a vector with a bold face letter with an arrow above it. For example
            \begin{align*}
                \vec{\textbf{V}}
            .\end{align*}
        \item \textbf{Displacement}: General term used to describe change in position.
        \item \textbf{The magnitude of a vector is the length of the arrow used to represent it}

        \item \textbf{Vector relations}
            \fig{.8}{./figures/vectorimage1.jpeg}

        \item \textbf{Scalars}: When a vector $\vec{A}$ is multiplied by a positive scalar $\alpha$, the result is a new vector $\vec{B}$ that is parallel to $\vec{A}$:
            \begin{align*}
                \vec{\textbf{B}} = \alpha \vec{\textbf{A}}
            .\end{align*}
            The magnitude of this new vector $\vec{\textbf{B}}$ is 
            \begin{align*}
                B = |\alpha|A
            .\end{align*}
            Where $B$ is the magnitude of $\vec{\textbf{B}} $ and $A$ is the magnitude of $\vec{\textbf{A}} $ 
        \item \textbf{Anti parallel vectors (opposing directions)}: Suppose we have two vectors $\vec{\textbf{A}}$ and $\vec{\textbf{B}}$ of equal magnitude. If there are antiparallel, we write
            \begin{align*}
                \vec{\textbf{A}} = -\vec{\textbf{B}} 
            .\end{align*}
        \item The vector sum of two (or more) vectors is called the \textbf{resultant vector} or, for short, the \textbf{resultant}
        \item The vector obtained by adding two vectors is called the \textbf{resultant}
        \item \textbf{Vector Laws }
            \begin{itemize}
                \item \textbf{Communitive law}: $\vec{A} + \vec{B} = \vec{B} + \vec{A}$
                \item \textbf{Assosiative law}: $(\vec{A} + \vec{B}) + \vec{C} = \vec{A} + (\vec{B} + \vec{C})$ 
                \item \textbf{Distributive law}: $\alpha_1 \vec{A} + \alpha_2 \vec{A} = (\alpha_1 + \alpha_2)\vec{A}$
            \end{itemize}
        \item \textbf{Vector addition}
            \begin{align*}
               \vec{\textbf{A}} + \vec{\textbf{B}} 
            .\end{align*}
            When two vectors are parallel, we can simply sum their magnitudes. However, if the vectors lie in different directions, the approach for vector addition involves finding their $x$ and $y$ components, summing them and then finding the magnitude. 
        \item \textbf{Vector Subtraction}
            \begin{align*}
                \vec{\textbf{A}} + (-\vec{\textbf{B}})
            .\end{align*}
            \bigbreak \noindent 
            When two vectors are aligned but point in exactly opposite directions, you can subtract their magnitudes (assuming you define one direction as positive and the other as negative) to find the net effect. This is a specific case of adding magnitudes where the direction is implicitly considered through subtraction.
        \item \textbf{Unit Vector notation}:  A unit vector in a normed vector space is a vector of length 1. We declare unit vectors with a hat instead of an arrow, consider the following example
            \begin{align*}
                \hat{\textbf{u}}
            .\end{align*}
            We usually denote the unit vector along the positive x-axis $\hat{i}$, the unit vector along the positive y-axis $\hat{j}$, and the unit vector along the positive z-axis $\hat{k}$
        \item \textbf{Unit vector example}: For example, instead of saying vector $\vec{D}_{AB}$ has a magnitude of $6.0\,\text{km}$ and a direction of northeast, we can introduce a unit vector $\hat{u}$ that points to the northeast and say succinctly that $\vec{D}_{AB} = (6.0\,\text{km})\hat{u}$. Then the southwesterly direction is simply given by the unit vector $-\hat{u}$. In this way, the displacement of $6.0\,\text{km}$ in the southwesterly direction is expressed by the vector
            \begin{align*}
                    \vec{D}_{BA} = (-6.0\,\text{km})\hat{u}.
            .\end{align*}
        \item \textbf{Parallelogram rule for resultant or difference vectors in two dimensions.}:
            \bigbreak \noindent 
            \fig{.8}{./figures/res.jpeg}
            \bigbreak \noindent 
            It follows from the parallelogram rule that neither the magnitude of the resultant vector nor the magnitude of the difference vector can be expressed as a simple sum or difference of magnitudes A and B, because the length of a diagonal cannot be expressed as a simple sum of side lengths.
        \item \textbf{tail-to-head geometric construction}
            \bigbreak \noindent 
            \fig{.8}{./figures/res2.jpeg}
        \item \textbf{Vector $x$ and $y$ components}: The $x$ component can be denoted, $\vec{A_{x}}$ the $y$ component can be denoted $\vec{A_{y}}$. Thus, the vector can be represented as
            \begin{align*}
                &\vec{A} = \vec{A_{x}} + \vec{A_{y}}
            .\end{align*}
        \item \textbf{Unit vectors of the axes}:     It is customary to denote the positive direction on the $x$-axis by the unit vector $\hat{i}$ and the positive direction on the $y$-axis by the unit vector $\hat{j}$. Unit vectors of the axes, $\hat{i}$ and $\hat{j}$, define two orthogonal directions in the plane. The $x$- and $y$- components of a vector can now be written in terms of the unit vectors of the axes:
            \bigbreak \noindent 
               \begin{equation}
                    \begin{cases}
                        \vec{A}_{x} &= A_{x}\hat{i}  \\
                         \vec{A}_{y} &= A_{y}\hat{j}  
                    \end{cases}
                \end{equation}
        \item \textbf{Component form of a vector}:
            \begin{align*}
                \vec{A} = A_{x}\hat{i} + A_{y}\hat{j}
            .\end{align*}
        \item \textbf{Finding components given initial and terminal points}: If we know the coordinates $b(x_b, y_b)$ of the origin point of a vector (where $b$ stands for "beginning") and the coordinates $e(x_e, y_e)$ of the end point of a vector (where $e$ stands for "end"), we can obtain the scalar components of a vector simply by subtracting the origin point coordinates from the end point coordinates:
            \begin{align*}
                A_x &= x_e - x_b  \\
                A_y &= y_e - y_b
            .\end{align*}
        \item \textbf{Magnitude $A$ of a vector with components $A_{x}$ and $A_{y}$}
            \begin{align*}
               &A^{2}  = A_{x}^{2} + A_{y}^{2} \\
               &A = \sqrt{A_{x}^{2} + A_{y}^{2}}
            .\end{align*}
            This equation works even if the scalar components of a vector are negative.
        \item \textbf{Finding theta for a vector (used for direction angles)}
            \begin{align*}
                \tan{\theta } = \frac{A_{y}}{A_{x}}
            .\end{align*}
        \item \textbf{Direction angles for vectors in the first}:
        \item \textbf{Direction angles for vectors in the second quadrant}:
            \begin{align*}
                \theta_{A} = \theta 
            .\end{align*}
        \item \textbf{Direction angles for vectors in the second quadrant}:
            \begin{align*}
                \theta_{A} = 180 - \theta 
            .\end{align*}
        \item \textbf{Direction angles for vectors in the second quadrant}:
            \begin{align*}
                \theta_{A} = 180 + \theta 
            .\end{align*}
        \item \textbf{Direction angles for vectors in the fourth quadrant}:
            \begin{align*}
                \theta_{A} = 360 - \theta 
            .\end{align*}
        \item \textbf{Finding $A_{x}$ and $A_{y}$  when the magnitude and direction angle are known}
        \begin{equation}
            \begin{cases}
                &A_{x} = A\cos{\theta_{A}} \\
                &A_{y} = A\sin{\theta_{A}} \\
            \end{cases}
        \end{equation}
        \item \textbf{Polar form}
            \begin{align*}
                &x = r\cos{\varphi} \\
                &y = r\sin{\varphi}
            .\end{align*}
        \item \textbf{Three dimensional plane}
            \bigbreak \noindent 
            \fig{.8}{./figures/zaxis.jpeg}
        \item \textbf{$z$ component of a vector}:
            \begin{align*}
                \vec{A}_{z} = A_{z}\hat{k} 
            .\end{align*}
            Where $A_{z}$ is given by 
            \begin{align*}
                z_{e} - z_{b}
            .\end{align*}
        \item \textbf{Vector in three dimensions}: A vector in three-dimensional space is the vector sum of its three vector component. 
            \begin{align*}
                \vec{A} = A_x \hat{i} + A_y \hat{j} + A_z \hat{k}.
            .\end{align*}
        \item \textbf{Magnitude of a vector in three dimensions}
            \begin{align*}
                A = \sqrt{A_x^2 + A_y^2 + A_z^2}.
            .\end{align*}
        \item \textbf{Null vector}: Denoted by 
            \begin{align*}
                \vec{0}
            .\end{align*}
            Has all components 0. Thus, 
            \begin{align*}
                \vec{0} = 0\hat{i} + 0\hat{j} + 0\hat{k}
            .\end{align*}
            Thus, it has no direction and no length
        \item Two vectors $\vec{A}$ and $\vec{B}$ are \textbf{equal vectors} if and only if their difference is the null vector:
        Hence, we can write $\vec{A} = \vec{B}$ if and only if the corresponding components of vectors $\vec{A}$ and $\vec{B}$ are equal:
        \[
        \vec{A} = \vec{B} \Leftrightarrow
        \left\{
            \begin{array}{l}
                A_x = B_x \\
                A_y = B_y \\
                A_z = B_z
            \end{array}
        \right.
        \]
    \item \textbf{Components of a resultant vector}:
        \begin{equation}
            \begin{cases}
                R_{x} = A_{x}  + B_{x} \\
                R_{y} = A_{y}  + B_{y} \\
                R_{z} = A_{z}  + B_{z} 
            \end{cases}
        \end{equation}

    \item \textbf{components of a resultant of many vectors}: if we are to sum up $N$ vectors $\vec{F}_1, \vec{F}_2, \vec{F}_3, \ldots, \vec{F}_N$, where each vector is $\vec{F}_k = F_{kx}\hat{i} + F_{ky}\hat{j} + F_{kz}\hat{k}$, the resultant vector $\vec{F}_R$ is
           \begin{equation}
                \begin{cases}
                    F_{R_{x}} = \summation{N}{k=1}\ F_{kx}\ = F_{1x} + F_{2x} + F_{3x} + ... + F_{Nx} \\
                    F_{R_{y}} = \summation{N}{k=1}\ F_{ky}\ = F_{1y} + F_{2y} + F_{3y} + ... + F_{Ny} \\
                    F_{R_{z}} = \summation{N}{k=1}\ F_{kz}\ = F_{1z} + F_{2z} + F_{3z} + ... + F_{Nz} \\
                \end{cases}
            \end{equation}
        With the component form 
        \begin{align*}
            \vec{F_{R}} = F_{R_{x}}\hat{i} + F_{R_{y}}\hat{j} + F_{R_{z}}\hat{k}
        .\end{align*}
    \item \textbf{Finding unit vector (Direction) of some vector}: Suppose we have some vector $\vec{V}$. Then 
        \begin{align*}
            \hat{V} = \frac{\vec{V}}{V}
        .\end{align*}
    \item \textbf{Dot product}
        \begin{align*}
            \vec{A} \cdot \vec{B} = AB\cos{\varphi} 
        .\end{align*}
        Where $\varphi$ is the angle between the vectors
    \item \textbf{Dot product of two parallel vectors}
        \begin{align*}
            \vec{A} \cdot \vec{B} = AB\cos{0^{\circ}} = AB
        .\end{align*}
    \item \textbf{Dot product of two anti-parallel vectors}
        \begin{align*}
            \vec{A} \cdot \vec{B} = AB\cos{180^{\circ}} = -AB
        .\end{align*}
    \item \textbf{Dot product of two orthogonal vectors}
        \begin{align*}
            \vec{A} \cdot \vec{B} = AB\cos{90^{\circ}} =  0
        .\end{align*}
    \item \textbf{Dot product of a vector with itself}
        \begin{align*}
            \vec{A} \cdot \vec{A} = AA\cos{0} = A^{2}
        .\end{align*}
    \item \textbf{Dot products of unit vectors}: Scalar products of the unit vector of an axis with other unit vectors of axes always vanish (equals 0) because these unit vectors are orthogonal:
    \item \textbf{Dot product of the same unit vector}: The dot product of the same unit vector is 1
        \begin{align*}
            \hat{i} \cdot \hat{i} = i^{2} = 1
        .\end{align*}
    \item \textbf{Using dot product to find scalar x-component}
        \begin{align*}
            \vec{A} \cdot \hat{i} = \norm{\vec{A}} \norm{\hat{i}}\cos{\theta_{A}} = A\cos{\theta_{A}}= A_{x}
        .\end{align*}
    \item \textbf{Using dot product to find scalar y-component}
        \begin{align*}
            \vec{A} \cdot \hat{j} = \norm{\vec{A}} \norm{\hat{j}}\cos{(90^{\circ} -\theta_{A})}= A\sin{\theta_{A}} = A_{y}
        .\end{align*}
    \item \textbf{Trig complementary angles}: For a right angle triangle, the sine of the complemenary angle is the cosine of the angle. And vice versa
        \begin{align*}
            &\sin{90-\theta} = \cos{\theta} \\
            &\cos{90-\theta} = \sin{\theta}
        .\end{align*}
    \item \textbf{Dot product second method of computation}
        \begin{align*}
            \vec{A} \cdot \vec{B} = A_{x}B_{x} + A_{y}B_{y} + A_{z}B_{z}
        .\end{align*}
    \item \textbf{Equation for $\cos{(\varphi)}$}
        \begin{align*}
            \cos{(\varphi)} = \frac{\vec{A} \cdot \vec{B}}{AB}
        .\end{align*}
    \item The \textbf{Si unit for work} is the joule (J), where 
        \begin{align*}
            1\ \text{J} = 1\ \text{N} \cdot \text{m} 
        .\end{align*}
    \item \textbf{The Work of a Force}: When force $\vec{F}$ pulls on an object and when it causes its displacement $\vec{D}$, we say the force performs work. The amount of work the force does is the scalar product $\vec{F} \cdot \vec{D}$.
    \item \textbf{Cross Product (Vector Product)}: 
        The vector product of two vectors $\vec{A}$ and $\vec{B}$ is denoted by $\vec{A} \times \vec{B}$ and is often referred to as a cross product. The vector product is a vector that has its direction perpendicular to both vectors $\vec{A}$ and $\vec{B}$. In other words, vector $\vec{A} \times \vec{B}$ is perpendicular to the plane that contains vectors $\vec{A}$ and $\vec{B}$. The magnitude of the vector product is defined as
        \begin{equation}
            \lVert \vec{A} \times \vec{B} \rVert = AB \sin \varphi,
        \end{equation}
        where angle $\varphi$, between the two vectors, is measured from vector $\vec{A}$ (first vector in the product) to vector $\vec{B}$ (second vector in the product), as indicated in Figure 2.29, and is between $0^\circ$ and $180^\circ$.
    \item     The \textbf{vector product vanishes} for pairs of vectors that are either \textbf{parallel} ($\varphi=0^\circ$) or \textbf{antiparallel} ($\varphi=180^\circ$) because $\sin 0^\circ = \sin 180^\circ = 0$.
    \item The cross product is \textbf{anti-communitive}
        \begin{align*}
            \vec{A} \times \vec{B} = -\vec{B} \times \vec{A}.
        .\end{align*}
    \item \textbf{Torque}: Denoted with the greek letter "tau" is the vector product of the distance between the pivot to force with the force:
        \begin{align*}
            \vec{\tau} = \vec{R} \times \vec{F}
        .\end{align*}
    \item \textbf{the cross product has the following distributive property:}
        \begin{align*}
            \vec{A} \times (\vec{B} + \vec{C}) = \vec{A} \times \vec{B} + \vec{A} \times \vec{C}
        .\end{align*}
    \item \textbf{Cross product of the same unit vectors}: The cross product of the same unit vectors is 0
    \item \textbf{Cross product between unit vectors}
        \begin{equation}
            \begin{cases}
                \hat{i} \times \hat{j} = +\hat{k} \\
                \hat{j} \times \hat{i} = -\hat{k} \\
                \hat{j} \times \hat{k} = +\hat{i} \\
                \hat{k} \times \hat{j} = -\hat{i} \\
                \hat{k} \times \hat{i} = +\hat{j} \\
                \hat{i} \times \hat{k} = -\hat{j}
            \end{cases}
        \end{equation}
        \bigbreak \noindent 
        \textbf{Notice:} The cross product of two different unit vectors is always a third unit vector.
    \item \textbf{Computation of the cross product}:
        \begin{align*}
            \vec{C} = \vec{A} \times \vec{B} = (A_yB_z - A_zB_y)\hat{i} + (A_zB_x - A_xB_z)\hat{j} + (A_xB_y - A_yB_x)\hat{k}
        .\end{align*}
        \pagebreak 
    \item \textbf{Average Velocity}: Velocity is the \textbf{vector quantity} of speed plus the direction. Velocity is 
        \begin{align*}
            \vec{v} &=\frac{\text{Displacement}}{Time} = \frac{\norm{\vec{d}}}{t}
        .\end{align*}
        Where $\norm{\vec{d}}$ is the magnitude of the displacement vector, and $t$ is the total elapsed time.
    \item To describe the position (location) of something, we give its distance from the orgin and its direction. This vector quantity is called the \textbf{position vector}, Denoted
        \begin{align*}
            \vec{r}
        .\end{align*}
        \bigbreak \noindent 
    \begin{figure}[ht]
        \centering
        \incfig{try2}
        \label{fig:try2}
    \end{figure}
    \item \textbf{Displacement} is defined as the change in the position vector. The final position vector minus the initial position vector
        \begin{align*}
            \Delta \vec{r} = \vec{r}_{f} - \vec{r}_{i}
        .\end{align*}
    \begin{figure}[ht]
        \centering
        \incfig{hellowrold}
        \label{fig:hellowrold}
    \end{figure}






    \end{itemize}

    \pagebreak 
    \subsubsection{Problems to remember}
    \begin{itemize}
        \item \textbf{Unit vectors}: A long measuring stick rests against a wall in a physics laboratory with its 200-cm end at the floor. A ladybug lands on the 100-cm mark and crawls randomly along the stick. It first walks 15 cm toward the floor, then it walks 56 cm toward the wall, then it walks 3 cm toward the floor again. Then, after a brief stop, it continues for 25 cm toward the floor and then, again, it crawls up 19 cm toward the wall before coming to a complete rest (Figure 2.8). Find the vector of its total displacement and its final resting position on the stick.
        \bigbreak \noindent 
        If we choose the direction along the stick toward the floor as the direction of unit vector $\hat{u}$, then the direction toward the floor is $+\hat{u}$ and the direction toward the wall is $-\hat{u}$. The ladybug makes a total of five displacements:
    \begin{align*}
        &\vec{D}_1 = (15\,\text{cm})(+\hat{u}), \quad \vec{D}_2  \\
        &= (56\,\text{cm})(-\hat{u}), \quad \vec{D}_3  \\
        &= (3\,\text{cm})(+\hat{u}), \quad \vec{D}_4  \\
        &= (25\,\text{cm})(+\hat{u}), \quad \text{and} \quad \vec{D}_5  \\
        &= (19\,\text{cm})(-\hat{u}).
    .\end{align*}
    The total displacement $\vec{D}$ is the resultant of all its displacement vectors.
    \bigbreak \noindent 
    The resultant of all the displacement vectors is
    \begin{align*}
        &\vec{D} = \vec{D}_1 + \vec{D}_2 + \vec{D}_3 + \vec{D}_4 + \vec{D}_5  \\
        &= (15\,\text{cm})(+\hat{u}) + (56\,\text{cm})(-\hat{u}) + (3\,\text{cm})(+\hat{u}) + (25\,\text{cm})(+\hat{u}) + (19\,\text{cm})(-\hat{u})  \\
        &= (15 - 56 + 3 + 25 - 19)\,\text{cm}\,\hat{u} = -32\,\text{cm}\,\hat{u}
    .\end{align*}
    In this calculation, we use the distributive law. The result reads that the total displacement vector points away from the 100-cm mark (initial landing site) toward the end of the meter stick that touches the wall. The end that touches the wall is marked 0 cm, so the final position of the ladybug is at the $(100 - 32)\,\text{cm} = 68\,\text{cm}$ mark.
    \end{itemize}

    \pagebreak 
    \subsection{Chapter 3: Motion along a straight line}
    \bigbreak \noindent 
    \subsubsection{Definitions and theorems}
    \begin{itemize}
        \item \textbf{Kinematics} is a subfield of physics, developed in classical mechanics, that describes the motion of points, bodies, and systems of bodies without considering the forces that cause them to move
        \item \textbf{Displacement}: Displacement $\Delta x$ is the change in position of an object:
            \begin{align*}
                \Delta x = x_f - x_0,
            .\end{align*}
            where $\Delta x$ is displacement, $x_f$ is the final position, and $x_0$ is the initial position.
        \item We define total displacement $\Delta x_{\text{Total}}$, as the sum of the individual displacements, and express this mathematically with the equation
            \begin{align*}
                \Delta x_{\text{Total}} = \sum \Delta x_i
            .\end{align*}
        \item \textbf{the distance traveled} is the sum of the magnitudes of the individual displacements:
            \begin{align*}
                x_{\text{total}} = \summation{n}{i=1}\ \Delta x_{i}
            .\end{align*}
        \item If the details of the motion at each instant are not important, the rate is usually expressed as the \textbf{average velocity}. If  $x_{1}$ and  $x_{2} $ are the positions of an object at times  $t_{1} $ and  $t_{2}$ , respectively, then 
            \begin{align*}
                \text{Average Velocity } =\ &\bar{v} =  \frac{\text{Displacement between two points}}{\text{Time needed to make the displacement}} \\
                &\bar{v} = \frac{\Delta x}{\Delta t} = \frac{x_{2} - x_{1}}{t_{2} - t_{1}}
            .\end{align*}
        
            This vector quantity is simply the total displacement between two points divided by the time taken to travel between them. The time taken to travel between two points is called the \textbf{elapsed time} $\Delta t$
        \item The \textbf{instantaneous velocity} of an object is the limit of the average velocity as the elapsed time approaches zero, or the derivative of $x$ with respect to $t$:
            \begin{align*}
                v(t) &= \lim\limits_{\Delta t \to 0}{\frac{f(t + \Delta t) - f(t)}{\Delta t}} \\
                 &= \frac{d}{dt}x(t) = \frac{d\vec{r}}{dt}.
            .\end{align*}
        \item We can calculate the \textbf{average speed} by finding the total distance traveled divided by the elapsed time:
            \begin{align*}
                \text{Average speed } = \bar{s} = \frac{\text{Total distance}}{\text{Elapsed time}}.
            .\end{align*}
        \item we can calculate the \textbf{instantaneous speed} from the magnitude of the instantaneous velocity:
            \begin{align*}
                \text{instantaneous speed } = \bigg\lvert v(t) \bigg\rvert
            .\end{align*}
        \item \textbf{Calculating Instantaneous Velocity}
            When calculating instantaneous velocity, we need to specify the explicit form of the position function $x(t)$. If each term in the $x(t)$ equation has the form of $A t^n$ where $A$ is a constant and $n$ is an integer, this can be differentiated using the power rule to be:
            \begin{align*}
                \frac{d(At^{n})}{dt}= Ant^{n-1}
            .\end{align*}
        \item \textbf{Average acceleration} is the rate at which velocity changes:
            \begin{align*}
                \bar{a} = \frac{\Delta v}{\Delta t} = \frac{v_f - v_0}{t_f - t_0},
            .\end{align*}
            where $\bar{a}$ is average acceleration, $v$ is velocity, and $t$ is time. (The bar over the $a$ means average acceleration.)
            \bigbreak \noindent 
            \textbf{Note:} acceleration occurs when velocity changes in magnitude (an increase or decrease in speed) or in direction, or both.
               \item \textbf{Acceleration as a vector:} Acceleration is a vector in the same direction as the change in velocity,  $\Delta v$ . Since velocity is a vector, it can change in magnitude or in direction, or both. Acceleration is, therefore, a change in speed or direction, or both.
         \bigbreak \noindent 
         Keep in mind that although acceleration is in the direction of the change in velocity, it is not always in the direction of motion. When an object slows down, its acceleration is opposite to the direction of its motion. Although this is commonly referred to as deceleration
        \item \textbf{Distance over constant acceleration}
            \begin{align*}
                d = \frac{1}{2}at^{2} 
            .\end{align*}
            Where $a$ is the acceleration, and $t$ is the time
        \item \textbf{instantaneous acceleration}
            \begin{align*}
            a(t) = \frac{d}{dt}v(t)
            .\end{align*}
        \item \textbf{Derivative of velocity function}: Suppose we have some function 
            \begin{align*}
                v(t) = (20 m/s) t - (10 m/s^{2})t^{2}
            .\end{align*}
            When we find the acceleration function, we are taking the derivative of the velocity function. Thus, we are finding the change in velocity with respect to time. Consequently, our terms become
            \begin{align*}
                a(t) = 20 m/s^{2} - (10m/s^{3})t
            .\end{align*}
            \textbf{Notice:} How we are dividing by an additional unit of time, thus the exponents for our seconds increases by one.
        \item \textbf{Simplified notation}: If we take initial time to be zero, and final quantitys without subscript, then we have 
            \begin{align*}
               &\Delta t = t\\
                &\Delta x = x - x_{0}\\
                &\Delta v = v- v_{0}\\
            .\end{align*}
        \item \textbf{Assumption of constant acceleration}: This assumption allows us to avoid using calculus to find instantaneous acceleration. Since acceleration is constant, the average and instantaneous accelerations are equal—that is,
            \begin{align*}
                \bar{a} = a = \text{constant}
            .\end{align*}
            Thus, we can use the symbol $a$ for acceleration at all times.
        \item \textbf{Final Position function} 
            \begin{align*}
                x = x_{0}  + \bar{v}t
            .\end{align*}
        \item \textbf{Average velocity under constant acceleration}
            \begin{align*}
                \bar{v} = \frac{v_{0} + v}{2} \quad \text{(Constant $a$)}
            .\end{align*}
            This reflects the fact that when acceleration is constant, $\bar{v}$ is just the simple average of the initial and final velocities.
        \item \textbf{Final Velocity function under constant acceleration}
            \begin{align*}
                v = v_{0} + at \quad \text{(Constant $a$)}
            .\end{align*}
        \item \textbf{Equation for final position under constant acceleration}
            \begin{align*}
                x = x_{0} +v_{0}t + \frac{1}{2}at^{2} \quad \text{(Constant $a$)}
            .\end{align*}
            When initial position and velocity are both zero, we have
            \begin{align*}
                x =  \frac{1}{2}at^{2} \quad \text{(Constant $a$)}
            .\end{align*}
        \item \textbf{relationships seen in final position under constant acceleration equation}
            \begin{itemize}
                \item Displacement depends on the square of the elapsed time when acceleration is not zero.
                \item If acceleration is zero, then initial velocity equals average velocity \(v_0 = \bar{v}\), and \(x = x_0 + v_0t + \frac{1}{2}at^2\) becomes \(x = x_0 + v_0t\).
            \end{itemize}
        \item \textbf{Final velocity equation (no time required)}
            \begin{align*}
                v^{2} = v_{0}^{2} + 2a(x-x_{0}) \quad \text{(Constant $a$)}
            .\end{align*}
        \item \textbf{additional insights into the general relationships among physical quantities:}
            \begin{itemize}
                \item The final velocity depends on how large the acceleration is and the distance over which it acts.
                \item For a fixed acceleration, a car that is going twice as fast doesn’t simply stop in twice the distance. It takes much farther to stop. (This is why we have reduced speed zones near schools.)
            \end{itemize}
        \item \textbf{acceleration in terms of velocities and displacement}
            \begin{align*}
                a  = \frac{v^{2} - v_{0}^{2}}{2(x-x_{0})}
            .\end{align*}
        \item \textbf{two-body pursuit problems}
            \begin{itemize}
                \item Find equations of motion for both bodies (with the same parameter)
                \item eliminate the parameter
                \item plug in knowns to solve for the unknown
            \end{itemize}
        \item if \textbf{air resistance and friction are negligible}, then in a given location all objects fall toward the center of Earth with the same \textbf{constant acceleration, independent of their mass.} 
        \item an object falling without air resistance or friction is defined to be in \textbf{free fall}
        \item The acceleration of free-falling objects is therefore called \textbf{acceleration due to gravity}
            \begin{align*}
                g = 9.8\ m/s^{2}
            .\end{align*}
            \textbf{Note:} If we define the upward direction as positive, then   $a=−g=-9.8m/s^{2},$ and if we define the downward direction as positive, then  $a=g=9.8m/s^{2}$
        \item \textbf{vertical displacement}: We denote vertical displacement with the symbol $y$
        \item \textbf{Kinematic equations for objects in free fall}: We assume here that acceleration equals -g (with the positive direction upward).
            \begin{align*}
                v &= v_{0} - gt \\
                y &= y_{0} + v_{0}t-\frac{1}{2}gt^{2} \\
                v^{2} &= v_{0}^{2} -2g(y-y_{0})
            .\end{align*}
        \item \textbf{Change in velocity}
            \begin{align*}
                \Delta v = a\Delta t
            .\end{align*}
            If you don't care about position this relates the acceleration,  intervals of time and changes in velocity
        \item \textbf{Change in position}
            \begin{align*}
                \Delta x = \frac{1}{2}(v_{f} - v_{i})\Delta t
            .\end{align*}
    \end{itemize}

    \pagebreak 
    \subsection{Chapter 4: Motion in two and three dimensions}
    \smallbreak \noindent
    \subsubsection{Definitions and theorems}
    \begin{itemize}
        \item \textbf{Location of a particle in space}
            \begin{align*}
                x &= x(t) \\
                y &= y(t) \\
                z &= z(t)
            .\end{align*}
            Where $x,y,z$ are functions of time ($t$). 
        \item \textbf{Position vector in space}
            \begin{align*}
                \vec{\mathbf{r}}(t) = x(t)\hat{\mathbf{i}} + y(t)\hat{\mathbf{j}} + z(t)\hat{\mathbf{k}}
            .\end{align*}
        \item \textbf{Displacement vector in space}: 
            Suppose we have a particle. At time $t_{1}$ the particle is located at $P_{1}$ with position vector $\vec{\mathbf{r}}(t_{1})$. At some later time $t_{2}$, the particle is located at $P_{2}$ with position vector $\vec{\mathbf{r}}(t_{2})$. The displacement vector $\vec{\Delta r}$ is found by subtracting $\vec{r}(t_1)$ from $\vec{r}(t_2)$:
            \begin{align*}
                \Delta \vec{\mathbf{r}} = \vec{\mathbf{r}}(t_{2}) - \vec{\mathbf{r}}(t_{1})
            .\end{align*}
            Thus, the displacement vector $\Delta \vec{\mathbf{r}} = \vec{\mathbf{r}}(t_{2}) - \vec{\mathbf{r}}(t_{1})$ is the vector from $P_{1}$ to $P_{2}$
        \item \textbf{Instantaneous velocity vector in two and three dimensions}: We can do the same operation in two and three dimensions, but we use vectors. The instantaneous velocity vector is now
            \begin{align*}
                \vec{\mathbf{v}} &= \lim\limits_{\Delta t \to 0}{\frac{\vec{\mathbf{r}}(t - \Delta t)-\vec{\mathbf{r}}(t)}{\Delta t}} \\
                &=\frac{d\vec{\mathbf{r}}}{dt}
            .\end{align*}
        \item \textbf{Velocity in component form}
            \begin{align*}
                \vec{\mathbf{v}}(t) = v_{x}\hat{\mathbf{i}} + v_{y}\hat{\mathbf{j}} + v_{z}\hat{\mathbf{k}}
            .\end{align*}
            Where 
            \begin{align*}
                v_{x}(t) = \frac{dx(t)}{dt}, \quad v_{y}(t) = \frac{dy(t)}{dt}, \quad v_{z}(t) = \frac{dz(t)}{dt}
            .\end{align*}
        \item \textbf{Average velocity in two and three dimensions}: 
            If only the average velocity is of concern, we have the vector equivalent of the one-dimensional average velocity for two and three dimensions
            \begin{align*}
                \vec{\mathbf{v}}_{\text{avg}} = \frac{\vec{\mathbf{r}}(t_{2})-\vec{\mathbf{r}}(t_{1})}{t_{2} - t_{1}}
            .\end{align*}
        \item \textbf{The Independence of Perpendicular Motions}: When we look at the three-dimensional equations for position and velocity written in unit vector notation, we see the components of these equations are separate and unique functions of time that do not depend on one another. Motion along the x direction has no part of its motion along the y and z directions, and similarly for the other two coordinate axes. Thus, the motion of an object in two or three dimensions can be divided into separate, independent motions along the perpendicular axes of the coordinate system in which the motion takes place.
        \item \textbf{Independence of motion}: In the kinematic description of motion, we are able to treat the horizontal and vertical components of motion separately. In many cases, motion in the horizontal direction does not affect motion in the vertical direction, and vice versa.
        \item \textbf{Instantaneous Acceleration}: This acceleration vector is the instantaneous acceleration and it can be obtained from the derivative with respect to time of the velocity function. The only difference in two or three dimensions is that these are now vector quantities. Taking the derivative with respect to time $\vec{\mathbf{v}}(t) $. We find
            \begin{align*}
               \vec{\mathbf{a}}(t) = \frac{d\vec{\mathbf{v}}(t)}{dt} 
            .\end{align*}
        \item \textbf{The acceleration in terms of components is}
            \begin{align*}
                \vec{\mathbf{a}}(t) = \frac{dv_{x}(t)}{dt}\hat{\mathbf{i}} + \frac{dv_{y}(t)}{dt}\hat{\mathbf{j}} + \frac{dv_{z}(t)}{dt}\hat{\mathbf{k}}
            .\end{align*}
        \item \textbf{acceleration in terms of the second derivative of the position function:}
            \begin{align*}
                \vec{\mathbf{a}}(t) = \frac{d^{2}x(t)}{dt^{2}}\hat{\mathbf{i}} + \frac{d^{2}y(t)}{dt^{2}}\hat{\mathbf{j}} + \frac{d^{2}z(t)}{dt^{2}}\hat{\mathbf{k}}
            .\end{align*}
        \item \textbf{Equations for position and velocity in the two and three dimensions (8)}. For simplicity I will only list the ones for the x-direction. However, keep in mind that these same equations hold for $y$ and $z$
            \begin{itemize}
                \item \textbf{Position with initial position and average velocity:} 
                    \begin{align*}
                        x(t)&=x_0+\left(v_x\right)_{\text {avg }} t
                    .\end{align*}
                \item \textbf{Final velocity with initial velocity, acceleration, and time}
                    \begin{align*}
                        v_x(t)&=v_{0 x}+a_x t
                    .\end{align*}
                \item \textbf{Position}
                    \begin{align*}
                        x(t)&=x_0+v_{0 x} t+\frac{1}{2} a_x t^2
                    .\end{align*}
                \item \textbf{Velocity with initial velocity, acceleration, final and initial position}
                    \begin{align*}
                        v_x^2(t)&=v_{0 x}^2+2 a_x\left(x-x_0\right) 
                    .\end{align*}
            \end{itemize}
        \item \textbf{These equations can be substituted into the equations for the position and velocity vectors in component form, and velocity vector in component form without the z-component to obtain the position vector and velocity vector as a function of time in two dimensions:}
            \begin{align*}
                \vec{\mathbf{r}}(t) &= x(t)\hat{\mathbf{i}} + y(t)\hat{\mathbf{j}}\\
                \vec{\mathbf{v}}(t)&= v_{x}(t)\hat{\mathbf{i}} + v_{y}(t)\hat{\mathbf{j}}
            .\end{align*}
        \item \textbf{Projectile motion} is the motion of an object thrown or projected into the air, subject only to acceleration as a result of gravity. Such objects are called \textbf{projectiles} and their path is called a \textbf{trajectory}
        \item \textbf{Acceleration in projectile motion}: Defining the positive direction to be upward, the components of acceleration are then very simple:
            \begin{align*}
                a_{y} &= -9.8\ m/s^{2} \\
                a_{x} &= 0
            .\end{align*}
             Because gravity is vertical, $a_{x} = 0$. If  $a_{x} = 0$, this means the initial velocity in the $x$ direction is equal to the final velocity in the $x$ direction, or  $v_{x} = v_{0x} $
            \item \textbf{Kinematic equations for motion in a uniform gravitational field}
                \begin{itemize}
                    \item Horizontal motion 
                        \begin{align*}
                            v_{0x} = v_{x}, \quad x = x_{0} +v_{x}t
                        .\end{align*}
                    \item \textbf{Average velocity (vertical)}
                    \item Basic average velocity 
                        \begin{align*}
                            V_{y,\text{avg}} = \frac{y-y_{0}}{\Delta t}
                        .\end{align*}
                    \item \textbf{Average velocity in special conditions}
                        \begin{align*}
                            V_{y,\text{avg}} = \frac{v_i +v_f}{2}
                        .\end{align*}
                        \textbf{Note:} gives the same result for average velocity in scenarios of uniform acceleration, such as projectile motion or free fall, under specific conditions. This formula calculates the average of the initial and final velocities, assuming that the acceleration is constant throughout the motion. It works perfectly for the vertical component of projectile motion when considering the ascent or descent separately, or any motion where the start and end points are symmetrical in terms of velocity but in opposite directions (e.g., going up and coming back down to the same height).
                    \item Vertical motion
                        \begin{align*}
                            y &= y_{0} + \frac{1}{2}(v_{0y} + v_{y})t \\
                            v_{y}&= v_{0y} + gt \\
                            y &=y_{0}+v_{0y}t + \frac{1}{2}gt^{2} \\
                            v^{2}_{y} &= v_{0y}^{2}+2g(y-y_{0})
                        .\end{align*}
                \end{itemize}
            \item \textbf{Projectile motion max height}
                \begin{align*}
                    h = \frac{v_{0y}^{2}}{2g}
                .\end{align*}
                This equation defines the maximum height of a projectile above its launch position and it depends only on the vertical component of the initial velocity.
            \item \textbf{Parabolic trajectory}: If the motion is parabolic, we should use the equation with $t^{2}$ to solve for $t$ and vertical displacement
            \item \textbf{Time of flight}: We can solve for the time of flight of a projectile that is both launched and impacts on a flat horizontal surface by performing some manipulations of the kinematic equations. We note the position and displacement in y must be zero at launch and at impact on an even surface. Thus, We set the displacement in y equal to zero and find
                \begin{align*}
                    T_{\text{tof}} = \frac{2(v_{0}\sin{\theta_{0}})}{g}
                .\end{align*}
                \textbf{Note:} This is the time of flight for a projectile both launched and impacting on a flat horizontal surface. This equation does not apply when the projectile lands at a different elevation than it was launched,
            \item \textbf{Trajectory}: The trajectory of a projectile can be found by eliminating the time variable $t$ from the kinematic equations for arbitrary $t$ and solving for $y(x)$. We take $x_{0} = y_{0} = 0$ so the projectile is launched from the origin. 
                \begin{align*}
                    y = (\tan{\theta_{0}})x - \left[\frac{g}{2(v_{0}\cos{\theta_{0}})^{2}}\right]x^{2}
                .\end{align*}
                \textbf{Note:} This trajectory equation is of the form $y=ax+bx^{2}$
                which is an equation of a parabola with coefficients
                \begin{align*}
                    a = \tan{\theta_{0}}, \quad b = -\frac{g}{2(v_{0}\cos{\theta_{0}})^{2}}
                .\end{align*}
            \item \textbf{Range}: From the trajectory equation we can also find the range, or the horizontal distance traveled by the projectile.
                \begin{align*}
                    R = \frac{v_{0}^{2}\sin{(2\theta_{0})}}{g}
                .\end{align*}

            \item \textbf{Note:} the last three equations (\textbf{time of flight, trajectory, and range}) are derived with g as negative, thus we do not make g negative when plugging into the equation. 
            \item \textbf{Centripetal Acceleration}: a particle moving in a circle at a constant speed has an acceleration with magnitude
                \begin{align*}
                    a_{c} = \frac{v^{2}}{r}
                .\end{align*}
                \textbf{Note:} The direction of the acceleration vector is toward the center of the circle
            \item \textbf{Position vector for a particle executing circular motion}:
                As the particle moves on the circle, its position vector sweeps out the angle $\theta$ with the x-axis. Vector $\vec{r}(t)$ making an angle $\theta$ with the x-axis is shown with its components along the x- and y-axes. The magnitude of the position vector is $A=|\vec{r}(t)|$ and is also the radius of the circle, so that in terms of its components,
                \begin{align*}
                    \vec{\mathbf{r}}(t) = A\cos(\omega t)\hat{\mathbf{i}} + A\sin(\omega t)\hat{\mathbf{j}}
                .\end{align*}
            \item \textbf{Average speed for circular motion}: If $T$ is the item it takes to go around the circle once, and the distance around a circle is $2\pi r$, then 
                \begin{align*}
                    v &= \frac{2\pi r}{T} \\
                      &\implies T = \frac{2\pi r}{v}
                .\end{align*}
            \item \textbf{Amplitude $(A)$}: In the contex of uniform circle motion, we define the amplitude $A$ as the radius of the circle. Thus,
                \begin{align*}
                    A = r
                .\end{align*}
            \item \textbf{Angular frequency}: In the previous equation, $\omega$ is a constant called the angular frequency of the particle.
                \bigbreak \noindent 
                The angular frequency has units of radians (rad) per second and is simply the number of radians of angular measure through which the particle passes per second. The angle  $\theta$ that the position vector has at any particular time is  $\omega t$
            \item \textbf{Angular frequency computation}:
                If $T$ is the period of motion, or the time to complete one revolution $(2\pi \text{ rad})$, then 
                \begin{align*}
                    \omega &= \frac{2\pi}{T} \\
                    &\implies T = \frac{2\pi}{\omega}
                .\end{align*}
            \item \textbf{Example: Finding $\omega$}: A flywheel is rotating at 21 rev/s. What is the total angle, in radians, through which a point on the flywheel rotates in 37 s?
                \bigbreak \noindent 
                To find $\omega$, we must first deduce $T$, if we complete 21 revolutions in one second, then the time to complete one revolution is $\frac{1}{21}s$. Thus, we have $\omega$ as 
                \begin{align*}
                    &\frac{2\pi}{\frac{1}{21}} = 41\pi
                .\end{align*}
                Which gives us 
                \begin{align*}
                    \theta  &= \omega t \\
                    &=41\pi(37s)
                .\end{align*}
            \item \textbf{Finding $\theta$ with $\omega$}
                \begin{align*}
                    \theta  = \omega t
                .\end{align*}
            \item \textbf{Velocity for circular motion}
                \begin{align*}
                    \vec{\mathbf{v}}(t) = \frac{d\vec{\mathbf{r}}(t)}{dt} = -A\omega\sin{(\omega t)}\hat{\mathbf{i}} + A\omega\cos{(\omega t)}\hat{\mathbf{j}}
                .\end{align*}
            \item \textbf{Acceleration for circular motion}
                \begin{align*}
                    \vec{\mathbf{a}}(t) = \frac{d\vec{\mathbf{v}}(t)}{dt} = -A\omega^{2}\cos{(\omega t)}\hat{\mathbf{i}} - A\omega^{2}\sin{(\omega t)}\hat{\mathbf{j}}
                .\end{align*}
                From this equation we see that the acceleration vector has magnitude $A\omega^2$ and is directed opposite the position vector, toward the origin, because $\vec{a}(t) = -\omega^2 \vec{r}(t)$.
            \item \textbf{Nonuniform Circular Motion Tangential Acceleration}:
                Circular motion does not have to be at a constant speed. A particle can travel in a circle and speed up or slow down, showing an acceleration in the direction of the motion.
                \smallbreak \noindent
                In uniform circular motion, the particle executing circular motion has a constant speed and the circle is at a fixed radius. If the speed of the particle is changing as well, then we introduce an additional acceleration in the direction tangential to the circle. Such accelerations occur at a point on a top that is changing its spin rate, or any accelerating rotor.  If the speed of the particle is changing, then it has a \textbf{tangential acceleration} that is the time rate of change of the magnitude of the velocity
                \begin{align*}
                    a_{T} = \bigg\lvert  \frac{d\vec{\mathbf{v}}}{dt} \bigg\rvert
                .\end{align*}
            \item \textbf{Tangential acceleration total acceleration}: 
                The direction of tangential acceleration is tangent to the circle whereas the direction of centripetal acceleration is radially inward toward the center of the circle. Thus, a particle in circular motion with a tangential acceleration has a total acceleration that is the vector sum of the centripetal and tangential accelerations
                \begin{align*}
                    \vec{\mathbf{a}} = \vec{\mathbf{a}}_{c} + \vec{\mathbf{a}}_{T}
                .\end{align*}
                Thus total acceleration can be found by finding the magnitude of this vector
                \begin{align*}
                    \norm{\vec{\mathbf{a}}} = \sqrt{\vec{\mathbf{a}}_{c}^{2} + \vec{\mathbf{a}}_{T}^{2}}
                .\end{align*}
                With 
                \begin{align*}
                    \tan{\theta} = \frac{y}{x}
                .\end{align*}
            \item \textbf{Reference frames:} When we say an object has a certain velocity, we must state it has a velocity with respect to a given reference frame. In most examples we have examined so far, this reference frame has been Earth.
            \item \textbf{Relative Motion in One Dimension Example}:
                Consider an example of a person sitting in a train moving east. If we choose east as the positive direction and Earth as the reference frame, then we can write the velocity of the train with respect to the Earth as $\vec{v}_{\text{TE}} = 10 \, \text{m/s} \, \hat{i}$ east, where the subscripts TE refer to train and Earth. Let’s now say the person gets up out of her seat and walks toward the back of the train at $2 \, \text{m/s}$. This tells us she has a velocity relative to the reference frame of the train. Since the person is walking west, in the negative direction, we write her velocity with respect to the train as $\vec{v}_{\text{PT}} = -2 \, \text{m/s} \, \hat{i}$. We can add the two velocity vectors to find the velocity of the person with respect to Earth. This relative velocity is written as
                \[
                    \vec{v}_{\text{PE}} = \vec{v}_{\text{PT}} + \vec{v}_{\text{TE}}.
                \]
            \item \textbf{Relative Velocity in Two Dimensions Example:}
                Consider a particle \(P\) and reference frames \(S\) and \(S'\), The position of the origin of \(S'\)
                as measured in \(S\) is \(\vec{r}_{S'S}\),
                the position of \(P\) as measured in \(S'\)
                is \(\vec{r}_{PS'}\),
                and the position of \(P\) as measured in \(S\) is \(\vec{r}_{PS}\).
                \bigbreak \noindent 
                \fig{.8}{./figures/2d.jpeg}
                \bigbreak \noindent 
                From this, we see 
                \begin{align*}
                    \vec{\mathbf{r}}_{PS} = \vec{\mathbf{r}}_{PS^{\prime}} + \vec{\mathbf{r}}_{S^{\prime}S}
                .\end{align*}
        \item \textbf{Relative velocities (Still using previous example)}: The relative velocities are the time derivatives of the position vectors. Therefore,
            \begin{align*}
                    \vec{\mathbf{v}}_{PS} = \vec{\mathbf{v}}_{PS^{\prime}} + \vec{\mathbf{v}}_{S^{\prime}S}
            .\end{align*}
            \bigbreak \noindent 
            So we see the velocity of a particle relative to S is equal to its velocity relative to  $S^{\prime}$ plus the velocity of $S^{\prime}$ relative to $S$
        \item \textbf{Relative accelerations (Still using same example)}:  We can also see how the accelerations are related as observed in two reference frames by differentiating
            \begin{align*}
                \vec{\mathbf{a}}_{PS} = \vec{\mathbf{a}}_{PS^{\prime}} + \vec{\mathbf{a}}_{S^{\prime}S}
            .\end{align*}
            We see that if the velocity of \(S'\) relative to \(S\) is a constant, then \(\vec{a}_{S'S} = 0\) and
            \[
                \vec{\mathbf{a}}_{PS} = \vec{\mathbf{a}}_{PS^{\prime}}.
            \]
            This says the acceleration of a particle is the same as measured by two observers moving at a constant velocity relative to each other.
        \item \textbf{Percent difference equation}: If $A$ is the experimental value, and $B$ is the actual value, then the percent difference is given by
            \begin{align*}
                \text{Percent difference } = \frac{\abs{A - B}}{A} \cdot 100\%
            .\end{align*}
        \end{itemize}
        \pagebreak 
        \subsection{Chapter 5: Newton's laws of motion}
        \bigbreak \noindent 
        \subsubsection{Definitions and Theorems}
        \begin{itemize}
            \item  \textbf{Dynamics} is the study of how forces affect the motion of objects and systems.
            \item \textbf{constraints of Newtonian mechanics:} Newton’s laws produce a good description of motion only when the objects are moving at speeds much less than the speed of light and when those objects are larger than the size of most molecules (about  $10^{-9}m$ in diameter).
            \item \textbf{Intuitive definition of force}: A push or a pull—is a good place to start.
                \bigbreak \noindent 
                \textbf{Note:} We know that a push or a pull has both magnitude and direction (therefore, it is a vector quantity), so we can define force as the push or pull on an object with a specific magnitude and direction. Force can be represented by vectors or expressed as a multiple of a standard force.
            \item \textbf{Standard force}: A quantitative definition of force can be based on some standard force, just as distance is measured in units relative to a standard length. One possibility is to stretch a spring a certain fixed distance, and use the force it exerts to pull itself back to its relaxed shape—called a restoring force—as a standard. The
            \item \textbf{Contact forces}: Contact forces are due to direct physical contact between objects.
            \item \textbf{Field forces}: Field forces, however, act without the necessity of physical contact between objects. They depend on the presence of a “field” in the region of space surrounding the body under consideration.
            \item \textbf{Field}: You can think of a field as a property of space that is detectable by the forces it exerts.
            \item \textbf{The newton}: 1 N is the force needed to accelerate an object with a mass of 1 kg at a rate of  1$m/s^{2} $. Hence, 1N=1$kg\cdot m/s^{2}$
            \item \textbf{Net external force}: The resultant of forces is call the net external force $\vec{\mathbf{F}}_{\text{net}}$ and is found by taking the vector sum of all external forces acting on an object or system
                \begin{align*}
                    \vec{\mathbf{F}}_{\text{net}} = \summation{}{}\  \vec{\mathbf{F}}\ = \vec{\mathbf{F}}_{1} + \vec{\mathbf{F}}_{2} + ...
                .\end{align*}
            \item \textbf{Newton's first law of motion}: A body at rest remains at rest or, if in motion, remains in motion at constant velocity unless acted on by a net external force.
                \bigbreak \noindent 
                \textbf{Note:} Newton’s first law says that there must be a cause for any change in velocity (a change in either magnitude or direction) to occur. This cause is a net external force,
            \item On a frictionless surface and ignoring air resistance, we can imagine an object sliding in a straight line indefinitely. Friction is thus the cause of slowing (consistent with Newton’s first law). The \textbf{object would not slow down if friction were eliminated.}
            \item \textbf{mass} is a measure of the amount of matter in something.
            \item \textbf{Gravitation} is the attraction of one mass to another
                \bigbreak \noindent 
                \textbf{Note:} such as the attraction between yourself and Earth that holds your feet to the floor. The magnitude of this attraction is your weight, and it is a force.
            \item \textbf{Inertia} is the ability of an object to resist changes in its motion—in other words, to resist acceleration.
                \bigbreak \noindent 
                \textbf{Note:} Newton’s first law is often called the \textbf{law of inertia}. The inertia of an object is measured by its mass.
            \item \textbf{Inertial reference frame}: In principle, we can make the net force on a body zero. If its velocity relative to a given frame is constant, then that frame is said to be inertial.
                \bigbreak \noindent 
                So by definition, an inertial reference frame is a reference frame in which Newton’s first law is valid. Newton’s first law applies to objects with constant velocity. From this fact, we can infer the following statement.
                \bigbreak \noindent 
                A reference frame moving at constant velocity relative to an inertial frame is also inertial. A reference frame accelerating relative to an inertial frame is not inertial.
            \item \textbf{Newton’s first law in vector form:}
                \begin{align*}
                    \vec{\mathbf{v}} = \text{constant when } \vec{\mathbf{F}}_{\text{net}} = \vec{\mathbf{0}}N
                .\end{align*}
            \item \textbf{static equilibrium}: Static systems do not change over time. Static equilibrium involves objects at rest
            \item \textbf{Dynamic equilibrium}: Dynamic systems change over time. Dynamic equilibrium involves objects in motion without acceleration,
            \item \textbf{a net force of zero means that an object is either at rest or moving with constant velocity,}
            \item \textbf{Acceleration is proportional to the net external force}. That is,
                \begin{align*}
                    a\ \propto\ \sum \vec{\mathbf{F}}
                .\end{align*}
            \item \textbf{It also seems reasonable that acceleration should be inversely proportional to the mass of the system.}
                \begin{align*}
                    a\ \propto\ \frac{1}{m}
                .\end{align*}
                In other words, the larger the mass (the inertia), the smaller the acceleration produced by a given force.
            \item \textbf{Experiments have shown that acceleration is exactly inversely proportional to mass, just as it is directly proportional to net external force.}
            \item \textbf{Newton's second law of motion}
                The acceleration of a system is directly proportional to and in the same direction as the net external force acting on the system and is inversely proportional to its mass. In equation form, Newton's second law is
                \[
                    \vec{a} = \frac{\vec{F}_{\text{net}}}{m},
                \]
                where $\vec{a}$ is the acceleration, $\vec{F}_{\text{net}}$ is the net force, and $m$ is the mass. This is often written in the more familiar form
                \[
                    \vec{F}_{\text{net}} = \sum \vec{F} = m\vec{a},
                \]
                but the first equation gives more insight into what Newton's second law means. When only the magnitude of force and acceleration are considered, this equation can be written in the simpler scalar form:
                \[
                    F_{\text{net}} = ma.
                \]
            \item \textbf{Component Form of Newton’s Second Law}
                The equations of motion in three dimensions can be represented as:
                \begin{align}
                    \sum F_x &= m a_x \\
                    \sum F_y &= m a_y \\
                    \sum F_z &= m a_z
                \end{align}
            \item \textbf{A body’s mass is a measure of its inertia}
            \item \textbf{Newton’s Second Law and Momentum}: Newton actually stated his second law in terms of momentum: “The instantaneous rate at which a body’s momentum changes is equal to the net force acting on the body.” (“Instantaneous rate” implies that the derivative is involved.) This can be given by the vector equation
                \begin{align*}
                    \vec{\mathbf{F}}_{\text{net}} = \frac{d\vec{\mathbf{p}}}{dt}
                .\end{align*}
                Momentum was described by Newton as “quantity of motion,” a way of combining both the velocity of an object and its mass.
            \item \textbf{Momentum}: For now, it is sufficient to define momentum $\vec{\mathbf{p}}$ as the product of the mass of the object $m$ and its velocity $\vec{\mathbf{v}}$
                \begin{align*}
                    \vec{\mathbf{p}} = m\vec{\mathbf{v}}
                .\end{align*}
            \item \textbf{Weight:} If air resistance is negligible, the net force on a falling object is the gravitational force, commonly called its weight  $\vec{\mathbf{w}}$ , or its force due to gravity acting on an object of mass $m$
                \bigbreak \noindent 
                Weight can be denoted as a vector because it has a direction; down is, by definition, the direction of gravity, and hence, weight is a downward force. The magnitude of weight is denoted as $w$.
                \bigbreak \noindent 
                The gravitational force on a mass is its weight. We can write this in vector form, where $\vec{\mathbf{w}}$ is weight and $m$ is mass, as
                \begin{align*}
                    \vec{\mathbf{w}} = m\vec{\mathbf{g}}
                .\end{align*}
                In scalar form, we can write
                \begin{align*}
                    w = mg
                .\end{align*}
            \item \textbf{When the net external force on an object is its weight, we say that it is in free fall}
            \item \textbf{Newton's third law of motion}:
                Whenever one body exerts a force on a second body, the first body experiences a force that is equal in magnitude and opposite in direction to the force that it exerts. Mathematically, if a body $A$ exerts a force $\vec{F}$ on body $B$, then $B$ simultaneously exerts a force $-\vec{F}$ on $A$, or in vector equation form,
                \[
                    \vec{\mathbf{F}}_{AB} = -\vec{\mathbf{F}}_{BA}
                \]
                \bigbreak \noindent 
                Newton’s third law represents a certain symmetry in nature: Forces always occur in pairs, and one body cannot exert a force on another without experiencing a force itself.
            \item  \textbf{Normal Force}: If the force supporting the weight of an object, or a load, is perpendicular to the surface of contact between the load and its support, this force is defined as \textbf{a normal force} and here is given by the symbol $\vec{\mathbf{N}}$.
                \begin{align*}
                    \vec{\mathbf{N}} = -m\vec{\mathbf{g}}
                .\end{align*}
                \textbf{Note:} The word normal means perpendicular to a surface.
                \bigbreak \noindent 
                For objects resting on horizontal surfaces, this becomes the scalar form 
                \begin{align*}
                    N = mg
                .\end{align*}
                \bigbreak \noindent 
                \textbf{Note:} The normal force can be less than the object’s weight if the object is on an incline.
            \item \textbf{Normal force is perpendicular to the surface}
            \item \textbf{Tension}: A \textbf{tension} is a force along the length of a medium; in particular, it is a pulling force that acts along a stretched flexible connector, such as a rope or cable.
                \smallbreak \noindent
                Any flexible connector, such as a string, rope, chain, wire, or cable, can only exert a pull parallel to its length; thus, a force carried by a flexible connector is a tension with a direction parallel to the connector.
                \smallbreak \noindent
                Consider the following figure
                \bigbreak \noindent 
                \fig{.8}{./figures/tension.jpeg}
                \bigbreak \noindent 
                If the 5.00-kg mass in the figure is stationary, then its acceleration is zero and the net force is zero. The only external forces acting on the mass are its weight and the tension supplied by the rope. Thus,
                \begin{align*}
                    F_{\text{net}} = T - w = 0 \\
                    \implies  T = w = mg
                .\end{align*}
                With this, (neglecting the mass of the rope), we see that the tension would be
                \begin{align*}
                    T &= mg = (5.00kg)(9.8 m/s^{2}) \\
                    &=49N
                .\end{align*}
                \bigbreak \noindent 
                \textbf{Observation}: If we cut the rope and insert a spring, the spring would extend a length corresponding to a force of 49.0 N, providing a direct observation and measure of the tension force in the rope.
            \item \textbf{When to use normal force}
                \begin{itemize}
                    \item \textbf{Contact Between Surfaces:} Use normal force in scenarios where two surfaces are in contact, such as an object resting on a table, a book lying on a shelf, or a box pushed against a wall.
                    \item \textbf{Supporting Weight:} If an object is supported by a surface (preventing it from falling due to gravity), there's a normal force exerted by the surface on the object, equal and opposite to the component of the object's weight perpendicular to the surface.
                    \item \textbf{Inclined Planes:} On an inclined plane, normal force acts perpendicular to the surface, supporting the component of the object's weight perpendicular to the plane. It's crucial for calculating the net force acting along the plane.
                \end{itemize}
            \item \textbf{When to not use normal force}
                \begin{itemize}
                    \item \textbf{No Direct Contact:} In scenarios where there's no direct physical contact between surfaces (e.g., objects in free fall, satellites in orbit, or a mass hanging from a string), normal force does not apply.
                    \item \textbf{Tension-Dominated Problems:} In problems primarily involving tension (such as a tightrope walker, a hanging pendulum, or objects connected by ropes over pulleys), the focus is on tension forces rather than normal forces.
                \end{itemize}
        \item \textbf{Tension T created when a perpendicular force  $(F_{\perp})$ is exerted at the middle of a flexible connector:}
            \begin{align*}
                T=\frac{F_{\perp}}{2\sin{\left(\theta \right)}}
            .\end{align*}
            The angle between the horizontal and the bent connector is represented by $\theta$. In this case, $T$ becomes large as $\theta$ approaches zero. Even
        \item \textbf{Hooke’s law}: A spring is a special medium with a specific atomic structure that has the ability to restore its shape, if deformed. To restore its shape, a spring exerts a restoring force that is proportional to and in the opposite direction in which it is stretched or compressed. This is the statement of a law known as Hooke’s law, which has the mathematical form
            \begin{align*}
                \vec{\mathbf{F}} = -k\vec{\mathbf{x}}
            .\end{align*}
            \bigbreak \noindent 
            \fig{1}{./figures/restore.jpeg}
                A spring exerts its force proportional to a displacement, whether it is compressed or stretched.
            \begin{enumerate}
                \item[(a)] The spring is in a relaxed position and exerts no force on the block.
                \item[(b)] The spring is compressed by displacement $\vec{\Delta x}_1$ of the object and exerts restoring force $-k\vec{\Delta x}_1$.
                \item[(c)] The spring is stretched by displacement $\vec{\Delta x}_2$ of the object and exerts restoring force $-k\vec{\Delta x}_2$.
            \end{enumerate}
        \item \textbf{Force on an inclined plane practice}: What force (in N) must be applied to a 250.0 kg crate on a frictionless plane inclined at $30^{\circ}$ to cause an acceleration of 7.6 $m/s^{2}$ up the plane? (Enter the magnitude.)
            \bigbreak \noindent 
            First, let's draw a diagram
            \bigbreak \noindent 
    \begin{figure}[ht]
        \centering
        \incfig{dagz}
        \label{fig:dagz}
    \end{figure}
    \bigbreak \noindent 
    Figure 3 shows how we set up our coordinate axis, and figure 2 shows how we find the angle between the weight vector and the negative x-axis
    \bigbreak \noindent 
    To find the Force vector $\vec{\mathbf{F}}$, we need to first find the net force in the horizontal direction. Since there is seemingly no vertical acceleration, we deduce that the net force must be zero. We find 
    \begin{align*}
        W_{x} = \vec{\mathbf{W}}\cos{\left(60^{\circ}\right)} &= mg\cos{\left(60^{\circ}\right)} \\
    .\end{align*}
    Thus the net force in the horizontal direction is given by
    \begin{align*}
        &F_{\text{net,x}} = \vec{\mathbf{F}} - W_{x} = ma \\
        &\implies \vec{\mathbf{F}} - wg\cos{\left(60^{\circ}\right)} = ma \\
        &\implies \vec{\mathbf{F}} = ma + mg\cos{\left(60\right)} \\
        &\therefore \vec{\mathbf{F}} = (250)(7.6) + (250)(9.8)\cos{\left(60^{\circ}\right)} \\
        &=3125
    .\end{align*}
\item \textbf{Average frictional force when stopping in car}
    \begin{align*}
        \abs{\vec{\mathbf{F}}_{s}} = m\abs{(\vec{\mathbf{a}})_{\text{avg}}}
    .\end{align*}

    \end{itemize}

    \pagebreak 
    \subsection{Chapter 6: Applications of Newton's Laws}
    \bigbreak \noindent 
    \subsubsection{Definitions and theorems}
    \begin{itemize}
        \item \textbf{Problem solving / Omitting forces } 
            \begin{itemize}
                \item Identify the physical principles involved by listing the givens and the quantities to be calculated.
                \item Sketch the situation, using arrows to represent all forces.
                \item Determine the system of interest. The result is a free-body diagram that is essential to solving the problem.
                \item Apply Newton’s second law to solve the problem. If necessary, apply appropriate kinematic equations from the chapter on motion along a straight line.
                \item Check the solution to see whether it is reasonable.
            \end{itemize}
            Suppose we want to lift a grand piano into a second-story apartment. Consider the sketch
            \bigbreak \noindent
            \fig{.7}{./figures/piano.jpeg}
            \bigbreak \noindent 
            In this case we want to find the tension of the rope, so the system of interest is the piano. Thus, the force that the piano exerts on the rope is a force by the system and not of use. We only care about external forces for whatever system we are analyzing
        \item \textbf{Weight and normal force are external}
        \item \textbf{When tensions are equal}
            \begin{itemize}
                \item \textbf{Single Rope with No Mass and no friction:} In a system where a massless rope passes over a frictionless pulley connecting two masses, the tension throughout the rope is constant if the pulley is ideal (massless and frictionless) and the rope is massless. This is because the rope cannot exert any force by itself, and there's no mechanism (like friction or pulley mass) to change the tension in different parts of the rope.
                \item \textbf{Static Equilibrium:} In a scenario where the system is in static equilibrium (not moving), and if we ignore the mass of the rope or the friction in the pulley, the tension on either side of the pulley must be equal to maintain equilibrium. This is because, in equilibrium, the sum of forces in any direction must be zero, and thus the forces (tensions) pulling on either side of the pulley must balance out.
            \end{itemize}
        \item \textbf{When Tensions are Not Equal}:
            \begin{itemize}
                \item \textbf{Rope with Mass:}
                    \begin{itemize}
                        \item If the rope has mass, the tension varies along the length of the rope. The tension is higher closer to where the heavier load is applied because it has to support more weight (including the weight of the rope itself).
                    \end{itemize}
                \item \textbf{Multiple Pulleys or Complex Systems:}
                    \begin{itemize}
                        \item In systems involving multiple pulleys or complex arrangements, tension can vary significantly throughout the system. Each segment of the rope can have different tensions due to varying angles of pull, different masses being lifted, or the mechanical advantage created by the pulleys. 
                    \end{itemize}
                \item \textbf{Varying Angle Measures:}
                    \begin{itemize}
                        \item When forces are applied at different angles, especially in systems involving pulleys or objects being pulled at angles, the tension in the rope or cable can differ based on how the components of the forces contribute to the tension.
                    \end{itemize}
                \item \textbf{Acceleration:} In systems where the masses are accelerating, the tension in the rope can differ if the rope passes over a pulley that changes the direction of the tension force. If one mass is accelerating downward, it might cause the rope on its side to have more tension compared to the other side, especially if the masses or forces acting on them are not symmetrical.
                \item \textbf{Non-Ideal Conditions:}: When the rope has mass or the pulley has friction, the tension can vary along the rope. For example, the part of the rope supporting a heavier mass will have more tension compared to the side with a lighter mass. Similarly, friction in the pulley can cause a difference in tension on either side because it requires additional force to overcome the friction, leading to higher tension on one side.
                \end{itemize}
                \item \textbf{Friction} is a force that opposes relative motion between systems in contact.
                \item \textbf{Sliding friction}: is parallel to the contact surfaces between systems and is always in a direction that opposes motion or attempted motion of the systems relative to each other. 
                \item \textbf{Static friction}: If two systems are in contact and stationary relative to one another, then the friction between them is called static friction. 
                \item \textbf{Kinetic friction}: If two systems are in contact and moving relative to one another, then the friction between them is called kinetic friction.
                \item \textbf{Magnitude of static friction}:
                    The magnitude of static friction $f_s$ is
                    \begin{equation}
                        f_s \leq \mu_s N
                    \end{equation}
                    where $\mu_s$ is the coefficient of static friction and $N$ is the magnitude of the normal force.
                \item  Static friction is a responsive force that increases to be equal and opposite to whatever force is exerted, up to its maximum limit. Once the applied force exceeds $f_{s}(max)$, the object moves. Thus,
                    \begin{align*}
                        f_{s}(max) = \mu_{s}N
                    .\end{align*}
                \item \textbf{Magnitude of kinetic friction}:
                    The magnitude of kinetic friction $f_k$ is given by
                    \begin{equation}
                        f_k = \mu_k N
                    \end{equation}
                    where $\mu_k$ is the coefficient of kinetic friction.
                \item \textbf{the coefficients of kinetic friction are less than their static counterparts.}
                \item \textbf{Coefficient of friction is a unitless quantity with a magnitude usually between 0 and 1.0. The actual value depends on the two surfaces that are in contact.}
                \item \textbf{neither formula is accurate for lubricated surfaces or for two surfaces sliding across each other at high speeds.}
                \item \textbf{Friction on an inclined plane}:  The basic physics is the same. We usually generalize the sloping surface and call it an inclined plane but then pretend that the surface is flat.
                \item \textbf{Angular velocity (radial acceleration)}, is the vector quantitiy $\omega$, as we know the term angular frequency refers to the scalar quantity $\omega = \frac{2\pi}{T}$, where $T$ is the time it takes to complete one revolution, and $\omega$ is given in rads/sec. This acceleration acts along the radius of the curved path and is thus also referred to as a radial acceleration.
                \item \textbf{Centripetal acceleration in terms of angular velocity}
                    \begin{align*}
                        a_{c} = r\omega^{2}
                    .\end{align*}
                \item \textbf{Any net force causing uniform circular motion is called a centripetal force.} The direction of a centripetal force is toward the center of curvature,
                \item \textbf{The direction of a centripetal force is toward the center of curvature, the same as the direction of centripetal acceleration.}
                \item \textbf{Equations for centripetal force}
                    \begin{align*}
                        F_{c} &= ma_{c} \\
                        F_{c} &= m \frac{v^{2}}{r}  \\
                              F_{c}&=mr\omega^{2}
                    .\end{align*}
                \item \textbf{Centripetal force  $\vec{\mathbf{F}}_{c}$ is always perpendicular to the path and points to the center of curvature,}
                \item \textbf{Friction in regards to cars and tires}:
                    \begin{itemize}
                        \item \textbf{Tires Rolling Without Slipping:} When a car is moving normally, and its tires are rolling without slipping, the friction between the tires and the road is static friction. Despite the car moving, the point of the tire that contacts the road is momentarily at rest relative to the road surface. This static friction is what allows the car to start moving from rest, stop, and turn without skidding. It's also the force that propels the car forward; the engine exerts a torque on the wheels, and it's the static frictional force at the contact patch of the tire that moves the car forward.
                        \item \textbf{Tires Slipping or Skidding:} When the tires are slipping or skidding on the road surface, the friction between the tires and the road is kinetic (sliding) friction. This occurs when the force applied (for example, through braking or accelerating too hard) exceeds the maximum static frictional force that can be developed between the tire and the road. Kinetic friction also acts when a car is drifting, where the tires are intentionally made to slip sideways.
                    \end{itemize}
                    \bigbreak \noindent 
                    \textbf{Note:} The key to understanding why static friction is involved with a moving car lies in the behavior of the tires in contact with the road. When a car moves and its tires are rolling without slipping, the point of contact between each tire and the road does not slide across the road surface. Instead, it momentarily "sticks" to the road before lifting off again as the tire rotates. This means that, at any given instant, the part of the tire touching the road is stationary relative to the road surface. The force preventing the tire from slipping and allowing it to roll is provided by static friction.
                \item \textbf{Finding theta for ideal banking (frictionless curve)}
                    \begin{align*}
                        \theta = \tan^{-1}{\left(\frac{v^{2}}{rg}\right)}
                    .\end{align*}
                \item \textbf{Drag force always opposes the motion of an object.}
                \item \textbf{Drag force (air resistance)} For most large objects such as cyclists, cars, and baseballs not moving too slowly, \textbf{the magnitude of the drag force $F_{D}$ is proportional to the square of the speed of the object.}
                    \begin{align*}
                        F_{D} \propto v^{2}
                    .\end{align*}
                     When taking into account other factors, this relationship becomes
                     \begin{align*}
                         F_{D} = \frac{1}{2}C\rho Av^{2}
                     .\end{align*}
                     where $C$ is the drag coefficient, $A$ is the area of the object facing the fluid, and $\rho$ is the density of the fluid.
                     \bigbreak \noindent 
                     This equation can also be written in a more generalized fashion as
                     \begin{align*}
                         F_{D} = bv^{n}
                     .\end{align*}
                     Where $b$ is a constant equivalent to $0.5C\rho A$
                     \bigbreak \noindent 
                     \textbf{Note:} The value of the drag coefficient C is determined empirically,
                \item \textbf{What is terminal velocity}: Terminal velocity is the constant speed that a freely falling object eventually reaches when the resistance of the medium through which it is falling prevents further acceleration.
                \item \textbf{Terminal Velocity}: Consider a skydiver falling through air under the influence of gravity. Once air resistance becomes equal to the force of gravity, the net force will be zero and there will be no acceleration. At this point, the person’s velocity remains constant and we say that the person has reached his terminal velocity. We have
                    \begin{align*}
                        mg&=\frac{1}{2}C\rho Av^{2} \\
                        v_{T}&=\sqrt{\frac{2mg}{\rho CA}}
                    .\end{align*}
                \item \textbf{The density of air is approximately}
                    \begin{align*}
                        \rho = 1.21 kg /m^{3}
                    .\end{align*}
                \item \textbf{Stoke's Law}: For a spherical object falling in a medium, the drag force is
                    \begin{align*}
                        F_{s} = 6\pi r\eta v
                    .\end{align*}
                where $r$ is the radius of the object, $\eta$ is the viscosity of the fluid, and $v$ is the object’s velocity.
        \item \textbf{A car approaches the top of a hill that is shaped like a vertical circle with a radius of 55.0m. What is the fastest speed that the car can go over the hill without losing contact with the ground?}. In this case, we have $F_{y} = mg-N = ma_{c} $. But once the car looses contact with the road (goes past its max velocity to stay on the road), normal force goes to zero because the car would no longer be in contact with the road. Thus, the maximum velocity would be
            \begin{align*}
                F_{y} = mg = a_{c}  \\
                \implies mg = m\frac{v^{2}}{r} \\
                \implies v = \sqrt{g \cdot r}
            .\end{align*}

    \end{itemize}

    \pagebreak 
    \subsection{Chapter 7: Work and Kinetic energy}
    \bigbreak \noindent 
    \subsubsection{Definitions and theorems}
    \begin{itemize}
        \item \textbf{work} is done on an object when energy is transferred to the object. In other words, work is done when a force acts on something that undergoes a displacement from one position to another. Forces can vary as a function of position, and displacements can be along various paths between two points. 
        \item \textbf{Increment of work $dW$}: We first define the increment of work $dW$ done by a force $\vec{F}$ acting through an infinitesimal displacement $\vec{dr}$ as the dot product of these two vectors:
            \begin{equation}
                dW = \vec{F} \cdot \vec{dr} = \left|\vec{F}\right| \left|\vec{dr}\right| \cos\theta.
            \end{equation}
        \item \textbf{Work done by a force}:
            We  add up the contributions for infinitesimal displacements, along a path between two positions, to get the total work.
            \bigbreak \noindent 
            The work done by a force is the integral of the force with respect to displacement along the path of the displacement:
            \begin{align*}
                W_{\text{AB}} = \int_{\text{path AB}} \vec{\mathbf{F}} \cdot d \vec{\mathbf{r}}
            .\end{align*}
        \item \textbf{SI unit for work}: The units of work are units of force multiplied by units of length, which in the SI system is newtons times meters, $N\cdot m$. This combination is called a joule, abbreviated $J$
        \item \textbf{American unit for work}: States, the unit of force is the pound (lb) and the unit of distance is the foot (ft), so the unit of work is the foot-pound  (ft$\cdot$lb).
        \item \textbf{Work done by constant forces and contact forces}: The simplest work to evaluate is that done by a force that is constant in magnitude and direction. In this case, we can factor out the force; the remaining integral is just the total displacement, which only depends on the end points A and B, but not on the path between them:
            \begin{align*}
                W_{\text{AB}} &= \vec{\mathbf{F}} \cdot \int_{A}^{B}\ d\vec{\mathbf{r}} \\
                &=\vec{\mathbf{F}} \cdot (\vec{\mathbf{r}}_{\text{B}} - \vec{\mathbf{r}}_{\text{A}}) \\
                &=\bigg\lvert \vec{\mathbf{F}} \bigg\rvert \bigg\lvert \vec{\mathbf{r}}_{\text{B}} - \vec{\mathbf{r}}_{\text{A}} \bigg\rvert\cos{\left(\theta \right)} 
            .\end{align*}
            That is, work is just $\vec{\mathbf{F}} \cdot \vec{\mathbf{d}}  = Fd\cos{\left(\theta \right)}$
        \item \textbf{Work of the normal force}. If the object being displaced never leaves the surface, then the displacement $d\vec{\mathbf{r}} $ is tangent to the surface. Since these two vectors are orthogonal, their dot product is zero. Thus, we have
            \begin{align*}
                dW_{N} = \vec{\mathbf{N}} \cdot d\vec{\mathbf{r}} = \vec{\mathbf{0}}
            .\end{align*}
            \bigbreak \noindent 
            \textbf{Note:} If the displacement $d\vec{\mathbf{r}}$ did have a relative component perpendicular to the surface, the object would either leave the surface or break through it, and there would no longer be any normal contact force.
        \item \textbf{"Tangent to the surface"}: When the displacement is tangent to the surface, it means the movement of the object is along the surface
            \bigbreak \noindent 
            \textbf{Note:} This is the same as saying the displacement is parallel to the surface.
        \item \textbf{Work done by kinetic friction}: 
            If the magnitude of $\vec{f}_k$ is constant (as it would be if all the other forces on the object were constant), then the work done by friction is
            \[
                W_{\text{fr}} = \int_{A}^{B} \vec{f}_k \cdot d\vec{r} = -f_k \int_{A}^{B} |d\vec{r}| = -f_k |l_{AB}|,
            \]
            \bigbreak \noindent 
            where  $\abs{l_{\text{AB}}}$ is the path length on the surface.
            \bigbreak \noindent 
            \textbf{Note:} kinetic friction $\vec{\mathbf{f}}_{k}$ is opposite to $d\vec{\mathbf{r}}$ relative to the surface, so the work done by kinetic friction is negative.
        \item \textbf{Work done by static friction}: The force of static friction does no work in the reference frame between two surfaces because there is never displacement between the surfaces.
            \bigbreak \noindent 
            As an external force, static friction can do work. Static friction can keep someone from sliding off a sled when the sled is moving and perform positive work on the person. 
        \item \textbf{the work done against a force is the negative of the work done by the force.}
        \item \textbf{Work done by gravity}: The work done by a constant force of gravity on an object depends only on the object’s weight and the difference in height through which the object is displaced.
            \begin{align*}
                W_{\text{grav},AB} = -mg\hat{\mathbf{j}} \cdot (\vec{\mathbf{r}}_B - \vec{\mathbf{r}}_A) = -mg(y_B - y_A)
            .\end{align*}
            \textbf{Note:} Gravity does negative work on an object that moves upward $(y_B > y_A)$, or, in other words, you must do positive work against gravity to lift an object upward. Alternately, gravity does positive work on an object that moves downward $(y_B < y_A)$, or you do negative work against gravity to “lift” an object downward, controlling its descent so it doesn’t drop to the ground. (“Lift” is used as opposed to “drop”.)
        \item \textbf{Work Done by Forces that Vary}: In general, forces may vary in magnitude and direction at points in space, and paths between two points may be curved. The infinitesimal work done by a variable force can be expressed in terms of the components of the force and the displacement along the path,
            \begin{align*}
                dW = F_{x}dx + F_{y}dy + F_{z}dz
            .\end{align*}
        \item \textbf{Springs, Hooke's law, and work (perfectly elastic spring)}:
            The force exerted by a perfectly elastic spring is governed by Hooke's law $\vec{F} = -k\Delta\vec{x}$, where $k$ is the spring constant and $\Delta\vec{x} = \vec{x} - \vec{x}_{\text{eq}}$ is the displacement from the spring's equilibrium position. This equilibrium position aligns with the spring's unstretched position in the absence of other forces or when they are neutralized. For work calculation by the spring force, the x-axis is aligned along the spring's length, with the origin at $x_{\text{eq}} = 0$. Thus, positive $x$ denotes stretching and negative $x$ compression. The work done by the spring force as $x$ varies from $x_A$ to $x_B$ is calculated under this framework.
            \bigbreak \noindent 
            The work done by the spring from \( x_A \) to \( x_B \) is given by:
            \begin{align*}
                W_{\text{spring},AB} &= \int_{A}^{B} F_x dx = -k \int_{A}^{B} x dx  \\
                &= -\frac{k}{2} \left[ x^2 \right]_{A}^{B} = -\frac{1}{2}k(x_B^2 - x_A^2).
            .\end{align*}
            \bigbreak \noindent 
            
     \textbf{Note:} Notice that $W_{AB}$ Depends only on the starting and ending points, $A$ and $B$, and is independent of the actual path between them, as long as it starts at $A$ and ends at $B$. That is, the actual path could involve going back and forth before ending.
    \item \textbf{Kinetic energy definition}: The kinetic energy of an object is the form of energy that it possesses due to its motion
 \item \textbf{Kinetic energy of a particle with mass $m$}:
     \begin{align*}
            K = \frac{1}{2}mv^{2}         
     .\end{align*}
  \item \textbf{Kinetic energy for a system of particles}:
      \begin{align*}
          k = \sum\frac{1}{2}mv^{2}
      .\end{align*}
    \item \textbf{Kinetic energy in terms of a particles momentum (single particle)}:
        that just as we can express Newton’s second law in terms of either the rate of change of momentum or mass times the rate of change of velocity, so the kinetic energy of a particle can be expressed in terms of its mass and momentum  $\vec{\mathbf{p}} = m\vec{\mathbf{v}}$, instead of its mass and velocity. Since $v=\frac{p}{m}$ , we see that
        \begin{align*}
            k = \frac{1}{2}\frac{p^{2}}{m} = \frac{p^{2}}{2m}
        .\end{align*}
    \item \textbf{Units of kinetic energy}: The units of kinetic energy are also the units of force times distance, which are the units of work, or joules.
    \item \textbf{Work-energy theorem}:
        The net work done on a particle equals the change in the particle's kinetic energy:
        \begin{equation}
            W_{\text{net}} = K_B - K_A.
        \end{equation}
    \item \textbf{Average power}: we first define average power as the work done during a time interval, divided by the interval,
        \begin{align*}
            \bar{P} = \frac{\Delta W}{\Delta t}
        .\end{align*}
    \item \textbf{Instantaneous power (or just plain power)}: Power is defined as the rate of doing work, or the limit of the average power for time intervals approaching zero,
        \begin{align*}
            P = \frac{dW}{dt}
        .\end{align*}
    \item \textbf{Work in constant power}: If the power is constant over a time interval, the average power for that interval equals the instantaneous power, and the work done by the agent supplying the power is
        \begin{align*}
            W = P\Delta t
        .\end{align*}
    \item \textbf{Work in varying power}: If the power during an interval varies with time, then the work done is the time integral of the power,
        \begin{align*}
            W = \int Pdt
        .\end{align*}
    \item \textbf{Unit for power}:  We can also define power as the rate of transfer of energy. Work and energy are measured in units of joules, so power is measured in units of joules per second, which has been given the SI name watts,
        \begin{align*}
            1 J/s = 1 W
        .\end{align*}
    \item \textbf{Horsepower}:
        \begin{align*}
            1hp = 746 W
        .\end{align*}
    \item \textbf{Power in terms of forces and velocity}
        \begin{align*}
            P = \frac{dW}{dt} &= \frac{\vec{\mathbf{F}} \cdot d\vec{\mathbf{r}}}{dt} = \vec{\mathbf{F}} \cdot \left(\frac{d\vec{\mathbf{r}}}{dt}\right) \\
            &=\vec{\mathbf{F}} \cdot \vec{\mathbf{v}}
        .\end{align*}
    \item \textbf{Finding angle when given grade}: Suppose we want to find the angle associated with a 15\% grade. This means for every unit increase in the x-direction, we move up 15\% or 0.15.
        \bigbreak \noindent 
\begin{figure}[ht]
    \centering
    \incfig{grade}
    \label{fig:grade}
\end{figure}
\bigbreak \noindent 
Thus, we see
\begin{align*}
    \tan{\left(\theta \right)} &= \frac{0.15}{1} \\
    \theta &=\tan^{-1}{\left(0.15\right)} \\
           &\approx 8.53^{\circ}
.\end{align*}
    \end{itemize}

    \pagebreak 
    \subsection{Chapter 8: Potential Energy and Conservation of Energy}
    \bigbreak \noindent 
    \subsubsection{Definitions and theorems}
    \begin{itemize}
        \item \textbf{Potential energy definition}: Potential energy is the energy an object posses by virtue of its position in a field
        \item \textbf{Potential energy difference}: We define the difference of potential energy from point A to point B as the negative of the work done:
            \begin{align*}
                \Delta U_{AB} = U_{B} - U_{A} = -W_{AB}
            .\end{align*}
        \item \textbf{Potential energy}: we need to define potential energy at a given position in such a way as to state standard values of potential energy on their own, rather than potential energy differences. We do this by rewriting the potential energy function in terms of an arbitrary constant,
            \begin{align*}
                \Delta U  = U(\vec{\mathbf{r}}) - U(\vec{\mathbf{r}}_{0})
            .\end{align*}
            \bigbreak \noindent 
            \textbf{Note:} the lowest height in a problem is usually defined as zero potential energy, or if an object is in space, the farthest point away from the system is often defined as zero potential energy. Then, the potential energy, with respect to zero at $\vec{\mathbf{r}}_{0}$, is just  $U(\vec{\mathbf{r}})$
        \item \textbf{Change in kinetic energy with no frition or air resistance}: As long as there is no friction or air resistance, the change in kinetic energy of the football equals negative of the change in gravitational potential energy of the football. This can be generalized to any potential energy:
            \begin{align*}
                \Delta K_{AB} = -\Delta U_{AB}
            .\end{align*}
        \item \textbf{Gravitational potential energy function}
            \begin{align*}
                U(y) = mgy + C
            .\end{align*}
            \bigbreak \noindent 
            \textbf{Note:} You can choose the value of the constant. However, for solving most problems, the most convenient constant to choose is zero for when $y=0$, which is the lowest vertical position in the problem. 
        \item \textbf{Elastic potential energy}:
            \begin{align*}
                U(x) = \frac{1}{2}kx^{2} + C 
            .\end{align*}
            \bigbreak \noindent 
            \textbf{Note:} If the spring force is the only force acting, it is simplest to take the zero of potential energy at $x=0$, when the spring is at its unstretched length. Then, the constant is zero. (Other choices may be more convenient if other forces are acting.)
        \item \textbf{Conservative Force}: Conservative force, in physics, any force, such as the gravitational force between Earth and another mass, whose work is determined only by the final displacement of the object acted upon. The total work done by a conservative force is independent of the path resulting in a given displacement and is equal to zero when the path is a closed loop
        \item \textbf{Non-Conservative Force}: Non-conservative forces are dissipative forces such as friction or air resistance. These forces take energy away from the system as the system progresses, energy that you can’t get back. These forces are path dependent; therefore it matters where the object starts and stops.
        \item \textbf{Work done by a conservative force}: The work done by a conservative force is independent of the path; in other words, the work done by a conservative force is the same for any path connecting two points:
            \[
                W_{AB,\text{path-1}} = \int_{A}^{B} \text{on path-1} \vec{F}_{\text{cons}} \cdot d\vec{r} = W_{AB,\text{path-2}} = \int_{A}^{B} \text{on path-2} \vec{F}_{\text{cons}} \cdot d\vec{r}.
            \]
        \item \textbf{Work done by a non-conservative force}: The work done by a non-conservative force depends on the path taken.
        \item \textbf{When is a force conservative}:
        A force is conservative if the work it does around any closed path is zero:
        \[
            W_{\text{closed path}} = \oint \vec{F}_{\text{cons}} \cdot d\vec{r} = 0.
        \]
        \item \textbf{Proving whether or not a force is conservative}:
            One answer is that the work done is independent of path if the infinitesimal work $\vec{F} \cdot d\vec{r}$ is an exact differential, the way the infinitesimal net work was equal to the exact differential of the kinetic energy, $dW_{\text{net}} = m\vec{v} \cdot d\vec{v} = d\left(\frac{1}{2}mv^2\right),$
            \bigbreak \noindent 
            There are mathematical conditions that you can use to test whether the infinitesimal work done by a force is an exact differential, and the force is conservative. These conditions only involve differentiation and are thus relatively easy to apply. In two dimensions, the condition for $\vec{F} \cdot d\vec{r} = F_x dx + F_y dy$ to be an exact differential is
            \begin{align*}
                \frac{dF_{x}}{dy} = \frac{dF_{y}}{dx}
            .\end{align*}
    \item \textbf{Non-conservative forces don't have potential energy}: we note that non-conservative forces do not have potential energy associated with them because the energy is lost to the system and can’t be turned into useful work later. So there is always a conservative force associated with every potential energy.
    \item \textbf{The infinitesimal increment of potential energy}:
        force. The infinitesimal increment of potential energy is the dot product of the force and the infinitesimal displacement,
        \begin{align*}
            dU = -\vec{\mathbf{F}} \cdot d\vec{\mathbf{\ell}} = -F_{\ell}d\ell \\
            \implies F_{\ell} = - \frac{dU}{d\ell}
        .\end{align*}
        \bigbreak \noindent 
        Here, we chose to represent the displacement in an arbitrary direction by  $d\vec{\mathbf{\ell}}$ so as not to be restricted to any particular coordinate direction.
        \bigbreak \noindent 
        In words, the component of a conservative force, in a particular direction, equals the negative of the derivative of the corresponding potential energy, with respect to a displacement in that direction.
    \item \textbf{Force with derivative of potential energy: One dimension}
        \begin{align*}
            \vec{\mathbf{F}} = f_{x}\,\hat{\mathbf{i}} = -\frac{dU}{dx}\,\hat{\mathbf{i}}
        .\end{align*}
    \item \textbf{Force with derivative of potential energy: Two dimensions}
        \begin{align*}
            \vec{\mathbf{F}} = f_{x}\,\hat{\mathbf{i}} + f_{y}\,\hat{\mathbf{j}} = -\frac{dU}{dx}\,\hat{\mathbf{i}} + -\frac{dU}{dy}\,\hat{\mathbf{j}}
        .\end{align*}
    \item \textbf{Mechanical energy}:
        mechanical energy is simply all the energy that an object has because of its motion and its position.
        \begin{align*}
           E = K + U 
        .\end{align*}
        Where $E$ denotes the mechanical energy
    \item \textbf{Conservation of Energy}:
        The mechanical energy $E$ of a particle stays constant unless forces outside the system or non-conservative forces do work on it, in which case, the change in the mechanical energy is equal to the work done by the non-conservative forces:
        \[ W_{\text{nc},AB} = \Delta(K + U)_{AB} = \Delta E_{AB}. \]
    \item \textbf{Position $x$ as a function of time $t$ with potential energy}
        \begin{align*}
            t = \int_{x_{0}}^{x}\ \frac{dx}{\sqrt{2[E-U(x)]/m}}
        .\end{align*}
    \item \textbf{Potential energy diagram example}
        \bigbreak \noindent 
        \fig{.6}{./figures/diagpot.jpeg}
        \bigbreak \noindent 
        The line at energy \(E\) represents the constant mechanical energy of the object, whereas the kinetic and potential energies, \(K_A\) and \(U_A\), are indicated at a particular height \(y_A\). You can see how the total energy is divided between kinetic and potential energy as the object's height changes.
    \item \textbf{Max height of a particle}: Using the fact that kinetic energy must be positive, we can write $K = E-U \geq 0 =\implies U \leq E$. If we use the gravitational potential energy reference point of zero at $y_{0}$, we can rewrite the gravitational potential energy $U$ as $mgy$. Solving for $y$ results in
        \begin{align*}
            y \leq \frac{E}{mg} = y_{\text{max}}
        .\end{align*}
    \item \textbf{Max speed of a particle}: At ground level $y_{0}$, the potential energy is zero, and the kinetic energy and the speed are maximum.
        \begin{align*}
            U_{0} &= 0 = E -K_{0} \\
            E&=K_{0}=\frac{1}{2}mv_{0}^{2} \\
            v_{0} &= \pm \sqrt{\frac{2E}{m}}
        .\end{align*}
        \bigbreak \noindent 
        \textbf{Note:} The maximum speed  $\pm v_{0}$ gives the initial velocity necessary to reach  $y_{\text{max}}$, the maximum height, and  $-v_{0}$ represents the final velocity, after falling from $y_{\text{max}}$.
    \item \textbf{Finding range of x}. Using the inequality $0 \leq K \leq E $, we can find the allowable range for x values. Considering the turning points, where all of the energy is potential, we have $K=0$ and $U=E$. For a elastic spring, we have
        \begin{align*}
            \frac{1}{2}kx^{2} &= E \\
            \implies x &= \pm \sqrt{\frac{2E}{k}}
        .\end{align*}
    \item \textbf{Finding range of $x$: Example}: Suppose we have the function $U(x) = 2(x^{4} - x^{2})$, and $E=-\frac{1}{4}$. To find the allowable range for $x$ values, we use the condition
        \begin{align*}
            &U+K = E \implies K = E - U \geq 0 \\
            &\implies -\frac{1}{4} - 2(x^{4}-x^{2}) \geq 0  \\
            &\implies 2\left(x^{4}-x^{2}\right) \leq -\frac{1}{4} \\
            &\implies 2\left(\left(x^{4}-\frac{1}{2}\right)^{2} - \frac{1}{4}\right) \leq -\frac{1}{4} \\
            &\implies \left(x^{4}-\frac{1}{2}\right)^{2} \leq \frac{-\frac{1}{4} + \frac{1}{2}}{2} \\
            &\implies \left(x^{4}-\frac{1}{2}\right)^{2} \leq \frac{1}{8} \\
            &\implies -\frac{1}{\sqrt{8}} + \frac{1}{2} \leq x^{2} \leq \frac{1}{\sqrt{8}} + \frac{1}{2}
        .\end{align*}
        \bigbreak \noindent 
        This represents two allowed regions, \(x_p \leq x \leq x_R\) and \(-x_R \leq x \leq -x_p\), where \(x_p = 0.38\) and \(x_R = 0.92\) (in meters).
    \item \textbf{Manipulating inequalitys: Squared terms}: For $a$ non-negative, we have
        \begin{align*}
            &x^{2} \leq a \\
            \implies -\sqrt{a} \leq\, &x \leq \sqrt{a}
        .\end{align*}
        \bigbreak \noindent 
        We also have
        \begin{align*}
           &a \leq x^{2} \leq b \quad \text{(Suppose)} \\
           &\implies \sqrt{a} \leq x \leq \sqrt{b} \quad \text{and} \quad -\sqrt{b} \leq x \leq -\sqrt{a}
        .\end{align*}
        \bigbreak \noindent 
        For $a,b$ non-negative. The implicition imposed by this is that $x$ can be within either ranges.

    \item \textbf{Finding equilibrium points}. To find the equilibrium points we just need to find relative extrema. Relative maxima is deemed unstable while relative minima is deemed stable

    \end{itemize}

    \pagebreak 
    \subsection{Chapter 13: Gravitation}
    \bigbreak \noindent 
    \subsubsection{Definitions and theorems}
    \begin{itemize}
        \item \textbf{Newton’s Law of Gravitation}:
            Newton's law of gravitation can be expressed as
            \begin{equation}
                \vec{F}_{12} = G \frac{m_1 m_2}{r^2} \hat{r}_{12}
            \end{equation}
            where $\vec{F}_{12}$ is the force on object 1 exerted by object 2 and $\hat{r}_{12}$ is a unit vector that points from object 1 toward object 2.
            \bigbreak \noindent 
            \fig{.8}{./figures/radial.jpeg}
            \bigbreak \noindent 
            \textbf{Note:} Notice how we take $r$ to be the distance between the center of mass for both objects.
        \item \textbf{Universal gravitational constant}: The constant $G$ is the previous equation is called the \textit{universal gravitational constant} and cavendish determined it to be 
            \begin{align*}
                G = 6.67 \times 10^{-11} N \cdot \frac{m^{2}}{kg^{2}}
            .\end{align*}
        \item \textbf{the law of gravitation applies to spherically symmetrical objects}, where the mass of each body acts as if it were at the center of the body.
        \item \textbf{Weight with NLOG}:
            We now know that this force is the gravitational force between the object and Earth. If we substitute $mg$ for the magnitude of $\vec{F}_{12}$ in Newton's law of universal gravitation, $m$ for $m_1$, and $M_E$ for $m_2$, we obtain the scalar equation
            \begin{equation}
                mg = G \frac{m M_E}{r^2}
            \end{equation}
            \bigbreak \noindent 
            where $r$ is the distance between the centers of mass of the object and Earth.
            \bigbreak \noindent 
            \textbf{Note:} The average radius of Earth is about 6370 km. Hence, for objects within a few kilometers of Earth’s surface, we can take $r=R_{E}$
        \item \textbf{Calculating $g$}, dividing the previous equation by $m$, we get
            \begin{equation}
                g = G \frac{M_E}{r^2}
            \end{equation}
        \item \textbf{Radius of the earth $R_{E}$}
            \begin{align*}
                R_{E} = 6.37 \times 10^{6}\, m
            .\end{align*}
        \item \textbf{Mass of the earth}:
            \begin{align*}
                M_{E} = 5.95 \times 10^{24}\, kg
            .\end{align*}
        \item \textbf{$g$ on the moons surface}
            \begin{align*}
                g = 1.6 m/s^{2}
            .\end{align*}
 
        \item \textbf{The gravitational field caused by mass $M$}
            \begin{align*}
                \vec{\mathbf{g}} = G \frac{M}{r^{2}}\hat{\mathbf{r}}
            .\end{align*}
            \bigbreak \noindent 
            We identify this vector field $\vec{\mathbf{g}}$ as the \textit{gravitational field caused by mass $M$}
        \item \textbf{Weight at the equator with earths spin}:
            \begin{align*}
                \sum F = F_{s} -mg = ma_{c}
            .\end{align*}
            Where $a_{c} = -\frac{v^{2}}{r} $
        \item \textbf{Results Away from the Equator}:
            At any other latitude $\lambda$, the situation is more complicated. The centripetal acceleration is directed toward point P in the figure, and the radius becomes $r = R_E \cos \lambda$. The vector sum of the weight and $\vec{F}_s$ must point toward point P, hence $\vec{F}_s$ no longer points away from the center of Earth. (The difference is small and exaggerated in the figure.)
            \bigbreak \noindent
            \fig{.7}{./figures/extrag.jpeg}
        \item \textbf{Gravity Away from the Surface}: Earlier we stated that the law of gravitation applies to spherically symmetrical objects, where the mass of each body acts as if it were at the center of the body. $\left(\vec{F}_{12} = G \frac{m_1 m_2}{r^2} \hat{r}_{12}\right)$ and $\left(g = G \frac{M_E}{r^2}\right) $. 
            \bigbreak \noindent 
            both equations are valid only for values of $r \geq R_{E}$. For $r < R_{E}$ these equations are not valid. However, we can determine g for these cases using a principle that comes from Gauss’s law,
            \bigbreak \noindent 
            A consequence of Gauss’s law, applied to gravitation, is that only the mass \textit{within} $r$ contributes to the gravitational force. Also, that mass, just as before, can be considered to be located at the center. The gravitational effect of the mass \textit{outside} $r$ has zero net effect.
            \bigbreak \noindent 
            For a spherical planet with constant density, the mass within $r$ is the density times the volume within $r$. This mass can be considered located at the center. Replacing $M_E$ with only the mass within $r$, $M = \rho \times (\text{volume of a sphere})$, and $R_E$ with $r$, we get 
            \[ g = \frac{G M_E}{R_E^2} = G \rho \left( \frac{4}{3} \pi r^3 \right) \frac{1}{r^2} = \frac{4}{3} G \rho \pi r. \]
            \bigbreak \noindent 
            \textbf{Note:} The value of $g$, and hence your weight, decreases linearly as you descend down a hole to the center of the spherical planet. At the center, you are weightless, as the mass of the planet pulls equally in all directions.
            \bigbreak \noindent 
            Actually, Earth’s density is not constant, nor is Earth solid throughout.
        \item \textbf{Gravitational Potential Energy beyond Earth}: We return to the definition of work and potential energy to derive an expression that is correct over larger distances.
            \begin{align*}
                U = - \frac{GM_{E}m}{r}
            .\end{align*}
        \item \textbf{Conservation of Energy with Gravitation}:
            \begin{align*}
                E &= K_{1} + U_{1} = K_{2} + U_{2}\\
                \implies &= \frac{1}{2}mv_{1}^{2} - \frac{GMm}{r_{1}} = \frac{1}{2}mv_{2}^{2} - \frac{GMm}{r_{2}}
            .\end{align*}
            \bigbreak \noindent 
            \textbf{Note:}
            Note that we use \(M\), rather than \(M_{E}\), as a reminder that we are not restricted to problems involving Earth. However, we still assume that \(m \ll M\). (For problems in which this is not true, we need to include the kinetic energy of both masses and use conservation of momentum to relate the velocities to each other. But the principle remains the same.)
        \item \textbf{Escape velocity:}
            \textit{Escape velocity} is often defined to be the \textbf{minimum initial velocity} of an object that is required to escape the surface of a planet (or any large body like a moon) and never return. As usual, we assume no energy lost to an atmosphere, should there be any.
            \bigbreak \noindent 
            Consider the case where an object is launched from the surface of a planet with an initial velocity directed away from the planet. With the minimum velocity needed to escape, the object would just come to rest infinitely far away, that is, the object gives up the last of its kinetic energy just as it reaches infinity, where the force of gravity becomes zero. Since $U \rightarrow 0 $ as $r \rightarrow \infty $
            \bigbreak \noindent 
            Thus, we find the escape velocity from the surface of an astronomical body of mass \(M\) and radius \(R\) by setting the total energy equal to zero. At the surface of the body, the object is located at \(r_1 = R\) and it has escape velocity \(v_1 = v_{\text{esc}}\). It reaches \(r_2 = \infty\) with velocity \(v_2 = 0\). Substituting the equation above and solving for $v_{esc}$, we find
            \begin{align*}
                v_{esc} = \sqrt{\frac{2GM}{R}}
            .\end{align*}
            \bigbreak \noindent 
            \textbf{Note:} Notice that m has canceled out of the equation. The escape velocity is the same for all objects, regardless of mass. Also, we are not restricted to the surface of the planet; R can be any starting point beyond the surface of the planet.
        \item \textbf{Not gravitationally Bound}: 
            As noted earlier, we see that \(U \rightarrow 0\) as \(r \rightarrow \infty\). If the total energy is zero, then as \(m\) reaches a value of \(r\) that approaches infinity, \(U\) becomes zero and so must the kinetic energy. Hence, \(m\) comes to rest infinitely far away from \(M\). It has "just escaped" \(M\). If the total energy is positive, then kinetic energy remains at \(r = \infty\) and certainly \(m\) does not return. When the total energy is zero or greater, then we say that \(m\) is not gravitationally bound to \(M\).
        \item \textbf{Gravitationally Bonud}:
            On the other hand, if the total energy is negative, then the kinetic energy must reach zero at some finite value of \(r\), where \(U\) is negative and equal to the total energy. The object can never exceed this finite distance from \(M\), since to do so would require the kinetic energy to become negative, which is not possible. We say \(m\) is gravitationally bound to \(M\).
        \item \textbf{Speed of orbit}
            \begin{align*}
                v_{\text{orbit}} = \sqrt{\frac{GM_{E}}{r}}
            .\end{align*}
        \item \textbf{Period of a circular orbit}:
            \begin{align*}
                T = 2\pi \sqrt{\frac{r^{3}}{GM_{E}}}
            .\end{align*}
        \item \textbf{Total energy for a circular orbit}:
            \begin{align*}
               E = K + U = -\frac{GM_{E}m}{2r} 
            .\end{align*}
            \bigbreak \noindent 
            For circular orbits, the magnitude of the kinetic energy is exactly one-half the magnitude of the potential energy.
        \item \textbf{Keplars first law}: Kepler’s first law states that every planet moves along an ellipse, with the Sun located at a focus of the ellipse.
        \item \textbf{Form of all conics}: There are four different conic sections, all given by the equation
            \begin{align*}
                \frac{a}{r} = 1+e\cos{\left(\theta \right)}
            .\end{align*}
            \bigbreak \noindent 
            The variables $r$ and $\theta$ are shown in the figure below. In the case of an ellipse. The constants $\alpha$ and $e$ are determined by the total energy and angular momentum of the satellite at a given point. The constant $e$ is called the eccentricity. The values of $\alpha$ and $e$ determine which of the four conic sections represents the path of the satellite.
            \bigbreak \noindent 
            \fig{.6}{./figures/kep.jpeg}
            \bigbreak \noindent 
            Every path taken by m is one of the four conic sections: a circle or an ellipse for bound or closed orbits, or a parabola or hyperbola for unbounded or open orbits.


        \item \textbf{Total energy for an elliptical orbit}
            \begin{align*}
                E = -\frac{GmM_{S}}{2a}
            .\end{align*}
            Where $M_{s}$ is the mass of the sun and $a$ is the semi-major axis
    \item \textbf{Keplar's second law}
        Kepler’s second law states that a planet sweeps out equal areas in equal times, that is, the area divided by time, called the areal velocity, is constant.
        \bigbreak \noindent 
        \fig{.6}{./figures/law2.jpeg}
        \bigbreak \noindent 
        The time it takes a planet to move from position \(A\) to \(B\), sweeping out area \(A_1\), is exactly the time taken to move from position \(C\) to \(D\), sweeping area \(A_2\), and to move from \(E\) to \(F\), sweeping out area \(A_3\). These areas are the same: \(A_1 = A_2 = A_3\).
    \item \textbf{semi-major axis}
        \begin{align*}
            a = \frac{1}{2}(\text{aphelion}\, + \, \text{perihelion})
        .\end{align*}
    \item \textbf{Keplar's third law}: Kepler’s third law states that the square of the period is proportional to the cube of the semi-major axis of the orbit. 
        \begin{align*}
            T^{2} = \frac{4\pi^{2}}{GM}a^{3}            
        .\end{align*}
    \item \textbf{Schwarzschild radius}:
        For any mass $M$, if that mass were compressed to the extent that its radius becomes less than the Schwarzschild radius, then the mass will collapse to a singularity, and anything that passes inside that radius cannot escape. Once inside $R_{S}$, the arrow of time takes all things to the singularity. (In a broad mathematical sense, a singularity is where the value of a function goes to infinity. In this case, it is a point in space of zero volume with a finite mass. Hence, the mass density and gravitational energy become infinite.) The Schwarzschild radius is given by
        \begin{align*}
            R_{S} = \frac{2GM}{c^{2}}
        .\end{align*}
        \bigbreak \noindent 
        \textbf{Note:} The Schwarzschild radius is also called the event horizon of a black hole.
    \end{itemize}

    \pagebreak 
    \subsection{Chapter 9: Linear momentum and collisions}
    \bigbreak \noindent 
    \subsubsection{Definitions and Theorems}
    \begin{itemize}
        \item \textbf{Momentum} is a quantity of motion and is defined as the vector 
            \begin{align*}
                \vec{\mathbf{p}} = m\vec{\mathbf{v}}
            .\end{align*}
            \bigbreak \noindent 
            So we see the momentum $p$ of an object is the product of its mass and its velocity:
            \bigbreak \noindent 
            \textbf{Note:} kinetic energy. It is perhaps most useful when determining whether an object’s motion is difficult or easy to change over a short time interval.
        \item \textbf{Impulse}: The product of a force and a time interval (over which that force acts) is called impulse, and is given the symbol  $\vec{\mathbf{J}}$. Formally we write
            \bigbreak \noindent 
            Let  $\vec{\mathbf{F}}(t)$
            be the force applied to an object over some differential time interval dt. The resulting impulse on the object is defined as
            \begin{align*}
                d\vec{\mathbf{J}} = \vec{\mathbf{F}}(t)dt
            .\end{align*}
            \bigbreak \noindent 
            The total impulse over the interval $t_{f} - t_{i}$ is
            \begin{align*}
                \vec{\mathbf{J}} = \int_{t_{1}}^{t_{2}}\ d\vec{\mathbf{J}} \quad \text{or} \quad \int_{t_{1}}^{t_{2}}\ \vec{\mathbf{F}}(t)\ dt
            .\end{align*}
        \item \textbf{Impulse with the mean value theorem}:
            To calculate the impulse using Equation 9.3, we need to know the force function $F(t)$, which we often don’t. However, a result from calculus is useful here: Recall that the average value of a function over some interval is calculated by
            \[
                f_{\text{ave}}(x) = \frac{1}{\Delta x} \int_{x_i}^{x_f} f(x) \, dx
            \]
            where $\Delta x = x_f - x_i$. Applying this to the time-dependent force function, we obtain
            \[
                \vec{F}_{\text{ave}} = \frac{1}{\Delta t} \int_{t_i}^{t_f} \vec{F}(t) \, dt.
            \]
            Therefore, from the equation above
            \[
                \vec{J} = \vec{F}_{\text{ave}} \Delta t.
            \]
            \bigbreak \noindent 
            \textbf{Note:} The idea here is that you can calculate the impulse on the object even if you don’t know the details of the force as a function of time; you only need the average force. In fact, though, the process is usually reversed: You determine the impulse (by measurement or calculation) and then calculate the average force that caused that impulse.
        \item \textbf{Impulse again}
            To calculate the impulse, a useful result follows from writing the force in Equation 9.3 as $\vec{F}(t) = m\vec{a}(t)$:
            \begin{align*}
                \vec{J} = \int_{t_i}^{t_f} \vec{F}(t) \, dt = m\int_{t_i}^{t_f} \vec{a}(t) \, dt = m[\vec{v}(t_f) - \vec{v}_I]. \\
                \therefore \vec{\mathbf{J}} = m\Delta\vec{\mathbf{v}}
            .\end{align*}
        \item \textbf{impulse-momentum theorem}:
            Because $\vec{mv}$ is the momentum of a system, $m\Delta\vec{v}$ is the change of momentum $\Delta\vec{p}$. This gives us the following relation,
            \bigbreak \noindent 
            An impulse applied to a system changes the system’s momentum, and that change of momentum is exactly equal to the impulse that was applied:
            \begin{align*}
                \vec{\mathbf{J}} = \Delta \vec{\mathbf{p}}
            .\end{align*}
        \item \textbf{Two crucial concepts in the impulse-momentum theorem:}
            \begin{itemize}
                \item Impulse is a vector quantity; an impulse of, say, $-(10\, \text{N} \cdot s) \hat{i}$ is very different from an impulse of $+(10\, \text{N} \cdot s) \hat{i}$; they cause completely opposite changes of momentum.
                \item An impulse does not cause momentum; rather, it causes a change in the momentum of an object. Thus, you must subtract the initial momentum from the final momentum, and---since momentum is also a vector quantity---you must take careful account of the signs of the momentum vectors.
            \end{itemize}
        \item \textbf{Problem solving: Impulse momentum theorem}
        \item \textbf{Average force with momentum}
            \begin{align*}
                \vec{\mathbf{F}}_{\text{ave}} = \frac{\Delta \vec{\mathbf{p}}}{\Delta t}
            .\end{align*}
        \item \textbf{Force with rate of change in momentum (Newton's second law in terms of momentum)}
            \begin{align*}
                \vec{\mathbf{F}} = \frac{d\vec{\mathbf{p}}}{dt}
            .\end{align*}
            \bigbreak \noindent 
            This says that the rate of change of the system’s momentum (implying that momentum is a function of time) is exactly equal to the net applied force (also, in general, a function of time). This is, in fact, Newton’s second law, written in terms of momentum rather than acceleration.
        \item \textbf{Newton's third law with momentum}
            \begin{align*}
                m_{1} \frac{d\vec{\mathbf{v}}_{1}}{dt} = -m_{2} \frac{d\vec{\mathbf{v}}_{2}}{dt}
            .\end{align*}
        \item \textbf{Newton's third law with momentum if mass remains constant}: We can then pull the masses inside the derivatives (see equation above)
            \begin{align*}
               \frac{d}{dt}(m_{1}\vec{\mathbf{v}}_{1}) = -\frac{d}{dt}(m_{2}\vec{\mathbf{v}}_{2}) 
            .\end{align*}
            and thus
            \begin{align*}
                \frac{d\vec{\mathbf{p}}_{1}}{dt} = - \frac{d\vec{\mathbf{p}}_{2}}{dt}
            .\end{align*}
            \bigbreak \noindent 
            This says that the rate at which momentum changes is the same for both objects. The masses are different, and the changes of velocity are different, but the rate of change of the product of $m$ and  $\vec{\mathbf{v}}$ are the same.
        \item \textbf{Newton's third law with momentum if mass remains constant (additive inverse version)}
            \begin{align*}
                \frac{d\vec{\mathbf{p}}_{1}}{dt} + \frac{d\vec{\mathbf{p}}_{2}}{dt} = 0
            .\end{align}
            \bigbreak \noindent 
            This says that during the interaction, although object 1’s momentum changes, and object 2’s momentum also changes, these two changes cancel each other out, so that the total change of momentum of the two objects together is zero.
            \bigbreak \noindent 
            Since the total combined momentum of the two objects together never changes, then we could write
            \begin{align*}
                &\frac{d}{dt} (\vec{\mathbf{p}}_{1} + \vec{\mathbf{p}}_{2}) = 0 \\
                &\implies \vec{\mathbf{p}}_{1} + \vec{\mathbf{p}}_{2} = \text{constant}
            .\end{align*}
        \item \textbf{Conservation laws}:
            If the value of a physical quantity is constant in time, we say that the quantity is conserved.
        \item \textbf{Requirements for Momentum Conservation}:
            There is a complication, however. A system must meet two requirements for its momentum to be conserved:
            \begin{itemize}
                \item The mass of the system must remain constant during the interaction.
                    As the objects interact (apply forces on each other), they may transfer mass from one to another; but any mass one object gains is balanced by the loss of that mass from another. The total mass of the system of objects, therefore, remains unchanged as time passes:
                    \begin{align*}
                        \bigg[\frac{dm}{dt}\bigg]_{\text{system}} = 0
                    .\end{align*}
                \item The net external force on the system must be zero.
                    As the objects collide, or explode, and move around, they exert forces on each other. However, all of these forces are internal to the system, and thus each of these internal forces is balanced by another internal force that is equal in magnitude and opposite in sign. As a result, the change in momentum caused by each internal force is cancelled by another momentum change that is equal in magnitude and opposite in direction. Therefore, internal forces cannot change the total momentum of a system because the changes sum to zero. However, if there is some external force that acts on all of the objects (gravity, for example, or friction), then this force changes the momentum of the system as a whole; that is to say, the momentum of the system is changed by the external force. Thus, for the momentum of the system to be conserved, we must have
                    \begin{align*}
                        \vec{\mathbf{F}}_{\text{ext}} = \vec{\mathbf{0}}
                    .\end{align*}
            \end{itemize}
        \item \textbf{Closed system}:
            A system of objects that meets these two requirements is said to be a closed system (also called an isolated system). Thus, the more compact way to express this is shown below.
        \item \textbf{Law of conservation of momentum}:
            The total momentum of a closed system is conserved:
            \begin{align*}
                \sum_{j=1}^{N}\vec{\mathbf{P}}_{j} = \text{constant}
            .\end{align*}
            \bigbreak \noindent 
            This statement is called the Law of \textbf{Conservation of Momentum}
            \bigbreak \noindent 
            \textbf{Note:} Note that there absolutely can be external forces acting on the system; but for the system’s momentum to remain constant, these external forces have to cancel, so that the net external force is zero. Billiard balls on a table all have a weight force acting on them, but the weights are balanced (canceled) by the normal forces, so there is no net force.
        \item \textbf{The meaning of a "system"}:
            A \textbf{system} (mechanical) is the collection of objects in whose motion (kinematics and dynamics) you are interested. If you are analyzing the bounce of a ball on the ground, you are probably only interested in the motion of the ball, and not of Earth; thus, the ball is your system. If you are analyzing a car crash, the two cars together compose your system
        \item \textbf{Problem solving: Conservation of momentum}:
            Using conservation of momentum requires four basic steps. The first step is crucial:
            \begin{enumerate}
                \item Identify a closed system (total mass is constant, no net external force acts on the system).
                \item Write down an expression representing the total momentum of the system before the “event” (explosion or collision).
                \item Write down an expression representing the total momentum of the system after the “event.”
                \item Set these two expressions equal to each other, and solve this equation for the desired quantity.
            \end{enumerate}
        \item \textbf{Explosions}:
            if the object is initially motionless, then the system (which is just the object) has no momentum and no kinetic energy. After the explosion, the net momentum of all the pieces of the object must sum to zero (since the momentum of this closed system cannot change). However, the system will have a great deal of kinetic energy after the explosion, although it had none before. Thus, we see that, although the momentum of the system is conserved in an explosion, the kinetic energy of the system most definitely is not; it increases. This interaction—one object becoming many, with an increase of kinetic energy of the system—is called an \textbf{explosion}.
            \bigbreak \noindent 
            \textbf{Note:} Where does the energy come from? Does conservation of energy still hold? Yes; some form of potential energy is converted to kinetic energy. In the case of gunpowder burning and pushing out a bullet, chemical potential energy is converted to kinetic energy of the bullet, and of the recoiling gun. For a bow and arrow, it is elastic potential energy in the bowstring.
        \item \textbf{Inelastic collisions}:
            two or more objects collide with each other and stick together, thus (after the collision) forming one single composite object. The total mass of this composite object is the sum of the masses of the original objects, and the new single object moves with a velocity dictated by the conservation of momentum. However, it turns out again that, although the total momentum of the system of objects remains constant, the kinetic energy doesn’t; but this time, the kinetic energy decreases. This type of collision is called inelastic.
        \item \textbf{Perfectly inelastic}:
            Any collision where the objects stick together will result in the maximum loss of kinetic energy (i.e., $K_{f}$ will be a minimum). Such a collision is called \textbf{perfectly inelastic}.
        \item \textbf{Elastic collisions}:
            The extreme case on the other end is if two or more objects approach each other, collide, and bounce off each other, moving away from each other at the same relative speed at which they approached each other. In this case, the total kinetic energy of the system is conserved. Such an interaction is called elastic.
        \item \textbf{Classifying collision types}:
            \begin{itemize}
                \item If \(0 < K_f < K_i\), the collision is inelastic.
                \item If \(K_f\) is the lowest energy, or the energy lost by both objects is the most, the collision is perfectly inelastic (objects stick together).
                \item If \(K_f = K_i\), the collision is elastic.
            \end{itemize}
        \item \textbf{Problem solving: Collisions}
            \begin{enumerate}
                \item Define a closed system.
                \item Write down the expression for conservation of momentum.
                \item If kinetic energy is conserved, write down the expression for conservation of kinetic energy; if not, write down the expression for the change of kinetic energy.
                \item You now have two equations in two unknowns, which you solve by standard methods.
            \end{enumerate}
        \item \textbf{Momentum in two dimensions}:
            \begin{align*}
                p_{f,x} = p_{1,i,x} + p_{2,i,x} \\
                p_{f,y} = p_{1,i,y} + p_{2,i,y}
            .\end{align*}
        \item \textbf{Force of a system of particles}: Note that the change of momentum of a system is entirely due the external forces, the sum of the chaneg in momentum of internal forces is zero
            \begin{align*}
                \vec{\mathbf{F}}_{\text{ext}} = \summation{N}{j=1}\ \frac{d\vec{\mathbf{p}_{j}}}{dt}\
            .\end{align*}
            We also know 
            \begin{align*}
                \vec{\mathbf{P}}_{\text{cm}} = \summation{N}{j=1}\ \vec{\mathbf{p}}_{j}\
            .\end{align*}
            \bigbreak \noindent 
            Thus, 
            \begin{align*}
                \vec{\mathbf{F}} = \frac{d \vec{\mathbf{p}}_{cm}}{dt}
            .\end{align*}
            \bigbreak \noindent 
        \item \textbf{Center of mass}:
            \begin{align*}
                \vec{\mathbf{r}}_{cm} = \frac{1}{M}\summation{N}{j=1}\ m_{j}\vec{\mathbf{r}}_{j} \
            .\end{align*}
        \item \textbf{Instantaneous velocity at center of mass}
            \begin{align*}
                \vec{\mathbf{v}}_{\text{cm}} &= \frac{d}{dt}\bigg(\frac{1}{M}\summation{N}{j=1}m_{j}\vec{\mathbf{r}_{j}}\ \bigg) \\
                    &=\frac{1}{M}\summation{N}{j=1}\ m_{j}\vec{\mathbf{v}}_{j}\
            .\end{align*}
        \item \textbf{Center of Mass of Continuous Objects}:
            If the object in question has its mass distributed uniformly in space, rather than as a collection of discrete particles, then  $m_{j} \rightarrow dm $ , and the summation becomes an integral:
            \begin{align*}
                \vec{\mathbf{r}}_{\text{cm}} = \frac{1}{M}\int \vec{\mathbf{r}}dm
            .\end{align*}
        \item \textbf{Conservation of momentum regarding center of mass}
            \begin{align*}
                M\vec{\mathbf{v}}_{\text{cm,f}} = M\vec{\mathbf{v}}_{\text{cm,i}}
            .\end{align*}
            Which implies, for a closed system with constant $M$, the velocites are equal 
        \item \textbf{Rocket equation}:
            \begin{align*}
                \Delta V = u\ln{\left(\frac{m_{0}}{m}\right)}
            .\end{align*}
            Where $\Delta v$ is the change in velocity of the rocket, $u$ is the effective exhaust velocity (which is the average velocity at which the exhaust gases are ejected from the rocket as seen from the rocket frame), $m_{0}$ is the initial mass, and $m$ is the final mass after burning some amount of fuel.
        \item \textbf{Rocket in a Gravitational Field}
            \begin{align*}
                \Delta V = u\ln{\left(\frac{m_{0}}{m}\right)} - g\Delta t
            .\end{align*}
    \item \textbf{Final velocity of each object in elastic collision}
        \begin{align*}
            v_{1,f} = \frac{v_{1,i}(m_{1} - m_{2}) + 2m_{2}v_{2,i}}{m_{1} + m_{2}} \\
            v_{2,f} = \frac{v_{2,i}(m_{2} - m_{1}) + 2m_{1}v_{1,i}}{m_{1} + m_{2}} \\
        .\end{align*}





    \end{itemize}


    \pagebreak 
    \unsect{Calculus III}

    \bigbreak \noindent 
    \subsection{Chapter 1: Parametric equations and polar coordinates}
    \subsubsection{Definitions and Theorems} 
    \begin{itemize}
        \item \textbf{Parametric equations, parameter}: 
            If \( x \) and \( y \) are continuous functions of \( t \) on an interval \( I \), then the equations
            \[ x = x(t) \quad \text{and} \quad y = y(t) \]
            are called \textbf{parametric equations} and \( t \) is called the \textbf{parameter}. 
        \item \textbf{Parametric Curve}: The set of points \( (x, y) \) obtained as \( t \) varies over the interval \( I \) is called the graph of the parametric equations. The graph of parametric equations is called a \textbf{parametric curve} or plane curve, and is denoted by \( C \).
        \item \textbf{Eliminating the parameter:} This allows us to rewrite the two equations as a single equation relating the variables x and y. Then we can apply any previous knowledge of equations of curves in the plane to identify the curve. 
            \begin{itemize}
                \item Solve one of the equations for $t$
                \item Plug the equation for $t$ into the equation still in terms of $t$
                \item Sketch the curve on the interval $I$
            \end{itemize}
        \item \textbf{Domain consideration when graphing by eliminating the parameter}: Suppose we have the equations 
            \begin{align*}
                &x = 2t^{2} \\
                &y = t^{4} + 1
            .\end{align*}
            If we solve $x$ for $t$, we get 
            \begin{align*}
                 t = \pm \sqrt{\frac{1}{2}x}
            .\end{align*}
            We see that the domain of $t$ is restricted to  $t \geq 0$, thus, the graph of this equation will only have the positive side.
        \item \textbf{Parameterizing a curve} is when we start with the equation of a curve and determine a pair of parametric equations for that curve.
            \begin{itemize}
                \item To find the first Parameterization, Define $x(t) = t $ and $y(t)$ as the function given, with $t$ instead of x 
                \item Verify there is no restriction on the domain of the original graph. Thus there is no restriction on the values of $t $
            \end{itemize}
            \begin{itemize}
                \item To find the second parameterization, choose some function for $x(t)$. Ensure that the domain is the set of all real numbers
                \item Plug $x(t)$ into the original function and solve to get $y(t)$
            \end{itemize}
        \item \textbf{Parametric equations for a cylcloid}
            \begin{align*}
                x(t) = a(t-\sin{t}), \quad y(t) = a(1-\cos{t})
            .\end{align*}
        \item \textbf{The general parametric equations for a hypocycloid are:}
            \begin{align*}
                &x(t) = (a-b)\ \cos{t} + b \cos{\left(\frac{a-b}{b}\right)}t \\
                &y(t) = (a-b)\ \sin{t} + b \sin{\left(\frac{a-b}{b}\right)}t \\
            .\end{align*}
        \item \textbf{Derivatives of Parametric Equations}:
            Consider the plane curve defined by the parametric equations \( x = x(t) \) and \( y = y(t) \). Suppose that \( x'(t) \) and \( y'(t) \) exist, and assume that \( x'(t) \neq 0 \). Then the derivative \( \frac{dy}{dx} \) is given by
            \[
                \frac{dy}{dx} = \frac{\frac{dy}{dt}}{\frac{dx}{dt}} = \frac{y'(t)}{x'(t)}.
            \]
        \item \textbf{Second order derivatives of parametric functions}
            \begin{align*}
                \frac{d^{2}y}{dx^{2}} = \frac{d}{dx}\left(\frac{dy}{dx}\right) = \frac{\left(\frac{d}{dt}\right)\left(\frac{dy}{dx}\right)}{\frac{dx}{dt}}
            .\end{align*}
        \item \textbf{Integral involving parametric equations}
            Consider the non-self-intersecting plane curve defined by the parametric equations
            \[
                x = x(t), \quad y = y(t), \quad a \leq t \leq b
            \]
            and assume that \( x(t) \) is differentiable. The area under this curve is given by
            \[
                A = \int_{a}^{b} y(t) x'(t) \, dt.
            \]
        \item \textbf{Arc Length of a Parametric Curve}:
            Consider the plane curve defined by the parametric equations
            \[
                x = x(t), \quad y = y(t), \quad t_1 \leq t \leq t_2
            \]
            and assume that \( x(t) \) and \( y(t) \) are differentiable functions of \( t \). Then the arc length of this curve is given by
            \[
                s = \int_{t_1}^{t_2} \sqrt{\left(\frac{dx}{dt}\right)^2 + \left(\frac{dy}{dt}\right)^2} \, dt.
            \]
            Or simply
            \begin{align*}
                s = \int_{a}^{b}\ \sqrt{1 + \left(\frac{dy}{dx}\right)^{2}}\ dx
            .\end{align*}
        \item \textbf{Surface area for a parametric curve}:
            The analogous formula for a parametrically defined curve is
            \[
                S = 2\pi \int_{a}^{b} y(t) \sqrt{(x'(t))^2 + (y'(t))^2} \, dt
            \]
            provided that \( y(t) \) is not negative on \([a, b]\).
        \item \textbf{Parametric equations for a circle (with $(h,k) = (0,0)$)}
            \begin{align*}
                &x(t) = r\cos{(t)} \\
                &y(t) = r\sin{(t)}
            .\end{align*}
            For $0 \leq t \leq 2\pi $
        \item \textbf{Parametric equations for the upper half of a semi-circle (with $(h,k) = (0,0)$)}
            \begin{align*}
                &x(t) = r\cos{(t)} \\
                &y(t) = r\sin{(t)}
            .\end{align*}
            For $0 \leq t \leq \pi $
        \item \textbf{Parametric equations for the lower half of a semi-circle (with $(h,k) = (0,0)$)}
            \begin{align*}
                &x(t) = r\cos{(t)} \\
                &y(t) = r\sin{(t)}
            .\end{align*}
            For $\pi \leq t \leq 2\pi $
        \item \textbf{Note: A graph has a horizontal tangent line when the derivative equals zero.}
        \item \textbf{Note: A graph has a vertical tangent line when the derivative does not exist}
        \item \textbf{Converting Points between Coordinate Systems}: 
            Given a point \( P \) in the plane with Cartesian coordinates \((x, y)\) and polar coordinates \((r, \theta)\), the following conversion formulas hold true:
            \begin{align*}
                x = r \cos \theta \quad \text{and} \quad y = r \sin \theta,
            .\end{align*}
            \begin{align*}
                r^2 = x^2 + y^2 \quad \text{and} \quad \tan \theta = \frac{y}{x}.
            .\end{align*}
            These formulas can be used to convert from rectangular to polar or from polar to rectangular coordinates.
        \item \textbf{Sometimes finding $\theta$ with the formula is not possible.} Consider the point $P(0,3)$, using the formula $\tan{\theta } = \frac{3}{0}$ is undefined. Instead, we graph the point $(0,3)$ and observe the angle between the $x$ and $y$ axis is $\frac{\pi}{2}$ 
        \item \textbf{The polar representation of a point is not unique.} Every point in the plane has an infinite number of representations in polar coordinates. However, each point in the plane has only one representation in the rectangular coordinate system.
        \item The line segment starting from the center of the graph going to the right (called the positive x-axis in the Cartesian system) is the \textbf{polar axis}. 
        \item The center point is the \textbf{pole}, or origin, of the coordinate system, and corresponds to  $r=0$.
        \item \textbf{If the value of  $r$ is positive}, move that distance along the terminal ray of the angle. \textbf{If it is negative}, move along the ray that is opposite the terminal ray of the given angle.
        \item A \textbf{Polar equation} has the form
            \begin{align*}
                r = f(\theta )
            .\end{align*}
        \item \textbf{Plotting a Curve in Polar Coordinates}
            \begin{enumerate}
                \item Create a table with two columns. The first column is for \( \theta \), and the second column is for \( r \).
                \item Create a list of values for \( \theta \).
                \item Calculate the corresponding \( r \) values for each \( \theta \).
                \item Plot each ordered pair \( (r, \theta) \) on the coordinate axes.
                \item Connect the points and look for a pattern.
            \end{enumerate}
            \pagebreak 
        \item \textbf{Types of polar curves}
     \bigbreak \noindent 
     \begin{tabularx}{\textwidth}{|X|X|X|}
        \hline
        Name & Equation & Example \\
        \hline
        Line passing through the pole with slope $\tan{K}$ & $\theta =K$ & \fig{.5}{./figures/14.png}\\
        \hline
        Circle & $r=a\cos{\theta} + b\sin{\theta} $ & \fig{.5}{./figures/16.png}\\
        \hline
        Spiral& $r=a+b\theta  $&\fig{.5}{./figures/15.png} \\
        \hline
     \end{tabularx}

     \pagebreak \bigbreak \noindent 
      \begin{tabularx}{\textwidth}{|X|X|X|}
        \hline
        Name & Equation & Example \\
        \hline
        Cardioid & \fig{.5}{./figures/20.png} & \fig{.5}{./figures/17.png}\\
        \hline
        Limacon & \fig{.5}{./figures/21.png} & \fig{.5}{./figures/18.png}\\
        \hline
        Rose & \fig{.5}{./figures/23.png}&\fig{.5}{./figures/19.png} \\
        \hline
     \end{tabularx}
     \bigbreak \noindent 
 \item      For the rose, \textbf{If the coefficient of $\theta$ is even}, the graph has twice as many petals as the coefficient. \textbf{If the coefficient of $\theta$ is odd, then the number of petals equals the coefficient.}

     \pagebreak 
    \item \textbf{Graphing a polar curve.}
        \begin{itemize}
            \item Make graph in rectangular system (use $\theta$ as horizontal axis and $r$ as vertical)
            \item Translate over to polar
            \item The circles represent values of $r$
        \end{itemize}
        \item \textbf{Area of a Region Bounded by a Polar Curve}:
             Suppose \( f \) is continuous and nonnegative on the interval \( \alpha \leq \theta \leq \beta \) with \( 0 < \beta - \alpha \leq 2\pi \). The area of the region bounded by the graph of \( r = f(\theta) \) between the radial lines \( \theta = \alpha \) and \( \theta = \beta \) is
                \[ A = \frac{1}{2} \int_{\alpha}^{\beta} [f(\theta)]^2 \, d\theta = \frac{1}{2} \int_{\alpha}^{\beta} r^2 \, d\theta. \]
        \item \textbf{Area between two polar curves}:
            \begin{align*}
                A = \frac{1}{2}\int_{\alpha}^{\beta}\ f(\theta)^{2} -g(\theta )^{2}\ d\theta 
            .\end{align*}
            Where $f(\theta ) \geq g(\theta ) \forall\ \alpha \leq \theta \leq \beta $

        \item \textbf{Bounds of integration for area outside some curve and inside some curve}: We find the bounds of integration the same way we found them for regularo functions, we find the points of intersection by setting the two functions equal to each other
        \item \textbf{Arc Length of a Curve Defined by a Polar Function}:
            Let \( f \) be a function whose derivative is continuous on an interval \( \alpha \leq \theta \leq \beta \). The length of the graph of \( r = f(\theta) \) from \( \theta = \alpha \) to \( \theta = \beta \) is
            \begin{align*}
              &L = \int_{\alpha}^{\beta} \sqrt{[f(\theta)]^2 + [f'(\theta)]^2} \, d\theta \\
              &= \int_{\alpha}^{\beta} \sqrt{r^2 + \left(\frac{dr}{d\theta}\right)^2} \, d\theta. 
          .\end{align*}
        \item \textbf{Absolute value bars}: Consider the integral
            \begin{align*}
                20 \int_{0}^{2\pi}\ \sqrt{\cos^{2}{\left(\frac{\theta }{2}\right)}}\ d\theta 
            .\end{align*}
            If we cancel out the sqrt and square, we must include absulute value bars. This is because $\cos{\left(\frac{\theta }{2}\right)}$ can be negative on the interval $[0,2\pi]$. The integral becomes
            \begin{align*}
                20 \int_{0}^{2\pi}\ \bigg\lvert \cos{\left(\frac{\theta}{2}\right)} \bigg\rvert\ d\theta 
            .\end{align*}
            To solve this, we use symmetry to change the bounds. Because the graph of the polar curve ($10+10\cos{\theta}$) is symmetric, we can integrate instead over the range $0,\pi$, and multiply the result by 2.
        \item \textbf{Derivative of polar equaotion}: Given some function $r=f(\theta )$. We can find the derivative with
            \begin{align*}
                \frac{dy}{dx} = \frac{\frac{dy}{d\theta}}{\frac{dx}{d\theta}} = \frac{\frac{dr}{d\theta }\sin{\theta } + r\cos{\theta }}{\frac{dr}{d\theta }\cos{\theta } - r\sin{\theta}}
            .\end{align*}
        \item \textbf{Find equation of tangent line given $r = f(\theta )$ and point $\theta$}. 
            \begin{itemize}
                \item Find $r$ with given $\theta$.
                \item Plug $r$ into $x=r\cos{\theta }$, $y=r\sin{\theta }$ to get point $P(x,y)$
                \item Find $\frac{dy}{dx}$ of $r=f(\theta)$
                \item Plug in $\theta$ to get slope $m$
                \item Use point slope form to find equation
            \end{itemize}






    \end{itemize}

    \pagebreak 
    \subsubsection{Problems to remember}
    \begin{itemize}
        \item \textbf{Eliminating the parameter:}
            Sometimes it is necessary to be a bit creative in eliminating the parameter. The parametric equations for this example are
            \begin{align*}
                x(t) = 4\cos{t}, \quad y(t) = 3\sin{t}
            .\end{align*}
            \bigbreak \noindent 
            Solving either equation for $t$ directly is not advisable because sine and cosine are not one-to-one functions. However, dividing the first equation by 4 and the second equation by 3 (and suppressing the $t$) gives us
            \begin{align*}
                \cos{t} = \frac{x}{4}, \quad \sin{t} = \frac{y}{3}
            .\end{align*}
            \bigbreak \noindent 
            Now use the Pythagorean identity  $\cos^{2}{t} + \sin^{2}{t} = 1$ and replace the expressions for $\sin{t}$ and $\cos{t}$ with the equivalent expressions in terms of $x$ and $y$. This gives
            \begin{align*}
        &\left(\frac{x}{4}\right)^{2} +  \left(\frac{y}{3}\right)^{2} = 1 \\
        &\frac{x^{2}}{16}  + \frac{y^{2}}{9} = 1
    .\end{align*}
    \bigbreak \noindent 
    This is the equation of a horizontal ellipse centered at the origin, with semimajor axis 4 and semiminor axis 3 as shown in the following graph. 
    \item \textbf{Convert from cartesian to polar}: Consider the point $P(1,1)$. First we find $r$ and $\theta$
        \begin{align*}
            &r^{2} = x^{2} + y^{2} \\
            &r = \sqrt{1^{2} + 1^{2}} \\
            &r=\sqrt{2} \\
            &\tan{\theta } = \frac{y}{x} = \frac{1}{1} \\
            &\theta  = \tan^{-1}{1} \\
            &\theta  = \frac{\pi}{4}
        .\end{align*}
        Thus we have the point $\left(\sqrt{2}, \frac{\pi}{4}\right) $
    \item \textbf{Graphing a polar equation}: Graph the curve defined by the function $r=4\sin{\theta}$. Identify the curve and rewrite the equation in rectangular coordinates.
        \bigbreak \noindent 
            Because the function is a multiple of a sine function, it is periodic with period  $2\pi$, so use values for $\theta$ between 0 and  $2\pi$
    \bigbreak \noindent 
    \begin{center}
        \begin{tabular}{cc|cc}
            \toprule
            \( \theta \) & \( r = 4\sin\theta \) & \( \theta \) & \( r = 4\sin\theta \) \\
            \midrule
            0 & 0 & \( \pi \) & 0 \\
            \( \frac{\pi}{6} \) & 2 & \( \frac{7\pi}{6} \) & -2 \\
            \( \frac{\pi}{4} \) & \( 2\sqrt{2} \approx 2.8 \) & \( \frac{5\pi}{4} \) & \( -2\sqrt{2} \approx -2.8 \) \\
            \( \frac{\pi}{3} \) & \( 2\sqrt{3} \approx 3.4 \) & \( \frac{4\pi}{3} \) & \( -2\sqrt{3} \approx -3.4 \) \\
            \( \frac{\pi}{2} \) & 4 & \( \frac{3\pi}{2} \) & -4 \\
            \( \frac{2\pi}{3} \) & \( 2\sqrt{3} \approx 3.4 \) & \( \frac{5\pi}{3} \) & \( -2\sqrt{3} \approx -3.4 \) \\
            \( \frac{3\pi}{4} \) & \( 2\sqrt{2} \approx 2.8 \) & \( \frac{7\pi}{4} \) & \( -2\sqrt{2} \approx -2.8 \) \\
            \( \frac{5\pi}{6} \) & 2 & \( \frac{11\pi}{6} \) & -2 \\
            \( 2\pi \) & 0 \\
            \bottomrule
        \end{tabular}
     \end{center}
     \bigbreak \noindent 
     Plotting these points gives the graph of a circle. The equation \( r = 4\sin\theta \) can be converted into rectangular coordinates by first multiplying both sides by \( r \). This gives the equation \( r^2 = 4r\sin\theta \). Next use the facts that \( r^2 = x^2 + y^2 \) and \( y = r\sin\theta \). This gives \( x^2 + y^2 = 4y \). To put this equation into standard form, subtract \( 4y \) from both sides of the equation and complete the square:
     \[
         x^2 + (y^2 - 4y) = x^2 + (y^2 - 4y + 4) = x^2 + (y - 2)^2 = 0 + 4.
     \]
     This is the equation of a circle with radius 2 and center \( (0, 2) \) in the rectangular coordinate system.

     \item \textbf{Finding the Arc Length of a Polar Curve}: 
         \begin{align*}
          Find the arc length of the polar curve 
              r = 2 + 2\cos{\theta }
          .\end{align*}
          \bigbreak \noindent 
               When \( \theta = 0 \), \( r = 2 + 2\cos(0) = 4 \). Furthermore, as \( \theta \) goes from \( 0 \) to \( 2\pi \), the cardioid is traced out exactly once. Therefore, these are the limits of integration. Using \( f(\theta) = 2 + 2\cos(\theta) \), \( \alpha = 0 \), and \( \beta = 2\pi \). Thus we have
     \begin{align*}
          &L = \int_{\alpha}^{\beta} \sqrt{[f(\theta)]^2 + [f'(\theta)]^2} \, d\theta  \\
          &= \int_{0}^{2\pi} \sqrt{[2 + 2\cos(\theta)]^2 + [-2\sin(\theta)]^2} \, d\theta  \\
          &= \int_{0}^{2\pi} \sqrt{4 + 8\cos(\theta) + 4\cos^2(\theta) + 4\sin^2(\theta)} \, d\theta  \\
          &= \int_{0}^{2\pi} \sqrt{4 + 8\cos(\theta) + 4(\cos^2(\theta) + \sin^2(\theta))} \, d\theta  \\
          &= \int_{0}^{2\pi} \sqrt{8 + 8\cos(\theta)} \, d\theta  \\
          &= 2 \int_{0}^{2\pi} \sqrt{2 + 2\cos(\theta)} \, d\theta.
     .\end{align*}
     Next, using the identity \( \cos(2\alpha) = 2\cos^2(\alpha) - 1 \), add 1 to both sides and multiply by 2. This gives \( 2 + 2\cos(2\alpha) = 4\cos^2(\alpha) \). Substituting \( \alpha = \frac{\theta}{2} \) gives \( 2 + 2\cos(\theta) = 4\cos^2\left(\frac{\theta}{2}\right) \), so the integral becomes
     \begin{align*}
          &L = 2 \int_{0}^{2\pi} \sqrt{2 + 2\cos(\theta)} \, d\theta  \\
          &= 2 \int_{0}^{2\pi} \sqrt{4\cos^2\left(\frac{\theta}{2}\right)} \, d\theta  \\
          &= 2 \int_{0}^{2\pi} 2 \left| \cos\left(\frac{\theta}{2}\right) \right| \, d\theta. 
     .\end{align*}
     The absolute value is necessary because the cosine is negative for some values in its domain. To resolve this issue, change the limits from \( 0 \) to \( \pi \) and double the answer. This strategy works because cosine is positive between \( 0 \) and \( \frac{\pi}{2} \). Thus,
     \begin{align*}
          &L = 4 \int_{0}^{2\pi} \left| \cos\left(\frac{\theta}{2}\right) \right| \, d\theta  \\
          &= 8 \int_{0}^{\pi} \cos\left(\frac{\theta}{2}\right) \, d\theta  \\
          &= 8 \left[ 2\sin\left(\frac{\theta}{2}\right) \right]_{0}^{\pi}  \\
          &= 16. 
     .\end{align*}


    \end{itemize}

    \pagebreak 
    \subsection{Chapter 2: Vectors in Space}

    \bigbreak \noindent 
    \subsubsection{Definitions and Theorems}
    \begin{itemize}
        \item The endpoints of the segment are called the \textbf{initial point} and the \textbf{terminal point} of the vector.
        \item The length of the line segment represents its magnitude. We use the notation  $\norm{\vec{v}}$ to denote the magnitude of the vector  $\vec{v}$.
        \item A vector with an initial point and terminal point that are the same is called the \textbf{zero vector}, denoted $\vec{0}$.
        \item Vectors with the same magnitude and direction are called \textbf{equivalent vectors}. We treat equivalent vectors as equal, even if they have different initial points. Thus, if  $\vec{v}$ and  $\vec{w}$ are equivalent, we write
      \begin{align*}
         \vec{v} = \vec{w} 
      .\end{align*}
  \item \textbf{Scalar Multiplication}:
      The product $k\vec{v}$ of a vector $\vec{v}$ and a scalar $k$ is a vector with a magnitude that is $|k|$ times the magnitude of $\vec{v}$, and with a direction that is the same as the direction of $\vec{v}$ if $k > 0$, and opposite the direction of $\vec{v}$ if $k < 0$. This is called scalar multiplication. If $k = 0$ or $\vec{v} = \vec{0}$, then $k\vec{v} = \vec{0}$.
    \item \textbf{Vector Addition}: 
        The sum of two vectors $\vec{v}$ and $\vec{w}$ can be constructed graphically by placing the initial point of $\vec{w}$ at the terminal point of $\vec{v}$. Then, the vector sum, $\vec{v} + \vec{w}$, is the vector with an initial point that coincides with the initial point of $\vec{v}$ and has a terminal point that coincides with the terminal point of $\vec{w}$. This operation is known as vector addition.
        \bigbreak \noindent 
        \fig{.8}{./figures/c21.jpeg}
    \item \textbf{Vector difference}: 
        We define $\vec{v} - \vec{w}$ as $\vec{v} + (-\vec{w}) = \vec{v} + (-1)\vec{w}$. The vector $\vec{v} - \vec{w}$ is called the vector difference. Graphically, the vector $\vec{v} - \vec{w}$ is depicted by drawing a vector from the terminal point of $\vec{w}$ to the terminal point of $\vec{v}$.
        \bigbreak \noindent 
        \fig{.8}{./figures/c22.jpeg}
    \item \textbf{Triangle inequality}: the length of any one side is less than the sum of the lengths of the remaining sides. So we have
        \begin{align*}
            \norm{\vec{v} + \vec{w}} \leq \norm{\vec{v}} + \norm{\vec{w}}
        .\end{align*}
    \item We call a vector with its initial point at the origin a \textbf{standard-position} vector.
    \item \textbf{Component form of a vector}:
        The vector with initial point $(0,0)$ and terminal point $(x,y)$ can be written in component form as
        \[
            \vec{v} = \langle x, y \rangle.
        \]
        The scalars $x$ and $y$ are called the components of $\mathbf{v}$.
    \item If a vector is in standard position, and its components are the same as the terminal point \textbf{}
    \item \textbf{Component form of a vector not in standard position}: 
        Let $\vec{v}$ be a vector with initial point $(x_i, y_i)$ and terminal point $(x_t, y_t)$. Then we can express $\vec{v}$ in component form as $\vec{v} = \langle x_t - x_i, y_t - y_i \rangle.$
    \item \textbf{Magnitude of vector}: If a vector is given by components $\langle x,y \rangle $. Then this is the vector with initial point at the orgin (0,0), and terminal point at (x,y). We find the magnitude of the vector with
        \begin{align*}
            \norm{\vec{v}} = \sqrt{x^{2} + y^{2}}
        .\end{align*}
        If the vector is not in standard position, and we have initial point $(x_{1}, x_{2})$ and terminal point $(x_{2}, y_{2})$, then we find the magnitude with 
        \begin{align*}
            \norm{\vec{v}} = \sqrt{(x_{2} - x_{1})^{2} + (y_{2} - y_{1})^{2}}
        .\end{align*}
    \item \textbf{Scalar multiplication, and vector addition (component form)}
        Let $\mathbf{v} = \langle x_1, y_1 \rangle$ and $\mathbf{w} = \langle x_2, y_2 \rangle$ be vectors, and let $k$ be a scalar.
        \begin{align*}
            &\text{Scalar multiplication: } k\mathbf{v} = \langle kx_1, ky_1 \rangle \\
            &\text{Vector addition: } \mathbf{v} + \mathbf{w} = \langle x_1, y_1 \rangle + \langle x_2, y_2 \rangle = \langle x_1 + x_2, y_1 + y_2 \rangle
        .\end{align*}
    \item \textbf{Properties of Vector Operations}:
        Let $\mathbf{u}$, $\mathbf{v}$, and $\mathbf{w}$ be vectors in a plane. Let $r$ and $s$ be scalars.
        \begin{enumerate}
            \item $\mathbf{u} + \mathbf{v} = \mathbf{v} + \mathbf{u}$ \quad (Commutative property)
            \item $(\mathbf{u} + \mathbf{v}) + \mathbf{w} = \mathbf{u} + (\mathbf{v} + \mathbf{w})$ \quad (Associative property)
            \item $\mathbf{u} + \mathbf{0} = \mathbf{u}$ \quad (Additive identity property)
            \item $\mathbf{u} + (-\mathbf{u}) = \mathbf{0}$ \quad (Additive inverse property)
            \item $r(s\mathbf{u}) = (rs)\mathbf{u}$ \quad (Associativity of scalar multiplication)
            \item $(r + s)\mathbf{u} = r\mathbf{u} + s\mathbf{u}$ \quad (Distributive property)
            \item $r(\mathbf{u} + \mathbf{v}) = r\mathbf{u} + r\mathbf{v}$ \quad (Distributive property)
            \item $1\mathbf{u} = \mathbf{u}, \quad 0\mathbf{u} = \mathbf{0}$ \quad (Identity and zero properties)
        \end{enumerate}
        \textbf{Note} $CAAA^{-1}(ASM)D^{2}(I\&Z)$
    \item \textbf{Finding components of a vector given the magnitude and the angle $\theta$}
        \begin{align*}
            &x = \norm{\vec{v}}\cos{\theta } \\
            &y = \norm{\vec{v}}\sin{\theta }
        .\end{align*}
    \item \textbf{Unit vector}: A unit vector is a vector with magnitude $1$. For any nonzero vector $\vec{v}$, we can use scalar multiplication to find a unit vector $\vec{u}$ that has the same direction as $\vec{v}$. To do this, we multiply the vector by the reciprocal of its magnitude:
        \[
            \vec{u} = \frac{1}{\lVert \vec{v} \rVert} \vec{v}.
        \]

    \item \textbf{xy-plane}: 
        \begin{align*}
            \{(x,y,0)\ :\ x,y \in \mathbb{R}\}
        .\end{align*}
        Described by the equation $z=0$
    \item \textbf{xz-plane}
        \begin{align*}
            \{(x,0,z)\ :\ x,z \in \mathbb{R}\}
        .\end{align*}
        Described by the equation $y=0$
    \item \textbf{yz-plane}
        \begin{align*}
            \{(0,y,z)\ :\ y,z \in \mathbb{R}\}
        .\end{align*}
        Described by the equation $x=0$
    \item the coordinate planes divide space between them into eight regions about the origin, called \textbf{octants}. The octants fill  $\mathbb{R}^{3}$ the same way that quadrants fill  $\mathbb{R}^{2}$,
    \item \textbf{Distance formula for three-dimensional space}:
        The distance $d$ between points $(x_1, y_1, z_1)$ and $(x_2, y_2, z_2)$ is given by the formula
        \[
            d = \sqrt{(x_2 - x_1)^2 + (y_2 - y_1)^2 + (z_2 - z_1)^2}.
        \]
    \item \textbf{Equations of planes parallel to coordinate planes}
        \begin{enumerate}
            \item The plane in space that is parallel to the xy-plane and contains point  (a,b,c) can be represented by the equation $z=c$.
            \item The plane in space that is parallel to the xz-plane and contains point  (a,b,c) can be represented by the equation  $y=b$.
            \item The plane in space that is parallel to the yz-plane and contains point  (a,b,c) can be represented by the equation  $x=a$
        \end{enumerate}
    \item         A \textbf{sphere} is the set of all points in space equidistant from a fixed point, the center of the sphere, just as the set of all points in a plane that are equidistant from the center represents a circle. In a sphere, as in a circle, the distance from the center to a point on the sphere is called the \textit{radius}.
    \item \textbf{Standard equation of a sphere}:
        The sphere with center $(a,b,c)$ and radius $r$ can be represented by the equation
        \[
            (x - a)^2 + (y - b)^2 + (z - c)^2 = r^2.
        \]
        This equation is known as the \textbf{standard equation of a sphere}.
    \item \textbf{Properties of vectors in space}:
        Let $\vec{v} = \langle x_1, y_1, z_1 \rangle$ and $\vec{w} = \langle x_2, y_2, z_2 \rangle$ be vectors, and let $k$ be a scalar.
        \begin{itemize}
            \item Scalar multiplication:
                \begin{align*}
                     k\vec{v} = \langle kx_1, ky_1, kz_1 \rangle
                .\end{align*}
            \item Vector addition: 
                \begin{align*}
                    \vec{v} + \vec{w} = \langle x_1, y_1, z_1 \rangle + \langle x_2, y_2, z_2 \rangle = \langle x_1 + x_2, y_1 + y_2, z_1 + z_2 \rangle
                .\end{align*}

            \item Vector subtraction: 
                \begin{align*}
                    \vec{v} - \vec{w} = \langle x_1, y_1, z_1 \rangle - \langle x_2, y_2, z_2 \rangle = \langle x_1 - x_2, y_1 - y_2, z_1 - z_2 \rangle
                .\end{align*}
            \item Vector magnitude: 
                \begin{align*}
                    \|\vec{v}\| = \sqrt{x_1^2 + y_1^2 + z_1^2}
                .\end{align*}
            \item Unit vector in the direction of $\vec{v}$:
                \begin{align*}
                    &\frac{1}{\|\vec{v}\|}\vec{v} = \frac{1}{\|\vec{v}\|}\langle x_1, y_1, z_1 \rangle  \\
                    &= \bigg\langle \frac{x_1}{\|\vec{v}\|}, \frac{y_1}{\|\vec{v}\|}, \frac{z_1}{\|\vec{v}\|} \bigg\rangle,\ \text{if } \vec{v} \neq 0
                .\end{align*}
            \end{itemize}
            \item \textbf{The dot product}:
                The dot product of vectors $\vec{u} = \langle u_1, u_2, u_3 \rangle$ and $\vec{v} = \langle v_1, v_2, v_3 \rangle$ is given by the sum of the products of the components
                \[
                    \vec{u} \cdot \vec{v} = u_1v_1 + u_2v_2 + u_3v_3.
                \]
                The dot product \textbf{does not} return a new vector, the result is a \textbf{scalar}
            \item \textbf{Properties of the dot product}
                Let $\vec{u}$, $\vec{v}$, and $\vec{w}$ be vectors, and let $c$ be a scalar.
                \begin{enumerate}
                    \item Commutative property: $\vec{u} \cdot \vec{v} = \vec{v} \cdot \vec{u}$
                    \item Distributive property: $\vec{u} \cdot (\vec{v} + \vec{w}) = \vec{u} \cdot \vec{v} + \vec{u} \cdot \vec{w}$
                    \item Associative property of scalar multiplication: $(c\vec{u} \cdot \vec{v}) = (c\vec{u}) \cdot \vec{v} = \vec{u} \cdot (c\vec{v})$
                    \item Property of magnitude: $\vec{v} \cdot \vec{v} = \|\vec{v}\|^2$
                \end{enumerate}
            \item \textbf{Evaluating a dot product}: 
                The dot product of two vectors is the product of the magnitude of each vector and the cosine of the angle between them:
                \begin{align*}
                    \vec{u} \cdot \vec{v} = \norm{\vec{u}} \cdot \norm{\vec{v}} \cdot \cos{\theta }
                .\end{align*}
            \item \textbf{Find the measure of the angle between two nonzero vectors}:
                \begin{align*}
                    \cos{\theta } = \frac{\vec{u} \cdot \vec{v}}{\norm{\vec{u}}\norm{\vec{v}}}
                .\end{align*}
                \bigbreak \noindent 
                \textbf{Note}: We are considering $0 \leq \theta  \leq \pi $
            \item \textbf{Vector Projection}: The vector projection of $\mathbf{v}$ onto $\mathbf{u}$ has the same initial point as $\mathbf{u}$ and $\mathbf{v}$ and the same direction as $\mathbf{u}$, and represents the component of $\mathbf{v}$ that acts in the direction of $\mathbf{u}$.
                \begin{align*}
                    \text{proj}_{\vec{u}}\vec{v} = \frac{\vec{v} \cdot \vec{u}}{\norm{\vec{u}}^{2}}\vec{u}
                .\end{align*}
                We say "The vector projection of $\vec{v}$ onto $\vec{u}$"
            \item \textbf{Scalar projection notation}: This is the length of the vector projection and is denoted
                \begin{align*}
                    \norm{\text{proj}_{\vec{u}}\vec{v}} = \text{comp}_{\vec{u}}\vec{v} = \frac{\vec{u} \cdot \vec{v}}{\norm{\vec{u}}}
                .\end{align*}
                \pagebreak 
            \item \textbf{Decompose some vector $\vec{v}$ into orthogonal components such that one of the component vectors has the same direction as  $\vec{u}$}
                \begin{itemize}
                    \item First, we compute $\vec{p} = \text{proj}_{\vec{u}}\vec{v} $
                    \item Then, we define $\vec{q}  = \vec{v} - \vec{p}$ 
                    \item Check that $\vec{q}$ and $\vec{p}$ are orthogonal by finding $\vec{q} \cdot \vec{p}$
                \end{itemize}
            \item \textbf{Work}:
                When a constant force is applied to an object so the object moves in a straight line from point $P$ to point $Q$, the work $W$ done by the force $\mathbf{F}$, acting at an angle $\theta$ from the line of motion, is given by
                \[ W = \vec{F} \cdot \overrightarrow{PQ} = \|\vec{F}\| \|\overrightarrow{PQ}\| \cos \theta. \]
            \item \textbf{Two vectors are orthogonal if}
                \begin{align*}
                    \vec{u} \cdot \vec{v} = 0
                .\end{align*}
            \item \textbf{Two vectors are parallel if}
                \begin{align*}
                    \exists \alpha\ \text{s.t } \alpha\vec{u} = \vec{v}
                .\end{align*}
            \item \textbf{Scalar projection componets of a vector}
                \begin{align*}
                    \vec{v} = \langle \text{comp}_{\hat{i}}\vec{v}, \text{comp}_{\hat{j}}\vec{v}, \text{comp}_{\hat{k}}\vec{v}\rangle
                .\end{align*}
            \item \textbf{Find direction angles for some vector}: Suppose we have some vector $\vec{v}$, to find the direction angles, we use the formula
                \begin{align*}
                    \cos{\theta} = \frac{\vec{v} \cdot \vec{u}}{\norm{\vec{v}} \norm{\vec{u}}}
                .\end{align*}
                With the unit vectors $\hat{i}, \hat{j}, \hat{k}$. This will gives angles $\alpha,\ \beta,\ \gamma $

            \item \textbf{The Cross Product}: 
                Let $\mathbf{u} = \langle u_1, u_2, u_3 \rangle$ and $\mathbf{v} = \langle v_1, v_2, v_3 \rangle$.
                Then, the cross product $\mathbf{u} \times \mathbf{v}$ is vector
                \begin{align*}
                    \mathbf{u} \times \mathbf{v} &= (u_2 v_3 - u_3 v_2) \mathbf{i} - (u_1 v_3 - u_3 v_1) \mathbf{j} + (u_1 v_2 - u_2 v_1) \mathbf{k}  \\
                                                 &= \langle u_2 v_3 - u_3 v_2, -(u_1 v_3 - u_3 v_1), u_1 v_2 - u_2 v_1 \rangle.
                .\end{align*}
                \bigbreak \noindent 
                \textbf{Note:} The cross product only works in $\mathbb{R}^{3}$, additionally, we measure the angle between $\vec{u}$ and $\vec{v}$ in $\vec{u} \times \vec{v}$ from $\vec{u}$ to $\vec{v}$
            \item \textbf{Cross product using matrix and discriminant}, suppose we have vectors $\vec{u}$ und $\vec{v}$. Then we can express them in matrix form as
                \begin{align*}
                    \vec{u} \times \vec{v}  =
                    \begin{bmatrix}
                        \hat{i} & \hat{j} & \hat{k} \\
                        u_{x} & u_{y} & u_{z} \\
                        v_{x} & v_{y} & v_{z}
                    \end{bmatrix}
                .\end{align*}
                Then we can find the discriminant of this matrix to compute the cross product
                \begin{align*}
                    \vec{u} \times \vec{v} = (u_{y}v_{z} - u_{z}v_{y})\hat{i} - (u_{x}v_{z}-u_{z}v_{x})\hat{k} + (u_{x}v_{y} - u_{y}v_{x})\hat{j}
                .\end{align*}
            \item \textbf{Right hand rule for cross product}: 
                The direction of $\mathbf{u} \times \mathbf{v}$ is given by the right-hand rule. If we hold the right hand out with the fingers pointing in the direction of $\mathbf{u}$, then curl the fingers toward vector $\mathbf{v}$, the thumb points in the direction of the cross product, as shown.
                \bigbreak \noindent 
                \fig{.8}{./figures/righthand.jpeg}
            \item \textbf{Properties of the Cross Product}
                Let $\mathbf{u}$, $\mathbf{v}$, and $\mathbf{w}$ be vectors in space, and let $c$ be a scalar.
                \begin{enumerate}
                    \item Anticommutative property: $\mathbf{u} \times \mathbf{v} = -(\mathbf{v} \times \mathbf{u})$
                    \item Distributive property: $\mathbf{u} \times (\mathbf{v} + \mathbf{w}) = \mathbf{u} \times \mathbf{v} + \mathbf{u} \times \mathbf{w}$
                    \item Multiplication by a constant: $c(\mathbf{u} \times \mathbf{v}) = (c\mathbf{u}) \times \mathbf{v} = \mathbf{u} \times (c\mathbf{v})$
                    \item Cross product of the zero vector: $\mathbf{u} \times \mathbf{0} = \mathbf{0} \times \mathbf{u} = \mathbf{0}$
                    \item Cross product of a vector with itself: $\mathbf{v} \times \mathbf{v} = \mathbf{0}$
                    \item Scalar triple product: $\mathbf{u} \cdot (\mathbf{v} \times \mathbf{w}) = (\mathbf{u} \times \mathbf{v}) \cdot \mathbf{w}$
                \end{enumerate}
                \textbf{Note:} (AC)D(MC)(ZV)(IT)(TP)
            \item \textbf{Magnitude of the Cross Product}:
                Let $\mathbf{u}$ and $\mathbf{v}$ be vectors, and let $\theta$ be the angle between them. Then, $\|\mathbf{u} \times \mathbf{v}\| = \|\mathbf{u}\| \cdot \|\mathbf{v}\| \cdot \sin \theta.$
            \item \textbf{Applications of the cross product}:
            \begin{itemize}
                \item Finding a vector orthogonal to two given vectors
                \item Computing areas of triangles and parallelograms
                \item Determining the volume of the three-dimensional geometric shape made of parallelograms known as a parallelepiped:w

            \end{itemize}
        \item \textbf{Area of a Parallelogram}
            If we locate vectors $\mathbf{u}$ and $\mathbf{v}$ such that they form adjacent sides of a parallelogram, then the area of the parallelogram is given by $\|\mathbf{u} \times \mathbf{v}\|$.
        \item \textbf{Triple Scalar Product}:

            The triple scalar product of vectors $\mathbf{u}$, $\mathbf{v}$, and $\mathbf{w}$ is $\mathbf{u} \cdot (\mathbf{v} \times \mathbf{w})$.
            \bigbreak \noindent 
            The triple scalar product is the determinant of the  $3\times 3$ matrix formed by the components of the vectors
        \item \textbf{triple scalar product identities}: 
            \begin{enumerate}[label=(\alph*)]
                \item $\mathbf{u} \cdot (\mathbf{v} \times \mathbf{w}) = -\mathbf{u} \cdot (\mathbf{w} \times \mathbf{v})$
                \item $\mathbf{u} \cdot (\mathbf{v} \times \mathbf{w}) = \mathbf{v} \cdot (\mathbf{w} \times \mathbf{u} = \mathbf{w} \cdot (\mathbf{u} \times \mathbf{v}))$
            \end{enumerate}
            \pagebreak 
        \item \textbf{parallelepiped}: Let $\mathbf{u}$ and $\mathbf{v}$ be two vectors in standard position. If $\mathbf{u}$ and $\mathbf{v}$ are not scalar multiples of each other, then these vectors form adjacent sides of a parallelogram.
            \bigbreak \noindent 
            Now suppose we add a third vector  $\mathbf{w}$ that does not lie in the same plane as  $\mathbf{u}$ and  $\mathbf{v}$ but still shares the same initial point. Then these vectors form three edges of a parallelepiped, a three-dimensional prism with six faces that are each parallelograms,
        \item \textbf{Volume of a Parallelepiped}:
            The volume of a parallelepiped with adjacent edges given by the vectors  $\mathbf{u}$, $\mathbf{u}$, and $\mathbf{w}$ is the absolute value of the triple scalar product:
            \begin{align*}
                V = \bigg\lvert \mathbf{u} \cdot (\mathbf{v} \times \mathbf{w}) \bigg\rvert
            .\end{align*}
        \item \textbf{Torque}:
            Measures the tendency of a force to produce rotation about an axis of rotation. Let $\mathbf{r}$ be a vector with an initial point located on the axis of rotation and with a terminal point located at the point where the force is applied, and let vector $\mathbf{F}$ represent the force. Then torque is equal to the cross product of $\mathbf{r}$ and $\mathbf{F}$:
            $$
            \tau = \mathbf{r} \times \mathbf{F}.
            $$
        \item \textbf{Choosing $\alpha$ to make parallel vectors equal}: Suppose we have two vectors $\mathbf{u}$ and $\mathbf{v}$, and $\exists\ \alpha \in \mathbb{R}$ s.t $\alpha \mathbf{v} = \mathbf{u}$. Then 
            \begin{align*}
                \alpha = \frac{\norm{\mathbf{u}}}{\norm{\mathbf{v}}}
            .\end{align*}
            Or if they are \textbf{anti-parallel}
            \begin{align*}
                \alpha = -\frac{\norm{\mathbf{u}}}{\norm{\mathbf{v}}}
            .\end{align*}
        \item \textbf{The zero vector is considered to be parallel to all vectors}
        \item \textbf{vector equation of a line}
            \begin{align*}
                \mathbf{r} = \mathbf{r}_{0} + t\mathbf{v} 
            .\end{align*}
            Where $\mathbf{v}$ is the direction vector (vector parallel to the line), $t$ is some scalar, and $\mathbf{r}$, $\mathbf{r}_{0}$ are position vectors
        \item \textbf{Parametric and Symmetric Equations of a Line}:
            A line $L$ parallel to vector $\mathbf{v}=\langle a,b,c \rangle$ and passing through point $P(x_0,y_0,z_0)$ can be described by the following parametric equations:
            \[
                x=x_0+ta, \quad y=y_0+tb, \quad \text{and} \quad z=z_0+tc.
            \]
            If the constants $a$, $b$, and $c$ are all nonzero, then $L$ can be described by the symmetric equation of the line:
            \[
                \frac{x-x_0}{a} = \frac{y-y_0}{b} = \frac{z-z_0}{c}.
            \]
            \bigbreak \noindent 
            \textbf{Note:} The parametric equations of a line are not unique. Using a different parallel vector or a different point on the line leads to a different, equivalent representation. Each set of parametric equations leads to a related set of symmetric equations, so it follows that a symmetric equation of a line is not unique either.
        \item \textbf{Vector equation of a line reworked}: Suppose we have some line, with points $P(x_{0}, y_{0}, z_{0})$, $Q(x_{1}, y_{1}, z_{1})$. Where $\mathbf{p} = \left\langle x_{0}, y_{0}, z_{0}\right\rangle $ and $\mathbf{Q} = \left\langle x_{1}, y_{1}, z_{1}\right\rangle $ are the correponding position vectors. Suppose we also have $\mathbf{r} = \left\langle x,y,z \right\rangle $. Then our vector equation for a line becomes 
            \begin{align*}
                \mathbf{r} = \mathbf{p} + t\left(\vec{PQ}\right)
            .\end{align*} 
            By properties of vectors, we get the vector equation of a line passing through points $P$ and $Q$ to be 
            \begin{align*}
                \mathbf{r} = (1-t)\mathbf{p} + t\mathbf{q}
            .\end{align*}
        \item \textbf{Equation of a line segment between two points $P$ and $Q$}: Using the result from the previous item, we find the vector equation of the line segment between  $P$ and $Q$ is
            \begin{align*}
                \mathbf{r} = (1-t)\mathbf{p} + t\mathbf{q}, \quad 0 \leq t \leq 1
            .\end{align*}
            \bigbreak \noindent 
            Because when $t=0$, $\mathbf{r} = \mathbf{p}$. When $t=1$, $\mathbf{r}=\mathbf{q}$
        \item \textbf{parametric equations for this line segment}
               \begin{equation}
                    \begin{cases}
                        x =  x_{0}+ t(x_{1} - x_{0})\\
                        y =  y_{0}+ t(y_{1} - y_{0})\\
                        z =  z_{0}+ t(z_{1} - z_{0})\\
                    \end{cases}
                \end{equation}
            For $0 \leq t \leq 1 $
        \item \textbf{Distance from a Point to a Line}:
            Let $L$ be a line in space passing through point $P$ with direction vector $\mathbf{v}$. If $M$ is any point not on $L$, then the distance from $M$ to $L$ is
            \[
                d = \frac{\left\| \overrightarrow{PM} \times \mathbf{v} \right\|}{\left\| \mathbf{v} \right\|}
            \]
        \item If two lines in space are not parallel, but do not intersect, then the lines are said to be \textbf{skew} lines
        \item \textbf{Classifying lines in space}
            \bigbreak \noindent 
            \fig{.8}{./figures/19.jpeg}
        \item \textbf{Testing for parallel, intersecting, or skew}: Suppose we have two sets of parametric equations for two lines.
            \begin{itemize}
                \item First we test for parallel lines, We find their direction vectors, if their direction vectors are parallel, the lines are parallel.
                \item If they are not parallel, we can test for intersection. This is best explained with an example
            \end{itemize}
            Suppose we have the equations 
            \begin{align*}
                &L_{1}:\ \quad x= 3+2t \quad y= 4-t \quad z = 1+3t \\
                &L_{2}:\ \quad x= 1+4s \quad y= 3-2s \quad z=4+5s
            .\end{align*}
            First, we equate each set of equations 
            \begin{align*}
                3+2t &= 1+4s \\
                4-t&=3-2s \\
                1+3t&=4+5s
            .\end{align*}
            \bigbreak \noindent 
            Next, we solve the first equation for one of the variables, in this case we choose to solve for $t$
            \begin{align*}
                t = 2s-1
            .\end{align*}
            Now, we want to plug this into the second equation. This gives
            \begin{align*}
                4-(2s-1) &= 3-2s \\
                         2&=0
            .\end{align*}
            Since we this statement is not true, we know the lines must not be intersecting. In this case, we know they must be skew
            \bigbreak \noindent 
            However, suppose we got some value for $s$, such as $s=1$, we can then plug this value into the equation we got for $t$ ($t=2s-1$) to get a value for $t$, after we have a value for $t$, we can plug both $s$ and $t$ into the third equation and evaluate the truthiness of its outcome. If the outcome is true, we have a valid solution for the system of equations.
            \bigbreak \noindent 
            \textbf{Note:} If the lines are known to be skew, then we compute the dot product of their direction angles to check for orthogonality.
            \bigbreak \noindent 
        \item given any three points that do not all lie on the same line, there is a unique \textbf{plane} that passes through these points. Just as a line is determined by two points, \textbf{a plane is determined by three.}
        \item given two distinct, intersecting lines, \textbf{there is exactly one plane containing both lines.}
        \item A plane is also \textbf{determined by a line and any point that does not lie on the line.}
        \item We say that  $\mathbf{n}$ is a normal vector, or perpendicular to the plane.
        \item \textbf{Vector equation of a plane}:
            Given a point $P$ and vector $\mathbf{n}$, the set of all points $Q$ satisfying the equation $\mathbf{n} \cdot \overrightarrow{PQ} = 0$ forms a plane. The equation
            \[
                \mathbf{n} \cdot \overrightarrow{PQ} = 0
            \]
            is known as the vector equation of a plane.

    \item \textbf{Scalar equation of a plane}:
        The scalar equation of a plane containing point $P=(x_0, y_0, z_0)$ with normal vector $\mathbf{n}=\langle a, b, c \rangle$ is
        \[
            a(x-x_0) + b(y-y_0) + c(z-z_0) = 0.
        \]
    \item \textbf{General form of the equation of a plane}:
        This equation (the one above) can be expressed as $ax + by + cz + d = 0$, where $d = -ax_0 - by_0 - cz_0$. This form of the equation is sometimes called the general form of the equation of a plane.
    \item any three points that do not all lie on the same line determine a plane. Given three such points, we can find an equation for the plane containing these points.
    \item \textbf{The Distance between a Plane and a Point}: 
        Suppose a plane with normal vector  $\mathbf{n}$ passes through point  $Q$. The distance  $d$ from the plane to a point  $P$ not in the plane is given by
        \begin{align*}
             d = \norm{\text{proj}_{\mathbf{n}}\vec{QP}} = \bigg\lvert \text{comp}_{\mathbf{n}}\vec{QP} \bigg\rvert = \frac{\bigg\lvert \vec{QP} \cdot \mathbf{n} \bigg\rvert}{\norm{\mathbf{n}}}
        .\end{align*}
    \item For two planes in space, we have only two possible relationships: \textbf{The two distinct planes are parallel or they intersect. When two planes are parallel, their normal vectors are parallel. When two planes intersect, the intersection is a line}
    \item \textbf{Finding line of intersection for two planes}: Suppose we are given the parametric equations for two planes
        \begin{itemize}
            \item Check if their normals are parallel, if not, we know the planes must intersect
            \item Find a common point that satisfies both equations (perhaps the origin)
            \item Eliminate one of the variables (perhaps by adding them)
            \item After elimination we should have one variable equal to one of the others, we then plug this into one of the equations to get two variables in terms of the third.
            \item Define this third variable in terms of $t$ to get $x$, $y$, $z$ in terms of $t$, these equations will be the paramteric equations for the line of intersection
        \end{itemize}
    \item \textbf{find the angle formed by the intersection of two planes:} We can use normal vectors to calculate the angle between the two planes.
        \bigbreak \noindent 
        We can find the measure of the angle $\theta$ between two intersecting planes by first finding the cosine of the angle, using the following equation:
        \begin{align*}
            \cos{\theta }= \frac{\bigg\lvert \mathbf{n}_{1} \cdot \mathbf{n}_{2} \bigg\rvert}{\norm{\mathbf{n}_{1}}\norm{\mathbf{n}_{2}}}
        .\end{align*}
        \textbf{We can then use the angle to determine whether two planes are parallel or orthogonal or if they intersect at some other angle.}
        \pagebreak 
    \item \textbf{Finding the distance between two parallel planes}: 
        Let $P(x_0, y_0, z_0)$ be a point. The distance from $P$ to plane $ax + by + cz + k = 0$ is given by
        \[
            d = \frac{\left| ax_0 + by_0 + cz_0 + k \right|}{\sqrt{a^2 + b^2 + c^2}}.
        \]
    \item \textbf{Find the intersection of a plane with a line}:
        \begin{itemize}
            \item First, we get the line in parametric form, with $t$ is the parameter.
            \item When then plug our parametric equations into the corresponding plane equation variables
            \item Solve for $t$,
            \item Plug $t$ into parametric equation to get point $(x,y,z)$
        \end{itemize}

    \end{itemize}
    \pagebreak 
    \subsection{Conic sections and quadric surfaces}
    \bigbreak \noindent 
    \subsubsection{Conic sections: Parabola, Ellipse, and Hyperbola}
    \begin{itemize}
    \item \textbf{Conic sections, The parabola}:
      A parabola is the set of all points in a plane equidistant from a fixed point $F$ (the focus) and a fixed line $\ell$ (the directrix) that lie in the plane.
    \item \textbf{Parabola that opens up}:
        Given a parabola opening upward with vertex located at  $(h,k)$ and focus located at  $(h,k+p)$, where $p$ is a constant, the equation for the parabola is given by
        \begin{align*}
            &y= \frac{1}{4p}(x-h)^{2} + k \\
              &\text{Vertex: } (h,k) \\
              &\text{Focus: } (h,k+p) \\
              &\text{Directrix } y = k-p \\
              &\text{AOS } x=h
        .\end{align*}
        This is the \textbf{standard form} of a parabola.
        \bigbreak \noindent 
        \fig{.5}{./figures/s7.png}
    \item \textbf{Parabola that opens down}:
        \begin{align*}
            &y = -\frac{1}{4p}(x-h)^{2} + k \\
              &\text{Vertex: } (h,k) \\
              &\text{Focus: } (h,k-p) \\
              &\text{Directrix } y = k+p \\
              &\text{AOS } x=h
        .\end{align*}
        \bigbreak \noindent 
        \fig{.5}{./figures/s8.png}
    \item \textbf{Parabola opens right}:
        \begin{align*}
            &x= \frac{1}{4p}(y-k)^{2} + h \\
            &\text{Vertex: } (h,k) \\
            &\text{Focus: } (h+p,k) \\
            &\text{Directrix } x=h-p \\
            &\text{AOS } y=k
        .\end{align*}
        \bigbreak \noindent 
        \fig{.5}{./figures/s10.png}
    \item \textbf{Parabola opens left}:
        \begin{align*}
            &x= -\frac{1}{4p}(y-k)^{2} + h \\
            &\text{Vertex: } (h,k) \\
            &\text{Focus: } (h-p,k) \\
            &\text{Directrix } x=h+p \\
            &\text{AOS } y=k
        .\end{align*}
        \bigbreak \noindent 
        \fig{.5}{./figures/s11.png}
    \item \textbf{Genaral form of a parabola}:
        The equation of a parabola can be written in the general form, though in this form the values of $h$, $k$, and $p$ are not immediately recognizable. The general form of a parabola is written as
        \begin{align*}
             ax^2 + bx + cy + d = 0 \quad \text{or} \quad ay^2 + bx + cy + d = 0
        .\end{align*}
        \textbf{Note:} The first equation represents a parabola that opens either up or down. The second equation represents a parabola that opens either to the left or to the right. To put the equation into standard form, use the method of completing the square.
    \item \textbf{Finding $h$ and $k$ for a parabola}. $h$ and $k$ for a parabola are given by
        \begin{align*}
            h &= -\frac{2b}{a} \\
            k &= f(h)
        .\end{align*}
    \item \textbf{Conic sections, The ellipse}:
        An ellipse is the set of all points for which the sum of their distances from two fixed points (the foci) is constant.
        \bigbreak \noindent 
        \fig{.6}{./figures/ell.jpeg}
        \bigbreak \noindent 
        \textbf{Foci:} There are two foci
        \smallbreak \noindent
        \textbf{Directrices}: There are two directrices (plural of directrix)
        \smallbreak \noindent
        \textbf{Major axis}: An ellipse has two axis, one short and one long. The major axis is the longer of the two and has length $2a$
        \smallbreak \noindent
        \textbf{Minor axis:} The shorter axis has length $2b$
        \smallbreak \noindent
        \textbf{Finding $c$ with $a$ and $b$}, to find $c$, we use
        \begin{align*}
            c^{2} =  a^{2} - b^{2}
        .\end{align*}
    \item \textbf{Ellipse with horizontal major axis}: This type of ellipse has the form
        \begin{align*}
            \frac{(x-h)^{2}}{a^{2}} + \frac{(y-k)^{2}}{b^{2}} = 1 \\
            \text{Center } (h,k) \\
            \text{Foci: } (h\pm c, k) \\
            \text{Directrices: } x=h\pm \frac{a^{2}}{c}
        .\end{align*}
    \item \textbf{Ellipse with vertical major axis:} This type of ellipse has the form
        \begin{align*}
             \frac{(x-h)^{2}}{b^{2}} + \frac{(y-k)^{2}}{a^{2}} = 1 \\
            \text{Center: } (h,k) \\
            \text{Foci: } (h, k\pm c) \\
            \text{Directrices: } y=k\pm \frac{a^{2}}{c}
        .\end{align*}
    \item \textbf{General form of an ellipse}: The equation of an ellipse is in general form if it is in the form
        \begin{align*}
            Ax^{2} +By^{2}   +Cx +Dy  +E = 0
        .\end{align*}
        where $A$ and $B$ are either both positive or both negative. To convert the equation from general to standard form, use the method of completing the square.
    \item \textbf{Conic sections, the Hyperbola}: A hyperbola is the set of all points where the difference between their distances from two fixed points (the foci) is constant.
        \bigbreak \noindent 
        \fig{.6}{./figures/hyper.jpeg}
        \bigbreak \noindent 
        \textbf{Foci:} There are two foci
        \smallbreak \noindent
        \textbf{Directrices}: There are two directrices (plural of directrix)
        \smallbreak \noindent
        \textbf{Asymptotes}:  There are two Asymptotes
        \smallbreak \noindent
        \textbf{Tranverse axis (major axis)} has length $2a$
        \smallbreak \noindent
        \textbf{Conjugate axis (minor axis)} has length $2b$
        \smallbreak \noindent
        \textbf{Finding $c$ with $a$ and $b$}, to find $c$, we use
        \begin{align*}
            c^{2} = a^{2} + b^{2}
        .\end{align*}
    \item \textbf{Hyperbola opening left and right (horizontal major axis)}
        \begin{align*}
            &\frac{(x-h)^{2}}{a^{2}} - \frac{(y-k)^{2}}{b^{2}} = 1 \\
            &\text{Center: } (h,k) \\
            &\text{Foci: } (h\pm c, k) \\
            &\text{Asymptotes: } y= k\pm \frac{b}{a}(x-h) \\
            &\text{Directrices: } x=h\pm \frac{a^{2}}{c}
        .\end{align*}
    \item \textbf{Hyperbola opening up and down (vertical major axis)}
        \begin{align*}
            &\frac{(y-k)^{2}}{a^{2}} - \frac{(x-h)}{b^{2}} = 1 \\
            &\text{Center: } (h,k) \\
            &\text{Foci: } (h, k\pm c) \\
            &\text{Asymptotes: } y= k\pm \frac{a}{b}(x-h) \\
            &\text{Directrices: } y=k\pm \frac{a^{2}}{c}
        .\end{align*}
    \item \textbf{Hyperbola general form}: The equation of a hyperbola is in general form if it is in the form  
        \begin{align*}
            Ax^{2} + By^{2} + Cx + Dy + E = 0
        .\end{align*}
        \bigbreak \noindent 
      where $A$ and $B$ have opposite signs. In order to convert the equation from general to standard form, use the method of completing the square.
    \item \textbf{Eccentricity and directrix of conic sections}:
    The eccentricity $e$ of a conic section is defined to be the distance from any point on the conic section to its focus, divided by the perpendicular distance from that point to the nearest directrix. This value is constant for any conic section, and can define the conic section as well:
    \begin{itemize}
        \item If $e=1$, the conic is a parabola.
        \item If $e<1$, it is an ellipse.
        \item If $e>1$, it is a hyperbola.
    \end{itemize}
    The eccentricity of a circle is zero. The directrix of a conic section is the line that, together with the point known as the focus, serves to define a conic section. Hyperbolas and noncircular ellipses have two foci and two associated directrices. Parabolas have one focus and one directrix.
    \bigbreak \noindent 
    \fig{.6}{./figures/ecc.jpeg}
    \item \textbf{Eccentricity for parabola}
        \begin{align*}
            e=1
        .\end{align*}
    \item \textbf{Eccentricity for ellipses}
        \begin{align*}
            e = \frac{c}{a}
        .\end{align*}
    \item \textbf{Eccentricity for hyperbolas}
        \begin{align*}
            e = \frac{c}{a}
        .\end{align*}

    \end{itemize}
    \bigbreak \noindent 
    \subsubsection{Quadric Surfaces}
    \begin{itemize}
    \item A set of lines parallel to a given line passing through a given curve is known as a \textbf{cylindrical surface}, or cylinder. The parallel lines are called \textbf{rulings}.
    \item The \textbf{traces} of a surface are the cross-sections created when the surface intersects a plane parallel to one of the coordinate planes.
    \item We have learned about surfaces in three dimensions described by first-order equations; these are planes. Some other common types of surfaces can be described by second-order equations. We can view these surfaces as three-dimensional extensions of the conic sections we discussed earlier: the ellipse, the parabola, and the hyperbola. We call these graphs \textbf{quadric surfaces}.
    \item \textbf{Quadric surfaces} are the graphs of equations that can be expressed in the form
        \begin{align*}
            Ax^2 + By^2 + Cz^2 + Dxy + Exz + Fyz + Gx + Hy + Jz + K = 0.
        .\end{align*}
        \textbf{Note:} When a quadric surface intersects a coordinate plane, the trace is a conic section.
    \item An \textbf{ellipsoid} is a surface described by an equation of the form
        \begin{align*}
            \frac{x^{2}}{a^{2}} + \frac{y^{2}}{b^{2}} + \frac{z^{2}}{c^{2}} = 1
        .\end{align*}
        \textbf{Note:} Set $x=0$ to see the trace of the ellipsoid in the $yz$-plane. To see the traces in the $xy$- and $xz$-planes, set $z=0$ and $y=0$, respectively. Notice that, if $a=b$, the trace in the $xy$-plane is a circle. Similarly, if $a=c$, the trace in the $xz$-plane is a circle and, if $b=c$, then the trace in the $yz$-plane is a circle. A sphere, then, is an ellipsoid with $a=b=c$.
        \bigbreak \noindent 
        \fig{.5}{./figures/s1.png}
    \item A \textbf{Sphere} is an ellipsoid with
        \begin{align*}
            a=b=c
        .\end{align*}
        \textbf{Traces:} All three traces are ellipses
    \item A quadric surface can have at most \textbf{two unique traces}, these traces make the name of the surface.
    \item \textbf{Hyperboloid of one sheet}
        \begin{align*}
            \frac{x^{2}}{a^{2}} + \frac{y^{2}}{b^{2}} - \frac{z^{2}}{c^{2}} = 1
        .\end{align*}
        Two of the variables have positive coefficients and one has a negative coefficient. The axis of the surface corresponds to the variable with the negative coefficient
        \bigbreak \noindent 
        \fig{.5}{./figures/s2.png}
        \bigbreak \noindent 
        \textbf{Traces:} One ellipse and two hypebolas
    \item \textbf{Hyperboloid of two sheets}
        \begin{align*}
            \frac{z^{2}}{c^{2}} - \frac{x^{2}}{a^{2}} - \frac{y^{2}}{b^{2}} = 1
        .\end{align*}
        Two of the variables have negative coefficients and one has a positive coefficient. The axis of the surface corresponds to the variable with the positive coefficient. The surface does not intersect the coordinate plane perpendicular to the axis
        \bigbreak \noindent 
        \fig{.5}{./figures/s3.png}
        \bigbreak \noindent 
        \textbf{Traces:} One ellipse (or the empty set (no trace)), and two hypebolas
    \item  \textbf{Elliptic cone}
        \begin{align*}
            \frac{x^{2}}{a^{2}} + \frac{y^{2}}{b^{2}} -\frac{z^{2}}{c^{2}} = 0
        .\end{align*}
        The axis of the surface corresponds to the variable with a negative coefficient. The traces in the coordinate planes parallel to the axis are intersecting lines
        \bigbreak \noindent 
        \fig{.5}{./figures/s4.png}
        \bigbreak \noindent 
        \textbf{Traces:} One ellipse and two hyperbolas. 
        \smallbreak \noindent
        In the $xz$-plane, we have a pair of lines that intersect at the orgin
        \smallbreak \noindent
        In the $yz$-plane, we have a pair of lines that intersect at the orgin
    \item \textbf{Elliptic paraboloid}
        \begin{align*}
            \frac{x^{2}}{a^{2} } + \frac{y^{2}}{b^{2}} = z
        .\end{align*}
        The axis of the surface corresponds to the linear variable
        \bigbreak \noindent 
        \fig{.5}{./figures/s5.png}
        \bigbreak \noindent 
        \textbf{Traces:} One ellipse and two parabolas
    \item \textbf{Hyperbolic paraboloid}
        \begin{align*}
            \frac{x^{2}}{a^{2} } - \frac{y^{2}}{b^{2}} = z
        .\end{align*}
        The axis of the surface corresponds to the linear variable
        \bigbreak \noindent 
        \fig{.5}{./figures/s6.png}
        \bigbreak \noindent 
        \textbf{Traces:} One hypebola and two parabolas

    \end{itemize}

    \subsection{Chapter 3: Vector-Valued Functions}
    \bigbreak \noindent 
    \subsubsection{Definitions and Theorems}
    \begin{itemize}
        \item A \textbf{vector-valued function} is a function of the form
            \[
                \mathbf{r}(t) = f(t)\mathbf{i} + g(t)\mathbf{j} \quad \text{or} \quad \mathbf{r}(t) = f(t)\mathbf{i} + g(t)\mathbf{j} + h(t)\mathbf{k},
            \]
            where the \textbf{component functions} $f$, $g$, and $h$, are real-valued functions of the parameter $t$. Vector-valued functions are also written in the form
            \[
                \mathbf{r}(t) = \langle f(t), g(t) \rangle \quad \text{or} \quad \mathbf{r}(t) = \langle f(t), g(t), h(t) \rangle.
            \]
            In both cases, the first form of the function defines a two-dimensional vector-valued function; the second form describes a three-dimensional vector-valued function.
            \bigbreak \noindent 
            \textbf{Note:} The parameter $t$ can lie between two real numbers: $a \leq t \leq b$ Another possibility is that the value of $t$ might take on all real numbers. Last, the component functions themselves may have domain restrictions that enforce restrictions on the value of $t$. We often use $t$ as a parameter because $t$ can represent time.
        \item \textbf{Graphing vector-valued functions: Plane curve}:
            The graph of a vector-valued function of the form \( \mathbf{r}(t) = f(t)\mathbf{i} + g(t)\mathbf{j} \) consists of the set of all \( (t, \mathbf{r}(t)) \), and the path it traces is called a \textbf{plane curve.}
        \item \textbf{Graphing vector-valued functions: Space curve}:
            The graph of a vector-valued function of the form \( \mathbf{r}(t) = f(t)\mathbf{i} + g(t)\mathbf{j} + h(t)\mathbf{k} \) consists of the set of all \( (t, \mathbf{r}(t)) \), and the path it traces is called a \textbf{space curve}.
        \item \textbf{Vector Parameterization}: 
            Any representation of a plane curve or space curve using a vector-valued function is called a \textbf{vector parameterization} of the curve.
        \item \textbf{Note on graphing vector-valued functions:} When graphing a vector-valued function, we typically graph the vectors in the domain of the function in standard position, because doing so guarantees the uniqueness of the graph.
        \item \textbf{How to graph vector-valued functions}: As with any graph, we start with a table of values. We then graph each of the vectors in the second column of the table in standard position and connect the terminal points of each vector to form a curve
            \bigbreak \noindent 
            \fig{.7}{./figures/sinewavong.jpeg}
        \item \textbf{similarity between vector-valued functions and parameterized curves.}:
            Given a vector-valued function \( \mathbf{r}(t) = f(t)\mathbf{i} + g(t)\mathbf{j} \), we can define \( x = f(t) \) and \( y = g(t) \). If a restriction exists on the values of \( t \) (for example, \( t \) is restricted to the interval \([a, b]\) for some constants \( a < b \)), then this restriction is enforced on the parameter. The graph of the parameterized function would then agree with the graph of the vector-valued function, except that the vector-valued graph would represent vectors rather than points. Since we can parameterize a curve defined by a function \( y = f(x) \), it is also possible to represent an arbitrary plane curve by a vector-valued function.
        \item \textbf{Limits of a Vector-Valued Function (Rigorous)}:
            A vector-valued function \(\mathbf{r}\) approaches the limit \(L\) as \(t\) approaches \(a\), written
            \[
                \lim_{t \to a} \mathbf{r}(t) = L,
            \]
            provided
            \[
                \lim_{t \to a} \|\mathbf{r}(t) - L\| = 0.
            \]
            \textbf{Note:} This is a rigorous definition of the limit of a vector-valued function. In practice, we use the following theorem:
        \item \textbf{Limit of a Vector-Valued Function}:
            Let \(f\), \(g\), and \(h\) be functions of \(t\). Then the limit of the vector-valued function  \(\mathbf{r}(t) = f(t)\mathbf{i} + g(t)\mathbf{j}\) as \(t\) approaches \(a\) is given by
            \[
                \lim_{t \to a} \mathbf{r}(t) = \left[ \lim_{t \to a} f(t) \right]\mathbf{i} + \left[ \lim_{t \to a} g(t) \right]\mathbf{j},
            \]
            provided the limits  \(\lim_{t \to a} f(t)\) and \(\lim_{t \to a} g(t)\) exist. Similarly, the limit of the vector-valued function  \(\mathbf{r}(t) = f(t)\mathbf{i} + g(t)\mathbf{j} + h(t)\mathbf{k}\) as \(t\) approaches \(a\) is given by
            \[
                \lim_{t \to a} \mathbf{r}(t) = \left[ \lim_{t \to a} f(t) \right]\mathbf{i} + \left[ \lim_{t \to a} g(t) \right]\mathbf{j} + \left[ \lim_{t \to a} h(t) \right]\mathbf{k},
            \]
            provided the limits  \(\lim_{t \to a} f(t)\), \(\lim_{t \to a} g(t)\), and \(\lim_{t \to a} h(t)\) exist.
        \item \textbf{Continuity of a vector-valued function}:
            Let \(f\), \(g\), and \(h\) be functions of \(t\). Then, the vector-valued function \(\mathbf{r}(t) = f(t)\mathbf{i} + g(t)\mathbf{j}\) is continuous at point \(t = a\) if the following three conditions hold:
            \begin{enumerate}
                \item \(\mathbf{r}(a)\) exists.
                \item \(\lim_{t \to a} \mathbf{r}(t)\) exists.
                \item \(\lim_{t \to a} \mathbf{r}(t) = \mathbf{r}(a)\).
            \end{enumerate}
            \bigbreak \noindent 
            Similarly, the vector-valued function \(\mathbf{r}(t) = f(t)\mathbf{i} + g(t)\mathbf{j} + h(t)\mathbf{k}\) is continuous at point \(t = a\) if the following three conditions hold:
            \begin{enumerate}
                \item \(\mathbf{r}(a)\) exists.
                \item \(\lim_{t \to a} \mathbf{r}(t)\) exists.
                \item \(\lim_{t \to a} \mathbf{r}(t) = \mathbf{r}(a)\).
            \end{enumerate}
        \item \textbf{Example problem}: Find the vector equation that represents the curve of intersection of the cylinder $x^{2} + y^{2}  = 9 $ and the surface  $z=x+3y$
            \bigbreak \noindent 
            To start, we can easily find $x(t)$ and $y(t)$. 
            \bigbreak \noindent 
            \begin{prop}
               \begin{align*}
                   x(t) &= 3\cos{t} \\
                   y(t) &= 3\sin{t}
               .\end{align*} 
               We can verify this simply
               \begin{align*}
                   \frac{1}{3}x &= \cos{t} \\
                   \frac{1}{3}y&=\sin{t}
               .\end{align*}
               If $\cos^{2}{t} + \sin^{2}{t} = 1 $, Then
               \begin{align*}
                   \frac{1}{9}x^{2} + \frac{1}{9}y^{2} = 1 \\
                   x^{2} + y^{2} = 9
               .\end{align*}
           \end{prop}
           To find the function for $z(t)$, we see that our surface is already solved for $z$, thus we simply replace $x$ and $y$ with what we have for $x(t)$ and $y(t)$
           \begin{align*}
               \implies z(t) = 3\cos{t} + 9 \sin{t}
           .\end{align*}
       \item \textbf{The derivative of a vector-valued function $r(t)$}
           The derivative of a vector-valued function $\mathbf{r}(t)$ is
           \begin{equation}
               \mathbf{r}'(t) = \lim_{\Delta t \to 0} \frac{\mathbf{r}(t + \Delta t) - \mathbf{r}(t)}{\Delta t},
           \end{equation}
           provided the limit exists. If $\mathbf{r}'(t)$ exists, then $\mathbf{r}$ is differentiable at $t$. If $\mathbf{r}'(t)$ exists for all $t$ in an open interval $(a, b)$, then $\mathbf{r}$ is differentiable over the interval $(a, b)$. For the function to be differentiable over the closed interval $[a, b]$, the following two limits must exist as well:
           \begin{equation}
               \mathbf{r}'(a) = \lim_{\Delta t \to 0^+} \frac{\mathbf{r}(a + \Delta t) - \mathbf{r}(a)}{\Delta t}
           \end{equation}
           and
           \begin{equation}
               \mathbf{r}'(b) = \lim_{\Delta t \to 0^-} \frac{\mathbf{r}(b + \Delta t) - \mathbf{r}(b)}{\Delta t}.
           \end{equation}
       \item \textbf{Differentiation of Vector-Valued Functions}
           Let $f$, $g$, and $h$ be differentiable functions of $t$.
           \begin{itemize}
               \item If $\mathbf{r}(t) = f(t)\mathbf{i} + g(t)\mathbf{j}$, then $\mathbf{r}'(t) = f'(t)\mathbf{i} + g'(t)\mathbf{j}$.
               \item If $\mathbf{r}(t) = f(t)\mathbf{i} + g(t)\mathbf{j} + h(t)\mathbf{k}$, then $\mathbf{r}'(t) = f'(t)\mathbf{i} + g'(t)\mathbf{j} + h'(t)\mathbf{k}$.
           \end{itemize}
           \pagebreak 
       \item \textbf{Properties of the Derivative of Vector-Valued Functions}
           Let $\mathbf{r}$ and $\mathbf{u}$ be differentiable vector-valued functions of $t$, let $f$ be a differentiable real-valued function of $t$, and let $c$ be a scalar.
           \begin{enumerate}
               \item $\frac{d}{dt}[c\mathbf{r}(t)] = c\mathbf{r}'(t)$, Scalar multiple
               \item $\frac{d}{dt}[\mathbf{r}(t) \pm \mathbf{u}(t)] = \mathbf{r}'(t) \pm \mathbf{u}'(t)$, Sum and difference
               \item $\frac{d}{dt}[f(t)\mathbf{u}(t)] = f'(t)\mathbf{u}(t) + f(t)\mathbf{u}'(t)$, Scalar product
               \item $\frac{d}{dt}[\mathbf{r}(t) \cdot \mathbf{u}(t)] = \mathbf{r}'(t) \cdot \mathbf{u}(t) + \mathbf{r}(t) \cdot \mathbf{u}'(t)$, Dot product
               \item $\frac{d}{dt}[\mathbf{r}(t) \times \mathbf{u}(t)] = \mathbf{r}'(t) \times \mathbf{u}(t) + \mathbf{r}(t) \times \mathbf{u}'(t)$, Cross product
               \item $\frac{d}{dt}[\mathbf{r}(f(t))] = \mathbf{r}'(f(t)) \cdot f'(t)$, Chain rule
               \item If $\mathbf{r}(t) \cdot \mathbf{r}(t) = c$, then $\mathbf{r}(t) \cdot \mathbf{r}'(t) = 0$.
           \end{enumerate}
       \item \textbf{principal unit tangent vector}
           Let $C$ be a curve defined by a vector-valued function $\mathbf{r}$, and assume that $\mathbf{r}'(t)$ exists when $t = t_0$. A tangent vector $\mathbf{v}$ at $t = t_0$ is any vector such that, when the tail of the vector is placed at point $\mathbf{r}(t_0)$ on the graph, vector $\mathbf{v}$ is tangent to curve $C$. Vector $\mathbf{r}'(t_0)$ is an example of a tangent vector at point $t = t_0$. Furthermore, assume that $\mathbf{r}'(t) \neq 0$. The principal unit tangent vector at $t$ is defined to be
           \begin{equation}
               \mathbf{T}(t) = \frac{\mathbf{r}'(t)}{\|\mathbf{r}'(t)\|},
           \end{equation}
           provided $\|\mathbf{r}'(t)\| \neq 0.$
        \item \textbf{the derivative provides a tangent vector to the curve represented by the function.} 
        \item \textbf{The Fundamental Theorem of Calculus applies to vector-valued functions as well.}
        \item \textbf{Integrals of Vector-Valued Functions}
            Let $f$, $g$, and $h$ be integrable real-valued functions over the closed interval $[a,b]$.
            The indefinite integral of a vector-valued function $\mathbf{r}(t) = f(t)\mathbf{i} + g(t)\mathbf{j} + h(t)\mathbf{k}$ is
            \begin{equation}
                \int [f(t)\mathbf{i} + g(t)\mathbf{j} + h(t)\mathbf{k}] \, dt = \left[ \int f(t) \, dt \right]\mathbf{i} + \left[ \int g(t) \, dt \right]\mathbf{j} + \left[ \int h(t) \, dt \right]\mathbf{k}.
            \end{equation}
            The definite integral of the vector-valued function is
            \begin{equation}
                \int_a^b [f(t)\mathbf{i} + g(t)\mathbf{j} + h(t)\mathbf{k}] \, dt = \left[ \int_a^b f(t) \, dt \right]\mathbf{i} + \left[ \int_a^b g(t) \, dt \right]\mathbf{j} + \left[ \int_a^b h(t) \, dt \right]\mathbf{k}.
            \end{equation}
        \item \textbf{Integration constant for the integral of a vector-valued function}
            \begin{align*}
                \int [f(t)\mathbf{i} + g(t)\mathbf{j}] \, dt &= \left[ \int f(t) \, dt \right]\mathbf{i} + \left[ \int g(t) \, dt \right]\mathbf{j} \\
                &= (F(t) + C_1)\mathbf{i} + (G(t) + C_2)\mathbf{j} \\
                &= F(t)\mathbf{i} + G(t)\mathbf{j} + C_1\mathbf{i} + C_2\mathbf{j} \\
                &= F(t)\mathbf{i} + G(t)\mathbf{j} + \mathbf{C}
            .\end{align*}
            where $\mathbf{C} = C_1\mathbf{i} + C_2\mathbf{j}$. Therefore, the integration constant becomes a constant vector.
            \pagebreak 
        \item \textbf{Arc Length for Vector Functions}
            \begin{enumerate}
                \item \textbf{Plane curve:} Given a smooth curve $C$ defined by the function $\mathbf{r}(t) = f(t)\mathbf{i} + g(t)\mathbf{j}$, where $t$ lies within the interval $[a,b]$, the arc length of $C$ over the interval is
                    \begin{equation}
                        s = \int_{a}^{b} \sqrt{[f'(t)]^2 + [g'(t)]^2} \, dt = \int_{a}^{b} \|\mathbf{r}'(t)\| \, dt.
                    \end{equation}

                \item \textbf{Space curve:} Given a smooth curve $C$ defined by the function $\mathbf{r}(t) = f(t)\mathbf{i} + g(t)\mathbf{j} + h(t)\mathbf{k}$, where $t$ lies within the interval $[a,b]$, the arc length of $C$ over the interval is
                    \begin{equation}
                        s = \int_{a}^{b} \sqrt{[f'(t)]^2 + [g'(t)]^2 + [h'(t)]^2} \, dt = \int_{a}^{b} \|\mathbf{r}'(t)\| \, dt.
                    \end{equation}
            \end{enumerate}
            \bigbreak \noindent 
            \textbf{Note:} Note that the formulas are defined for smooth curves: curves where the vector-valued function $r(t)$ is differentiable with a non-zero derivative. The smoothness condition guarantees that the curve has no cusps (or corners) that could make the formula problematic.
        \item \textbf{Arc-length function definition}: If a vector-valued function represents the position of a particle in space as a function of time, then the arc-length function measures how far that particle travels as a function of time.
        \item \textbf{Arc-length function}:
            Let $\mathbf{r}(t)$ describe a smooth curve for $t \geq a$. Then the arc-length function is given by
            \begin{equation}
                s(t) = \int_{a}^{t} \|\mathbf{r}'(u)\| \, du
            \end{equation}
            $\mathscr{F}$urthermore, $\frac{ds}{dt} = \|\mathbf{r}'(t)\| > 0$. If $\|\mathbf{r}'(t)\| = 1$ for all $t \geq a$, then the parameter $t$ represents the arc length from the starting point at $t = a$.
            \bigbreak \noindent 
            \textbf{Note:} If a vector-valued function represents the position of a particle in space as a function of time, then the arc-length function measures how far that particle travels as a function of time.
            \bigbreak \noindent 
            Since  $s(t)$ measures distance traveled as a function of time,  $s^{\prime}(t)$ measures the speed of the particle at any given time.
            \bigbreak \noindent 
        \item \textbf{Arc-length parametrization}
            \begin{enumerate}
                \item Find $s(t)$
                \item Solve $s(t)$ for $t$
                \item Plug expression for $t$ into $\vec{\mathbf{r}}(t)$ to get $\vec{\mathbf{r}}(s)$
            \end{enumerate}
            The vector-valued function is now written in terms of the parameter $s$. Since the variable $s$ represents the arc length, we call this an arc-length parameterization of the original function  $r(t)$. One advantage of finding the arc-length parameterization is that the distance traveled along the curve starting from  $s=0$ is now equal to the parameter $s$
            \bigbreak \noindent 
            \textbf{Eg:} Suppose we have the function $r(t) = 4\cos{t}\ \hat{\mathbf{i}} + 4\sin{t}\ \hat{\mathbf{j}}$ for $t \geq 0$. First, we find $s(t)$
            \begin{align*}
                r^{\prime}(u) &= -4\sin{(u)}\ \hat{\mathbf{i}} + 4\cos{(u)}\ \hat{\mathbf{k}} \\
                \implies s(t) &= \int_{0}^{t}\ \sqrt{16\sin^{2}{u} + 16\cos^{2}{u}}\ du \\
                &=\int_{0}^{t}\ 4\ du =4u \bigg|^{0}_{t} = 4t 
            .\end{align*}
            Now we solve for $t$
            \begin{align*}
                s = 4t \implies t=\frac{1}{4}s
            .\end{align*}
            Pluggin back into $\vec{\mathbf{r}}(t)$ we get 
            \begin{align*}
                \vec{\mathbf{r}}(s) = 4\cos{\left(\frac{1}{4}s\right)}\ \hat{\mathbf{i}} + 4\sin{\left(\frac{1}{4}s\right)}\ \hat{\mathbf{j}}
            .\end{align*}
            This is the arc-length parameterization of $\mathbf{r}(t)$. Since the original restriction on $t$ was given by $t \geq 0$, the restriction on $s$ becomes $\frac{s}{4} \geq 0$, or $s \geq 0$.
        \item \textbf{Curvature}: The concept of curvature provides a way to measure how sharply a smooth curve turns.
            Let $C$ be a smooth curve in the plane or in space given by $\mathbf{r}(s)$, where $s$ is the arc-length parameter. The curvature $\kappa$ at $s$ is
            \begin{equation}
                \kappa = \left\|\frac{d\mathbf{T}}{ds}\right\| = \|\mathbf{T}'(s)\|.
            \end{equation}
        \item \textbf{Alternate formulas for curvature}
            If $C$ is a smooth curve given by $\mathbf{r}(t)$, then the curvature $\kappa$ of $C$ at $t$ is given by
            \begin{equation}
                \kappa = \frac{\|\mathbf{T}'(t)\|}{\|\mathbf{r}'(t)\|}.
            \end{equation}
            If $C$ is a three-dimensional curve, then the curvature can be given by the formula
            \begin{equation}
                \kappa = \frac{\|\mathbf{r}'(t) \times \mathbf{r}''(t)\|}{\|\mathbf{r}'(t)\|^3}.
            \end{equation}
            If $C$ is the graph of a function $y=f(x)$ and both $y'$ and $y''$ exist, then the curvature $\kappa$ at point $(x,y)$ is given by
            \begin{equation}
                \kappa = \frac{|y''|}{[1+(y')^2]^{3/2}}.
            \end{equation}
        \item \textbf{The curvature of a circle} is given by 
            \begin{align*}
                \frac{1}{\text{radius}}
            .\end{align*}
        \item \textbf{Principal unit normal vector}
            Let $C$ be a three-dimensional smooth curve represented by $\mathbf{r}$ over an open interval $I$. If $\mathbf{T}'(t) \neq 0$, then the principal unit normal vector at $t$ is defined to be
            \begin{equation}
                \mathbf{N}(t) = \frac{\mathbf{T}'(t)}{\|\mathbf{T}'(t)\|}.
            \end{equation}
        \item \textbf{Principal unit binormal vector}
            The binormal vector at $t$ is defined as
            \begin{equation}
                \mathbf{B}(t) = \mathbf{T}(t) \times \mathbf{N}(t),
            \end{equation}
            where $\mathbf{T}(t)$ is the unit tangent vector.
            \bigbreak \noindent 
            \textbf{Note:} the binormal vector will already be of magnitude one
        \item \textbf{We can only find a binormal vector for a space curve, not a two dimensional curve (plane curve)}
        \item \textbf{Normal plane}: The unit normal vector and the binormal vector form a plane that is perpendicular to the curve at any point on the curve, called the normal plane
        \item These three vectors form a frame of reference in three-dimensional space called the \textbf{Frenet frame of reference (also called the TNB frame)}
        \item The plane determined by the vectors $\vec{\mathbf{T}}$ and $\vec{\mathbf{N}}$ forms the \textbf{osculating plane of $C$ at any point $P$ on the curve.}
    \end{itemize}

    \pagebreak 
    \subsection{Chapter 4: Differentiation of Functions of Several Variables}
    \bigbreak \noindent 
    \subsubsection{Definitions and Theorems}
    \begin{itemize}
        \item \textbf{Function of two variables}:
            A function of two variables $z=f(x,y)$ maps each ordered pair $(x,y)$ in a subset $D$ of the real plane $\mathbb{R}^2$ to a unique real number $z$. The set $D$ is called the domain of the function. The range of $f$ is the set of all real numbers $z$ that has at least one ordered pair $(x,y) \in D$ such that $f(x,y) = z$ as shown in the following figure.
        \item \textbf{Surface}: The graph of a function  $z=(x,y)$ of two variables is called a surface.
        \item Given a function  $f(x,y)$ and a number $c$ in the range of $f$, a \textbf{level curve of a function of two variables} for the value  $c$ is defined to be the set of points satisfying the equation  $f(x,y)=c$
        \item A graph of the various level curves of a function is called a \textbf{contour map.}
        \item Level curves are always graphed in the  $xy$-plane, but as their name implies, \textbf{vertical traces} are graphed in the  $xz$- or  $yz$-planes.
        \item Consider a function $z=f(x,y)$ with domain $D \subseteq \mathbb{R}^2$. A \textbf{vertical trace} of the function can be either the set of points that solves the equation $f(a,y)=z$ for a given constant $x=a$ or $f(x,b)=z$ for a given constant $y=b$.
        \item \textbf{Limit laws for functions of two variables}
            Let $f(x,y)$ and $g(x,y)$ be defined for all $(x,y) \neq (a,b)$ in a neighborhood around $(a,b)$, and assume the neighborhood is contained completely inside the domain of $f$. Assume that $L$ and $M$ are real numbers such that $\lim_{(x,y) \to (a,b)} f(x,y) = L$ and $\lim_{(x,y) \to (a,b)} g(x,y) = M$, and let $c$ be a constant. Then each of the following statements holds:
            \bigbreak \noindent 
            \textbf{Constant Law:}
            \[
                \lim_{(x,y) \to (a,b)} c = c \tag{4.2}
            \]
            \textbf{Identity Laws:}
            \[
                \lim_{(x,y) \to (a,b)} x = a \tag{4.3}
            \]
            \[
                \lim_{(x,y) \to (a,b)} y = b \tag{4.4}
            \]
            \textbf{Sum Law:}
            \[
                \lim_{(x,y) \to (a,b)} (f(x,y) + g(x,y)) = L + M \tag{4.5}
            \]
            \textbf{Difference Law:}
            \[
                \lim_{(x,y) \to (a,b)} (f(x,y) - g(x,y)) = L - M \tag{4.6}
            \]
            \textbf{Constant Multiple Law:}
            \[
                \lim_{(x,y) \to (a,b)} (cf(x,y)) = cL \tag{4.7}
            \]
            \textbf{Product Law:}
            \[
                \lim_{(x,y) \to (a,b)} (f(x,y)g(x,y)) = LM \tag{4.8}
            \]
            \textbf{Quotient Law:}
            \[
                \lim_{(x,y) \to (a,b)} \frac{f(x,y)}{g(x,y)} = \frac{L}{M} \quad \text{for } M \neq 0 \tag{4.9}
            \]
            \textbf{Power Law:}
            \[
                \lim_{(x,y) \to (a,b)} (f(x,y))^n = L^n \tag{4.10}
            \]
            for any positive integer $n$.
            \textbf{Root Law:}
            \[
                \lim_{(x,y) \to (a,b)} \sqrt[n]{f(x,y)} = \sqrt[n]{L} \tag{4.11}
            \]
            for all $L$ if $n$ is odd and positive, and for $L \geq 0$ if $n$ is even and positive provided that $f(x, y) \geq 0$ for all $(x, y) \neq (a, b)$ in neighborhood of $(a, b)$.
        \item \textbf{Method to show that a limit does not exist}. To show that a limit exists can be quite challenging. To show that the limit exists is to show that it is the same along each and every path (infinitely many). To show that it does not exist, however, we only need to show that the limit differs for at least two unique paths
            \bigbreak \noindent 
            \textbf{Example}: Show that the following limit does not exist
            \begin{align*}
                \lim\limits_{(x,y) \to (0,0)}{\frac{x^{2}-y^{2}}{2x^{2} + y^{2}}}
            .\end{align*}
                We see that evaluating this limit directly yields an undefined result. Thus, we shall show that the limit differs among paths taken. Suppose we imagine our point at (0,0), if we choose to go along the path $x=0$, our limit becomes
                \begin{align*}
                    \lim\limits_{(0,y) \to (0,0)}{\frac{-y^{2}}{y^{2}}} = -1
                .\end{align*}
                Likewise, lets go along the line $y=0$, we get
                \begin{align*}
                    \lim\limits_{(x,0) \to (0,0)}{\frac{x^{2}}{2x^{2}}} = \frac{1}{2}
                .\end{align*}
                Since $-1 \neq \frac{1}{2}$. We assert the limit does not exist at point (0,0) and hence move forward
            \item \textbf{ L'Hospital's Rule in limits of functions of two or more variables}. It is sadly the case that we cannot use L'Hospital's Rule for limits of two variables, we can however use it when showing the limits are different among different paths. If we choose our path such that one variable vanishs, then the function is now of one variable and we are free to use L'Hospital's Rule to evaluate it.
                \bigbreak \noindent 
                \textbf{Example}:
                \begin{align*}
                    \lim\limits_{(x,y) \to (0,0)}{\frac{3xy}{3x^{2} + y^{2}}}
                .\end{align*}
                If we choose our path of interest to be along $y=x$, we get 
                \begin{align*}
                    \lim\limits_{(x,x) \to (0,0)}{\frac{3x^{2}}{4x^{2}}} = \lim\limits_{x \to 0}{\frac{3x^{2}}{4x^{2}}}
                .\end{align*}
                We are now free to use L'Hospital's Rule, although it is clearly not necessary in this case.
            \item \textbf{Using polar coordinates to evaluate limits}
                \begin{align*}
                    \lim\limits_{(x,y) \to (0,0)}{\frac{xy}{\sqrt{x^{2}+y^{2}}}}
                .\end{align*}
                We swap $x$ and $y$ to polar form
                \begin{align*}
                    &\lim\limits_{r \to 0}{\frac{r\cos{\left(\theta\right)}r\sin{\left(\theta\right)}}{\sqrt{r^{2}\cos^{2}{\left(\theta\right)}+r^{2}\sin^{2}{\left(\theta \right)}}}} \\
                    &=\lim\limits_{r \to 0}{r\cos{\left(\theta\right)}\sin{\left(\theta\right)}} \\
                    &=0
                .\end{align*}
            \item \textbf{Squeeze theorem for multivariable limits}: Suppose we have
                \begin{align*}
                    \lim\limits_{(x,y) \to (0,0)}{ \frac{5x^{2}y}{x^{2}+y^{2}}   }
                .\end{align*}
                We want to consider the positive function, thus we examine
                \begin{align*}
                    &\lim\limits_{(x,y) \to (0,0)}{\bigg\lvert \frac{5x^{2}y}{x^{2}+y^{2}} \bigg\rvert} \\
                    &=\lim\limits_{(x,y) \to (0,0)}{\frac{5x^{2}\abs{y}}{x^{2}+y^{2}} }
                .\end{align*}
                \bigbreak \noindent 
                We remark that 
                \begin{align*}
                    \lim\limits_{(x,y) \to (a,b)}{\bigg\lvert f(x,y) \bigg\rvert} = \bigg\lvert \lim\limits_{(x,y) \to (a,b)}{f(x,y)} \bigg\rvert
                .\end{align*}
                \bigbreak \noindent 
                We notice
                \begin{align*}
                    0 \leq \frac{5x^{2}}{x^{2}+y^{2}} \leq 5
                .\end{align*}
                From here, we multiply all sides by $\abs{y}$, note that this is why we wanted to consider the absolute value of the function, so we can peacfully manipulate the inequality
                \begin{align*}
                    0 \leq \frac{5x^{2}\abs{y}}{x^{2}+y^{2}} \leq 5\abs{y}
                .\end{align*}
                Now, we take limits and notice the squeeze
                \begin{align*}
                    \lim\limits_{(x,y) \to (0,0)}{0} \leq \frac{5x^{2}\abs{y}}{x^{2}+y^{2}} \leq \lim\limits_{(x,y) \to (0,0)}{5\abs{y}}
                .\end{align*}
                \bigbreak \noindent 
                Thus, the limit of the middle function must be zero.
            \item \textbf{Limits of functions of three variables}. Similar to the functions of two variables, we can choose a path along an axis by having two of the variables be zero. Furthermore, we can go along a curve thats parametric. Thus, we set $x,y$ and $z$ to functions of a parameter $t$. This allows us to travel along a curve $C$. The best way to approach this is to choose the functinos of $t$ such that the degrees in both the numerator and denominator match up.
            \item \textbf{Interior Points and Boundary Points}:
                Let $S$ be a subset of $\mathbb{R}^2$ 
                \begin{enumerate}[label=(\alph*)]
                    \item A point $P_0$ is called an \textbf{interior point} of $S$ if there is a $\delta$-disk centered around $P_0$ contained completely in $S$.
                    \item A point $P_0$ is called a \textbf{boundary point} of $S$ if every $\delta$-disk centered around $P_0$ contains points both inside and outside $S$.
                    \item $S$ is called an \textbf{open set} if every point of S is an interior point. 
                    \item $S$ is called a \textbf{closed set} if it contains all its boundary points.
                    \item An open set  $S$ is a \textbf{connected set} if it cannot be represented as the union of two or more disjoint, nonempty open subsets 
                    \item A set $S$ is a \textbf{region} if it is open, connected, and nonempty.
                \end{enumerate}
                \bigbreak \noindent 
                \fig{.7}{./figures/boundary.jpeg}
            \item \textbf{Continuity of a function of two variables}:
                A function $f(x,y)$ is continuous at a point $(a,b)$ in its domain if the following conditions are satisfied:
                \begin{enumerate}
                    \item $f(a,b)$ exists.
                    \item $\lim_{(x,y) \to (a,b)} f(x,y)$ exists.
                    \item $\lim_{(x,y) \to (a,b)} f(x,y) = f(a,b)$.
                \end{enumerate}
            \item \textbf{The Sum of Continuous Functions Is Continuous}
            \item \textbf{The Product of Continuous Functions Is Continuous}
            \item \textbf{The Composition of Continuous Functions Is Continuous}
            \item \textbf{Sketching graphs for domain and continuity}
                \begin{itemize}
                    \item Identify domain restrictions
                    \item If domain restriction is an inequality, change inequality to equality and solve for $y$
                    \item Graph function of $y$, if inequality is strict, curve should be dotted. If non-strict, solid
                    \item Test points outside/inside (or above/below for line) with original inequality (from domain restriction), shade regions that yield true
                        \bigbreak \noindent 
                        \textbf{Example:} Graph the set of points of continuity for the following function:
                        \begin{align*}
                            \ln{(x+6y)}
                        .\end{align*}
                        We see 
                        \begin{align*}
                            x+6y > 0  \quad \text{(strict)}\\
                            \implies y = -\frac{1}{6}x
                        .\end{align*}
                        Graph:
                    \begin{figure}[ht]
                        \centering
                        \incfig{graphermane1233}
                        \label{fig:graphermane1233}
                    \end{figure}
                    \bigbreak \noindent 
                    The shaded regoin is found with test points $T_{1}(0,-3)$ and $T_{2}(0,3)$ (red points). We see
                    \begin{align*}
                        0 + 6(-3) \not&> 0 \\
                        0 + 6(3) &> 0 
                    .\end{align*}
                \end{itemize}
            \item \textbf{Constant of proportionality}
                \begin{itemize}
                    \item Direct proportionality
                        \begin{align*}
                            y = kx
                        .\end{align*}
                        Where $k$ is the constant of proportionality
                \end{itemize}
                The constant of proportionality can be determined if you know the values of the two variables at a specific point. For direct proportionality, if you know a pair of values $(x_{0}, y_{0}) $, you can find  $k$ by rearranging the formula:
                \begin{align*}
                    k = \frac{y}{x}
                .\end{align*}
                \item Inversely proportional
                \begin{align*}
                    y = \frac{k}{x}
                .\end{align*}
                For inverse proportionality, given a pair of values $(x_{0}, y_{0})$, $k$ can be found as 
                \begin{align*}
                    k=xy
                .\end{align*}
                \bigbreak \noindent 
                \textbf{We use $\propto$ notation without the constant, but when we replace it with equality, we need to introduce the constant to maintain equality}
            \item \textbf{Limit definition of a partial derivative}:
                Let $f(x,y)$ be a function of two variables. Then the partial derivative of $f$ with respect to $x$, written as $\frac{\partial f}{\partial x}$, or $f_x$, is defined as
                \begin{align*}
                    \frac{\partial f}{\partial x} = \lim_{h \to 0} \frac{f(x+h,y) - f(x,y)}{h}
                .\end{align*}
                The partial derivative of $f$ with respect to $y$, written as $\frac{\partial f}{\partial y}$, or $f_y$, is defined as
                \begin{align*}
                    \frac{\partial f}{\partial y} = \lim_{k \to 0} \frac{f(x,y+k) - f(x,y)}{k}
                .\end{align*}
                \bigbreak \noindent 
                \textbf{Note:} The logic remains the same for functions of three variables
            \item \textbf{Fast computation of partial derivatives}: To compute partial derivaties without using the limit definition, we let the variable that we arn't interested in be a constant, this allows us to compute the derivative as if it were ordinary.
            \item \textbf{Interpertation of partial derivatives}: The partial derivative in the x-direction is the slope for slight movement in the x-direction, etc
            \item \textbf{higher-order partial derivatives}:
                There are four second-order partial derivatives for any function (provided they all exist):
                \begin{align*}
                    \frac{\partial^2 f}{\partial x^2} &= \frac{\partial}{\partial x}\left[\frac{\partial f}{\partial x}\right], \\
                    \frac{\partial^2 f}{\partial x \partial y} &= \frac{\partial}{\partial x}\left[\frac{\partial f}{\partial y}\right], \\
                    \frac{\partial^2 f}{\partial y \partial x} &= \frac{\partial}{\partial y}\left[\frac{\partial f}{\partial x}\right], \\
                    \frac{\partial^2 f}{\partial y^2} &= \frac{\partial}{\partial y}\left[\frac{\partial f}{\partial y}\right].
                .\end{align*}
                \bigbreak \noindent 
                \textbf{Note:} in this notation, the order in which we take derivatives goes from right to left (of the denominator)
            \item \textbf{Higher-order partial derivatives alternate notation}:
                An alternative notation for each is $f_{xx}$, $f_{yx}$, $f_{xy}$, and $f_{yy}$, respectively. Higher-order partial derivatives calculated with respect to different variables, such as $f_{xy}$ and $f_{yx}$, are commonly called mixed partial derivatives.
                \bigbreak \noindent 
                \textbf{Note:} with this notation, the order is from left to right
            \item \textbf{Tangent plane:} Let $P_0=(x_0,y_0,z_0)$ be a point on a surface $S$, and let $C$ be any curve passing through $P_0$ and lying entirely in $S$. If the tangent lines to all such curves $C$ at $P_0$ lie in the same plane, then this plane is called the tangent plane to $S$ at $P_0$.
                \bigbreak \noindent 
                \fig{.5}{./figures/tplane.jpeg}
            \item \textbf{For a tangent plane to a surface to exist at a point on that surface, it is sufficient for the function that defines the surface to be differentiable at that point,}
            \item \textbf{Equation of a tangent plane}: Let \(S\) be a surface defined by a differentiable function \(z = f(x, y)\), and let \(P_0 = (x_0, y_0)\) be a point in the domain of \(f\). Then, the equation of the tangent plane to \(S\) at \(P_0\) is given by
                \begin{align*}
                    z = f(x_0, y_0) + f_x(x_0, y_0)(x - x_0) + f_y(x_0, y_0)(y - y_0)
                .\end{align*}
            \item \textbf{Linear approximation with tangent planes}: 
                Given a function \(z=f(x,y)\) with continuous partial derivatives that exist at the point \((x_0,y_0)\), the linear approximation of \(f\) at the point \((x_0,y_0)\) is given by the equation
                \[ L(x,y) = f(x_0,y_0) + f_x(x_0,y_0)(x-x_0) + f_y(x_0,y_0)(y-y_0) \]
            \item \textbf{a surface is considered to be smooth at point  $P$ if a tangent plane to the surface exists at that point.}
            \item \textbf{For a tangent plane to exist at the point  $(x_{0}, y_{0})$ the partial derivatives must therefore exist at that point. However, this is not a sufficient condition for smoothness,}
            \item \textbf{Differentiability (book definition)}: A function \(f(x,y)\) is differentiable at a point \(P(x_0,y_0)\) if, for all points \((x,y)\) in a \(\delta\) disk around \(P\), we can write
                \begin{equation}
                    f(x,y) = f(x_0,y_0) + f_x(x_0,y_0)(x-x_0) + f_y(x_0,y_0)(y-y_0) + E(x,y)
                \end{equation}
                where the error term \(E\) satisfies
                \[
                    \lim_{(x,y) \to (x_0,y_0)} \frac{E(x,y)}{\sqrt{(x-x_0)^2 + (y-y_0)^2}} = 0.
                \]
            \item \textbf{Differentiability (Professor's definition)}: The geometric concept of having a tangent plane at $(x_{0}, y_{0})$ is equivalent to the approximation $f(x,y) \approx L(x,y)$ being sufficiently good. When this happens, we call $f(x,y)$ differentiable at $(x_{0}, y_{0})$. Formally,
                \begin{align*}
                    \lim\limits_{(x,y) \to (x_{0}, y_{0})}{E(x,y)} = 0
                .\end{align*}
                Where 
                \begin{align*}
                    E(x,y) = \frac{f(x,y) - L(x,y)}{\bigg\lvert (x,y) - (x_{0}, y_{0}) \bigg\rvert}
                .\end{align*}

            \item \textbf{Differentiability critera}: $f(x,y)$ is differentiable at $(x_{0}, y_{0})$ iff 
                \begin{enumerate}
                    \item $f(x,y) = L(x) + E(x,y) \cdot \bigg\lvert (x,y) - (x_{0}, y_{0}) \bigg\rvert $
                    \item $\lim\limits_{(x,y) \to (x_{0}, y_{0})}{E(x,y)}  =0 $
                \end{enumerate}
                \bigbreak \noindent 
                \textbf{}
                Differentiability for functions of several variables (e.g., $f(x,y)$) extends the concept from single-variable calculus. A function  $f(x,y)$ is differentiable at a point $(x_{0}, y_{0})$ if it can be well approximated by a linear function (the tangent plane) near that point.
            \item \textbf{Show that a function is differentiable}: Show that 
                \begin{align*}
                    f(x,y) = 2x^{2} - 4y
                .\end{align*}
                Is differentiable at the point $(2,-3)$
                \bigbreak \noindent 
                First, we find $f(x,y)$, $f_{x}(x,y)$, and $f_{y}(x,y)$. We find
                \begin{align*}
                    f(2,-3) &= 20 \\
                    f_{x}(2,-3) &= 8 \\
                    f_{y}(2,-3) &= -4
                .\end{align*}
                \bigbreak \noindent 
                This implies
                \begin{align*}
                    f(x,y) &= L(x) + E(x,y) \\
                           &\implies E(x,y) = f(x,y) - L(x,y) \\
                           &\therefore E(x,y) = 2x^{2} -8x+8
                .\end{align*}
                \bigbreak \noindent 
            Now we need to show that $\lim\limits_{(x,y) \to (x_{0}, y_{0})}{\frac{E(x,y)}{\sqrt{(x-x_{0})^{2}+(y-y_{0})^{2}}}} =0$. Thus,
            \begin{align*}
            &\lim\limits_{(x,y) \to (2,-3)}{\frac{2x^{2}-8x+8}{\sqrt{(x-2)^{2}+(y+3)^{2}}}}  \\
            &=\lim\limits_{(x,y) \to (2,-3)}{\frac{2(x-2)^{2}}{\sqrt{(x-2)^{2}+(y+3)^{2}}}}  \\ 
            &\leq \lim\limits_{(x,y) \to (2,-3)}{\frac{2((x-2)^{2}+(y+3)^{2})}{\sqrt{(x-2)^{2}+(y+3)^{2}}}} \\
            &=\lim\limits_{(x,y) \to (2,-3)}{2\sqrt{(x-2)^{2} + (y+3)^{2}}} \\
            &=0
            .\end{align*}
            \bigbreak \noindent 
            Since $E(x,y) \geq 0$ for any value of $x$ or $y$, the original limit must be equal to zero. Therefore, $f(x,y) = 2x^{2}-4y$ is differentiable  at point $(2,-3) $
        \item \textbf{Differentiability Implies Continuity}:
        Let \(z = f(x,y)\) be a function of two variables with \((x_0,y_0)\) in the domain of \(f\). If \(f(x,y)\) is differentiable at \((x_0,y_0)\), then \(f(x,y)\) is continuous at \((x_0,y_0)\).
    \item \textbf{Continuity of First Partials Implies Differentiability}:
        Let \(z = f(x,y)\) be a function of two variables with \((x_0,y_0)\) in the domain of \(f\). If \(f(x,y)\), \(f_x(x,y)\), and \(f_y(x,y)\) all exist in a neighborhood of \((x_0,y_0)\) and are continuous at \((x_0,y_0)\), then \(f(x,y)\) is differentiable there.
    \item \textbf{Total differential}:
        Let \(z = f(x,y)\) be a function of two variables with \((x_0,y_0)\) in the domain of \(f\), and let \(\Delta x\) and \(\Delta y\) be chosen so that \((x_0 + \Delta x, y_0 + \Delta y)\) is also in the domain of \(f\). If \(f\) is differentiable at the point \((x_0,y_0)\), then the differentials \(dx\) and \(dy\) are defined as
        \[dx = \Delta x \quad \text{and} \quad dy = \Delta y.\]
        The differential \(dz\), also called the total differential of \(z = f(x,y)\) at \((x_0,y_0)\), is defined as
        \[dz = f_x(x_0,y_0)dx + f_y(x_0,y_0)dy.\]
    \item \textbf{\Delta z}
        \begin{align*}
            \Delta z = f(x+\Delta x, y+\Delta y) - f(x,y)
        .\end{align*}
        \bigbreak \noindent 
        We use  $dz$ to approximate  $\Delta z$, so
        \begin{align*}
            \Delta z \approx dz = f_{x}(x_{0},y_{0})dx + f_{y}(x_{0},y_{0})dy
        .\end{align*}
        With can further approimate with
        \begin{align*}
            f(x + \Delta x, y+\Delta y)  &= f(x,y) + \Delta z \\
            &\approx f(x,y) +f_{x}(x_{0},y_{0})\Delta x + f_{y}(x_{0}, y_{})\Delta y
        .\end{align*}
    \item \textbf{Chain Rule for One Independent Variable}:
        Suppose that \(x = g(t)\) and \(y = h(t)\) are differentiable functions of \(t\) and \(z = f(x,y)\) is a differentiable function of \(x\) and \(y\). Then \(z = f(x(t), y(t))\) is a differentiable function of \(t\) and
        \[
            \frac{dz}{dt} = \frac{\partial z}{\partial x} \cdot \frac{dx}{dt} + \frac{\partial z}{\partial y} \cdot \frac{dy}{dt},
        \]
        where the ordinary derivatives are evaluated at \(t\) and the partial derivatives are evaluated at \((x,y)\).
    \item \textbf{Chain Rule for Two Independent Variables}:
        Suppose \(x = g(u,v)\) and \(y = h(u,v)\) are differentiable functions of \(u\) and \(v\), and \(z = f(x,y)\) is a differentiable function of \(x\) and \(y\). Then, \(z = f(g(u,v),h(u,v))\) is a differentiable function of \(u\) and \(v\), and
        \[
            \frac{\partial z}{\partial u} = \frac{\partial z}{\partial x} \frac{\partial x}{\partial u} + \frac{\partial z}{\partial y} \frac{\partial y}{\partial u}
        \]
        and
        \[
            \frac{\partial z}{\partial v} = \frac{\partial z}{\partial x} \frac{\partial x}{\partial v} + \frac{\partial z}{\partial y} \frac{\partial y}{\partial v}.
        \]
    \item \textbf{Generalized Chain Rule}:
        Let \(w = f(x_1, x_2, \ldots, x_m)\) be a differentiable function of \(m\) independent variables, and for each \(i \in \{1, \ldots, m\}\), let \(x_i = x_i(t_1, t_2, \ldots, t_n)\) be a differentiable function of \(n\) independent variables. Then
        \[
            \frac{\partial w}{\partial t_j} = \frac{\partial w}{\partial x_1} \frac{\partial x_1}{\partial t_j} + \frac{\partial w}{\partial x_2} \frac{\partial x_2}{\partial t_j} + \cdots + \frac{\partial w}{\partial x_m} \frac{\partial x_m}{\partial t_j}
        \]
        for any \(j \in \{1, 2, \ldots, n\}\).
    \item \textbf{Chain rule by a tree diagram}. Suppose we have some function $z=f(x,y)$, where $x=g(t)$ and $y=h(t)$, so $z$ has two variables $x$ and $y$, both of which depend on $t$. To find the formula for the derivative $\frac{dz}{dt}$, we can create a tree diagram
        \bigbreak \noindent 
    \begin{figure}[ht]
        \centering
        \incfig{tree}
        \label{fig:tree}
    \end{figure}
    \bigbreak \noindent 
    \bigbreak \noindent 
    The way we use the diagram is simple. We want to find $\frac{dz}{dt}$, which you notice is an ordinary derivative, because the leaf nodes are all the same variable. First, we start with the left side, we take derivatives all the way down, multiplying, and then the same for the right side. We sum the two sides. Thus this becomes
    \begin{align*}
        \frac{dz}{dt} = \frac{\delta z}{\delta x} \cdot \frac{dx}{dt}  + \frac{\delta z}{\delta y} \cdot \frac{dy}{dt}
    .\end{align*}
    \bigbreak \noindent 
    \item \textbf{Changing variables after computing derivative}: After we compute the derivative, we need to change all instances of $x$ and $y$ by using the equation that these variables are equal to. This way our final answer is in terms of the leaf node variables (see the tree diagram above)
    \item \textbf{Implicit Differentiation of a Function of Two or More Variables}:
        Suppose the function \(z = f(x,y)\) defines \(y\) implicitly as a function \(y = g(x)\) of \(x\) via the equation \(f(x,y) = 0\). Then
        \[
            \frac{dy}{dx} = -\frac{\partial f/\partial x}{\partial f/\partial y}
        \]
        provided \(\partial f/\partial y(x,y) \neq 0\).
        \bigbreak \noindent 
        If the equation \(f(x,y,z) = 0\) defines \(z\) implicitly as a differentiable function of \(x\) and \(y\), then
        \[
            \frac{\partial z}{\partial x} = -\frac{\partial f/\partial x}{\partial f/\partial z} \quad \text{and} \quad \frac{\partial z}{\partial y} = -\frac{\partial f/\partial y}{\partial f/\partial z}
        \]
        as long as \(\partial f/\partial z(x,y,z) \neq 0\).
    \item \textbf{Tangent plane for a surface defined implicitly}
        \begin{align*}
            F_{x}(x_{0},y_{0},z_{0})(x-x_{0}) + F_{y}(x_{0}, y_{0}, z_{0})(y-y_{0}) + F_{z}(x_{0}, y_{0}, z_{0})(z-z_{0}) = 0
        .\end{align*}
    \item \textbf{Direction Derivatives limit definition}:
        Suppose $z=f(x,y)$ is a function of two variables with a domain of $D$. Let $(a,b) \in D$ and define $\mathbf{u}=\cos\theta \mathbf{i} + \sin\theta \mathbf{j}$. Then the directional derivative of $f$ in the direction of $\mathbf{u}$ is given by
        \begin{equation}
            D_{\mathbf{u}}f(a,b) = \lim_{h \to 0} \frac{f(a+h\cos\theta, b+h\sin\theta) - f(a,b)}{h},
        \end{equation}
        provided the limit exists.
    \item \textbf{Directional derivatives with partial derivatives}:
        Let $z=f(x,y)$ be a function of two variables $x$ and $y$, and assume that $f_x$ and $f_y$ exist and $f(x, y)$ is differentiable everywhere. Then the directional derivative of $f$ in the direction of $\mathbf{u}=\cos\theta \mathbf{i} + \sin\theta \mathbf{j}$ is given by
        \[
            D_{\mathbf{u}}f(x,y) = f_x(x,y)\cos\theta + f_y(x,y)\sin\theta.
        \]
    \item \textbf{Gradient}:
        Let $z=f(x,y)$ be a function of $x$ and $y$ such that $f_x$ and $f_y$ exist. The vector $\nabla f(x,y)$ is called the gradient of $f$ and is defined as
        \begin{equation}
        \nabla f(x,y) = f_x(x,y)\mathbf{i} + f_y(x,y)\mathbf{j}.
        \end{equation}
        The vector $\nabla f(x,y)$ is also written as “grad $f$.”
    \item \textbf{Divide by the norm}: If the vector that is given for the direction of the derivative is not a unit vector, then it is only necessary to divide by the norm of the vector.
    \item \textbf{Directional derivative with the gradient}:
        \begin{align*}
            D_{u}f(x,y) = \nabla f(x,y) \cdot \mathbf{u}
        .\end{align*}
    \item \textbf{Properties of the Gradient}:
        Suppose the function $z=f(x,y)$ is differentiable at $(x_0,y_0)$ (Figure 4.41).
        \begin{enumerate}
            \item If $\nabla f(x_0,y_0) = 0$, then $D_{\mathbf{u}}f(x_0,y_0) = 0$ for any unit vector $\mathbf{u}$.
            \item If $\nabla f(x_0,y_0) \neq 0$, then $D_{\mathbf{u}}f(x_0,y_0)$ is maximized when $\mathbf{u}$ points in the same direction as $\nabla f(x_0,y_0)$. The maximum value of $D_{\mathbf{u}}f(x_0,y_0)$ is $\|\nabla f(x_0,y_0)\|$.
            \item If $\nabla f(x_0,y_0) \neq 0$, then $D_{\mathbf{u}}f(x_0,y_0)$ is minimized when $\mathbf{u}$ points in the opposite direction from $\nabla f(x_0,y_0)$. The minimum value of $D_{\mathbf{u}}f(x_0,y_0)$ is $-\|\nabla f(x_0,y_0)\|$.
        \end{enumerate}
    \item \textbf{Gradient Is Normal to the Level Curve}:
        Suppose the function $z=f(x,y)$ has continuous first-order partial derivatives in an open disk centered at a point $(x_0,y_0)$. If $\nabla f(x_0,y_0) \neq 0$, then $\nabla f(x_0,y_0)$ is normal to the level curve of $f$ at $(x_0,y_0)$.
    \item \textbf{Finding a tangent vector with normal vector}: Suppose we use the theorem above to find the vector tangent to a level curve at some point, we can then find a tangent vector by reversing the components and multiplying either one by negative one.
        \bigbreak \noindent 
        Example:
        \begin{align*}
            &\nabla f(-2,1) = 9\hat{\mathbf{i}} + 18\hat{\mathbf{j}} \quad \text{(would be normal to some level curve)} \\
            &-18\hat{\mathbf{i}} -9\hat{\mathbf{j}} \quad \text{Tangent vector at (-2,1)}
        .\end{align*}
        \bigbreak \noindent 
        \fig{.7}{./figures/nabla.jpeg}
    \item \textbf{Gradient in three variables}: 
        Let $w=f(x,y,z)$ be a function of three variables such that $f_x$, $f_y$, and $f_z$ exist. The vector $\nabla f(x,y,z)$ is called the gradient of $f$ and is defined as
        \begin{equation}
            \nabla f(x,y,z) = f_x(x,y,z)\mathbf{i} + f_y(x,y,z)\mathbf{j} + f_z(x,y,z)\mathbf{k}.
        \end{equation}
        $\nabla f(x,y,z)$ can also be written as $\text{grad}f(x,y,z)$.
        \pagebreak 
    \item \textbf{Directional derivative for functions of three variables (limit definition)}:
            Suppose $w=f(x,y,z)$ is a function of three variables with a domain of $D$. Let $(x_0,y_0,z_0) \in D$ and let $\mathbf{u}=\cos\alpha\mathbf{i}+\cos\beta\mathbf{j}+\cos\gamma\mathbf{k}$ be a unit vector. Then, the directional derivative of $f$ in the direction of $\mathbf{u}$ is given by
        \begin{equation}
        D_{\mathbf{u}}f(x_0,y_0,z_0) = \lim_{t \to 0} \frac{f(x_0 + t\cos\alpha, y_0 + t\cos\beta, z_0 + t\cos\gamma) - f(x_0,y_0,z_0)}{t},
        \end{equation}
        provided the limit exists.
        \bigbreak \noindent 
        \textbf{Note:} The components of the unit vector are called the \textbf{directional cosines}
    \item \textbf{Directional derivative for functions of three variables (p.d definition)}:
        Let $f(x,y,z)$ be a differentiable function of three variables and let $\mathbf{u}=\cos\alpha\mathbf{i}+\cos\beta\mathbf{j}+\cos\gamma\mathbf{k}$ be a unit vector. Then, the directional derivative of $f$ in the direction of $\mathbf{u}$ is given by
        \begin{align*}
            D_{\mathbf{u}}f(x,y,z) &= \nabla f(x,y,z) \cdot \mathbf{u}  \\
            &= f_x(x,y,z)\cos\alpha + f_y(x,y,z)\cos\beta + f_z(x,y,z)\cos\gamma.
        .\end{align*}
    \item \textbf{Critical points}: 
        Let $z = f(x, y)$ be a function of two variables that is defined on an open set containing the point $(x_0, y_0)$. The point $(x_0, y_0)$ is called a critical point of a function of two variables $f$ if one of the two following conditions holds:
        \begin{enumerate}
            \item $f_x(x_0, y_0) = f_y(x_0, y_0) = 0$
            \item Either $f_x(x_0, y_0)$ or $f_y(x_0, y_0)$ does not exist.
        \end{enumerate}
    \item \textbf{Local max / Absolute max (global max)}:
        Let $z = f(x, y)$ be a function of two variables that is defined and continuous on an open set containing the point $(x_0, y_0)$. Then $f$ has a local maximum at $(x_0, y_0)$ if
        \[ f(x_0, y_0) \geq f(x, y) \]
        for all points $(x, y)$ within some disk centered at $(x_0, y_0)$. The number $f(x_0, y_0)$ is called a local maximum value. If the preceding inequality holds for every point $(x, y)$ in the domain of $f$, then $f$ has a global maximum (also called an absolute maximum) at $(x_0, y_0)$.
    \item \textbf{Local min / Absolute min (global min)}:
        The function $f$ has a local minimum at $(x_0, y_0)$ if
        \[ f(x_0, y_0) \leq f(x, y) \]
        for all points $(x, y)$ within some disk centered at $(x_0, y_0)$. The number $f(x_0, y_0)$ is called a local minimum value. If the preceding inequality holds for every point $(x, y)$ in the domain of $f$, then $f$ has a global minimum (also called an absolute minimum) at $(x_0, y_0)$.
    \item \textbf{Local Extremum}:
        If $f(x_0, y_0)$ is either a local maximum or local minimum value, then it is called a local extremum.
    \item \textbf{Fermat’s Theorem for Functions of Two Variables}: 
        Let $z = f(x, y)$ be a function of two variables that is defined and continuous on an open set containing the point $(x_0, y_0)$. Suppose $f_x$ and $f_y$ each exists at $(x_0, y_0)$. If $f$ has a local extremum at $(x_0, y_0)$, then $(x_0, y_0)$ is a critical point of $f$.
    \item \textbf{Saddle point}: Given the function $z = f(x, y)$, the point $(x_0, y_0, f(x_0, y_0))$ is a saddle point if both $f_x(x_0, y_0) = 0$ and $f_y(x_0, y_0) = 0$, but $f$ does not have a local extremum at $(x_0, y_0)$.
    \item \textbf{Second derivative test}:
        Let $z = f(x, y)$ be a function of two variables for which the first- and second-order partial derivatives are continuous on some disk containing the point $(x_0, y_0)$. Suppose $f_x(x_0, y_0) = 0$ and $f_y(x_0, y_0) = 0$. Define the quantity
        \[ D = f_{xx}(x_0, y_0)f_{yy}(x_0, y_0) - (f_{xy}(x_0, y_0))^2. \]
        \begin{enumerate}
            \item[I.] If $D > 0$ and $f_{xx}(x_0, y_0) > 0$, then $f$ has a local minimum at $(x_0, y_0)$.
            \item[II.] If $D > 0$ and $f_{xx}(x_0, y_0) < 0$, then $f$ has a local maximum at $(x_0, y_0)$.
            \item[III.] If $D < 0$, then $f$ has a saddle point at $(x_0, y_0)$.
            \item[IV.] If $D = 0$, then the test is inconclusive.
        \end{enumerate}
        \bigbreak \noindent 
        \fig{.6}{./figures/saddle.jpeg}
    \item \textbf{Extreme Value Theorem}:
        A continuous function $f(x,y)$ on a closed and bounded set $D$ in the plane attains an absolute maximum value at some point of $D$ and an absolute minimum value at some point of $D$.
    \item \textbf{Finding extreme values}:
        Assume $z=f(x,y)$ is a differentiable function of two variables defined on a closed, bounded set $D$. Then $f$ will attain the absolute maximum value and the absolute minimum value, which are, respectively, the largest and smallest values found among the following:
        \begin{enumerate}
            \item The values of $f$ at the critical points of $f$ in $D$.
            \item The values of $f$ on the boundary of $D$.
        \end{enumerate}
    \item \textbf{Method of Lagrange Multipliers: One Constraint}:
        Let $f$ and $g$ be functions of two variables with continuous partial derivatives at every point of some open set containing the smooth curve $g(x,y)=0$. Suppose that $f$, when restricted to points on the curve $g(x,y)=0$, has a local extremum at the point $(x_0,y_0)$ and that $\nabla g(x_0,y_0) \neq 0$. Then there is a number $\lambda$ called a Lagrange multiplier, for which
        \[
            \nabla f(x_0,y_0) = \lambda \nabla g(x_0,y_0).
        \]
    \item \textbf{Problem-Solving Strategy: Steps for Using Lagrange Multipliers}:
        Follow these steps to solve the optimization problem using Lagrange multipliers:
        \begin{enumerate}
            \item Determine the objective function $f(x,y)$ and the constraint function $g(x,y)$. Does the optimization problem involve maximizing or minimizing the objective function?
            \item Set up a system of equations using the following template:
                \begin{align*}
                    \nabla f(x_0,y_0) &= \lambda \nabla g(x_0,y_0)  \\
                    g(x_0,y_0) &= 0.
                .\end{align*}
            \item Solve for $x_0$ and $y_0$.
            \item The largest of the values of $f$ at the solutions found in step 3 maximizes $f$; the smallest of those values minimizes $f$.
        \end{enumerate}
    \item \textbf{Problems with Two Constraints}:
        The method of Lagrange multipliers can be applied to problems with more than one constraint. In this case, the optimization function, $w$, is a function of three variables:
        \[
            w = f(x, y, z)
        \]
        and it is subject to two constraints:
        \[
            g(x, y, z) = 0 \quad \text{and} \quad h(x, y, z) = 0.
        \]
        There are two Lagrange multipliers, $\lambda_1$ and $\lambda_2$, and the system of equations becomes
        \begin{align*}
            \nabla f(x_0, y_0, z_0) &= \lambda_1 \nabla g(x_0, y_0, z_0) + \lambda_2 \nabla h(x_0, y_0, z_0) \\
            g(x_0, y_0, z_0) &= 0 \\
            h(x_0, y_0, z_0) &= 0
        .\end{align*}
    \item \textbf{Using gradient to find tangent plane to level surface}: Suppose we have some function $z = f(x,y)$, we rearrange the function such that it becomes $F(x,y,z) = 0$. We then use the fact that $\nabla F(x_{0}, y_{0}, z_{0}) $ is normal to the surface $F(x,y,z)$ at $P_{0} $ 
    \end{itemize}

    \pagebreak 
    \subsection{Chapter 5: Multiple integration}
    \bigbreak \noindent 
    \subsubsection{Definitions and theorems}
    \begin{itemize}
        \item \textbf{Double integral intro 1: Recangular region $R$ and solid $S$}:
            Consider a continuous function $f(x,y) \geq 0$ of two variables defined on the closed rectangle $R$:
            \[
                R = [a,b] \times [c,d] = \{(x,y) \in \mathbb{R}^2 \,|\, a \leq x \leq b, \, c \leq y \leq d\}
            \]
            \bigbreak \noindent 
            The graph of $f$ represents a surface above the $xy$-plane with equation $z = f(x,y)$ where $z$ is the height of the surface at the point $(x,y)$. Let $S$ be the solid that lies above $R$ and under the graph of $f$. The base of the solid is the rectangle $R$ in the $xy$-plane. We want to find the volume $V$ of the solid $S$.
            \bigbreak \noindent 
            \fig{.6}{./figures/andstuff.jpeg}
        \item \textbf{Double integral intro 2: Divisions of $R$}:
            We divide the region $R$ into small rectangles $R_{ij}$, each with area $\Delta A$ and with sides $\Delta x$ and $\Delta y$. We do this by dividing the interval $[a,b]$ into $m$ subintervals and dividing the interval $[c,d]$ into $n$ subintervals. Hence, $\Delta x = \frac{b-a}{m}$, $\Delta y = \frac{d-c}{n}$, and $\Delta A = \Delta x \Delta y$.
            \bigbreak \noindent 
            \fig{.6}{./figures/nstuff2.jpeg}
        \item \textbf{Double integral intro 3: volume of the subregions}:
            The volume of a thin rectangular box above $R_{ij}$ is $f(x^*_{ij}, y^*_{ij})\Delta A$, where $(x^*_{ij}, y^*_{ij})$ is an arbitrary sample point in each $R_{ij}$ as shown in the following figure.
            \bigbreak \noindent 
            \fig{.6}{./figures/nstuff3.jpeg}
        \item \textbf{Double integral intro 4: Double Reimann sum}:
            Using the same idea for all the subrectangles, we obtain an approximate volume of the solid $S$ as 
            \[ V \approx \sum_{i=1}^{m}\sum_{j=1}^{n} f(x^*_{ij}, y^*_{ij})\Delta A. \]
            This sum is known as a double Riemann sum and can be used to approximate the value of the volume of the solid. Here, the double sum means that for each subrectangle, we evaluate the function at the chosen point, multiply by the area of each rectangle, and then add all the results.
            \bigbreak \noindent 
            As we have seen in the single-variable case, we obtain a better approximation to the actual volume if $m$ and $n$ become larger.
            \[ V = \lim_{m,n \to \infty} \sum_{i=1}^{m}\sum_{j=1}^{n} f(x^*_{ij}, y^*_{ij})\Delta A \]
            or
            \[ V = \lim_{\Delta x, \Delta y \to 0} \sum_{i=1}^{m}\sum_{j=1}^{n} f(x^*_{ij}, y^*_{ij})\Delta A. \]
            Note that the sum approaches a limit in either case, and the limit is the volume of the solid with the base $R$. Now we are ready to define the double integral.

        \item \textbf{The double integral over a recangular region}:
            The double integral of the function $f(x,y)$ over the rectangular region $R$ in the $xy$-plane is defined as
            \begin{equation}
                \iint_{R} f(x,y) \, dA = \lim_{m,n \to \infty} \sum_{i=1}^{m}\sum_{j=1}^{n} f(x^*_{i}, y^*_{j})\Delta A 
            \end{equation}
            \bigbreak \noindent 
            \textbf{Note:} If $f(x,y) \geq 0$, then the volume $V$ of the solid $S$, which lies above $R$ in the $xy$-plane and under the graph of $f$, is the double integral of the function $f(x,y)$ over the rectangle $R$. If the function is ever negative, then the double integral can be considered a “signed” volume in a manner similar to the way we defined net signed area in The Definite Integral.
            \bigbreak \noindent 
            \textbf{Example:} Suppose we have the surface defined by $z=f(x,y) = 3x^{2} + y$ with the region $[0,2] \times [0,2]$
            \begin{align*}
                \iint_{R} 3x^{2} + y\ dA = \lim\limits_{m,n \to \infty}{\summation{m}{i=1}\summation{n}{j=1} \left[(x^{*}_{ij})^{2} + y^{*}_{ij}\right]\Delta A}
            .\end{align*}
        \item \textbf{Double integral exists and function is integrable}:
            The double integral of the function $z = f(x,y)$ exists provided that the function $f$ is not too discontinuous. If the function is bounded and continuous over $R$ except on a finite number of smooth curves, then the double integral exists and we say that $f$ is integrable over $R$.
        \item \textbf{Double integral with $dx$ and $dy$}: 
            Since $\Delta A = \Delta x \Delta y = \Delta y \Delta x$, we can express $dA$ as $dx\,dy$ or $dy\,dx$. This means that, when we are using rectangular coordinates, the double integral over a region $R$ denoted by $\iint_{R} f(x,y) \, dA$ can be written as $\iint_{R} f(x,y) \, dx\,dy$ or $\iint_{R} f(x,y) \, dy\,dx$.
        \item \textbf{Properties of double integrals}:
            Assume that the functions $f(x,y)$ and $g(x,y)$ are integrable over the rectangular region $R$; $S$ and $T$ are subregions of $R$; and assume that $m$ and $M$ are real numbers.
            \begin{enumerate}[label=\Roman*.]
                \item The sum $f(x,y) + g(x,y)$ is integrable and
                    \[\iint_{R} [f(x,y) + g(x,y)] \, dA = \iint_{R} f(x,y) \, dA + \iint_{R} g(x,y) \, dA.\]
                \item If $c$ is a constant, then $cf(x,y)$ is integrable and
                    \[\iint_{R} cf(x,y) \, dA = c \iint_{R} f(x,y) \, dA.\]
                \item If $R = S \cup T$ and $S \cap T = \emptyset$ except an overlap on the boundaries, then
                    \[\iint_{R} f(x,y) \, dA = \iint_{S} f(x,y) \, dA + \iint_{T} f(x,y) \, dA.\]
                \item If $f(x,y) \geq g(x,y)$ for $(x,y)$ in $R$, then
                    \[\iint_{R} f(x,y) \, dA \geq \iint_{R} g(x,y) \, dA.\]
                \item If $m \leq f(x,y) \leq M$, then
                    \[m \times A(R) \leq \iint_{R} f(x,y) \, dA \leq M \times A(R).\]
                \item In the case where $f(x,y)$ can be factored as a product of a function $g(x)$ of $x$ only and a function $h(y)$ of $y$ only, then over the region $R = \{(x,y) \,|\, a \leq x \leq b, \, c \leq y \leq d\}$, the double integral can be written as
                    \[\iint_{R} f(x,y) \, dA = \left( \int_{a}^{b} g(x) \, dx \right) \left( \int_{c}^{d} h(y) \, dy \right).\]
                    \bigbreak \noindent 
                    \textbf{Example:} Evaluate the integral $\iint_{R} ye^{x} \cos(x) \, dA$ over the region $R = \{(x,y) \, | \, 0 \leq x \leq \frac{\pi}{2}, 0 \leq y \leq 1\}$.
                    \begin{align*}
                        &\left(\int_0^{\frac{\pi}{2}}\cos{\left(x\right)}\,dx\right)\left(\int_0^{1}e^{y}\,dy\right) \\
                        &=e-1
                    .\end{align*}
            \end{enumerate}
        \item \textbf{Illustrating property V}: Over the region $R = \{(x,y) \,|\, 1 \leq x \leq 3, \, 1 \leq y \leq 2\}$, we have $2 \leq x^2 + y^2 \leq 13$. Find a lower and an upper bound for the integral $\iint_{R} (x^2 + y^2) \, dA$.
            For a lower bound, integrate the constant function $2$ over the region $R$. For an upper bound, integrate the constant function $13$ over the region $R$.
            \begin{align*}
                \int_{1}^{2} \int_{1}^{3} 2 \, dx \, dy &= \int_{1}^{2} [2x]_{1}^{3} \, dy = \int_{1}^{2} 2(2) \, dy = 4y \bigg|_{1}^{2} = 4(2-1) = 4 \\
             \int_{1}^{2} \int_{1}^{3} 13 \, dx \, dy  &= \int_{1}^{2} [13x]_{1}^{3} \, dy = \int_{1}^{2} 13(2) \, dy = 26y \bigg|_{1}^{2} = 26(2-1) = 26.
            .\end{align*}
            Hence, we obtain $4 \leq \iint_{R} (x^2 + y^2) \, dA \leq 26$.

        \item \textbf{Iterated integrals}:
            Assume $a$, $b$, $c$, and $d$ are real numbers. We define an iterated integral for a function $f(x,y)$ over the rectangular region $R = [a,b] \times [c,d]$ as
            \begin{enumerate}[label=(\alph*)]
                \item \relax
                \begin{align*}
                    \int_{a}^{b} \int_{c}^{d} f(x,y) dy\ dx = \int_{a}^{b} \left[ \int_{c}^{d} f(x,y) \, dy \right] dx
                .\end{align*}
            \item \relax
                \begin{align*}
                    \int_{c}^{d} \int_{a}^{b} f(x,y) dx\ dy = \int_{c}^{d} \left[ \int_{a}^{b} f(x,y) \, dx \right] dy
                .\end{align*}
            \end{enumerate}
            \bigbreak \noindent 
            \textbf{Note:} The notation 
            \[
                \int_{a}^{b} \left[ \int_{c}^{d} f(x,y) \, dy \right] dx
            \]
            means that we integrate $f(x,y)$ with respect to $y$ while holding $x$ constant. Similarly, the notation 
            \[
                \int_{c}^{d} \left[ \int_{a}^{b} f(x,y) \, dx \right] dy
            \]
            means that we integrate $f(x,y)$ with respect to $x$ while holding $y$ constant.
        \item \textbf{Fubini’s Theorem}:
            Suppose that $f(x,y)$ is a function of two variables that is continuous over a rectangular region $R = \{(x,y) \in \mathbb{R}^2 \,|\, a \leq x \leq b, \, c \leq y \leq d\}$. Then we see from the figure below that the double integral of $f$ over the region equals an iterated integral,
            \[
                \iint_{R} f(x,y) \, dA = \iint_{R} f(x,y) \, dx \, dy = \int_{a}^{b} \left( \int_{c}^{d} f(x,y) \, dy \right) dx = \int_{c}^{d} \left( \int_{a}^{b} f(x,y) \, dx \right) dy.
            \]
            More generally, Fubini’s theorem is true if $f$ is bounded on $R$ and $f$ is discontinuous only on a finite number of continuous curves. In other words, $f$ has to be integrable over $R$.
            \bigbreak \noindent 
            \fig{.6}{./figures/fubini.jpeg}
        \item \textbf{Area of a region $R$}:
            The area of the region $R$ is given by  
            \begin{align*}
                A(R) = \iint_{R}1\,dA
            .\end{align*}
        \item \textbf{Recall: Average value of a function of one variable}: the average value of a function of one variable on an interval  $[a,b]$ is given by
            \begin{align*}
                \bar{f} = \frac{1}{b-a}\int_{a}^{b}\ f(x)\ dx
            .\end{align*}
        \item \textbf{Average value of a function of two variables}: The average value of a function of two variables over a region $R$ is given by
            \begin{align*}
                \bar{f} = \frac{1}{A(R)} \iint_{R}f(x,y)\,dA
            .\end{align*}
        \item \textbf{Non rectangular region $D$}:
            Since \(D\) is bounded on the plane, there must exist a rectangular region \(R\) on the same plane that encloses the region \(D\), that is, a rectangular region \(R\) exists such that \(D\) is a subset of \(R\) (\(D \subseteq R\)).
            \bigbreak \noindent 
            We extend the definition of the function to include all points on the rectangular region \(R\) and then use the concepts and tools from the preceding section. But how do we extend the definition of \(f\) to include all the points on \(R\)? We do this by defining a new function \(g(x,y)\) on \(R\) as follows:
            \[
                g(x,y) = 
                \begin{cases} 
                    f(x,y) & \text{if } (x,y) \text{ is in } D \\
                    0 & \text{if } (x,y) \text{ is in } R \text{ but not in } D
                \end{cases}
            \]
            \bigbreak \noindent 
            \textbf{Note:} we assume the boundary to be a piecewise smooth and continuous simple closed curve. We must be careful about \(g(x,y)\) and verify that \(g(x,y)\) is an integrable function over the rectangular region \(R\). This happens as long as the region \(D\) is bounded by simple closed curves.
        \item \textbf{Types of planar bounded regions}:
            A region \(D\) in the \((x,y)\)-plane is of \textbf{Type I} if it lies between two vertical lines and the graphs of two continuous functions \(g_1(x)\) and \(g_2(x)\). 
            \[ D = \{(x,y) \mid a \leq x \leq b, g_1(x) \leq y \leq g_2(x)\}. \]
            \bigbreak \noindent 
            \fig{.5}{./figures/type1.jpeg}
            A region \(D\) in the \(xy\)-plane is of \textbf{Type II} if it lies between two horizontal lines and the graphs of two continuous functions \(h_1(y)\) and \(h_2(y)\).
            \[ D = \{(x,y) \mid c \leq y \leq d, h_1(y) \leq x \leq h_2(y)\}. \]
            \bigbreak \noindent 
            \fig{.5}{./figures/type2.jpeg}
        \item \textbf{Double Integrals over Nonrectangular Regions}:
            Suppose \(g(x,y)\) is the extension to the rectangle \(R\) of the integrable function \(f(x,y)\) defined on the region \(D\), where \(D\) is inside \(R\). Then \(g(x,y)\) is integrable and we define the double integral of \(f(x,y)\) over \(D\) by
            \[
                \iint_{D} f(x,y) \, dA = \iint_{R} g(x,y) \, dA.
            \]
            \bigbreak \noindent 
            \textbf{Note:} The equality works because the values of \(g(x,y)\) are \(0\) for any point \((x,y)\) that lies outside \(D\), and hence these points do not add anything to the integral. However, it is important that the rectangle \(R\) contains the region \(D\).
        \item \textbf{Fubini’s Theorem (Strong Form)}:
            For a function \(f(x,y)\) that is continuous on a region \(D\) of Type I, we have
            \begin{equation}
                \iint_{D} f(x,y) \, dA = \iint_{D}f(x,y)dy\, dx= \int_{a}^{b} \left[ \int_{g_1(x)}^{g_2(x)} f(x,y) \, dy \right] dx.
            \end{equation}
            Similarly, for a function \(f(x,y)\) that is continuous on a region \(D\) of Type II, we have
            \begin{equation}
                \iint_{D} f(x,y) \, dA  = \iint_{D}f(x,y)dx\, dy= \int_{c}^{d} \left[ \int_{h_1(y)}^{h_2(y)} f(x,y) \, dx \right] dy.
            \end{equation}
        \item \textbf{Decomposing Regions into Smaller Regions}:
            Suppose the region \(D\) can be expressed as \(D = D_1 \cup D_2\) where \(D_1\) and \(D_2\) do not overlap except at their boundaries. Then
            \[
                \iint_{D} f(x,y) \, dA = \iint_{D_1} f(x,y) \, dA + \iint_{D_2} f(x,y) \, dA.
            \]
        \item \textbf{We can always change our region from type 1 to type 2 to make our iterated integral easier to solve}
        \item \textbf{Area of a plane bounded by a region $D$}:
            The area of a plane-bounded region \(D\) is defined as the double integral \(\iint_{D} 1 \, dA\).
        \item \textbf{Average value of a function over a general region}:
            If \(f(x,y)\) is integrable over a plane-bounded region \(D\) with positive area \(A(D)\), then the average value of the function is
            \[
                f_{\text{ave}} = \frac{1}{A(D)} \iint_{D} f(x,y) \, dA.
            \]
            Note that the area is \(A(D) = \iint_{D} 1 \, dA\).
        \item \textbf{Fubini’s Theorem for Improper Integrals}:
            If $D$ is a bounded rectangle or simple region in the plane defined by $\{(x,y) : a \leq x \leq b, g(x) \leq y \leq h(x)\}$ and also by $\{(x,y) : c \leq y \leq d, j(y) \leq x \leq k(y)\}$, and $f$ is a nonnegative function on $D$ with finitely many discontinuities in the interior of $D$, then
            \[
                \iint_D f \, dA = \int_{x=a}^{x=b} \int_{y=g(x)}^{y=h(x)} f(x,y) \, dy \, dx = \int_{y=c}^{y=d} \int_{x=j(y)}^{x=k(y)} f(x,y) \, dx \, dy.
            \]
            \bigbreak \noindent 
            \textbf{Note:} It is very important to note that we required that the function be nonnegative on  $D$ for the theorem to work. We consider only the case where the function has finitely many discontinuities inside $D$.
        \item \textbf{Improper Integrals on an Unbounded Region}:
            If $R$ is an unbounded rectangle such as $R = \{(x,y) : a \leq x < \infty, c \leq y < \infty\}$, then when the limit exists, we have
            \begin{align*}
                \iint_R f(x,y) \, dA  &\\
                &= \lim_{(b,d) \to (\infty, \infty)} \int_a^b \left( \int_c^d f(x,y) \, dy \right) dx  \\
                &= \lim_{(b,d) \to (\infty, \infty)} \int_c^d \left( \int_a^b f(x,y) \, dx \right) dy
            .\end{align*}

            \begin{align*}
                \lim\limits_{(b,d) \to (\infty,\infty)}{\frac{1}{4}(1-e^{-b^{2}})(1-e^{-d^{2}})}
            .\end{align*}
        \item \textbf{Joint density function}:
            Consider a pair of continuous random variables \(X\) and \(Y\), such as the birthdays of two people or the number of sunny and rainy days in a month. The joint density function \(f\) of \(X\) and \(Y\) satisfies the probability that \((X,Y)\) lies in a certain region \(D\):
            \[
                P((X,Y) \in D) = \iint_D f(x,y) \, dA.
            \]
            Since the probabilities can never be negative and must lie between \(0\) and \(1\), the joint density function satisfies the following inequality and equation:
            \[
                f(x,y) \geq 0 \quad \text{and} \quad \iint_{\mathbb{R}^2} f(x,y) \, dA = 1.
            \]
        \item \textbf{Independent random variable classification}:
            The variables \(X\) and \(Y\) are said to be independent random variables if their joint density function is the product of their individual density functions:
            \[
                f(x,y) = f_1(x) f_2(y).
            \]

        \item \textbf{Probability theory expected values}:
            In probability theory, we denote the expected values \(E(X)\) and \(E(Y)\), respectively, as the most likely outcomes of the events. The expected values \(E(X)\) and \(E(Y)\) are given by
            \[
                E(X) = \iint_S x f(x,y) \, dA \quad \text{and} \quad E(Y) = \iint_S y f(x,y) \, dA,
            \]
            where \(S\) is the sample space of the random variables \(X\) and \(Y\).
        \item \textbf{Double integral in polar coordinates}:
            The double integral of the function $f(r,\theta)$ over the polar rectangular region $R$ in the $r\theta$-plane is defined as
            \begin{align*}
                &\iint_R f(r,\theta) \, dA  \\
                &= \lim_{m,n \to \infty} \sum_{i=1}^{m} \sum_{j=1}^{n} f(r^*_{ij}, \theta^*_{ij}) \Delta A  \\
                &= \lim_{m,n \to \infty} \sum_{i=1}^{m} \sum_{j=1}^{n} f(r^*_{ij}, \theta^*_{ij}) r^*_{ij} \Delta r \Delta \theta.
            .\end{align*}
        \item \textbf{Iterated integral for polar regions}
            \begin{align*}
                &\iint_R f(r,\theta )\, dA = \iint_R f(r,\theta )r \, dr \, d\theta \\
                &=\int_{\theta =\alpha}^{\theta =\beta} \int_{r=a}^{r=b}f(r,\theta )r dr\, d\theta 
            .\end{align*}
        \item \textbf{Integral when changinig from $xy$ to $r\theta$ }
            \begin{align*}
                \iint_R f(x,y)\, dA = \iint_R f(r\cos{\left(\theta \right)}, r\sin{\left(\theta \right)})r\, dr\, d\theta 
            .\end{align*}
            \textbf{NOTICE:} The extra $r$ at the end of the integrand
        \item \textbf{Double Integrals over General Polar Regions}:
            If $f(r,\theta)$ is continuous on a general polar region $D$ as described above, then the double integral over $D$ can be expressed as:
            \[
                \iint_D f(r,\theta) r \, dr \, d\theta = \int_{\theta=\alpha}^{\theta=\beta} \int_{r=h_1(\theta)}^{r=h_2(\theta)} f(r,\theta) r \, dr \, d\theta
            \]
        \item \textbf{Area of a polar region}
            \begin{align*}
                A = \int_{\alpha}^{\beta}\int_{h_{1}(\theta )}^{h_{2}(\theta )}r\, dr\, d\theta 
            .\end{align*}
        \item \textbf{Equation of an arbitrary cone with radius $a$ and height $h$}
            \begin{align*}
                z = h - \frac{h}{a}\sqrt{x^{2} = y^{2}}
            .\end{align*}
        \item \textbf{Entire $xy$ plane region to polar region}: Suppose we have the region $\mathbb{R}^{2}$, ie the entire $xy$-plane. This can be seen as
            \begin{align*}
                0 \leq \theta  \leq 2\pi, \quad 0 \leq r < \infty
            .\end{align*}
        \item \textbf{Triple integral}:
            The triple integral of a function $f(x,y,z)$ over a rectangular box $B$ is defined as
            \[
                \lim_{l,m,n \to \infty} \sum_{i=1}^{l} \sum_{j=1}^{m} \sum_{k=1}^{n} f(x^*_{ijk}, y^*_{ijk}, z^*_{ijk}) \Delta x \Delta y \Delta z = \iiint_B f(x,y,z) \, dV
            \]
            if this limit exists. 
            \bigbreak \noindent 
            \textbf{Note:} When the triple integral exists on $B$, the function $f(x,y,z)$ is said to be integrable on $B$. Also, the triple integral exists if $f(x,y,z)$ is continuous on $B$. Therefore, we will use continuous functions for our examples. However, continuity is sufficient but not necessary; in other words, $f$ is bounded on $B$ and continuous except possibly on the boundary of $B$.
            \bigbreak \noindent 
            The sample point $(x^*_{ijk}, y^*_{ijk}, z^*_{ijk})$ can be any point in the rectangular sub-box $B_{ijk}$ and all the properties of a double integral apply to a triple integral.
            \textbf{Fubini's Theorem for Triple Integrals}
            If $f(x,y,z)$ is continuous on a rectangular box $B = [a,b] \times [c,d] \times [e,f]$, then
            \[
                \iiint_B f(x,y,z) \, dV = \int_e^f \int_c^d \int_a^b f(x,y,z) \, dx \, dy \, dz.
            \]
            This integral is also equal to any of the other five possible orderings for the iterated triple integral.
        \item \textbf{Triple integral over general regions}:
            The triple integral of a continuous function $f(x,y,z)$ over a general three-dimensional region
            \[
                E = \left\{ (x,y,z) \mid (x,y) \in D, u_1(x,y) \leq z \leq u_2(x,y) \right\}
            \]
            in $\mathbb{R}^3$, where $D$ is the projection of $E$ onto the $xy$-plane, is
            \[
                \iiint_E f(x,y,z) \, dV = \iint_D \left[ \int_{u_1(x,y)}^{u_2(x,y)} f(x,y,z) \, dz \right] dA.
            \]
        \item \textbf{Volume of a general region $E$ with triple integrals}: To find the volume of a general  region $E$, we use
            \begin{align*}
                \iiint_E dV
            .\end{align*}
        \item \textbf{Average Value of a Function of Three Variables}
            If $f(x,y,z)$ is integrable over a solid bounded region $E$ with positive volume $V(E)$, then the average value of the function is
            \[
                f_{\text{ave}} = \frac{1}{V(E)} \iiint_E f(x,y,z) \, dV.
            \]
            Note that the volume is $V(E) = \iiint_E 1 \, dV$.
        \item \textbf{Equation of a plane with intercepts $(a,0,0),\ (0,b,0),\ (0,0,c) $}
            \begin{align*}
                \frac{x}{a} + \frac{y}{b} + \frac{z}{c} = 1
            .\end{align*}
        \item \textbf{Cylindrical coordinates}:
            In three-dimensional space \(\mathbb{R}^3\), a point with rectangular coordinates \((x,y,z)\) can be identified with cylindrical coordinates \((r,\theta,z)\) and vice versa. We can use these same conversion relationships, adding \(z\) as the vertical distance to the point from the \(xy\)-plane as shown in the following figure.
            \bigbreak \noindent 
            \fig{.8}{./figures/cyl.jpeg}
        \item \textbf{Fubini’s Theorem in Cylindrical Coordinates}:
            Suppose that \(g(x,y,z)\) is continuous on a portion of a circular cylinder \(B\), which when described in cylindrical coordinates looks like \(B = \{(r,\theta,z) \mid a \leq r \leq b, \alpha \leq \theta \leq \beta, c \leq z \leq d\}\).
            \bigbreak \noindent 
            Then \(g(x,y,z) = g(r \cos \theta, r \sin \theta, z) = f(r, \theta, z)\) and
            \[
                \iiint_B g(x,y,z) \, dV = \int_c^d \int_\alpha^\beta \int_a^b f(r, \theta, z) r \, dr \, d\theta \, dz.
            \]
            \bigbreak \noindent 
            \textbf{Note:} The iterated integral may be replaced equivalently by any one of the other five iterated integrals obtained by integrating with respect to the three variables in other orders.
        \item \textbf{Cylindrical region is a general solid}:
            If the cylindrical region over which we have to integrate is a general solid, we look at the projections onto the coordinate planes. Hence the triple integral of a continuous function \(f(r, \theta, z)\) over a general solid region \(E = \{(r, \theta, z) \mid (r, \theta) \in D, u_1(r, \theta) \leq z \leq u_2(r, \theta)\}\) in \(\mathbb{R}^3\), where \(D\) is the projection of \(E\) onto the \(r\theta\)-plane, is
            \[
                \iiint_E f(r, \theta, z) r \, dr \, d\theta \, dz = \iint_D \left[ \int_{u_1(r, \theta)}^{u_2(r, \theta)} f(r, \theta, z) \, dz \right] r \, dr \, d\theta.
            \]
            In particular, if \(D = \{(r, \theta) \mid g_1(\theta) \leq r \leq g_2(\theta), \alpha \leq \theta \leq \beta\}\), then we have
            \[
                \iiint_E f(r, \theta, z) r \, dr \, d\theta = \int_{\theta=\alpha}^{\theta=\beta} \int_{r=g_1(\theta)}^{r=g_2(\theta)} \int_{z=u_1(r, \theta)}^{z=u_2(r, \theta)} f(r, \theta, z) r \, dz \, dr \, d\theta.
            \]
            Similar formulas exist for projections onto the other coordinate planes. We can use polar coordinates in those planes if necessary.

        \item \textbf{Spherical coordinates}:
            In three-dimensional space \(\mathbb{R}^3\) in the spherical coordinate system, we specify a point \(P\) by its distance \(\rho\) from the origin, the polar angle \(\theta\) from the positive \(x\)-axis (same as in the cylindrical coordinate system), and the angle \(\phi\) from the positive \(z\)-axis and the line \(OP\). Note that \(\rho \geq 0\) and \(0 \leq \phi \leq \pi\).
            \bigbreak \noindent 
            \fig{.8}{./figures/sphere.jpeg}
            \bigbreak \noindent 
            \textbf{Note:} Spherical coordinates are useful for triple integrals over regions that are symmetric with respect to the origin.
    \item \textbf{Rectangular to spherical}
        \begin{align*}
            x &= \rho\sin{\left(\varphi\right)}\cos{\left(\theta \right)} \\
            y &= \rho\sin{\left(\varphi\right)}\sin{\left(\theta \right)} \\
            z &= \rho\cos{\left(\varphi\right)} \\
            \rho^{2} &= x^{2} + y^{2} + z^{2} \\
            \tan{\left(\theta \right)} &= \frac{y}{x} \\
            \varphi &= \cos^{-1}{\left(\frac{z}{\sqrt{x^{2} + y^{2} + z^{2}}}\right)}
        .\end{align*}
    \item \textbf{Spherical to cylindrical}
        \begin{align*}
            r &= \rho\sin{\left(\varphi\right)} \\
            \theta  &= \theta  \\
            z &= \rho\cos{\left(\varphi\right)} 
        .\end{align*}
    \item \textbf{Cylindrical to spherical}
        \begin{align*}
            \rho &= \sqrt{r^{2} + z^{2}} \\
            \theta  &= \theta  \\
            \varphi &= \cos^{-1}{\left(\frac{z}{\sqrt{r^{2} + z^{2}}}\right)}
        .\end{align*}
    \item \textbf{solid regions that are convenient to express in spherical coordinates.}
        \bigbreak \noindent 
        \fig{.8}{./figures/scon.jpeg}
    \item \textbf{Fubini’s Theorem for Spherical Coordinates}:
        If \(f(\rho, \theta, \phi)\) is continuous on a spherical solid box \(B = [a,b] \times [\alpha,\beta] \times [\gamma,\psi]\), then
        \[
            \iiint_B f(\rho, \theta, \phi) \rho^2 \sin \phi \, d\rho \, d\phi \, d\theta = \int_{\theta=\alpha}^{\theta=\beta} \int_{\phi=\gamma}^{\phi=\psi} \int_{\rho=a}^{\rho=b} f(\rho, \theta, \phi) \rho^2 \sin \phi \, d\rho \, d\phi \, d\theta.
        \]
        This iterated integral may be replaced by other iterated integrals by integrating with respect to the three variables in other orders.
    \item \textbf{Spherical region is a general solid}:
    \item \textbf{Jacobian}:
            The Jacobian of the \(C^1\) transformation \(T(u,v) = (g(u,v), h(u,v))\) is denoted by \(J(u,v)\) and is defined by the \(2 \times 2\) determinant
            \[
                J(u,v) = \left| \frac{\partial(x,y)}{\partial(u,v)} \right| = \left| \begin{array}{cc}
                    \frac{\partial x}{\partial u} & \frac{\partial x}{\partial v} \\
                    \frac{\partial y}{\partial u} & \frac{\partial y}{\partial v}
                \end{array} \right| = \left( \frac{\partial x}{\partial u} \frac{\partial y}{\partial v} - \frac{\partial x}{\partial v} \frac{\partial y}{\partial u} \right).
            \]
        \item A transformation \(T: G \to \mathbb{R}^2\), defined as \(T(u, v) = (x, y)\), is said to be a \textbf{one-to-one} transformation if no two points map to the same image point.














 









 






    \end{itemize}




\end{document}
