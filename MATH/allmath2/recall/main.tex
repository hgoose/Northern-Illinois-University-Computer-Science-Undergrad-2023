\documentclass{report}

\input{~/dev/latex/template/preamble.tex}
\input{~/dev/latex/template/macros.tex}

\title{\Huge{}}
\author{\huge{Nathan Warner}}
\date{\huge{}}
\fancyhf{}
\rhead{}
\fancyhead[R]{\itshape Warner} % Left header: Section name
\fancyhead[L]{\itshape\leftmark}  % Right header: Page number
\cfoot{\thepage}
\renewcommand{\headrulewidth}{0pt} % Optional: Removes the header line
%\pagestyle{fancy}
%\fancyhf{}
%\lhead{Warner \thepage}
%\rhead{}
% \lhead{\leftmark}
%\cfoot{\thepage}
%\setborder
% \usepackage[default]{sourcecodepro}
% \usepackage[T1]{fontenc}

% Change the title
\hypersetup{
    pdftitle={Math concepts to remember}
}

\begin{document}
    % \maketitle
        \begin{titlepage}
       \begin{center}
           \vspace*{1cm}
    
           \textbf{Practice problems for important concepts in calculus}
    
           \vspace{0.5cm}
            
                
           \vspace{1.5cm}
    
           \textbf{Nathan Warner}
    
           \vfill
                
                
           \vspace{0.8cm}
         
           \includegraphics[width=0.4\textwidth]{~/niu/seal.png}
                
           Computer Science \\
           Northern Illinois University\\
           January 23, 2023 \\
           United States\\
           
                
       \end{center}
    \end{titlepage}
    \tableofcontents
    \pagebreak 
    \unsect{Linear Approximation}
    \bigbreak \noindent 
    \begin{concept}
        Linear approximation is a mathematical method used to estimate the value of a function at a certain point based on its value and slope at another point. 
    \end{concept}
    \bigbreak \noindent 
    \begin{remark}
        \begin{align*}
            f(x) \approx L(x) = f(a) + f^{\prime}(a)(x-a)
        .\end{align*}
    \end{remark}
    \bigbreak \noindent 
    Suppose we have the quantity $\sqrt{80}$, and we wish to approximate it. For this we will use the fact that $\sqrt{81} = 9$. 
    
    
\end{document}
