\documentclass{report}

\input{~/dev/latex/template/preamble.tex}
\input{~/dev/latex/template/macros.tex}

\title{\Huge{}}
\author{\huge{Nathan Warner}}
\date{\huge{}}
\fancyhf{}
\rhead{}
\fancyhead[R]{\itshape Warner} % Left header: Section name
\fancyhead[L]{\itshape\leftmark}  % Right header: Page number
\cfoot{\thepage}
\renewcommand{\headrulewidth}{0pt} % Optional: Removes the header line
%\pagestyle{fancy}
%\fancyhf{}
%\lhead{Warner \thepage}
%\rhead{}
% \lhead{\leftmark}
%\cfoot{\thepage}
%\setborder
% \usepackage[default]{sourcecodepro}
% \usepackage[T1]{fontenc}

% Change the title
\hypersetup{
    pdftitle={Units and Measurement}
}

\begin{document}
    % \maketitle
        \begin{titlepage}
       \begin{center}
           \vspace*{1cm}
    
           \textbf{Chapter II} \\ 
            Vectors
    
           \vspace{0.5cm}
            
                
           \vspace{1.5cm}
    
           \textbf{Nathan Warner}
    
           \vfill
                
                
           \vspace{0.8cm}
         
           \includegraphics[width=0.4\textwidth]{~/niu/seal.png}
                
           Computer Science \\
           Northern Illinois University\\
           January 22, 2023 \\
           United States\\
           
                
       \end{center}
    \end{titlepage}
    \tableofcontents
    \pagebreak 
    \unsect{2.1 Scalars and Vectors}
    \bigbreak \noindent 
    Many familiar physical quantities can be specified completely by giving a single number and the appropriate unit. For example, “a class period lasts 50 min” or “the gas tank in my car holds 65 L” or “the distance between two posts is 100 m.” A physical quantity that can be specified completely in this manner is called a \textbf{scalar quantity}. Scalar is a synonym of “number.” Time, mass, distance, length, volume, temperature, and energy are examples of \textbf{scalar quantities}.
    \bigbreak \noindent 
    Physical quantities specified completely by giving a number of units (magnitude) and a direction are called \textbf{vector quantities}. Examples of vector quantities include displacement, velocity, position, force, and torque.
    \bigbreak \noindent 
    In the language of mathematics, physical vector quantities are represented by mathematical objects called \textbf{vectors}
    \bigbreak \noindent 
    We can add or subtract two vectors, and we can multiply a vector by a scalar or by another vector, but we cannot divide by a vector. The operation of division by a vector is not defined.
    \bigbreak \noindent 
    to distinguish between a vector and a scalar quantity, we adopt the common convention that a letter in bold type with an arrow above it denotes a vector, and a letter without an arrow denotes a scalar. For example, a distance of 2.0 km, which is a scalar quantity, is denoted by d = 2.0 km, whereas a displacement of 2.0 km in some direction, which is a vector quantity, is denoted by $\vec{\textbf{d}}$
    \bigbreak \noindent 
    Suppose you tell a friend on a camping trip that you have discovered a terrific fishing hole 6 km from your tent. It is unlikely your friend would be able to find the hole easily unless you also communicate the direction in which it can be found with respect to your campsite. You may say, for example, “Walk about 6 km northeast from my tent.” The key concept here is that you have to give not one but two pieces of information—namely, the distance or magnitude (6 km) and the direction (northeast).
    \bigbreak \noindent 
    Displacement is a general term used to describe a change in position, such as during a trip from the tent to the fishing hole. Displacement is an example of a vector quantity. If you walk from the tent (location A) to the hole (location B), the vector $\vec{D}$, representing your displacement, is drawn as the arrow that originates at point A and ends at point B. The arrowhead marks the end of the vector. The direction of the displacement vector $\vec{D}$ is the direction of the arrow. The length of the arrow represents the magnitude $D$ of vector $\vec{D}$. Here, $D = 6\,\text{km}$. Since the magnitude of a vector is its length, which is a positive number, the magnitude is also indicated by placing the absolute value notation around the symbol that denotes the vector; so, we can write equivalently that $D \equiv \lVert \vec{D} \rVert$. To solve a vector problem graphically, we need to draw the vector $\vec{D}$ to scale.

    \pagebreak \bigbreak \noindent 
    Suppose your friend walks from the campsite at A to the fishing pond at B and then walks back: from the fishing pond at B to the campsite at A. The magnitude of the displacement vector $\vec{D}_{AB}$ from A to B is the same as the magnitude of the displacement vector $\vec{D}_{BA}$ from B to A (it equals $6\,\text{km}$ in both cases), so we can write $D_{AB} = D_{BA}$. However, vector $\vec{D}_{AB}$ is not equal to vector $\vec{D}_{BA}$ because these two vectors have different directions: $\vec{D}_{AB} \neq \vec{D}_{BA}$. In Figure 2.3, vector $\vec{D}_{BA}$ would be represented by a vector with an origin at point B and an end at point A, indicating vector $\vec{D}_{BA}$ points to the southwest, which is exactly $180^\circ$ opposite to the direction of vector $\vec{D}_{AB}$. We say that vector $\vec{D}_{BA}$ is \textbf{antiparallel} to vector $\vec{D}_{AB}$ and write $\vec{D}_{AB} = -\vec{D}_{BA}$, where the minus sign indicates the antiparallel direction.
    \bigbreak \noindent 
    Two vectors that have identical directions are said to be \textbf{parallel} vectors—meaning, they are parallel to each other. Two parallel vectors $\vec{A}$ and $\vec{B}$ are equal, denoted by $\vec{A} = \vec{B}$, if and only if they have equal magnitudes $\lVert \vec{A} \rVert = \lVert \vec{B} \rVert$. Two vectors with directions perpendicular to each other are said to be \textbf{orthogonal} vectors. These relations between vectors are illustrated in Figure 2.5.
    \bigbreak \noindent 
    \fig{.8}{./figures/1.jpeg}
    
    \pagebreak \bigbreak \noindent 
    \subsection{Algebra of Vectors in One Dimension}
    \bigbreak \noindent 
    \begin{dfn}[Scalars]
        When a vector $\vec{A}$ is multiplied by a positive scalar $\alpha$, the result is a new vector $\vec{B}$ that is parallel to $\vec{A}$:
        \begin{align*}
            \vec{\textbf{B}} = \alpha \vec{\textbf{A}}
        .\end{align*}
        The magnitude of this new vector $\vec{\textbf{B}}$ is 
        \begin{align*}
            B = |\alpha|A
        .\end{align*}
        Where $B$ is the magnitude of $\vec{\textbf{B}} $ and $A$ is the magnitude of $\vec{\textbf{A}} $ 
    \end{dfn}
    \bigbreak \noindent 
    \subsection{Vector Laws}
    \begin{itemize}
        \item \textbf{Communitive law}: $\vec{A} + \vec{B} &= \vec{B} + \vec{A}$
        \item \textbf{Assosiative law}: $(\vec{A} + \vec{B}) + \vec{C} &= \vec{A} + (\vec{B} + \vec{C})$ 
        \item \textbf{Distributive law}: $\alpha_1 \vec{A} + \alpha_2 \vec{A} = (\alpha_1 + \alpha_2)\vec{A}$
    \end{itemize}
    \bigbreak \noindent 
    \subsection{Unit vectors}
    \bigbreak \noindent 
    When adding many vectors in one dimension, it is convenient to use the concept of a unit vector. A unit vector, which is denoted by a letter symbol with a hat, such as $\hat{u}$, has a magnitude of one and does not have any physical unit so that $|\hat{u}| \equiv u = 1$. The only role of a unit vector is to specify direction. For example, instead of saying vector $\vec{D}_{AB}$ has a magnitude of $6.0\,\text{km}$ and a direction of northeast, we can introduce a unit vector $\hat{u}$ that points to the northeast and say succinctly that $\vec{D}_{AB} = (6.0\,\text{km})\hat{u}$. Then the southwesterly direction is simply given by the unit vector $-\hat{u}$. In this way, the displacement of $6.0\,\text{km}$ in the southwesterly direction is expressed by the vector
    \[\vec{D}_{BA} = (-6.0\,\text{km})\hat{u}.\]

    \bigbreak \noindent 
    \begin{exm}[A Ladybug Walker]
        A long measuring stick rests against a wall in a physics laboratory with its 200-cm end at the floor. A ladybug lands on the 100-cm mark and crawls randomly along the stick. It first walks 15 cm toward the floor, then it walks 56 cm toward the wall, then it walks 3 cm toward the floor again. Then, after a brief stop, it continues for 25 cm toward the floor and then, again, it crawls up 19 cm toward the wall before coming to a complete rest (Figure 2.8). Find the vector of its total displacement and its final resting position on the stick.
    \end{exm}
    \bigbreak \noindent 
    \textcolor{red}{\textit{Solution.}}
    If we choose the direction along the stick toward the floor as the direction of unit vector $\hat{u}$, then the direction toward the floor is $+\hat{u}$ and the direction toward the wall is $-\hat{u}$. The ladybug makes a total of five displacements:
    \begin{align*}
        \vec{D}_1 &= (15\,\text{cm})(+\hat{u}) \\
        \vec{D}_2 &= (56\,\text{cm})(-\hat{u}) \\
        \vec{D}_3 &= (3\,\text{cm})(+\hat{u})  \\
        \vec{D}_4 &= (25\,\text{cm})(+\hat{u})  \\
        \vec{D}_5 &= (19\,\text{cm})(-\hat{u})
    .\end{align*}
    The total displacement $\vec{D}$ is the resultant of all its displacement vectors.
    \bigbreak \noindent 
    The resultant of all the displacement vectors is
    \begin{align*}
        &\vec{D} = \vec{D}_1 + \vec{D}_2 + \vec{D}_3 + \vec{D}_4 + \vec{D}_5  \\
        &= (15\,\text{cm})(+\hat{u}) + (56\,\text{cm})(-\hat{u}) + (3\,\text{cm})(+\hat{u}) + (25\,\text{cm})(+\hat{u}) + (19\,\text{cm})(-\hat{u})  \\
        &= (15 - 56 + 3 + 25 - 19)\,\text{cm}\,\hat{u} = -32\,\text{cm}\,\hat{u}
    .\end{align*}
    In this calculation, we use the distributive law. The result reads that the total displacement vector points away from the 100-cm mark (initial landing site) toward the end of the meter stick that touches the wall. The end that touches the wall is marked 0 cm, so the final position of the ladybug is at the $(100 - 32)\,\text{cm} = 68\,\text{cm}$ mark.

    \bigbreak \noindent 
    \subsection{Algebra of Vectors in Two Dimensions}
    \bigbreak \noindent 
    When vectors lie in a plane—that is, when they are in two dimensions—they can be multiplied by scalars, added to other vectors, or subtracted from other vectors in accordance with the general laws, However, the addition rule for two vectors in a plane becomes more complicated than the rule for vector addition in one dimension. We have to use the laws of geometry to construct resultant vectors, followed by trigonometry to find vector magnitudes and directions. 
    \bigbreak \noindent 
    For a geometric construction of the sum of two vectors in a plane, we follow the \textbf{parallelogram rule}. Suppose two vectors $\vec{A}$ and $\vec{B}$ are at the arbitrary positions shown in Figure 2.10. Translate either one of them in parallel to the beginning of the other vector, so that after the translation, both vectors have their origins at the same point. Now, at the end of vector $\vec{A}$ we draw a line parallel to vector $\vec{B}$ and at the end of vector $\vec{B}$ we draw a line parallel to vector $\vec{A}$ (the dashed lines in Figure 2.10). In this way, we obtain a parallelogram. From the origin of the two vectors we draw a diagonal that is the resultant $\vec{R}$ of the two vectors: $\vec{R} = \vec{A} + \vec{B}$ (Figure 2.10(a)). The other diagonal of this parallelogram is the vector difference of the two vectors $\vec{D} = \vec{A} - \vec{B}$. Notice that the end of the difference vector is placed at the end of vector $\vec{A}$.
    \bigbreak \noindent 
    \fig{.8}{./figures/2.jpeg}
    \bigbreak \noindent 
    It follows from the parallelogram rule that neither the magnitude of the resultant vector nor the magnitude of the difference vector can be expressed as a simple sum or difference of magnitudes A and B, because the length of a diagonal cannot be expressed as a simple sum of side lengths.
    \bigbreak \noindent 
    Drawing the resultant vector of many vectors can be generalized by using the following tail-to-head geometric construction. Suppose we want to draw the resultant vector $\vec{R}$ of four vectors $\vec{A}$, $\vec{B}$, $\vec{C}$, and $\vec{D}$ (Figure 2.12(a)). We select any one of the vectors as the first vector and make a parallel translation of a second vector to a position where the origin (“tail”) of the second vector coincides with the end (“head”) of the first vector. Then, we select a third vector and make a parallel translation of the third vector to a position where the origin of the third vector coincides with the end of the second vector. We repeat this procedure until all the vectors are in a head-to-tail arrangement like the one shown in Figure 2.12. We draw the resultant vector $\vec{R}$ by connecting the origin (“tail”) of the first vector with the end (“head”) of the last vector. The end of the resultant vector is at the end of the last vector. Because the addition of vectors is associative and commutative, we obtain the same resultant vector regardless of which vector we choose to be first, second, third, or fourth in this construction.
    \bigbreak \noindent 
    \fig{.8}{./figures/3.jpeg}




    
    





    \pagebreak 
    \unsect{2.2 Coordinate Systems and Components of a Vector}
    \bigbreak \noindent 
    In a rectangular (Cartesian) $xy$-coordinate system in a plane, a point in a plane is described by a pair of coordinates $(x, y)$. In a similar fashion, a vector $\vec{A}$ in a plane is described by a pair of its vector coordinates. The $x$-coordinate of vector $\vec{A}$ is called its $x$-component and the $y$-coordinate of vector $\vec{A}$ is called its $y$-component. The vector $x$-component is a vector denoted by $\vec{A}_x$. The vector $y$-component is a vector denoted by $\vec{A}_y$. In the Cartesian system, the $x$ and $y$ vector components of a vector are the orthogonal projections, as illustrated in Figure 2.16, of this vector onto the $x$- and $y$-axes, respectively. In this way, following the parallelogram rule for vector addition, each vector on a Cartesian plane can be expressed as the vector sum of its vector components:
    \begin{align*}
        \vec{A} = \vec{A_{x}} + \vec{A_{y}}
    .\end{align*}
    \bigbreak \noindent 
    \fig{.8}{./figures/3.jpeg}
    \bigbreak \noindent 
    It is customary to denote the positive direction on the $x$-axis by the unit vector $\hat{i}$ and the positive direction on the $y$-axis by the unit vector $\hat{j}$. Unit vectors of the axes, $\hat{i}$ and $\hat{j}$, define two orthogonal directions in the plane. As shown in Figure 2.16, the $x$- and $y$- components of a vector can now be written in terms of the unit vectors of the axes:
       \begin{equation}
            \begin{cases}
                \vec{A}_x &= A_x \hat{i}, \\
                \vec{A}_y &= A_y \hat{j}.
            \end{cases}
        \end{equation}
    \bigbreak \noindent 
    The vectors $\vec{A}_x$ and $\vec{A}_y$ defined by Equation 2.11 are the vector components of vector $\vec{A}$. The numbers $A_x$ and $A_y$ that define the vector components in Equation 2.11 are the scalar components of vector $\vec{A}$. Combining Equation 2.10 with Equation 2.11, we obtain the component form of a vector:
    \[
        \vec{A} = A_x \hat{i} + A_y \hat{j}.
    \]
    \bigbreak \noindent 
    If we know the coordinates $b(x_b, y_b)$ of the origin point of a vector (where $b$ stands for "beginning") and the coordinates $e(x_e, y_e)$ of the end point of a vector (where $e$ stands for "end"), we can obtain the scalar components of a vector simply by subtracting the origin point coordinates from the end point coordinates:
    \begin{align*}
        A_x &= x_e - x_b  \\
        A_y &= y_e - y_b
    .\end{align*}

    \pagebreak 
    \begin{exm}[Displacement of a Mouse Pointer]
        A mouse pointer on the display monitor of a computer at its initial position is at point $(6.0\, \text{cm}, 1.6\, \text{cm})$ with respect to the lower left-side corner. If you move the pointer to an icon located at point $(2.0\, \text{cm}, 4.5\, \text{cm})$, what is the displacement vector of the pointer?
    \end{exm}
    \bigbreak \noindent 
    \textcolor{red}{\textit{Solution.}}
    We identify $x_b = 6.0$, $x_e = 2.0$, $y_b = 1.6$, and $y_e = 4.5$, where the physical unit is $1\, \text{cm}$. The scalar $x$- and $y$-components of the displacement vector are
    \begin{align*}
        &D_x = x_e - x_b = (2.0 - 6.0)\, \text{cm} = -4.0\, \text{cm}, \\
        &D_y = y_e - y_b = (4.5 - 1.6)\, \text{cm} = +2.9\, \text{cm}.
    .\end{align*}
    The vector component form of the displacement vector is
    \[
        \vec{D} = D_x \hat{i} + D_y \hat{j} = (-4.0\, \text{cm}) \hat{i} + (2.9\, \text{cm}) \hat{j} = (-4.0 \hat{i} + 2.9 \hat{j})\, \text{cm}.
    \]
    \bigbreak \noindent 
    \fig{.8}{./figures/4.jpeg}
    \bigbreak \noindent 
    The vector $x$-component $\vec{D}_x = -4.0 \hat{i} = 4.0(-\hat{i})$ of the displacement vector has the magnitude $\lVert \vec{D}_x \rVert = \lvert -4.0 \rvert \lVert \hat{i} \rVert = 4.0$ because the magnitude of the unit vector is $\lVert \hat{i} \rVert = 1$. Notice, too, that the direction of the $x$-component is $-\hat{i}$, which is antiparallel to the direction of the $+x$-axis; hence, the $x$-component vector $\vec{D}_x$ points to the left, as shown in Figure 2.17. The scalar $x$-component of vector $\vec{D}$ is $D_x = -4.0$.
    \bigbreak \noindent 
    Similarly, the vector $y$-component $\vec{D}_y = +2.9 \hat{j}$ of the displacement vector has magnitude $\lVert \vec{D}_y \rVert = \lvert 2.9 \rvert \lVert \hat{j} \rVert = 2.9$ because the magnitude of the unit vector is $\lVert \hat{j} \rVert = 1$. The direction of the $y$-component is $+\hat{j}$, which is parallel to the direction of the $+y$-axis. Therefore, the $y$-component vector $\vec{D}_y$ points up, as seen in Figure 2.17. The scalar $y$-component of vector $\vec{D}$ is $D_y = +2.9$. The displacement vector $\vec{D}$ is the resultant of its two vector components.
    \bigbreak \noindent 
    The vector component form of the displacement vector Equation 2.14 tells us that the mouse pointer has been moved on the monitor $4.0\, \text{cm}$ to the left and $2.9\, \text{cm}$ upward from its initial position.

    \pagebreak \bigbreak \noindent 
    When the vector lies either in the first quadrant or in the fourth quadrant, where component $A_x$ is positive, the angle $\theta$ is identical to the direction angle $\theta_A$. For vectors in the fourth quadrant, angle $\theta$ is negative, which means that for these vectors, direction angle $\theta_A$ is measured clockwise from the positive $x$-axis. Similarly, for vectors in the second quadrant, angle $\theta$ is negative. 
    \bigbreak \noindent 
    When the vector lies in either the second or third quadrant, where component $A_x$ is negative, the direction angle is $\theta_A = \theta + 180^\circ$.
    \bigbreak \noindent 
    \fig{.8}{./figures/5.jpeg}
    \bigbreak \noindent 
    \begin{exm}[Magnitude and Direction of the Displacement Vector]
        In the previous example, we found the displacement vector $\vec{D}$ of the mouse pointer. We identify its scalar components $D_x = -4.0\, \text{cm}$ and $D_y = +2.9\, \text{cm}$ and substitute into the equations mentioned above to find the magnitude $D$ and direction $\theta_D$, respectively.
    \end{exm}
    \bigbreak \noindent 
    \textcolor{red}{\textit{Solution.}}
    The magnitude of vector $\vec{D}$ is
    \[
        D = \sqrt{D_x^2 + D_y^2} = \sqrt{(-4.0\, \text{cm})^2 + (2.9\, \text{cm})^2} = \sqrt{(4.0)^2 + (2.9)^2}\, \text{cm} = 4.9\, \text{cm}.
    \]
    The direction angle is
    \[
        \tan \theta = \frac{D_y}{D_x} = \frac{+2.9\, \text{cm}}{-4.0\, \text{cm}} = -0.725 \Rightarrow \theta = \tan^{-1}(-0.725) = -35.9^\circ.
    \]
    Vector $\vec{D}$ lies in the second quadrant, so its direction angle is
    \[
        \theta_D = \theta + 180^\circ = -35.9^\circ + 180^\circ = 144.1^\circ.
    \]
    \bigbreak \noindent 
    The quotient of the adjacent side $A_x$ to the hypotenuse $A$ is the cosine function of direction angle $\theta_A$, $A_x/A = \cos \theta_A$, and the quotient of the opposite side $A_y$ to the hypotenuse $A$ is the sine function of $\theta_A$, $A_y/A = \sin \theta_A$. When magnitude $A$ and direction $\theta_A$ are known, we can solve these relations for the scalar components:
       \begin{equation}
            \begin{cases}
                &A_{x} = A\cos{\theta_{A}} \\
                &A_{y} = A\sin{\theta_{A}} \\
            \end{cases}
        \end{equation}

        \pagebreak \bigbreak \noindent 
        \nt{
            When calculating vector components with this equation, care must be taken with the angle. The direction angle  θA
            of a vector is the angle measured counterclockwise from the positive direction on the x-axis to the vector. The clockwise measurement gives a negative angle.
        }


        \bigbreak \noindent 
        \subsection{Polar Coordinates}
        \bigbreak \noindent 
        To describe locations of points or vectors in a plane, we need two orthogonal directions. In the Cartesian coordinate system these directions are given by unit vectors $\hat{i}$ and $\hat{j}$ along the $x$-axis and the $y$-axis, respectively. The Cartesian coordinate system is very convenient to use in describing displacements and velocities of objects and the forces acting on them. However, it becomes cumbersome when we need to describe the rotation of objects. When describing rotation, we usually work in the polar coordinate system.
        \bigbreak \noindent 
        In the polar coordinate system, the location of point $P$ in a plane is given by two polar coordinates (Figure 2.20). The first polar coordinate is the radial coordinate $r$, which is the distance of point $P$ from the origin. The second polar coordinate is an angle $\varphi$ that the radial vector makes with some chosen direction, usually the positive $x$-direction. In polar coordinates, angles are measured in radians, or rads. The radial vector is attached at the origin and points away from the origin to point $P$. This radial direction is described by a unit radial vector $\hat{r}$. The second unit vector $\hat{t}$ is a vector orthogonal to the radial direction $\hat{r}$. The positive $+\hat{t}$ direction indicates how the angle $\varphi$ changes in the counterclockwise direction. In this way, a point $P$ that has coordinates $(x, y)$ in the rectangular system can be described equivalently in the polar coordinate system by the two polar coordinates $(r,\varphi)$. Equation 2.17 is valid for any vector, so we can use it to express the $x$- and $y$-coordinates of vector $\vec{r}$. In this way, we obtain the connection between the polar coordinates and rectangular coordinates of point $P$.
        \begin{align*}
            &x = r\cos{\phi} \\
            &y = r\sin{\phi}
        .\end{align*}

        \pagebreak \bigbreak \noindent 
        \subsection{Vectors in Three Dimensions}
        \bigbreak \noindent 
        To specify the location of a point in space, we need three coordinates $(x, y, z)$, where coordinates $x$ and $y$ specify locations in a plane, and coordinate $z$ gives a vertical position above or below the plane. Three-dimensional space has three orthogonal directions, so we need not two but three unit vectors to define a three-dimensional coordinate system. In the Cartesian coordinate system, the first two unit vectors are the unit vector of the $x$-axis $\hat{i}$ and the unit vector of the $y$-axis $\hat{j}$. The third unit vector $\hat{k}$ is the direction of the $z$-axis (Figure 2.21). The order in which the axes are labeled, which is the order in which the three unit vectors appear, is important because it defines the orientation of the coordinate system. The order $x$-$y$-$z$, which is equivalent to the order $\hat{i}$ - $\hat{j}$ - $\hat{k}$, defines the standard right-handed coordinate system (positive orientation). Notice that while previously the $y$-axis was used for the upward direction, in three dimensions we usually choose $z$ to represent the vertical. A right-handed coordinate system with $z$ up and with one of the remaining axes pointing to the right and the other out of the page must have the $y$-axis pointing right and $x$ out of the page, as illustrated here.
        \fig{.8}{./figures/}
        \bigbreak \noindent 
        In three-dimensional space, vector $\vec{A}$ has three vector components: the $x$-component $\vec{A}_x = A_x \hat{i}$, which is the part of vector $\vec{A}$ along the $x$-axis; the $y$-component $\vec{A}_y = A_y \hat{j}$, which is the part of $\vec{A}$ along the $y$-axis; and the $z$-component $\vec{A}_z = A_z \hat{k}$, which is the part of the vector along the $z$-axis. A vector in three-dimensional space is the vector sum of its three vector component. 
        \[
            \vec{A} = A_x \hat{i} + A_y \hat{j} + A_z \hat{k}.
        \]
        If we know the coordinates of its origin $b(x_b, y_b, z_b)$ and of its end $e(x_e, y_e, z_e)$, its scalar components are obtained by taking their differences: $A_x$ and $A_y$ are given by the equations mentioned previously, and the $z$-component is given by
        \[
            A_z = z_e - z_b.
        \]
        Magnitude $A$ is obtained by generalizing magnitude formula in two dimensions to three dimensions:
        \[
            A = \sqrt{A_x^2 + A_y^2 + A_z^2}.
        \]






    


    





    \pagebreak 
    \unsect{2.3 Algebra of Vectors}


    \pagebreak 
    \unsect{2.4 Products of Vectors}



\end{document}
