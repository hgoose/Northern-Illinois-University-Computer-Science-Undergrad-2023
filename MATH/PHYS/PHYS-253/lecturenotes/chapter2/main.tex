\documentclass{report}

\input{~/dev/latex/template/preamble.tex}
\input{~/dev/latex/template/macros.tex}

\title{\Huge{}}
\author{\huge{Nathan Warner}}
\date{\huge{}}
\fancyhf{}
\rhead{}
\fancyhead[R]{\itshape Warner} % Left header: Section name
\fancyhead[L]{\itshape\leftmark}  % Right header: Page number
\cfoot{\thepage}
\renewcommand{\headrulewidth}{0pt} % Optional: Removes the header line
%\pagestyle{fancy}
%\fancyhf{}
%\lhead{Warner \thepage}
%\rhead{}
% \lhead{\leftmark}
%\cfoot{\thepage}
%\setborder
% \usepackage[default]{sourcecodepro}
% \usepackage[T1]{fontenc}

% Change the title
\hypersetup{
    pdftitle={Units and Measurement}
}

\begin{document}
    % \maketitle
        \begin{titlepage}
       \begin{center}
           \vspace*{1cm}
    
           \textbf{Chapter II} \\ 
            Vectors
    
           \vspace{0.5cm}
            
                
           \vspace{1.5cm}
    
           \textbf{Nathan Warner}
    
           \vfill
                
                
           \vspace{0.8cm}
         
           \includegraphics[width=0.4\textwidth]{~/niu/seal.png}
                
           Computer Science \\
           Northern Illinois University\\
           January 22, 2023 \\
           United States\\
           
                
       \end{center}
    \end{titlepage}
    \tableofcontents
    \pagebreak 
    \unsect{2.1 Scalars and Vectors}
    \bigbreak \noindent 
    Many familiar physical quantities can be specified completely by giving a single number and the appropriate unit. For example, “a class period lasts 50 min” or “the gas tank in my car holds 65 L” or “the distance between two posts is 100 m.” A physical quantity that can be specified completely in this manner is called a \textbf{scalar quantity}. Scalar is a synonym of “number.” Time, mass, distance, length, volume, temperature, and energy are examples of \textbf{scalar quantities}.
    \bigbreak \noindent 
    Physical quantities specified completely by giving a number of units (magnitude) and a direction are called \textbf{vector quantities}. Examples of vector quantities include displacement, velocity, position, force, and torque.
    \bigbreak \noindent 
    In the language of mathematics, physical vector quantities are represented by mathematical objects called \textbf{vectors}
    \bigbreak \noindent 
    We can add or subtract two vectors, and we can multiply a vector by a scalar or by another vector, but we cannot divide by a vector. The operation of division by a vector is not defined.
    \bigbreak \noindent 
    to distinguish between a vector and a scalar quantity, we adopt the common convention that a letter in bold type with an arrow above it denotes a vector, and a letter without an arrow denotes a scalar. For example, a distance of 2.0 km, which is a scalar quantity, is denoted by d = 2.0 km, whereas a displacement of 2.0 km in some direction, which is a vector quantity, is denoted by $\vec{\textbf{d}}$




    \pagebreak 
    \unsect{2.2 Coordinate Systems and Components of a Vector}

    \pagebreak 
    \unsect{2.3 Algebra of Vectors}


    \pagebreak 
    \unsect{2.4 Products of Vectors}



\end{document}
