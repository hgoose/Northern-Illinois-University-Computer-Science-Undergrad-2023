\documentclass{report}

\input{~/dev/latex/template/preamble.tex}
\input{~/dev/latex/template/macros.tex}

\title{\Huge{}}
\author{\huge{Nathan Warner}}
\date{\huge{}}
\fancyhf{}
\rhead{}
\fancyhead[R]{\itshape Warner} % Left header: Section name
\fancyhead[L]{\itshape\leftmark}  % Right header: Page number
\cfoot{\thepage}
\renewcommand{\headrulewidth}{0pt} % Optional: Removes the header line
%\pagestyle{fancy}
%\fancyhf{}
%\lhead{Warner \thepage}
%\rhead{}
% \lhead{\leftmark}
%\cfoot{\thepage}
%\setborder
% \usepackage[default]{sourcecodepro}
% \usepackage[T1]{fontenc}

% Change the title
\hypersetup{
    pdftitle={Units and Measurement}
}

\begin{document}
    % \maketitle
        \begin{titlepage}
       \begin{center}
           \vspace*{1cm}
    
           \textbf{Chapter I} \\ 
            Units and Measurement
    
           \vspace{0.5cm}
            
                
           \vspace{1.5cm}
    
           \textbf{Nathan Warner}
    
           \vfill
                
                
           \vspace{0.8cm}
         
           \includegraphics[width=0.4\textwidth]{~/niu/seal.png}
                
           Computer Science \\
           Northern Illinois University\\
           January 22, 2023 \\
           United States\\
           
                
       \end{center}
    \end{titlepage}
    \tableofcontents
    \pagebreak 
    \unsect{1.1 The Scope and Scale of Physics}
    \bigbreak \noindent 
    \subsection{Order of magnitude}
    The order of magnitude of a number is the power of 10 that most closely approximates it. Thus, the order of magnitude refers to the scale (or size) of a value. Each power of 10 represents a different order of magnitude. For example, \( 10^1, 10^2, 10^3, \) and so forth, are all different orders of magnitude, as are \( 10^0 = 1, 10^{-1}, 10^{-2}, \) and \( 10^{-3} \). To find the order of magnitude of a number, take the base-10 logarithm of the number and round it to the nearest integer, then the order of magnitude of the number is simply the resulting power of 10. For example, the order of magnitude of 800 is \( 10^3 \) because \( \log_{10}800 \approx 2.903, \) which rounds to 3. Similarly, the order of magnitude of 450 is \( 10^3 \) because \( \log_{10}450 \approx 2.653, \) which rounds to 3 as well. Thus, we say the numbers 800 and 450 are of the same order of magnitude: \( 10^3 \). However, the order of magnitude of 250 is \( 10^2 \) because \( \log_{10}250 \approx 2.397, \) which rounds to 2.
    \bigbreak \noindent 
    An equivalent but quicker way to find the order of magnitude of a number is first to write it in scientific notation and then check to see whether the first factor is greater than or less than \( \sqrt{10} = 10^{0.5} \approx 3 \). The idea is that \( \sqrt{10} = 10^{0.5} \) is halfway between \( 1 = 10^0 \) and \( 10 = 10^1 \) on a log base-10 scale. Thus, if the first factor is less than \( \sqrt{10} \), then we round it down to 1 and the order of magnitude is simply whatever power of 10 is required to write the number in scientific notation. On the other hand, if the first factor is greater than \( \sqrt{10} \), then we round it up to 10 and the order of magnitude is one power of 10 higher than the power needed to write the number in scientific notation. For example, the number 800 can be written in scientific notation as \( 8 \times 10^2 \). Because 8 is bigger than \( \sqrt{10} \approx 3 \), we say the order of magnitude of 800 is \( 10^{2+1} = 10^3 \). The number 450 can be written as \( 4.5 \times 10^2 \), so its order of magnitude is also \( 10^3 \) because 4.5 is greater than 3. However, 250 written in scientific notation is \( 2.5 \times 10^2 \) and 2.5 is less than 3, so its order of magnitude is \( 10^2 \).

    \pagebreak 
    \subsection{Known ranges of length, mass, and time}
    \bigbreak \noindent 
    The vastness of the universe and the breadth over which physics applies are illustrated by the wide range of examples of known lengths, masses, and times (given as orders of magnitude) in Figure 1. Examining this table will give you a feeling for the range of possible topics in physics and numerical values.
    \bigbreak \noindent 
    \fig{.8}{./figures/1.jpeg}

    \pagebreak 
    \unsect{1.2 Units and Standards}
    \bigbreak \noindent 
    Two major systems of units are used in the world: SI units (for the French Système International d’Unités), also known as the metric system, and English units (also known as the customary or imperial system). English units were historically used in nations once ruled by the British Empire and are still widely used in the United States. English units may also be referred to as the foot–pound–second (fps) system, as opposed to the centimeter–gram–second (cgs) system. You may also encounter the term SAE units, named after the Society of Automotive Engineers. Products such as fasteners and automotive tools (for example, wrenches) that are measured in inches rather than metric units are referred to as SAE fasteners or SAE wrenches.




    
\end{document}
