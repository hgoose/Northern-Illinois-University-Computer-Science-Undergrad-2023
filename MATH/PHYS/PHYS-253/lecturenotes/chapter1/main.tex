\documentclass{report}

\input{~/dev/latex/template/preamble.tex}
\input{~/dev/latex/template/macros.tex}

\title{\Huge{}}
\author{\huge{Nathan Warner}}
\date{\huge{}}
\fancyhf{}
\rhead{}
\fancyhead[R]{\itshape Warner} % Left header: Section name
\fancyhead[L]{\itshape\leftmark}  % Right header: Page number
\cfoot{\thepage}
\renewcommand{\headrulewidth}{0pt} % Optional: Removes the header line
%\pagestyle{fancy}
%\fancyhf{}
%\lhead{Warner \thepage}
%\rhead{}
% \lhead{\leftmark}
%\cfoot{\thepage}
%\setborder
% \usepackage[default]{sourcecodepro}
% \usepackage[T1]{fontenc}

% Change the title
\hypersetup{
    pdftitle={Units and Measurement}
}

\begin{document}
    % \maketitle
        \begin{titlepage}
       \begin{center}
           \vspace*{1cm}
    
           \textbf{Chapter I} \\ 
            Units and Measurement
    
           \vspace{0.5cm}
            
                
           \vspace{1.5cm}
    
           \textbf{Nathan Warner}
    
           \vfill
                
                
           \vspace{0.8cm}
         
           \includegraphics[width=0.4\textwidth]{~/niu/seal.png}
                
           Computer Science \\
           Northern Illinois University\\
           January 22, 2023 \\
           United States\\
           
                
       \end{center}
    \end{titlepage}
    \tableofcontents
    \pagebreak 
    \unsect{1.1 The Scope and Scale of Physics}
    \bigbreak \noindent 
    \subsection{Order of magnitude}
    The order of magnitude of a number is the power of 10 that most closely approximates it. Thus, the order of magnitude refers to the scale (or size) of a value. Each power of 10 represents a different order of magnitude. For example, \( 10^1, 10^2, 10^3, \) and so forth, are all different orders of magnitude, as are \( 10^0 = 1, 10^{-1}, 10^{-2}, \) and \( 10^{-3} \). To find the order of magnitude of a number, take the base-10 logarithm of the number and round it to the nearest integer, then the order of magnitude of the number is simply the resulting power of 10. For example, the order of magnitude of 800 is \( 10^3 \) because \( \log_{10}800 \approx 2.903, \) which rounds to 3. Similarly, the order of magnitude of 450 is \( 10^3 \) because \( \log_{10}450 \approx 2.653, \) which rounds to 3 as well. Thus, we say the numbers 800 and 450 are of the same order of magnitude: \( 10^3 \). However, the order of magnitude of 250 is \( 10^2 \) because \( \log_{10}250 \approx 2.397, \) which rounds to 2.
    \bigbreak \noindent 
    An equivalent but quicker way to find the order of magnitude of a number is first to write it in scientific notation and then check to see whether the first factor is greater than or less than \( \sqrt{10} = 10^{0.5} \approx 3 \). The idea is that \( \sqrt{10} = 10^{0.5} \) is halfway between \( 1 = 10^0 \) and \( 10 = 10^1 \) on a log base-10 scale. Thus, if the first factor is less than \( \sqrt{10} \), then we round it down to 1 and the order of magnitude is simply whatever power of 10 is required to write the number in scientific notation. On the other hand, if the first factor is greater than \( \sqrt{10} \), then we round it up to 10 and the order of magnitude is one power of 10 higher than the power needed to write the number in scientific notation. For example, the number 800 can be written in scientific notation as \( 8 \times 10^2 \). Because 8 is bigger than \( \sqrt{10} \approx 3 \), we say the order of magnitude of 800 is \( 10^{2+1} = 10^3 \). The number 450 can be written as \( 4.5 \times 10^2 \), so its order of magnitude is also \( 10^3 \) because 4.5 is greater than 3. However, 250 written in scientific notation is \( 2.5 \times 10^2 \) and 2.5 is less than 3, so its order of magnitude is \( 10^2 \).

    \pagebreak 
    \subsection{Known ranges of length, mass, and time}
    \bigbreak \noindent 
    The vastness of the universe and the breadth over which physics applies are illustrated by the wide range of examples of known lengths, masses, and times (given as orders of magnitude) in Figure 1. Examining this table will give you a feeling for the range of possible topics in physics and numerical values.
    \bigbreak \noindent 
    \fig{.8}{./figures/1.jpeg}

    \pagebreak 
    \unsect{1.2 Units and Standards}
    \bigbreak \noindent 
    Two major systems of units are used in the world: SI units (for the French Système International d’Unités), also known as the metric system, and English units (also known as the customary or imperial system). English units were historically used in nations once ruled by the British Empire and are still widely used in the United States. English units may also be referred to as the foot–pound–second (fps) system, as opposed to the centimeter–gram–second (cgs) system. You may also encounter the term SAE units, named after the Society of Automotive Engineers. Products such as fasteners and automotive tools (for example, wrenches) that are measured in inches rather than metric units are referred to as SAE fasteners or SAE wrenches.

    \bigbreak \noindent 
    \subsection{SI Units: Base and Derived Units}
    \bigbreak \noindent 
    In any system of units, the units for some physical quantities must be defined through a measurement process. These are called the base quantities for that system and their units are the system’s base units. All other physical quantities can then be expressed as algebraic combinations of the base quantities. Each of these physical quantities is then known as a derived quantity and each unit is called a derived unit. The choice of base quantities is somewhat arbitrary, as long as they are independent of each other and all other quantities can be derived from them. Typically, the goal is to choose physical quantities that can be measured accurately to a high precision as the base quantities. The reason for this is simple. Since the derived units can be expressed as algebraic combinations of the base units, they can only be as accurate and precise as the base units from which they are derived.
    \bigbreak \noindent 
    \begin{tabularx}{\textwidth}{|X|X|}
        \hline
        ISQ Base Quantity & SI Base Unit \\
        \hline
        Length & meter (m) \\
        \hline
        Mass & kilogram (kg) \\
        \hline
        Time & second (s) \\
        \hline
        Electrical current & ampere (A) \\
        \hline
        Thermodynamic temperature & kelvin (K) \\
        \hline
        Amount of substance & mole (mol) \\
        \hline
        Luminous intensity & candela (cd) \\
        \hline
    \end{tabularx}
    \bigbreak \noindent 
    \begin{center}
        \textit{Table 1}
    \end{center}
    \bigbreak \noindent 
    Derived quantities are formed from the base quantities listed in Table 1. Area, calculated as the product of two lengths, is expressed in square meters (m$^2$). Volume is in cubic meters (m$^3$), speed in meters per second (m/s), and density in kilograms per cubic meter (kg/m$^3$). Angles, measured in radians, are the ratio of arc length to radius in a circle. Other derived quantities include mass flow rate (kg/s), volume flow rate (m$^3$/s), electric charge (A$\cdot$s), and mass flux density [kg/(m$^2$$\cdot$s)]. All physical quantities and units can be derived from the seven base quantities and SI base units.

    \pagebreak 
    \subsection{Units of Time, Length, and Mass: The Second, Meter, and Kilogram}
    \bigbreak \noindent 
    \subsubsection{The second}
    \bigbreak \noindent 
    The SI unit for time, the second (abbreviated s), has a long history. For many years it was defined as 1/86,400 of a mean solar day. More recently, a new standard was adopted to gain greater accuracy and to define the second in terms of a nonvarying or constant physical phenomenon (because the solar day is getting longer as a result of the very gradual slowing of Earth’s rotation). Cesium atoms can be made to vibrate in a very steady way, and these vibrations can be readily observed and counted. In 1967, the second was redefined as the time required for 9,192,631,770 of these vibrations to occur. Note that this may seem like more precision than you would ever need, but it isn’t—GPSs rely on the precision of atomic clocks to be able to give you turn-by-turn directions on the surface of Earth, far from the satellites broadcasting their location.
    \bigbreak \noindent 
    \subsubsection{The meter}
    \bigbreak \noindent 
    The SI unit for length is the meter (abbreviated m); its definition has also changed over time to become more precise. The meter was first defined in 1791 as 1/10,000,000 of the distance from the equator to the North Pole. This measurement was improved in 1889 by redefining the meter to be the distance between two engraved lines on a platinum–iridium bar now kept near Paris. By 1960, it had become possible to define the meter even more accurately in terms of the wavelength of light, so it was again redefined as 1,650,763.73 wavelengths of orange light emitted by krypton atoms. In 1983, the meter was given its current definition (in part for greater accuracy) as the distance light travels in a vacuum in 1/299,792,458 of a second . This change came after knowing the speed of light to be exactly 299,792,458 m/s.
    \bigbreak \noindent 
    \subsubsection{The kilogram}
    \bigbreak \noindent 
    The SI unit for mass is the kilogram (abbreviated kg); From 1795–2018 it was defined to be the mass of a platinum–iridium cylinder kept with the old meter standard at the International Bureau of Weights and Measures near Paris. However, this cylinder has lost roughly 50 micrograms since it was created. Because this is the standard, this has shifted how we defined a kilogram. Therefore, a new definition was adopted in May 2019 based on the Planck constant and other constants which will never change in value. We will study Planck’s constant in quantum mechanics, which is an area of physics that describes how the smallest pieces of the universe work. The kilogram is measured on a Kibble balance (see Figure 1.10). When a weight is placed on a Kibble balance, an electrical current is produced that is proportional to Planck’s constant. Since Planck’s constant is defined, the exact current measurements in the balance define the kilogram.

    \pagebreak 
    \subsection{Metric Prefixes}
    \bigbreak \noindent 
    SI units are part of the metric system, which is convenient for scientific and engineering calculations because the units are categorized by factors of 10. Table 1.2 lists the metric prefixes and symbols used to denote various factors of 10 in SI units. For example, a centimeter is one-hundredth of a meter (in symbols, \(1 \, \text{cm} = 10^{-2} \, \text{m}\)) and a kilometer is a thousand meters (\(1 \, \text{km} = 10^{3} \, \text{m}\)). Similarly, a megagram is a million grams (\(1 \, \text{Mg} = 10^{6} \, \text{g}\)), a nanosecond is a billionth of a second (\(1 \, \text{ns} = 10^{-9} \, \text{s}\)), and a terameter is a trillion meters (\(1 \, \text{Tm} = 10^{12} \, \text{m}\)).
    \bigbreak \noindent 
    \begin{center}
        \begin{tabular}{@{}llllll@{}}
            \toprule
            Prefix & Symbol & Meaning & Prefix & Symbol & Meaning \\ 
            \midrule
            yotta- & Y & $10^{24}$ & yocto- & y & $10^{-24}$ \\
            zetta- & Z & $10^{21}$ & zepto- & z & $10^{-21}$ \\
            exa-   & E & $10^{18}$ & atto-  & a & $10^{-18}$ \\
            peta-  & P & $10^{15}$ & femto- & f & $10^{-15}$ \\
            tera-  & T & $10^{12}$ & pico-  & p & $10^{-12}$ \\
            giga-  & G & $10^9$   & nano-  & n & $10^{-9}$  \\
            mega-  & M & $10^6$   & micro- & $\mu$ & $10^{-6}$ \\
            kilo-  & k & $10^3$   & milli- & m & $10^{-3}$ \\
            hecto- & h & $10^2$   & centi- & c & $10^{-2}$ \\
            deka-  & da & $10^1$  & deci-  & d & $10^{-1}$ \\
            \bottomrule
        \end{tabular}
    \end{center}
    \bigbreak \noindent 
    The only rule when using metric prefixes is that you cannot ``double them up.'' For example, if you have measurements in petameters (1 Pm = $10^{15}$ m), it is not proper to talk about megagigameters, although $10^6 \times 10^9 = 10^{15}$. In practice, the only time this becomes a bit confusing is when discussing masses. As we have seen, the base SI unit of mass is the kilogram (kg), but metric prefixes need to be applied to the gram (g), because we are not allowed to ``double-up'' prefixes. Thus, a thousand kilograms ($10^3$ kg) is written as a megagram (1 Mg) since
    \bigbreak \noindent 
    \[ 10^3 \, \text{kg} = 10^3 \times 10^3 \, \text{g} = 10^6 \, \text{g} = 1 \, \text{Mg}. \]
    \bigbreak \noindent 
    Incidentally, $10^3$ kg is also called a metric ton, abbreviated t. This is one of the units outside the SI system considered acceptable for use with SI units.
    \bigbreak \noindent 
    As we see in the next section, metric systems have the advantage that conversions of units involve only powers of 10. There are 100 cm in 1 m, 1000 m in 1 km, and so on. In nonmetric systems, such as the English system of units, the relationships are not as simple---there are 12 in. in 1 ft, 5280 ft in 1 mi, and so on.
    \bigbreak \noindent 
    Another advantage of metric systems is that the same unit can be used over extremely large ranges of values simply by scaling it with an appropriate metric prefix. The prefix is chosen by the order of magnitude of physical quantities commonly found in the task at hand. For example, distances in meters are suitable in construction, whereas distances in kilometers are appropriate for air travel, and nanometers are convenient in optical design. With the metric system there is no need to invent new units for particular applications. Instead, we rescale the units with which we are already familiar.

    \pagebreak 
    \unsect{1.3 Unit Conversion}
    \bigbreak \noindent 
    Suppose we have units in meters and we want to convert to kilometers. We need to determine a conversion factor relating meters to kilometers. A conversion factor is a ratio that expresses how many of one unit are equal to another unit. We know that there are 1000 m in 1 km. Now we can set up our unit conversion. We write the units we have and then multiply them by the conversion factor so the units cancel out 
    \begin{align*}
        80 \cancel{m} \times \frac{1km}{1000\cancel{m}} = 0.080km
    .\end{align*}
    \bigbreak \noindent 
    we can get the same answer just as easily by noting that
    \begin{align*}
        80 \, \text{m} = 8.0 \times 10^1 \, \text{m} = 8.0 \times 10^{-2} \, \text{km} = 0.080 \, \text{km},
    .\end{align*}
    \bigbreak \noindent 
    \begin{exm}
        The distance from the university to home is 10 mi and it usually takes 20 min to drive this distance. Calculate the average speed in meters per second (m/s). (Note: Average speed is distance traveled divided by time of travel.)
        \bigbreak \noindent 
        \textbf{Note:} There are 1609 meters in 1 mile
    \end{exm}
    \bigbreak \noindent 
    \textcolor{red}{\textit{Solution.}} 
    First, we can compute the average speed with the units given $\left(\frac{miles}{minute}\right)$
    \begin{align*}
        \text{Average Speed} = \frac{\text{miles}}{\text{minute}} = \frac{10}{20} = 0.5\ mi/min
    .\end{align*}
    \bigbreak \noindent 
    Now we simply convert to m/s
    \begin{align*}
        &\frac{0.5\ \cancel{mi}}{1\ \cancel{min}} \times \frac{1\ \cancel{min}}{60\ sec} \times \frac{1609\ m}{1\ \cancel{mi}} \\
        &\approx 13 m/s
    .\end{align*}
    \bigbreak \noindent 
    \begin{exm}
        The density of iron is  $7.86 g/cm^{3}$ under standard conditions. Convert this to $kg/m^{3}$.
    \end{exm}
    \bigbreak \noindent 
    \textcolor{red}{\textit{Solution.}}
    \begin{align*}
        &\frac{7.86\ g}{1\ cm^{3}} \times \left(\frac{100\ cm}{1\ m}\right)^{3} \times \frac{1\ kg}{1000\ g} \\
        &= \frac{7.86(100^{3})(1\ kg)}{1000(1\ m)} \\
        &=7.86 \cdot 10^{3} kg/m^{3}
    .\end{align*}

    \pagebreak 
    \unsect{Dimensional Analysis}
    \bigbreak \noindent 
    The dimension of any physical quantity expresses its dependence on the base quantities as a product of symbols (or powers of symbols) representing the base quantities.
    \bigbreak \noindent 
    \begin{center}
        \begin{tabular}{ll}
            \hline
            Base Quantity & Symbol for Dimension \\
            \hline
            Length & L \\
            Mass & M \\
            Time & T \\
            Current & I \\
            Thermodynamic temperature & $\Theta$ \\
            Amount of substance & N \\
            Luminous intensity & J \\
            \hline
        \end{tabular}
    \end{center}
    \bigbreak \noindent 
    For example, a measurement of length is said to have dimension \( L \) or \( L^1 \), a measurement of mass has dimension \( M \) or \( M^1 \), and a measurement of time has dimension \( T \) or \( T^1 \). Like units, dimensions obey the rules of algebra. Thus, area is the product of two lengths and so has dimension \( L^2 \), or length squared. Similarly, volume is the product of three lengths and has dimension \( L^3 \), or length cubed. Speed has dimension length over time, \( L/T \) or \( LT^{-1} \). Volumetric mass density has dimension \( M/L^3 \) or \( ML^{-3} \), or mass over length cubed. In general, the dimension of any physical quantity can be written as \( L^aM^bT^cI^d\Theta^eN^fJ^g \) for some powers \( a, b, c, d, e, f, \) and \( g \). We can write the dimensions of a length in this form with \( a=1 \) and the remaining six powers all set equal to zero: \( L^1 = L^1M^0T^0I^0\Theta^0N^0J^0 \). Any quantity with a dimension that can be written so that all seven powers are zero (that is, its dimension is \( L^0M^0T^0I^0\Theta^0N^0J^0 \)) is called dimensionless (or sometimes “of dimension 1,” because anything raised to the zero power is one). Physicists often call dimensionless quantities pure numbers.
    \bigbreak \noindent 
    Physicists often use square brackets around the symbol for a physical quantity to represent the dimensions of that quantity. For example, if \( r \) is the radius of a cylinder and \( h \) is its height, then we write \([r] = L\) and \([h] = L\) to indicate the dimensions of the radius and height are both those of length, or \( L \). Similarly, if we use the symbol \( A \) for the surface area of a cylinder and \( V \) for its volume, then \([A] = L^2\) and \([V] = L^3\). If we use the symbol \( m \) for the mass of the cylinder and \( \rho \) for the density of the material from which the cylinder is made, then \([m] = M\) and \([\rho] = M L^{-3}\).
    \bigbreak \noindent 
    The importance of the concept of dimension arises from the fact that any mathematical equation relating physical quantities must be \textbf{dimensionally consistent}
    \begin{itemize}
        \item Every term in an expression must have the same dimensions; it does not make sense to add or subtract quantities of differing dimension (think of the old saying: “You can’t add apples and oranges”). In particular, the expressions on each side of the equality in an equation must have the same dimensions.
        \item The arguments of any of the standard mathematical functions such as trigonometric functions (such as sine and cosine), logarithms, or exponential functions that appear in the equation must be dimensionless. These functions require pure numbers as inputs and give pure numbers as outputs.
    \end{itemize}
    \bigbreak \noindent 
    If either of these rules is violated, an equation is not dimensionally consistent and cannot possibly be a correct statement of physical law.
    \bigbreak \noindent 
    One further point that needs to be mentioned is the effect of the operations of calculus on dimensions. We have seen that dimensions obey the rules of algebra, just like units, but what happens when we take the derivative of one physical quantity with respect to another or integrate a physical quantity over another? The derivative of a function is just the slope of the line tangent to its graph and slopes are ratios, so for physical quantities \( v \) and \( t \), we have that the dimension of the derivative of \( v \) with respect to \( t \) is just the ratio of the dimension of \( v \) over that of \( t \):
    \[
        \left[ \frac{dv}{dt} \right] = \frac{[v]}{[t]}.
    \]
    Similarly, since integrals are just sums of products, the dimension of the integral of \( v \) with respect to \( t \) is simply the dimension of \( v \) times the dimension of \( t \):
    \[
        \left[ \int v \, dt \right] = [v] \cdot [t].
    \]
    By the same reasoning, analogous rules hold for the units of physical quantities derived from other quantities by integration or differentiation.

    \pagebreak 
    \unsect{1.5 Estimation and Fermi calulations}
    \bigbreak \noindent 
    To make some progress in estimating, you need to have some definite ideas about how variables may be related. The following strategies may help you in practicing the art of estimation:
    \bigbreak \noindent 
    \begin{itemize}
        \item \textbf{Get big lengths from smaller lengths.} When estimating lengths, remember that anything can be a ruler. Thus, imagine breaking a big thing into smaller things, estimate the length of one of the smaller things, and multiply to get the length of the big thing. For example, to estimate the height of a building, first count how many floors it has. Then, estimate how big a single floor is by imagining how many people would have to stand on each other’s shoulders to reach the ceiling. Last, estimate the height of a person. The product of these three estimates is your estimate of the height of the building. It helps to have memorized a few length scales relevant to the sorts of problems you find yourself solving. For example, knowing some of the length scales in Figure 1 might come in handy. Sometimes it also helps to do this in reverse—that is, to estimate the length of a small thing, imagine a bunch of them making up a bigger thing. For example, to estimate the thickness of a sheet of paper, estimate the thickness of a stack of paper and then divide by the number of pages in the stack. These same strategies of breaking big things into smaller things or aggregating smaller things into a bigger thing can sometimes be used to estimate other physical quantities, such as masses and times.
        \item \textbf{Get areas and volumes from lengths.} When dealing with an area or a volume of a complex object, introduce a simple model of the object such as a sphere or a box. Then, estimate the linear dimensions (such as the radius of the sphere or the length, width, and height of the box) first, and use your estimates to obtain the volume or area from standard geometric formulas. If you happen to have an estimate of an object’s area or volume, you can also do the reverse; that is, use standard geometric formulas to get an estimate of its linear dimensions.
        \item \textbf{Get masses from volumes and densities.} When estimating masses of objects, it can help first to estimate its volume and then to estimate its mass from a rough estimate of its average density (recall, density has dimension mass over length cubed, so mass is density times volume). For this, it helps to remember that the density of air is around 1 kg/m3, the density of water is 103 kg/m3, and the densest everyday solids max out at around 104 kg/m3. Asking yourself whether an object floats or sinks in either air or water gets you a ballpark estimate of its density. You can also do this the other way around; if you have an estimate of an object’s mass and its density, you can use them to get an estimate of its volume.
        \item \textbf{If all else fails, bound it.} For physical quantities for which you do not have a lot of intuition, sometimes the best you can do is think something like: Well, it must be bigger than this and smaller than that. For example, suppose you need to estimate the mass of a moose. Maybe you have a lot of experience with moose and know their average mass offhand. If so, great. But for most people, the best they can do is to think something like: It must be bigger than a person (of order 102 kg) and less than a car (of order 103 kg). If you need a single number for a subsequent calculation, you can take the geometric mean of the upper and lower bound—that is, you multiply them together and then take the square root. For the moose mass example, this would be
            \[
                (10^2 \times 10^3)^{0.5} = 10^{2.5} = 10^{0.5} \times 10^2 \approx 3 \times 10^2 \, \text{kg}.
            \]
        The tighter the bounds, the better. Also, no rules are unbreakable when it comes to estimation. If you think the value of the quantity is likely to be closer to the upper bound than the lower bound, then you may want to bump up your estimate from the geometric mean by an order or two of magnitude.
        \item \textbf{One “sig. fig.” is fine.} There is no need to go beyond one significant figure, or one digit in the coefficient of an expression in scientific notation, when doing calculations to obtain an estimate. In most cases, the order of magnitude is good enough. The goal is just to get in the ballpark figure, so keep the arithmetic as simple as possible.
        \item \textbf{Ask yourself: Does this make any sense?} Last, check to see whether your answer is reasonable. How does it compare with the values of other quantities with the same dimensions that you already know or can look up easily? If you get some wacky answer (for example, if you estimate the mass of the Atlantic Ocean to be bigger than the mass of Earth, or some time span to be longer than the age of the universe), first check to see whether your units are correct. Then, check for arithmetic errors. Then, rethink the logic you used to arrive at your answer. If everything checks out, you may have just proved that some slick new idea is actually bogus.
    \end{itemize}
    \bigbreak \noindent 

    \pagebreak 
    \unsect{Significant Figures}
    \bigbreak \noindent 
    Science is based on observation and experiment—that is, on measurements. \textbf{Accuracy} is how close a measurement is to the accepted reference value for that measurement. For example, let’s say we want to measure the length of standard printer paper. The packaging in which we purchased the paper states that it is 11.0 in. long. We then measure the length of the paper three times and obtain the following measurements: 11.1 in., 11.2 in., and 10.9 in. These measurements are quite accurate because they are very close to the reference value of 11.0 in. In contrast, if we had obtained a measurement of 12 in., our measurement would not be very accurate. Notice that the concept of accuracy requires that an accepted reference value be given.
    \bigbreak \noindent 
    The \textbf{precision} of measurements refers to how close the agreement is between repeated independent measurements (which are repeated under the same conditions).
    \bigbreak \noindent 
    \subsection{Accuracy, Precision, Uncertainty, and Discrepancy}
    \bigbreak \noindent 
    The precision of a measuring system is related to the \textbf{uncertainty} in the measurements whereas the accuracy is related to the \textbf{discrepancy} from the accepted reference value.
    \bigbreak \noindent 
    Discrepancy (or “measurement error”) is the difference between the measured value and a given standard or expected value.
    \bigbreak \noindent If the measurements are not very precise, then the uncertainty of the values is high. If the measurements are not very accurate, then the discrepancy of the values is high.
    \bigbreak \noindent 
    \subsection{Percent uncertainty}
    \bigbreak \noindent 
    Another method of expressing uncertainty is as a percent of the measured value. If a measurement \( A \) is expressed with uncertainty \( \delta A \), the percent uncertainty is defined as
    \[
    \text{Percent uncertainty} = \frac{\delta A}{A} \times 100\%.
    \]

    \pagebreak \bigbreak \noindent 
    \begin{exm}[Percent Uncertainty]
        A grocery store sells 5-lb bags of apples. Let’s say we purchase four bags during the course of a month and weigh the bags each time. We obtain the following measurements:
        \begin{item}
            \item Week 1 weight: 4.8 lb
            \item Week 2 weight: 5.3 lb
            \item Week 3 weight: 4.9 lb
            \item Week 4 weight: 5.4 lb
        \end{item}
        \bigbreak \noindent 
        We then determine the average weight of the 5-lb bag of apples is $5.1 \pm 0.3\ lb$ from using half of the range. What is the percent uncertainty of the bag’s weight?
    \end{exm}
    \bigbreak \noindent 
    \textcolor{red}{\textit{Solution.}} 
    First, observe that the average value of the bag’s weight, $A$, is 5.1 lb. The uncertainty in this value,  $\delta A$, is 0.3 lb. 
  \bigbreak \noindent 
  Substitute the values into the equation:
    \[
        \text{Percent uncertainty} = \frac{\delta A}{A} \times 100\% = \frac{0.3 \, \text{lb}}{5.1 \, \text{lb}} \times 100\% = 5.9\% \approx 6\%.
    \]
    \bigbreak \noindent 
    We can conclude the average weight of a bag of apples from this store is \(5.1 \, \text{lb} \pm 6\%\). Notice the percent uncertainty is dimensionless because the units of weight in \(\delta A = 0.2 \, \text{lb}\) canceled those in \(A = 5.1 \, \text{lb}\) when we took the ratio.

    \bigbreak \noindent 
    \subsubsection{Uncertainties in calculations}
    \bigbreak \noindent 
    If the measurements going into the calculation have small uncertainties (a few percent or less), then the method of adding percents can be used for multiplication or division. This method states the percent uncertainty in a quantity calculated by multiplication or division is the sum of the percent uncertainties in the items used to make the calculation. For example, if a floor has a length of \(4.00 \, \text{m}\) and a width of \(3.00 \, \text{m}\), with uncertainties of \(2\%\) and \(1\%\), respectively, then the area of the floor is \(12.0 \, \text{m}^2\) and has an uncertainty of \(3\%\). (Expressed as an area, this is \(0.36 \, \text{m}^2\) [\(12.0 \, \text{m}^2 \times 0.03\)], which we round to \(0.4 \, \text{m}^2\) since the area of the floor is given to a tenth of a square meter.)
    \bigbreak \noindent 
    \subsection{Significant figures in calculations}
    \bigbreak \noindent 
    When combining measurements with different degrees of precision with the mathematical operations of addition, subtraction, multiplication, or division, then the number of significant digits in the final answer can be no greater than the number of significant digits in the least-precise measured value. There are two different rules, one for multiplication and division and the other for addition and subtraction. There are two rules
    \begin{itemize}
        \item \textbf{For multiplication and division}, the result should have the same number of significant figures as the quantity with the least number of significant figures entering into the calculation 
        \item \textbf{For addition and subtraction}, the answer can contain no more decimal places than the least-precise measurement.
    \end{itemize}





    







    
    









    
\end{document}
