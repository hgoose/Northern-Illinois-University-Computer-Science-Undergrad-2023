\documentclass{report}

\input{~/dev/latex/template/preamble.tex}
\input{~/dev/latex/template/macros.tex}

\title{\Huge{}}
\author{\huge{Nathan Warner}}
\date{\huge{}}
\fancyhf{}
\rhead{}
\fancyhead[R]{\itshape Warner} % Left header: Section name
\fancyhead[L]{\itshape\leftmark}  % Right header: Page number
\cfoot{\thepage}
\renewcommand{\headrulewidth}{0pt} % Optional: Removes the header line
%\pagestyle{fancy}
%\fancyhf{}
%\lhead{Warner \thepage}
%\rhead{}
% \lhead{\leftmark}
%\cfoot{\thepage}
%\setborder
% \usepackage[default]{sourcecodepro}
% \usepackage[T1]{fontenc}

% Change the title
\hypersetup{
    pdftitle={Chapter 5 Example Problems}
}

\begin{document}
    % \maketitle
        \begin{titlepage}
       \begin{center}
           \vspace*{1cm}
    
           \textbf{Example Problems} \\
           Chapter 5: Newton's laws of motion
    
           \vspace{0.5cm}
            
                
           \vspace{1.5cm}
    
           \textbf{Nathan Warner}
    
           \vfill
                
                
           \vspace{0.8cm}
         
           \includegraphics[width=0.4\textwidth]{~/niu/seal.png}
                
           Computer Science \\
           Northern Illinois University\\
           February 24, 2024 \\
           United States\\
           
                
       \end{center}
    \end{titlepage}
    \tableofcontents
    \pagebreak 
    \unsect{5.2: Newton's first law of motion}
    \bigbreak \noindent 
    \textbf{Problem 1.} When Does Newton’s First Law Apply to Your Car?
    \bigbreak \noindent 
    Newton’s laws can be applied to all physical processes involving force and motion, including something as mundane as driving a car.
    \begin{enumerate}[label=(\alph*)]
        \item Your car is parked outside your house. Does Newton’s first law apply in this situation? Why or why not?
        \item Your car moves at constant velocity down the street. Does Newton’s first law apply in this situation? Why or why not?
    \end{enumerate}
    \bigbreak \noindent 
    \textcolor{red}{\textit{Solution.}} We are considering the first part of Newton’s first law, dealing with a body at rest; in (b), we look at the second part of Newton’s first law for a body in motion.
    \begin{enumerate}[label=(\alph*)]
        \item When your car is parked, all forces on the car must be balanced; the vector sum is 0 N. Thus, the net force is zero, and Newton’s first law applies. The acceleration of the car is zero, and in this case, the velocity is also zero.
        \item When your car is moving at constant velocity down the street, the net force must also be zero according to Newton’s first law. The car’s frictional force between the road and tires opposes the drag force on the car with the same magnitude, producing a net force of zero. The body continues in its state of constant velocity until the net force becomes nonzero. Realize that a net force of zero means that an object is either at rest or moving with constant velocity, that is, it is not accelerating. What do you suppose happens when the car accelerates? We explore this idea in the next section.
    \end{enumerate}

    \pagebreak 
    \unsect{5.3: Newton's second law of motion}
    \bigbreak \noindent 
    \textbf{Problem 1.} What Acceleration Can a Person Produce When Pushing a Lawn Mower?
    \bigbreak \noindent 
    Suppose that the net external force (push minus friction) exerted on a lawn mower is 51 N (about 11 lb.) parallel to the ground. The mass of the mower is 24 kg. What is its acceleration?
    \bigbreak \noindent 
    \fig{1}{./figures/1.jpeg}
    \bigbreak \noindent 
    \textcolor{red}{\textit{Solution.}} This problem involves only motion in the horizontal direction; we are also given the net force, indicated by the single vector, but we can suppress the vector nature and concentrate on applying Newton's second law. Since $F_{\text{net}}$ and $m$ are given, the acceleration can be calculated directly from Newton's second law as $F_{\text{net}} = ma$.
    \bigbreak \noindent 
    The magnitude of the acceleration $a$ is $a = \frac{F_{\text{net}}}{m}$. Entering known values gives
    \[ a = \frac{51\, \text{N}}{24\, \text{kg}}. \]
    Substituting the unit of kilograms times meters per square second for newtons yields
    \[ a = \frac{51\, \text{kg} \cdot \text{m}/\text{s}^2}{24\, \text{kg}} = 2.1\, \text{m}/\text{s}^2. \]

    \pagebreak \bigbreak \noindent 
    \textbf{Problem 2.} Which Force Is Bigger?
    \bigbreak \noindent 
    \fig{1}{./figures/2.jpeg}
    \begin{enumerate}
        \item[(a)] The car shown in Figure 5.13 is moving at a constant speed. Which force is bigger, $\vec{F}_{\text{friction}}$ or $\vec{F}_{\text{drag}}$? Explain.
        \item[(b)] The same car is now accelerating to the right. Which force is bigger, $\vec{F}_{\text{friction}}$ or $\vec{F}_{\text{drag}}$? Explain.
    \end{enumerate}
    \bigbreak \noindent 
    \textcolor{red}{\textit{Solution.}}
    \begin{enumerate}[label=(\alph*)]
        \item The forces are equal. According to Newton's first law, if the net force is zero, the velocity is constant. In this case, for the car to move at a constant speed, the net external force acting on it must be zero. This implies that the magnitude of the force of friction $\vec{F}_{\text{friction}}$ that acts opposite to the direction of motion must be equal to the magnitude of the drag force $\vec{F}_{\text{drag}}$ that also acts in the opposite direction to the motion of the car.
        \item     In this case, $\vec{F}_{\text{friction}}$ must be larger than $\vec{F}_{\text{drag}}$. According to Newton's second law, a net force is required to cause acceleration. For the car to accelerate to the right, the total force acting in the direction of acceleration (which includes the force of friction that now acts to propel the car forward, assuming it's the force exerted by the ground on the car) must be greater than the total force acting in the opposite direction (which includes the drag force).
    \end{enumerate}

    \pagebreak \bigbreak \noindent 
    \textbf{Problem 3.} What Rocket Thrust Accelerates This Sled?
    \bigbreak \noindent 
    Before space flights carrying astronauts, rocket sleds were used to test aircraft, missile equipment, and physiological effects on human subjects at high speeds. They consisted of a platform that was mounted on one or two rails and propelled by several rockets.
    \bigbreak \noindent 
    Calculate the magnitude of force exerted by each rocket, called its thrust $T$, for the four-rocket propulsion system. The sled's initial acceleration is $49\, \text{m/s}^2$, the mass of the system is $2100\, \text{kg}$, and the force of friction opposing the motion is $650\, \text{N}$.
    \bigbreak \noindent 
    \fig{1}{./figures/3.jpeg}
    \bigbreak \noindent 
    Although forces are acting both vertically and horizontally, we assume the vertical forces cancel because there is no vertical acceleration. This leaves us with only horizontal forces and a simpler one-dimensional problem. Directions are indicated with plus or minus signs, with right taken as the positive direction. 
    \bigbreak \noindent 
    Since acceleration, mass, and the force of friction are given, we start with Newton's second law and look for ways to find the thrust of the engines. We have defined the direction of the force and acceleration as acting "to the right," so we need to consider only the magnitudes of these quantities in the calculations. Hence we begin with
    \[
        F_{\text{net}} = ma
    \]
    where $F_{\text{net}}$ is the net force along the horizontal direction. We can see from the figure that the engine thrusts add, whereas friction opposes the thrust. In equation form, the net external force is
    \[
        F_{\text{net}} = 4T - f.
    \]
    Substituting this into Newton's second law gives us
    \[
        F_{\text{net}} = ma = 4T - f.
    \]
    Using a little algebra, we solve for the total thrust $4T$:
    \[
        4T = ma + f.
    \]
    Substituting known values yields
    \[
        4T = ma + f = (2100\,\text{kg})(49\,\text{m/s}^2) + 650\,\text{N}.
    \]
    Therefore, the total thrust is
    \[
        4T = 1.0 \times 10^5\,\text{N},
    \]
    and the individual thrusts are
    \[
        T = \frac{1.0 \times 10^5\,\text{N}}{4} = 2.5 \times 10^4\,\text{N}.
    \]

    \bigbreak \noindent 
    \textbf{Problem 4.} Force on a Soccer Ball






    
\end{document}
