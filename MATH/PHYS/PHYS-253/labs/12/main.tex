\documentclass{report}

\input{~/dev/latex/template/preamble.tex}
\input{~/dev/latex/template/macros.tex}

\title{\Huge{}}
\author{\huge{Nathan Warner}}
\date{\huge{}}
\fancyhf{}
\rhead{}
\fancyhead[R]{\itshape Warner} % Left header: Section name
\fancyhead[L]{\itshape\leftmark}  % Right header: Page number
\cfoot{\thepage}
\renewcommand{\headrulewidth}{0pt} % Optional: Removes the header line
%\pagestyle{fancy}
%\fancyhf{}
%\lhead{Warner \thepage}
%\rhead{}
% \lhead{\leftmark}
%\cfoot{\thepage}
%\setborder
% \usepackage[default]{sourcecodepro}
% \usepackage[T1]{fontenc}

% Change the title
\hypersetup{
    pdftitle={}
}

\begin{document}
    % \maketitle
    %     \begin{titlepage}
    %    \begin{center}
    %        \vspace*{1cm}
    %
    %        \textbf{Lab Report}
    %
    %        \vspace{0.5cm}
    %         Skyscaper
    %             
    %        \vspace{1.5cm}
    %
    %        \textbf{Nathan Warner}
    %
    %        \vfill
    %             
    %             
    %        \vspace{0.8cm}
    %      
    %        \includegraphics[width=0.4\textwidth]{~/niu/seal.png} \\
    %        
    %             
    %    \end{center}
    % \end{titlepage}
    \begin{center}
        \begin{Huge}
            Moment of Inertia Lab
        \end{Huge}
        \begin{Large}
            \bigbreak \noindent 
            Nate Warner
            \smallbreak \noindent
            April 27, 2024
            \bigbreak \noindent 
            Section 253D, 6:00 PM Tuesday 
        \end{Large}
    \end{center}
    \pagebreak 
    \tableofcontents
    \pagebreak \bigbreak \noindent 
    \begin{center}
    \textbf{Abstract}
    \end{center}
    \begin{adjustwidth}{.3in}{.3in}
        \hspace{\parindent} 
    \end{adjustwidth}

    \bigbreak \noindent 
    \section{Theory}
    \bigbreak \noindent 
    The aim of this laboratory exercise is to determine the moment of inertia for three distinct objects and to compare these experimental values with their theoretical predictions provided by the given formulas.
    \bigbreak \noindent 
    What exactly is the moment of inertia? In essence, it represents how resistant an object is to changes in its rotational speed. To visualize, consider turning left in a vehicle; you would feel your body shift to the right due to inertia. The sharper the turn, the more pronounced this shift. But how do we quantify the moment of inertia in a laboratory setting? We will explore this shortly.
    \bigbreak \noindent 
    Imagine a ball descending a slope. This ball possesses both rotational and translational velocities, contributing separately to its total kinetic energy, which can be expressed as:
    $$
    K_{\text{ToT}} = K_{\text{Trans}} + K_{\text{Rot}}
    $$
    \bigbreak \noindent 
    Translational kinetic energy is calculated using the object's mass and velocity, whereas rotational kinetic energy is determined using the moment of inertia, \(I\), and the angular velocity. Here, we focus specifically on the center of mass:
    $$
    \begin{aligned}
        K_{\text{Tman}} & = \frac{1}{2} m v_{CM}^2 \\
        K_{\text{flot}} & = \frac{1}{2} I_{CM} \omega^2
    \end{aligned}
    $$
    \bigbreak \noindent 
    Integrating these two formulas into equation (1), we derive:
    $$
    K = \frac{1}{2} m \tau_{CM}^2 + \frac{1}{2} I_{CM} \omega^2
    $$
    \bigbreak \noindent 
    From our previous experiments involving an axle, we know that linear velocity can be described using the radius of circular motion, \(R\), and the angular velocity (\(\omega\)):
    $$
    v = \omega R
    $$
    \bigbreak \noindent 
    This relationship allows us to substitute \(v\) in our kinetic energy equation to express \(K\) purely in terms of angular velocity:
    $$
    \begin{aligned}
        K & = \frac{1}{2} m \omega^2 R^2 + \frac{1}{2} I_{CM} \omega^2 \\
          & = \frac{1}{2}(m R^2 + I_{CM}) \omega^2
    \end{aligned}
    $$
    \bigbreak \noindent 
    This parenthetical expression represents the "Parallel Axis Theorem," which accounts for rotation about axes other than the center of mass. Considering our rolling ball, its rotation occurs not about its center, but along the ground, altering its moment of inertia:
    $$
    I = m R^2 + I_{CM}
    $$
    \bigbreak \noindent 
    Consequently, we simplify equation (3) as follows:
    $$
    K = \frac{1}{2} I \omega^2
    $$
    \bigbreak \noindent 
    In this experiment, we avoid collisions, simplifying the application of the conservation of energy. Total energy \(E\) is the sum of potential energy \(U\) and kinetic energy. Using the conservation principle, we find:
    $$
    \begin{aligned}
        K_i + U_i & = K_f + U_f \\
        \frac{1}{2} I \omega_i^2 + m g h_i & = \frac{1}{2} I \omega^2 + m g h_f
    \end{aligned}
    $$
    \bigbreak \noindent 
    Reorganizing these terms segregates kinetic and potential energy on opposite sides of the equation:
    $$
    \frac{1}{2} I(\omega_i^2 - \omega_f^2) = m g(h_f - h_i)
    $$
    \bigbreak \noindent 
    By isolating \(I\) and applying equation (4), we can determine the moment of inertia about the center of mass:
    $$
    \begin{gathered}
        I = \frac{2 m g(h_f - h_i)}{\omega_i^2 - \omega_j^2} \\
        I_{CM} + m R^2 = \frac{2 m g(h_f - h_i)}{\omega_i^2 - \omega_f^2} \\
        I_{CM} = \frac{2 m g(h_f - h_i)}{\omega_i^2 - \omega_f^2} - m R^2
    \end{gathered}
    $$
    \bigbreak \noindent 
    This illustrates that we can indeed calculate the moment of inertia about the center of mass for our objects using measurements obtainable directly in the laboratory.


    \bigbreak \noindent 
    \subsection{Theoretical Moments of Inertia}
    \bigbreak \noindent 
    The following equations will be the ones that you use to calculate the theoretical values for the moments of inertia and what you will compare your experimental results to.
    \begin{enumerate}
        \item A solid cylinder of mass $M$ and radius $R$ about its center:
        $$
        I_{C M}=\frac{1}{2} M R^2
        $$
        \item An annular cylinder (or ring) of inner radius $R_1$ and outer radius $R_2$ :
        $$
        I_{C M}=\frac{1}{2} M\left(R_1^2+R_2^2\right)
        $$
        \item A solid spbere
        $$
        I_{C M}=\frac{2}{5} M R^2
        $$
    \end{enumerate}

    \bigbreak \noindent 
    \subsection{Procedure}
    \begin{enumerate}
        \item Measure the mass of the three objects and record them. Additionally, measure the initial and final heights, which are the top and bottom of the incline.
        \item Use a string to measure the circumference of the sphere in order to calculate the radius \( c = 2\pi R \). Use a ruler to measure the diameters of the two cylinders \( D = 2R \).
        \item Measure the distance each of the objects travels through the photogate, denoted as \( d \). Start by placing a ruler on the track at one of the photogates and positioning the object at the higher side of the track so that the light on top of the photogate is just about to turn on. Allow the object to roll slowly through the photogate until the light turns off. The ruler will allow you to measure the start and stop distance. Record each of the distances: \( d_{\text{cylinder}}, d_{\text{ring}}, \) and \( d_{\text{sphere}} \). Note: the ring will have two instances of the light turning on and off due to the nature of a ring not having a center.
        \item Repeat the previous step for the second photogate and record.
        \item Place the solid cylinder on the high end of the track and start the LoggerPro timer. Once the timer is started, allow the cylinder to roll down the ramp. Record the time difference between states 1 and 0 for each of the photogate and calculate the initial and final angular velocities using the following equation:
            \[
                \omega = \frac{d_{\text{object}}}{R\Delta t}
            \]
            where \(\Delta t\) is the time difference between state 1 and state 0. Note that for the ring (with two radii), you have to choose which radius to use. There is one that makes more obvious sense. Give this thought and choose carefully.
        \item Repeat Step 4 for both the sphere and the ring. Again, it is important to note that the ring will have two states of 1 and 0 through the photogate, so to calculate \(\Delta t\), you will need to use the time difference between the first state 1 and second state 0.
    \end{enumerate}


    \bigbreak \noindent 
    \section{Data}
    \bigbreak \noindent 
    We start this experiment by collecting some measurements
    \bigbreak \noindent 
    \begin{center}
        \begin{tabular}{c|c}
            $m_{sphere} $ & 0.130kg \\
            $m_{ring}$  & & 0.068kg\\
            $m_{cylinder}$ & 0.024kg\\
            $R_{cylinder} $ &  0.01625m \\
            $R_{ring} $ & 0.01625m\\
            $R_{sphere} $ & 0.1m
            \hline
        \end{tabular}
    \end{center}
    \tc{Displays the measurements taken before commencing the experiment}
    \bigbreak \noindent 
    Next, we split the experiment into three separate trials, for the sphere, ring, and cylinder. In each trail, we find $h_{i}$, $h_{f}$, and $\Delta t$. Where $h_{i}$ is the starting height of the object, $h_{f}$ is the final height of the object, and $\Delta t$ is the time it takes to get from $h_{i}$ to $h_{f}$ (as measured by the photogate sensors). We also measure $d$, the length that the object travels.
    \bigbreak \noindent 
    \nt{We measure $h_{i}$ at the position just before the photogate, and $h_{f}$ directly after the photogate}
    \bigbreak \noindent 
    These measurements are found in the following table
    \bigbreak \noindent 
    \begin{center}
        \begin{tabular}{c|c|c|c|c}
            Trial (object)  &$h_{i}\, (m)$ & $h_{f}\, (m)$ & $d\, (m)$ & $\Delta t\, (s)$ \\
            \hline
            Cylinder & 0.08 & 0.065 & 0.035 & 0.18\\
            Ring  & 0.08 & 0.065 & 0.04 & 0.04\\
            Sphere  & 0.08 & 0.065 & 0.0375 & 0.16
        \end{tabular}
    \end{center}
    \tc{Shows the data collected for each object}

    \bigbreak \noindent 
    \section{Results}
    \bigbreak \noindent 
    We now find the initial angular velocity for each object using equation $\omega = \frac{d}{R\delta T} $. We present this computation using the cylinder trail and then displays results for the remaining two trials in a following table.
    \bigbreak \noindent 
    Recall we found (for the cylinder) $d = 0.035m$, $R=0.01625m$, and $\Delta t = 0.18$. Thus, we find the final angular velocity to be 
    \begin{align*}
        \omega &= \frac{d}{R\Delta T} = \frac{0.035m}{(0.01625m)(0.18s)} \\
        &=12 \frac{rad}{s}
    .\end{align*}
    \bigbreak \noindent 
    \nt{The initial angular velocity for each object is zero, only the final is calculated.}

    \bigbreak \noindent 
    \begin{center}
        \begin{tabular}{c|c}
             Trail & $\omega_{f}$  $\left(\frac{rad}{s}\right) $\\
            \hline
             Cylinder & 12\\
             Ring & 59.55\\
             Sphere & 15.12
        \end{tabular}
    \end{center}
    \tc{Shows $\omega_{f}$ for each object}
    \bigbreak \noindent 
    We now move to calculating the theoretical and experimental moments of inertia for each object. Again, we present the computations for only the cylinder.
    \bigbreak \noindent 
    \textbf{Recall:}
    \bigbreak \noindent 
     A solid cylinder of mass $M$ and radius $R$ about its center has a theoretical moment of inertia given by:
    $$
    I_{C M}=\frac{1}{2} M R^2
    $$
    \bigbreak \noindent 
    Whereas we use equation 6 to find the experimental, given by
    $$
    \begin{gathered}
        I_{CM} = \frac{2 m g(h_f - h_i)}{\omega_i^2 - \omega_f^2} - m R^2
    \end{gathered}
    $$
    \bigbreak \noindent 
    First, we calculate the theoretical
    \begin{align*}
        I_{cm} &= \frac{1}{2}MR^{2} \\
        &=\frac{1}{2}(0.068kg)(0.01625m)^{2} \\
        &=8.97 \times 10^{-6}kg \cdot m^{2}
    .\end{align*}
    \bigbreak \noindent 
    Using the equation the experimental, we get
    \begin{align*}
        I_{CM} &= \frac{2 m g(h_f - h_i)}{\omega_i^2 - \omega_f^2} - m R^2 \\
               &=\frac{2(0.068kg)g(0.065m - 0.08m)}{0 - 12^{2}} - (0.068kg)(0.01625m)^{2} \\
               &=1.2 \times 10^{-4}kg \cdot m^{2}
    .\end{align*}
    \bigbreak \noindent 
    The remaining calculations are found in the following table
    \begin{center}
        \begin{tabular}{c|c|c}
            Trail & Theoretical $I_{CM}\, (kg \cdot m^{2})$ & Experimental $I_{CM}\, (kg \cdot m^{2})$ \\
            \hline
            Cylinder & $8.97 \times 10^{-6}$ & $1.2\times 10^{-4}$ \\
            Ring & $6.14 \times 10^{-6}$&  $4.34 \times 10^{-6} $\\
            Sphere  & $1.3 \times 10^{-5} $& $1.3 \times 10^{-4}$
        \end{tabular}
    \end{center}
    \tc{Shows the results after comuting the theoretical and experimental moments of interia for each object}

    \bigbreak \noindent 
    \subsection{Percent Error}
    \bigbreak \noindent 
    \subsubsection{Cylinder}
    \begin{align*}
        \%\text{err} &= \frac{\abs{8.97 \times 10^{-6 } - 2 \times 10^{-4} }}{8.97 \times 10^{-6}} \cdot 100\% \\
        &=2129\%
    .\end{align*}
    \bigbreak \noindent 
    \subsubsection{Ring}
    \begin{align*}
        \%\text{err} &= \frac{\abs{6.14 \times 10^{-6 } - 4.34 \times 10^{-6} }}{6.14 \times 10^{-6}} \cdot 100\% \\
        &=29\%
    .\end{align*}
    \bigbreak \noindent 
    \subsubsection{Sphere}
    \begin{align*}
        \%\text{err} &= \frac{\abs{1.3 \times 10^{-5 } - 1.3 \times 10^{-4} }}{1.3 \times 10^{-5}} \cdot 100\% \\
        &=900\%
    .\end{align*}




    \bigbreak \noindent 
    \section{Discussion}
    \bigbreak \noindent 
    One plausible factor contributing to these discrepancies could be errors in the experimental setup and methodology. For instance, the approximation in measuring the exact dimensions of the objects, such as the radii and mass, can lead to variations in the calculated moment of inertia. Additionally, the frictional forces acting at the contact points between the objects and the incline were not accounted for in the theoretical calculations. These unconsidered frictional forces can alter the angular velocities, thus impacting the kinetic energy calculations and leading to deviations from the expected theoretical values.
    \bigbreak \noindent 
    Moreover, the experimental setup itself, including the alignment of the photogates and the precision of the timing devices, may introduce errors. Timing inaccuracies, particularly in capturing the exact moment an object passes through a photogate, can result in incorrect measurements of angular velocities. It's also possible that slight misalignments or uneven surfaces of the ramp could have influenced the objects' motion, affecting both translational and rotational kinetic energies.

    \bigbreak \noindent 
    \section{Conclusion}
    \bigbreak \noindent 
    In conclusion, this experiment aimed to measure the moment of inertia for three different objects and compare these experimental results to theoretical predictions. While the results had differences between the theoretical and experimental values, the experiment successfully demonstrated the principles underlying rotational dynamics and the conservation of energy. 










    
\end{document}
