\documentclass{report}

\input{~/dev/latex/template/preamble.tex}
\input{~/dev/latex/template/macros.tex}

\title{\Huge{}}
\author{\huge{Nathan Warner}}
\date{\huge{}}
\fancyhf{}
\rhead{}
\fancyhead[R]{\itshape Warner} % Left header: Section name
\fancyhead[L]{\itshape\leftmark}  % Right header: Page number
\cfoot{\thepage}
\renewcommand{\headrulewidth}{0pt} % Optional: Removes the header line
%\pagestyle{fancy}
%\fancyhf{}
%\lhead{Warner \thepage}
%\rhead{}
% \lhead{\leftmark}
%\cfoot{\thepage}
%\setborder
% \usepackage[default]{sourcecodepro}
% \usepackage[T1]{fontenc}

% Change the title
\hypersetup{
    pdftitle={Coin Toss Lab}
}

% Page geometry
\geometry{
  left=1in,
  right=1in,
  top=1in,
  bottom=1in
}

\usepackage{changepage}

\begin{document}
    % \maketitle
    %     \begin{titlepage}
    %    \begin{center}
    %        \vspace*{1cm}
    %
    %        \textbf{Lab Report}
    %
    %        \vspace{0.5cm}
    %         Skyscaper
    %             
    %        \vspace{1.5cm}
    %
    %        \textbf{Nathan Warner}
    %
    %        \vfill
    %             
    %             
    %        \vspace{0.8cm}
    %      
    %        \includegraphics[width=0.4\textwidth]{~/niu/seal.png} \\
    %        
    %             
    %    \end{center}
    % \end{titlepage}
    \begin{center}
        \begin{Huge}
            Coin Toss Lab Report
        \end{Huge}
        \begin{Large}
            \bigbreak \noindent 
            Nate Warner
            \smallbreak \noindent
            Febuary 6, 2024
            \smallbreak \noindent
            Lab Partner(s): Craig Dobbe
            \bigbreak \noindent 
            Section 253D, 6:00 PM Tuesday 
        \end{Large}
    \end{center}
    \pagebreak 
    \tableofcontents
    \pagebreak 
    \begin{center}
        \textbf{Abstract}
    \end{center}
    \begin{adjustwidth}{.3in}{.3in}
        \hspace{\parindent} This laboratory session blends theoretical concepts with practical experiments to enhance the lecture material. In this session, we become acquainted with the mathematical principles of averages, standard deviation, and uncertainties in experimental work.
    \end{adjustwidth}
    
    \bigbreak \noindent 
    \section{Theory}
    \bigbreak \noindent 
    \fig{.8}{./figures/1.png}{Frequency distribution for adult males. Image from the Sheffield University School of Health Science.}
    \bigbreak \noindent 
    Many experimental outcomes often manifest as a Gaussian distribution. Figure 1 shows the distribution of height measurements for 100 adult males at the Sheffield University School of Health Science. However, what insights can we derive from this kind of distribution? 

    \bigbreak \noindent 
    \subsection{Average}
    \bigbreak \noindent 
    Usually, our description of data starts with computing the average. The average is determined by summing all the data points in a set and then dividing by the total number of points to find the middle value. This concept can be mathematically represented as
    \begin{align*}
        &\bar{x} = \frac{\summation{n}{i=1}\ x_{i}\ }{N} \\
        &\bar{x} =\frac{\summation{n}{i=1}\ x_{i}f_{i}\ }{\summation{n}{i=1}\ f_{i}\ } 
    .\end{align*}
    \bigbreak \noindent 
    We use the second equation when each data point $x_{i}$ has some frequency attached.

    \bigbreak \noindent 
    \subsection{Standard Deviation}
    \bigbreak \noindent 
    The standard deviation measures the spread of your data's distribution. It reflects the width of the bell curve; a wider curve indicates data that is more dispersed, while a narrower, taller curve suggests data that is more concentrated and precise. Specifically, moving one standard deviation from the average encompasses 33\% of all data points. This concept can be mathematically expressed with the following equation
    \begin{align*}
        \sigma = \sqrt{\frac{\summation{n}{i=1}\ (x_{i} - \bar{x})^{2}\ }{N-1}}
    .\end{align*}
    \bigbreak \noindent 
    \subsection{Error Propagation}
    \bigbreak \noindent 
    The final key component of data analysis we'll explore is \textit{error propagation}, which assesses the uncertainty in calculated values due to the inherent imprecision of measurement systems. For instance, when using a meter stick with centimeter markings, any measurement includes an estimated millimeter value, representing your uncertainty. For example, measuring a person's height might result in a measurement expressed as $175.4 \pm 0.1\, \si{\centi\meter}$, indicating the uncertainty in the measurement.
    \bigbreak \noindent 
    For this lab, we will need two crucial formulas, uncertainty in perimeter, and uncertainty in area. They are given by 
    \begin{align*}
        \sigma_{\text{perimeter}} &= \sqrt{(\sigma_{\text{length}})^{2} + (\sigma_{\text{width}})^{2}} \\
        \sigma_{\text{area}}&= \sqrt{\left(\frac{\sigma_{\text{length}}}{length}\right)^{2} + \left(\frac{\sigma_{\text{width}}}{width}\right)^{2}}
    .\end{align*}

    \pagebreak 
    \section{Data}

    \bigbreak \noindent 
    \subsection{Coin Toss}

    \bigbreak \noindent 
    \subsubsection{Colab Coin Flipping Simulation Code}
    \bigbreak \noindent 
    \fig{.7}{./figures/3.png}{The following code will simulate flipping a coin n times}

    \bigbreak \noindent 
    \subsubsection{Data Obtained From Simulations}
    \bigbreak \noindent 
    \fig{.7}{./figures/4.png}

    \bigbreak \noindent 
    \subsubsection{Computation of the analysis}
    \bigbreak \noindent 
    \fig{.7}{./figures/5.png}

    \bigbreak \noindent 
    \subsection{Table Measurements}
    \bigbreak \noindent 
    \subsubsection{Colab Data and Computation}
    \bigbreak \noindent 
    \fig{.7}{./figures/6.png}

    \bigbreak \noindent 
    \section{Results}
    \bigbreak \noindent 
    \subsection{Coin Toss}
    The results obtained from the coin toss simulations are given by table.
    \bigbreak \noindent 
    \begin{center}
        \begin{tabular}{c|c|c}
            Trail Specifics & Mean & Standard Deviation \\
            \hline
            10 Trails of 10 Tosses  & 5.6 & 1.43\\
            10 Trails of 5 Tosses  &1.9 & 0.7 \\
            10 Trails of 100 Tosses  & 50.44 & 4.95\\
            20 Trails of 10 Tosses  &5.75 & 1.48\\
        \end{tabular}
    \end{center}

    \bigbreak \noindent 
    \subsection{Table Measurements}
    \bigbreak \noindent 
    The results obtained from the table measurements and Colab computations are expressed in the following list. 
    \bigbreak \noindent 
    \begin{itemize}
        \item Area: $11689.08 \pm 0.0015\, \si{\square\centi\meter}$ 
        \item Perimeter: $458.20 \pm 0.014\, \si{\centi\meter}$
    \end{itemize}

    \bigbreak \noindent 
    \section{Discussion}
    \bigbreak \noindent 
    Jumping into this lab was like getting a real taste of what data and its odd behaviors are all about. The coin toss part was super eye-opening. You hear about probabilities in lectures, but actually seeing them play out in a bunch of coin flips is a whole different experience. It was cool to see the theory we've been talking about actually show up in the results, especially when we got that neat 50-50 split after a ton of tosses. 
    \bigbreak \noindent 
    And then there was the table measuring exercise. Despite the measuring stick, we still had to rely on a bit of estimation. Figuring out the uncertainties in the table's dimensions was really interesting. Precision isn't only about using equipment, it's about understanding the equipment's limitations and what the numbers are actually saying.


    \bigbreak \noindent 
    \section{Conclusion}
    \bigbreak \noindent 
    In wrapping up this lab, I found the hands-on approach to exploring probabilities and uncertainties both challenging and enlightening. It was fascinating to see theoretical concepts come to life through the coin toss simulations and table measurements. These experiments highlighted the unpredictable nature of real-world data.
    \bigbreak \noindent 
    I appreciated the lesson in precision and the importance of understanding the tools we use. It was a reminder that there's always room for improvement, especially in how we measure and interpret data. This lab was a success in my eyes, it deepened my understanding of the subject matter and sharpened my analytical skills. 










    
\end{document}
