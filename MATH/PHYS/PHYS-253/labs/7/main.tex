\documentclass{report}

\input{~/dev/latex/template/preamble.tex}
\input{~/dev/latex/template/macros.tex}

\title{\Huge{}}
\author{\huge{Nathan Warner}}
\date{\huge{}}
\fancyhf{}
\rhead{}
\fancyhead[R]{\itshape Warner} % Left header: Section name
\fancyhead[L]{\itshape\leftmark}  % Right header: Page number
\cfoot{\thepage}
\renewcommand{\headrulewidth}{0pt} % Optional: Removes the header line
%\pagestyle{fancy}
%\fancyhf{}
%\lhead{Warner \thepage}
%\rhead{}
% \lhead{\leftmark}
%\cfoot{\thepage}
%\setborder
% \usepackage[default]{sourcecodepro}
% \usepackage[T1]{fontenc}

% Change the title
\hypersetup{
    pdftitle={}
}

\begin{document}
% \maketitle
%     \begin{titlepage}
%    \begin{center}
%        \vspace*{1cm}
%
%        \textbf{Lab Report}
%
%        \vspace{0.5cm}
%         Skyscaper
%             
%        \vspace{1.5cm}
%
%        \textbf{Nathan Warner}
%
%        \vfill
%             
%             
%        \vspace{0.8cm}
%      
%        \includegraphics[width=0.4\textwidth]{~/niu/seal.png} \\
%        
%             
%    \end{center}
% \end{titlepage}
\begin{center}
    \begin{Huge}
        Simple Harmonic Motion (SHM) Lab
    \end{Huge}
    \begin{Large}
        \bigbreak \noindent 
        Nate Warner
        \smallbreak \noindent
        March 19, 2023
        \bigbreak \noindent 
        Section 253D, 6:00 PM Tuesday 
    \end{Large}
\end{center}
\pagebreak 
\tableofcontents
\pagebreak \bigbreak \noindent 
\begin{center}
    \textbf{Abstract}
\end{center}
\begin{adjustwidth}{.3in}{.3in}
    \hspace{\parindent} This study explores the dynamics of simple harmonic motion (SHM) through the analysis of a mass-spring system. By attaching a mass to a vertically suspended spring, we investigate how the system oscillates around a new equilibrium position, characterized by a displacement $\Delta L$ from the spring's rest length. The restoring force, $F = -ky$, where $y$ is the displacement from equilibrium and $k$ is the spring constant, initiates oscillation. The motion is described by $y= A\sin(\omega t + \phi)$, where $A$ is the amplitude, $\omega$ the angular frequency, and $\phi$ a phase constant. The experiment aims to validate the theoretical model that predicts the oscillation's frequency and amplitude based on the mass $m$ and spring constant $k$, and how gravitational force influences the equilibrium position.
    \bigbreak \noindent 
    The experimental procedure involves attaching a mass to a spring and using a motion detector to record the oscillations. Adjustments to the setup ensure accurate data collection on displacement and velocity over time. Preliminary runs validate the setup's correctness, with further data analysis expected to confirm the theoretical predictions of SHM behavior, including the influence of gravity on the system's equilibrium position.
\end{adjustwidth}

\bigbreak \noindent 
\section{Theory}
\bigbreak \noindent 
Many things involve vibrations or oscillations. A fundamental example of a vibrating system is a mass attached to a spring. Consider a spring suspended vertically; its initial length without any attached mass is $L$, known as its rest or equilibrium length. Attaching a mass extends the spring by $\Delta L$, setting a new equilibrium position at $L + \Delta L$ from the support point. Pulling the mass slightly further down introduces a restoring force, $F = -ky$, where $y$ is the extension from the equilibrium position, and $k$ is the spring's force constant. The negative sign signifies that the force is opposite the mass's displacement. This restoring force initiates oscillation in the mass, with its period influenced by the mass and the spring constant. The motion of this mass-spring system, portrays what we call \textit{simple harmonic motion (SHM)}, and its vertical movement can be described by the equation:
\bigbreak \noindent 
\begin{equation}
    y= A\sin{\left(2\pi ft + \phi\right)} = A\sin{\left(\omega t + \phi\right)}
\end{equation}
\bigbreak \noindent 
In the equation, $y$ represents the vertical displacement from the equilibrium position, $A$ is the amplitude of the motion, $f$ is the frequency of oscillation, $t$ is time, and $\phi$ is a phase constant. The angular frequency is denoted by $\omega$, and it is known that the period of oscillation $T$ is the inverse of the frequency, $T = \frac{1}{f}$.
\bigbreak \noindent 
The origin of the simple harmonic motion (SHM) equation can be traced to the dynamics of a mass block $m$ subjected to gravitational force $mg$ and a restoring force $-ky$, where $y$ is its position. The net force is the sum of these forces. By Newton's second law, we have
\bigbreak \noindent 
\begin{equation}
    \sum F = mg - ky = ma 
\end{equation}
\bigbreak \noindent 
where $a$ is the rate of change of the velocity
\begin{align*}
    a = \frac{dv}{dt}
.\end{align*}
\bigbreak \noindent 
and the velocity is the rate of change of the position $y$
\begin{equation}
    \begin{split}
        v &= \frac{dy}{dt} \\
        \implies a &= \frac{d^{2}y}{dt^{2}}
    \end{split}
\end{equation}
\bigbreak \noindent 
Putting this together, we find an equation to solve
\begin{equation}
    mg - ky = m \frac{dy^{2}}{dt^{2}}
\end{equation}
\bigbreak \noindent 
A method for solving this involves proposing a solution. Suppose we guess $y=A\sin{\left(\omega t + \phi\right) + B} $. Where $A$, $\omega$, $\phi$, and $B$ are potential constants or things we need to solve for
\bigbreak \noindent 
To find the solution we need to take the first and then second derivative of $y$
\bigbreak \noindent 
\begin{equation}
    v=\frac{dy}{dt} = A\omega\cos{\left(\omega t + \phi\right)}
\end{equation}
and
\bigbreak \noindent 
\begin{equation}
    a = \frac{dy^{2}}{dt^{2}} = -A\omega^{2}\sin{\left(\omega t + \phi\right)}
\end{equation}
Plugging this into the equation we want to solve for we get
\begin{equation}
    mg -k(A\sin{\left(\omega t + \phi\right)}) -kB = -m\omega^{2}(A\sin{\left(\omega t + \phi\right)})
\end{equation}
\bigbreak \noindent 
Observing the equation, it is noted that the left-hand side contains two terms independent of time, whereas the right-hand side lacks such terms. Consequently, the time-independent terms on the left must negate each other, leading to $mg - kB = 0$ or equivalently, $B = \frac{mg}{k}$. Examining the time-dependent terms, we find that the terms on both sides of the equation, encapsulated within parentheses and exhibiting time dependence, must match. Thus, for the equation to be consistently valid, it is required that
\begin{equation}
    -k = -m\omega^{2} 
\end{equation}
or
\begin{equation}
    \omega = \sqrt{(k/m)} 
\end{equation}
\bigbreak \noindent 
Thus, the frequency of the oscillation is fixed by $m$ and $k$, and
\begin{equation}
    y= \frac{A\sin{\left(\omega t + \phi\right)}  + (mg)}{k}
\end{equation}
Where $\omega = \sqrt{(k/m)}$, and $A$ and $\phi$ as free parameters
\bigbreak \noindent 
Observing the equation, it is evident that the left side contains two terms that are independent of time, whereas the right side lacks such terms. Consequently, the two time-independent terms on the left must negate each other: $mg - kB = 0$ or $B = \frac{mg}{k}$. Examining the time-dependent terms, we notice that the terms on both sides with time dependence (enclosed in parentheses) match, indicating that for the equation to be consistently valid, we must establish:
\bigbreak \noindent 
The term $\left(\frac{mg}{k}\right)$ represents a shift in the equilibrium position attributable to gravitational forces. This implies that even when the spring is stationary, the gravitational pull ($mg$ downwards) is counterbalanced by a spring force ($-ky_0$ upwards), leading to $mg - ky_0 = 0$ or $y_0 = \frac{mg}{k}$.

\bigbreak \noindent 
\subsection{Objectives}
\begin{itemize}
    \item Analyze how the movement of a mass-spring system aligns with the theoretical model of simple harmonic motion.
    \item Identify the amplitude, period, and phase shift of the observed simple harmonic motion, as well as calculate the spring's stiffness constant.
\end{itemize}


\bigbreak \noindent 
\subsection{Procedure}
\begin{itemize}
    \item Connect the Motion Detector to PORT 1 on the Lab Pro Interface, ensuring it's linked to the computer via USB and the power cable is connected. Position the motion detector securely on a low stool.
    \item Use a meter stick to precisely position the motion detector directly beneath the spring assembly, ensuring it's exactly aligned and unobstructed. Attach an index card to the bottom of the mass holder to improve detection accuracy.
    \item Launch the Logger Pro software and open the “SHM Exp 15” file from the Physics with Computers experiments. This displays distance vs. time and velocity vs. time graphs.
    \item Adjust the sampling rate to 20 samples per second using the software's clock icon.
    \item Modify the position graph settings to a Y-axis range of -0.3m to 0.3m.
    \item Calibrate the motion detector by ensuring it's stationary, then zeroing it and reversing its direction in the sensor settings. Verify it reads "0"; if not, recalibrate.
    \item Conduct a preliminary test by lifting and releasing the mass about 6-8 cm to ensure it oscillates vertically. Start data collection, which stops after 10 seconds. The graph should display a smooth sinusoidal curve; any irregularities mean the detector needs repositioning.
\end{itemize}


\bigbreak \noindent 
\section{Data}
\bigbreak \noindent 
This experiment consists of two trials, where each trial in conducted with a different mass. Each trial will have two iterations, these iterations will have the same mass but differ in $\Delta y$
\bigbreak \noindent 
Consequently, we have four sets of data to analyze. Each set of data will be fitted with a position vs time curve, and a velocity vs time curve.
\bigbreak \noindent 
\begin{remark}
    The data recorded for each iteration is quite long. Thus, only a subset will be presented in this report. Note that the graphs displayed in a following section will be comprised of all points in the data set 
\end{remark}

\pagebreak 
\subsection{Colab code}
\bigbreak \noindent 
\fig{.7}{./figures/code.png}



\pagebreak 
\subsection{Trial 1}
\bigbreak \noindent 
Before we table the values for trial one, we remark the constants
\begin{align*}
        &\text{Mass:}\ 0.05kg\\
        &\text{Equilibrium Distance ($y_{0}$):}\ 0.759m
    .\end{align*}
    \bigbreak \noindent 
    \nt{The equilibrium distance is the distance from the sensor to the mass, when the mass in resting in its equilibrium position.}
    \bigbreak \noindent 
    The following table lists the values for $\Delta y$ amongst each trial, where $\Delta y$ represents the distance (in m) that the spring was lifted upward before releasing
    \begin{center}
        \begin{tabular}{c|c}
            Trial & $\Delta y$ (m) \\
            \hline
            1 & 0.06 \\
            2 & 0.08
        \end{tabular}
    \end{center}
    \tc{}
    \bigbreak \noindent 
    \subsubsection{Iteration 1 Data Sets}
    \bigbreak \noindent 
    The following data represents a sample of size 25
    \bigbreak \noindent 
        \begin{center}
            \begin{tabular}{c|c|c}
                Time (s) & Position (m) & Velocity (m/s)\\
                \hline
                0.0333&	-0.057281&	-0.006866866867\\
                0.0666&	-0.0581385	&0.05035702369\\
                0.0999&	-0.055223&	0.1416291291\\
                0.1332&	-0.048363&	0.210726977\\
                0.1665&	-0.039788&	0.2068643644\\
                0.1998&	-0.035672&	0.272957958\\
                0.2331&	-0.023324&	0.4296083584\\
                0.2664&	-0.005145&	0.481539039\\
                0.2997&	0.0097755&	0.4493506006\\
                0.333 &  0.024353 &   0.4235998499\\
                0.3663&	0.0382445&	0.3828278278\\
                0.3996&	0.0502495&	0.3133008008\\
                0.4329&	0.0591675&	0.2339026527\\
                0.4662&	0.065856 &   0.1510710711\\
                0.4995&	0.069286 &   0.05965590591\\
                0.5328&	0.0701435&	-0.04635135135\\
                0.5661&	0.065856 &   -0.1240327828\\
                0.5994&	0.061054 &   -0.1596546547\\
                0.6327&	0.0560805&	-0.239481982\\
                0.666 &  0.0454475&	-0.3420558058\\
                0.6993&	0.032585 &   -0.399994995\\
                0.7326&	0.0183505&	-0.4175913413\\
                0.7659&	0.0046305&	-0.416732983\\
                0.7992&	-0.009604&	-0.3957032032\\
            \end{tabular}
        \end{center}

    \bigbreak \noindent 
    \subsubsection{Iteration 2 Data Sets}
    \bigbreak \noindent 
    The following data represents a sample of size 25
    \bigbreak \noindent 
    \begin{center}
        \begin{tabular}{c|c|c}
            Time (s) & Position (m) & Velocity (m/s)
            \hline
        \end{tabular}
    \end{center}


    \bigbreak \noindent 
    \subsection{Trial 2}
    \bigbreak \noindent 
    Again, we remark the constants. 
    \begin{align*}
         &\text{Mass:}\ 0.1kg\\
         &\text{Equilibrium Distance ($y_{0}$):}\ 0.6258m
     .\end{align*}
     \bigbreak \noindent 
     Along with the distances $\Delta y$
     \begin{center}
         \begin{tabular}{c|c}
             Trial & $\Delta y$ (m) \\
             \hline
             1 & 0.06 \\
             2 & 0.08
         \end{tabular}
     \end{center}
     \tc{}

     \bigbreak \noindent 
     \subsubsection{Data set}






     \bigbreak \noindent 
     \section{Results}
     \bigbreak \noindent 

     \bigbreak \noindent 
     \subsection{Trial 1}
     \bigbreak \noindent 
     \subsubsection{Iteration 1: Graphs}
     \bigbreak \noindent 
     \fig{.5}{./figures/set1.png}
     \bigbreak \noindent 
     \subsubsection{Iteration 1: Sine Wave Fitting Equation}
     \begin{align*}
        &\text{Position vs time: } 0.0461\sin{(6.6025t + -1.8734)} \\
        &\text{Velocity vs time: } 0.2989\sin{(6.6021t + -0.2901)}
    .\end{align*}

    \bigbreak \noindent 
    \subsubsection{Iteration 2: Graphs}
    \bigbreak \noindent 
    \fig{.5}{./figures/set2.png}

    \bigbreak \noindent 
    \subsubsection{Iteration 2: Sine Wave Fitting Equation}
    \begin{align*}
        &\text{Position vs time: } 0.0634\sin{(6.6235t + 1.6918)} \\
        &\text{Velocity vs time: } -0.4148\sin{(6.6221t + 0.1449)}
    .\end{align*}

    \bigbreak \noindent 
    \subsection{Trial 2}

    \bigbreak \noindent 
    \subsubsection{Iteration 1: Graphs} 
    \fig{.5}{./figures/set3.png}


    \bigbreak \noindent 
\subsubsection{Iteration 1: Sine Wave Fitting Equation}}
\begin{align*}
        &\text{Position vs time: } 0.0545\sin{(5.1754t + -0.9313)} \\
        &\text{Velocity vs time: } 0.0636\sin{(4.7262t + 1.9529)}
    .\end{align*}

    \bigbreak \noindent 
    \subsubsection{Iteration 2: Graphs} 
    \fig{.5}{./figures/set4.png}

    \bigbreak \noindent 
\subsubsection{Iteration 2: Sine Wave Fitting Equation}}
\begin{align*}
        &\text{Position vs time: } 0.0773\sin{(5.1737t + -1.1965)} \\
        &\text{Velocity vs time: } 0.3971\sin{(5.1736t + 0.3824)}
    .\end{align*}


    \bigbreak \noindent 
    \section{Discussion}
    \bigbreak \noindent 

    \bigbreak \noindent 
    \section{Conclusion}
    \bigbreak \noindent 











\end{document}
