\documentclass{report}

\input{~/dev/latex/template/preamble.tex}
\input{~/dev/latex/template/macros.tex}

\title{\Huge{}}
\author{\huge{Nathan Warner}}
\date{\huge{}}
\fancyhf{}
\rhead{}
\fancyhead[R]{\itshape Warner} % Left header: Section name
\fancyhead[L]{\itshape\leftmark}  % Right header: Page number
\cfoot{\thepage}
\renewcommand{\headrulewidth}{0pt} % Optional: Removes the header line
%\pagestyle{fancy}
%\fancyhf{}
%\lhead{Warner \thepage}
%\rhead{}
% \lhead{\leftmark}
%\cfoot{\thepage}
%\setborder
% \usepackage[default]{sourcecodepro}
% \usepackage[T1]{fontenc}

% Change the title
\hypersetup{
    pdftitle={}
}

\begin{document}
% \maketitle
%     \begin{titlepage}
%    \begin{center}
%        \vspace*{1cm}
%
%        \textbf{Lab Report}
%
%        \vspace{0.5cm}
%         Skyscaper
%             
%        \vspace{1.5cm}
%
%        \textbf{Nathan Warner}
%
%        \vfill
%             
%             
%        \vspace{0.8cm}
%      
%        \includegraphics[width=0.4\textwidth]{~/niu/seal.png} \\
%        
%             
%    \end{center}
% \end{titlepage}
\begin{center}
    \begin{Huge}
        Simple Harmonic Motion (SHM) Lab
    \end{Huge}
    \begin{Large}
        \bigbreak \noindent 
        Nate Warner
        \smallbreak \noindent
        March 19, 2023
        \bigbreak \noindent 
        Section 253D, 6:00 PM Tuesday 
    \end{Large}
\end{center}
\pagebreak 
\tableofcontents
\pagebreak \bigbreak \noindent 
\begin{center}
    \textbf{Abstract}
\end{center}
\begin{adjustwidth}{.3in}{.3in}
    \hspace{\parindent} This study explores the dynamics of simple harmonic motion (SHM) through the analysis of a mass-spring system. By attaching a mass to a vertically suspended spring, we investigate how the system oscillates around a new equilibrium position, characterized by a displacement $\Delta L$ from the spring's rest length. The restoring force, $F = -ky$, where $y$ is the displacement from equilibrium and $k$ is the spring constant, initiates oscillation. The motion is described by $y= A\sin(\omega t + \phi)$, where $A$ is the amplitude, $\omega$ the angular frequency, and $\phi$ a phase constant. The experiment aims to validate the theoretical model that predicts the oscillation's frequency and amplitude based on the mass $m$ and spring constant $k$, and how gravitational force influences the equilibrium position.
    \bigbreak \noindent 
    The experimental procedure involves attaching a mass to a spring and using a motion detector to record the oscillations. Adjustments to the setup ensure accurate data collection on displacement and velocity over time. Preliminary runs validate the setup's correctness, with further data analysis expected to confirm the theoretical predictions of SHM behavior, including the influence of gravity on the system's equilibrium position.
\end{adjustwidth}

\bigbreak \noindent 
\section{Theory}
\bigbreak \noindent 
Many things involve vibrations or oscillations. A fundamental example of a vibrating system is a mass attached to a spring. Consider a spring suspended vertically; its initial length without any attached mass is $L$, known as its rest or equilibrium length. Attaching a mass extends the spring by $\Delta L$, setting a new equilibrium position at $L + \Delta L$ from the support point. Pulling the mass slightly further down introduces a restoring force, $F = -ky$, where $y$ is the extension from the equilibrium position, and $k$ is the spring's force constant. The negative sign signifies that the force is opposite the mass's displacement. This restoring force initiates oscillation in the mass, with its period influenced by the mass and the spring constant. The motion of this mass-spring system, portrays what we call \textit{simple harmonic motion (SHM)}, and its vertical movement can be described by the equation:
\bigbreak \noindent 
\begin{equation}
    y= A\sin{\left(2\pi ft + \phi\right)} = A\sin{\left(\omega t + \phi\right)}
\end{equation}
\bigbreak \noindent 
In the equation, $y$ represents the vertical displacement from the equilibrium position, $A$ is the amplitude of the motion, $f$ is the frequency of oscillation, $t$ is time, and $\phi$ is a phase constant. The angular frequency is denoted by $\omega$, and it is known that the period of oscillation $T$ is the inverse of the frequency, $T = \frac{1}{f}$.
\bigbreak \noindent 
The origin of the simple harmonic motion (SHM) equation can be traced to the dynamics of a mass block $m$ subjected to gravitational force $mg$ and a restoring force $-ky$, where $y$ is its position. The net force is the sum of these forces. By Newton's second law, we have
\bigbreak \noindent 
\begin{equation}
    \sum F = mg - ky = ma 
\end{equation}
\bigbreak \noindent 
where $a$ is the rate of change of the velocity
\begin{align*}
    a = \frac{dv}{dt}
.\end{align*}
\bigbreak \noindent 
and the velocity is the rate of change of the position $y$
\begin{equation}
    \begin{split}
        v &= \frac{dy}{dt} \\
        \implies a &= \frac{d^{2}y}{dt^{2}}
    \end{split}
\end{equation}
\bigbreak \noindent 
Putting this together, we find an equation to solve
\begin{equation}
    mg - ky = m \frac{dy^{2}}{dt^{2}}
\end{equation}
\bigbreak \noindent 
A method for solving this involves proposing a solution. Suppose we guess $y=A\sin{\left(\omega t + \phi\right) + B} $. Where $A$, $\omega$, $\phi$, and $B$ are potential constants or things we need to solve for
\bigbreak \noindent 
To find the solution we need to take the first and then second derivative of $y$
\bigbreak \noindent 
\begin{equation}
    v=\frac{dy}{dt} = A\omega\cos{\left(\omega t + \phi\right)}
\end{equation}
and
\bigbreak \noindent 
\begin{equation}
    a = \frac{dy^{2}}{dt^{2}} = -A\omega^{2}\sin{\left(\omega t + \phi\right)}
\end{equation}
Plugging this into the equation we want to solve for we get
\begin{equation}
    mg -k(A\sin{\left(\omega t + \phi\right)}) -kB = -m\omega^{2}(A\sin{\left(\omega t + \phi\right)})
\end{equation}
\bigbreak \noindent 
Observing the equation, it is noted that the left-hand side contains two terms independent of time, whereas the right-hand side lacks such terms. Consequently, the time-independent terms on the left must negate each other, leading to $mg - kB = 0$ or equivalently, $B = \frac{mg}{k}$. Examining the time-dependent terms, we find that the terms on both sides of the equation, encapsulated within parentheses and exhibiting time dependence, must match. Thus, for the equation to be consistently valid, it is required that
\begin{equation}
    -k = -m\omega^{2} 
\end{equation}
or
\begin{equation}
    \omega = \sqrt{(k/m)} 
\end{equation}
\bigbreak \noindent 
Thus, the frequency of the oscillation is fixed by $m$ and $k$, and
\begin{equation}
    y= \frac{A\sin{\left(\omega t + \phi\right)}  + (mg)}{k}
\end{equation}
Where $\omega = \sqrt{(k/m)}$, and $A$ and $\phi$ as free parameters
\bigbreak \noindent 
Observing the equation, it is evident that the left side contains two terms that are independent of time, whereas the right side lacks such terms. Consequently, the two time-independent terms on the left must negate each other: $mg - kB = 0$ or $B = \frac{mg}{k}$. Examining the time-dependent terms, we notice that the terms on both sides with time dependence (enclosed in parentheses) match, indicating that for the equation to be consistently valid, we must establish:
\bigbreak \noindent 
The term $\left(\frac{mg}{k}\right)$ represents a shift in the equilibrium position attributable to gravitational forces. This implies that even when the spring is stationary, the gravitational pull ($mg$ downwards) is counterbalanced by a spring force ($-ky_0$ upwards), leading to $mg - ky_0 = 0$ or $y_0 = \frac{mg}{k}$.

\bigbreak \noindent 
\subsection{Objectives}
\begin{itemize}
    \item Analyze how the movement of a mass-spring system aligns with the theoretical model of simple harmonic motion.
    \item Identify the amplitude, period, and phase shift of the observed simple harmonic motion, as well as calculate the spring's stiffness constant.
\end{itemize}


\bigbreak \noindent 
\subsection{Procedure}
\begin{itemize}
    \item Connect the Motion Detector to PORT 1 on the Lab Pro Interface, ensuring it's linked to the computer via USB and the power cable is connected. Position the motion detector securely on a low stool.
    \item Use a meter stick to precisely position the motion detector directly beneath the spring assembly, ensuring it's exactly aligned and unobstructed. Attach an index card to the bottom of the mass holder to improve detection accuracy.
    \item Launch the Logger Pro software and open the “SHM Exp 15” file from the Physics with Computers experiments. This displays distance vs. time and velocity vs. time graphs.
    \item Adjust the sampling rate to 20 samples per second using the software's clock icon.
    \item Modify the position graph settings to a Y-axis range of -0.3m to 0.3m.
    \item Calibrate the motion detector by ensuring it's stationary, then zeroing it and reversing its direction in the sensor settings. Verify it reads "0"; if not, recalibrate.
    \item Conduct a preliminary test by lifting and releasing the mass about 6-8 cm to ensure it oscillates vertically. Start data collection, which stops after 10 seconds. The graph should display a smooth sinusoidal curve; any irregularities mean the detector needs repositioning.
\end{itemize}


\bigbreak \noindent 
\section{Data}
\bigbreak \noindent 
This experiment consists of two trials, where each trial in conducted with a different mass. Each trial will have two iterations, these iterations will have the same mass but differ in $\Delta y$
\bigbreak \noindent 
Consequently, we have four sets of data to analyze. Each set of data will be fitted with a position vs time curve, and a velocity vs time curve.
\bigbreak \noindent 
\begin{remark}
    The data recorded for each iteration is quite long. Thus, only a subset will be presented in this report. Note that the graphs displayed in a following section will be comprised of all points in the data set 
\end{remark}

\pagebreak 
\subsection{Colab code}
\bigbreak \noindent 
\fig{.7}{./figures/code.png}



\pagebreak 
\subsection{Trial 1}
\bigbreak \noindent 
Before we table the values for trial one, we remark the constants
\begin{align*}
        &\text{Mass:}\ 0.05kg\\
        &\text{Equilibrium Distance ($y_{0}$):}\ 0.759m
    .\end{align*}
    \bigbreak \noindent 
    \nt{The equilibrium distance is the distance from the sensor to the mass, when the mass in resting in its equilibrium position.}
    \bigbreak \noindent 
    The following table lists the values for $\Delta y$ amongst each trial, where $\Delta y$ represents the distance (in m) that the spring was lifted upward before releasing
    \begin{center}
        \begin{tabular}{c|c}
            Trial & $\Delta y$ (m) \\
            \hline
            1 & 0.06 \\
            2 & 0.08
        \end{tabular}
    \end{center}
    \tc{}
    \bigbreak \noindent 
    \subsubsection{Iteration 1 Data Set}
    \bigbreak \noindent 
    The following data represents a sample of size 25
    \bigbreak \noindent 
        \begin{center}
            \begin{tabular}{c|c|c}
                Time (s) & Position (m) & Velocity (m/s)\\
                \hline
                0.0333&	-0.057281&	-0.006866866867\\
                0.0666&	-0.0581385	&0.05035702369\\
                0.0999&	-0.055223&	0.1416291291\\
                0.1332&	-0.048363&	0.210726977\\
                0.1665&	-0.039788&	0.2068643644\\
                0.1998&	-0.035672&	0.272957958\\
                0.2331&	-0.023324&	0.4296083584\\
                0.2664&	-0.005145&	0.481539039\\
                0.2997&	0.0097755&	0.4493506006\\
                0.333 &  0.024353 &   0.4235998499\\
                0.3663&	0.0382445&	0.3828278278\\
                0.3996&	0.0502495&	0.3133008008\\
                0.4329&	0.0591675&	0.2339026527\\
                0.4662&	0.065856 &   0.1510710711\\
                0.4995&	0.069286 &   0.05965590591\\
                0.5328&	0.0701435&	-0.04635135135\\
                0.5661&	0.065856 &   -0.1240327828\\
                0.5994&	0.061054 &   -0.1596546547\\
                0.6327&	0.0560805&	-0.239481982\\
                0.666 &  0.0454475&	-0.3420558058\\
                0.6993&	0.032585 &   -0.399994995\\
                0.7326&	0.0183505&	-0.4175913413\\
                0.7659&	0.0046305&	-0.416732983\\
                0.7992&	-0.009604&	-0.3957032032\\
                0.7992	& -0.009604&	-0.3957032032 
            \end{tabular}
        \end{center}
        \tc{The following table represents a subset from the full data set of size 25}

    \bigbreak \noindent 
    \subsubsection{Iteration 2 Data Set}
    \bigbreak \noindent 
    The following data represents a sample of size 25
    \bigbreak \noindent 
    \begin{center}
        \begin{tabular}{c|c|c}
            Time (s) & Position (m) & Velocity (m/s) \\
            \hline
            0.0333	&0.0999845	&-0.2137312312\\
            0.0666	&0.093639	&-0.2798248248\\
            0.0999	&0.082663	&-0.3819694695\\
            0.1332	&0.0680855	&-0.4918393393\\
            0.1665	&0.0495635	&-0.5716666667\\
            0.1998	&0.029498	&-0.6115803303\\
            0.2331	&0.008575	&-0.6210222723\\
            0.2664	&-0.012348	&-0.5922672673\\
            0.2997	&-0.031213	&-0.531753003\\
            0.333	&-0.04802	&-0.4497797798\\
            0.3663	&-0.0612255	&-0.3583646146\\
            0.3996	&-0.07203	&-0.2540740741\\
            0.4329	&-0.0783755	&-0.1351914414\\
            0.4662	&-0.080948	&-0.01502127127\\
            0.4995	&-0.079576	&0.1111574074\\
            0.5328	&-0.0732305	&0.2218856356\\
            0.5661	&-0.064827	&0.3364764765\\
            0.5994	&-0.0512785	&0.4639426927\\
            0.6327	&-0.032928	&0.5313238238\\
            0.666	&-0.015435	&0.5570745746\\
            0.6993	&0.004116	&0.5712374875\\
            0.7326	&0.0231525	&0.5549286787\\
            0.7659	&0.0409885	&0.5321821822\\
            0.7992	&0.0588245	&0.4918393393\\
            0.8325	&0.074088	&0.4205955956
        \end{tabular}
    \end{center}
    \tc{The following table represents a subset from the full data set of size 25}


    \bigbreak \noindent 
    \subsection{Trial 2}
    \bigbreak \noindent 
    Again, we remark the constants. 
    \begin{align*}
         &\text{Mass:}\ 0.1kg\\
         &\text{Equilibrium Distance ($y_{0}$):}\ 0.6258m
     .\end{align*}
     \bigbreak \noindent 
     Along with the distances $\Delta y$
     \begin{center}
         \begin{tabular}{c|c}
             Trial & $\Delta y$ (m) \\
             \hline
             1 & 0.06 \\
             2 & 0.08
         \end{tabular}
     \end{center}
     \tc{}

     \bigbreak \noindent 
     \subsubsection{Iteration 1 Data set}
     \bigbreak \noindent 
     The following data represents a sample of size 25
     \bigbreak \noindent 
     \begin{center}
         \begin{tabular}{c|c|c}
             Time (s) & Position (m) & Velocity (m/s) \\
             \hline
             0.0333	&-0.045619	&0.2781081081\\
             0.0666	&-0.0365295	&0.2912696029\\
             0.0999	&-0.026411	&0.3094381882\\
             0.1332	&-0.0159495	&0.3308971471\\
             0.1665	&-0.0042875	&0.3416266266\\
             0.1998	&0.007203	&0.3313263263\\
             0.2331	&0.017493	&0.333043043\\
             0.2664	&0.0293265	&0.3351889389\\
             0.2997	&0.0403025	&0.3038588589\\
             0.333	&0.0499065	&0.2446321321\\
             0.3663	&0.056595	&0.1811136136\\
             0.3996	&0.0619115	&0.1214577077\\
             0.4329	&0.0646555	&0.06652277277\\
             0.4662	&0.066199	&0.01888388388\\
             0.4995	&0.0660275	&-0.03347597598\\
             0.5328	&0.0639695	&-0.08755255255\\
             0.5661	&0.0601965	&-0.14119995\\
             0.5994	&0.054537	&-0.1922722723\\
             0.6327	&0.047334	&-0.2373360861\\
             0.666	&0.0385875	&-0.2703828829\\
             0.6993	&0.029155	&-0.2918418418\\
             0.7326	&0.019208	&-0.3128716216\\
             0.7659	&0.008232	&-0.3287512513\\
             0.7992	&-0.002744	&-0.3381931932
         \end{tabular}
     \end{center}
     \tc{The following table represents a subset from the full data set of size 25}




     \pagebreak 
     \subsubsection{Iteration 2 Data set}
     \bigbreak \noindent 
     The following data represents a sample of size 25
     \bigbreak \noindent 
     \begin{center}
         \begin{tabular}{c|c|c}
             Time (s) & Position (m) & Velocity (m/s) \\
             \hline
             0.0333	&-0.0818055	&0.3150175175\\
             0.0666	&-0.0715155	&0.3341875209\\
             0.0999	&-0.060025	&0.3669481982\\
             0.1332	&-0.046991	&0.4025700701\\
             0.1665	&-0.033271	&0.4416253754\\
             0.1998	&-0.0176645	&0.4841141141\\
             0.2331	&-0.0008575	&0.5132982983\\
             0.2664	&0.016807	&0.5163025526\\
             0.2997	&0.033957	&0.48626001\\
             0.333	&0.049392	&0.4399086587\\
             0.3663	&0.063112	&0.3896946947\\
             0.3996	&0.0759745	&0.3068631131\\
             0.4329	&0.083349	&0.2291816817\\
             0.4662	&0.0910665	&0.1583671171\\
             0.4995	&0.094325	&0.06952702703\\
             0.5328	&0.095354	&-0.003433433433\\
             0.5661	&0.093982	&-0.06523523524\\
             0.5994	&0.0910665	&-0.1326163664\\
             0.6327	&0.085407	&-0.218023023\\
             0.666	&0.0766605	&-0.3120132633\\
             0.6993	&0.064484	&-0.3939864865\\
             0.7326	&0.050078	&-0.4497797798\\
             0.7659	&0.0343	&-0.4849724725\\
             0.7992	&0.0176645	&-0.5034271772\\
             0.8325	&0.000343	&-0.4905518018
         \end{tabular}
     \end{center}
     \tc{The following table represents a subset from the full data set of size 25}








     \pagebreak 
     \section{Results}
     \bigbreak \noindent 

     \bigbreak \noindent 
     \subsection{Trial 1}
     \bigbreak \noindent 
     \subsubsection{Iteration 1: Graphs}
     \bigbreak \noindent 
     \fig{.5}{./figures/set1.png}
     \bigbreak \noindent 
     \subsubsection{Iteration 1: Sine Wave Fitting Equation}
     \begin{align*}
        &\text{Position vs time: } 0.0461\sin{(6.6025t + -1.8734)} \\
        &\text{Velocity vs time: } 0.2989\sin{(6.6021t + -0.2901)}
    .\end{align*}

    \bigbreak \noindent 
    \subsubsection{Iteration 2: Graphs}
    \bigbreak \noindent 
    \fig{.5}{./figures/set2.png}

    \bigbreak \noindent 
    \subsubsection{Iteration 2: Sine Wave Fitting Equation}
    \begin{align*}
        &\text{Position vs time: } 0.0634\sin{(6.6235t + 1.6918)} \\
        &\text{Velocity vs time: } -0.4148\sin{(6.6221t + 0.1449)}
    .\end{align*}

    \pagebreak 
    \subsection{Trial 2}

    \bigbreak \noindent 
    \subsubsection{Iteration 1: Graphs} 
    \fig{.5}{./figures/set3.png}


    \bigbreak \noindent 
\subsubsection{Iteration 1: Sine Wave Fitting Equation}
\begin{align*}
        &\text{Position vs time: } 0.0545\sin{(5.1754t + -0.9313)} \\
        &\text{Velocity vs time: } 0.0636\sin{(4.7262t + 1.9529)}
    .\end{align*}

    \bigbreak \noindent 
    \subsubsection{Iteration 2: Graphs} 
    \fig{.5}{./figures/set4.png}

    \bigbreak \noindent 
\subsubsection{Iteration 2: Sine Wave Fitting Equation}
\begin{align*}
        &\text{Position vs time: } 0.0773\sin{(5.1737t + -1.1965)} \\
        &\text{Velocity vs time: } 0.3971\sin{(5.1736t + 0.3824)}
    .\end{align*}

    \pagebreak 
    \subsection{Values of $A$, $\omega$, and $\phi$ (Position vs time)}
    \bigbreak \noindent 
    The following table lists the values as seen in the equations above
    \begin{center}
        \begin{tabular}{c|c|c|c}
            Iteration & Amplitude ($A$) & Frequency $(\omega)$ & Phase shift $(\phi)$ \\ 
            \hline
            1 & 0.0461 & 6.2025 & -1.8734 \\ 
            2 & 0.0634 & 6.235 & 1.6918 \\
            3 &0.0545 & 5.1754 & -0.9313\\
            4&0.0773 & 5.1737 & -1.1965 \\
        \end{tabular}
    \end{center}
    \tc{Shows the values of amplitude, frequency, and phase shift among each of the position vs time graphs}

    \bigbreak \noindent 
    \subsection{Values of $A$, $\omega$, and $\phi$ (Velocity vs time)}
    \begin{center}
        \begin{tabular}{c|c|c|c}
            Iteration & Amplitude ($A$) & Frequency $(\omega)$ & Phase shift $(\phi)$ \\ 
            \hline
            1 & 0.2989 & 6.2025 & -1.8734\\
            2 & -0.4148 & 6.6221 & 0.1449 \\
            3 & 0.0636 & 4.7262 & 1.9529 \\
            4 & 0.3971 & 5.1736 & 0.3824\\
        \end{tabular}
    \end{center}
    \tc{Shows the values of amplitude, frequency, and phase shift among each of the velocity vs time graphs}

    \pagebreak 
    \section{Data Analysis}
    \bigbreak \noindent 
        Througout all trials and iterations, the graphs for position vs time and velocity vs time resemble sinusodial oscillations. On each graph, a sine curve can be precisely fitted on the data points, and we can extract an equation in the form $A\sin{\left(\omega t + \phi\right)}$. However, they differ by the fact that the velocity vs time graphs seem to have faster oscillations.  
        \bigbreak \noindent 
        When velocity is zero, the mass has two possible locations. The first location would be when the mass is at its equilibrium position. The second is after the mass is lifted but before it is realeased and begins oscillations
        \bigbreak \noindent 
        Velocity is greatest at the start of the movement, as $t$ varies from 0 to 20, accerelation due to gravity is decelerating the movement
        \bigbreak \noindent 
        Furthermore, since both trials varied in mass, it follows that the equilibrium position relative to the motion device would differ. The heavier the mass, the farther the spring stretches thus briging it closer to the motion sensor. In our equation for simple harmonic motion, we have $\frac{mg}{k}$ as the shift in the equilibrium position due to gravity, as seen in the theory section. The larger the mass, the larger the numerator, hence we get a larger number.
        \bigbreak \noindent 
        If we examine tables seven and eight, we notice for the position vs time graph, as mass increases from 0.05kg to 0.1kg, the amplitude increases whereas the frequency and phase shift decreases. For the velocity vs time graphs, we notice the same.
        \bigbreak \noindent 
        Refering to equation 10, we find that our prediction of $\omega$ being fixed by $k$ and $m$ is verified.
        \bigbreak \noindent 
        Using equation 9, we find the spring constant. We show this computation for the first iteration and then list the results for the remaining three in a following table.
        \bigbreak \noindent 
        \nt{We evaluate the spring constant using the position vs time graphs. Recall that the mass for the first two iterations is $0.05kg$, whereas the mass for the last two iterations is $0.1kg$}
        \begin{align*}
            k &= m\omega^{2} \\
            \implies k &= 0.05(6.2025)^{2} = 1.9236 \frac{N}{m}
        .\end{align*}
        \bigbreak \noindent 
        \begin{center}
            \begin{tabular}{c|c}
                Iteration & Spring constant ($k$) \\
                \hline
                1 & 1.9236\\ 
                2 & 1.94376\\
                3 & 2.6785\\
                4  & 2.6767
            \end{tabular}
        \end{center}
        \tc{Shows the values found for the spring constant $k$ among each iteration}
        \bigbreak \noindent 
        From these results, we see that the spring constant is relatively the same in iterations where the mass is the same, but the spring constant seems to change when the mass increases. This discrepancy could be attributed to several factors
        \begin{itemize}
            \item \textbf{Measurement Errors:} Inaccuracies in measuring the oscillation frequency ($\omega$), the mass ($m$), or both can lead to different $k$ values. Precision in measurements is crucial.
            \item \textbf{Experimental Setup:} Variations in the setup for different masses, such as changes in the initial displacement or how the spring is mounted, can affect the system's behavior.
            \item \textbf{Non-ideal Spring Behavior:} Real springs may not perfectly obey Hooke's law, especially if they are stretched beyond their limit or if the spring has been deformed or damaged. This can make the spring appear to have a different $k$ under different masses.
            \item \textbf{Assumption of Simple Harmonic Motion (SHM):} The calculation assumes perfect SHM, where energy loss (e.g., due to air resistance or internal friction within the spring) is negligible. These factors can affect the oscillation and thus the computed $k$.
        \end{itemize}


    \bigbreak \noindent 
    \section{Discussion}
    \bigbreak \noindent 
    In this Simple Harmonic Motion (SHM) Lab, we find the relationship between the mass of an object attached to a vertically suspended spring, the spring constant, and the oscillation frequency and amplitude. Our findings mostly supported the theoretical model of SHM, demonstrating that the frequency of oscillation is indeed determined by the mass and the spring constant, as predicted by $\omega = \sqrt{(k/m)}$. This correlation highlights the principles of SHM, where an increase in mass results in a decrease in oscillation frequency due to the greater inertia the spring must overcome. Additionally, the experiment validated the theoretical prediction that the equilibrium position shifts with the addition of mass, demonstrating the impact of gravitational force on the system.
    \bigbreak \noindent
    However, the experiment also faced limitations similar to those encountered in the Atwood Machine Lab, such as the assumptions of an ideal system without air resistance or friction. These factors, along with potential experimental errors like misalignment of the motion detector or inaccuracies in mass measurement, could lead to discrepancies between our experimental results and theoretical predictions. 


    \bigbreak \noindent 
    \section{Conclusion}
    \bigbreak \noindent 
    This lab's investigation into simple harmonic motion provided a practical examination of the connection between mass, spring constant, and oscillation behavior, reinforcing principles through evidence. Our findings show the fundamental concepts of SHM, demonstrating the system's sensitivity to changes in mass and spring constant. While the experiment highlighted potential challenges in aligning theoretical predictions with real-world observations, highlighting the impact of assumptions and measurement errors, it also showed the importance of practical design in physics. Ultimately, this lab not only deepened our understanding of SHM but also illustrated the critical nature of accounting for external factors and potential discrepancies in experimental physics.











\end{document}
