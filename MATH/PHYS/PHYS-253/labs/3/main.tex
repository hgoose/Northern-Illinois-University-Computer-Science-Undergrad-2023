\documentclass{report}

\input{~/dev/latex/template/preamble.tex}
\input{~/dev/latex/template/macros.tex}

\title{\Huge{}}
\author{\huge{Nathan Warner}}
\date{\huge{}}
\fancyhf{}
\rhead{}
\fancyhead[R]{\itshape Warner} % Left header: Section name
\fancyhead[L]{\itshape\leftmark}  % Right header: Page number
\cfoot{\thepage}
\renewcommand{\headrulewidth}{0pt} % Optional: Removes the header line
%\pagestyle{fancy}
%\fancyhf{}
%\lhead{Warner \thepage}
%\rhead{}
% \lhead{\leftmark}
%\cfoot{\thepage}
%\setborder
% \usepackage[default]{sourcecodepro}
% \usepackage[T1]{fontenc}

% Change the title
\hypersetup{
    pdftitle={}
}

\begin{document}
    % \maketitle
    %     \begin{titlepage}
    %    \begin{center}
    %        \vspace*{1cm}
    %
    %        \textbf{Lab Report}
    %
    %        \vspace{0.5cm}
    %         Skyscaper
    %             
    %        \vspace{1.5cm}
    %
    %        \textbf{Nathan Warner}
    %
    %        \vfill
    %             
    %             
    %        \vspace{0.8cm}
    %      
    %        \includegraphics[width=0.4\textwidth]{~/niu/seal.png} \\
    %        
    %             
    %    \end{center}
    % \end{titlepage}
    \begin{center}
        \begin{Huge}
            Force Table Lab
        \end{Huge}
        \begin{Large}
            \bigbreak \noindent 
            Nate Warner
            \smallbreak \noindent
            February 13, 2024
            \bigbreak \noindent 
            Section 253D, 6:00 PM Tuesday 
        \end{Large}
    \end{center}
    \pagebreak 
    \tableofcontents
    \pagebreak \bigbreak \noindent 
    \begin{center}
    \textbf{Abstract}
    \end{center}
    \begin{adjustwidth}{.3in}{.3in}
        \hspace{\parindent} Every measurable quantity falls into one of two categories: scalar or vector. Scalars possess only magnitude, such as the number of students in a class, an object's mass, or its speed. Vectors, however, have both magnitude and direction, with velocity, force, and acceleration being prime examples. For instance, stating a car travels at 60 mph describes its speed (a scalar), but "60 mph due east" specifies both speed and direction, making it a vector (velocity). Unlike scalars, which combine through simple arithmetic, vector addition requires accounting for both magnitude and direction. In this lab, we'll explore how to find the resultant of multiple force vectors and practice vector addition both graphically and analytically. This lab focuses on the techniques we can perform on vector quantitys, specificially to determine the counterweight (equilibrant) needed to balance a force table. We start with two weights at arbitrary mass and angles, and deduce the thirds mass and position needed for equilibrium 
    \end{adjustwidth}

    \bigbreak \noindent 
    \section{Theory}
    \bigbreak \noindent 
    \subsection{Vector Representation}
    A vector quantity is characterized by both its magnitude and direction. It is typically depicted as an arrow, where the arrow's direction indicates the vector's direction and its length represents the vector's magnitude. In mathematical notation, a vector is denoted by \( \vec{A} \), or as a boldface letter \( \mathbf{A} \). The negative of a vector, \( -\vec{A} \), has the same magnitude but points in the opposite direction. In vector theory these two vectors are defined to be "anti-parallel", as illustrated in Figure 1.
    \bigbreak \noindent 
    For graphical representation, vectors are placed in a Cartesian coordinate system with their tail at the origin (standard position). The vector's direction is determined by the angle \( \theta \) (theta), measured from the positive x-axis to the vector, as depicted in Figure 2.
    \bigbreak \noindent 
    \begin{figure}[ht]
        \centering
        \incfig{fig1}
        \label{fig:fig1}
    \end{figure}
    \fc{Shows the "anti-parallel" relationship between two vectors $\vec{\mathbf{A}}$ and $\vec{\mathbf{B}}$.}
    % \begin{center}
    %     \textbf{Figure 1:} \textit{Shows the "anti-parallel" relationship between two vectors $\vec{\mathbf{A}}$ and $\vec{\mathbf{B}}$.}
    % \end{center}
    \pagebreak \bigbreak \noindent 
    \begin{figure}[ht]
        \centering
        \incfig{fig23}
        \label{fig:fig23}
    \end{figure}
    \bigbreak \noindent 
    \fc{Shows a vector $\vec{\mathbf{V}}$ placed at standard position}
    % \begin{center}
    %     \textbf{Figure 2:} \textit{Shows a vector $\vec{\mathbf{V}}$ placed at standard position}
    % \end{center}
    \bigbreak \noindent 
    \subsection{Components of Vectors}
    \bigbreak \noindent 
    An important technique when working with vectors mathematically is to break them down into their \(x\) and \(y\) components. Figure 3 shows he we can use trigonometry to find these components
    \bigbreak \noindent 
    \begin{figure}[ht]
        \centering
        \incfig{figure3}
        \label{fig:figure3}
    \end{figure}
    \bigbreak \noindent 
    \fc{Shows how we create a right triangle out of a vectors components}
    \bigbreak \noindent 
    We see from figure 3 that a vector can be deconstructed into $x$ and $y$ components to form a right triangle, from this we can use trigonometry to find values for $x$ and $y$
    \bigbreak \noindent 
    We know that for any right triangle, we can relate their sides and angles using the functions $\sin{\theta}$, $\cos{\theta}$, and $\tan{\theta}$. More generally, we know
    \begin{align*}
        \sin{\theta} &= \frac{opposite}{hypotenuse} \\
        \cos{\theta} &= \frac{adjacent}{hypotenuse} \\
        \tan{\theta} &= \frac{opposite}{adjacent}
    .\end{align*}
    \bigbreak \noindent 
    Refering back to figure 3, we now see that 
    \begin{align*}
        \cos{\theta} &= \frac{\mathbf{V}_{x}}{\vec{\mathbf{V}}} \implies \mathbf{V}_{x} = \vec{\mathbf{V}}\cos{\theta} \\
        \sin{\theta} &= \frac{\mathbf{V}_{y}}{\vec{\mathbf{V}}} \implies \mathbf{V}_{y} = \vec{\mathbf{V}}\sin{\theta} \\
        \tan{\theta} &= \frac{\mathbf{V}_{y}}{\mathbf{V}_{x}} \implies \theta = \tan^{-1}{\frac{\mathbf{V}_{y}}{\mathbf{V}_{x}}}
    .\end{align*}
    \bigbreak \noindent 
    We now have a way to breakdown a vector into $x$ and $y$ components given magnitude and angle $\theta$
    \bigbreak \noindent 
    \begin{remark}
        We note that the direction angle of a vector can be found using the tangent function described previously.
    \end{remark}
    

    \bigbreak \noindent 
    \subsection{Magnitude of a vector}
    \bigbreak \noindent 
    Furthermore, from figure 3 we see that given a vectors $x$ and $y$ components, we can use that fact that for a right triangle, the following equations holds 
    \begin{align*}
        a^{2} + b^{2} = c^{2}
    .\end{align*}
    Where $a,b$, and $c$ are the sides of the triangle (in order from shortest to longest). Thus, we find the magnitude of a vector with
    \begin{align*}
        \mathbf{V}_{x}^{2} + \mathbf{V}_{y}^{2} &= \vec{\mathbf{V}}^{2} \\
        \implies \sqrt{\mathbf{V}_{x}^{2} + \mathbf{V}_{y}^{2}} &= \vec{\mathbf{V}} 
    .\end{align*}
    \bigbreak \noindent 
    \subsection{Resultant}
    \bigbreak \noindent 
Finding the resultant of a vector is a straightforward process. Suppose we have two vectors $\vec{\mathbf{A}}$ and $\vec{\mathbf{B}}$, with components $\vec{\mathbf{A}} = \langle \mathbf{A}_{x}, \mathbf{A}_{y}\rangle$, $\vec{\mathbf{B}} = \left\langle \mathbf{B}_{x}, \mathbf{B}_{y} \right\rangle $. We define $\vec{\mathbf{R}} = \vec{\mathbf{A}} + \vec{\mathbf{B}}$ as the resultant vector, with components
    \bigbreak \noindent 
       \begin{equation}
            \begin{cases}
                \mathbf{R}_{x} &=  \mathbf{A}_{x} + \mathbf{B}_{x}\\
                \mathbf{R}_{y} &=  \mathbf{A}_{y} + \mathbf{B}_{y}
            \end{cases}
        \end{equation}
        \bigbreak \noindent 
        This is represented graphically in figure 4.
        \pagebreak \bigbreak \noindent 
    \begin{figure}[ht]
        \centering
        \incfig{figure4}
        \label{fig:figure4}
    \end{figure}
    \bigbreak \noindent 
    \fc{Shows how we can represent the resultant vector graphically. We place the tail of $\vec{\mathbf{B}}$ at the terminal point of $\vec{\mathbf{A}}$. Then draw the resultant vector with initial point at the origin and terminal point connecting with the terminal point of $\vec{\mathbf{B}}$ (The yellow vector).}
    \bigbreak \noindent 
    \subsection{Objective}
    \bigbreak \noindent 
    The objective of this experiment is to find the equilibrant of one or more known forces using a force table and compare the results to those obtained by analytical methods.
    \bigbreak \noindent 
    \subsection{Equipment}
    \begin{itemize}
        \item Force table
        \item Ruler
        \item Strings
        \item Weight hangers
        \item Assorted weights
        \item Bubble level
    \end{itemize}
    \bigbreak \noindent 
    \subsection{Force of the vectors}
    \bigbreak \noindent 
    In this lab, we make use of a force table (figure 5). On each pan, we have some mass (in kg). To find the force, we use Newton's second law
    \begin{align*}
        f = ma
    .\end{align*}
    Where $f$ is the force, $m$ is the mass, and $a$ is the acceleration due to gravity. If our mass is measured in kg, and the acceleration due to gravity is measured in $m/s^{2}$, the resulting computation will yield a result of units $kg \cdot m/s^{2}$, also know as the Newton ($N$).
    \pagebreak 
    \fig{.8}{./figures/1.png}
    \fc{Shows an example of a force table}
    \bigbreak \noindent 
    \subsection{Setup}
    \bigbreak \noindent 
    Given two force vectors, we will determine the third force that will produce equilibrium in the system. This third force, known as the equilibrant, will be equal and opposite to the resultant of the two known forces. we will use a force table, as shown in Figure 8, to work with force vectors. The force table is a circular platform mounted on a tripod stand, with adjustable screws on the tripod legs for leveling the platform. The platform features degree markings on its surface for angle measurements. Pulleys can be clamped at any position along the platform's edge; in this lab, we will use three pulleys. Strings are attached to a central ring, passed over the pulleys, and masses are hung on the other end.
    \bigbreak \noindent 
    The hanging masses generate a tension force in each string, proportional to the gravitational force, such that the tension force equals the gravitational force. For instance, doubling the mass doubles the force. When forces are balanced, the ring centers on the table; if unbalanced, the ring touches the central post's side. The force due to each hanging mass is given by \(mg\), where \(g\) is the acceleration due to gravity. To facilitate angle reading, assume the x-axis runs from the 180-degree mark to the 0-degree mark, with 0 degrees indicating the positive x direction, and the y-axis from the 270-degree mark to the 90-degree mark, with 90 degrees as the positive y direction. Refer to Figure 9.

    \pagebreak 
    \section{Raw Data}
    \bigbreak \noindent 
    This lab will be comprised of three trials. Each trial will have two force vectors with angle $\theta$ and mass $m\ kg$. The numbers for each trial are found in the following table.
    \bigbreak \noindent 
    \begin{center}
        \begin{tabular}{c|c|c|c|c}
            Trial & Vector 1 mass & Vector 1 angle & Vector 2 mass & Vector 2 angle \\
            \hline
            1 & 0.15kg & $60^{\circ}$ & 0.15kg & $300^{\circ}$\\
            2 & 0.15kg & $60^{\circ}$ & 0.2kg & $330^{\circ}$\\
            3 & 0.15kg & $30^{\circ}$ & 0.15kg & $330^{\circ}$
        \end{tabular}
    \end{center}
    \bigbreak \noindent 
    \tc{Shows the masses and angles used for both vectors in each trial}
    % \begin{center}
    %     \textbf{Table 1:} \textit{Shows the masses and angles used for both vectors in each trial}
    % \end{center}

    \bigbreak \noindent 
    \subsection{Hypothesis}
    \bigbreak \noindent 
    Before making any computations, we first attempt to estimate the mass and angle needed on vector three to achieve equilibrium. This is summarized in the following table. We analyze these initial guesses in section 3.4
    \bigbreak \noindent 
    \begin{center}
        \begin{tabular}{c|c|c}
            Trial & Mass & Angle \\
            \hline 
            1 &0.2kg & $180^{\circ}$\\
            2 & 0.25kg & $185^{\circ}$\\
            3 & 0.2kg & $180^{\circ}$\\
        \end{tabular}
    \end{center}
    \bigbreak \noindent 
    \tc{Shows the hypothesized mass and angle needed to achieve equilibrium with vector three}
    % \begin{center}
    %     \textbf{Table 2:} \textit{Shows the hypothesized mass and angle needed to achieve equilibrium with vector three}
    % \end{center}
    \bigbreak \noindent 

    \bigbreak \noindent 
    \section{Results}
    \smallbreak \noindent
    \subsection{Trial 1}
    \bigbreak \noindent 
    We have force vectors 
    \begin{align*}
        &\vec{\mathbf{F}_{1}} = 0.15\ kg(9.8\ m/s^{2}) = 1.5N, \quad \theta_{F_{1}} = 60^{\circ} \\
        &\vec{\mathbf{F}_{2}} = 0.15\ kg(9.8\ m/s^{2}) = 1.5N, \quad \theta_{F_{2}} = 300^{\circ}
    .\end{align*}
    \bigbreak \noindent 
    We deconstruct these vectors into their $x$ and $y$ components
    \begin{align*}
        \vec{\mathbf{F}}_{1x} &= 1.5N\cos{60^{\circ}} = 0.75N, \quad \vec{\mathbf{F}}_{1y} = 1.5N\sin{60^{\circ}} = 1.3N \\
        \vec{\mathbf{F}}_{2x} &= 1.5N\cos{300^{\circ}} = 0.75N, \quad \vec{\mathbf{F}}_{2y} = 1.5N\sin{300^{\circ}} = -1.3N 
    .\end{align*}
    We then find the resultant vector $\vec{\mathbf{R}} = \vec{\mathbf{F}}_{1} + \vec{\mathbf{F}}_{2} $
    \begin{align*}
        \vec{\mathbf{R}} &= \left\langle 0.75N + 0.75N,\ 1.3N + (-1.3)N \right\rangle \\
        \vec{\mathbf{R}} &= \left\langle 1.5N,\ 0N\right\rangle
    .\end{align*}
    With magnitude and direction angle
    \begin{align*}
        \norm{\vec{\mathbf{R}}} &= \sqrt{1.5^{2} + 0^{2}} =1.5N  \\
        \tan{\theta } &= \frac{\mathbf{R}_{y}}{\mathbf{R}_{x}} = \frac{0}{1.5} = 0 \\
        &\theta  = \tan^{-1}{0} = 0^{\circ}
    .\end{align*}
    With the magnitude, we can find the mass by dividing by the acceleration due to gravity ($g$). This is a reversal of Newton's second law. $f=ma \implies m = \frac{f}{a} $
    \begin{align*}
        m &= \frac{1.5N}{9.8m/s^{2}} = \frac{1.5 kg \cdot m/s^{2}}{9.8 m/s^{2}}\\
        &=0.15kg
    .\end{align*}
    Now, to find the vector needed to bring the force table to equilibrium we simply find the vector anti parallel to $\vec{\mathbf{R}}$. The magnitude will remain the same but the angle will become
    \begin{align*}
        &180^{\circ} + \theta \\
        &=180^{\circ} + 0^{\circ} =  180^{\circ}
    .\end{align*}
    \bigbreak \noindent 
    \textbf{Note:} We define the equilibrium vector $\vec{\mathbf{E}} = -\vec{\mathbf{R}}$, with components 
    \begin{align*}
        \left\langle -1.5N,\ 0N \right\rangle
    .\end{align*}
    \bigbreak \noindent 
    These findings are represented by figure \thefigtitle
    \pagebreak \bigbreak \noindent 
    \begin{figure}[ht]
        \centering
        \incfig{fig8}
        \label{fig:fig8}
    \end{figure}
    \fc{Shows the initial force vectors, the resultant vector, and the vector needed for equilibrium}
    % \begin{center}
    %     \textbf{Figure 6:} \textit{Shows the initial force vectors, the resultant vector, and the vector needed for equilibrium}     
    % \end{center}

    \bigbreak \noindent 
    \subsection{Trial 2}
    \bigbreak \noindent 
    Trial 2 is executed in the same way as trail 1. We find
    \begin{align*}
        \vec{\mathbf{R}} &= \left\langle 2.435N,\ 0.29N \right\rangle \\
        \norm{\vec{\mathbf{R}}} &= \sqrt{(2.43N)^{2} + (0.29N)^{2}} = 2.45N\\
        \theta &= \tan^{-1}{\frac{0.29N}{2.435N}} = 6.79^{\circ}
    .\end{align*}
    With mass
    \begin{align*}
        \frac{2.45N}{9.8 m/s^{2}} = 0.25kg
    .\end{align*}
    \bigbreak \noindent 
    We then define $\vec{\mathbf{E}} = -\vec{\mathbf{R}}$, this vector will have the same magnitude as the resultant vector ($2.45N$), but the angle becomes $180^{\circ} +  6.79^{\circ} = 186.79^{\circ}$. These results are found in the following figure
    \bigbreak \noindent 
    \textbf{Note:} $\vec{\mathbf{E}}$ has components 
    \begin{align*}
        \left\langle -2.435N,\ -0.29N \right\rangle
    .\end{align*}
    \pagebreak \bigbreak \noindent 
    \begin{figure}[ht]
        \centering
        \incfig{fig9}
        \label{fig:fig9}
    \end{figure}
    \fc{Shows the initial force vectors, the resultant vector, and the vector needed for equilibrium}
    % \begin{center}
    %     \textbf{Figure 7:} \textit{Shows the initial force vectors, the resultant vector, and the vector needed for equilibrium}     
    % \end{center}
    \bigbreak \noindent 
    \subsection{Trial 3}
    \bigbreak \noindent 
    Again, we follow the same steps in trial 1. We find 
    \begin{align*}
        \vec{\mathbf{R}} &= \left\langle 2.6N,\ 0N \right\rangle \\
        \norm{\vec{\mathbf{R}}} &= \sqrt{(2.6N)^{2} + (0N)^{2}} =2.6N \\
        \theta  &= \tan^{-1}{\frac{0N}{2.6N}} = 0^{\circ}
    .\end{align*}
    \bigbreak \noindent 
    With mass
    \begin{align*}
        &\frac{2.6N}{9.8m/s^{2}} \\
        &0.27kg
    .\end{align*}
    \bigbreak \noindent 
    We then define $\vec{\mathbf{E}} = -\vec{\mathbf{R}}$, this vector will have the same magnitude as the resultant vector, but the angle becomes $180^{\circ} +  0^{\circ} = 180^{\circ}$. These results are found in the following figure
    \bigbreak \noindent 
    \textbf{Note:} $\vec{\mathbf{E}}$ has components 
    \begin{align*}
        \left\langle -2.6N,\ 0N \right\rangle
    .\end{align*}
    \pagebreak \bigbreak \noindent 
    \begin{figure}[ht]
        \centering
        \incfig{fig11}
        \label{fig:fig11}
    \end{figure}
    \bigbreak \noindent 
    \fc{Shows the initial force vectors, the resultant vector, and the vector needed for equilibrium}


    \bigbreak \noindent 
    \subsection{Data analysis}
    \bigbreak \noindent 
    The results found during the lab are summarized in the following table
    \bigbreak \noindent 
    \begin{center}
        \begin{tabular}{c|c|c}
            Trial & Mass Needed & Angle measure \\
            \hline 
            1 & 0.15kg & $180^{\circ}$\\
            2 & 0.25kg & $186.79^{\circ}$\\
            3 & 0.27kg & $180^{\circ}$\\
        \end{tabular}
    \end{center}
    \tc{Displays the mass and angle found for the equilibrant vector in each trial}
    \bigbreak \noindent 
    We then refer back to table 2. To calculate the percent difference, we use the following formula. If $A$ is the experimental value, and $B$ is the actual value, then the percent difference is given by
    \begin{align*}
        \text{Percent difference } = \frac{\abs{A - B}}{\frac{A+B}{2}} \cdot 100\%
    .\end{align*}
    \bigbreak \noindent 
    For trial one, we have 
    \begin{align*}
        \text{Percent difference}_{M} &= \frac{\abs{0.2 - 0.15}}{\frac{0.2 + 0.15}{2}}  \cdot 100\% = 28.6\%\\
        \text{Percent difference}_{A} &= \frac{\abs{180-180}}{\frac{180+180}{2}} \cdot 100\% = 0\% \\
    .\end{align*}
    \bigbreak \noindent 
    We then do the same for the last two trials, the findings are summarized in the following table
    \bigbreak \noindent 
    \begin{center}
        \begin{tabular}{c|c|c}
            Trial & Mass Percent Difference & Angle Percent Difference \\
            \hline 
            1 & 28.6\% & 0\%\\
            2 & 0\% & 1\%\\
            3 & 30\% & 0\%
        \end{tabular}
    \end{center}


    \bigbreak \noindent 
    \subsection{Colab}
    \bigbreak \noindent 
    \figcapman{.5}{./figures/2.png}{Shows the variables declared to be used in the computations}{9}
    \bigbreak \noindent 
    \figcapman{.5}{./figures/3.png}{Shows the computations for trial 1}{10}
    \bigbreak \noindent 
    \figcapman{.5}{./figures/4.png}{Shows the computations for trial 2}{11}
    \bigbreak \noindent 
    \figcapman{.5}{./figures/5.png}{Shows the computations for trial 3}{12}
    \bigbreak \noindent 











    \bigbreak \noindent 
    \section{Discussion}
    \bigbreak \noindent 
    The experiment aimed to explore vector addition and equilibrium through the use of a force table, comparing calculated values for equilibrant forces with those determined experimentally. The percent differences between calculated and experimental values for mass across the three trials were 28.6\%, 0\%, and 30\%, respectively, with angle percent differences being relatively low, at 0\%, 1\%, and 0\%.
    \bigbreak \noindent 
    There are a couple obvious factors that could account for the discrepancies between the calculated and experimental values. First, the precision of the equipment, including the accuracy of the mass measurements and the sensitivity of the angle measurements on the force table, could have contributed to the variance. Small errors in positioning the masses or reading the angles could have impacted the results. 
    \bigbreak \noindent 
    Additionally, the assumption of ideal conditions does not account for real-world factors such as friction or air resistance, which could affect the force on the hanging masses. 
    \bigbreak \noindent 
    \bigbreak \noindent 
    Before conducting the calculations, the outcomes were not entirely obvious, especially the angle needed for equilibrium. While the general direction was quite obvious, the exact angle and mass was non-trivial. This highlights the importance of using mathematical techniques to correctly compute the precise force and angle needed, rather than relying on estimation and guesswork.
    \bigbreak \noindent 

    \bigbreak \noindent 
    \section{Conclusion}
    \bigbreak \noindent 
    This lab provided a hands-on opportunity to explore vector addition and equilibrium. I found the process of determining the equilibrant force interesting, it offered a clear demonstration of the principles discussed in class. 
    \bigbreak \noindent 
    This experience definitely enhanced my understanding of vectors, giving a new perspective to the knowledge gained from the course. 
    \bigbreak \noindent 
    % Overall, the lab was an enriching experience, blending theory with practice in a manner that was both informative and enjoyable. It underscored the importance of meticulous experimental setup and the value of integrating theoretical knowledge with practical skills in physics.
        In conclusion, the experiment successfully demonstrated the principles of vector addition and equilibrium. The deviation between the calculated and experimental values offered insight into the nature of experimental physics. 











    
\end{document}
