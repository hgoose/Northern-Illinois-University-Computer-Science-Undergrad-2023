\documentclass{report}

\input{~/dev/latex/template/preamble.tex}
\input{~/dev/latex/template/macros.tex}

\title{\Huge{}}
\author{\huge{Nathan Warner}}
\date{\huge{}}
\fancyhf{}
\rhead{}
\fancyhead[R]{\itshape Warner} % Left header: Section name
\fancyhead[L]{\itshape\leftmark}  % Right header: Page number
\cfoot{\thepage}
\renewcommand{\headrulewidth}{0pt} % Optional: Removes the header line
%\pagestyle{fancy}
%\fancyhf{}
%\lhead{Warner \thepage}
%\rhead{}
% \lhead{\leftmark}
%\cfoot{\thepage}
%\setborder
% \usepackage[default]{sourcecodepro}
% \usepackage[T1]{fontenc}

% Change the title
\hypersetup{
    pdftitle={}
}

\begin{document}
    % \maketitle
    %     \begin{titlepage}
    %    \begin{center}
    %        \vspace*{1cm}
    %
    %        \textbf{Lab Report}
    %
    %        \vspace{0.5cm}
    %         Skyscaper
    %             
    %        \vspace{1.5cm}
    %
    %        \textbf{Nathan Warner}
    %
    %        \vfill
    %             
    %             
    %        \vspace{0.8cm}
    %      
    %        \includegraphics[width=0.4\textwidth]{~/niu/seal.png} \\
    %        
    %             
    %    \end{center}
    % \end{titlepage}
    \begin{center}
        \begin{Huge}
            Ballistic Pendulum
        \end{Huge}
        \begin{Large}
            \bigbreak \noindent 
            Nate Warner
            \smallbreak \noindent
            April 5, 2024
            \bigbreak \noindent 
            Section 253D, 6:00 PM Tuesday 
        \end{Large}
    \end{center}
    \pagebreak 
    \tableofcontents
    \pagebreak \bigbreak \noindent 
    \begin{center}
    \textbf{Abstract}
    \end{center}
    \begin{adjustwidth}{.3in}{.3in}
        \hspace{\parindent} 
    \end{adjustwidth}

    \bigbreak \noindent 
    \section{Theory}
    \bigbreak \noindent 
    The aim of this experiment is to explore the interplay between kinetic and potential energy by launching a ball into a pendulum cup and quantifying the cup's ascent. Previously, we delved into projectile motion, focusing on how the travel distance of a ball is influenced by its initial speed and height. The formula for calculating the range, given the initial velocity and height, is:
    \begin{equation}
        R=v_0 \sqrt{\frac{2 h}{g}}
    \end{equation}
    \bigbreak \noindent
    To deduce the initial velocity from the range, we rearrange the above equation as follows:
    \begin{equation}
        v_0=R \sqrt{\frac{g}{2 h}}
    \end{equation}
    \bigbreak \noindent
    This allows us to compute the ball's initial momentum upon launch, utilizing the momentum definition:
    \begin{equation}
        p_i=m_{\text {ball }} v_0
    \end{equation}
    \bigbreak \noindent
    Upon collision between the ball and the cup, momentum is conserved despite energy loss, leading to the equation:
    \begin{equation}
        m_b v_0=\left(m_b+m_c\right) v_i
    \end{equation}
    \bigbreak \noindent
    Here, $m_b$ and $m_c$ represent the masses of the ball and cup, respectively, and $v_i$ is their combined velocity post-collision. Solving for $v_i$, we obtain:
    \begin{equation}
        v_i=\frac{m_b v_0}{m_b+m_c}
    \end{equation}
    \bigbreak \noindent
    Next, we calculate the system's initial kinetic energy, which, in the absence of further collisions, should match the potential energy at the peak of the swing. The kinetic and potential energy equations are:
    \begin{equation}
        \begin{split}
            K=\frac{1}{2} m v^2 \\
            U=m g h
        \end{split}
    \end{equation}
    \bigbreak \noindent
    Energy conservation implies:
    \begin{equation}
        (K+U)_i=(K+U)_f
    \end{equation}
    \bigbreak \noindent
    Initially, with $h=0$, potential energy is negligible. Conversely, at the peak, kinetic energy is null. Calculating $K_i$ and $U_f$:
    \begin{equation}
        \begin{split}
            K & =\frac{1}{2}\left(m_b+m_c\right)\left(\frac{m_b v_0}{m_b+m_c}\right)^2 \\
              & =\frac{1}{2} \frac{m_b^2 v_0^2}{m_b+m_c}
        \end{split}
    \end{equation}
    \bigbreak \noindent
    And potential energy as:
    \begin{equation}
        U=\left(m_b+m_c\right) g h
    \end{equation}
    \bigbreak \noindent
    We compare these energies using the percent difference formula:
    \begin{equation}
        \% \text { Diff }=\frac{K-U}{\frac{1}{2}(K+U)} \times 100 \%
    \end{equation}


    \bigbreak \noindent
    \subsection{Appratus used in the experiment}
    \bigbreak \noindent 
    \fig{1}{./figures/1.png}
    \fc{Displays the device used in the lab}
    \bigbreak \noindent 
    Observing Figure \thefigtitle, it's evident that the setup includes a spring-loaded launcher, a ball, a cup attached to a pendulum capable of swinging freely, and a supporting mount, with each element identified from right to left in the illustration. The experiment's goal is to verify the principle of energy conservation. To achieve this, we must ascertain the maximum height attained by the cup and ball as they swing upwards. This measurement can be straightforwardly obtained by counting the notches the cup ascends to on the mount; a lower position corresponds to fewer notches. For convenience, a table is provided to correlate the number of notches with the specific height reached by the cup, as detailed further below.
    \bigbreak \noindent 
    \begin{center}
        \begin{tabular}{c|c}
            Notch Measurement & Height (mm) \\
            \hline
            0 &65 \\ 
            10 & 73\\ 
            20  & 80\\
            30 & 88\\ 
            40  &96
        \end{tabular}
    \end{center}
    \tc{Shows the relationship Between Notch Measurement and Height.}
    \bigbreak \noindent 
    The table provides values only for the cup positioned at $0, 10, 20, 30$, or $40$ notches. If the cup halts at a notch count not explicitly listed, such as somewhere between these specified values, we employ interpolation to estimate the corresponding height. This method involves creating a proportional relationship using the known values from the table to calculate the unknown height. Suppose the cup stops at notch $x$, situated between two known notch counts $n_1$ (the lower count) and $n_2$ (the higher count), with their respective heights being $h_1$ and $h_2$. The task is to find the height $y$ that aligns with notch $x$.
    \bigbreak \noindent
    We establish a ratio, akin to finding a percentage. The ratio's left side compares the distance between $n_1$ and $x$ to the distance between $n_1$ and $n_2$, expressed as
    \begin{equation}
        \frac{x-n_1}{n_2-n_1}
    \end{equation}
    \bigbreak \noindent
    This ratio should mirror the relationship between the corresponding heights on the right side, thus
    \begin{equation}
        \frac{x-n_1}{n_2-n_1}=\frac{y-h_1}{h_2-h_1}
    \end{equation}
    \bigbreak \noindent
    For a clearer understanding, consider the cup stopping at $24$ notches. Here, $n_1$ is $20$ (with a height of $80 \, \mathrm{cm}$) and $n_2$ is $30$ (with a height of $88 \, \mathrm{cm}$). To find the height $y$ corresponding to $24$ notches, we set up the ratio as follows:
    \begin{equation}
        \frac{24-20}{30-20}=\frac{y-80}{88-80}
    \end{equation}
    Solving this equation allows us to determine $y$, the height corresponding to the $24$ notch count.


    \bigbreak \noindent 
    \subsection{Procedure}
    \begin{enumerate}
        \item Start by making sure that the cup, and only the cup, is stuck on the mount. With the ball attached to the end of the spring gun, push the ball back so that the spring compresses (it may take some strength). Pull the trigger and watch where the ball initially hits the ground. Run this at least four times to get a good idea of where the ball will land.
        \item Tape a piece of white paper on the ground, with the center of the paper in the same general spot as where the ball hits the ground. Tape a piece of carbon paper on top of the white paper with the black side facing the ground. Shoot the ball in the same manner as Step 1 a total of ten times.
        \item Remove the carbon paper. You should have 10 distinct marks on the white paper. Measure from the end of the table to each of the marks. Add up each of the measurements and take the average. Use the average distance and add it to the distance from the end of the table to where the ball initially starts its flight to get the total range.
        \item To get the height that the ball falls, simply measure from the floor to the height where the ball begins its projectile motion.
        \item Remove the cup from the mount by pushing up to the top of the mount until it is free. Once it is free, push it to the side of the mount and guide its path back to the central point. Again, compress the spring gun. Once the cup is stationary, shoot the ball into the cup and record the number of notches that the ball and the cup travel.
        \item Repeat Step 5 a total of ten times and take the average number of notches.
        \item Measure the mass of the ball mb by using a scale. Record the mass of the cup, which is written on the apparatus.
    \end{enumerate}







    \pagebreak 
    \section{Data}
    \bigbreak \noindent 
    We start this experiment by measuring a few key items that are essential in our computations. We measure the mass of the cup $(m_{c})$, the mass of the ball $(m_{b})$, the height between the ground and the starting location of the ball $(h)$
    \bigbreak \noindent 
    \begin{center}
        \begin{tabular}{c|c}
            \hline
            $m_{c}$ & 0.2635\ (kg) \\
            $m_{b} $ & 0.0648\ (kg)\\
            $h $ & 0.847\ (m)
        \end{tabular}
    \end{center}
    \tc{Displays the initial three measurements taken before beginning the experiment}
    \bigbreak \noindent 
    With these measurements, we further the experiment by shooting the ball ten times and seeing where it lands on the floor. After all ten iterations, we determine the mean distance that the ball travaled, we will call this measurement $R$ and we find it to be roughly $0.868m$
    \bigbreak \noindent 
    Next, we shoot the ball an additional ten times, but instead of letting it hit the floor, we shoot the ball into the cup and record the number of notches the ball and cup travel. The findings for this stage of the experiment are expressed in the following table.
    \bigbreak \noindent
    \begin{center}
        \begin{tabular}{c|c}
            Iteration  &  Noches \\
            \hline
            1&10 \\
            2& 5\\ 
            3& 8\\
            4& 10\\
            5&6 \\
            6& 8\\
            7& 11\\
            8& 8\\
            9& 11\\
            10&12
        \end{tabular}
    \end{center}
    \tc{Displays the relationship between the number of notches and the height measurement}
    \bigbreak \noindent 
    We find the mean number of notches to be
    \begin{align*}
        \bar{x} &= \frac{\sum x_{i}}{n} \\
        &=8.9\ \text{notches}
    .\end{align*}
    We denote $\bar{x} = \bar{n} = 8.9\ \text{notches}$


    \bigbreak \noindent 
    \section{Results}
    \bigbreak \noindent 
    Now that we have the data required to analyze the results of this experiment, we begin by using equation two to compute the initial velocity of the ball after it leaves the spring gun. Recall that we found $R=0.868m$
    \begin{align*}
        v_0&=R \sqrt{\frac{g}{2 h}} \\
           &=0.868m \cdot \sqrt{\frac{9.8 m/s^{2}}{2(0.847m)}} \\
           &\approx 2.088m/s
    .\end{align*}
    \bigbreak \noindent 
    We can then find the kinetic eneregy of the ball and the cup. By equation 8
    \begin{align*}
        K & =\frac{1}{2} \frac{m_b^2 v_0^2}{m_b+m_c} \\
          &= \frac{\frac{1}{2}(0.0648kg)^{2}(2.088 m/s)^{2}}{0.0648kg+ 0.2635kg} \\
          &=0.02788J
    .\end{align*}
    \bigbreak \noindent 
    Next, using the result we found for the mean number of notches ($\bar{n}=8.9$), we can use table one equation twelve to find the corresponding height.
    \begin{align*}
        &\frac{x-n_1}{n_2-n_1}=\frac{y-h_1}{h_2-h_1} \\
         &\implies\frac{8.9-0}{10-0} = \frac{y-65}{73-65} \\
         &\implies\frac{8.9}{10} = \frac{y-65}{8}  \\
         &\implies 8.9(8) = 10y - 605 \\
         &\implies 71.2 + 650 = 10y \\
         &\implies y = \frac{71.2+650}{10} \\
         &\implies y=72.12mm = 0.07212m
    .\end{align*}
    \bigbreak \noindent 
    With this and equation nine, we find the potential energy of the system
    \begin{align*}
        U&=\left(m_b+m_c\right) g y \\
         &=(0.0648kg + 0.2635kg) \cdot 9.8 m/s^{2} \cdot 0.07212m \\
         &=0.232J
    .\end{align*}
    \bigbreak \noindent 
    Lastly, using equation 10, we find the percent difference between the kinetic energy and potential energy.
    \begin{align*}
        \% \text { Diff }&=\frac{K-U}{\frac{1}{2}(K+U)} \times 100 \% \\
                         &= \frac{0.02788J - 0.232J}{\frac{1}{2}(0.02788J + 0.232J)} \cdot 100\% \\
                         \approx 157\%
    .\end{align*}

    \bigbreak \noindent 
    \section{Discussion}
    \bigbreak \noindent 
    In this experiment, we aimed to demonstrate the conservation of momentum and the transformation of kinetic energy into potential energy. Theoretically, this should result in equivalent initial kinetic and final potential energies. However, our findings showed a significant percent difference of approximately 157\% between these energies. 
    \bigbreak \noindent 
    This big difference could be because of a few simple things we might not have fully considered. First, not all the ball's energy might have gone into making the pendulum swing higher. Some energy could get lost when the ball hits the cup, but we didn't account for where this energy goes. Also, the pendulum swinging through the air faces air resistance, and there's some friction where it pivots, which can slow it down a bit and make it not reach as high.
    \bigbreak \noindent 
    Another reason for the difference could be how we measured the height the pendulum reached. We used a method to estimate the height based on the number of notches the pendulum moved, but this method isn't perfect. It assumes each notch represents the same height increase, which might not be exactly true.
    \bigbreak \noindent 
    To get better results next time, we could try to be more careful with how we measure the pendulum's height and think about other simple ways energy might be getting lost or not fully used to lift the pendulum. Even doing the experiment a few more times could help us get a more accurate average result.

    \bigbreak \noindent 
    \section{Conclusion}
    \bigbreak \noindent 
    In conclusion, our ballistic pendulum experiment aimed to explore the principles of kinetic and potential energy conservation through a hands-on approach. While the theoretical framework suggested a direct transformation of the ball's kinetic energy into the pendulum's potential energy, our results indicated a significant difference. This percent difference shows the difference between theoretical and real-world physics, where ideal conditions are hard to replicate. Factors such as energy loss during the collision, air resistance, and friction at the pendulum's pivot point played a crucial role in the observed outcome. However, this experiment provided important insights into the dynamics of energy transformation and conservation laws.










    
\end{document}
