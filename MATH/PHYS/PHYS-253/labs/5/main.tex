\documentclass{report}

\input{~/dev/latex/template/preamble.tex}
\input{~/dev/latex/template/macros.tex}

\title{\Huge{}}
\author{\huge{Nathan Warner}}
\date{\huge{}}
\fancyhf{}
\rhead{}
\fancyhead[R]{\itshape Warner} % Left header: Section name
\fancyhead[L]{\itshape\leftmark}  % Right header: Page number
\cfoot{\thepage}
\renewcommand{\headrulewidth}{0pt} % Optional: Removes the header line
%\pagestyle{fancy}
%\fancyhf{}
%\lhead{Warner \thepage}
%\rhead{}
% \lhead{\leftmark}
%\cfoot{\thepage}
%\setborder
% \usepackage[default]{sourcecodepro}
% \usepackage[T1]{fontenc}

% Change the title
\hypersetup{
    pdftitle={}
}

\begin{document}
    % \maketitle
    %     \begin{titlepage}
    %    \begin{center}
    %        \vspace*{1cm}
    %
    %        \textbf{Lab Report}
    %
    %        \vspace{0.5cm}
    %         Skyscaper
    %             
    %        \vspace{1.5cm}
    %
    %        \textbf{Nathan Warner}
    %
    %        \vfill
    %             
    %             
    %        \vspace{0.8cm}
    %      
    %        \includegraphics[width=0.4\textwidth]{~/niu/seal.png} \\
    %        
    %             
    %    \end{center}
    % \end{titlepage}
    \begin{center}
        \begin{Huge}
            Projectile Motion Lab Report
        \end{Huge}
        \begin{Large}
            \bigbreak \noindent 
            Nate Warner
            \smallbreak \noindent
            February 29, 2024
            \bigbreak \noindent 
            Section 253D, 6:00 PM Tuesday 
        \end{Large}
    \end{center}
    \pagebreak 
    \tableofcontents
    \pagebreak \bigbreak \noindent 
    \begin{center}
    \textbf{Abstract}
    \end{center}
    \begin{adjustwidth}{.3in}{.3in}
        \hspace{\parindent}  
        This lab involves the analysis of real world projectile motion through two experimental phases, focusing on the relationship between a projectile's vertical and horizontal positions, and determining its range. In the first phase, we find the projectile's vertical position ($y$) as a function of its horizontal position ($x$) using kinematic equations, considering the initial velocity in the x-direction and the effect of gravity. This phase directly establishes an equation for $y$ in terms of $x$, involving the initial height (h) from which the projectile was launched. The second phase aimed to calculate the projectile's range (R), using initial conditions and kinematic principles to find an equation for the range based on the initial velocity and height. Experimental procedures involved dropping a ball down a ramp and through a tube, measuring distances and times to calculate velocities and ranges. Data analysis included plotting measurements, fitting polynomial equations, and calculating percent errors to evaluate the accuracy of theoretical predictions against experimental results. 
    \end{adjustwidth}
    \bigbreak \noindent 
    \bigbreak \noindent 
    \section{Theory}
    \bigbreak \noindent 
    \begin{figure}[ht]
        \centering
        \incfig{fig1}
        \label{fig:fig1}
    \end{figure}
    \fc{Shows an example of simple projectile motion starting at $x=0$, $y=h$ and ending at $y=0$, $x=R$}
    \bigbreak \noindent 
    Initially, our goal is to determine the projectile's vertical position (y) based on its horizontal position (x), beginning with an analysis of kinematic equations in both the x and y directions.
    \begin{equation}
        \begin{split}
            x(t) &= x_{0} + v_{0,x}t + \frac{1}{2}a_{x}t^{2}  \\
            y(t) &= y_{0} + v_{0,y}t + \frac{1}{2}a_{y}t^{2} 
        \end{split}
    \end{equation}
    \bigbreak \noindent 
    The kinematic principles remain unchanged; however, our experimental setup imposes specific conditions. As illustrated in Figure 1, launching a projectile with initial velocity solely in the $x$ direction results in the $y$ velocity becoming zero. Furthermore, since gravity acts only downwards without a horizontal component, the $x$ acceleration also becomes zero. Consequently, this simplifies our equations to
    \bigbreak \noindent 
    \begin{equation}
        \begin{split}
            x(t) &= x_{0} + v_{0}t \\
            y(t)&=y_{0} -\frac{1}{2}gt^{2}
        \end{split}
    \end{equation}
    \bigbreak \noindent 
    By initializing the $x$ position to zero, we simplify to $x = vt$. Solving for time allows us to incorporate this into the second equation for the $y$ direction.
    \begin{equation}
        \begin{split}
            y(t) &= y_{0} -\frac{1}{2}g\left(\frac{x}{v}\right)^{2}  \\
            &=y_{0}  - \frac{g}{2v^{2}}x^{2}
        \end{split}
    \end{equation}
    \bigbreak \noindent 
    This yields a formula resembling $y = ax^2 + bx + c$ with $b = 0$. The initial vertical position, $y_0$, representing the starting height of the ball, is denoted as $h$. Consequently, we arrive at the final expression for $y$ in terms of $x$:
    \bigbreak \noindent 
    \begin{equation}
       y(x) = -\frac{g}{2v^{2}} x^{2} + h
    \end{equation}
    \bigbreak \noindent 
    The subsequent phase of the lab aims to determine the range, or the horizontal distance the ball covers. Starting from the initial conditions outlined in (1), with $v_{0,y} = 0$ and $a_x = 0$, we maintain our equations as presented in (2). When the ball reaches the ground, its $y$ position becomes $y(t) = 0$, and the horizontal distance it has covered is denoted as $x(t) = R$, where $R$ represents the range. Thus, equation (2) is modified accordingly.
    \begin{equation}
       \begin{split}
           R &= v_{0,x}t \\
           0 &= h-\frac{1}{2}gt^{2}
       \end{split}
    \end{equation}
    \bigbreak \noindent 
    In this step, by solving the equation for $y$ in terms of time $t$, we can estimate the projectile's range using only our equipment measurements, eliminating the need to directly measure airtime. By rearranging the equation to isolate $t$ after shifting the height $h$ across, we obtain
    \begin{equation}
       t = \sqrt{\frac{2h}{g}} 
    \end{equation}
    \bigbreak \noindent 
    Next, We put this into the range equation we have in (5), this gives 
    \begin{equation}
       R = v_{0,x}\sqrt{\frac{2h}{g}} 
    \end{equation}
    \bigbreak \noindent 
    This part describes a method to calculate the ball's range without directly timing its flight. However, measuring the initial velocity is necessary. The lab setup includes a tube and photogates to determine the time the ball takes to pass between them, enabling the calculation of the initial $x$ velocity.
    \begin{equation}
       v_{0,x}  = \frac{\Delta x}{\Delta t}
    \end{equation}
    \bigbreak \noindent 
    where $\Delta x$ is the distance between photogates and $\Delta t$ is the time it takes to go between them.

    \bigbreak \noindent 
    \section{Data}
    \bigbreak \noindent 
    \subsection{Part one}
    \bigbreak \noindent 
    The data collected in part one can be found in the following table
    \bigbreak \noindent 
    \begin{center}
        \begin{tabular}{c|c}
            Horizontal (cm) & Height (cm) \\
        \hline
            0 & 28.5\\
            2 & 28.2\\
            4 & 27.8\\
            6 & 27.1\\
            8 & 26.7\\
            10 & 25.2
        \end{tabular}
    \end{center}
    \tc{Shows the data collected in part one of the experiment}

    \bigbreak \noindent 
    \subsection{Part One: Colab code}
    \bigbreak \noindent 
    \fig{.5}{./figures/1.png}
    \fc{Shows the code used to generate the scatter plot describing data found in part one of the lab}

    \bigbreak \noindent 
    \subsection{Part Two}
    \bigbreak \noindent 
    The distance between the photogates was measured to be 87cm. The height from the floor to the tubing was measured to be 79.5cm. The time it took for the ball bearing to pass through the gap in the photogates is expressed in the following table
    \bigbreak \noindent 
    \begin{center}
        \begin{tabular}{c|c|c}
            Trial & $t_{1} (s)$ & $t_{2} (s)$ \\
            \hline
            1 & 1.06 & 1.54  \\
            2 & 0.74& 1.23  \\
            3 & 0.81 & 1.39 \\
            4 & 2.68 & 3.23 \\
            5 & 1.24 & 1.74
        \end{tabular}
    \end{center}
    \tc{Lists the time it takes for the bearing to pass through the gap between the two photogates in five trials}
    \bigbreak \noindent 

    \bigbreak \noindent 
    \section{Results}
    \bigbreak \noindent 
    \subsection{Part one: Colab Graph and Equation of line of best fit}
    \bigbreak \noindent 
    \fig{.6}{./figures/2.png}
    \fc{Shows the scatter plot of data found in part one, the red line represents the line of best fit.}
    \bigbreak \noindent 
    We see the equation given by 
    \begin{align*}
        y = -0.03x^{2} -0.04 + 28.44
    .\end{align*}
    \bigbreak \noindent 
    The fitted equation, \(y = -0.03x^2 - 0.04x + 28.44\), portrays similarities to equation 4, \(y(x) = -\frac{g}{2v^2} x^2 + h\). It shows that the ball's path is a parabolic curve, which is what we expected. The \(x\) term in the fitted equation could be possibly be attributed to some small things the simple theory didn't cover. 

    \bigbreak \noindent 
    \subsection{Part two}
    \bigbreak \noindent 
    Using table two, we now find $\Delta t$ for each trial. Each $\Delta t$ is found using the following equation
    \begin{align*}
        \Delta t = t_{2} - t_{1}
    .\end{align*}
    \bigbreak \noindent 
    \begin{center}
        \begin{tabular}{c|c|c|c}
            Trial & $t_{1} (s)$ & $t_{2} (s) $ & $\Delta t$ \\
            \hline
            1 & 1.06 & 1.54 &0.48 \\ 
            2 & 0.74& 1.23 & 0.49\\
            3 & 0.81 & 1.39 &0.58\\
            4 & 2.68 & 3.23 & 0.56\\
            5 & 1.24 & 1.74 & 0.50
        \end{tabular}
    \end{center}
    \tc{Shows $\Delta t$ for each trial}
    \bigbreak \noindent 
    Thus, we have an average time $(\Delta x)_{\text{avg}} = \frac{\sum \Delta x}{n} = 0.521s$ 
    \bigbreak \noindent 
    With this information, we use equation eight to find initial velocities. Recall from \textit{data} that $\Delta x$ was measured to be 0.87m, this is the distance between the two photogates. For the first trial, we find
    \begin{align*}
        v_{0,x} &= \frac{\Delta x}{\Delta t}\\
        &=\frac{.87m}{0.48s} = 1.81 m/s
    .\end{align*}
    \bigbreak \noindent 
    The velocities found for the remaining four $\Delta t$ calculations are expresesd in the following table
    \bigbreak \noindent 
    \begin{center}
        \begin{tabular}{c|c|c}
            Trial & $\Delta t$ & $v_{0,x}$\\
        \hline
            1 & 0.48 & 1.81\\
            2 & 0.49 & 1.78\\
            3 & 0.58 & 1.5\\
            4  & 0.56 & 1.55\\
            5 & 0.50 & 1.74
        \end{tabular}
    \end{center}
    \tc{Shows $v_{0,x}$ for each trial}
    \bigbreak \noindent 
    Thus, we have an average velocity
    \begin{align*}
        (v_{0,x})_{\text{avg}} = \frac{\sum v_{0,x}}{n} = 1.68 m/s
    .\end{align*}
    \bigbreak \noindent 
    Next, using equation seven and our average velocity, we find our expected range to be
    \begin{align*}
        R &= v_{0,x}\sqrt{\frac{2h}{g}} \\
        &=1.7 m/s \cdot \sqrt{\frac{2(0.795 m)}{9.8 m/s^{2}}} = 0.672 m
    .\end{align*}
    \bigbreak \noindent 
    Recall that we found the height from the ground to the tubing to be $h=0.795m$
    \bigbreak \noindent 
    After ten trials of dropping the ball down the tube, letting it hit the carbon paper on the ground, and recording each distance, we find the average range to be
    \begin{align*}
        R = .615m
    .\end{align*}
    \bigbreak \noindent 
    \subsubsection{Percent Error}
    \bigbreak \noindent 
    Now that we have our calculated and measured range, we find the percent error. If $A$ is the calculated value and $B$ is the measured value, then the percent error is given by
    \begin{align*}
        \text{Percent Error } &= \frac{\bigg\lvert A-b \bigg\rvert}{A} \cdot 100\% \\
       &=\frac{\bigg\lvert 0.672-0.615 \bigg\rvert}{0.672} \cdot 100\% = 8.48\%
    .\end{align*}








    \bigbreak \noindent 
    \section{Discussion}
    \bigbreak \noindent 
    In our exploration of projectile motion, the experiment yielded an average range that deviated from the theoretical predictions, as seen by the percent error of 8.48%. This discrepancy, while small, stems from quite obvious causes
    \bigbreak \noindent 
    Firstly, the initial velocity calculation plays a large role in finding the projectile's range. The precision of time measurements between the photogates impacts the accuracy of the initial velocity ($v_{0,x}$) calculation. Variations in the timing, possibly due to differences in the ball's release, could introduce some non-negliable inconsistencies in the velocity calculations.
    \bigbreak \noindent 
    Secondly, ignoring air resistance stands out as a significant simplification. In reality, air resistance affects the projectile's motion, especially over longer distances, potentially slowing the projectile more than the model predicts and thus reducing the actual range compared to the theoretical range.
    \bigbreak \noindent 
    Additionally,  the height measurement from which the projectile was launched could introduce errors. Any inaccuracy in measuring this height would affect the comuptation of the expected range.

    \bigbreak \noindent 
    \section{Conclusion}
    \bigbreak \noindent 
    This lab was about studying how things move when they're thrown or shot through the air, using ideas we've learned in class. We measured how far a ball goes and how it moves, which helped us understand more about motion and gravity. Even though we found some differences between what we expected and what actually occured, doing these experiments was particularly useful. It showed us how the material we learned in portrays in real world scenarios.
    \bigbreak \noindent 
    In short, this lab was a great way to see theory in action, especially about how things fly. The differences we saw also taught us that measuring stuff in the real world can be tricky and doesn't always match up perfectly with what we think will happen.










    
\end{document}
