\documentclass{report}

\input{~/dev/latex/template/preamble.tex}
\input{~/dev/latex/template/macros.tex}

\title{\Huge{}}
\author{\huge{Nathan Warner}}
\date{\huge{}}
\fancyhf{}
\rhead{}
\fancyhead[R]{\itshape Warner} % Left header: Section name
\fancyhead[L]{\itshape\leftmark}  % Right header: Page number
\cfoot{\thepage}
\renewcommand{\headrulewidth}{0pt} % Optional: Removes the header line
%\pagestyle{fancy}
%\fancyhf{}
%\lhead{Warner \thepage}
%\rhead{}
% \lhead{\leftmark}
%\cfoot{\thepage}
%\setborder
% \usepackage[default]{sourcecodepro}
% \usepackage[T1]{fontenc}

% Change the title
\hypersetup{
    pdftitle={}
}

\begin{document}
    % \maketitle
    %     \begin{titlepage}
    %    \begin{center}
    %        \vspace*{1cm}
    %
    %        \textbf{Lab Report}
    %
    %        \vspace{0.5cm}
    %         Skyscaper
    %             
    %        \vspace{1.5cm}
    %
    %        \textbf{Nathan Warner}
    %
    %        \vfill
    %             
    %             
    %        \vspace{0.8cm}
    %      
    %        \includegraphics[width=0.4\textwidth]{~/niu/seal.png} \\
    %        
    %             
    %    \end{center}
    % \end{titlepage}
    \begin{center}
        \begin{Huge}
            Atwood Machine Lab
        \end{Huge}
        \begin{Large}
            \bigbreak \noindent 
            Nate Warner
            \smallbreak \noindent
            March 9, 2024
            \bigbreak \noindent 
            Section 253D, 6:00 PM Tuesday 
        \end{Large}
    \end{center}
    \pagebreak 
    \tableofcontents
    \pagebreak \bigbreak \noindent 
    \begin{center}
    \textbf{Abstract}
    \end{center}
    \begin{adjustwidth}{.3in}{.3in}
        \hspace{\parindent}  This lab explores the dynamics of a two-mass pulley system, focusing on the relationship between mass differences and acceleration. The experiment is designed to show how differences between two masses, $m_{1}$ and $m_{2}$ suspended over a pulley, affect the system's acceleration. By using kinematic equations and Newton's second law, we derive an equation to calculate the acceleration based on the mass difference and gravity. The experimental procedure involves measuring the time it takes for the masses to move a known distance, adjusting the mass difference, and repeating the measurements to gather data for analysis. This data is then used to plot acceleration against mass difference, allowing for the determination of the gravitational acceleration, $g$, by analyzing the slope of the resulting line. 
    \end{adjustwidth}

    \bigbreak \noindent 
    \section{Theory}
    \bigbreak \noindent 
    \begin{figure}[ht]
        \centering
        \incfig{pulley}
        \label{fig:pulley}
    \end{figure}
    \fc{Shows the pulley system known as the "Atwood Machine"}
    \bigbreak \noindent 
    In this laboratory session, we examine a straightforward setup involving two masses connected by a string and suspended over a pulley. The force diagrams for each mass reveal that gravity exerts a downward force, countered by an upward tension force, $T$. With equal masses, the system remains static with no acceleration. However, differing masses introduce acceleration due to the unbalanced forces. Utilizing vertical forces, we apply kinematic equations to determine the acceleration of the masses as follows:
    \begin{equation}
    y(t) = y_0 + v_0t + \frac{1}{2}at^2
    \end{equation}
    Assuming initial motion from rest, and defining the distance between two photogates as $h$, the equation simplifies to:
    \begin{equation}
    h = \frac{1}{2}at^2
    \end{equation}
    This allows us to calculate acceleration directly:
    \begin{equation}
    a = \frac{2h}{t^2}
    \end{equation}
    Beyond this, we consider the system's total force. Each mass experiences gravitational force ($F = mg$), leading to a net force that is the difference in gravitational forces on the two masses:
    \begin{equation}
    F_{\text{tot}} = F_2 - F_1 = m_2g - m_1g = (m_2 - m_1)g
    \end{equation}
    This net force equals the total mass times acceleration ($F_{\text{tot}} = Ma$), with $M$ being the sum of the two masses:
    \begin{equation}
    F_{\text{tot}} = (m_1 + m_2)a
    \end{equation}
    By equating the expressions for $F_{\text{tot}}$, we derive an alternative formula for acceleration:
    \begin{equation}
    a = \frac{(m_2 - m_1)g}{m_2 + m_1}
    \end{equation}
    To account for non-ideal conditions like pulley friction and inertia, we adjust the total mass to include half the pulley's mass ($\frac{m_p}{2}$), resulting in the refined acceleration equation:
    \begin{equation}
    a = \frac{(m_2 - m_1)g}{m_1 + m_2 + \frac{m_p}{2}}
    \end{equation}

    \bigbreak \noindent 
    \section{Data}
    \bigbreak \noindent 
    The data accumulated for this lab consists of eight trials. For each trial, the total mass remains the same, however five grams are taken from $m_{1}$ and added to $m_{2}$. The mass of the pulley remains the same throughout each trial, it was measured to be $0.482g$. The height of the gap between the photogates also remains constant throughout the lab, it was measured to be $0.982m$. After each trail, the time it took to pass through the photogates was captured. The data for each trial is expressed in the following table.
    \bigbreak \noindent 
    \begin{center}
        \begin{tabular}{c|c|c|c}
            $m_{1}$ (kg) & $m_{2}$ (kg) & Mass Difference (kg) & time (s) \\
        \hline
            0.150 & 0.175& 0.025 & 0.623 \\
            0.145 & 0.180& 0.035 & 0.610 \\
            0.140 & 0.185& 0.045 & 0.600 \\
            0.135 & 0.190& 0.055& 0.590 \\
            0.130 & 0.195& 0.065& 0.590 \\
            0.125 &  0.200& 0.075& 0.580 \\
            0.120 &  0.205& 0.085& 0.570 \\
            0.115 &  0.210& 0.095& 0.560 \\
        \end{tabular}
    \end{center}
    \tc{Shows the data gathered from eight trials of the experiment, for a total of eight differences in mass. Note that the total mass remains constant}

    \pagebreak 
    \subsection{Colab Code: Graph Acceleration vs Mass Difference}
    \bigbreak \noindent 
    \fig{.4}{./figures/1.png}
    \fc{Shows the code used to generate the scatter plot}

    \bigbreak \noindent 
    \section{Results}
    \bigbreak \noindent 
    \subsection{Acceleration}
    \bigbreak \noindent 
    With the data found in table one, we use equation (3) to find the acceleration for each trial. If trial one has $\Delta t = 0.623$, the acceleration is given by
    \begin{align*}
        a &= \frac{2h}{t^{2}} \\
        &=\frac{2(0.982m)}{(0.623s)^{2}} \\
        &=5.06m/s^{2}
    .\end{align*}
    \bigbreak \noindent 
    The findings for the all eight trials are expressed in the following table.
    \begin{center}
        \begin{tabular}{c|c}
            Time (s) & Acceleration \\
        \hline
            0.623  & 5.06\\
            0.610  & 5.28\\
            0.600  & 5.45\\
            0.590  & 5.64\\
            0.590  & 5.64\\
            0.580  & 5.84\\
            0.570  & 6.04\\
            0.560  & 6.26\\
        \end{tabular}
    \end{center}
    \tc{Shows the acceleration found for each trial using equation (3)}

    \bigbreak \noindent 
    \subsection{Scatter plot: acceleration vs mass difference}
    \fig{.5}{./figures/2.png}
    \fc{Shows the graph of acceleration vs mass difference}
    \bigbreak \noindent 
    Analysis by linear regression of the scatter plot yields the equation
    \begin{align*}
        y &= 15.92x + 4.70 \quad R^{2} = 0.98 \\
        \implies m &= 15.92
    .\end{align*}
    \bigbreak \noindent 
    \subsection{Calculating g}
    \bigbreak \noindent 
    Using the equation
    \begin{align*}
        \text{Slope} = \frac{g}{m_{1} + m_{2} + \frac{m_{p}}{2}}
    .\end{align*}
    We find the value of $g$ for trial one to be
    \begin{align*}
        g &= \text{slope}\left(m_{1} + m_{2} + \frac{m_{p}}{2}\right) \\
        &=15.92m/s^{2}\left(0.150kg + 0.175kg+\frac{0.482kg}{2}\right) \\
        &=9.01m/s^{2}
    .\end{align*}
    \bigbreak \noindent 
    Since the slope and total mass of the system remains constant througout each trial, value for $g$ above is also constant among each trial.

    \bigbreak \noindent 
    \subsection{Percent Error}
    \bigbreak \noindent 
    We now calculate the percent error. This will determine the error in the experimental value if we define $g=9.8 m/s^{2}$ to be the accepted value.
    \begin{align*}
    \text{\%Error} &= \frac{\abs{A-B}}{A} \cdot 100\% \\
                      &=\frac{\abs{9.01 - 9.81}}{9.81} \cdot 100 \%  \\
                      &=8.15\%
    .\end{align*}

    \bigbreak \noindent 
    \section{Discussion}
    \bigbreak \noindent 
    In the Atwood Machine Lab, we show how differences in mass can affect the acceleration of a two-mass pulley system. The experiment's, which involved varying the mass difference between $m_1$ and $m_2$, was crucial for finding the relationship between mass and acceleration. Altering only one mass while keeping the other constant could have drastically changed the nature of the results, if each trial had a different total mass, the computation of $g$ would be different among each trial, given  each trial its own unique percent error .
    \bigbreak \noindent 
    Introducing initial velocity by releasing the cylinder from above the photogates would have complicated our measurements by deviating from the initial rest assumption. This additional variable would result in quicker times through the photogates, skewing the calculated acceleration. Adjusting for this initial velocity would require modifying the kinematic equations.
    \bigbreak \noindent 
    Furthermore, the assumption of a frictionless pulley, while simplifying the theoretical analysis, does not compare to real-world conditions where friction and inertia play significant roles. In reality, these factors would decrease the system's acceleration, suggesting that our simplified model might overestimate the actual acceleration. 

    \bigbreak \noindent 
    \section{Conclusion}
    \bigbreak \noindent 
    This lab focused on the dynamics of a two-mass pulley system, applying theoretical concepts to a real-world experiment. By changing the mass differences throughout each trial and observing the accelerations, we gained insights into the principles of motion and force. 
    \bigbreak \noindent 
    In conclusion, this Lab served as an excellent demonstration of physics in real-world situations, reinforcing our understanding of how forces influence motion. The variances observed between expected and actual results demonstrated the challenges of empirical measurement, proving the importance of accounting for external factors in experimental physics.










    
\end{document}
