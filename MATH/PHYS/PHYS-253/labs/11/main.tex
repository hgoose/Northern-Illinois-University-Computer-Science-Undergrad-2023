\documentclass{report}

\input{~/dev/latex/template/preamble.tex}
\input{~/dev/latex/template/macros.tex}

\title{\Huge{}}
\author{\huge{Nathan Warner}}
\date{\huge{}}
\fancyhf{}
\rhead{}
\fancyhead[R]{\itshape Warner} % Left header: Section name
\fancyhead[L]{\itshape\leftmark}  % Right header: Page number
\cfoot{\thepage}
\renewcommand{\headrulewidth}{0pt} % Optional: Removes the header line
%\pagestyle{fancy}
%\fancyhf{}
%\lhead{Warner \thepage}
%\rhead{}
% \lhead{\leftmark}
%\cfoot{\thepage}
%\setborder
% \usepackage[default]{sourcecodepro}
% \usepackage[T1]{fontenc}

% Change the title
\hypersetup{
    pdftitle={}
}

\begin{document}
    % \maketitle
    %     \begin{titlepage}
    %    \begin{center}
    %        \vspace*{1cm}
    %
    %        \textbf{Lab Report}
    %
    %        \vspace{0.5cm}
    %         Skyscaper
    %             
    %        \vspace{1.5cm}
    %
    %        \textbf{Nathan Warner}
    %
    %        \vfill
    %             
    %             
    %        \vspace{0.8cm}
    %      
    %        \includegraphics[width=0.4\textwidth]{~/niu/seal.png} \\
    %        
    %             
    %    \end{center}
    % \end{titlepage}
    \begin{center}
        \begin{Huge}
           Pendulum Periods 
        \end{Huge}
        \begin{Large}
            \bigbreak \noindent 
            Nate Warner
            \smallbreak \noindent
            April 23, 2024
            \bigbreak \noindent 
            Section 253D, 6:00 PM Tuesday 
        \end{Large}
    \end{center}
    \pagebreak 
    \tableofcontents
    \pagebreak 
    \section{Theory}
    \bigbreak \noindent 
    \begin{figure}[ht]
        \centering
        \incfig{fc1}
        \label{fig:fc1}
    \end{figure}
    \fc{Shows a simple pendulum}
    \bigbreak \noindent 
    In Figure 1, we observe the fundamental behavior of a pendulum. By securing one end of a string to a mass, $m$, and the other end to a ceiling point, it becomes clear through physical intuition that the pendulum will exhibit a consistent back-and-forth motion. Indeed, it does. When displaced at an angle, $\theta$, from the vertical (represented by the dashed line in Figure 1), the mass follows an arc. The arc length, denoted as $x$, is defined by
    $$
    x = L \theta
    $$
    \bigbreak \noindent 
    where $x$ is the arc length, $\theta$ is the angle from the vertical in radians, and $L$ is the string's length. Conveniently positioning our axes, we let the $\mathrm{y}$-axis align with the string. This orientation simplifies breaking down the gravitational force into $\mathrm{x}$ and y-components, as depicted in Figure 1. The x-component acts leftward, serving as a "restoring force" that drives the system back to its initial position. This force is quantified as
    $$
    F = -m g \sin(\theta)
    $$
    \bigbreak \noindent 
    The negative sign in Equation 2 indicates the nature of the restoring force. Given the dependency of Equations 1 and 2 on $\theta$, we solve for $\theta$ in Equation 1:
    $$
    \theta = \frac{x}{L}
    $$
    \bigbreak \noindent 
    Substituting this back into Equation 2 yields:
    $$
    F = -m g \sin\left(\frac{x}{L}\right)
    $$
    \bigbreak \noindent 
    Here, we apply the "small angle approximation," valid for angles up to about $15^{\circ}$, where $\sin(\theta) \approx \theta$. This approximation simplifies our understanding of pendular motion and is demonstrated at the experiment's onset. Using this approximation, the force equation becomes:
    $$
    F = -\frac{m g}{L} x
    $$
    \bigbreak \noindent 
    Remarkably, this mirrors Hooke's Law for harmonic motion, where $F = -k x$, and defining $k$ as $\frac{m g}{L}$ confirms the analogy. To determine the pendulum's period, or the time for a complete cycle, we utilize the general formula for simple harmonic motion:
    $$
    T = 2\pi \sqrt{\frac{m}{k}}
    $$
    \bigbreak \noindent 
    With $k = \frac{m g}{L}$ from Equation 3, the formula simplifies, canceling out the mass, leaving the period dependent solely on $L$:
    $$
    \begin{aligned}
        T &= 2\pi \sqrt{\frac{m}{m g / L}} \\
          &= 2\pi \sqrt{\frac{L}{g}}
    \end{aligned}
    $$
    \bigbreak \noindent 
    This reveals that neither mass nor displacement $\theta$ affects the period. Only $L$ influences $T$, specifically through its square root, indicating a linear relationship between $T^2$ and $L$:
    $$
    T^2 = \frac{4 \pi^2}{g} L
    $$
    \bigbreak \noindent 
    \subsection{Experimental Procedure}
    \begin{enumerate}
        \item The pendulum is currently at a length of about 1 meter long with a 100 gram mass attached to it.
            The LoggerPro setup should already have a program set up to measure the period as this pendulum
            swings. Run a test trial by displacing the pendulum and letting it swing. When you press collect, it
            should start displaying the period during each of the swings.

        \item Once you collect five data points on the graph, take the average of the five periods and record that
            average into the proper column in the table provided.

        \item Repeat step 2 for 150 gram, 200 gram, and 250 gram masses.

        \item Once complete with the masses, remove any excess mass from the bob, and start measuring the angle
            by which you displace the pendulum for each trial. You should have a measurement of \(5^\circ\), \(10^\circ\), \(15^\circ\),
            and \(20^\circ\). Again, record the average period for each of these displacements into the proper column in
            the table provided.

        \item Adjust the length of the pendulum by wrapping it around a pole. Measure from the bottom of the pole
            to the top of the mass: this will be your new length \(L\). Find the average over 5 periods and record the
            average for this new length.

        \item Repeat Step 5 so that you get the average period of five different lengths. Record all of your data in
            the tables below.
    \end{enumerate}

    \pagebreak 
    \section{Data}
    \bigbreak \noindent 
    The data collected during the lab is split among three tables
    \bigbreak \noindent 
    \begin{table}[h]
        \centering
        \begin{tabular}{cc}
            \toprule
            Mass (g) & Avg Period (s) \\
            \midrule
            100 & 1.882 \\
            150 & 1.879 \\
            200 & 1.879 \\
            250 & 1.879 \\
            \bottomrule
        \end{tabular}
        \caption{Average Periods for Different Masses}
    \end{table}
    \bigbreak \noindent 
    \begin{table}[h]
        \centering
        \begin{tabular}{cc}
            \toprule
            Amplitude (degrees) & Avg Period (s) \\
            \midrule
            5 & 1.125 \\
            10 & 1.879 \\
            15 & 1.885 \\
            20 & 1.889 \\
            \bottomrule
        \end{tabular}
        \caption{Average Periods for Different Amplitudes}
    \end{table}
    \bigbreak \noindent 
    \begin{table}[h]
        \centering
        \begin{tabular}{ccc}
            \toprule
            Length (m) & Avg Period (s) & Period Squared (sec\(^2\)) \\
            \midrule
            0.81 & 1.879 & 3.53 \\
            0.73 & 1.8 & 3.24 \\
            0.62 & 1.669 & 2.786 \\
            0.5 & 1.516 & 2.298 \\
            0.44 & 1.453 & 2.11 \\
            \bottomrule
        \end{tabular}
        \caption{Average Periods and Period Squared for Different Lengths}
    \end{table}

    \bigbreak \noindent 
    \section{Results}
    \bigbreak \noindent 
    We start the data analysis by contructing a scatter plot for the three tables listed above.
    \bigbreak \noindent 
    \subsection{Scatter plot: Mass vs Average Period}
    \bigbreak \noindent 
    \fig{.5}{./figures/1.png}

    \bigbreak \noindent 
    \subsection{Scatter plot: Period vs Amplitude}
    \bigbreak \noindent 
    \fig{.5}{./figures/2.png}

    \bigbreak \noindent 
    \subsection{Scatter plot: Period vs Length}
    \bigbreak \noindent 
    \fig{.5}{./figures/3.png}

    \bigbreak \noindent 
    \subsection{Scatter plot: Period$^{2}$ vs Length}
    \bigbreak \noindent 
    \fig{.7}{./figures/Screenshot_20240423_133458.png}

    \bigbreak \noindent 
    \textbf{Equation of fitted line:}
    \begin{align*}
        y = 3.9109x + 0.368, \quad \text{with } r^{2} = 0.999
    .\end{align*}

    \bigbreak \noindent 
    \subsection{Analysis of the period$^{2}$ vs length plot}
    \bigbreak \noindent 
    Using equation six, and the slope found from the fitted line equation, we find gravity ($g$) to be
    \begin{align*}
        m &= \frac{4\pi^{2}}{g} \implies g = \frac{4\pi^{2}}{m}\\
          &= \frac{4\pi^{2}}{3.9109} \\
          &=10.0945
    .\end{align*}
    Thus, in this lab we have calculated the quantity of gravity to be $10.0945 \frac{m}{s^{2}}$. Using the accepted value $9.8 \frac{m}{s^{2}}$, we compute the percent error with
    \begin{align*}
        \text{\%Err} &= \frac{\abs{A-B}}{A} \cdot 100\% \\
                     &= \frac{\abs{9.8 - 10.0945}}{9.8} \cdot 100\% \\
                     &=3.005\%
    .\end{align*}



    \bigbreak \noindent 
    \section{Discussion}
    \bigbreak \noindent 
    \bigbreak \noindent
    The experiments conducted on pendulum periods provided insights into the fundamental properties of harmonic motion. Notably, our results confirm that the mass of the pendulum does not significantly affect the period of the pendulum, consistent with the theoretical predictions for simple harmonic oscillators. As shown in the data and reinforced by the scatter plots, the period values remained relatively constant despite varying masses (1.882 s for 100 g and 1.879 s for 250 g). This observation supports the hypothesis that, for small angles, the period of a simple pendulum is independent of its mass, primarily influenced by the length of the string and the acceleration due to gravity.
    \bigbreak \noindent
    Displacement from the vertical (angle of amplitude) also showed minimal influence on the period until reaching larger angles, which is expected under the small angle approximation. The theory suggests that $\sin(\theta) \approx \theta$ should hold true for angles up to about $15^\circ$, which our data corroborated, as periods did not vary drastically up to $20^\circ$. Regarding the value of gravity derived from the slope of the Period Squared versus Length plot ($10.0945 \, \text{m/s}^2$), it demonstrated a minor percent error (3.005\%) relative to the accepted value of $9.8 \, \text{m/s}^2$. This discrepancy likely arises from errors in the experimental setup, such as the sensitivity of the photogate sensor or minor alignment issues in the pendulum's release mechanism, which could affect the initial angle measurement and the resultant period calculation.
    \bigbreak \noindent

    \bigbreak \noindent 
    \section{Conclusion}
    \bigbreak \noindent 
    In this laboratory experiment, we investigated the periodic behavior of a simple pendulum and explored how various factors such as mass, amplitude of swing, and length of the pendulum influence its period. Our findings show the classical theory of simple harmonic motion, demonstrating that the pendulum's period is independent of its mass and mostly not affected by the amplitude of swing, provided the swings are within small angles. 










    
\end{document}
