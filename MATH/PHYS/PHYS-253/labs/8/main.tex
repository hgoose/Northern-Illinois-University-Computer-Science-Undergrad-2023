\documentclass{report}

\input{~/dev/latex/template/preamble.tex}
\input{~/dev/latex/template/macros.tex}

\title{\Huge{}}
\author{\huge{Nathan Warner}}
\date{\huge{}}
\fancyhf{}
\rhead{}
\fancyhead[R]{\itshape Warner} % Left header: Section name
\fancyhead[L]{\itshape\leftmark}  % Right header: Page number
\cfoot{\thepage}
\renewcommand{\headrulewidth}{0pt} % Optional: Removes the header line
%\pagestyle{fancy}
%\fancyhf{}
%\lhead{Warner \thepage}
%\rhead{}
% \lhead{\leftmark}
%\cfoot{\thepage}
%\setborder
% \usepackage[default]{sourcecodepro}
% \usepackage[T1]{fontenc}

% Change the title
\hypersetup{
    pdftitle={}
}

\begin{document}
    % \maketitle
    %     \begin{titlepage}
    %    \begin{center}
    %        \vspace*{1cm}
    %
    %        \textbf{Lab Report}
    %
    %        \vspace{0.5cm}
    %         Skyscaper
    %             
    %        \vspace{1.5cm}
    %
    %        \textbf{Nathan Warner}
    %
    %        \vfill
    %             
    %             
    %        \vspace{0.8cm}
    %      
    %        \includegraphics[width=0.4\textwidth]{~/niu/seal.png} \\
    %        
    %             
    %    \end{center}
    % \end{titlepage}
    \begin{center}
        \begin{Huge}
            Axel Lab
        \end{Huge}
        \begin{Large}
            \bigbreak \noindent 
            Nate Warner
            \smallbreak \noindent
            March 27, 2024
            \bigbreak \noindent 
            Section 253D, 6:00 PM Tuesday 
        \end{Large}
    \end{center}
    \pagebreak 
    \tableofcontents
    \pagebreak \bigbreak \noindent 
    \begin{center}
    \textbf{Abstract}
    \end{center}
    \begin{adjustwidth}{.3in}{.3in}
        \hspace{\parindent} This experiment investigates the relationship between the radius of circular motion and centripetal force, with gravity as the medium. By using a mass-spring system, where a mass attached to a spring is balanced by a weight hanging over the edge of a table, the experiment demonstrates the principles of equilibrium and restoring force. The experiment induces circular motion in the mass-spring system, which streches the spring further and establishing a new equilibrium state. This experiment is used to explore circular motion, how the centripetal force correlates with the radius of the motion under constant mass conditions. The centripetal force is calculated using the mass of the rotating object, the angular velocity, and the radius of the motion, with angular velocity derived from the time taken for a set number of rotations. The experiment involves varying the radius of motion and measuring the change in the centripetal force. The data collected from multiple trials at different radii are analyzed to compare the gravitational force (acting on the hanging mass) and the centripetal force (acting on the rotating mass).
    \end{adjustwidth}

    \bigbreak \noindent 
    \section{Theory}
    \bigbreak \noindent 
    In this lab, we look into the effects of circular motion's radius on the centripetal force, with gravity as the primary force. Imagine a scenario where a mass is initially in equilibrium due to the tension of a spring. When this spring is stretched, it exerts a force to bring the mass back to its equilibrium state. Our experiment involves elongating a spring to create a restoring force. This is done by attaching one end to a mass while the other end hangs over the edge of a table. By placing a weight on the hanging end, the spring is stretched, and it reaches a new equilibrium in its extended position, even though the system is stationary. The force that the spring uses to pull the mass back is counteracted by the gravitational pull acting downwards. We write
    \begin{equation}
        F_{s} = Mg
    \end{equation}
    \bigbreak \noindent 
    Where \(M\) represents the mass of the hanging weight, we start by elongating the spring through the addition of the hanging mass. We induce further extension of the spring by spinning the apparatus, ensuring that the mass reaches a required radius, $r$.
    \bigbreak \noindent 
    As the object undergoes circular motion, we define its angular velocity, \(\omega\), which is the arc traveled by the object in one second. This is essential for understanding the motion, and allows us to make a connection between the angular velocity and linear velocity. We define the relationship between the angular and linear velocity as 
    \begin{equation}
        v = \omega r
    \end{equation}
    \bigbreak \noindent 
    We know by Newton's first law, an object will maintain its state of motion unless acted upon by an external force. In the context of circular motion, this suggests that a force must be applied to keep an object moving in a circle. This force which constantly pulls the object toward the center of its circular path, is known as the centripetal force, denoted by \( F_c \). Newton's second law gives us the formula for calculating this force. The law states that the force acting on an object is equal to the mass of the object multiplied by its acceleration (\( F = ma \)). Therefore, for circular motion, the centripetal force (\( F_c \)) is given by
    \begin{equation}
        F_c = m a_c
    \end{equation}
    \bigbreak \noindent 
    If there is no variability in the mass of the object, then we can always find the centripetal accerelation. As it turns out, for uniform circular motion, we use
    \begin{equation}
       a_{c} = \frac{v^{2}}{r} 
    \end{equation}
    \bigbreak \noindent 
    Where $v$ is the linear velocity. We assertain $\omega$ with the simple fact that it takes a certain time, $t$, to go around a circle $n$ times. Thus, we define $T = \frac{t}{n}$, which as you see gives us the time it takes to complete one rotation of $2\pi \, \text{rad}$. This gives omega as
    \begin{equation}
        \omega = \frac{2\pi}{\frac{t}{n}} = \frac{2\pi n}{t}
    \end{equation}
    \bigbreak \noindent 
    Plugging equation two into equation four and using this result for omega, we find the centripetal force given by
    \begin{align*}
        F_{c} &= ma_{c} \\
        &=m \frac{v^{2}}{r} \\
        &=m \frac{\omega^{2}r^{2}}{r} \\
        \therefore F_{c} &=m \frac{4\pi^{2}n^{2}}{t^{2}}r
    .\end{align*}

    \bigbreak \noindent 
    An example of centripetal force is described in the following figure
    \bigbreak \noindent 
    \begin{figure}[ht]
        \centering
        \incfig{fig2}
        \label{fig:fig2}
    \end{figure}
    \bigbreak \noindent 
    \fc{Shows centripetal acceleration and velocity for a object executing uniform circular motion. Notice that the acceleration points towards the center of the circle and the velocity is tangent to the curve}

    \bigbreak \noindent 
    \subsection{Procedure}
    \begin{enumerate}
        \item Locate the two marks on the device that indicate the edges of the axle's base. Calculate the distance $D$ between these marks. The axle's radius $R$ is then half of this distance, calculated as $R = \frac{D}{2}$.
        \item Unhook the pendulum bob from the spring to measure its weight, labeling this as mass $m$, as indicated in Equation 6. Afterwards, reattach the bob to the spring and string.
        \item Move the indicator as close to the axle as possible. Record the distance $d$ from the base of the axle to the indicator. This distance will be different for each experiment. To find the total radius of the motion, add the radius $R$ from step 1 to $d$, giving $r = \frac{D}{2} + d$.
            \textbf{Note:} For precise rotation of the bob, it's crucial to align the bob and its string perfectly straight. Adjust the counterweight at the top of the axle until it lines up directly with the indicator's position.
        \item Attach the string to the front of the bob and then to the hanger. Drape it over the pulley to start stretching the spring. Gradually add weight to the hanger until the bob aligns directly above the indicator. Document the total weight hung, then remove it from the string and secure the string to prevent interference during the axle's rotation.
        \item With the hanging mass removed and the string secured, manually rotate the axle until the bob is positioned directly above the indicator again. Begin timing once it's aligned, using LoggerPro to record the total time for exactly 50 rotations ($n = 50$).
            \textbf{Note:} Continuous spinning of the axle is necessary due to internal resistance; if left to spin freely, it will slow and eventually stop, leading to significant data inaccuracies.
        \item Shift the indicator 2 cm closer to the pulley and measure the new distance $d$ to determine the updated radius $r$. Repeat steps 4 and 5 for this new setup.
        \item Execute step 6 a total of five times, each with the indicator moved and a new measurement of $d$ taken, to complete a series of five trials.
    \end{enumerate}

    \pagebreak 
    \section{Data}
    \bigbreak \noindent 
    This experiment has 5 distinct trials. In each trial, the indicator was moved by some small distance. This change in distance requires a remeasure for $d$, which in turn changes the measure for $r$, recall that $d$ is the distance between the indicator and the base
    \bigbreak \noindent 
    In each trial, we spin the system such that it passes through the Photogate a total of 50 times, after the spins are completed, we take an arbitrary $\Delta t$. That is, the time difference between two passes through the Photogate, this will be useful in future calculations.
    \bigbreak \noindent 
    \begin{center}
        \begin{tabular}{c|c|c|c|c|c|c|c}
            Trial & Bob mass (g) & Hanging Mass (g) & $D$ (cm) & $d$ (cm) & $r$ (cm) & $n$ & $\Delta t$ (s) \\
            \midrule
            1 & 447 & 500 & 6.25 & 10.3 & 13.425 & 50 & 0.713 \\
            2 & 447 & 950 & 6.25 & 16 & 19.125 & 50 & 0.565 \\
            3 & 447 & 825 & 6.25 & 14 & 17.125 & 50 & 0.605 \\
            4 & 447 & 600 & 6.25 & 12.25 & 15.375 & 50 & 0.648 \\
            5 & 447 & 525 & 6.25 & 11 & 14.125 & 50 & 0.673 
        \end{tabular}
    \end{center}
    \tc{Expresses the data collected during the experiment}
    \bigbreak \noindent 
    The value $R$ is defined to be the radius of the axel. It is given by
    \begin{align*}
        R = \frac{D}{2}
    .\end{align*}
    The following figure describes the variables $r$, $d$, and $D$
    \bigbreak \noindent 
    \begin{figure}[ht]
        \centering
        \incfig{fig3}
        \label{fig:fig3}
    \end{figure}
    \fc{Describes the variables $r$, $d$, and $D$}

    \pagebreak 
    \section{Results}
    \bigbreak \noindent 
    \subsection{$F_{s}$ and $F_{c}$}
    \bigbreak \noindent 
    Using the results from table 1, and equation 1, we find the force on the spring. To show this computation, we use the results from trial 1. Recall that we defined $F_{s} = Mg$ is the theory section of this report.
    \begin{align*}
        F_{s} &= Mg = .5kg(9.8m/s^{2}) \\
        &=4.9N
    .\end{align*}
    \bigbreak \noindent 
    Where $M$ is the hanging mass. To find the centripetal force $F_{c}$, we use equation 6, with $m$ as the mass of the bob. Regarding trial 1, we find
    \begin{align*}
        F_{c} = m \frac{4\pi^{2}n^{2}}{t^{2}}r
    .\end{align*}
    Since we recoreded the time it takes to complete only one rotation (as seen by $\Delta t$ in table 1), we use the fact that $T = \frac{t}{n} $ and rewrite
    \begin{align*}
        F_{c} = m \frac{4\pi^{2}}{\Delta t^{2}}r
    .\end{align*}
    Using data from trial 1 we find
    \begin{align*}
        F_{c} &= \frac{4(.447kg)\pi^{2}}{(0.713s)^{2}}(.13425m) \\
        &=4.66N
    .\end{align*}
    \bigbreak \noindent 
    The results for the remaining four trials are expressed in the following table
    \bigbreak \noindent 
    \begin{center}
        \begin{tabular}{c|c|c}
            Trial & $F_{s} (N)$ &$F_{c} (N)$ \\
            \hline
            1 & 4.9 & 4.66\\
            2 & 9.31 & 10.5724 \\
            3 &8.085 & 8.2563 \\
            4 &5.88 & 6.4615 \\
            5 &5.145 & 5.5033
        \end{tabular}
    \end{center}
    \tc{Displays the forces found for each trial of the lab}


    \bigbreak \noindent 
    \subsection{Percent error}
    \bigbreak \noindent 
    We know examine the percent error between the spring force and the centripetal force. We use the following equation
    \begin{align*}
        \text{\%Error} = \frac{\abs{F_{c} - F_{s}}}{\frac{1}{2}(F_{c} + F_{s})} \cdot 100\%
    .\end{align*}
    \bigbreak \noindent 
    Using the first trial as an example, we find
    \begin{align*}
        \text{\%Error} &= \frac{\abs{4.66 - 4.9}}{\frac{1}{2}(4.66+4.9)} \cdot 100\% \\
        &\approx 5\%
    .\end{align*}
    \bigbreak \noindent 
    The results for the remaining trial can be found in the following table
    \bigbreak \noindent 
    \begin{center}
        \begin{tabular}{c|c}
            Trial & Percent Error \\
            \hline
            1 & 5\%\\
            2 & 13\%\\
            3 & 2\%\\
            4 & 9\%\\
            5 & 7\%\\
        \end{tabular}
    \end{center}
    \bigbreak \noindent 
    \begin{remark}
        The percent differences between the spring force ($F_s$) and the centripetal force ($F_c$) observed in the experiment are noteworthy. 
        \bigbreak \noindent 
         The discrepancies between $F_s$ and $F_c$ can arise from measurement inaccuracies. Precise measurement of distances ($D$, $d$, and $r$) and time intervals ($\Delta t$) is challenging and subject to human error.  
        \bigbreak \noindent 
        Secondly, the assumption that the mass remains constant throughout the experiment might not account for the dynamic nature of the system, especially considering air resistance and friction at the axle. These unaccounted forces could alter the effective mass of the rotating system, thus affecting the centripetal force calculation.
        \bigbreak \noindent 
        To reduce these differences:
        \begin{itemize}
            \item Enhancing the precision of measurements. 
            \item Minimizing external influences such as air resistance and friction 
            \item Conducting multiple readings for each trial and calculating an average 
        \end{itemize}
    \end{remark}

    \bigbreak \noindent 
    \section{Discussion}
    \bigbreak \noindent 
    In this experiment, we dived into how changing the radius of a circle affects the force needed to keep something moving in that circle, basically looking at the tug-of-war between the pull of the spring and gravity. Our results pretty much lined up with what physics textbooks say: if you make the circle bigger, you need more force to keep things moving in a loop. This makes sense because you're stretching the spring more, so it pulls back harder.
    \bigbreak \noindent 
    But, just like in any science experiment, things weren't perfect. We had to assume our setup was ideal, ignoring stuff like air pushing against the moving parts and the friction where the axle spins. Plus, measuring everything exactly right is tough. Even though we tried our best, there's always a bit of guesswork involved, which shows up as small errors in our results.
    \bigbreak \noindent 
    To get closer to the textbook predictions next time, we could use better tools for measuring and maybe find a way to cut down on things like air resistance and friction. Also, doing the experiment in a more controlled setting might help keep those pesky outside forces in check.
    \bigbreak \noindent 

    \section{Conclusion}
    \bigbreak \noindent 
    In conclusion, the experiment conducted aimed to investigate the relationship between the radius of circular motion and the centripetal force required to sustain that motion. The results obtained from the experiment align with theoretical predictions, confirming that an increase in the radius necessitates a corresponding increase in centripetal force to maintain circular motion. This outcome corroborates the principles of physics regarding circular motion and centripetal force.
    \bigbreak \noindent 
    However, the experiment encountered limitations due to the assumption of an ideal system, excluding factors such as air resistance and friction, and potential inaccuracies in measurement. These limitations may have contributed to discrepancies between the theoretical and experimental results.










    
\end{document}
