\documentclass{report}

\input{~/dev/latex/template/preamble.tex}
\input{~/dev/latex/template/macros.tex}

\title{\Huge{}}
\author{\huge{Nathan Warner}}
\date{\huge{}}
\fancyhf{}
\rhead{}
\fancyhead[R]{\itshape Warner} % Left header: Section name
\fancyhead[L]{\itshape\leftmark}  % Right header: Page number
\cfoot{\thepage}
\renewcommand{\headrulewidth}{0pt} % Optional: Removes the header line
%\pagestyle{fancy}
%\fancyhf{}
%\lhead{Warner \thepage}
%\rhead{}
% \lhead{\leftmark}
%\cfoot{\thepage}
%\setborder
% \usepackage[default]{sourcecodepro}
% \usepackage[T1]{fontenc}

% Change the title
\hypersetup{
    pdftitle={}
}

\begin{document}
    % \maketitle
    %     \begin{titlepage}
    %    \begin{center}
    %        \vspace*{1cm}
    %
    %        \textbf{Lab Report}
    %
    %        \vspace{0.5cm}
    %         Skyscaper
    %             
    %        \vspace{1.5cm}
    %
    %        \textbf{Nathan Warner}
    %
    %        \vfill
    %             
    %             
    %        \vspace{0.8cm}
    %      
    %        \includegraphics[width=0.4\textwidth]{~/niu/seal.png} \\
    %        
    %             
    %    \end{center}
    % \end{titlepage}
    \begin{center}
        \begin{Huge}
            Incline Plane Lab
        \end{Huge}
        \begin{Large}
            \bigbreak \noindent 
            Nate Warner
            \smallbreak \noindent
            February 22, 2024
            \smallbreak \noindent
            PHYS 253
            \bigbreak \noindent 
            Section 253D, 6:00 PM Tuesday 
        \end{Large}
    \end{center}
    \pagebreak 
    \tableofcontents
    \pagebreak \bigbreak \noindent 
    \begin{center}
    \textbf{Abstract}
    \end{center}
    \begin{adjustwidth}{.3in}{.3in}
        \hspace{\parindent} This study aims to determine the acceleration due to gravity ($g$) by analyzing a cart's motion down an inclined track at three different angles. Using a motion detector and LoggerPro software, the data for the position and velocity were collected for nine trials (three per angle). Data analysis for this lab involves creating scatter plots and trendlines to compute experimental values for $g$. The results are then compared to the accepted $g$ value (9.81 $m/s^{2}$) to evaluate the experiment's accuracy through percent error calculations. 
    \end{adjustwidth}

    \bigbreak \noindent 
    \section{Theory}
    \bigbreak \noindent 
    From our everyday observations, we know that a ball will accelerate downhill if nothing stops it. Now, we aim to quantify this acceleration. To start, we must determine the slope of the incline by measuring its height at two points, $h_{1}$ and $h_{2}$, set apart by a distance, $\ell$. Referring to Figure \thefigtitle, these measurements help us identify two smaller triangles within a larger one, sharing the same angle, $\theta$, with the horizontal. Using basic trigonometry, we can calculate this angle from our measurement
    \begin{equation}
        \tan{\left(\theta\right)}  &= \frac{\text{opp}}{\text{hyp}} = \frac{h_{2}-h_{1}}{\ell} \\
       \implies \theta  = \tan^{-1}{\left(\frac{h_{2}-h_{1}}{\ell}\right)}
    \end{equation}
    \bigbreak \noindent 
    This measurement then enables us to correlate the acceleration we observe with g, the gravitational acceleration value.
    \bigbreak \noindent 
    \begin{figure}[ht]
        \centering
        \incfig{fig1}
        \label{fig:fig1}
    \end{figure}
    \fc{Shows the similarity between the smaller and larger triangle, as well as labeling the measurements required for finding the angle $\theta$}
    \bigbreak \noindent 
    \begin{remark}
        We have the flexibility to set up our coordinate systems in a manner that simplifies describing the ball's motion, positioning the x-direction parallel to the incline does just this (Figure \thefigtitle). We can then consider the kinematic equations for position and velocity.
        \begin{align*}
            v(t) &= v_{0} + at \\
            x(t) &= x_{0} + v_{0}t + \frac{1}{2}at^{2}
        .\end{align*}
    \end{remark}
    \pagebreak \bigbreak \noindent 
    \begin{figure}[ht]
        \centering
        \incfig{fig2}
        \label{fig:fig2}
    \end{figure}
    \fc{Shows how we define our coordinate access}
    \bigbreak \noindent 
    Given the ability to define our coordinate system in such a way to aid in our computations, we set the ball's starting position at $x=0$. With the ball beginning from a stationary state, its initial velocity, $v_{0}$, is also zero. Consequently, this simplification leads to the equation
    \begin{equation}
        v(t) = at
    \end{equation}
    \bigbreak \noindent 
    Next, we account for gravity acting directly downwards, at an angle to the plane. The component of gravity along the x-axis is responsible for the acceleration down the plane, which we define within our coordinate system as (figure \thefigtitle)
    \begin{equation}
       a=g\sin{\left(\theta\right)} 
    \end{equation}
    \bigbreak \noindent
    \begin{figure}[ht]
        \centering
        \incfig{figab}
        \label{fig:figab}
    \end{figure}
    \fc{Shows the relationship between $g$ and $g_{x}$, as defined by our coordinate axis}
    \bigbreak \noindent 
    We now substitude this value for a into equation 2.
    \begin{equation}
       v(t) = g\sin{\left(\theta \right)}t
    \end{equation}
    \bigbreak \noindent 
    When conducting measurements of position and velocity during the experiment, plotting the data should reveal a linear relationship between velocity and time, akin to the standard linear equation \(y = mx + b\), where \(m\) represents the slope of the line. By analogy, with \(y = mx + b\), \(t\) serves as our independent variable, making the slope (\(m\)) proportional to \(g\) through the relationship
    \begin{equation}
        m = g \sin(\theta) \Rightarrow g = \frac{m}{\sin(\theta)} 
    \end{equation}
    We can then evaluate our experimental \(g\) value against the accepted value of \(9.81 \, \text{m/s}^2\) by calculating the percent error as follows:
    \begin{equation}
        \%Error = \left| \frac{\text{Accepted} - \text{Measured}}{\text{Accepted}} \right| \times 100\%
    \end{equation}
    

    \bigbreak \noindent 
    \section{Data}
    \bigbreak \noindent 
    The measurements taken for each trial are displayed in the following table
    \bigbreak \noindent 
    \begin{center}
        \begin{tabular}{c|c|c|c|c}
            Trial no. & $H_{1}$ & $H_{2}$ & $\ell$ & $\theta$\\
            \hline
            1 & 13.5cm & 17cm &30cm & $7.6^{\circ}$\\
            2 & 12.5cm & 16cm & 31cm & $6.4^{\circ}$\\
            3  &9.5cm & 12.5cm & 31cm & $5.5^{\circ}$
        \end{tabular}
    \end{center}
    \tc{Shows the measurements and angle $\theta$ for each trial. Recall that each trial will have three iterations}

    \bigbreak \noindent 
    \subsection{Colab data: Functions for graph creations}
    \bigbreak \noindent 
    \fig{.4}{./figures/1.png}
    \fc{Shows the functions used to create the graphs}

    \bigbreak \noindent 
    \subsection{Colab data: Trial 1}
    \fig{.4}{./figures/2.png}
    \fc{Shows the data points in trial one iterations}


    \bigbreak \noindent 
    \subsection{Colab data: Trial 2}
    \fig{.44}{./figures/3.png}
    \fc{Shows the data points in trial two iterations}


    \bigbreak \noindent 
    \subsection{Colab data: Trial 3}
    \fig{.5}{./figures/4.png}
    \fc{Shows the data points in trial three iterations}


    \pagebreak 
    \section{Results}
    \smallbreak \noindent
    \subsection{Colab graphs and computations: Trial 1}
    \smallbreak \noindent
    \subsubsection{Iteration 1 Graphs}
    \fig{.18}{./figures/it1.png}
    \fc{}
    \subsubsection{Iteration 1 Equations}
    \bigbreak \noindent 
    The linear equation for the position vs time graph is given by
    \begin{align*}
        y = 0.88x -0.73
    .\end{align*}
    \bigbreak \noindent 
    The linear equation for the velocity vs time graph is given by
    \begin{align*}
        y=1.18x -0.63
    .\end{align*}

   \bigbreak \noindent 
    \subsubsection{Iteration 2 Graphs}
    \bigbreak \noindent 
    \fig{.3}{./figures/it2.png}
    \fc{}
    \bigbreak \noindent 
    \subsubsection{Iteration 2 Equations}
    The linear equation for the position vs time graph is given by
    \begin{align*}
        y = 0.53x+0.44
    .\end{align*}
    \bigbreak \noindent 
    The linear equation for the velocity vs time graph is given by
    \begin{align*}
        y=-2.01x+2.23
    .\end{align*}

   \bigbreak \noindent 
    \subsubsection{Iteration 3 Graphs}
    \fig{.3}{./figures/it3.png}
    \fc{}


    \bigbreak \noindent 
    \subsubsection{Iteration 3 Equations}
    \bigbreak \noindent 
    The linear equation for the position vs time graph is given by
    \begin{align*}
        y=.99x-0.83
    .\end{align*}
    \bigbreak \noindent 
    The linear equation for the velocity vs time graph is given by
    \begin{align*}
        y=1.61x-.99
    .\end{align*}


    \pagebreak 
    \subsection{Colab graphs and computations: Trial 2}
    \bigbreak \noindent 
    \subsubsection{Iteration 1 Graphs}
    \fig{.45}{./figures/iter4.png}
    \fc{}


    \bigbreak \noindent 
    \subsubsection{Iteration 1 Equations}
    \bigbreak \noindent 
    The linear equation for the position vs time graph is given by
    \begin{align*}
        y=0.71x-0.61
    .\end{align*}
    \bigbreak \noindent 
    The linear equation for the velocity vs time graph is given by
    \begin{align*}
        y=2.06x-1.94
    .\end{align*}

   \bigbreak \noindent 
    \subsubsection{Iteration 2 Graphs}
    \fig{.45}{./figures/iter5.png}
    \fc{}


    \bigbreak \noindent 
    \subsubsection{Iteration 2 Equations}
    \bigbreak \noindent 
        The linear equation for the position vs time graph is given by
    \begin{align*}
        y=0.89x-0.68
    .\end{align*}
    \bigbreak \noindent 
    The linear equation for the velocity vs time graph is given by
    \begin{align*}
        y=1.13x-0.47
    .\end{align*}


   \bigbreak \noindent 
    \subsubsection{Iteration 3 Graphs}
    \fig{.45}{./figures/iter6.png}
    \fc{}


   \bigbreak \noindent 
    \subsubsection{Iteration 3 Equations}
    The linear equation for the position vs time graph is given by
    \begin{align*}
        y=0.92x-0.68
    .\end{align*}
    With $R^{2} = 1$
    \bigbreak \noindent 
    The linear equation for the velocity vs time graph is given by
    \begin{align*}
        y=1.15x-0.45
    .\end{align*}
    With $R^{2} = 1$



    \bigbreak \noindent 
    \subsection{Colab graphs and computations: Trial 3}

    \bigbreak \noindent 
    \subsubsection{Iteration 1 Graphs}
    \fig{.45}{./figures/iter7.png}
    \fc{}


    \bigbreak \noindent 
    \subsubsection{Iteration 1 Equations}
    \bigbreak \noindent 
    The linear equation for the position vs time graph is given by
    \begin{align*}
        y=0.73x-0.55
    .\end{align*}
    With $R^{2} = 1$
    \bigbreak \noindent 
    The linear equation for the velocity vs time graph is given by
    \begin{align*}
        y=0.80x-0.29
    .\end{align*}
    With $R^{2} = 1$

    \bigbreak \noindent 
    \subsubsection{Iteration 2 Graphs}
    \fig{.45}{./figures/iter8.png}
    \fc{}


    \bigbreak \noindent 
    \subsubsection{Iteration 2 Equations}
    The linear equation for the position vs time graph is given by
    \begin{align*}
        y = 0.71x -0.45 
    .\end{align*}
    With $R^{2} = 1$
    \bigbreak \noindent 
    The linear equation for the velocity vs time graph is given by
    \begin{align*}
        y=0.83x-0.23
    .\end{align*}
    With $R^{2} = 0.99$

    \bigbreak \noindent 
    \subsubsection{Iteration 3 Graphs}
    \fig{.45}{./figures/iter9.png}
    \fc{}


    \bigbreak \noindent 
    \subsubsection{Iteration 3 Equations}
    The linear equation for the position vs time graph is given by
    \begin{align*}
        y = 0.67x -0.82 
    .\end{align*}
    With $R^{2} = 1$
    \bigbreak \noindent 
    The linear equation for the velocity vs time graph is given by
    \begin{align*}
        y=0.92x-0.90
    .\end{align*}
    With $R^{2} = 0.98$

    \bigbreak \noindent 
    \subsection{Experimental values for $g$}
    \bigbreak \noindent 
    The computation for the experimental values of $g$ shall be shown using trial one $\theta =7.6^{\circ}$, and iteration one $m=0.88$. Using equation 5, we see
    \begin{align*}
        g &= \frac{m}{\sin{\left(\theta\right)}} = \frac{1.18}{\sin{\left(7.6^{\circ}\right)}} \\
         &=8.92 m/s^{2}
    .\end{align*}
    Note that we are using $m$ from the velocity vs time graphs
    \bigbreak \noindent 
    The values for the remaining trials are found in the following table
    \pagebreak \bigbreak \noindent 
    \begin{center}
        \begin{tabular}{c|c|c|c}
            Trial \& Iteration & $\theta$ & $m$ & $g$\\
            \hline
            1, 1 & $7.6^{\circ}$ & 1.18 & 8.92\\
            1, 2 & $7.6^{\circ}$ & 2.01 & 15.2\\
            1, 3 & $7.6^{\circ}$ & 1.61 & 12.17 \\
            2, 1 & $6.4^{\circ}$ & 2.06 & 18.48 \\
            2, 2  & $6.4^{\circ}$ & 1.13 & 10.14\\
            2, 3 & $6.4^{\circ}$ & 1.15 & 10.23\\
            3, 1 & $5.5^{\circ}$ & 0.80 & 8.35\\
            3, 2 & $5.5^{\circ}$ & 0.83 & 8.7\\
            3, 3 & $5.5^{\circ}$ & 0.92  &9.6
        \end{tabular}
    \end{center}
    \tc{Shows the experimental values of $g$ among each trial and iteration}

    \bigbreak \noindent 
    \subsection{Percent Error}
    \bigbreak \noindent 
    We can then evaluate our experimental $g$ value against the accepted value of 9.81 $m/s^{2}$ by calculating the percent error. We demonstrate this with trial one iteration one. We compare our experimental value $g =8.92 m/s^{2}$ with the accepted value $9.8 m/s^{2} $
    \begin{align*}
        \text{Percent Error } &= \bigg\lvert \frac{9.8 m/s^{2} - 8.92 m/s^{2}}{9.8 m/s^{2}} \bigg\rvert \times 100\% \\
        &=9\%
    .\end{align*}
    The value for the remaining trials are found in the following table
    \bigbreak \noindent 
    \begin{center}
    \begin{tabular}{c|c|c|c|c}
        Trial \& Iteration & $\theta$ & $m$ & $g$ & \% Error \\
        \hline
        1, 1 & $7.6^{\circ}$ & 1.18 & 8.92 & 9\% \\
        1, 2 & $7.6^{\circ}$ & 2.01 & 15.2 & 55.10\% \\
        1, 3 & $7.6^{\circ}$ & 1.61 & 12.17 & 24.18\% \\
        2, 1 & $6.4^{\circ}$ & 2.06 & 18.48 & 88.57\% \\
        2, 2  & $6.4^{\circ}$ & 1.13 & 10.14 & 3.47\% \\
        2, 3 & $6.4^{\circ}$ & 1.15 & 10.23 & 4.39\% \\
        3, 1 & $5.5^{\circ}$ & 0.80 & 8.35 & 14.80\% \\
        3, 2 & $5.5^{\circ}$ & 0.83 & 8.7 & 11.22\% \\
        3, 3 & $5.5^{\circ}$ & 0.92  & 9.6 & 2.04\% \\
    \end{tabular}
    \end{center}
    \tc{Shows the percent error among experimental values of $g$}


    \bigbreak \noindent 
    \section{Discussion}
    \bigbreak \noindent 
    The investigation into measuring the acceleration due to gravity, \(g\), revealed a range of experimental values, some that are close to the accepted \(9.81 \, \text{m/s}^2\) and others far from it, with percent errors spanning from \(2.04\%\) to \(88.57\%\). This discrepancy shows the sensitivity of our measurements due to various factors, likely the precision in determining the angle, \(\theta\).
    \bigbreak \noindent 
    Furthermore, Friction and air resistance were not accounted for in our theoretical model, which played a role in the experimental outcomes. The assumption of a frictionless environment greatly simplifies the theoretical model but doesn't align with realistic conditions.

    \bigbreak \noindent 
    \section{Conclusion}
    \bigbreak \noindent 
    This lab explored the experimental determination of the acceleration due to gravity, $g$, providing a application of theoretical concepts discussed in our physics course. The process of measuring $g$ across various inclines not only deepened my understanding of gravitational acceleration but also showed the importance of precision in physics. Despite the challenges, the experiment was a valuable exercise in applying theoretical knowledge to real-world scenarios.
    \bigbreak \noindent 
    In conclusion, the lab effectively showed the connection between theory and practice, demonstrating the principles of gravitational acceleration. The variations observed between the experimental and accepted values of $g$ shows the complexities of physical measurements.










    
\end{document}
