\documentclass{report}

\input{~/dev/latex/template/preamble.tex}
\input{~/dev/latex/template/macros.tex}

\title{\Huge{}}
\author{\huge{Nathan Warner}}
\date{\huge{}}
\fancyhf{}
\rhead{}
\fancyhead[R]{\itshape Warner} % Left header: Section name
\fancyhead[L]{\itshape\leftmark}  % Right header: Page number
\cfoot{\thepage}
\renewcommand{\headrulewidth}{0pt} % Optional: Removes the header line
%\pagestyle{fancy}
%\fancyhf{}
%\lhead{Warner \thepage}
%\rhead{}
% \lhead{\leftmark}
%\cfoot{\thepage}
%\setborder
% \usepackage[default]{sourcecodepro}
% \usepackage[T1]{fontenc}

% Change the title
\hypersetup{
    pdftitle={}
}

\begin{document}
    % \maketitle
    %     \begin{titlepage}
    %    \begin{center}
    %        \vspace*{1cm}
    %
    %        \textbf{Lab Report}
    %
    %        \vspace{0.5cm}
    %         Skyscaper
    %             
    %        \vspace{1.5cm}
    %
    %        \textbf{Nathan Warner}
    %
    %        \vfill
    %             
    %             
    %        \vspace{0.8cm}
    %      
    %        \includegraphics[width=0.4\textwidth]{~/niu/seal.png} \\
    %        
    %             
    %    \end{center}
    % \end{titlepage}
    \begin{center}
        \begin{Huge}
            Collision Lab
        \end{Huge}
        \begin{Large}
            \bigbreak \noindent 
            Nate Warner
            \smallbreak \noindent
            April 16, 2024
            \bigbreak \noindent 
            Section 253D, 6:00 PM Tuesday 
        \end{Large}
    \end{center}
    \pagebreak 
    \tableofcontents
    \pagebreak \bigbreak \noindent 
    \begin{center}
    \textbf{Abstract}
    \end{center}
    \begin{adjustwidth}{.3in}{.3in}
        \hspace{\parindent} 
    \end{adjustwidth}

    \bigbreak \noindent 
    \section{Theory}
    \bigbreak \noindent 
    When objects come into contact while in motion (note that one may instead begin stationary), this interaction is commonly referred to as a \textit{collision}, during which momentum and energy may be exchanged between them. We can categorize collisions into two primary types: elastic and inelastic. 
    \bigbreak \noindent 
    Momentum is quantified by its velocity, which is the rate at which it covers distance relative to a point of reference over some period of time. We denote
    \begin{equation}
        v=\frac{\ell}{t}
    \end{equation}
    \bigbreak \noindent
    Where $\ell$ is the length of the distance covered, and $t$ is the time that it took to cover such distance.
    \bigbreak \noindent 
    Considering an object has mass, its momentum, denoted by $p$, is calculated by multiplying its velocity by its mass.
    \begin{equation}
        p=m v
    \end{equation}
    \bigbreak \noindent 
    In scenarios involving several objects, the total momentum is the sum of the individual momenta.
    \begin{equation}
        p_{\mathrm{TOT}}=p_1+p_2+\ldots=m_1 v_1+m_2 v_2+\ldots
    \end{equation}
    \bigbreak \noindent 
    We first explore inelastic collisions, which is when two objects stick together and form a single entity after they collide. In this case, the momentum remains conserved, but kinetic energy does not.
    \begin{equation}
        \begin{aligned}
            K_i & \neq K_f \\
            K_{1, i}+K_{2, i} & \neq K_{1, f}+K_{2, f} \\
            \frac{1}{2} m_1 v_{1, i}^2+\frac{1}{2} m_2 v_{2, i}^2 & \neq \frac{1}{2} m_1 v_{1, f}^2+\frac{1}{2} m_2 v_{2, f}^2
        \end{aligned}
    \end{equation}
    \bigbreak \noindent 
    Thus, in these scenarios, only momentum is preserved from start to finish.
    \begin{equation}
        \begin{aligned}
            p_{\mathrm{TOT}, i} & =p_{\mathrm{TOT}, f} \\
            p_{1, i}+p_{2, i} & =p_{1, f}+p_{2, f} \\
            m_1 v_{1, i}+m_2 v_{2, i} & =\left(m_1+m_2\right) v_f
        \end{aligned}
    \end{equation}
    \bigbreak \noindent 
    Next, we have elastic collisions, where both kinetic energy and momentum remain conserved.
    \begin{equation}
        \begin{aligned}
            K_i & =K_f \\
            p_i & =p_f
        \end{aligned}
    \end{equation}
    \bigbreak \noindent 
    Such collisions can be visualized with two billiard balls or hockey pucks gliding across ice, where they bounce off each other without sticking. Applying the known equations for momentum and kinetic energy, we get the following equations
    \begin{equation}
        \begin{aligned}
            \frac{1}{2} m_1 v_{1, i}^2+\frac{1}{2} m_2 v_{2, i}^2 & =\frac{1}{2} m_1 v_{1, f}^2+\frac{1}{2} m_2 v_{2, f}^2 \\
            m_1 v_{1, i}+m_2 v_{2, i} & =m_1 v_{1, f}+m_2 v_{2, f}
        \end{aligned}
    \end{equation}
    \bigbreak \noindent 
    Ideally, these systems should not experience energy losses from friction, sound, or other energy transformations. However, this is not always the case when considering non-ideal systems, we measure this loss as a percentage, defined as the absolute difference between the initial and final values relative to the initial value, multiplied by 100\%.
    \begin{equation}
        \% \operatorname{Loss}=\frac{\left|\varphi_f-\varphi_i\right|}{\varphi_i} \times 100 \%
    \end{equation}
    \bigbreak \noindent 
    Where $\varphi$ denotes kinetic energy.

    \bigbreak \noindent 
    \subsection{Procedure}
    \bigbreak \noindent 
    In this experiment, we are going to test four scenarios, two elastic collisions and two inelastic collisions. These first steps are general steps that we will just get out of the way first.
    \begin{enumerate}
        \item Mass out the two carts and record them. If the carts are too heavy, we have 500 g cylinders that can hook onto the scale.
        \item Mass out the extra weight that will go onto one of the carts and record.
        \item Measure the length of the metal flag that goes on the carts. This will be the length that you use to find the velocity of each of the carts.
        \item Place the flags on the carts such that the long edge is facing the side of the track and place the carts on the track.
    \end{enumerate}

    \bigbreak \noindent 
    \subsection{Image of the setup}
    \bigbreak \noindent 
    \fig{1}{./figures/1.png}
    \fc{Illustrates the setup of the carts and the photogates used in the experiment}


    \bigbreak \noindent 
    \section{Data}
    \bigbreak \noindent 
    Before we branch off to the specific cases, we record a few measurements that will remain constant throughout each trial. These measurements include the mass of the light cart $(m_{L})$, the mass of the heavy cart $(m_{H})$, and the length $\ell$ that will be used to compute the velocity, this length is the length of the flag above the cart.
    \bigbreak \noindent 
    \begin{center}
        \begin{tabular}{c|c}
            Name & Measurement \\
            \hline
            $m_{L}$ & 0.565 kg\\
            $m_{H}$  & 1.082 kg\\
            $\ell$ & 0.06 m
        \end{tabular}
    \end{center}
    \tc{Displays the constants that will be used among all four trials}

    \bigbreak \noindent 
    \subsection{Inelastic: Light cart stationary}
    \bigbreak \noindent 
    We begin this trial but placing the light cart ($m_{L}$) inbetween the two photogates (see figure 1). We then send the heavy cart towards the light cart. The heavy cart passes through the first photogate, then collides with the stationary cart. The two carts will stick together after the collision.
    \bigbreak \noindent 
    Once the collided carts pass through the second photogate, we will have two time mesaurements, the time it took for the heavy cart to pass through the first photogate, and the time it took for the collided carts to pass through the second photogate. These measurements are displayed in the following table
    \bigbreak \noindent 
    \begin{center}
        \begin{tabular}{c|c}
            Cart & Time difference ($\Delta t$ s) \\
            \hline
            Heavy $(m_{i\rightarrow p_{1}}) $ & 0.13855 \\
            Joined $(m_{f \rightarrow p_{2}})$ & 0.148
        \end{tabular}
    \end{center}
    \tc{Shows the time it took for the objects to pass through each photogate}
    \bigbreak \noindent 
    The notation used in this table $(m_{i \rightarrow p_{1}})$, is used to denote the inital mass moving through photogate 1, and $(m_{f \rightarrow p_{2}})$ denoting final mass moving through photogate 2
    \bigbreak \noindent 
    \nt{We denote the first time difference $\Delta t_{1}$, and the second time difference $\Delta t_{2}$}

    \bigbreak \noindent 
    \subsection{Inelastic: Heavy cart stationary}
    \bigbreak \noindent 
    This second trial is similar to the previous, but this time the heavy cart is positioned in the middle (stationary).
    \bigbreak \noindent 
    Again, we display the time measurements in the following table
    \bigbreak \noindent 
    \begin{center}
        \begin{tabular}{c|c}
            Cart & Time difference ($\Delta t$ s) \\
            \hline
            Light $(m_{i\rightarrow p_{1}}) $ & 0.134 \\
            Joined $(m_{f\rightarrow p_{2}}) $& 0.204
        \end{tabular}
    \end{center}
    \tc{Shows the time it took for the objects to pass through each photogate}


    \pagebreak 
    \subsection{Elastic: Light cart stationary}
    \bigbreak \noindent 
    This trial will be similar to the first, where the light cart is placed in the middle. However, we flip the carts around such that the magnets of each cart do not face eachother. Hence, they will not stick together after the collision.
    \bigbreak \noindent 
    This trial will have three time measuremens, whereas the last two trials had two. The first will be the time it took for the leftmost cart to pass through the first photogate ($m_{i \rightarrow p_{1}} $). The second will be the time it took for the stationary (middle) mass to pass through the second photogate after collision ($m_{s \rightarrow p_{2}} $), and the third will be the time it took for the initial mass to pass through the second photogate after collision $(m_{i \rightarrow p_{2}})$.
    \bigbreak \noindent 
    \begin{center}
        \begin{tabular}{c|c}
            Cart & Time difference ($\Delta t$ s) \\
            \hline
            $m_{i\rightarrow p_{1}} $& 0.085\\
            $m_{s\rightarrow p_{2}} $ & 0.172\\
            $m_{i \rightarrow p_{2}}$ &0.645
        \end{tabular}
    \end{center}
    \tc{Shows the time it took for the objects to pass through each photogate}


    

    \bigbreak \noindent 
    \subsection{Inelastic: Heavy cart stationary}
    \bigbreak \noindent 
    I regret to inform that this trial was not able to be conducted. Technical difficulties were experienced.



    \bigbreak \noindent 
    \section{Results}
    \bigbreak \noindent 
    Using the data found in the previous tables, we compute the velocities, momentums, and kinetic energys for each trial. 

    \bigbreak \noindent 
    \subsection{Inelastic: Light cart stationary}
    \bigbreak \noindent 
    We start with the velocites. As discussed previously, we show these computations in this section and then simply display the results in the proceeding sections, without computations. 
    \bigbreak \noindent 
    Using the length of the flag recorded at the start of the experiment $(\ell = 0.06m)$, and the times found in table two, we find
    \begin{align*}
        v_{m_{i}\rightarrow p_{1}} = \frac{\ell}{\Delta t} = \frac{0.06m}{0.1385s} = 0.363\, m/s \\
        v_{m_{f} \rightarrow p_{2}} = \frac{\ell}{\Delta t} = \frac{0.06m}{0.148s} = 0.405\, m/s
    .\end{align*}
    \bigbreak \noindent 
    Next, we use equation four to find the momenta
    \begin{align*}
        p_{i} &= m_{i}v_{m_{i}\rightarrow p_{1}} = 1.08\, kg \cdot 0.363\, m/s = 0.392\, kg \cdot \frac{m}{s} \\
        p_{f} &= (m_{i} + m_{f})v_{m_{f} \rightarrow p_{2}} = 1.645\, kg \cdot 0.405\, m/s = 0.667\, kg \cdot \frac{m}{s}
    .\end{align*}
    \bigbreak \noindent 
    Finally, we find the kinetic energys
    \begin{align*}
        K_{i} = \frac{1}{2}m_{i}v_{m_{i} \rightarrow p_{1}}^{2} = \frac{1}{2}(1.08\, kg)(0.363\, m/s)^{2} = 0.071\, J \\
        K_{f} = \frac{1}{2}m_{f}v_{m_{f} \rightarrow p_{2}}^{2} = \frac{1}{2}(1.645\, kg)(0.405\, m/s)^{2} = 0.135\, J
    .\end{align*}

    \bigbreak \noindent 
    \subsection{Results tables for all trials}
    The following table displays the results for the trial we just analyzed, as well as the remaining two.
    \bigbreak \noindent 
    \subsubsection{Velocity table: Inelastic trials}
    \bigbreak \noindent 
    \begin{center}
        \begin{tabular}{c|c|c}
            Trial & $v_{i}$ $m/s$ & $v_{f}$ $m/s$\\
            \hline
            1 & 0.363  & 0.405 \\ 
            2  & 0.448  & 0.294 
        \end{tabular}
    \end{center}
    \tc{Displays the velocities found during the first two inelastic trials}
    \bigbreak \noindent 
    \subsubsection{Velocity table: Elastic trial}
    \begin{center}
        \begin{tabular}{c|c|c|c}
            Trial & $v_{m_{i} \rightarrow p_{1}}$ $m/s$ & $v_{m_{s} \rightarrow p_{2}}$ $m/s$ & $v_{m_{i} \rightarrow p_{2}}$ $m/s$ \\
            \hline
            3 & 0.705 & 0.349 & 0.0938
        \end{tabular}
    \end{center}
    \tc{Displays the velocities found during the elastic trial}

    \bigbreak \noindent 
    \subsubsection{Momenta and Energy table}
    \begin{center}
        \begin{tabular}{c|c|c|c|c}
            Trial & $P_{i}$ $kg \cdot \frac{m}{s} $ & $P_{f}$ $kg \cdot \frac{m}{s} $ & $K_{i}$ $J$ & $K_{f}$ $J$ \\
            \hline
            1 & 0.392 & 0.667 & 0.071 & 0.135 \\ 
            2  & 0.253 & 0.484 & 0.057 & 0.071\\ 
            3  & 0.7614 & 0.298 & 0.268 & 0.039
        \end{tabular}
    \end{center}
    \tc{Shows the momentum and energy found in each trial}

    \bigbreak \noindent 
    \subsection{Percent loss: Momenta}
    \bigbreak \noindent 
    We now find the percent loss among each trials initial and final momentum. For this, we use equation 6. We show this computation for the first trial, and then report the remaining findings in a following table
    \begin{align*}
        \% \operatorname{Loss}&=\frac{\left|\varphi_f-\varphi_i\right|}{\varphi_i} \times 100 \% \\
                              &= \frac{\abs{0.667 kg \cdotm/s - 0.392 kg \cdot m/s}}{0.392 kg \cdot m/s} \times 100\% \\
                              &=70\%
    .\end{align*}

    \bigbreak \noindent 
    \subsection{Percent loss: Energy}
        \begin{align*}
        \% \operatorname{Loss}&=\frac{\left|\varphi_f-\varphi_i\right|}{\varphi_i} \times 100 \% \\
                              &= \frac{\abs{0.135J - 0.071J}}{0.071J} \times 100\% \\
                              &=90\%
    .\end{align*}

    \bigbreak \noindent 
    \subsection{Percent loss data table}
    \begin{center}
        \begin{tabular}{c|c|c}
            Trial  &  Momentum percent loss & Energy percent loss \\
            \hline
            1 &70\% & 90\%\\ 
            2 & 91\% & 24\%\\ 
            3 & 61\% & 85\%\\
        \end{tabular}
    \end{center}



    \bigbreak \noindent 
    \section{Discussion}
    \bigbreak \noindent 
    In this experiment, we explored how momentum and energy behave during collisions, focusing on the difference between elastic and inelastic collisions.  We showed the principle of momentum conservation, which holds regardless of the collision type. Our procedures ranged from measuring the mass of moving carts to recording their velocities after the collision, which allowed us to look into the nuances of kinetic energy behavior and momentum transfer.
    \bigbreak \noindent 
    However, things didn't always go as smoothly as we hoped. We noticed some inconsistencies in our momentum calculations. For instance, in inelastic collisions where the carts stuck together, it looked like we ended up with more momentum than we started with, which shouldn't happen. We think this might have been due to errors in how we measured the time it took for the carts to pass through the sensors, which affected our velocity calculations. It's a good reminder that even small mistakes in measurement can lead to big discrepancies in results, and it's important to be as accurate as possible.
    \bigbreak \noindent 
    When initial and final momentum measurements don't match up in experiments, several factors can be at play, beyond just measurement errors. Friction between the moving objects and the surfaces can lead to a loss of kinetic energy, affecting the final momentum measurement. Any external force acting on the system can introduce additional momentum. In a perfectly isolated system, momentum would be conserved, but external forces can add or subtract from the system's total momentum.




    \bigbreak \noindent 
    \section{Conclusion}
    \bigbreak \noindent 
    In conclusion, this lab's look into the types of collisions has not only deepened our understanding of the concepts of momentum and energy conservation but has also provided a look at the challenges of capturing these principles in real-world scenarios. Despite setting up our experiments to explore both elastic and inelastic collisions, we encountered unexpected outcomes that were vastly different from the theoretical predictions. These discrepancies show us the complexities of experiments, where factors such as measurement errors in timing and velocity determinations can influence the results. The apparent momentum gain in some inelastic collisions hinted at underlying inaccuracies in our measurement techniques. How we computed the final velocity for the inelastic collisions most likely contained some fatal errors, which caused the resulting momentum to be greater than the initial.









    
\end{document}
