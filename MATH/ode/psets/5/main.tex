 \documentclass{report}
 
 \input{~/dev/latex/template/preamble.tex}
 \input{~/dev/latex/template/macros.tex}
 
 \title{\Huge{}}
 \author{\huge{Nathan Warner}}
 \date{\huge{}}
 \fancyhf{}
 \rhead{}
 \fancyhead[R]{\itshape Warner} % Left header: Section name
 \fancyhead[L]{\itshape\leftmark}  % Right header: Page number
 \cfoot{\thepage}
 \renewcommand{\headrulewidth}{0pt} % Optional: Removes the header line
 %\pagestyle{fancy}
 %\fancyhf{}
 %\lhead{Warner \thepage}
 %\rhead{}
 % \lhead{\leftmark}
 %\cfoot{\thepage}
 %\setborder
 % \usepackage[default]{sourcecodepro}
 % \usepackage[T1]{fontenc}
 
 % Change the title
 \hypersetup{
     pdftitle={}
 }

 \geometry{
  left=1in,
  right=1in,
  top=1in,
  bottom=1in
}
 
 \begin{document}
     % \maketitle
     %     \begin{titlepage}
     %    \begin{center}
     %        \vspace*{1cm}
     % 
     %        \textbf{}
     % 
     %        \vspace{0.5cm}
     %         
     %             
     %        \vspace{1.5cm}
     % 
     %        \textbf{Nathan Warner}
     % 
     %        \vfill
     %             
     %             
     %        \vspace{0.8cm}
     %      
     %        \includegraphics[width=0.4\textwidth]{~/niu/seal.png}
     %             
     %        Computer Science \\
     %        Northern Illinois University\\
     %        United States\\
     %        
     %             
     %    \end{center}
     % \end{titlepage}
     % \tableofcontents
    \pagebreak \bigbreak \noindent
    Nate Warner \ \quad \quad \quad \quad \quad \quad \quad \quad \quad \quad \quad \quad  MATH 336 \quad  \quad \quad \quad \quad \quad \quad \quad \quad \ \ \quad \quad Spring 2026
    \begin{center}
        \textbf{Problem set 5 - Due: Monday, February 16}
    \end{center}
    \bigbreak \noindent 
    \begin{mdframed}
        1. 
        \begin{align*}
            x^{2}y^{\prime} = xy + y^{2}
        .\end{align*}
    \end{mdframed}
    \bigbreak \noindent 
    We have
    \begin{align*}
        x^{2}y^{\prime} = xy + y^{2} \implies x^{2}\; dy -(xy+y^{2})\; dx = 0
    .\end{align*}
    Notice that this is homogeneous, so let $y = xv$, then $y^{\prime} = xv^{\prime} + v $, and
    \begin{align*}
        x^{2}y^{\prime} &= xy  + y^{2} \implies x^{2}(xv^{\prime} + v) = x^{2}v + x^{2}v^{2} \\
        \implies x^{3}v^{\prime} + x^{2}v &= x^{2}v + x^{2}v^{2}  \implies x^{3}v^{\prime} = x^{2}v^{2} \\
        \implies xv^{\prime} &= v^{2}  \implies \frac{1}{v^{2}}\; dv = \frac{1}{x}\; dx
    .\end{align*}
    So,
    \begin{align*}
        \int \frac{1}{v^{2}}\; dv &= \int \frac{1}{x}\; dx \implies -\frac{1}{v} = \ln{\left(\left\lvert x \right\rvert\right)} + C \\
        \implies v &= \frac{1}{C - \ln{\left\lvert x \right\rvert}}
    .\end{align*}
    Now, we switch back to $y$, so
    \begin{align*}
        \frac{y}{x} &= \frac{1}{C - \ln{\left(\left\lvert x \right\rvert\right)}} \implies y = \frac{x}{C-\ln{\left(\left\lvert x \right\rvert\right)}}
    .\end{align*}


    \bigbreak \noindent 
    \begin{mdframed}
        2. 
        \begin{align*}
           xyy^{\prime} = y^{2} +x\sqrt{4x^{2} + y^{2}}
        .\end{align*}
    \end{mdframed}
    \bigbreak \noindent 
    We have
    \begin{align*}
        xy\; dy - (y^{2} + x\sqrt{4x^{2} + y^{2}})\; dx = 0
    .\end{align*}
    Let $M(x,y) = -(y^{2} + x\sqrt{4x^{2}} + y^{2}) $, and $N(x,y) = xy $. We check for homogeneity.
    \begin{align*}
        N(\lambda x, \lambda y) = (\lambda x)(\lambda y) = \lambda^{2}xy
    ,\end{align*}
    and
    \begin{align*}
        M( \lambda x, \lambda y) &= -((\lambda y)^{2} + (\lambda x)\sqrt{4(\lambda x)^{2} + (\lambda y)^{2}}) \\
        &= -(\lambda^{2} y^{2} + \lambda x\sqrt{4\lambda^{2}x^{2} + \lambda^{2}y^{2}}) \\
        &= -(\lambda^{2}y^{2} + \lambda^{2} x\sqrt{4x^{2} + y^{2}})
    .\end{align*}
    Thus, they are homogeneous of the same degree, and the DE is therefore homogeneous. So, let $y = xv$, then $y^{\prime} = xv^{\prime} + v $, and 
    \begin{align*}
        xyy^{\prime} &= y^{2} + x\sqrt{4x^{2} + y^{2}} \implies x(xv)(xv^{\prime} + v) = x^{2}v^{2} + x\sqrt{4x^{2} + x^{2}v^{2}} \\
        \implies x^{3}vv^{\prime} + x^{2}v^{2} &= x^{2}v^{2} + x\sqrt{4x^{2} + x^{2}v^{2}} \implies x^{3}vv^{\prime} = x\sqrt{4x^{2} + x^{2}v^{2}} \\
        \implies x^{3}vv^{\prime} &= x(x)\sqrt{4 + v^{2}} = x^{2}\sqrt{4+v^{2}} \implies xvv^{\prime} = \sqrt{4+v^{2}}
    .\end{align*}
    So, we can separate to get
    \begin{align*}
        \frac{v}{\sqrt{4+v^{2}}} \; dv = \frac{1}{x}\; dx
    .\end{align*}
    Thus,
    \begin{align*}
        \int \frac{v}{\sqrt{4+v^{2}}}\; dv = \int \frac{1}{x}\; dx = \ln{\left\lvert x \right\rvert}  + C
    .\end{align*}
    For the LHS integral, Let $u = 4+v^{2}$, then $\frac{1}{2}\; du = v\; dv $, so
    \begin{align*}
        \int \frac{v}{\sqrt{4+v^{2}}}\; dv = \frac{1}{2}\int \frac{1}{\sqrt{u}}\; du = \frac{1}{2}(2)\sqrt{u} = \sqrt{4+v^{2}}
    .\end{align*}
    Thus,
    \begin{align*}
        \sqrt{4+v^{2}} = \ln{\left\lvert x \right\rvert} + C
    .\end{align*}
    Then,
    \begin{align*}
        \sqrt{4 + \left(\frac{y}{x}\right)^{2}} = \ln{\left(\left\lvert x \right\rvert\right)} + C
    \end{align*}
    is an implicit solution.



    \bigbreak \noindent 
    \begin{mdframed}
        3. 
        \begin{align*}
            x^{2}y^{\prime} + 2xy = 5y^{3}
        .\end{align*}
    \end{mdframed}
    \bigbreak \noindent 
    We have
    \begin{align*}
        y^{\prime} + \frac{2}{x}y = \frac{5}{x^{2}}y^{3}
    .\end{align*}
    Which is a Bernoulli DE, so let $ v = y^{1-3} = y^{-2}$, then $\frac{dv}{dx} = -\frac{2}{y^{3}}y^{\prime} $, so $y^{\prime} = -\frac{1}{2}y^{3}v^{\prime}$. Then,
    \begin{align*}
        -\frac{1}{2}y^{3}v^{\prime} + \frac{2}{x}y = \frac{5}{x^{2}}y^{3} \implies -\frac{1}{2}v^{\prime} + \frac{2}{x}y^{-2} = \frac{5}{x^{2}} \\
        \implies v^{\prime} - \frac{4}{x}v &= -\frac{10}{x^{2}}
    ,\end{align*}
    which is linear in $v$. Now, define
    \begin{align*}
        \mu(x) &= e^{-4\int \frac{1}{x}\; dx} = e^{-4 \ln{\left(x\right)}} = \frac{1}{x^{4}}
    .\end{align*}
    So,
    \begin{align*}
        \left(\frac{1}{x^{4}}v\right)^{\prime} &= -\frac{10}{x^{2}}\left(\frac{1}{x^{4}}\right) = -\frac{10}{x^{6}}
    .\end{align*}
    Thus,
    \begin{align*}
        \frac{1}{x^{4}}v &= -10\int x^{-6}\; dx = -\frac{10}{-5}x^{-5} + C = \frac{2}{x^{5}} + C \\
        \implies v &= x^{4}\left(\frac{2}{x^{5}} + C\right) = \frac{2}{x} + Cx^{4} \\
        \implies \frac{1}{y^{2}} &= \frac{2}{x} + Cx^{4}
    .\end{align*}
    This is an implicit solution.



    \bigbreak \noindent 
    \begin{mdframed}
        4. 
        \begin{align*}
            xy^{\prime} + 6y = 3xy^{\frac{4}{3}}
        .\end{align*}
    \end{mdframed}
    \bigbreak \noindent 
    We have
    \begin{align*}
        y^{\prime} + \frac{6}{x}y = 3y^{\frac{4}{3}}
    .\end{align*}
    So, let $v = y^{1-\frac{4}{3}} = y^{-\frac{1}{3}}$, then $v^{\prime} = -\frac{1}{3}y^{-\frac{4}{3}}y^{\prime}$, so $y^{\prime} = -3y^{\frac{4}{3}}v^{\prime} $. Then,
    \begin{align*}
        -3y^{\frac{4}{3}}v^{\prime} + \frac{6}{x}y &= 3y^{\frac{4}{3}} \implies -3v^{\prime} + \frac{6}{x}y^{-\frac{1}{3}} = 3 \\
        \implies v^{\prime} -\frac{2}{x}v &= -1
    .\end{align*}
    Notice that this is linear in $v$, so define
    \begin{align*}
        \mu(x) = e^{-2\int \frac{1}{x}\; dx} = e^{-2\ln{\left(x\right)}} = \frac{1}{x^{2}}
    .\end{align*}
    Then,
    \begin{align*}
        \left(\frac{1}{x^{2}}v\right)^{\prime} &= -\frac{1}{x^{2}}
    .\end{align*}
    So,
    \begin{align*}
        \frac{1}{x^{2}}v &= -\int \frac{1}{x^{2}}\; dx \implies \frac{1}{x^{2}}v = \frac{1}{x} + C
    .\end{align*}
    Thus,
    \begin{align*}
        v &= x^{2}\left(\frac{1}{x} + C\right) = x + Cx^{2}
    .\end{align*}
    Therefore,
    \begin{align*}
        \frac{1}{y^{\frac{1}{3}}} &= x + Cx^{2}
    \end{align*}
    is an implicit solution.


    \bigbreak \noindent 
    \begin{mdframed}
        5.
        \begin{align*}
            (3x^{2} + 2y^{2})\; dx + (4xy + 6y^{2})\; dy = 0
        .\end{align*}
    \end{mdframed}
    \bigbreak \noindent 
    We can check if this DE is exact by letting $M(x,y) = 3x^{2} + 2y^{2}$, and $N(x,y) =4xy + 6y^{2} $ and checking if $M_{y} = N_{x}$. We have
    \begin{align*}
        M_{y} = 4y, \quad N_{x} = 4y
    .\end{align*}
    Thus, the DE is exact. Now, we can find $F(x,y)$ by integrating $M$ with respect to $x$, since $M = F_{x}$. So,
    \begin{align*}
        F(x,y) = \int M\; dx = \int (3x^{2} + 2y^{2})\; dx = x^{3} + 2xy^{2} + h(y)
    .\end{align*}
    Now, we differentiate with respect to $y$, 
    \begin{align*}
        \frac{\partial F}{\partial y} &= 4xy + h^{\prime}(y) 
    .\end{align*}
    If we set this equal to $F_{y} = N$, we see that
    \begin{align*}
        4xy + h^{\prime}(y) = 4xy + 6y^{2}
    .\end{align*}
    So, $h^{\prime}(y) = 6y^{2}$. Therefore,
    \begin{align*}
        h(y) = \int h^{\prime}(y)\; dy = \int 6y^{2}\; dy = 2y^{3} + K
    .\end{align*}
    Thus,
    \begin{align*}
        F(x,y) &= x^{3} + 2xy^{2} + 2y^{3} + K
    .\end{align*}
    But, $F(x,y) = C_0$, so
    \begin{align*}
        x^{3} + 2xy^{2} + 2y^{3} + K = C_0 \implies x^{3} + 2xy^{2} + 2y^{3} = C
    \end{align*}
    Is an implicit solution.


    \bigbreak \noindent 
    \begin{mdframed}
        6. 
        \begin{align*}
            \left(x^{3} + \frac{y}{x}\right)\; dx + (y^{2} + \ln{\left(x\right)})\; dy = 0
        .\end{align*}
    \end{mdframed}
    \bigbreak \noindent 
    We check $M_{y}(x,y) = \left(x^{3} + \frac{y}{x}\right)_{y}$ against $N_{x}(x,y) = (y^{2} + \ln{\left(x\right)})_{x}$
    \begin{align*}
        M_{y} = \frac{1}{x}, \quad N_{x} = \frac{1}{x}
    .\end{align*}
    Thus, the DE is exact. So,
    \begin{align*}
        F(x,y) = \int M\; dx = \int \left(x^{3} + \frac{y}{x}\right)\; dx = \frac{1}{4}x^{4} + y\ln{\left(x\right)} + h(y)
    .\end{align*}
    So,
    \begin{align*}
        F_{y} = \ln{\left(x\right)} + h^{\prime}(y)
    .\end{align*}
    Thus,
    \begin{align*}
        \ln{\left(x\right)} + h^{\prime}(y) = y^{2} + \ln{\left(x\right)} \implies h^{\prime}(y) = y^{2}
    ,\end{align*}
    so $h(y) = \frac{1}{3}y^{3} $. Now, we have that
    \begin{align*}
        F(x,y) = \frac{1}{4}x^{4} +y\ln{\left(x\right)} + \frac{1}{3}y^{3} + K
    .\end{align*}
    So, since $F(x,y) = C_0 $,
    \begin{align*}
       \frac{1}{4}x^{4} + y\ln{\left(x\right)}  + \frac{1}{3}y^{3} = C
    \end{align*}
    is an implicit solution.


    \bigbreak \noindent 
    \begin{mdframed}
        7. 
        \begin{align*}
            (1+ye^{xy})\; dx  + (2y + xe^{xy})\; dy = 0
        .\end{align*}
    \end{mdframed}
    \bigbreak \noindent 
    We have
    \begin{align*}
        M_{y} = e^{xy} + xye^{xy}, \quad N_{x} = e^{xy} + xye^{xy}
    .\end{align*}
    Thus, the DE is exact, and
    \begin{align*}
        F(x,y) = \int (1+ye^{xy})\; dx = x + \frac{y}{y}e^{xy} + h(y) = x + e^{xy} + h(y)
    .\end{align*}
    Thus,
    \begin{align*}
        F_{y} = xe^{xy} + h^{\prime}(y)
    .\end{align*}
    So,
    \begin{align*}
        xe^{xy} + h^{\prime}(y) = 2y + xe^{xy} \implies h^{\prime}(y) = 2y \implies h(y) = y^{2} + K
    .\end{align*}
    Thus,
    \begin{align*}
        F(x,y) = x + e^{xy} + y^{2} + K
    ,\end{align*}
    and 
    \begin{align*}
        x + e^{xy} + y^{2} = C 
    \end{align*}
    is an implicit solution.





\end{document}
