 \documentclass{report}
 \input{~/dev/latex/template/preamble.tex}
 \input{~/dev/latex/template/macros.tex}
 
 \title{\Huge{}}
 \author{\huge{Nathan Warner}}
 \date{\huge{}}
 \fancyhf{}
 \rhead{}
 \fancyhead[R]{\itshape Warner} % Left header: Section name
 \fancyhead[L]{\itshape\leftmark}  % Right header: Page number
 \cfoot{\thepage}
 \renewcommand{\headrulewidth}{0pt} % Optional: Removes the header line
 %\pagestyle{fancy}
 %\fancyhf{}
 %\lhead{Warner \thepage}
 %\rhead{}
 % \lhead{\leftmark}
 %\cfoot{\thepage}
 %\setborder
 % \usepackage[default]{sourcecodepro}
 % \usepackage[T1]{fontenc}
 
 % Change the title
 \hypersetup{
     pdftitle={}
 }

 \geometry{
  left=1in,
  right=1in,
  top=1in,
  bottom=1in
}
 
 \begin{document}
     % \maketitle
     %     \begin{titlepage}
     %    \begin{center}
     %        \vspace*{1cm}
     % 
     %        \textbf{}
     % 
     %        \vspace{0.5cm}
     %         
     %             
     %        \vspace{1.5cm}
     % 
     %        \textbf{Nathan Warner}
     % 
     %        \vfill
     %             
     %             
     %        \vspace{0.8cm}
     %      
     %        \includegraphics[width=0.4\textwidth]{~/niu/seal.png}
     %             
     %        Computer Science \\
     %        Northern Illinois University\\
     %        United States\\
     %        
     %             
     %    \end{center}
     % \end{titlepage}
     % \tableofcontents
    \pagebreak \bigbreak \noindent
    Nate Warner \ \quad \quad \quad \quad \quad \quad \quad \quad \quad \quad \quad \quad  MATH 336 \quad  \quad \quad \quad \quad \quad \quad \quad \quad \ \ \quad \quad Spring 2026
    \begin{center}
        \textbf{Problem set 3 - Due: Monday, Febuary 2}
    \end{center}
    \bigbreak \noindent 
    \begin{mdframed}
        1. Consider the differential equation
        \begin{align*}
            y\frac{dy}{dx} = 4x
        .\end{align*}
        \begin{enumerate}[label=(\alph*)]
            \item Verify that $4x^{2} - y^{2} = C$ gives a one-parameter family of implicit solutions. 
            \item State the existence and uniqueness theorem for the IVP 
                \begin{align*}
                    \begin{cases}
                        \frac{dy}{dx} = f(x,y), \\
                        y(x_{0}) = y_{0}
                    \end{cases}
                .\end{align*}
            \item If $x_{0} \ne 0$, does this theorem guarantee the existence of a solution to the IVP
                \begin{align*}
                    \begin{cases}
                        y\frac{dy}{dx} =4x, \\
                        y(x_{0}) = 0
                    \end{cases}
                .\end{align*}
            \item Give two distinct solutions to
                \begin{align*}
                    \begin{cases}
                        y\frac{dy}{dx}      =4x, \\
                        y(0) = 0
                    \end{cases}
                .\end{align*}
                \textbf{Hint:} consider $y = kx$ for some constant $k$).
        \end{enumerate}
    \end{mdframed}
    \bigbreak \noindent 
    a.) We differentiate the proposed solution implicitly,
    \begin{align*}
        \frac{d}{dx}\left(4x^{2} - y^{2}\right) &= \frac{d}{dx}C \\
        \implies 8x -2y \frac{dy}{dx} &= 0 \\
        \implies y\frac{dy}{dx} &= \frac{-8x}{-2} = 4x
    .\end{align*}
    Thus, verified.
    \bigbreak \noindent 
    b.) If $f(x,y)$ is continuous around the initial point $(x_{0}, y_{0})$, then a solution exists on an open interval containing $x_{0}$. If $\frac{\delta f}{\delta y}$ is continuous around $(x_{0}, y_{0})$, then the solution is unique on that interval.
    \bigbreak \noindent 
    c.) No, since $f(x,y) = \frac{4x}{y}$ is not continuous at the initial point $(x_{0}, 0)$ for any $x_{0}$, the theorem does not guarantee a solution on an open interval containing $x_{0}$.
    \bigbreak \noindent 
    d.) If we consider $y = kx$, for some constant $k$, then
    \begin{align*}
        \frac{dy}{dx} = k
    .\end{align*}
    Using the differential equation $y \frac{dy}{dx} = 4x$, we see that
    \begin{align*}
        (kx)(k) &= 4x \implies k^{2}x = 4x \implies k^{2} = 4 \implies k = \pm 2
    .\end{align*}
    Thus, two distinct solutions are
    \begin{align*}
        y(x) = 2x, \quad y(x) = -2x
    ,\end{align*}
    which both satisfy the initial condition $y(0) = 0$.

    \bigbreak \noindent 
    \begin{mdframed}
        2. Solve the following differential equation
        \begin{align*}
            \frac{dy}{dx} = \frac{(x-1)y^{5}}{x^{2}(2y^{3}-y)}
        .\end{align*}
    \end{mdframed}
    \bigbreak \noindent 
    We have
    \begin{align*}
        \frac{dy}{dx} = \frac{x-1}{x^{2}}\left(\frac{y^{5}}{2y^{3}-y}\right)
    .\end{align*}
    So,
    \begin{align*}
        \frac{2y^{3}-y}{y^{5}}\; dy &= \frac{x-1}{x^{2}}\; dx \\
        \implies \int \frac{2y^{3}-y}{y^{5}}\; dy &= \int \frac{x-1}{x^{2}}\; dx \\ 
        \implies \int (2y^{-2} - y^{-4})\; dy &= \int (x^{-1}-x^{-2})\; dx \\
        \implies -\frac{2}{y} + \frac{1}{3y^{3}} &= \ln{\left\lvert x \right\rvert} + \frac{1}{x} + C
    .\end{align*}

    \pagebreak \bigbreak \noindent 
    \begin{mdframed}
        3. Solve the following differential equation
        \begin{align*}
            \frac{dy}{dx} = 2xy^{2} + 3x^{2}y^{2}, \quad y(1) = -1
        .\end{align*}
    \end{mdframed}
    \bigbreak \noindent 
    We have
    \begin{align*}
        \frac{dy}{dx} = 2xy^{2} + 3x^{2}y^{2} = y^{2}(2x + 3x^{2})
    .\end{align*}
    So,
    \begin{align*}
        \frac{1}{y^{2}}\; dy &= 2x + 3x^{2}\; dx \\
        \implies \int \frac{1}{y^{2}}\; dy &= \int (2x + 3x^{2})\; dx \\
        \implies -\frac{1}{y} &= x^{2} + x^{3} + C
    .\end{align*}
    Using the initial condition $y(1) = -1 $, 
    \begin{align*}
        -\frac{1}{-1} &= 1^{2} + 1^{3} + C \implies 1 = 1+ 1 + C \implies C = -1
    .\end{align*}
    Thus,
    \begin{align*}
        -\frac{1}{y} &= x^{2} + x^{3} - 1
    .\end{align*}

    \pagebreak \bigbreak \noindent 
    \begin{mdframed}
        4. Solve the following differential equation
        \begin{align*}
            \frac{dy}{dx} = 1 + x + y + xy
        .\end{align*}
    \end{mdframed}
    \bigbreak \noindent 
    We have
    \begin{align*}
        \frac{dy}{dx} = 1 +x + y + xy = 1+x + y(1+x) = (1+x)(1+y)
    .\end{align*}
    So,
    \begin{align*}
        \frac{1}{1+y}\; dy &= 1+x\; dx \implies \int \frac{1}{1+y}\; dy = \int 1+x\; dx \\
        \implies \ln{\left\lvert 1+y \right\rvert} &= x + \frac{1}{2}x^{2} + C_{0} \\
        \implies \left\lvert 1+y \right\rvert &= e^{x+\frac{1}{2}x^{2}+C_{0}} = C_{1}e^{x+\frac{1}{2}x^{2}} \\
                                              \implies 1+y &= Ce^{x+\frac{1}{2}x^{2}} \\
                                              \implies y &= Ce^{x+\frac{1}{2}x^{2}} - 1
    .\end{align*}

    \pagebreak \bigbreak \noindent 
    \begin{mdframed}
        5. Solve the following differential equation
        \begin{align*}
            \frac{dy}{dx} = 6e^{2x-y}, \quad y(0) = 0
        .\end{align*}
    \end{mdframed}
    \bigbreak \noindent 
    We have
    \begin{align*}
        \frac{dy}{dx} &= 6e^{2x-y} = \frac{6e^{2x}}{e^{y}} \\
        \implies e^{y}\; dy &= 6e^{2x}\; dx \\
        \implies \int e^{y}\; dy &= \int 6e^{2x}\; dx \\
        \implies e^{y} &= 3e^{2x} + C
    .\end{align*}
    With $y(0) = 0$,
    \begin{align*}
        e^{0} &= 3e^{0} + C \implies 1 = 3 + C \implies  C = -2
    .\end{align*}
    Thus, 
    \begin{align*}
        e^{y} = 3e^{2x} - 2
    .\end{align*}



\end{document}
