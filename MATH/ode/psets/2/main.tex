 \documentclass{report}
 
 \input{~/dev/latex/template/preamble.tex}
 \input{~/dev/latex/template/macros.tex}
 
 \title{\Huge{}}
 \author{\huge{Nathan Warner}}
 \date{\huge{}}
 \fancyhf{}
 \rhead{}
 \fancyhead[R]{\itshape Warner} % Left header: Section name
 \fancyhead[L]{\itshape\leftmark}  % Right header: Page number
 \cfoot{\thepage}
 \renewcommand{\headrulewidth}{0pt} % Optional: Removes the header line
 %\pagestyle{fancy}
 %\fancyhf{}
 %\lhead{Warner \thepage}
 %\rhead{}
 % \lhead{\leftmark}
 %\cfoot{\thepage}
 %\setborder
 % \usepackage[default]{sourcecodepro}
 % \usepackage[T1]{fontenc}
 
 % Change the title
 \hypersetup{
     pdftitle={}
 }

 \geometry{
  left=1in,
  right=1in,
  top=1in,
  bottom=1in
}
 
 \begin{document}
     % \maketitle
     %     \begin{titlepage}
     %    \begin{center}
     %        \vspace*{1cm}
     % 
     %        \textbf{}
     % 
     %        \vspace{0.5cm}
     %         
     %             
     %        \vspace{1.5cm}
     % 
     %        \textbf{Nathan Warner}
     % 
     %        \vfill
     %             
     %             
     %        \vspace{0.8cm}
     %      
     %        \includegraphics[width=0.4\textwidth]{~/niu/seal.png}
     %             
     %        Computer Science \\
     %        Northern Illinois University\\
     %        United States\\
     %        
     %             
     %    \end{center}
     % \end{titlepage}
     % \tableofcontents
    \pagebreak \bigbreak \noindent
    Nate Warner \ \quad \quad \quad \quad \quad \quad \quad \quad \quad \quad \quad \quad  MATH 336 \quad  \quad \quad \quad \quad \quad \quad \quad \quad \ \ \quad \quad Spring 2026
    \begin{center}
        \textbf{Problem set 2 - Due: Monday, January 26}
    \end{center}
    \bigbreak \noindent 
    \begin{mdframed}
        1. Suppose a student comes into a class of 33 students and starts spreading a rumor about what will be on the first exam. Derive a differential equation whose solution gives the number of students \( S(t) \) who have heard the rumor at time \( t \), assuming that the rate at which the rumor spreads is proportional to the number of interactions between students who have heard the rumor and those who have not. (Recall that the number of interactions is given by the product of the two groups.) What is the initial condition \( S(0) \)?
    \end{mdframed}
    \bigbreak \noindent 
    Since the number of students who have heard the rumor is $S(t)$, the number of students who have not heard the rumor is therefore $33-S(t)$. Thus, the interaction is given by
    \begin{align*}
        S(t)(33-S(t))
    .\end{align*}
    The rate at which the rumor spreads with respect to time $t$ is $\frac{dS}{dt}$, which is proportional to the interaction. Thus,
    \begin{align*}
        \frac{dS}{dt} = kS(t)(33-S(t))
    .\end{align*}
    Note that $S(0) = 1$, which represents the student who started the rumor.

    \begin{mdframed}
        2. Newton’s law of cooling (or warming) states that the rate at which the temperature of an object changes is proportional to the difference between the temperature of the object and the temperature of the surrounding environment. A plate of ice cream is removed from a freezer where the temperature is $20^\circ\mathrm{F}$ and left to thaw on a table in a room where the temperature is $80^\circ\mathrm{F}$. Derive a differential equation whose solution gives the temperature $T(t)$ of the ice cream at time $t$. What is the initial condition $T(0)$?
    \end{mdframed}
    \bigbreak \noindent 
    The rate at which the temperature of an object changes is proportional to the difference between the temperature of the object and the temperature of the surrounding environment. Thus,
    \begin{align*}
        \frac{dT}{dt} = k(T - T_{a})
    ,\end{align*}
    where $T_{a}$ is the ambient temperature. More precisely, the rate of change is given by
    \begin{align*}
        \frac{dT}{dt} = -k(T-T_{a})
    .\end{align*}
    This is due to the fact that we require positive change (warming up) when $T < T_{a}$, and negative change (cooling down) when $T > T_{a} $. Since the ambient temperature is $T_{a} = 80$, the differential equation is
    \begin{align*}
        \frac{dT}{dt} = -k(T-80)
    .\end{align*}
    We note that the initial condition is $T(0) =20$, which represents the temperature of the object at the moment it was placed onto the ambient environment (table).

    \pagebreak \bigbreak \noindent 
    \begin{mdframed}
        3. Consider the differential equation 
        \begin{align*}
            \frac{dy}{dx} = 2-3xy 
        .\end{align*}
        Create a slope field at the indicated points.
    \end{mdframed}
    \bigbreak \noindent 
    \fig{1}{./figures/1.png}

    \begin{mdframed}
        4. Using the initial condition at $(10,0)$, sketch a potential solution to the differential equation $y = y - x - 3$
    \end{mdframed}
    \bigbreak \noindent 
    \fig{1}{./figures/3.png}



\end{document}
