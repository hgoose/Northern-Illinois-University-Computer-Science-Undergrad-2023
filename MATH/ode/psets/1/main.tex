 \documentclass{report}
 
 \input{~/dev/latex/template/preamble.tex}
 \input{~/dev/latex/template/macros.tex}
 
 \title{\Huge{}}
 \author{\huge{Nathan Warner}}
 \date{\huge{}}
 \fancyhf{}
 \rhead{}
 \fancyhead[R]{\itshape Warner} % Left header: Section name
 \fancyhead[L]{\itshape\leftmark}  % Right header: Page number
 \cfoot{\thepage}
 \renewcommand{\headrulewidth}{0pt} % Optional: Removes the header line
 %\pagestyle{fancy}
 %\fancyhf{}
 %\lhead{Warner \thepage}
 %\rhead{}
 % \lhead{\leftmark}
 %\cfoot{\thepage}
 %\setborder
 % \usepackage[default]{sourcecodepro}
 % \usepackage[T1]{fontenc}
 
 % Change the title
 \hypersetup{
     pdftitle={}
 }

 \geometry{
  left=1in,
  right=1in,
  top=1in,
  bottom=1in
}
 
 \begin{document}
     % \maketitle
     %     \begin{titlepage}
     %    \begin{center}
     %        \vspace*{1cm}
     % 
     %        \textbf{}
     % 
     %        \vspace{0.5cm}
     %         
     %             
     %        \vspace{1.5cm}
     % 
     %        \textbf{Nathan Warner}
     % 
     %        \vfill
     %             
     %             
     %        \vspace{0.8cm}
     %      
     %        \includegraphics[width=0.4\textwidth]{~/niu/seal.png}
     %             
     %        Computer Science \\
     %        Northern Illinois University\\
     %        United States\\
     %        
     %             
     %    \end{center}
     % \end{titlepage}
     % \tableofcontents
    \pagebreak \bigbreak \noindent
    Nate Warner \ \quad \quad \quad \quad \quad \quad \quad \quad \quad \quad \quad \quad  MATH 336 \quad  \quad \quad \quad \quad \quad \quad \quad \quad \ \ \quad \quad Spring 2026
    \begin{center}
        \textbf{Problem set 1 - Due: Friday, January 23}
    \end{center}
    \bigbreak \noindent 
    \begin{mdframed}
        1. Solve the following ODE
        \begin{align*}
            \frac{dy}{dx} = \frac{\cos{\left(\frac{1}{x}\right)}}{x^{3}}
        .\end{align*}
    \end{mdframed}
    \bigbreak \noindent 
    First, we see that 
    \begin{align*}
        \frac{dy}{dx} = \frac{\cos{\left(\frac{1}{x}\right)}}{x^{3}}
    \end{align*}
    implies that
    \begin{align*}
        dy = \frac{\cos{\left(\frac{1}{x}\right)}}{x^{3}} dx
    .\end{align*}
    So, we integrate both sides,
    \begin{align*}
        \int\; dy = \int \frac{\cos{\left(\frac{1}{x}\right)}}{x^{3}}\; dx
    .\end{align*}
    Thus,
    \begin{align*}
        y + C_1 = \int \frac{\cos{\left(\frac{1}{x}\right)}}{x^{3}}\; dx
    .\end{align*}
    To integrate the right hand side, we first let $u = \frac{1}{x}$. Then,
    \begin{align*}
        \frac{du}{dx} = \frac{d}{dx}\frac{1}{x}  = -\frac{1}{x^{2}}
    .\end{align*}
    So,
    \begin{align*}
        du = -\frac{1}{x^{2}}\; dx \implies -du = \frac{1}{x^{2}}\; dx
    .\end{align*}
    With this, we see that
    \begin{align*}
        \frac{1}{x^{3}}\; dx = \frac{1}{x} \cdot \frac{1}{x^{2}}\; dx = \frac{1}{x} \cdot (-du) = u(-du) = -u\; du
    .\end{align*}
    So,
    \begin{align*}
        \int \frac{\cos{\left(\frac{1}{x}\right)}}{x^{3}}\; dx = -\int u\cos{\left(u\right)}\; du
    .\end{align*}
    Now, we can integrate by parts. Let $\hat{u} = u$, and $dv = \cos{\left(u\right)} du $. Then,
    \begin{align*}
        \frac{d\hat{u}}{du} = 1 \implies d\hat{u} = du
    ,\end{align*}
    and
    \begin{align*}
        \int\; dv  &= v = \int \cos{\left(u\right)}\; du = \sin{\left(u\right)} \\
        \implies v &= \sin{\left(u\right)}
    .\end{align*}
    Now, since
    \begin{align*}
        \int \hat{u}\; dv = \hat{u}v - \int v\; d\hat{u} = \hat{u}v - \int v\; du
    ,\end{align*}
    we have
    \begin{align*}
        -\int u \cos{\left(u\right)}\; du &= - \left(u\sin{\left(u\right)} - \int \sin{\left(u\right)}\; du\right) \\
        &= - \left(u\sin{\left(u\right)} + \cos{\left(u\right)}\right) \\
        &= -u\sin{\left(u\right)} - \cos{\left(u\right)}
    .\end{align*}
    Therefore, 
    \begin{align*}
        y + C_1 &= \int \frac{\cos{\left(\frac{1}{x}\right)}}{x^{3}}\; dx = -\frac{1}{x}\sin{\left(\frac{1}{x}\right)} - \cos{\left(\frac{1}{x}\right)} \\
        \implies y &= -\frac{1}{x}\sin{\left(\frac{1}{x}\right)} - \cos{\left(\frac{1}{x}\right)} + C
    .\end{align*}


    \begin{mdframed}
        2. Solve the following ODE
        \begin{align*}
            \frac{dy}{dx} =  \frac{x}{\sqrt{x^{2}-7}}
        .\end{align*}
    \end{mdframed}
    \bigbreak \noindent 
    We have
    \begin{align*}
        dy = \frac{x}{\sqrt{x^{2}- 7}}\; dx \\
        \implies \int\; dy  = y = \int \frac{x}{\sqrt{x^{2} - 7}}\; dx
    .\end{align*}
    If we let $u= x^{2} - 7 $, then
    \begin{align*}
        \frac{du}{dx} = 2x \implies du = 2x\; dx \implies x\; dx = \frac{1}{2}\; du
    .\end{align*}
    Thus,
    \begin{align*}
        y = \frac{1}{2}\int \frac{1}{u^{\frac{1}{2}}}\; du
    .\end{align*}
    Now, since
    \begin{align*}
        \int u^{-\frac{1}{2}}\; du = \frac{u^{-\frac{1}{2} + 1}}{-\frac{1}{2} + 1} = 2u^{\frac{1}{2}} =2\sqrt{u}
    ,\end{align*}
    we have
    \begin{align*}
        y = \frac{1}{2}\left(2\sqrt{u}\right) = \sqrt{u} = \sqrt{x^{2} - 7 } + C 
    .\end{align*}

    \pagebreak \bigbreak \noindent 
    \begin{mdframed}
        3. Solve the following ODE
        \begin{align*}
            \frac{dy}{dx} = \frac{x}{\sqrt{1+x^{2}}}
        .\end{align*}
    \end{mdframed}
    \bigbreak \noindent 
    Similar to the last problem, 
    \begin{align*}
        dy = \frac{x}{\sqrt{1+x^{2}}}\; dx
    ,\end{align*}
    so
    \begin{align*}
        y = \int \frac{x}{\sqrt{1+x^{2}}}\; dx
    .\end{align*}
    Let $u=1+x^{2}$, so
    \begin{align*}
        du = 2x\; dx \implies x\; dx = \frac{1}{2}\; du
    .\end{align*}
    Thus,
    \begin{align*}
        y = \frac{1}{2}\int \frac{1}{\sqrt{u}}\; du
    .\end{align*}
    Which is exactly the same as the last problem. So, we know that
    \begin{align*}
        y = \sqrt{u} = \sqrt{1+x^{2}} + C 
    .\end{align*}


    \begin{mdframed}
        1. Solve the following Initial value problem
        \begin{align*}
            \begin{cases}
                \frac{dy}{dx} = x^{2}e^{x} \\
                y(0) = 2
            \end{cases}
        .\end{align*}
    \end{mdframed}
    \bigbreak \noindent 
    We have
    \begin{align*}
        dy = x^{2}e^{x}\; dx
    ,\end{align*}
    so
    \begin{align*}
        y = \int x^{2}e^{x}\; dx
    .\end{align*}
    If we let $u = x^{2}$, and $dv = e^{x}\; dx $, then
    \begin{align*}
        du = 2x\; dx
    ,\end{align*}
    and 
    \begin{align*}
        v = e^{x}
    .\end{align*}
    Thus, by integration by parts, we have
    \begin{align*}
        y = x^{2}e^{x} - \int 2xe^{x}\; dx
    .\end{align*}
    Now, we preform a second integration by parts for the integral $\int 2xe^{x}\; dx$. Let $u = 2x$, $dv = e^{x}\; dx $. Then,
    \begin{align*}
        du = 2\; dx
    ,\end{align*}
    and 
    \begin{align*}
        v = e^{x}
    .\end{align*}
    Thus,
    \begin{align*}
        y &= x^{2}e^{x} -\left(2xe^{x} - 2\int e^{x}\; dx\right) = x^{2}e^{x} - \left(2xe^{x} -2e^{x}\right) \\
        &= x^{2}e^{x} - 2xe^{x}+2e^{x} + C
    .\end{align*}
    With the initial value $y(0) = 2 $, 
    \begin{align*}
        2 = 0^{2}e^{0}-2(0)e^{0}+2e^{0} + C \implies 2 = 2 + C 
    .\end{align*}
    Thus, $C = 0 $, and
    \begin{align*}
        y         &= x^{2}e^{x} - 2xe^{x}+2e^{x} = e^{x}\left(x^{2}-2x + 2\right)
    .\end{align*}





\end{document}
