 \documentclass{report}
 
 \input{~/dev/latex/template/preamble.tex}
 \input{~/dev/latex/template/macros.tex}
 
 \title{\Huge{}}
 \author{\huge{Nathan Warner}}
 \date{\huge{}}
 \fancyhf{}
 \rhead{}
 \fancyhead[R]{\itshape Warner} % Left header: Section name
 \fancyhead[L]{\itshape\leftmark}  % Right header: Page number
 \cfoot{\thepage}
 \renewcommand{\headrulewidth}{0pt} % Optional: Removes the header line
 %\pagestyle{fancy}
 %\fancyhf{}
 %\lhead{Warner \thepage}
 %\rhead{}
 % \lhead{\leftmark}
 %\cfoot{\thepage}
 %\setborder
 % \usepackage[default]{sourcecodepro}
 % \usepackage[T1]{fontenc}
 
 % Change the title
 \hypersetup{
     pdftitle={}
 }

 \geometry{
  left=1in,
  right=1in,
  top=1in,
  bottom=1in
}
 
 \begin{document}
     % \maketitle
     %     \begin{titlepage}
     %    \begin{center}
     %        \vspace*{1cm}
     % 
     %        \textbf{}
     % 
     %        \vspace{0.5cm}
     %         
     %             
     %        \vspace{1.5cm}
     % 
     %        \textbf{Nathan Warner}
     % 
     %        \vfill
     %             
     %             
     %        \vspace{0.8cm}
     %      
     %        \includegraphics[width=0.4\textwidth]{~/niu/seal.png}
     %             
     %        Computer Science \\
     %        Northern Illinois University\\
     %        United States\\
     %        
     %             
     %    \end{center}
     % \end{titlepage}
     % \tableofcontents
    \pagebreak \bigbreak \noindent
    Nate Warner \ \quad \quad \quad \quad \quad \quad \quad \quad \quad \quad \quad \quad  MATH 336 \quad  \quad \quad \quad \quad \quad \quad \quad \quad \ \ \quad \quad Spring 2026
    \begin{center}
        \textbf{Problem set 4 - Due: Monday, February 9}
    \end{center}
    \bigbreak \noindent 
    \begin{mdframed}
        1. Solve the initial value problem
        \begin{align*}
            xy^{\prime} + 2y =3x, \quad y(1) = 5
        .\end{align*}
    \end{mdframed}
    \bigbreak \noindent 
    This is a first order linear differential equation, with standard form
    \begin{align*}
        y^{\prime} + \frac{2}{x}y = 3
    .\end{align*}
    So, let $P(x) = \frac{2}{x} $, $Q(x) = 3$, and define
    \begin{align*}
        \mu(x) = e^{\int P(x)\; dx} = e^{2\int\frac{1}{x}\; dx} = e^{2\ln{\left(x\right)}} = x^{2}
    .\end{align*}
    So,
    \begin{align*}
        y^{\prime} + \frac{2}{x}y &= 3 \implies x^{2}y^{\prime} + 2xy = 3x^{2} \\
        \implies (x^{2}y)^{\prime} &= 3x^{2} \implies x^{2}y = \int 3x^{2}\; dx \\
        \implies x^{2}y &= x^{3} + C \implies y = x + \frac{C}{x^{2}}
    .\end{align*}
    With the initial condition $y(1) = 5 $,
    \begin{align*}
        5 &= 1 + \frac{c}{1^{2}} \implies C = 4
    .\end{align*}
    Thus,
    \begin{align*}
        y(x) &= x + \frac{4}{x^{2}}
    .\end{align*}
    Where the initial of definition of this solution is $(0,\infty)$.

    \bigbreak \noindent 
    \begin{mdframed}
        2. Solve the initial value problem
        \begin{align*}
            xy^{\prime} - y = x, \quad y(1) = 7
        .\end{align*}
    \end{mdframed}
    \bigbreak \noindent 
    This is a linear first order differential equation with standard form
    \begin{align*}
        y^{\prime} - \frac{1}{x}y &= 1
    .\end{align*}
    Let $P(x) = -\frac{1}{x}$, $Q(x) = 1 $, and define
    \begin{align*}
        \mu(x) = e^{\int P(x)\; dx} = e^{- \int \frac{1}{x}\; dx} = e^{- \ln{\left(x\right)}} = \frac{1}{x}
    .\end{align*}
    So,
    \begin{align*}
        y^{\prime} - \frac{1}{x}y &= 1 \implies \frac{1}{x}y^{\prime} - \frac{1}{x^{2}}y = \frac{1}{x} \\
        \implies \left(\frac{1}{x}y\right)^{\prime} &= \frac{1}{x} \implies  \frac{1}{x}y = \int \frac{1}{x}\; dx \\
        \implies \frac{1}{x}y &= \ln{\left\lvert x \right\rvert} + C \implies y = x\ln{\left\lvert x \right\rvert} + Cx
    .\end{align*}
    With the initial condition $y(1) = 7 $,
    \begin{align*}
        7 &= (1)\ln{\left(1\right)} + C(1)  \implies C = 7
    .\end{align*}
    Thus,
    \begin{align*}
        y(x) &= x\ln{\left\lvert x \right\rvert} + 7x
    .\end{align*}
    Notice that the initial of definition is $(0,\infty)$, so
    \begin{align*}
        y(x) &= x\ln{\left(x\right)} + 7x
    \end{align*}
    since $x > 0$ for all $x$ in the interval.

    \bigbreak \noindent 
    \begin{mdframed}
        3. Solve the initial value problem
        \begin{align*}
            y^{\prime} + 2xy = x, \quad y(0) = -2
        .\end{align*}
    \end{mdframed}
    \bigbreak \noindent 
    This is a first order linear differential equation already in standard form. Let $P(x) = 2x$, $Q(x) = x$, and define
    \begin{align*}
        \mu(x) = e^{\int P(x)\; dx} = e^{2\int x\; dx} = e^{x^{2}}
    .\end{align*}
    Then,
    \begin{align*}
        y^{\prime} + 2xy &= x \implies e^{x^{2}}y^{\prime} + 2xe^{x^{2}}y = xe^{x^{2}} \\
        \implies (e^{x^{2}}y)^{\prime} &= xe^{x^{2}} \implies e^{x^{2}}y = \int xe^{x^{2}}\; dx
    .\end{align*}
    Regarding the integral $\int xe^{x^{2}}\; dx $, let $u = x^{2}$, then $du = 2x\; dx$, so $\frac{1}{2}\; du = x\; dx$. Thus,
    \begin{align*}
        \int xe^{x^{2}}\; dx = \frac{1}{2}\int e^{u}\; du = \frac{1}{2}e^{u} = \frac{1}{2}e^{x^{2}} + C
    .\end{align*}
    So,
    \begin{align*}
        e^{x^{2}}y &= \int xe^{x^{2}}\; dx \implies e^{x^{2}}y = \frac{1}{2}e^{x^{2}} \\
        \implies y &= \frac{1}{e^{x^{2}}}\left(\frac{1}{2}e^{x^{2}} + C\right) = \frac{1}{2} + \frac{C}{e^{x^{2}}}
    .\end{align*}
    With the initial condition $y(0) = -2$, 
    \begin{align*}
        -2 &= \frac{1}{2} + \frac{C}{e^{0}} \implies C = -2 - \frac{1}{2} = -\frac{5}{2}
    .\end{align*}
    Thus,
    \begin{align*}
        y(x) &= \frac{1}{2} - \frac{5}{2e^{x^{2}}} = \frac{1}{2}\left(1-\frac{5}{e^{x^{2}}}\right)
    \end{align*}
    with interval of definition $(-\infty, \infty)$.


    \pagebreak \bigbreak \noindent 
    \begin{mdframed}
        4. Solve the initial value problem
        \begin{align*}
            (1+x)y^{\prime} + y = \cos{\left(x\right)}, \quad y(0) = 1
        .\end{align*}
    \end{mdframed}
    \bigbreak \noindent 
    This is a first order linear differential equation with standard form
    \begin{align*}
        y^{\prime} + \frac{1}{1+x}y &= \frac{\cos{\left(x\right)}}{1+x} 
    .\end{align*}
    Let $P(x) = \frac{1}{1+x}$, $Q(x) = \frac{\cos{\left(x\right)}}{1+x}$, and define
    \begin{align*}
        \mu(x) = e^{\int P(x)\; dx} = e^{\int \frac{1}{1+x}\; dx} = e^{\ln{\left(1+x\right)}} = 1+x
    .\end{align*}
    Then, 
    \begin{align*}
        y^{\prime} + \frac{1}{1+x}y &= \frac{\cos{\left(x\right)}}{1+x}  \implies (1+x)y^{\prime} + y = \cos{\left(x\right)} \\
        \implies ((1+x)y)^{\prime} &= \cos{\left(x\right)} \implies (1+x)y = \int \cos{\left(x\right)}\; dx \\
        \implies (1+x)y &= \sin{\left(x\right)} + C \implies y = \frac{1}{1+x}\left(\sin{\left(x\right)} + C\right)
    .\end{align*}
    With the initial condition $y(0) = 1$, 
    \begin{align*}
        1 = \frac{1}{1+0}\left(\sin{\left(0\right)} + C\right) \implies C = 1
    .\end{align*}
    Thus,
    \begin{align*}
        y(x) = \frac{1}{1+x}\left(\sin{\left(x\right)} + 1\right)
    \end{align*}
    with interval of definition $(-1, \infty)$.


    \bigbreak \noindent 
    \begin{mdframed}
        5. A tank contains 1000 liters of a solution consisting of 100 kg of salt dissolved in water. Pure water is pumped into the tank at the rate of $5$ L/s, and the mixture, kept uniform by stirring, is pumped out at the same rate. How long will it be until only 10 kg of salt remains in the tank? 
    \end{mdframed}
    \bigbreak \noindent 
    Let $x(t)$ be the amount of salt (in kg) in the tank at time $t$, and $v(t)$ be the volume of liquid in the tank at time $t$, we have
    \begin{align*}
        v(t) = 1000 + (5-5)t  = 1000\; \text{L}
    .\end{align*}
    Now, the inflow of salt into the tank is given by
    \begin{align*}
        0\; \frac{\text{kg}}{\text{L}} \left(5\; \frac{\text{L}}{\text{s}}\right)   = 0\; \frac{\text{kg}}{\text{s}}
    .\end{align*}
    The outflow of salt water from the tank is given by
    \begin{align*}
        \frac{x(t)}{v(t)}\; \frac{\text{kg}}{\text{L}} \left(5\; \frac{\text{L}}{\text{s}}\right) = \frac{5x(t)}{1000}\; \frac{\text{kg}}{\text{s}}
    .\end{align*}
    Thus, the differential equation is 
    \begin{align*}
        \frac{dx}{dt} = 0 - \frac{5}{1000}x(t) = \frac{1}{200}x(t)
    .\end{align*}
    Observe that the standard from is
    \begin{align*}
        \frac{dx}{dt} + \frac{1}{200}x(t) &= 0
    .\end{align*}
    Notice that this is a first order linear differential equation. Let $P(t) = \frac{1}{200}$, $Q(t) = 0$, and define
    \begin{align*}
        \mu(t) &= e^{\int P(t)\; dt} = e^{\int \frac{1}{200}\; dt} = e^{\frac{1}{200}t}
    .\end{align*}
    Then, 
    \begin{align*}
        e^{\frac{1}{200}t}\frac{dx}{dt} + \frac{e^{\frac{1}{200}t}}{200} &= 0 \implies \left(e^{\frac{1}{200}t}x(t)\right)^{\prime} = 0 \\
        \implies e^{\frac{1}{200}t}x(t)&= \int 0\; dt \implies x(t) = \frac{C}{e^{\frac{1}{200}t}}
    .\end{align*}
    With the initial condition $x(0) = 100$,
    \begin{align*}
        100 = \frac{C}{e^{0}} \implies C = 100
    .\end{align*}
    Thus, 
    \begin{align*}
        x(t) = \frac{100}{e^{\frac{1}{200}}t}
    .\end{align*}
    Now we can find $x(10)$,
    \begin{align*}
        10 &= \frac{100}{e^{\frac{1}{200}}t} \implies e^{\frac{1}{200}t} = 10 \\
        \implies \frac{t}{200} &= \ln{\left(10\right)}
    .\end{align*}
    Thus, $t = 200\ln{\left(10\right)} $.

    \begin{mdframed}
        6. A tank initially contains 60 gal of pure water. Brine containing 1 lb of salt per gallon enters the tank at 2 gal/min, and the (perfectly mixed) solution leaves the tank at 3 gal/min; thus the tank is empty after exactly 1 h. 
        \begin{enumerate}[label=(\alph*)]
            \item Find the amount of salt in the tank after t minutes. 
            \item What is the maximum amount of salt ever in the tank?
        \end{enumerate}
    \end{mdframed}
    \bigbreak \noindent 
    Let $x(t)$ be the amount of salt in the tank at time $t$, and $v(t)$ be the volume of liquid in the tank at time $t$, we have
    \begin{align*}
        v(t) = 60 + (2-3)t = 60 - t\; \text{gal}
    .\end{align*}
    The inflow of salt into the tank is given by
    \begin{align*}
        1\; \frac{\text{lb}}{\text{gal}} \left(2\; \frac{\text{gal}}{\text{min}}\right) = 2\; \frac{\text{lbs}}{\text{min}}
    .\end{align*}
    The outflow of solution is given by
    \begin{align*}
        \frac{x(t)}{v(t)}\; \frac{\text{lbs}}{\text{gal}}\left(3\; \frac{\text{gal}}{\text{min}}\right) = \frac{3x(t)}{v(t)}\; \frac{\text{lbs}}{\text{min}}
    .\end{align*}
    Thus, the differential equation is
    \begin{align*}
        \frac{dx}{dt} &= 2 - \frac{3}{60-t}x(t) \implies \frac{dx}{dt} + \frac{3}{60-t}x(t) = 2
    .\end{align*}
    Let $P(t) = \frac{3}{60-t}$, $Q(t) = 2$, and define
    \begin{align*}
        \mu(t) &= e^{\int P(t)\; dt} = e^{3\int \frac{1}{60-t}\; dt} = e^{-3\ln{\left(60-t\right)}} = \frac{1}{(60-t)^{3}}
    .\end{align*}
    So,
    \begin{align*}
        \frac{dx}{dt} + \frac{3}{60-t}x(t) &= 2  = \frac{1}{(60-t)^{3}}\frac{dx}{dt} + \frac{3}{(60-t)^{4}}x(t) = \frac{2}{(60-t)^{3}} \\
        \implies \left(\frac{1}{(60-t)^{3}}x(t)\right)^{\prime} &= \frac{2}{(60-t)^{3}} \implies \frac{1}{(60-t)^{3}} x(t) = 2\int \frac{1}{(60-t)^{3}}\; dt
    .\end{align*}
    Regarding $ 2\int \frac{1}{(60-t)^{3}}\; dt$, let $u = 60-t$, then $-du = dt$. So,
    \begin{align*}
        2\int \frac{1}{(60-t)^{3}}\; dt = -2\int u^{-3}\; du = u^{-2} = \frac{1}{(60-t)^{2}} + C
    .\end{align*}
    Thus, 
    \begin{align*}
        x(t) = (60-t)^{3}\left(\frac{1}{(60-t)^{2}} + C\right) = (60-t) + C(60-t)^{3}
    .\end{align*}
    With the initial condition $y(0) =0$, 
    \begin{align*}
        0 = 60 + 60^{3}C \implies C = -\frac{60}{60^{3}} = -\frac{1}{60^{2}}
    .\end{align*}
    Thus,
    \begin{align*}
        x(t) &= (60-t) -\frac{1}{60^{2}}(60-t)^{3}
    .\end{align*}
    If we differentiate $x(t)$ with respect to $t$, we find
    \begin{align*}
        \frac{dx}{dt} &= -1 - \frac{1}{60^{2}}(3(60-t)^{2}(-1)) = \frac{3}{60^{2}}(60-t)^{2} - 1
    .\end{align*}
    Setting equal to zero gives
    \begin{align*}
        \frac{3}{60^{2}}(60-t)^{2} - 1 &= 0 \implies 60-t = \frac{60}{\sqrt{3}} \\
        \implies t = 60 - \frac{60}{\sqrt{3}} = 60-20\sqrt{3} \approx 25.4
    .\end{align*}
    Thus, the maximum amount of salt in the tank occurs at $t=25.4$ minutes.

    \pagebreak \bigbreak \noindent 
    \begin{mdframed}
        7. A 400-gal tank initially contains 100 gal of brine containing 50 lb of salt. Brine containing 1 lb of salt per gallon enters the tank at the rate of 5 gal/s, and the well mixed brine in the tank flows out at the rate of 3 gal/s. How much salt will the tank contain when it is full of brine?
    \end{mdframed}
    \bigbreak \noindent 
    Let $x(t)$ be the amount of salt at time $t$, and $v(t)$ be the amount of liquid at time $t$, we have
    \begin{align*}
        v(t) = 100 + (5 - 3)t = 100 + 2t = 2(50+t)
    .\end{align*}
    The inflow of salt into the tank is given by
    \begin{align*}
        1(5) = 5\; \frac{\text{lbs}}{\text{s}}
    .\end{align*}
    The outflow is given by
    \begin{align*}
        \frac{x(t)}{v(t)}\left(3\right) = \frac{3x(t)}{100+2t}\; \frac{\text{lbs}}{\text{s}}
    .\end{align*}
    Thus, the differential equation is given by
    \begin{align*}
        \frac{dx}{dt}  = 5 - \frac{3}{100+2t}x(t) \implies \frac{dx}{dt} + \frac{3}{100+2t}x(t) = 5
    .\end{align*}
    Let $P(t) = \frac{3}{100+2t}$, $Q(t) =  5$, and define
    \begin{align*}
        \mu(t) &= e^{\int P(t)\; dt} = e^{3\int \frac{1}{100+2t}\; dt} = e^{\frac{3}{2}\int \frac{1}{50+t}\; dt} = e^{\frac{3}{2}\ln{\left(50+t\right)}} = (50+t)^{\frac{3}{2}}
    .\end{align*}
    Then,
    \begin{align*}
        \frac{dx}{dt} + \frac{3}{100+2t}x(t) &= 5 \implies (50+t)^{\frac{3}{2}}\frac{dx}{dt} + \frac{3(50+t)^{\frac{3}{2}}}{2(50+t)} = 5(50+t)^{\frac{3}{2}} \\
        \implies \left((50+t)^{\frac{3}{2}}x(t)\right)^{\prime} &= 5(50+t)^{\frac{3}{2}} \implies (50+t)^{\frac{3}{2}}x(t) = \int 5(50+t)^{\frac{3}{2}}\; dt \\
        \implies x(t) &= \frac{1}{(50+t)^{\frac{3}{2}}}\left(2(50+t)^{\frac{5}{2}} + C\right) = 2(50+t) + \frac{C}{(50+t)^{\frac{3}{2}}}
    .\end{align*}
    With the initial condition $x(0) = 50 $, 
    \begin{align*}
        50 = 2(50) + \frac{C}{50^{\frac{3}{2}}} \implies C = -50(50^{\frac{3}{2}}) = -50^{\frac{5}{2}}
    .\end{align*}
    Thus,
    \begin{align*}
        x(t) &= 2(50+t) + \frac{-50^{\frac{5}{2}}}{(50+t)^{\frac{3}{2}}}
    .\end{align*}
    To find how much salt will be in the tank when the tank is full of brine, we observe that the tank is full of brine when $v(t) = 400$, so
    \begin{align*}
        2(50+t) = 400 \implies 50 + t = 200 \implies t = 150
    .\end{align*}
    Thus,
    \begin{align*}
        x(150) = 2(50 + 150) - \frac{50^{\frac{5}{2}}}{(50+150)^{\frac{3}{2}}} = 400 - \frac{50^{\frac{5}{2}}}{200^{\frac{3}{2}}} \approx 393.75\; \text{lbs}
    .\end{align*}



\end{document}
