\documentclass{report}

\input{~/dev/latex/template/preamble.tex}
\input{~/dev/latex/template/macros.tex}

\title{\Huge{}}
\author{\huge{Nathan Warner}}
\date{\huge{}}
\fancyhf{}
\rhead{}
\fancyhead[R]{\itshape Warner} % Left header: Section name
\fancyhead[L]{\itshape\leftmark}  % Right header: Page number
\cfoot{\thepage}
\renewcommand{\headrulewidth}{0pt} % Optional: Removes the header line
%\pagestyle{fancy}
%\fancyhf{}
%\lhead{Warner \thepage}
%\rhead{}
% \lhead{\leftmark}
%\cfoot{\thepage}
%\setborder
% \usepackage[default]{sourcecodepro}
% \usepackage[T1]{fontenc}

% Change the title
\hypersetup{
    pdftitle={Exam 1}
}

\begin{document}
    % \maketitle
        \begin{titlepage}
       \begin{center}
           \vspace*{1cm}
    
           \textbf{Exam 1}
    
           \vspace{0.5cm}
            
                
           \vspace{1.5cm}
    
           \textbf{Nathan Warner}
    
           \vfill
                
                
           \vspace{0.8cm}
         
           \includegraphics[width=0.4\textwidth]{~/niu/seal.png}
                
           Computer Science \\
           Northern Illinois University\\
           United States\\
           
                
       \end{center}
    \end{titlepage}
    \tableofcontents
    \pagebreak 
    \unsect{Axioms}
    \item         \textbf{Axiom of distance}: For all points $P,Q$
        \begin{enumerate}
            \item $PQ \geq 0 $
            \item $PQ = 0 \iff P=Q $
            \item $PQ = QP $
        \end{enumerate}
    \item         \textbf{Axioms of incidence}
        \begin{enumerate}
            \item There are at least two different lines
            \item Each line contains at least two different points
            \item Each pair of points are together in at least one line
            \item Each pair of points $P,Q$, with $PQ < \omega$ are together in at most one line
        \end{enumerate}
    \item \textbf{Betweenness of points axiom (Ax. BP)}: If $A,B,C$ are distinct, collinear points, and if $AB + BC \leq \omega$, then there exists a betweenness relation among $A,B,C$
        \bigbreak \noindent 
        What this is really saying is that if \textbf{any} of $AB + BC$, $BA + AC$, $AC + CB$ is $ \leq \omega$, then there is a betweenness relation.
        \bigbreak \noindent 
        \textbf{Note:} If Ax.BP is true for a plane $\mathbb{P}$, and if $AB + BC \leq \omega$ for distinct collinear $A,B,C$, then there is a betweenness relation, but not necessarily $ A\text{-}B\text{-}C $
        \bigbreak \noindent 
        When $\omega = \infty$, then for any distinct collinear $A,B,C$, $AB +BC  < \infty = \omega $, so there will be a betweenness relation
    \item \textbf{Quadrichotomy Axiom for Points (Ax.QP)}: If $A,B,C,X$ are distinct, collinear points, and if $ A\text{-}B\text{-}C$. Then, at least one of the following must hold
        \begin{align*}
            X\text{-}A\text{-}B, \quad A\text{-}X\text{-}B, \quad B\text{-}X\text{-}C, \quad \text{or } \quad B\text{-}C\text{-}X
        \end{align*}
        \bigbreak \noindent 
        Thus, Ax.QP says that whenever $ A\text{-}B\text{-}C$ (say on line $\ell$), then any other point $X$ on line $\ell$ is in either $ \overrightarrow{BA} $ or $ \overrightarrow{BC} $. That is,
        \begin{align*}
            \ell = \overrightarrow{BA} \cup \overrightarrow{BC}
        \end{align*}

    \item \textbf{Nontriviality Axiom (Ax.N)}: For any point $A$ on a line $\ell$ there exists a point $B$ on $\ell$ with $0 < AB < \omega$
        \bigbreak \noindent 
        This axiom is true for the planes in which $\omega = \infty$ ($\mathbb{E}$, $\mathbb{M}$, $\mathbb{H}$, $\mathbb{G}$, $\mathbb{R}^{3}$, $\hat{\mathbb{E}} $, ws)
        \bigbreak \noindent 
        This axiom is also true for $\mathbb{S}$ and Fano, where $\omega < \infty $


    \pagebreak 
    \unsect{Definitions}

    \pagebreak 
    \unsect{Theorems}
    \begin{itemize}
        \item \textbf{Theorem 6.1 (Symmetry of betweenness)}. For a general plane $\mathbb{P}$ with points, lines, distance, and satisfy the seven axioms, $A-B-C \iff C-B-A$
        \item \textbf{Theorem 6.2 (UMT)}: If $A-B-C$ then $B-A-C$ and $A-C-B$ are false.
        \item \textbf{Theorem 7.6}: For any point $A$ on a line $\ell$ there exists a point $C$ not on $\ell$ with $0 < AC <\omega$ 
    \item \textbf{Triangle inequality for the line}: If $A,B,C$ are any three distinct, collinear points, then 
        \begin{align*}
            AB + BC \geq AC 
        \end{align*}
    \item \textbf{Rule of insertion}: 
        \begin{itemize}
            \item If $ A\text{-}B\text{-}C$ and $ A\text{-}X\text{-}B$, then $ A\text{-}X\text{-}B\text{-}C $
            \item If $ A\text{-}B\text{-}C$ and $ B\text{-}X\text{-}C$, then $ A\text{-}B\text{-}X\text{-}C $
        \end{itemize}
    \end{itemize}

    \pagebreak 
    \unsect{Propositions}
    \begin{itemize}
        \item \textbf{Proposition 6.3}
            \begin{enumerate}[label=(\alph*)]
                \item $\overline{AB}$ lies in one line, the line $\overleftrightarrow{AB} $
                \item $\overline{AB} = \overline{BA} $
                \item If $x\in \overline{AB}$, with $X \ne B$, then $AX < AB $
            \end{enumerate}
        \item \textbf{Proposition 6.4}: Let $A,B,C,D$ be collinear points with $0 < AB < \omega$, $0< CD<\omega$, and $\overline{AB} = \overline{CD}$, then
        \begin{enumerate}[label=(\alph*)]
                \item Either $\{A,B\} = \{C,D\}$ or $\{A,B\} \cap \{C,D\} = \varnothing$
                \item $AB = CD$
            \end{enumerate}
        \item \textbf{Proposition 7.1}: If $A\text{-}B\text{-}C$ and $A\text{-}C\text{-}D$, then $A,B,C,D$ are distinct and collinear 
        \item \textbf{Proposition 7.2} If $A\text{-}B\text{-}C\text{-}D$, then $A,B,C,D$ are distinct and collinear, and $D\text{-}C\text{-}B\text{-}A $
        \item \textbf{Proposition 7.5}: If $X \ne Y$ are points distinct from $A$ or ray $\overrightarrow{AB}$, then at least one of $ A\text{-}X\text{-}Y$ or $ A\text{-}Y\text{-}X$ or $X,Y$ in $ \overline{AB}$ is true.
        \item \textbf{Important fact}:  Suppose $X$ is a point on a ray $\overrightarrow{AB}$ in a general plane.
            \begin{enumerate}
                \item If $ A\text{-}X\text{-}B$ then $AX < AB $
                \item If $ A\text{-}B\text{-}X$ then $AX > AB $
                \item IF $X = B$ then $AX = AB$
            \end{enumerate}
    \end{itemize}







    
\end{document}
