\documentclass{report}

\input{~/dev/latex/template/preamble.tex}
\input{~/dev/latex/template/macros.tex}

\title{\Huge{}}
\author{\huge{Nathan Warner}}
\date{\huge{}}
\fancyhf{}
\rhead{}
\fancyhead[R]{\itshape Warner} % Left header: Section name
\fancyhead[L]{\itshape\leftmark}  % Right header: Page number
\cfoot{\thepage}
\renewcommand{\headrulewidth}{0pt} % Optional: Removes the header line
%\pagestyle{fancy}
%\fancyhf{}
%\lhead{Warner \thepage}
%\rhead{}
% \lhead{\leftmark}
%\cfoot{\thepage}
%\setborder
% \usepackage[default]{sourcecodepro}
% \usepackage[T1]{fontenc}

% Change the title
\hypersetup{
    pdftitle={Exam 1}
}

\begin{document}
    % \maketitle
        \begin{titlepage}
       \begin{center}
           \vspace*{1cm}
    
           \textbf{Exam 1}
    
           \vspace{0.5cm}
            
                
           \vspace{1.5cm}
    
           \textbf{Nathan Warner}
    
           \vfill
                
                
           \vspace{0.8cm}
         
           \includegraphics[width=0.4\textwidth]{~/niu/seal.png}
                
           Computer Science \\
           Northern Illinois University\\
           United States\\
           
                
       \end{center}
    \end{titlepage}
    \tableofcontents
    \pagebreak 
    \unsect{Axioms}
    \item         \textbf{Axiom of distance}: For all points $P,Q$
        \begin{enumerate}
            \item $PQ \geq 0 $
            \item $PQ = 0 \iff P=Q $
            \item $PQ = QP $
        \end{enumerate}
    \item         \textbf{Axioms of incidence}
        \begin{enumerate}
            \item There are at least two different lines
            \item Each line contains at least two different points
            \item Each pair of points are together in at least one line
            \item Each pair of points $P,Q$, with $PQ < \omega$ are together in at most one line
        \end{enumerate}
    \item \textbf{Betweenness of points axiom (Ax. BP)}: If $A,B,C$ are distinct, collinear points, and if $AB + BC \leq \omega$, then there exists a betweenness relation among $A,B,C$
        \bigbreak \noindent 
        What this is really saying is that if \textbf{any} of $AB + BC$, $BA + AC$, $AC + CB$ is $ \leq \omega$, then there is a betweenness relation.
        \bigbreak \noindent 
        \textbf{Note:} If Ax.BP is true for a plane $\mathbb{P}$, and if $AB + BC \leq \omega$ for distinct collinear $A,B,C$, then there is a betweenness relation, but not necessarily $ A\text{-}B\text{-}C $
        \bigbreak \noindent 
        When $\omega = \infty$, then for any distinct collinear $A,B,C$, $AB +BC  < \infty = \omega $, so there will be a betweenness relation
    \item \textbf{Quadrichotomy Axiom for Points (Ax.QP)}: If $A,B,C,X$ are distinct, collinear points, and if $ A\text{-}B\text{-}C$. Then, at least one of the following must hold
        \begin{align*}
            X\text{-}A\text{-}B, \quad A\text{-}X\text{-}B, \quad B\text{-}X\text{-}C, \quad \text{or } \quad B\text{-}C\text{-}X
        \end{align*}
        \bigbreak \noindent 
        Thus, Ax.QP says that whenever $ A\text{-}B\text{-}C$ (say on line $\ell$), then any other point $X$ on line $\ell$ is in either $ \overrightarrow{BA} $ or $ \overrightarrow{BC} $. That is,
        \begin{align*}
            \ell = \overrightarrow{BA} \cup \overrightarrow{BC}
        \end{align*}

    \item \textbf{Nontriviality Axiom (Ax.N)}: For any point $A$ on a line $\ell$ there exists a point $B$ on $\ell$ with $0 < AB < \omega$
        \bigbreak \noindent 
        This axiom is true for the planes in which $\omega = \infty$ ($\mathbb{E}$, $\mathbb{M}$, $\mathbb{H}$, $\mathbb{G}$, $\mathbb{R}^{3}$, $\hat{\mathbb{E}} $, ws)
        \bigbreak \noindent 
        This axiom is also true for $\mathbb{S}$ and Fano, where $\omega < \infty $
     \item \textbf{Real ray Axiom (Ax.RR)}: For any ray $ \overrightarrow{AB}$, and for any real number $s $ with $0 \leq s \leq \omega$, there is a point $X$ in $\overrightarrow{AB}$ with $AX = s$



    \pagebreak 
    \unsect{Definitions}
    \begin{itemize}
        \item \textbf{Definition (Endpoints)}. Point $A$ is called an endpoint of ray $\overrightarrow{AB} $
        \item \textbf{Definition (Interior points and length for a segment):} Given a segment $ \overline{AB}$, $A$ and $B$ are called its endpoints. All other points of $\overline{AB}$ are called \textbf{Interior points} of $\overline{AB}$
            \bigbreak \noindent 
            Distance $AB$ is called the \textbf{length} of $\overline{AB} $
            \bigbreak \noindent 
            The interior of $\overline{AB}$, denoted $\text{Int}\overline{AB}$ or $\overline{AB}^{0}$, means the set of all interior points of $\overline{AB}$. That is, $\text{Int}\overline{AB} = \overline{AB}^{0} = \{X: A\text{-}X\text{-}B\}$
        \item \textbf{Definition}. Assume $\omega < \infty$. Let $A$ be a point on a line $m$. The unique point $A_{m}^{*}$ on $m$ such that $AA_{m}^{*} = \omega$ is called the \textbf{antipode} of $A$ on $m$. Thus,
            \begin{align*}
                \begin{cases}
                    A,A_{m}^{*} \text{ are on m, }  AA_{m}^{*} = \omega \\
                    \text{and } A\text{-}X\text{-}A_{m}^{*} \text{ for all other points $X$ on $m$}
                \end{cases}
            \end{align*}
        \item \textbf{Definition (interior points of a ray)}: Let \( h = \overrightarrow{AB} \) be a ray.  
            All points of \( h \) that are not endpoints of \( h \) are called \textit{interior points} of \( h \).  
            \bigbreak \noindent 
            The \textit{interior} of \( h \) is the set of all interior points of \( h \),  
            and is denoted by \( h^\circ \), \( \overline{AB}^\circ \), or \( \text{Int } \overrightarrow{AB} \).
        \item \textbf{Definition (Opposite rays)}: Two rays with the same endpoint whose union is a line are called \textbf{opposite rays}
        \item \textbf{Notation:} Denote the ray opposite to ray $h$ by $h^{\prime}$. So, $\overrightarrow{AB}^{\prime}$ means the ray opposite $\overrightarrow{AB} $

    \end{itemize}

    \pagebreak 
    \unsect{Theorems}
    \begin{itemize}
        \item \textbf{Theorem 6.1 (Symmetry of betweenness)}. For a general plane $\mathbb{P}$ with points, lines, distance, and satisfy the seven axioms, $A-B-C \iff C-B-A$
        \item \textbf{Theorem 6.2 (UMT)}: If $A-B-C$ then $B-A-C$ and $A-C-B$ are false.
        \item \textbf{Theorem 7.6}: For any point $A$ on a line $\ell$ there exists a point $C$ not on $\ell$ with $0 < AC <\omega$ 
    \item \textbf{Triangle inequality for the line}: If $A,B,C$ are any three distinct, collinear points, then 
        \begin{align*}
            AB + BC \geq AC 
        \end{align*}
    \item \textbf{Rule of insertion}: 
        \begin{itemize}
            \item If $ A\text{-}B\text{-}C$ and $ A\text{-}X\text{-}B$, then $ A\text{-}X\text{-}B\text{-}C $
            \item If $ A\text{-}B\text{-}C$ and $ B\text{-}X\text{-}C$, then $ A\text{-}B\text{-}X\text{-}C $
        \end{itemize}
        \item \textbf{Theorem 8.1}: If $\omega = \infty$, then $\mathbb{D} = [0,\infty$); if $\omega < \infty$, then $\mathbb{D} = [0,\omega] $
        \item \textbf{Theorem 8.2} Each segment, ray, and line has infinitely many points.
        \item \textbf{Theorem 8.3}. If $X \ne Y$ are points different from $A$ on ray $\overrightarrow{AB}$, then one of $ A\text{-}X\text{-}Y$ or $ A\text{-}Y\text{-}X$ is true.
        \item \textbf{Theorem 8.4}. If $C$ is any point on ray $ \overrightarrow{AB}$ with $ 0 < AC < \omega$, then $ \overrightarrow{AC} = \overrightarrow{AB} $
        \item \textbf{Theorem 8.6 (UDR)} For any ray $ \overrightarrow{AB}$ and any real number $s$ with $0 \leq s \leq \omega$, there is a \textbf{unique} point $X$ on $\overrightarrow{AB}$ with $AX = s$. $X$ is in $\overline{AB}$ if and only if $s \leq   AB $
        \item \textbf{Theorem 9.1 (Antipode on line theorem)}: Let $A$ be a point on a line $m$ (in a plane with the 11 axioms). Assume that $\omega < \infty$. Then, there exists a unique point $A^{*}_{m}$ on $m$ such that $AA_{m}^{*} = \omega$. Further, if $X$ is any other point on $m$, then $ A\text{-}X\text{-}A^{*}_{m} $
        \item \textbf{Theorem 9.2 (Almost-uniqueness for Quadrichotomy)}:  
            Suppose that \( A, B, C, X \) are distinct points on a line \( m \),  
            and that \( A - B - C \). Then \textbf{\textit{exactly one}} of the following holds:  
            \[
                X - A - B, \quad A - X - B, \quad B - X - C, \quad B - C - X
            \]
            with the \textbf{\textit{only exception}} that both \( X - A - B \) and \( B - C - X \) are true  
            when \( \omega < \infty \) and \( X = B_m^* \).
            \bigbreak \noindent 
            (Note that \( B_m^* - A - B \) and \( B - C - B_m^* \) \textbf{\textit{are both true}} by Thm. 9.1)
        \item \textbf{Theorem 9.4}.
            If \( h \) is a ray with two endpoints \( A \) and \( P \),  
            then \( \omega < \infty \) and \( P = A_m^* \), where \( m \) is the carrier of \( h \) (\( h \subseteq m \)).
        \item \textbf{Theorem 9.6 (Opposite ray theorem)}: If $ B\text{-}A\text{-}C$, then $\overrightarrow{AB}$ and $\overrightarrow{AC}$ are opposite rays
            \bigbreak \noindent 
            Also, for $m = \overleftrightarrow{AB}$
            \begin{align*}
                \overrightarrow{AB} \cap \overrightarrow{AC} = 
                \begin{cases}
                    \{A\}     & \text{ if } \omega = \infty \\
                    \{A, A_{m}^{*}\}     & \text{ if } \omega<\infty
                \end{cases}
            \end{align*}
        \item \textbf{Corollary 9.7}: Each ray has a unique opposite ray.
        \item \textbf{Corollary 9.8}: Let $A,B$ be points on line $m$ with $0 <AB<\omega <\infty$. Then $\overrightarrow{AB}^{\prime} = \overrightarrow{AB_{m}^{*}} $
        \item \textbf{Corollary 9.9}: Let $A,B$ be points on line $m$ with $ 0 < AB < \omega < \infty$. Then, $ m = \overline{AB} \cup \overline{BA_{m}^{*}} \cup \overline{A_{m}^{*}B_{m}^{*}} \cup \overline{B_{m}^{*}A}$, with the interiors of these segments being disjoint.
        \item \textbf{Theorem 9.10}: Let $A,B$ be points on line $m$ with $0 < AB < \omega < \infty$ . Let $C \ne A,B,A_{m}^{*}, B_{m}^{*} $ be another point on $m$. Then there is no betweenness relation for $A,B,C$ if and only if $C \in \overline{A_{m}^{*}B_{m}^{*}}^{0}$
        \item \textbf{Definition}. A subset $S$ of $\mathbb{P}$ is \textbf{convex} if for each pair of points $X \ne Y$ in $S$ with $XY < \omega$, $\overline{XY} \subseteq S$ holds.
        \item \textbf{Theorem 10.1}: If $S_{1}$ and $S_{2}$ are convex sets in $\mathbb{P}$, then so is $S_{1} \cap S_{2}$
        \item \textbf{Theorem 10.2}: Segments, rays, and lines are convex.
        \item \textbf{Definition}: A pair of sets $H,K$ in $\mathbb{P}$ is called \textbf{opposed around a line $m$} if 
            \begin{itemize}
                \item $H,K \ne \varnothing $
                \item $H,K$ are convex
                \item $H \cap K = \varnothing $
                \item $H \cup K = \mathbb{P} - m$
            \end{itemize}
        \item \textbf{Theorem 10.3} Let $H,K$ be sets opposed around a line $m$ in $\mathbb{P}$. Suppose that $A,C$ are points so that $C \in m$, $A \in H$, $AC < \omega$. Then, $\text{Int}\overrightarrow{CA} \subseteq H$, and $\text{Int}\overrightarrow{CA}^{\prime} \subseteq K $
        \item \textbf{Corollary 10.4}: let $H,K$ be sets opposed around a line $m$, let $A,B$ be points not on $m$, with $ A\text{-}X\text{-}B$ for some point $X \in m$. Then, $A,B$ lie one in each of $H$ and $K$, in some order.

    \end{itemize}

    \pagebreak 
    \unsect{Propositions}
    \begin{itemize}
        \item \textbf{Proposition 6.3}
            \begin{enumerate}[label=(\alph*)]
                \item $\overline{AB}$ lies in one line, the line $\overleftrightarrow{AB} $
                \item $\overline{AB} = \overline{BA} $
                \item If $x\in \overline{AB}$, with $X \ne B$, then $AX < AB $
            \end{enumerate}
        \item \textbf{Proposition 6.4}: Let $A,B,C,D$ be collinear points with $0 < AB < \omega$, $0< CD<\omega$, and $\overline{AB} = \overline{CD}$, then
        \begin{enumerate}[label=(\alph*)]
                \item Either $\{A,B\} = \{C,D\}$ or $\{A,B\} \cap \{C,D\} = \varnothing$
                \item $AB = CD$
            \end{enumerate}
        \item \textbf{Proposition 7.1}: If $A\text{-}B\text{-}C$ and $A\text{-}C\text{-}D$, then $A,B,C,D$ are distinct and collinear 
        \item \textbf{Proposition 7.2} If $A\text{-}B\text{-}C\text{-}D$, then $A,B,C,D$ are distinct and collinear, and $D\text{-}C\text{-}B\text{-}A $
        \item \textbf{Proposition 7.5}: If $X \ne Y$ are points distinct from $A$ or ray $\overrightarrow{AB}$, then at least one of $ A\text{-}X\text{-}Y$ or $ A\text{-}Y\text{-}X$ or $X,Y$ in $ \overline{AB}$ is true.
        \item \textbf{Important fact}:  Suppose $X$ is a point on a ray $\overrightarrow{AB}$ in a general plane.
            \begin{enumerate}
                \item If $ A\text{-}X\text{-}B$ then $AX < AB $
                \item If $ A\text{-}B\text{-}X$ then $AX > AB $
                \item IF $X = B$ then $AX = AB$
            \end{enumerate}
        \item \textbf{Proposition 8.11} Let $A,B$ be any two points on line $m$, with $0 < AB <\omega$. Then, there exists a point $C$ on $m$ with $ C\text{-}A\text{-}B$ and $ CB < \omega$.
        \item \textbf{Proposition 8.5}: A ray has at most two endpoints
        \item \textbf{Proposition 8.7}: Let $\overline{AB}$ be a segment and $X,Y \in \overline{AB}$. Then, $XY \leq AB$, and if $XY = AB$, then $\{X,Y\} = \{A,B\}$
        \item \textbf{Proposition 8.8} If $\overline{AB} = \overline{CD}$, then $\{A,B\}  = \{C,D\}$
        \item \textbf{Proposition 8.9}: In each segment $\overline{AB}$ there is a unique point $M$, called the \textbf{midpoint} of $\overline{AB} $, with the property that $AM = \frac{1}{2}AB$. Further, $AM = MB $
        \item \textbf{Proposition 9.3}: Assume \( \omega < \infty \). Let \( A, B \) be points on line \( m \)  
            with \( 0 < AB < \omega \). Then  
            \begin{enumerate}
                \item[(a)] \( \overrightarrow{AB} = \overline{AB} \cup \overline{BA_m^*} \) and \( \overline{AB}^{\circ} \cap \overline{BA_m^*}^{\circ} = \varnothing \).
                \item[(b)] \( \overrightarrow{AB} = \overrightarrow{A_m^* B} \), so that if \( A \) is an endpoint of a ray  
                    with carrier \( m \), then so is \( A_m^* \).
            \end{enumerate}
        \item \textbf{Proposition between} Let $\overrightarrow{AB}$ and $\overrightarrow{AC}$ be opposite rays, and points $X \in \text{Int}\overrightarrow{AB}$, $Y \in \text{Int}\overrightarrow{AC} $ with $AX + AY \leq \omega$, then $ X\text{-}A\text{-}Y$

    \end{itemize}







    
\end{document}
