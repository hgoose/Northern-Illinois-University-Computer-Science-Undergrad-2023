\documentclass{report}

\input{~/dev/latex/template/preamble.tex}
\input{~/dev/latex/template/macros.tex}

\title{\Huge{}}
\author{\huge{Nathan Warner}}
\date{\huge{}}
\fancyhf{}
\rhead{}
\fancyhead[R]{\itshape Warner} % Left header: Section name
\fancyhead[L]{\itshape\leftmark}  % Right header: Page number
\cfoot{\thepage}
\renewcommand{\headrulewidth}{0pt} % Optional: Removes the header line
%\pagestyle{fancy}
%\fancyhf{}
%\lhead{Warner \thepage}
%\rhead{}
% \lhead{\leftmark}
%\cfoot{\thepage}
%\setborder
% \usepackage[default]{sourcecodepro}
% \usepackage[T1]{fontenc}

% Change the title
\hypersetup{
    pdftitle={Exam 1}
}

\begin{document}
    % \maketitle
        \begin{titlepage}
       \begin{center}
           \vspace*{1cm}
    
           \textbf{Exam 1}
    
           \vspace{0.5cm}
            
                
           \vspace{1.5cm}
    
           \textbf{Nathan Warner}
    
           \vfill
                
                
           \vspace{0.8cm}
         
           \includegraphics[width=0.4\textwidth]{~/niu/seal.png}
                
           Computer Science \\
           Northern Illinois University\\
           United States\\
           
                
       \end{center}
    \end{titlepage}
    \tableofcontents
    \unsect{Part 1 (Study of points on lines, and distance)}
    \bigbreak \noindent 
    \subsection{Axioms}
    \item         \textbf{Axiom of distance}: For all points $P,Q$
        \begin{enumerate}
            \item $PQ \geq 0 $
            \item $PQ = 0 \iff P=Q $
            \item $PQ = QP $
        \end{enumerate}
    \item         \textbf{Axioms of incidence}
        \begin{enumerate}
            \item There are at least two different lines
            \item Each line contains at least two different points
            \item Each pair of points are together in at least one line
            \item Each pair of points $P,Q$, with $PQ < \omega$ are together in at most one line
        \end{enumerate}
    \item \textbf{Betweenness of points axiom (Ax. BP)}: If $A,B,C$ are distinct, collinear points, and if $AB + BC \leq \omega$, then there exists a betweenness relation among $A,B,C$
        \bigbreak \noindent 
        What this is really saying is that if \textbf{any} of $AB + BC$, $BA + AC$, $AC + CB$ is $ \leq \omega$, then there is a betweenness relation.
        \bigbreak \noindent 
        \textbf{Note:} If Ax.BP is true for a plane $\mathbb{P}$, and if $AB + BC \leq \omega$ for distinct collinear $A,B,C$, then there is a betweenness relation, but not necessarily $ A\text{-}B\text{-}C $
        \bigbreak \noindent 
        When $\omega = \infty$, then for any distinct collinear $A,B,C$, $AB +BC  < \infty = \omega $, so there will be a betweenness relation
    \item \textbf{Quadrichotomy Axiom for Points (Ax.QP)}: If $A,B,C,X$ are distinct, collinear points, and if $ A\text{-}B\text{-}C$. Then, at least one of the following must hold
        \begin{align*}
            X\text{-}A\text{-}B, \quad A\text{-}X\text{-}B, \quad B\text{-}X\text{-}C, \quad \text{or } \quad B\text{-}C\text{-}X
        \end{align*}
        \bigbreak \noindent 
        Thus, Ax.QP says that whenever $ A\text{-}B\text{-}C$ (say on line $\ell$), then any other point $X$ on line $\ell$ is in either $ \overrightarrow{BA} $ or $ \overrightarrow{BC} $. That is,
        \begin{align*}
            \ell = \overrightarrow{BA} \cup \overrightarrow{BC}
        \end{align*}

    \item \textbf{Nontriviality Axiom (Ax.N)}: For any point $A$ on a line $\ell$ there exists a point $B$ on $\ell$ with $0 < AB < \omega$
        \bigbreak \noindent 
        This axiom is true for the planes in which $\omega = \infty$ ($\mathbb{E}$, $\mathbb{M}$, $\mathbb{H}$, $\mathbb{G}$, $\mathbb{R}^{3}$, $\hat{\mathbb{E}} $, ws)
        \bigbreak \noindent 
        This axiom is also true for $\mathbb{S}$ and Fano, where $\omega < \infty $
     \item \textbf{Real ray Axiom (Ax.RR)}: For any ray $ \overrightarrow{AB}$, and for any real number $s $ with $0 \leq s \leq \omega$, there is a point $X$ in $\overrightarrow{AB}$ with $AX = s$
        \item \textbf{Separation Axiom Ax.S}: for each line $m$, there exists a pair of opposite halfplanes with edge $m$. 



    \pagebreak 
    \subsection{Definitions}
    \begin{itemize}
        \item \textbf{Definition (Endpoints)}. Point $A$ is called an endpoint of ray $\overrightarrow{AB} $
        \item \textbf{Definition (Interior points and length for a segment):} Given a segment $ \overline{AB}$, $A$ and $B$ are called its endpoints. All other points of $\overline{AB}$ are called \textbf{Interior points} of $\overline{AB}$
            \bigbreak \noindent 
            Distance $AB$ is called the \textbf{length} of $\overline{AB} $
            \bigbreak \noindent 
            The interior of $\overline{AB}$, denoted $\text{Int}\overline{AB}$ or $\overline{AB}^{0}$, means the set of all interior points of $\overline{AB}$. That is, $\text{Int}\overline{AB} = \overline{AB}^{0} = \{X: A\text{-}X\text{-}B\}$
        \item \textbf{Definition}. Assume $\omega < \infty$. Let $A$ be a point on a line $m$. The unique point $A_{m}^{*}$ on $m$ such that $AA_{m}^{*} = \omega$ is called the \textbf{antipode} of $A$ on $m$. Thus,
            \begin{align*}
                \begin{cases}
                    A,A_{m}^{*} \text{ are on m, }  AA_{m}^{*} = \omega \\
                    \text{and } A\text{-}X\text{-}A_{m}^{*} \text{ for all other points $X$ on $m$}
                \end{cases}
            \end{align*}
        \item \textbf{Definition (interior points of a ray)}: Let \( h = \overrightarrow{AB} \) be a ray.  
            All points of \( h \) that are not endpoints of \( h \) are called \textit{interior points} of \( h \).  
            \bigbreak \noindent 
            The \textit{interior} of \( h \) is the set of all interior points of \( h \),  
            and is denoted by \( h^\circ \), \( \overline{AB}^\circ \), or \( \text{Int } \overrightarrow{AB} \).
        \item \textbf{Definition (Opposite rays)}: Two rays with the same endpoint whose union is a line are called \textbf{opposite rays}
        \item \textbf{Notation:} Denote the ray opposite to ray $h$ by $h^{\prime}$. So, $\overrightarrow{AB}^{\prime}$ means the ray opposite $\overrightarrow{AB} $
        \item \textbf{Definition}: Let $H,K$ be opposite halfplanes with edge $m$. Two points in the same halfplane are said to be on the \textbf{same side} of $m$. 
        \item \textbf{Definition}: $A^{*} $ is called the \textbf{antipode} of $A$

    \end{itemize}

    \pagebreak 
    \subsection{Theorems}
    \begin{itemize}
        \item \textbf{Theorem 6.1 (Symmetry of betweenness)}. For a general plane $\mathbb{P}$ with points, lines, distance, and satisfy the seven axioms, $A-B-C \iff C-B-A$
        \item \textbf{Theorem 6.2 (UMT)}: If $A-B-C$ then $B-A-C$ and $A-C-B$ are false.
        \item \textbf{Theorem 7.6}: For any point $A$ on a line $\ell$ there exists a point $C$ not on $\ell$ with $0 < AC <\omega$ 
    \item \textbf{Triangle inequality for the line}: If $A,B,C$ are any three distinct, collinear points, then 
        \begin{align*}
            AB + BC \geq AC 
        \end{align*}
    \item \textbf{Rule of insertion}: 
        \begin{itemize}
            \item If $ A\text{-}B\text{-}C$ and $ A\text{-}X\text{-}B$, then $ A\text{-}X\text{-}B\text{-}C $
            \item If $ A\text{-}B\text{-}C$ and $ B\text{-}X\text{-}C$, then $ A\text{-}B\text{-}X\text{-}C $
        \end{itemize}
        \item \textbf{Theorem 8.1}: If $\omega = \infty$, then $\mathbb{D} = [0,\infty$); if $\omega < \infty$, then $\mathbb{D} = [0,\omega] $
        \item \textbf{Theorem 8.2} Each segment, ray, and line has infinitely many points.
        \item \textbf{Theorem 8.3}. If $X \ne Y$ are points different from $A$ on ray $\overrightarrow{AB}$, then one of $ A\text{-}X\text{-}Y$ or $ A\text{-}Y\text{-}X$ is true.
        \item \textbf{Theorem 8.4}. If $C$ is any point on ray $ \overrightarrow{AB}$ with $ 0 < AC < \omega$, then $ \overrightarrow{AC} = \overrightarrow{AB} $
        \item \textbf{Theorem 8.6 (UDR)} For any ray $ \overrightarrow{AB}$ and any real number $s$ with $0 \leq s \leq \omega$, there is a \textbf{unique} point $X$ on $\overrightarrow{AB}$ with $AX = s$. $X$ is in $\overline{AB}$ if and only if $s \leq   AB $
        \item \textbf{Theorem 9.1 (Antipode on line theorem)}: Let $A$ be a point on a line $m$ (in a plane with the 11 axioms). Assume that $\omega < \infty$. Then, there exists a unique point $A^{*}_{m}$ on $m$ such that $AA_{m}^{*} = \omega$. Further, if $X$ is any other point on $m$, then $ A\text{-}X\text{-}A^{*}_{m} $
        \item \textbf{Theorem 9.2 (Almost-uniqueness for Quadrichotomy)}:  
            Suppose that \( A, B, C, X \) are distinct points on a line \( m \),  
            and that \( A - B - C \). Then \textbf{\textit{exactly one}} of the following holds:  
            \[
                X - A - B, \quad A - X - B, \quad B - X - C, \quad B - C - X
            \]
            with the \textbf{\textit{only exception}} that both \( X - A - B \) and \( B - C - X \) are true  
            when \( \omega < \infty \) and \( X = B_m^* \).
            \bigbreak \noindent 
            (Note that \( B_m^* - A - B \) and \( B - C - B_m^* \) \textbf{\textit{are both true}} by Thm. 9.1)
        \item \textbf{Theorem 9.4}.
            If \( h \) is a ray with two endpoints \( A \) and \( P \),  
            then \( \omega < \infty \) and \( P = A_m^* \), where \( m \) is the carrier of \( h \) (\( h \subseteq m \)).
        \item \textbf{Theorem 9.6 (Opposite ray theorem)}: If $ B\text{-}A\text{-}C$, then $\overrightarrow{AB}$ and $\overrightarrow{AC}$ are opposite rays
            \bigbreak \noindent 
            Also, for $m = \overleftrightarrow{AB}$
            \begin{align*}
                \overrightarrow{AB} \cap \overrightarrow{AC} = 
                \begin{cases}
                    \{A\}     & \text{ if } \omega = \infty \\
                    \{A, A_{m}^{*}\}     & \text{ if } \omega<\infty
                \end{cases}
            \end{align*}
        \item \textbf{Corollary 9.7}: Each ray has a unique opposite ray.
        \item \textbf{Corollary 9.8}: Let $A,B$ be points on line $m$ with $0 <AB<\omega <\infty$. Then $\overrightarrow{AB}^{\prime} = \overrightarrow{AB_{m}^{*}} $
        \item \textbf{Corollary 9.9}: Let $A,B$ be points on line $m$ with $ 0 < AB < \omega < \infty$. Then, $ m = \overline{AB} \cup \overline{BA_{m}^{*}} \cup \overline{A_{m}^{*}B_{m}^{*}} \cup \overline{B_{m}^{*}A}$, with the interiors of these segments being disjoint.
        \item \textbf{Theorem 9.10}: Let $A,B$ be points on line $m$ with $0 < AB < \omega < \infty$ . Let $C \ne A,B,A_{m}^{*}, B_{m}^{*} $ be another point on $m$. Then there is no betweenness relation for $A,B,C$ if and only if $C \in \overline{A_{m}^{*}B_{m}^{*}}^{0}$
        \item \textbf{Definition}. A subset $S$ of $\mathbb{P}$ is \textbf{convex} if for each pair of points $X \ne Y$ in $S$ with $XY < \omega$, $\overline{XY} \subseteq S$ holds.
        \item \textbf{Theorem 10.1}: If $S_{1}$ and $S_{2}$ are convex sets in $\mathbb{P}$, then so is $S_{1} \cap S_{2}$
        \item \textbf{Theorem 10.2}: Segments, rays, and lines are convex.
        \item \textbf{Definition}: A pair of sets $H,K$ in $\mathbb{P}$ is called \textbf{opposed around a line $m$} if 
            \begin{itemize}
                \item $H,K \ne \varnothing $
                \item $H,K$ are convex
                \item $H \cap K = \varnothing $
                \item $H \cup K = \mathbb{P} - m$
            \end{itemize}
        \item \textbf{Theorem 10.3} Let $H,K$ be sets opposed around a line $m$ in $\mathbb{P}$. Suppose that $A,C$ are points so that $C \in m$, $A \in H$, $AC < \omega$. Then, $\text{Int}\overrightarrow{CA} \subseteq H$, and $\text{Int}\overrightarrow{CA}^{\prime} \subseteq K $
        \item \textbf{Corollary 10.4}: let $H,K$ be sets opposed around a line $m$, let $A,B$ be points not on $m$, with $ A\text{-}X\text{-}B$ for some point $X \in m$. Then, $A,B$ lie one in each of $H$ and $K$, in some order.
        \item \textbf{Definition}: Let $m$ be a line. Sets $H,K$ are called \textbf{opposite halfplanes with edge $m$} if:
            \bigbreak \noindent 
            \begin{align*}
                &H,K \text{ are opposed around $m$, and whenever } X \in H, Y \in K \text{ and } XY < \omega, \\ &\text{ then, } \overline{XY} \cap m \ne \varnothing
            \end{align*}
        \item \textbf{Theorem 10.5}: Suppose that $m$ is a line  so that there exists a pair $H,K$ of opposite half planes with edge $m$. Suppose also that $\omega < \infty$ and $A$ is a point on $m$. If $B$ is any point in $\mathbb{P}$ with $AB = \omega$, then $B \in m$ (so $B = A_{m}^{*}$, and there is only one point $B$ in all of $\mathbb{P}$ with $AB = \omega$)
            \bigbreak \noindent 
            In other words, let $H,K$ be opposite halfplanes with edge a line $m$, let $A \in m$, $\omega < \infty$. If $B \in \mathbb{P}$, $AB = \omega$, then $B \in m$, and $B$ unique in $\mathbb{P}$
        \item \textbf{Theorem 10.6}: Suppose that there is a pair $H,K$ of opposite halfplanes with edge $m$. Let $A \ne B$ be points not on $m$. Then, 
            \begin{align*}
                A,B \text{ lie one in each of $H,K$ } \iff \text{ there is a point $X$ on $m$ such that $ A\text{-}X\text{-}B $}
            \end{align*}
        \item \textbf{Corollary 10.7 (Needs proof)}: Suppose that there is a pair $H,K$ of opposite halfplanes with edge a line $m$. Then, $H,K$ is the only pair of sets opposed around $m$.
        \item \textbf{Theorem 10.8}: Suppose that $\omega < \infty$. For each point $ A$, there is exactly one point $A^{*}$ in $\mathbb{P}$ with $AA^{*} = \omega$. Also, every line through $A$ goes through $A^{*}$ as well.
        \item \textbf{Corollary 10.9}: Suppose that $\omega < \infty$. For any line $m$ and point $P$, there are just two possibilities:
            \begin{align*}
               \begin{cases}
                   P,P^{*} &\text{ both on $m$}     \\
                   P, P^{*} &\text{on opposite sides of $m$}
               \end{cases}
            \end{align*}

        \item \textbf{Theorem 10.10 (Pasch's Axioms) (needs proof)}: Let $A,B,C$ be three noncollinear points. Let $X$ be a point with $ B\text{-}X\text{-}C $, and $m$ a line through $X$ but not through $A,B,$ or $C$. Then, exactly one of
            \begin{enumerate}
                \item $m$ contains a point $Y$ with $ A\text{-}Y\text{-}C$
                \item $m$ contains a point $Z$ with $ A\text{-}Z\text{-}B $
            \end{enumerate}
        \item \textbf{Theorem 10.11}: Assume that $\omega < \infty$. Then, any two distinct lines must have a point (in fact, a pair of antipodes) in common.


    \end{itemize}

    \pagebreak 
    \subsection{Propositions}
    \begin{itemize}
        \item \textbf{Proposition 6.3}
            \begin{enumerate}[label=(\alph*)]
                \item $\overline{AB}$ lies in one line, the line $\overleftrightarrow{AB} $
                \item $\overline{AB} = \overline{BA} $
                \item If $x\in \overline{AB}$, with $X \ne B$, then $AX < AB $
            \end{enumerate}
        \item \textbf{Proposition 6.4}: Let $A,B,C,D$ be collinear points with $0 < AB < \omega$, $0< CD<\omega$, and $\overline{AB} = \overline{CD}$, then
        \begin{enumerate}[label=(\alph*)]
                \item Either $\{A,B\} = \{C,D\}$ or $\{A,B\} \cap \{C,D\} = \varnothing$
                \item $AB = CD$
            \end{enumerate}
        \item \textbf{Proposition 7.1}: If $A\text{-}B\text{-}C$ and $A\text{-}C\text{-}D$, then $A,B,C,D$ are distinct and collinear 
        \item \textbf{Proposition 7.2} If $A\text{-}B\text{-}C\text{-}D$, then $A,B,C,D$ are distinct and collinear, and $D\text{-}C\text{-}B\text{-}A $
        \item \textbf{Proposition 7.5}: If $X \ne Y$ are points distinct from $A$ or ray $\overrightarrow{AB}$, then at least one of $ A\text{-}X\text{-}Y$ or $ A\text{-}Y\text{-}X$ or $X,Y$ in $ \overline{AB}$ is true.
        \item \textbf{Important fact}:  Suppose $X$ is a point on a ray $\overrightarrow{AB}$ in a general plane.
            \begin{enumerate}
                \item If $ A\text{-}X\text{-}B$ then $AX < AB $
                \item If $ A\text{-}B\text{-}X$ then $AX > AB $
                \item IF $X = B$ then $AX = AB$
            \end{enumerate}
        \item \textbf{Proposition 8.11} Let $A,B$ be any two points on line $m$, with $0 < AB <\omega$. Then, there exists a point $C$ on $m$ with $ C\text{-}A\text{-}B$ and $ CB < \omega$.
        \item \textbf{Proposition 8.5}: A ray has at most two endpoints
        \item \textbf{Proposition 8.7}: Let $\overline{AB}$ be a segment and $X,Y \in \overline{AB}$. Then, $XY \leq AB$, and if $XY = AB$, then $\{X,Y\} = \{A,B\}$
        \item \textbf{Proposition 8.8} If $\overline{AB} = \overline{CD}$, then $\{A,B\}  = \{C,D\}$
        \item \textbf{Proposition 8.9}: In each segment $\overline{AB}$ there is a unique point $M$, called the \textbf{midpoint} of $\overline{AB} $, with the property that $AM = \frac{1}{2}AB$. Further, $AM = MB $
        \item \textbf{Proposition 9.3}: Assume \( \omega < \infty \). Let \( A, B \) be points on line \( m \)  
            with \( 0 < AB < \omega \). Then  
            \begin{enumerate}
                \item[(a)] \( \overrightarrow{AB} = \overline{AB} \cup \overline{BA_m^*} \) and \( \overline{AB}^{\circ} \cap \overline{BA_m^*}^{\circ} = \varnothing \).
                \item[(b)] \( \overrightarrow{AB} = \overrightarrow{A_m^* B} \), so that if \( A \) is an endpoint of a ray  
                    with carrier \( m \), then so is \( A_m^* \).
            \end{enumerate}
        \item \textbf{Proposition between} Let $\overrightarrow{AB}$ and $\overrightarrow{AC}$ be opposite rays, and points $X \in \text{Int}\overrightarrow{AB}$, $Y \in \text{Int}\overrightarrow{AC} $ with $AX + AY \leq \omega$, then $ X\text{-}A\text{-}Y$
        \item \textbf{Proposition Noncollinear}: If $A,B,C$ are three noncollinear points (not all on the same line), then $AB, AC,BC$ all less than $\omega$.

    \end{itemize}

    \pagebreak 
    \unsect{Part 2 (Study of rays in pencils, angles, angle measures, triangles)}
    \bigbreak \noindent 
    \subsection{Axioms}
    \begin{itemize}
        \item \textbf{Measure axioms}:
            \begin{enumerate}
                \item [M1]: For all coterminal rays $p,q$, $0 \leq pq \leq 180$
                \item [M2]: $pq = 0 \iff p=q$
                \item [M3]: $pq = qp$
                \item [M4]: $pq = 180 \iff q=p^{\prime} $
            \end{enumerate}
                \item \textbf{Betweenness of rays axiom (Ax.BR)}: If $a,b,c$ are distinct, coterminal rays, and if $ab+bc \leq 180$, then there exists a betweenness relation among $a,b,c$
                    \bigbreak \noindent 
                    Thus, if no betweenness relation exists, then
                    \begin{align*}
                        ab + bc > 180 \\
                        ac + cb > 180 \\
                        ba + ac > 180
                    \end{align*}
                \item \textbf{Quadrichotomy of Rays Axiom (Ax.QR)}: If $a,b,c,x$ are distinct, coterminal rays, and if $ a\text{-}b\text{-}c$, then at least one of the following must hold
                    \begin{align*}
                        x\text{-}a\text{-}b \quad a\text{-}x\text{-}b \quad b\text{-}x\text{-}c \quad b\text{-}c\text{-}x
                    \end{align*}
                    \bigbreak \noindent 
                    So, Ax.QR says that whenever $ a\text{-}b\text{-}c$ (say in pencil $P$), then any other ray in $P$ is in either fan $\overrightarrow{ba}$ or fan $\overrightarrow{bc} $ (so $P = \overrightarrow{ba} \cup \overrightarrow{bc} $)
                \item \textbf{Real fan axiom (Ax.RF)}: For any fan $\overrightarrow{ab} $ and for any real number $t$ with $ 0 \leq t \leq 180$, there is a ray $r$ in $\overrightarrow{ab} $ with $ar = t $
                    \bigbreak \noindent 
                    Ax.RF says every real number from 0 to 180 produces at least one ray in the fan
                    \bigbreak \noindent 
                    \textbf{Note:} Ax.RF is one version of what is sometimes called the \textbf{Protractor Axiom}
                \item \textbf{Compatibility Axiom (Ax.C)}: Let $A,B,C$ be points on line $m$, and $X$ a point not on $m$. If $ A\text{-}B\text{-}C$, then $ \overrightarrow{XA}\text{-}\overrightarrow{XB}\text{-}\overrightarrow{XC} $
        \item \textbf{Side-angle-side axiom (Ax.SAS)}: If under the correspondence $ABC \leftrightarrow XYZ$ between the vertices of $ \triangle ABC$ and those of $ \triangle XYZ$, two sides of $ \triangle ABC$ are congruent to the corresponding two sides of $\triangle XYZ$, and the angle included between these two sides of $ \triangle ABC$ is congruent to the corresponding angle of $\triangle XYZ$, then $\triangle ABC \cong \triangle XYZ$

    \end{itemize}

    \pagebreak 
    \subsection{Definitions}
    \begin{itemize}
        \item \textbf{Definition: \textit{Coterminal rays}}: Rays with the same endpoint
        \item \textbf{Definition: \textit{Angle}}: $\underline{ab} = a \cup b $, where $a,b$ are coterminal rays
        \item \textbf{Definition: \textit{Pencil of rays at point $A$}}: The set of all rays with endpoint $A$: denote by $P_{A}$ or just $P$
            \bigbreak \noindent 
            When $\omega < \infty$, each ray $h = \overrightarrow{AB} = \overrightarrow{A^{*}B}$, so $P_{A} = P_{A^{*}} $. $h^{\prime} $ is the opposite ray to $h$, as before
        \item \textbf{Undefined Term \textit{Angle distance function, or angle measure}}: A function $\mu$ from all pairs $(p,q) $ of coterminal rays to $\mathbb{R}$
            \bigbreak \noindent 
            We abbreviate the angular distance between rays $p,q$, or the angle measure of the angle $pq$, $\mu(p,q)$ as $pq$ 
        \item \textbf{Angular distance in $\mathbb{E}$, $\hat{\mathbb{E}}$, $\mathbb{M} $}: The usual measure in degrees (0 to 180)
            \begin{align*}
                pq = \cos^{-1}{\left(\frac{1+mn}{\sqrt{1+m^{2}}\sqrt{1+n^{2}}}\right)}
            \end{align*}
        \item \textbf{Angular distance in $\mathbb{H}$}:
            \begin{align*}
                \mu_{\mathbb{H}}(p,q) = \cos^{-1}{\left(\frac{1+mn-bc}{\sqrt{1+m^{2}-b^{2}}\sqrt{1+n^{2}-c^{2}}}\right)} 
            \end{align*}
        \item \textbf{Definition \textit{(betweenness for rays)}}: Ray $b$ lies \textbf{between} rays $a$ and $c$ ($ a\text{-}b\text{-}c $) provided that
            \begin{enumerate}[label=(\alph*)]
                \item $a,b,c$ are different, coterminal
                \item $ab + bc = ac $
            \end{enumerate}

        \item \textbf{Definition \textit{(Wedge, fan)}}: Let $p,q$ be coterminal rays with $0<pq<180$.
            \begin{itemize}
                \item \textbf{Wedge $\overline{pq} = \{p,q\} \cup \{r: p\text{-}r\text{-}q\}$}
                \item \textbf{Fan $\overrightarrow{pq} = \{p,q\} \cup \{r: p\text{-}r\text{-}q\} \cup \{r: p\text{-}q\text{-}r\}$}
            \end{itemize}
        \item \textbf{Definition \textit{(quad betweenness)}}: $ a\text{-}b\text{-}c\text{-}d $ means that all four of 
            \begin{align*}
                a\text{-}b\text{-}c \quad a\text{-}b\text{-}d \quad a\text{-}c\text{-}d \quad b\text{-}c\text{-}d
            \end{align*}
            are true
        \item \textbf{Notation and terminology}: Recall that $\hcancel{pq}$ means $p \cup q$, then union of the rays. Measure of $\hcancel{pq} $ means the angular distance $pq$
            \bigbreak \noindent 
            Suppose $p = \overrightarrow{BA}$, $ q = \overrightarrow{BC}$. Then, write
            \begin{align*}
                \hcancel{pq} = \underline{\angle ABC} = \underline{\angle CBA}
            \end{align*}
            Or just $\underline{\angle B}$ when clear, and
            \begin{align*}
                pq = \angle ABC = \angle CBA
            \end{align*}
            or just $\angle B$.
        \item \textbf{Definition}: 
            \begin{itemize}
                \item \textbf{Zero angle:} $\underline{pq}$ is a \textbf{zero angle} if $pq = 0 $ ($\iff p = q$)
                \item \textbf{Straight angle}: If $pq = 180 (\iff p = q^{\prime}) $
                \item \textbf{Proper angle:} if $0 < pq < 180 $
                \item \textbf{acute angle}: if $ 0 < pq < 90$
                \item \textbf{right angle}: if $ pq = 90$
                \item \textbf{obtuse angle}: if $ 90 < pq < 180$
            \end{itemize}
        \item \textbf{Definition}: The ray $b$ from the midpoint proposition is called the \textbf{bisector} of angle $\underline{pq}$ 
        \item \textbf{Definition: Congruence}: Two segments $\overline{AB}$ and $ \overline{XY}$ are \textbf{congruent} $(\cong)$ if they have the same length: $\overline{AB} \cong \overline{XY} $ means $AB = XY$
            \bigbreak \noindent 
            Two angles $\angle CAB$ and $\angle ZXY$ are congruent if they have the same angle measure
            \bigbreak \noindent 
            Two triangles $\triangle ABC$ and $\triangle XYZ$ are congruent under the correspondence $A\leftrightarrow X$, $B \leftrightarrow Y, C\leftrightarrow Z$ (Write as $ABC \leftrightarrow XYZ $) if 
            \begin{align*}
                \overline{AB} \cong \overline{XY},\quad \overline{BC} \cong\overline{YZ} ,\quad \overline{AC} \cong \overline{XZ}
            .\end{align*}
            and 
            \begin{align*}
                \angle ABC \cong \angle XYZ, \quad \angle CAB \cong \angle ZXY,\quad \angle BCA \cong \angle YZX
            .\end{align*}
            denote this by $\triangle ABC \cong \triangle XYZ $
        \item \textbf{Definition: Absolute plane}: An \textbf{absolute plane} $\mathbb{P}$ is a set of points $\mathbb{P}$ with lines, distance, and angular distance (all undefined terms), such that all 21 axioms are true. The three planes above are absolute planes
        \item \textbf{Definition: types of triangles}
            \begin{itemize}
                \item A triangle is \textbf{isosceles} if two sides have the same length
                \item \textbf{Equilateral} if all three sides have the same length
                \item \textbf{Equiangular} if all three angles have the same measure  
            \end{itemize}
            \textbf{Note:} A triangle can be called \textbf{scalene} if all all three sides have different lengths and all three angles have different measures



    \end{itemize}

    \pagebreak 
    \subsection{Theorems}
    \begin{itemize}
        \item \textbf{Theorem 11.1 \textit{(symmetry of betweenness)}}: $ a\text{-}b\text{-}c \iff c\text{-}b\text{-}a$
        \item \textbf{Theorem 11.3 \textit{UMT}}: If $ a\text{-}b\text{-}c$, then $ b\text{-}a\text{-}c$ and $ a\text{-}c\text{-}b$ are false.
        \item \textbf{Theorem 11.2 (non-triviality)}: For any ray $p$ there is a coterminal ray $q$ so that $0 < pq < 180$
        \item \textbf{Theorem \textit{(Triangle inequality for rays)}}: If $a,b,c$ are three distinct, coterminal rays, then $ab + bc \geq ac$
        \item \textbf{Theorem 11.5 \textit{(Rule of insertion for rays)}}:
            \begin{enumerate}[label=(\alph*)]
                \item If $ a\text{-}b\text{-}c$ and $ a\text{-}r\text{-}b$, then $ a\text{-}r\text{-}b\text{-}c $
                \item If $ a\text{-}b\text{-}c $ and $ b\text{-}r\text{-}c $, then $ a\text{-}b\text{-}r\text{-}c $
            \end{enumerate}
        \item \textbf{Theorem 11.6 (Unique angular distance for fans)}: For any fan $\overrightarrow{pq}$ and any real number $t$ with $0 \leq t \leq 180$, there is a unique ray $r$ in $\overrightarrow{pq}$ with $pr = t$. $r$ is in $\overline{pq} $ if and only if $t \leq  pq$
        \item \textbf{Theorem 11.8}: If ray $a$ lies in pencil $P$, then $ a\text{-}r\text{-}a^{\prime} $ for every other ray $r$ in $P$
        \item \textbf{Theorem 11.9 (Almost uniqueness of quadrichotomy for rays)}: Suppose that $a,b,c,r$ are distinct rays in a pencil $P$, and that $ a\text{-}b\text{-}c$. Then, \textbf{exactly} one of 
            \begin{align*}
                r\text{-}a\text{-}b \quad a\text{-}r\text{-}b \quad b\text{-}r\text{-}c \quad b\text{-}c\text{-}r
            \end{align*}
            With the exception that both $ r\text{-}a\text{-}b $ and $ b\text{-}c\text{-}r$ are true when $r = b^{\prime} $
        \item \textbf{Theorem 11.10 (Opposite fan theorem)}: Let $p,q,r$ be rays in pencil $P$ such that $ q\text{-}p\text{-}r$. Then, $ \overrightarrow{pq} \cup \overrightarrow{pr} = P$, and $ \overrightarrow{pq} \cap \overrightarrow{pr} = \{p,p^{\prime}\} $
        \item \textbf{Corollary 11.11}: If $p,q$ are rays in pencil $P$ with $0 < pq < 180$, then $P = \overrightarrow{pq} \cup \overrightarrow{pq^{\prime}} $ and $\overrightarrow{pq} \cap \overrightarrow{pq^{\prime}} = \{p,p^{\prime}\}$
        \item \textbf{Theorem 12.2 (Fan: halfplane)}: Let $H,K$ be opposite halfplanes with edge line $\ell$, point $B \in H$. Let $X,A$ be points on $\ell$ with $0 < AX < \omega$. Let $h = \overrightarrow{XA}$, $k = \overrightarrow{XB}$. Then, $H $ consists of all points on all rays of the fan $\overrightarrow{hk}$, except for the points of $\ell$
            \bigbreak \noindent 
            That is, $P \in H \iff P \in j^{0}$, where $j^{0}$ is the interior of some ray $j \in \overrightarrow{hk}$, $j \ne h$ or $h^{\prime}$
        \item \textbf{Corollary 12.3}: Let $z$ by any number with $0 < z < 180$. For any ray $\overrightarrow{AB}$ there are exactly two rays $h,k$ in $P_{A}$ such that $\overrightarrow{AB}h = z = \overrightarrow{AB}k$. Furthermore, $h^{0}$ and $k^{0}$ lie in opposite halfplanes with edge $\overleftrightarrow{AB} $
        \item \textbf{Theorem 12.4 (The Crossbar Theorem)}: If $\underline{hk}$ is a proper angle with vertex (common endpoint) $X$, if $A \in h^{0}$ (so $h = \overrightarrow{XA}$), $C \in k^{0}$ (so $k = \overrightarrow{XC}$), and $ h\text{-}j\text{-}k$, then there is an interior point $B$ of $j$ with $ A\text{-}B\text{-}C$
        \item \textbf{Theorem 13.1 (ASA)}: If under the correspondence $ABC \leftrightarrow XYZ$, two angles and the included side of $\triangle ABC$ are congruent, respectively, to the corresponding two angles and included side of $\triangle XYZ$, then $\triangle ABC \cong \triangle XYZ $
        \item \textbf{Theorem 13.2 (pons asinorum ("Bride of asses"))} In any $\triangle ABC$, 
            $$ AB = AC \iff \angle ACB = \angle ABC $$
        \item \textbf{Corollary 13.3}: A triangle is equilateral if and only if it is equiangular
        \item \textbf{Theorem 13.4 (SSS)}: If in $\triangle ABC$ and $\triangle XYZ$, $\overline{AB} \cong \overline{XY}$, $ \overline{BC} \cong \overline{YZ}$ and $\overline{CA} \cong \overline{ZX}$, then 
            \begin{align*}
                \triangle ABC \cong \triangle XYZ
            .\end{align*}

    \end{itemize}

    \pagebreak 
    \subsection{Propositions}
    \begin{itemize}
        \item \textbf{Proposition 11.14}
            \begin{enumerate}[label=(\alph*)]
                \item If $\omega < \infty$, then $\angle ABC = \angle AB^{*}C$
                \item If $P \in \overrightarrow{BA}^{0} $ and $Q \in \overrightarrow{BC}^{0} $, then $\angle ABC = \angle PBQ$
                    \bigbreak \noindent 
            \end{enumerate}
        \item \textbf{Proposition 11.15 (Midpoint)}: If $\underline{pq}$ is a proper angle, then there is exactly one ray $b$ in the wedge $\overline{pq}$ so that $pb  = \frac{1}{2}pq $
      
    \end{itemize}



    \pagebreak 
    \subsection{Duals of results from chapters 8 and 9}
    \bigbreak \noindent 
    \subsubsection{Theorems (14)}
    \begin{itemize}
        \item \textbf{Theorem 8.1D}: The set of angle measures $\mathbb{D} = [0,180]$
        \item \textbf{Theorem 8.2D}: All wedges, fans, pencils have infinitely many rays
        \item \textbf{Theorem 8.3D}: Let $x\ne y$ be distinct from $a$ on fan $\overrightarrow{ab}$. Then, exactly one of 
            \begin{align*}
                a\text{-}x\text{-}y \quad \text{ or } \quad a\text{-}y\text{-}x
            .\end{align*}
        \item \textbf{Theorem 8.4D}: Let $\overrightarrow{ab}$ be a fan. If $c \in \overrightarrow{ab}$, $0 < c < 180$, then $\overrightarrow{ab} = \overrightarrow{ac}$
        \item \textbf{Theorem 8.6D}: Stated in theorem 11.6
        \item \textbf{Theorem 9.1D}: Let ray $a$ be in pencil $P$, there exists a unique fan $a^{\prime} \in P$ such that $aa^{\prime} = 180$. For all other rays $x\in P$, $ a\text{-}x\text{-}a^{\prime} $
        \item \textbf{Theorem 9.2D}: Stated in theorem 11.8
        \item \textbf{Theorem 9.4D}: If $ap = 180$ in some fan $h$, then $p = a^{\prime}$.
        \item \textbf{Theorem 9.6D}: Stated in theorem 11.9
        \item \textbf{Theorem 9.7D}: Each fan has a unique opposite fan.
        \item \textbf{Theorem 9.8D}: Let rays $a,b \in P$, if $0 < ab < 180$, then fan $\overrightarrow{ab}^{\prime} = \overrightarrow{ab^{\prime}} $
        \item \textbf{Theorem 9.9D}: Let rays $a,b \in P$, if $0 < ab < 180$, then $P = \overline{ab} \cup \overline{ab^{\prime}} \cup \overline{ba^{\prime}} \cup \overline{b^{\prime}a^{\prime}}$, where the interiors of these wedges are disjoint.
        \item \textbf{Theorem 9.10D}: Let rays $a,b \in P$, if $0 < ab < 180$, and $c$ is some other ray in $P$, then there exists no betweenness relation among $a,b,c$ if and only if $c \in \overline{a^{\prime}b^{\prime}} $
    \end{itemize}

    \bigbreak \noindent 
    \subsubsection{Propositions}
    \begin{itemize}
        \item \textbf{Proposition 8.11D}: Let $a,b \in P$, $0 < ab < 180$, there exists $c\in P$ such that $ c\text{-}a\text{-}b$, $cb < 180$
        \item \textbf{Proposition 8.5D}: A fan has at most two terminal rays 
        \item \textbf{Proposition 8.7D}: Let $\overline{ab}$ be a wedge, for all $x,y \in \overline{ab}$, $xy \leq ab$, if $xy = ab$, then $\{x,y\} = \{a,b\}$
        \item \textbf{Proposition 8.8D}: If $\overline{ab}  = \overline{cd}$, then $\{a,b\} = \{c,d\}$
        \item \textbf{Proposition 8.9D}: Stated in proposition 11.15
        \item \textbf{Proposition 9.3D}: Let $a,b \in P$ such that $0 < ab < 180$. Then,
            \begin{itemize}
                \item Fan $\overrightarrow{ab} = \overline{ab} \cup \overline{ba^{\prime}}$, with $\overline{ab} \cap \overline{ba^{\prime}}  = \varnothing$
                \item Fan $\overrightarrow{ab} = \overrightarrow{a^{\prime}b} $
            \end{itemize}
    \end{itemize}

    \pagebreak 
    \unsect{Part 3: Study of perpendiculars}
    \bigbreak \noindent 
    \subsection{Definitions}
    \begin{itemize}
        \item \textbf{Definition: Supplementary angles}: Two angles are \textbf{supplementary} if their measures sum to 180.
        \item \textbf{Definition}: Angles $\underline{hk}, \underline{rs}$ are \textbf{vertical} if $\{r,s\}  = \{h^{\prime}, k^{\prime}\}$
        \item \textbf{Definition: Perpendicular}: Two intersecting lines $m,n$ are \textbf{perpendicular} (at point of intersection $B$) if the four angles they determine at $B$ are right angles, we write $m\perp n$ (at $B$)
        \item \textbf{Definition: The perpendicular bisector}: The \textbf{perpendicular bisector} of a segment $\overline{AB}$ is the line perpendicular to $\overleftrightarrow{AB}$ at the midpoint $M$ of $\overline{AB}$
        \item \textbf{Definition: Pole}: Point $A$ is a \textbf{Pole} of line $m$ if there exists a point $X$ on $m$ such that 
            \begin{align*}
                \overleftrightarrow{AX} \perp m \text{ and } AX = \frac{\omega}{2}
            .\end{align*}
        \item \textbf{Definition: \textit{Right triangle}}: A \textbf{right triangle} is a triangle with exactly \textbf{one} right angle.
        \item \textbf{Definition: \textit{Hypotenuse}}: In a right triangle, the \textbf{hypotenuse} is the side opposite the right angle. The \textbf{legs} are the other two sides
        \item \textbf{Definition: \textit{Birectangular triangle}}: A triangle with exactly \textbf{two} right angles is a \textbf{birectangular} (e.g $\triangle ABC$ on $\mathbb{S}$ with $B,C$ on equator, $A = $ north pole). 
        \item \textbf{Definition: \textit{Trirectangular triangle}}: A triangle with three right angles is \textbf{trirectangular}
        \item \textbf{Definition: \textit{small triangle}}: A triangle is \textbf{small} if all sides have length $< \frac{\omega}{2}$. (So when $\omega = \infty$, every triangle is small).
            \bigbreak \noindent 
            If $\triangle ABC$ has more than one right angle (say $\angle B = \angle C = 90$), then $ \overleftrightarrow{AB}, \overleftrightarrow{AC}$ both perpendicular to $ \overleftrightarrow{BC}$, so thm 14.5 implies $A$ is a pole for $\overleftrightarrow{BC}$. Then, Thm 14.6 implies $AB = AC = \frac{\omega}{2}$, which implies $\triangle ABC$ is \textbf{not} small.
        \item \textbf{Definition: \textit{Cevian}:} A \textbf{Cevian} is a segment from a vertex of a triangle to a point on the opposite side.
        \item \textbf{Definition: \textit{exterior and remote interior angles}:} Given $ \triangle ABC$, and $D$ a point with $ B\text{-}C\text{-}D$, then $ \underline{\angle ACD} $ is called an \textbf{exterior angle} of $\triangle ABC$, and $ \underline{ \angle A}, \underline{ \angle B}$ are called the \textbf{remote interior angles} (relative to $ \underline{\angle ACD} $)
        \item \textbf{Definition:} for any line $m$ and point $A$, the \textbf{distance between $A$ and $m$}, denoted $d(A,m)$, is the minimum distance $AX$ for all points $X$ on $m$.
            \bigbreak \noindent 
            \textbf{Note:} If $A$ is on $m$, then $d(A,m) = AA = 0$

    \end{itemize}


    \pagebreak 
    \subsection{Theorems}
    \begin{itemize}
        \item \textbf{Theorem 14.1 (Supplementary angles theorem)}: If $h,j$ are coterminal rays, then $\underline{hj}$ and $\underline{jh^{\prime}} $ are supplementary
        \item \textbf{Theorem 14.2 (Vertical angles theorem)}: Vertical angles are congruent
        \item \textbf{Theorem 14.3}: Through any point $A$ on a line $m$, there is exactly one line $n$ perpendicular to $m$
        \item \textbf{Theorem 14.9 (needs proof)}: Every point of the perpendicular bisector of a segment is equidistant from the endpoints of the segment: $AX = BX$ for all $X$ on the perpendicular bisector
        \item \textbf{Theorem 14.10 (converse of 14.9)}: Let $m = \overleftrightarrow{AB}$, suppose that line $n\ne m$ meets $m$ at the midpoint $M$ of $\overline{AB}$. Suppose that there is some point $X$ on $n$, not on $m$, so that $AX = BX$. Then, $n \perp n$ at $M$
        \item \textbf{Theorem 14.4}: Through a point $A$ not on a given line $m$ there is at least one line $n$ perpendicular to $m$
        \item \textbf{Theorem 14.5}: If there are two different lines through a point $A$ and perpendicular to a line $m$, then $A$ is a pole of $m$.
        \item \textbf{Theorem 14.6}: If $A$ is a pole of line $m$, then every line through $A$ is perpendicular to $m$, and meets $m$ at a point distance $\frac{\omega}{2} $ from $A$. Also, every line perpendicular to $m$ goes through $A$
        \item \textbf{Corollary 14.7}: Suppose $\omega < \infty$, each line $m$ has exactly two poles, $A$ and $A^{*}$
        \item \textbf{Theorem 15.1 (Cevian theorem)}: Suppose $\omega < \infty$, if $AB < \frac{\omega}{2}$, and $AC \leq \frac{\omega}{2}$ in $\triangle ABC$, and if $ B\text{-}D\text{-}C $ (so $\overline{AD} $ is a cevian of $\triangle ABC$), then $AD < \frac{\omega}{2}$
        \item \textbf{Theorem 15.3 (EAI)}: An exterior angle of a small triangle has larger measure than either remote interior angle
        \item \textbf{Corollary 15.4 (needs proof)}: The nonright angles of a small right triangle are accute
        \item \textbf{Corollary 15.5 (needs proof)}: The base angles of an isosceles triangle whose congruent sides are $< \frac{\omega}{2}$ are acute.
        \item \textbf{Theorem 15.7 (The triangle inequality)}: In any $\triangle ABC$, 
            \begin{align*}
                AB + BC > AC
            .\end{align*}
        \item \textbf{Corollary 15.8}: For any points $A,B,C$, $AB + BC \geq AC$. 
        \item \textbf{Theorem 16.1 (Comparison theorem)}: If one angle of a triangle is larger than a second, then the side opposite the lager angle is longer than the side opposite the smaller angle; and conversely. 
            \bigbreak \noindent 
            That is, in $\triangle ABC$, 
            \begin{align*}
                \angle B > \angle C \iff AC > AB
            .\end{align*}
        \item \textbf{Corollary 16.2 (Needs proof)}: The hypotenuse of a small right triangle is its longest side
        \item \textbf{Theorem 16.3}: Suppose that in $\triangle ABC$, $ \angle C = 90$ and $AC  < \frac{\omega}{2} $. Then, $\underline{\angle B}$ is acute and $AB  > AC$.
        \item \textbf{Theorem 16.8}: Let $m$ be a line, $C \in m$, $A \not\in m$, $\overleftrightarrow{AC} \perp m$
            \begin{enumerate}[label=(\alph*)]
                \item If $AC < \frac{\omega}{2}$ then $d(A,m) = AC$; and $AC < AX$, all $X \ne C$ on $m$
                \item If $AC = \frac{\omega}{2}$ (so $\omega<\infty$), then $d(A,m) = \frac{\omega}{2} = AX$, all $X \in m$
                \item If $AC > \frac{\omega}{2}$ (so $\omega<\infty$), then $d(A,m)  = \omega - AC = AC^{*}$; and $AC^{*} < AX$, all $X \ne C^{*}$ on $m$
            \end{enumerate}
            \bigbreak \noindent 


    \end{itemize}

    \pagebreak 
    \subsection{Propositions}
    \begin{itemize}
        \item \textbf{Proposition 15.6}: If $AB < \frac{\omega}{2}$, and $ BC \leq \frac{\omega}{2}$ in $ \triangle ABC$, then $AB + BC > AC$
    \end{itemize}

    \pagebreak 
    \unsect{Planes}
    \begin{itemize}
        \item \textbf{Euclidean Distance}:
            \begin{align*}
                e(AB) = \sqrt{(x_{2} - x_{1})^{2} + (y_{2} - y_{1})^{2}} = \left\lvert x_{1} - x_{2} \right\rvert \sqrt{1+m^{2}}
            .\end{align*}
            \bigbreak \noindent 
            All 21 axioms are true for $\mathbb{E}$
        \item \textbf{Hyperbolic Distance}:
            \begin{align*}
                d_{\mathbb{H}}(AB) = \ln{\left(\frac{e(AN)e(BM)}{e(AM)e(BN)}\right)}
            .\end{align*}
            All 21 axioms are true for $\mathbb{E}$
        \item \textbf{Spherical Distance}:
            \begin{align*}
                d_{\mathbb{S}} = r\theta = r\cos^{-1}{\left(\frac{ax+by+cz}{r^{2}}\right)}
            .\end{align*}
            All 21 axioms are true for $\mathbb{E}$
        \item \textbf{Minkowski distance}:
            \begin{align*}
                d_{\mathbb{M}}(AB) = \left\lvert x_{2} - x_{1} \right\rvert + \left\lvert y_{2} - y_{1} \right\rvert = \left\lvert x_{1} - x_{2} \right\rvert (1+\abs{m})
            .\end{align*}
            The first 20 axioms are true for $\mathbb{M}$
        \item \textbf{Gap distance}:
            \begin{align*}
                d_{\mathbb{G}} = \begin{cases}
                    e(AB) - e(CD) & \text{ if $A,B$ on same side} \\
                    e(AB) & \text{otherwise}
                \end{cases}
            .\end{align*}
            The first 11 axioms are true for $\mathbb{G}$
    \end{itemize}

    \bigbreak \noindent 
    \subsection{IO}
    \bigbreak \noindent 
    Consider $\mathbb{P} = \{A,B,C,D,E,F\}$, $\mathbb{L}:\ \ell = \{A,B,C,D\}, m = \{A,E\}, n  = \{C,E\} , v = \{D,E\} $, and distance
    \bigbreak \noindent 
    \begin{align*}
        \begin{array}{c|ccccc}
                   &A&B&C&D&E \\ 
            A & 0 & 3 & 1 & 2 & 4\\
            B &  3 & 0 & 2  & 1 & 4\\
            C &  1 & 2 & 0 & 3 & 4\\
            D & 2 &1 & 3 & 0 & 4\\
            E & 4 & 4& 4 & 4 & 0\\
        \end{array}
    \end{align*}
    \bigbreak \noindent 
    The first 9 axioms are true for the IO model.

    \bigbreak \noindent 
    \subsection{TDM}
    \bigbreak \noindent 
    Let $\mathbb{P}$ be any set of at least three elements. Let $\mathbb{L}$ be the collection of all two element subsets of $\mathbb{P} $
    \bigbreak \noindent 
    Define distance as follows: For all $x\ne y \in \mathbb{P}$, 
    \begin{align*}
        \begin{cases}
            xy &= 1 \\
            xx &= 0
        \end{cases}
    \end{align*}
    \bigbreak \noindent 
    The first 9 axioms are true for the TDM.

    \bigbreak \noindent 
    \subsection{Fano}
    \bigbreak \noindent 
    The Fano plane is a \textit{projective plane of order two.}
    \bigbreak \noindent 
    There are three points on each line, and three points through each line
    \bigbreak \noindent 
    Which points on which line? Write points in alphabetical order in three rows, start with $A$, then $B$, then with $D$
    \begin{align*}
            &A \ B \ C \ D \ E \ F \\
            &B \ C \ D \ E \ F \ A\\
            &D \ E \ F \ A B \ C \ \\
    \end{align*}
    Note that the columns give the lines
    \bigbreak \noindent 
    Each point is an ordered triple $(x,y,z) $, where $x,y,z$ are integers mod $2$
    \bigbreak \noindent 
    \begin{align*}
        \begin{cases}
            0 & \text{ stands for all even numbers}     \\
            1 & \text{ stands for all odd numbers}     
        \end{cases}
    \end{align*}
    We further note that $\text{odd } + \text{ odd} = \text{ even}$. Or, $1 +1 = 0 $. Other than that it is business as usual... $0+0 =0,\ 1+0 = 0 + 1 =  1$
    \bigbreak \noindent 
    We have the points
    \begin{align*}
            &A(1,0,0) \quad B(1,1,0) \quad D(0,1,0) \quad E(0,0,1) \\
            &C(1,1,1) \quad F(1,0,1) \quad G(0,1,1) \quad \text{No point }:\ (0,0,0)
    \end{align*}
    \bigbreak \noindent 
    Given points $P,Q$, find the third point collinear with $P,Q$. We simply add the coordinate triples for $P,Q$. For example, suppose $A(1,0,0), B(1,1,0)$. Then,
    \begin{align*}
        (1,0,0) + (1,1,0) = (0,1,0) = D
    \end{align*}
    \bigbreak \noindent 
    We define distance for Fano points, but its not Euclidean distance
    \bigbreak \noindent 
    Given points $P,Q$,
    \begin{align*}
        d(PQ) = \text{ number of different respective coordinates}
    \end{align*}
    \bigbreak \noindent 
    For example, $B(1,1,0), G(0,1,1)$ implies $d(BG) = 2 $
    \bigbreak \noindent 
    The first 10 axioms are true for the Fano plane.

    \bigbreak \noindent 
    \subsection{White stripes}
    \bigbreak \noindent 
    Let $\ell, m$ be two parallel lines in $\mathbb{E}$
    \bigbreak \noindent 
    Define $\mathbb{P} = \{\text{all points on $\ell $}\} \cup \{\text{all points on $m$}\}$, and $\mathbb{L} = \ell,m$, and all two point sets $\{P,Q\}$ where $P$ on $\ell$, $Q$ on $m$. Define distance $d = $ Euclidean distance $e(PQ)$
    \bigbreak \noindent 
    Note that the seven axioms are true statements for $ws$, and $\mathbb{D} = [0,\infty),\ \omega = \infty$
    \bigbreak \noindent 
    The first 10 axioms are true for WS






    \pagebreak 
    \unsect{Which axioms took out which planes}
    \begin{itemize}
        \item \textbf{$\mathbb{\hat{E}}$}: Ax.SAS
        \item \textbf{$\mathbb{M}$}: Ax.Sas
        \item \textbf{$\mathbb{G}$}: Ax.S
        \item \textbf{$\mathbb{R}^{3}$}: Ax.S
        \item \textbf{IO}: Ax.N
        \item \textbf{TDM}: Ax.N
        \item \textbf{WS}: Ax.RR
        \item \textbf{Fano}: Ax.RR
    \end{itemize}

    \pagebreak 
    \unsect{Other computational stuff}
    \bigbreak \noindent 
    \subsection{Angle measure in $\mathbb{H}$ and $\mathbb{E}$}
    \bigbreak \noindent 
    In $\mathbb{H}$, Project Euclidean $\underline{\angle QPR}$ vertically up to the demispherical dome; measure the angle formed by the tangent lines to the circular arcs.
    \bigbreak \noindent 
    \begin{align*}
        \angle QPR = \cos^{-1}{\left(\frac{1+mn-bc}{\sqrt{1+m^{2}-b^{2}}\sqrt{1+n^{2}-c^{2}}}\right)}
    .\end{align*}
    \bigbreak \noindent 
    in $\mathbb{E}$, we have
    \begin{align*}
        \angle QPR = \cos^{-1}{\left(\frac{1+mn}{\sqrt{1+m^{2}}\sqrt{1+n^{2}}}\right)}
    .\end{align*}


    \pagebreak 
    \unsect{Exam 2 questions}
    \bigbreak \noindent 
    \begin{mdframed}
        1. Prove that for any four distinct points on a line, there must be a betweenness relation among some three of them
    \end{mdframed}
    \bigbreak \noindent 
    \textbf{\textit{Proof.}} Consider the distance $AB$, if $AB = \omega$, then by Thm 9.1, $ A\text{-}C\text{-}B$ and $ A\text{-}D\text{-}B$, since $B = A^{*}$. 
    \bigbreak \noindent 
    If $AB < \omega$, then $\overrightarrow{AB}$ defined. By Coroll. 11.11, $\overrightarrow{AB} \cup \overrightarrow{AB}^{\prime}  = \ell$, where $\ell$ is the line that contains $A,B,C,D$. If $C$ or $D$ or both are in $\overrightarrow{AB}$, then there is a betweenness relation among $A,B,C,D$ by definition of the ray $\overrightarrow{AB}$. If $C,D \in \overrightarrow{AB}^{\prime}$, then by Thm 8.3, either $  A\text{-}C\text{-}D$ or $ A\text{-}D\text{-}C$. Hence, a betweenness relation exists.


    % \bigbreak \noindent 
    % Case 1.) Assume $\omega = \infty$. In this case, name the points such that they appear in the order $A,B,C,D$ from left to right.
    % \bigbreak \noindent 
    % Since $AB < \omega = \infty$, ray $\overrightarrow{AB}$ is defined. Thus, $C,D \in \overrightarrow{AB}$, and therefore $ A\text{-}B\text{-}C$, $ A\text{-}B\text{-}D$, by definition of the ray $\overrightarrow{AB}$ and the ordering of the points. Thus, some betweenness relation exists.
    % \bigbreak \noindent 
    % Case 2.) Assume $\omega < \infty$. Consider the points $A,B$. By theorem 9.10, there exists no betweenness relation among $A,B,C$ iff $C \in \overline{A^{*}B^{*}}^{0}$. If $C \not\in \overline{A^{*}B^{*}}^{0}$, then there exists a betweenness relation among $A,B,C$ and we are done.
    % \bigbreak \noindent 
    % So, we consider the scenario when $C \in \overline{A^{*}B^{*}}^{0}$. So, no betweenness relation exists among $A,B,C$. We now consider the point $D$.
    % \bigbreak \noindent 
    % We note that $A^{*} \ne B^{*} \ne C^{*} \ne D^{*}$, since $A,B,C,D$ distinct. If any two antipodes were equal, then this would contradict theorem 9.1 (and 10.8), which guarantees uniqueness of antipodes for distinct points.
    % \bigbreak \noindent 
    % By coroll 9.10, there exists no between relation among $A,B,D$ iff $ D \in  \overline{A^{*}B^{*}}^{0}$, no betweenness relation among $A,C,D$ iff $D \in \overline{A^{*}C^{*}}^{0}$, no betweenness relation among $B,C,D$ iff $D \in \overline{B^{*}C^{*}}^{0}$. 
    % \bigbreak \noindent 
    % If none of these segment membership conditions are upheld, then there exists a betweenness relation.
    % \bigbreak \noindent 
    % Without loss of generality, assume $D \in \overline{A^{*}B^{*}}^{0}$. Then, since $A^{*} \ne B^{*} \ne C^{*}\ne D^{*}$, $D$ not in any of the other segments mentioned and there exists a betweenness relation.
    % \bigbreak \noindent 
    % Thus, there always exists a betwenness relation among some three given four distinct collinear points.

    \pagebreak \bigbreak \noindent 
    \begin{mdframed}
        2. Given distinct points $P,Q,R,S$ on a line $m$, with $ P\text{-}R\text{-}S$ and $PS < \omega < \infty$, answer true or false and explain why in each case
        \begin{enumerate}[label=(\alph*)]
            \item $R^{*} \in \overrightarrow{PR} $
            \item $\overrightarrow{PR} = \overrightarrow{PS}$
            \item There is exactly one point $X$ on $m$ with $PX = \frac{\omega}{2}$
            \item $\overrightarrow{RP}$ and $\overrightarrow{RS}$ have only on point in common.
            \item $ PQ = PQ^* $
            \item $\overrightarrow{PS} \cap \overrightarrow{RS} = \{S\} $
        \end{enumerate}
    \end{mdframed}
    \bigbreak \noindent 
    a.) False, if $R^{*} \in \overrightarrow{PR}$, then one of $ P\text{-}R\text{-}R^{*}$ or $ P\text{-}R^{*}\text{-}R$. Suppose $ P\text{-}R\text{-}R^{*} $, then $PR + RR^{*} = PR^{*}$, implies $ RR^{*} = \omega < PR^{*}$. But, since $P \ne R$, $PR^{*} < \omega$, a contradiction. Next, suppose $ P\text{-}R^{*}\text{-}R$, then $ PR^{*} + R^{*}R = PR$, which implies $R^{*}R = \omega < PR$, but since $\overrightarrow{PR}$ defined, $PR < \omega$, so $\omega < \omega$, another contradiction. Thus, $R^{*} \not\in  \overrightarrow{PR}$
    \bigbreak \noindent 
    b.) True, since $S \in \overrightarrow{PR}$ (by definition of $ P\text{-}R\text{-}S$), and $PS < \omega$, $\overrightarrow{PR} = \overrightarrow{PS}$ but theorem 8.4
    \bigbreak \noindent 
    c.) False, there are exactly two points $X,Y \in m$ such that $PX = PY = \frac{\omega}{2} $, one in $\overrightarrow{PR}$, and one in $\overrightarrow{PR}^{\prime}$, by theorem 8.4 and the fact that $\overrightarrow{PR} \cup \overrightarrow{PR}^{\prime} = m $
    \bigbreak \noindent 
    d.), False, by theorem 9.6 since $\omega < \infty$, the intersection is instead $\{R,R^{*}\} $
    \bigbreak \noindent 
    e.) False, assume for the sake of contradiction that $PQ = PQ^{*}$, first consider the case when $PQ = \omega$. Thus, either $P = Q^{*} $ or $Q = P^{*}$, if they both happen simultaneously, then 
    \begin{align*}
        PQ = PQ^{*} \implies PP^{*} = Q^{*}Q^{*} \implies \omega = 0
    .\end{align*}
    a contradiction, thus not both at once. If $P = Q^{*}, Q \ne P^{*}$, then
    \begin{align*}
        PQ = PQ^{*} \implies Q^{*}Q = Q^{*}Q^{*} \implies \omega = 0
    .\end{align*}
    another contradiction. If $Q = P^{*}, P \ne Q^{*}$, then
    \begin{align*}
        PQ = PQ^{*} \implies PP^{*} = PQ^{*} \implies \omega = PQ^{*}
    .\end{align*}
    but, $P \ne Q$. So, another contradiction. Thus, for $PQ = PQ^{*}$, $PQ = QP < \omega$. So, $\overrightarrow{QP}$ is defined. By prop 9.3, $\overrightarrow{QP} = \overrightarrow{Q^{*}P}$, and $Q \in \overrightarrow{Q^{*}P} $ implies one of $ Q^{*}\text{-}Q\text{-}P$ or $ Q^{*}\text{-}P\text{-}Q$. Assume $ Q^{*}\text{-}Q\text{-}P$. Then, $Q^{*}Q + QP = Q^{*}P $, which implies $ Q^{*}Q = \omega < Q^{*}P $, and since $P \ne Q$, a contradiction.
    \bigbreak \noindent 
    assume $ Q^{*}\text{-}P\text{-}Q$, then $ Q^{*}P + PQ = Q^{*}Q = \omega$, and thus $PQ = \omega - Q^{*}P$. A contradiction, since we assumed $PQ = PQ^{*}$, Therefore, $PQ \ne PQ^{*} $

    \bigbreak \noindent 
    f.) False, $ P\text{-}R\text{-}S$ implies $ R \in \overrightarrow{PS}$, and we know that $R \in \overrightarrow{RS}$.

    \pagebreak \bigbreak \noindent 
    \begin{mdframed}
        3. Let $m,n$ be lines with $m \cap n = \varnothing $, let $H,K$ be opposite halfplanes with edge $m$. Prove 
        \begin{enumerate}[label=(\alph*)]
            \item $\omega = \infty$
            \item Either $n \subseteq H$ or $n \subseteq K$
        \end{enumerate}
    \end{mdframed}
    \bigbreak \noindent 
    a.) Assume for the sake of contradiction that $\omega < \infty$. Then, by theorem 10.11, all lines have a point in common. Thus, $m \cap n \ne \varnothing$, a contradiction. So, $\omega = \infty $
    \bigbreak \noindent 
    b.) Let $A \in n$, choose $B \in n$ such that $0 < AB < \omega $, such points exist by Ax.RR. We note that by theorem 10.2, all segments, rays, lines are convex. Thus, for $X,Y \in n$, $\overline{XY} \subseteq n$. Thus, $\overline{AB} \subseteq n$. Suppose that $A \in H$, $B \in K$. Then, by theorem 10.6, there exists a point $X \in m$ such that $ A\text{-}X\text{-}B$. But, this implies that $X \in \overline{AB}$. Thus, $X \in \overline{AB} \implies X \in n$, and $X \in m$, which means $n \cap m \ne \varnothing$, a contradiction. 
    \bigbreak \noindent 
    So, $n$ must consist entirely of points in $H$ or in $K$ but cannot have points in both, so $n \subseteq H$ or $n \subseteq K$.

    \pagebreak 
    \bigbreak \noindent 
    \begin{mdframed}
        4. Let $A,B,C$ be three noncollinear points. Let $D,E$ be points with $ A\text{-}D\text{-}C$ and $ A\text{-}E\text{-}B$. Prove
        \begin{enumerate}[label=(\alph*)]
            \item $\angle BCE + \angle   ECA = \angle BCA$ 
            \item $\overrightarrow{CE}$ meets $\overline{BD} $
            \item $\angle BCA + \angle BCA^{*}  = 180$
            \item $ \overrightarrow{BD}\text{-}\overrightarrow{BC}\text{-}\overrightarrow{BA^{*}} $
        \end{enumerate}
    \end{mdframed}
    \bigbreak \noindent 
    a.) By Ax.C, $ A\text{-}E\text{-}B$ and pt $C$ yields $ \overrightarrow{CA}\text{-}CE\text{-}CB = \overrightarrow{CB}\text{-}\overrightarrow{CE}\text{-}\overrightarrow{CA}$, which implies
    \begin{align*}
        \overrightarrow{CB}\overrightarrow{CE} + \overrightarrow{CE}\overrightarrow{CA} = \overrightarrow{CB}\overrightarrow{CA}
    .\end{align*}
    Since $\overrightarrow{CB}\overrightarrow{CE} = \angle BCE$, $\overrightarrow{CE}\overrightarrow{CA} = \angle BCA$, and $\overrightarrow{CB}\overrightarrow{CA} = \angle BCA$, $ \angle BCE + \angle ECA = \angle BCA$
    \bigbreak \noindent 
    b.) Since $D \in \overrightarrow{CA}^{0}$, $B \in \overrightarrow{CB}^{0} $, and $ \overrightarrow{CA}\text{-}\overrightarrow{CE}\text{-}\overrightarrow{CB} $, by the Crossbar theorem, there exits a point $F \in \overrightarrow{CE}^{0}$ such that $ B\text{-}F\text{-}D$. Thus, $ \overrightarrow{CE}$ meets $ \overline{BD}$ at $F$
    \bigbreak \noindent 
    c.) $ A\text{-}D\text{-}C$ implies $ A,D,C$ collinear. Thus, $A,A^{*}, C$ collinear since $A^{*}$ on the same line as $A$ (thm 10.5). Thus, by theorem 9.1, $ A\text{-}C\text{-}A^{*}$, so $\overrightarrow{CA}$ and $\overrightarrow{CA^{*}} $ are opposite rays, and $ \overrightarrow{CA}\overrightarrow{CA^{*}}  = 180$ (Ax.M4). Further, we have by theorem 10.8 that $ \overrightarrow{CA}\text{-}\overrightarrow{CB}\text{-}\overrightarrow{CA^{*}} $. Thus, 
    \begin{align*}
        \overrightarrow{CA}\overrightarrow{CB} + \overrightarrow{CB}\overrightarrow{CA^{*}} = \overrightarrow{CA}\overrightarrow{CA^{*}} = 180 \\
    .\end{align*}
    Note that $\overrightarrow{CA}\overrightarrow{CB}  = \overrightarrow{CB}\overrightarrow{CA}$ (Ax.M3). Thus, we have that
    \begin{align*}
        \angle BCA + \angle BCA^{*} = \angle ACA^{*} = 180
    .\end{align*}
    \bigbreak \noindent 
    d.) First, we show that $ D\text{-}C\text{-}A^{*} = A^{*}\text{-}C\text{-}D$. Observe that since $ A\text{-}D\text{-}C$, $D \in \overrightarrow{AC}$, more specifically, $ D \in \overline{AC}$. By prop 9.3, $ \overrightarrow{AC} = \overrightarrow{A^{*}C}$. So, $D \in \overrightarrow{A^{*}C}$. Thus, one of 
    \begin{align*}
        A^{*}\text{-}D\text{-}C, \quad \text{ or } \quad A^{*}\text{-}C\text{-}D
    .\end{align*}
    Assume $ A^{*}\text{-}D\text{-}C$. Then, $D \in \overline{A^{*}C}$, which by prop 6.3  $\overline{A^{*}C} = \overline{CA^{*}}$, and by prop 9.3 $\overrightarrow{AC} = \overline{AC} \cup \overline{CA^{*}}$, with $ \overline{AC}^{0} \cap \overline{CA^{*}}^{0} = \varnothing$. Thus, $D \in \overline{CA^{*}}$ and $ D \in \overline{AC} $ is a contradiction, and we infact have that $ A^{*}\text{-}C\text{-}D = D\text{-}C\text{-}A^{*}$. 
    \bigbreak \noindent 
    From here, since $A,B,C$ noncollinear, $B$ not on the same line with $A,C$. Thus, Ax.C implies
    \begin{align*}
        \overrightarrow{BD}\text{-}\overrightarrow{BC}\text{-}\overrightarrow{BA^{*}}
    .\end{align*}
    \endpf


    \pagebreak \bigbreak \noindent 
    \begin{mdframed}
        5. Given distinct rays $p,q,r,s$ in a pencil $P$ with $ p\text{-}r\text{-}s$, $ps < 180$, $pq=85$, $qr = 70 $, answer T/F and explain why in each case
        \begin{enumerate}[label=(\alph*)]
            \item There is a betweenness relation among $p,q,r$
            \item $r^{\prime}$ is in $\overrightarrow{pr}$
            \item $pq^{\prime}  = 85$
            \item $\overrightarrow{pr} = \overrightarrow{ps} $
            \item There is exactly one ray $x$ in $P$ with $px = 100 $
            \item If $ p\text{-}q\text{-}r$ then $ q\text{-}r\text{-}s$
        \end{enumerate}
    \end{mdframed}
    \bigbreak \noindent 
    a.) True, Since
    \begin{align*}
        pq + qr = 70 + 85 = 155 < 180,
    .\end{align*}
    Ax.BR says there exists a betweenness relation among $p,q,r$
    \bigbreak \noindent 
    b.) False, if $r^{\prime} \in \overrightarrow{pr}$, then one of 
    \begin{align*}
        p\text{-}r^{\prime}\text{-}r,\ \quad p\text{-}r\text{-}r^{\prime}
    .\end{align*}
    Assume $ p\text{-}r^{\prime}\text{-}r $, then $pr^{\prime} + r^{\prime}r = pr$, which implies $rr^{\prime} =180 < pr$. By axiom $Ax.M1$, $ 0 \leq pr \leq 180 $. Thus, a contradiction.
    \bigbreak \noindent 
    Assume $ p\text{-}r\text{-}r^{\prime}$, then $ pr + rr^{\prime} = pr^{\prime}$, which implies $rr\pime = 180 < pr^{\prime}$, another contradiction. 
    \bigbreak \noindent 
    Thus, $r^{\prime} \not \in \overrightarrow{pr} $
    \bigbreak \noindent 
    c.) False, by theorem 11.8, $ q\text{-}p\text{-}q^{\prime}$, thus $ qp + pq^{\prime} = qq^{\prime} = 180$, which implies $pq^{\prime} = 180 - 85 = 95 $
    \bigbreak \noindent 
    d.) True, by $ p\text{-}r\text{-}s$, $s \in \overrightarrow{pr}$, by the dual of theorem 8.4, since $ ps < 180, \overrightarrow{pr} = \overrightarrow{ps} $
    \bigbreak \noindent 
    e.) False, by theorem 12.3, there are exactly two rays $x,y$ in $P$ with $px = py = 100$.
    \bigbreak \noindent 
    f.) False, contradicts theorem 11.3 (UMT for rays)


    \pagebreak \bigbreak \noindent 
    \begin{mdframed}
        6. Let $B \ne C$ be points on the same side of line $\overleftrightarrow{AX}$. Prove that exactly one of the following is true
        \begin{align*}
            A\text{-}B\text{-}C, \quad A\text{-}C\text{-}B, \quad \overrightarrow{AX}\text{-}\overrightarrow{AB}\text{-}\overrightarrow{AC}, \quad \overrightarrow{AX}\text{-}\overrightarrow{AC}\text{-}\overrightarrow{AB}
        .\end{align*}
    \end{mdframed}
    \bigbreak \noindent 
    \textbf{\textit{Proof.}} We first note that since $B,C \ne \overleftrightarrow{AX}$, $A,B,C,X$ noncollinear, so all pairs of distances less than $\omega$ by theorem 10.8 or proposition non-collinear. Thus, $\overrightarrow{AX},\overrightarrow{AB}$ defined. Call $\overrightarrow{AX} = h$, $ \overrightarrow{AB} = k$. By theorem 12.2, 
    \begin{align*}
        P \in H \iff P \in j^{0}, \text{ for } j\in \overrightarrow{hk},\ j\ne h, h^{\prime}
    .\end{align*}
    \bigbreak \noindent 
    Suppose that $C \in \overrightarrow{AB}^{0} = k^{0}$. Then, either $ A\text{-}B\text{-}C$ or $ A\text{-}C\text{-}B$ by definition of a ray. Note that if one occurs, the other cannot by the Unique Middle Theorem.
    \bigbreak \noindent 
    Further, note that by Thm. 8.4, $\overrightarrow{AB} = \overrightarrow{AC}$, since we stated above that $AC < \omega$ and $C \in \overrightarrow{AB}^{0}$. Thus, both $ \overrightarrow{AX}\text{-}\overrightarrow{AB}\text{-}\overrightarrow{AC}$ and $ \overrightarrow{AX}\text{-}\overrightarrow{AC}\text{-}\overrightarrow{AB}$ cannot occur since $ \overrightarrow{AB}, \overrightarrow{AC}$ are not distinct rays.
    \bigbreak \noindent 
    Suppose that $C \not\in k^{0}$, then $A,B,C$ noncollinear, and thus $ A\text{-}B\text{-}C$ and $ A\text{-}C\text{-}B$ cannot occur. Call the ray that $C$ is in $\overrightarrow{AC}$. Since $C \not \in \overleftrightarrow{AX}$, $\overrightarrow{AC} \ne \overrightarrow{AX}$, and $\overrightarrow{AC} \ne \overrightarrow{AB}$ since $C \not\in \overrightarrow{AB} $. Thus, $\overrightarrow{AB}, \overrightarrow{AC},\overrightarrow{AX}$ are distinct, coterminal rays, and by definition of $\overrightarrow{hk} = \overrightarrow{\overrightarrow{AX}\overrightarrow{AB}}$, where $\overrightarrow{AC} \in \overrightarrow{hk}$, exactly one of 
    \begin{align*}
        \overrightarrow{AX}\text{-}\overrightarrow{AB}\text{-}\overrightarrow{AC}, \quad \overrightarrow{AX}\text{-}\overrightarrow{AC}\text{-}\overrightarrow{AB}
    .\end{align*}
    \endpf


    \pagebreak 
    \unsect{Final exam questions}
    \begin{mdframed}
        1. For each example below of a plane, state in full the first axiom, from our list of 21 that is false, explain why it is false. If no axioms are false, state none 
    \end{mdframed}
    \begin{enumerate}[label=(\alph*)]
        \item \textbf{$\mathbb{G}$}: Ax.s is false for the gap plane, you cannot guarantee two opposite halfplanes
        \item $TDM$: Ax.N, all lines only have two points and the distance between those two points is $ \omega = 1$, so let $\ell$ be a line with points $A,B$, where $AB = 1 = \omega$, there is no point $C \in \overleftrightarrow{AB}$ such that $ 0 < AC < \omega$, so Ax.N false.
        \item $\mathbb{S}$: None
        \item Fano: Ax.RR, set of distances is $\mathbb{D} = \{0,1,2,3\}$, so on any ray, you will never find for example a point that makes distance $\frac{1}{2}$ with the endpoint of the ray.
        \item $\mathbb{M}$: Ax.SAS, the way distance is defined on $\mathbb{M}$, if two sides and the angle between them are congruent for two triangles, the third side is not guaranteed to be congruent.
        \item WS: Ax.RR, all the two point lines have rays since $\omega = \infty$, but you don't have infinitely many points. If $\overrightarrow{PB}$ is a ray for the line $\{P,Q\}$, say $PQ = 10$, you don't have a point $X$ such that $PX = 5$ for example, which would be true if Ax.RR were true.
    \end{enumerate}

    \bigbreak \noindent 
    \begin{mdframed}
        2. 
        \begin{enumerate}[label=(\alph*)]
            \item Define each of: ray, interior of a ray, opposite rays, opposite halfplanes, and sketch what each of these looks like in $\mathbb{H}$ and $\mathbb{S}$
            \item Fan and sketch what it looks like in $\mathbb{E}$ 
            \item Define the diameter $\omega$ of a plane, and explain what $\omega$ equals for each of $\mathbb{H}, \mathbb{S}(r)$, and the Fano plane.
        \end{enumerate}
    \end{mdframed}
    \bigbreak \noindent 
    a.) If $A,B\in \ell$, $ 0 < AB < \omega$, then the ray $\overrightarrow{AB} = \{X:\ A\text{-}X\text{-}B\} \cup \{X:\ A\text{-}B\text{-}X\} $, the interior of a ray is the set of all points in the ray that are not endpoints of the ray.
    \bigbreak \noindent 
    Two rays are opposite if there union is a line. Sets $H,K$ are opposite halfplanes with edge a line $\ell$ if $H,K$ are sets opposed around $\ell$, and for all $X \in H,\ Y \in K$, $\overline{XY} \cap m \ne \varnothing$
    \bigbreak \noindent 
    c.) The diameter $\omega$ of a plane is the least upper bound (supremum) for the set of distance $\mathbb{D}$. In $\mathbb{S}(r)$, the lines are great circles, and distance is defined as the length of the minor arc, so since the circumference of the great circle is $2\pi r$, and the length of a minor arc is at most half the circumference, then $\omega = \frac{2\pi r}{r} = \pi r$. In $\mathbb{H}$, $\omega = \infty$, the set of distances is unbounded.
    \bigbreak \noindent 
    In Fano, the set of distances is $\mathbb{D} = \{0,1,2,3\}$, so $\omega =\infty$. This is because the distance between two points on the Fano plane is the number of differences in the respective coordinates. Each point is a ordered triple, so the maximum number of differences between two points is three .



    \bigbreak \noindent 
    \begin{mdframed}
        3. Given points $A,B$ with $0 < AB < \omega < \infty$
        \begin{enumerate}[label=(\alph*)]
            \item Prove $A^{*}B^{*} = AB$ (hint: consider $A^{*}B$)
            \item What is the dual of (a)
            \item Prove if $A,B,C$ are noncollinear, then so are $A^{*}B^{*}C^{*}$, and $\triangle A^{*}B^{*}C^{*}  \cong \triangle ABC$
        \end{enumerate}
    \end{mdframed}
    \bigbreak \noindent 
    a.) We have by Theorem 9.1 $ A\text{-}B\text{-}A^{*}$, so $ A^{*}B = \omega - AB$, we also have that $ B\text{-}A^{*}\text{-}B^{*}$, so $ A^{*}B^{*} = \omega -A^{*}B$. Thus,
    \begin{align*}
        A^{*}B^{*} = \omega - A^{*}B = \omega - (\omega - AB) = AB
    .\end{align*}
    \bigbreak \noindent 
    b.) If $h,j$ are rays with $0 < hj < 180$, then $h^{\prime}j^{\prime} = hj $
    \bigbreak \noindent 
    c.) Assume for the sake of contradiction that $A^{*}, B^{*}, C^{*}$ are actually collinear, but Theorem 10.5 / 10.8 implies that $A$ collinear with $A^{*}$, $B$ collinear with $B^{*}$, and $C$ collinear with $C^{*}$, so then $A^{*}, B^{*}, C^{*}, A,B,C$ would be collinear, which contradicts $A,B,C$ being noncollinear
    \bigbreak \noindent 
    consider the triangles $\triangle ABC$ and $\triangle A^{*}B^{*}C^{*}$, by part (a), we have that $A^{*}B^{*} = AB$, $B^{*}C^{*} = BC$, and  $A^{*}C^{*} = AC$, which implies $ \overline{AB} \cong  \overline{A^{*}B^{*}}, \overline{BC} \cong \overline{B^{*}C^{*}}$, and $\overline{AC} \cong \overline{A^{*}C^{*}} $, so by Theorem 13.4 (SSS), $ \triangle ABC \cong \triangle A^{*}B^{*}C^{*}$

    \bigbreak \noindent 
    \begin{mdframed}
        4. Given an absolute plane with $\omega < \infty$, $H,K$ opposite halfplanes with edge $m = \overleftrightarrow{BC}$, point $A$ in $H$ with $ \overleftrightarrow{AB} \perp \overleftrightarrow{BC}$, and $AB = \frac{\omega}{3} $, point $D$ on $\overleftrightarrow{BC}$ but not on $\overrightarrow{BC} $. Answer True or False and explain in each case
        \begin{enumerate}[label=(\alph*)]
            \item $B$ is the closest point on $m$ to $A$
            \item $D$ is on $\overrightarrow{BC^{*}} $
            \item There is a pole $P$ for $m$ on $\overrightarrow{BA}$
            \item If $X \in \overrightarrow{AB}$ but $X \not\in \overline{AB}$, then $X \in  K$
            \item If $Y \in H$ and $ \underline{\angle YBC}$ is obtuse, then $\overrightarrow{BA}$ meets $\overline{YC} $
            \item If $Z \in H$ and $ \underline{\angle ZBC}$ is acute, then $ \overrightarrow{BD}\text{-}\overrightarrow{BA}\text{-}\overrightarrow{BZ}\text{-}\overrightarrow{BC} $
        \end{enumerate}
    \end{mdframed}
    \bigbreak \noindent 
    a.) True, by Theorem 16.8, $d(A,m) = AB$ since $ AB < \frac{\omega}{2}$
    \bigbreak \noindent 
    b.) True, by theorems 9.6 and 9.8, $\overrightarrow{BC^{*}} \cup \overrightarrow{BC} = m$,  $ D \not\in  \overrightarrow{BC}$ implies $ D \in \overrightarrow{BC^{*}} $
    \bigbreak \noindent 
    c.) True, by Ax.RR, there exists a point $P$ on $\overrightarrow{BA}$ such that $BP = \frac{\omega}{2}$, and $ \overleftrightarrow{BA} = \overleftrightarrow{BP} \perp m$, so $P$ is a pole for $m$.
    \bigbreak \noindent 
    d.) True, by theorem 10.4, $\text{Int}(\overrightarrow{BA}) \subseteq H$, so $ \text{Int}(\overrightarrow{BA}^{\prime}) = \text{Int}(\overrightarrow{BA^{*}}) \subseteq K$. Since $X \in \overrightarrow{AB},\ X\not\in \overline{AB}$ implies that $ A\text{-}B\text{-}X = X\text{-}B\text{-}A$, which implies that $X \not\in \overrightarrow{BA}$, since neither $ B\text{-}X\text{-}A$ or $ B\text{-}A\text{-}X$ are true by the UMT, so $X \in \overrightarrow{BA^{*}}$. Since $X \in \overrightarrow{BA^{*}}$, and $X \ne B$ (since $ X \not\in \overline{AB}$), $X\in \text{Int}(\overrightarrow{BA^{*}})$, so $X \in K$.
    \bigbreak \noindent 
    e.) True, consider the fan $\overrightarrow{\overrightarrow{BC}\overrightarrow{BA}}$. By theorem 12.2 (fan: halfplane) $H$ is composed of all interior points of the rays in the fan $\overrightarrow{\overrightarrow{BC}\overrightarrow{BA}}$, except for the points on $\overleftrightarrow{BC} $. Since $Y \in H$, $\overrightarrow{BA} \in \overrightarrow{\overrightarrow{BC}\overrightarrow{BA}}$.
    \bigbreak \noindent 
    Since $\overleftrightarrow{AB} \perp \overleftrightarrow{BC}$, $ \overrightarrow{BC}\overrightarrow{BA} = 90$. Since $\underline{\overrightarrow{BC}\overrightarrow{BY}}$ obtuse, $\overrightarrow{BC}\overrightarrow{BY} > 90$, so $\overrightarrow{BY} \not\in \overline{\overrightarrow{BC}\overrightarrow{BA}} $ by Theorem 11.6. Thus, $ \overrightarrow{BC}\text{-}\overrightarrow{BA}\text{-}\overrightarrow{BY} $.
    \bigbreak \noindent 
    $C \in \text{Int}(\overrightarrow{BC})$, $Y \in \text{Int}(\overrightarrow{BY})$, $ \overrightarrow{BC}\text{-}\overrightarrow{BA}\text{-} \overrightarrow{BY}$ and the Crossbar Theorem implies there is a point $E \in \text{Int}(\overrightarrow{BA})$ such that $ Y\text{-}E\text{-}C$, so $ \overrightarrow{BA}$ meets $\overline{YC}$
    \bigbreak \noindent 
    f.) True, using the same argument above, $\overrightarrow{BZ} \in \overrightarrow{\overrightarrow{BC}\overrightarrow{BA}}$, and $ \overrightarrow{BC}\overrightarrow{BZ} < 90$ implies $ \overrightarrow{BZ} \in \overline{\overrightarrow{BC}\overrightarrow{BA}}$, so $ \overrightarrow{BC}\text{-}\overrightarrow{BZ}\text{-}\overrightarrow{BA} $. Also, $D \in \overleftrightarrow{BC},\ D \not\in \overrightarrow{BC}$ implies $ D \in \overrightarrow{BC^{*}}$, and $BD < \omega $ (since $D \not\in \overrightarrow{BC}$ implies $ D \ne B^{*} $), so $\overrightarrow{BC^{*}} = \overrightarrow{BD}$ by theorem 8.4. So, by theorem 11.6, $ \overrightarrow{BD}\text{-}\overrightarrow{BA}\text{-}\overrightarrow{BC}$.
    \bigbreak \noindent 
    $ \overrightarrow{BC}\text{-}\overrightarrow{BZ}\text{-}\overrightarrow{BA} = \overrightarrow{BA}\text{-}\overrightarrow{BZ}\text{-}\overrightarrow{BC}$, $ \overrightarrow{BD}\text{-}\overrightarrow{BA}\text{-}\overrightarrow{BC} $ and the ROI gives $ \overrightarrow{BD}\text{-}\overrightarrow{BA}\text{-}\overrightarrow{BZ}\text{-}\overrightarrow{BC}$, as desired.

    \bigbreak \noindent 
    \begin{mdframed}
        5. Given $\triangle ABC$ with $\angle A > \angle B$. Prove there is a point $D$ on $\overline{BC}$ with $AD = BD$
    \end{mdframed}
    \bigbreak \noindent 
    Consider the triangle $\triangle ABC$, further consider the fan $\overrightarrow{\overrightarrow{AB}\overrightarrow{AC}}$. Let $ z = \angle B$, by Theorem 11.6, there exists a ray $j \in \overrightarrow{\overrightarrow{AB}\overrightarrow{AC}}$ such that $\overrightarrow{AB}j = z$. Further, $j \in \overline{\overrightarrow{AB}\overrightarrow{AC}} $ since $ z = \angle B < \overrightarrow{AB}\overrightarrow{AC} = \angle A $.
    \bigbreak \noindent 
    Since $j \in \overline{\overrightarrow{AB}\overrightarrow{AC}}$, we have $ \overrightarrow{AB}\text{-}j\text{-}\overrightarrow{AC}$,  by the Crossbar Theorem (12.4), since $B \in \overrightarrow{AB}^{\circ}$, and $C \in \overrightarrow{AC}^{\circ}$, there exists a point $D \in j^{0}$ such that $ B\text{-}D\text{-}C$. 
    \begin{figure}[ht]
        \centering
        \incfig{tri1}
        \label{fig:tri1}
    \end{figure}
    \bigbreak \noindent 
    Consider the triangle $\triangle ADB$, since $ \angle DAB = \angle DBA = \angle B$, Pons Asinorum implies $ AD = BD$.

    \bigbreak \noindent 
    \begin{mdframed}
        6. Given $A,B,C$ noncollinear points, and $ B\text{-}X\text{-}C$, prove for any real number $z$ with $0 \leq z \leq 180$, there is a ray $j$ with endpoint $X$ so that $\overrightarrow{XC}j  = z$, and $j$ meets either $\overline{AB}$  or $\overline{AC}$
    \end{mdframed}
    \bigbreak \noindent 
    Consider the fan $\overrightarrow{\overrightarrow{XC}\overrightarrow{XA}}$, by Ax.RF, for all $z \in \mathbb{R}$, $ 0 \leq z \leq 180$, there exists a ray $j  \in \overrightarrow{\overrightarrow{XC}\overrightarrow{XA}}$ such that $\overrightarrow{XC}j = z$. Since $ j \in \overrightarrow{\overrightarrow{XC}\overrightarrow{XA}}$, either $ \overrightarrow{XC}\text{-}j\text{-}\overrightarrow{XA}$ or $ \overrightarrow{XC}\text{-}\overrightarrow{XA}\text{-}j$
    \bigbreak \noindent 
    Assume $ \overrightarrow{XC}\text{-}j\text{-}\overrightarrow{XA}$. $C \in \overrightarrow{XC}^{\circ}$, $ A \in \overrightarrow{XA}^{\circ}$ and the Crossbar Theorem implies there exists a point $D \in j^{0}$ such that $ A\text{-}D\text{-}C$, so $ j$ meets $\overline{AC}$
    \bigbreak \noindent 
    By the dual of proposition 9.3, $\overrightarrow{\overrightarrow{XC}\overrightarrow{XA}} = \overline{\overrightarrow{XC}\overrightarrow{XA}} \cup \overrightarrow{XA}\overrightarrow{XB}$, since $ \overrightarrow{XB} $ is opposite to $\overrightarrow{XA}$, by $ B\text{-}X\text{-}C $ and Theorem 9.6.
    \bigbreak \noindent 
    If $ \overrightarrow{XC}\text{-}\overrightarrow{XA}\text{-}j$, then $ \overrightarrow{XA}\text{-}j\text{-}\overrightarrow{XB}$ by the fact above. $ A \in \overrightarrow{XA}^{\circ}$, $ B \in \overrightarrow{XB}^{\circ} $ and the crossbar theorem implies there exists a point $E \in j^{0}$ such that $ A\text{-}E\text{-}B$, so $j$ meets the segment $\overline{AB} $

    \bigbreak \noindent 
    \begin{mdframed}
        7. Given $\triangle ABC$ with $AB < AC < \frac{\omega}{2}$, $M$ is the midpoint of $\overline{BC}$, and $  B\text{-}C\text{-}D$, prove 
        \begin{enumerate}[label=(\alph*)]
            \item $\overrightarrow{AB}\text{-}\overrightarrow{AM}\text{-}\overrightarrow{AC}\text{-}\overrightarrow{AD} $
            \item $\triangle ABM$ and $\triangle ACM$ are small
        \end{enumerate}
        Then, order the angles $\angle AMC, \angle ACM, \angle ACD, \angle ABM$ from smallest to largest, prove your answer
    \end{mdframed}
    \bigbreak \noindent 
    $ B\text{-}C\text{-}D$ implies $ B,C,D$ collinear, and $M$ the midpoint of $\overline{BC}$ implies $ M$ collinear with $B,C,D$, so $ B,M,C,D$ collinear points. $\triangle ABC$ implies $ A,B,C$ noncollinear, so $A$ not collinear with $B,M,C,D$.
    \bigbreak \noindent 
    $ B\text{-}M\text{-}C$ and Ax.C implies $ \overrightarrow{AB}\text{-}\overrightarrow{BM}\text{-}\overrightarrow{BC}$, $ B\text{-}C\text{-}D $ and Ax.C implies $ \overrightarrow{AB}\text{-}\overrightarrow{AC}\text{-}\overrightarrow{AD}$, these two relations and the ROI yields $ \overrightarrow{AB}\text{-}\overrightarrow{AM}\text{-}\overrightarrow{AC}\text{-}\overrightarrow{AD}$, as desired.
    \bigbreak \noindent 
    By the Cevian Theorem (15.1), $ AM < \frac{\omega}{2}$. Since $M$ is the midpoint of segment $ \overline{BC}$, we have $BM = MC = \frac{1}{2}BC$, so $ BC = 2BM = 2MC$. Since $ A,B,C$ noncollinear, $BC < \omega$ by Theorem 10.8, so $ BC < \omega$. Thus,
    \begin{align*}
        2BM &< \omega \\
        \implies BM &< \frac{\omega}{2}
    .\end{align*}
    And similarly, $MC <\frac{\omega}{2}$, so all sides $BA,BC,BM,MC < \frac{\omega}{2}$, thus both $\triangle ABM$ and $ \triangle ACM$ are small.
    \bigbreak \noindent 
    We note that $AB = AC$ implies $\overleftrightarrow{AM} \perp \overleftrightarrow{BC}$ at $M$ by Theorem 14.10, so both $\triangle AMB, \triangle AMC$ are small right triangles. By theorem 15.4, $ \angle ABM$ and $ \angle ACM $ are acute.
    \bigbreak \noindent 
    We also note that $\angle ACD$ exterior to $\triangle AMC$, so $\angle ACD > \angle AMC > 90$. Lastly, $ AB < AC$ implies $ \angle B > \angle C$, by the Comparison Theorem. Thus,
    \begin{align*}
        \angle ACM < \angle ABM < \angle AMC < \angle ACD
    .\end{align*}

    \pagebreak \bigbreak \noindent 
    \begin{mdframed}
        8. Given $P$ is a pole for line $\overleftrightarrow{BC}$, $M =$ midpoint of $\overline{PB}$, $N =$ midpoint of $\overline{P^{*}C}$, prove
        \begin{enumerate}[label=(\alph*)]
            \item $\triangle MBC  \cong \triangle NCB$
            \item $ \overrightarrow{BN}\text{-}\overrightarrow{BC}\text{-}\overrightarrow{BM} $
        \end{enumerate}
    \end{mdframed}
    \bigbreak \noindent 
    We have the following setup
    \bigbreak \noindent 
    \begin{figure}[ht]
        \centering
        \incfig{tris}
        \label{fig:tris}
    \end{figure}
    \bigbreak \noindent 
    First, we note that by Theorem 14.7, since $P$ is a pole for the line $\overleftrightarrow{BC}$, so is $P^{*}$. Thus, by definition of a pole, $PB = P^{*}C = \frac{\omega}{2}$.
    \bigbreak \noindent 
    Since $M$ the midpoint of $\overline{PB}$, and $N$ the midpoint of $\overline{P^{*}C}$, we have
    \begin{align*}
        PM = MB = \frac{1}{2}PB = \frac{1}{2} \cdot  \frac{\omega}{2} = \frac{\omega}{4} 
    .\end{align*}
    And 
    \begin{align*}
        P^{*}N = NC = \frac{1}{2}P^{*}C = \frac{1}{2} \cdot \frac{\omega}{2} = \frac{\omega}{4}
    .\end{align*}
    So, $\overline{MB} \cong \overline{NC}$. Also, notice that $ \overline{BC} \cong \overline{BC} $
    \bigbreak \noindent 
    Next, we consider the triangles $\triangle PMC,\ \triangle P^{*}NB$, since $P$ and $P^{*}$ are poles for $\overleftrightarrow{BC}$, $P^{*}B = PC = \frac{\omega}{2}$, so $\overline{P^{*}B} \cong \overline{PC}$. And, by the facts above, $\overline{P^{*}N} \cong \overline{PM}$
    \pagebreak \bigbreak \noindent 
    \begin{figure}[ht]
        \centering
        \incfig{tris3}
        \label{fig:tris3}
    \end{figure}
    \bigbreak \noindent 
    But, notice that $\angle BP^{*}N = \angle BP^{*}C$, since $N$ being the midpoint of $\overline{P^{*}C}$ implies $ P^{*}\text{-}N\text{-}C$, which implies $C \in \overrightarrow{P^{*}N} $. And similarly, $\angle  CPM = \angle CPB = \angle BPC$, and by proposition 11.14, $\angle BP^{*}C = \angle BPC$. Thus, $ \triangle BP^{*}N \cong \triangle CPM$ by Theorem 14.4 (SSS), so $ \overline{BN} \cong \overline{CM}$, and therefore $ BN = CM$.
    \bigbreak \noindent 
    So, again by Theorem 14.4 (SSS), $\triangle MBC \cong \triangle NCB$.
    \bigbreak \noindent 
    Next, observe that Since $ P^{*}\text{-}N\text{-}C$, $P^{*},N,C$ collinear. By Theorem 10.5/10.8, $P$ collinear with $P^{*}$, so $N,C,P$ collinear. $\triangle BNC$ implies $B$ not collinear with $N,C$. 
    \bigbreak \noindent 
    Since $ P^{*}\text{-}N\text{-}C$, and $ P^{*}\text{-}C\text{-}P$ by Theorem 9.1, by the ROI we have $ P^{*}\text{-}N\text{-}C\text{-}P $, thus $ N\text{-}C\text{-}P$.
    \bigbreak \noindent 
    $ N\text{-}C\text{-}P$ and $B$ not collinear with $N,C,P$ implies $ \overrightarrow{BN}\text{-}\overrightarrow{BC}\text{-}\overrightarrow{BP}$ by Ax.C.
     

    \bigbreak \noindent 
    \begin{mdframed}
        9. Given $\omega < \infty$, lines $m\ne n$, $P$ a pole for $m$, prove
        \begin{itemize}
            \item $P$ is not also a pole for $n$
            \item If $Q $ is a pole for $n$, then $PQ < \omega$, and $\overleftrightarrow{PQ}$ is perpendicular to both $m$ and $n$
        \end{itemize}
    \end{mdframed}
    \bigbreak \noindent 
    a.) Since $P$ a pole for $m$, all lines through $P$ meet $m$ at a point $X$ distance $\frac{\omega}{2 }$ from $P$ (Theorem 14.6). By Theorem 14.7, $P^{*}$ also a pole for $m$. Consider all lines $\overleftrightarrow{PX}$, where $PX = \frac{\omega}{2}$, note that $\overleftrightarrow{PX} = \overrightarrow{PX} \cup \overrightarrow{PX^{*}}$ by definition of opposite rays and Coroll. 9.7. Also, $PX^{*} = \omega - PX = \omega - \frac{\omega}{2} = \frac{\omega}{2} $. By theorem 8.6, in each ray $\overrightarrow{PX}, \overrightarrow{PX^{*}}$, there is exactly one point $Y \in \overrightarrow{PX}$ such that $PY = \frac{\omega}{2}$, and one point $Z  \in \overrightarrow{PX^{*}}$ such that $PZ = \frac{\omega}{2}$, so $Y = X, Z = X^{*} $, and all lines through $P$ meet at a distance $\frac{\omega}{2}$ from $m$. So, all points distance $\frac{\omega}{2}$ from $P$ must be on $m$.
    \bigbreak \noindent 
    Assume for the sake of contradiction that $P$ is a pole for $n$, then all points on $n$ meet at a distance $\frac{\omega}{2}$ from $P$, by Theorem 14.6, thus $n = m$, but $n \ne m$, a contradiction. So, $P$ cannot be a pole for $m$.
    \bigbreak \noindent 
    b.) Assume for the sake of contradiction that $PQ = \omega$, then by Theorem 10.8, $Q = P^{*}$, and $Q$ in the same line with $P$. 
    \bigbreak \noindent 
    Then, $P^{*}$ would be a pole for $n$, which means $P$ would be a pole for $n$ by theorem 14.7, which we saw above is impossible since $n\ne m$. So, $ PQ < \omega$
    \bigbreak \noindent 
    By theorem 14.6, all lines through $P$ are perpendicular to $m$, and all lines through $Q$ are perpendicular to $n$, since $\overleftrightarrow{PQ}$ goes through $P$ and $Q$, $\overleftrightarrow{PQ} \perp m$ and $\overleftrightarrow{PQ} \perp n $.






    
\end{document}
