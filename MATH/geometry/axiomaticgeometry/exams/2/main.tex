\documentclass{report}

\input{~/dev/latex/template/preamble.tex}
\input{~/dev/latex/template/macros.tex}

\title{\Huge{}}
\author{\huge{Nathan Warner}}
\date{\huge{}}
\fancyhf{}
\rhead{}
\fancyhead[R]{\itshape Warner} % Left header: Section name
\fancyhead[L]{\itshape\leftmark}  % Right header: Page number
\cfoot{\thepage}
\renewcommand{\headrulewidth}{0pt} % Optional: Removes the header line
%\pagestyle{fancy}
%\fancyhf{}
%\lhead{Warner \thepage}
%\rhead{}
% \lhead{\leftmark}
%\cfoot{\thepage}
%\setborder
% \usepackage[default]{sourcecodepro}
% \usepackage[T1]{fontenc}

% Change the title
\hypersetup{
    pdftitle={Exam 1}
}

\begin{document}
    % \maketitle
        \begin{titlepage}
       \begin{center}
           \vspace*{1cm}
    
           \textbf{Exam 1}
    
           \vspace{0.5cm}
            
                
           \vspace{1.5cm}
    
           \textbf{Nathan Warner}
    
           \vfill
                
                
           \vspace{0.8cm}
         
           \includegraphics[width=0.4\textwidth]{~/niu/seal.png}
                
           Computer Science \\
           Northern Illinois University\\
           United States\\
           
                
       \end{center}
    \end{titlepage}
    \tableofcontents
    \unsect{Part 1}
    \bigbreak \noindent 
    \subsection{Axioms}
    \item         \textbf{Axiom of distance}: For all points $P,Q$
        \begin{enumerate}
            \item $PQ \geq 0 $
            \item $PQ = 0 \iff P=Q $
            \item $PQ = QP $
        \end{enumerate}
    \item         \textbf{Axioms of incidence}
        \begin{enumerate}
            \item There are at least two different lines
            \item Each line contains at least two different points
            \item Each pair of points are together in at least one line
            \item Each pair of points $P,Q$, with $PQ < \omega$ are together in at most one line
        \end{enumerate}
    \item \textbf{Betweenness of points axiom (Ax. BP)}: If $A,B,C$ are distinct, collinear points, and if $AB + BC \leq \omega$, then there exists a betweenness relation among $A,B,C$
        \bigbreak \noindent 
        What this is really saying is that if \textbf{any} of $AB + BC$, $BA + AC$, $AC + CB$ is $ \leq \omega$, then there is a betweenness relation.
        \bigbreak \noindent 
        \textbf{Note:} If Ax.BP is true for a plane $\mathbb{P}$, and if $AB + BC \leq \omega$ for distinct collinear $A,B,C$, then there is a betweenness relation, but not necessarily $ A\text{-}B\text{-}C $
        \bigbreak \noindent 
        When $\omega = \infty$, then for any distinct collinear $A,B,C$, $AB +BC  < \infty = \omega $, so there will be a betweenness relation
    \item \textbf{Quadrichotomy Axiom for Points (Ax.QP)}: If $A,B,C,X$ are distinct, collinear points, and if $ A\text{-}B\text{-}C$. Then, at least one of the following must hold
        \begin{align*}
            X\text{-}A\text{-}B, \quad A\text{-}X\text{-}B, \quad B\text{-}X\text{-}C, \quad \text{or } \quad B\text{-}C\text{-}X
        \end{align*}
        \bigbreak \noindent 
        Thus, Ax.QP says that whenever $ A\text{-}B\text{-}C$ (say on line $\ell$), then any other point $X$ on line $\ell$ is in either $ \overrightarrow{BA} $ or $ \overrightarrow{BC} $. That is,
        \begin{align*}
            \ell = \overrightarrow{BA} \cup \overrightarrow{BC}
        \end{align*}

    \item \textbf{Nontriviality Axiom (Ax.N)}: For any point $A$ on a line $\ell$ there exists a point $B$ on $\ell$ with $0 < AB < \omega$
        \bigbreak \noindent 
        This axiom is true for the planes in which $\omega = \infty$ ($\mathbb{E}$, $\mathbb{M}$, $\mathbb{H}$, $\mathbb{G}$, $\mathbb{R}^{3}$, $\hat{\mathbb{E}} $, ws)
        \bigbreak \noindent 
        This axiom is also true for $\mathbb{S}$ and Fano, where $\omega < \infty $
     \item \textbf{Real ray Axiom (Ax.RR)}: For any ray $ \overrightarrow{AB}$, and for any real number $s $ with $0 \leq s \leq \omega$, there is a point $X$ in $\overrightarrow{AB}$ with $AX = s$
        \item \textbf{Separation Axiom Ax.S}: for each line $m$, there exists a pair of opposite halfplanes with edge $m$. 



    \pagebreak 
    \subsection{Definitions}
    \begin{itemize}
        \item \textbf{Definition (Endpoints)}. Point $A$ is called an endpoint of ray $\overrightarrow{AB} $
        \item \textbf{Definition (Interior points and length for a segment):} Given a segment $ \overline{AB}$, $A$ and $B$ are called its endpoints. All other points of $\overline{AB}$ are called \textbf{Interior points} of $\overline{AB}$
            \bigbreak \noindent 
            Distance $AB$ is called the \textbf{length} of $\overline{AB} $
            \bigbreak \noindent 
            The interior of $\overline{AB}$, denoted $\text{Int}\overline{AB}$ or $\overline{AB}^{0}$, means the set of all interior points of $\overline{AB}$. That is, $\text{Int}\overline{AB} = \overline{AB}^{0} = \{X: A\text{-}X\text{-}B\}$
        \item \textbf{Definition}. Assume $\omega < \infty$. Let $A$ be a point on a line $m$. The unique point $A_{m}^{*}$ on $m$ such that $AA_{m}^{*} = \omega$ is called the \textbf{antipode} of $A$ on $m$. Thus,
            \begin{align*}
                \begin{cases}
                    A,A_{m}^{*} \text{ are on m, }  AA_{m}^{*} = \omega \\
                    \text{and } A\text{-}X\text{-}A_{m}^{*} \text{ for all other points $X$ on $m$}
                \end{cases}
            \end{align*}
        \item \textbf{Definition (interior points of a ray)}: Let \( h = \overrightarrow{AB} \) be a ray.  
            All points of \( h \) that are not endpoints of \( h \) are called \textit{interior points} of \( h \).  
            \bigbreak \noindent 
            The \textit{interior} of \( h \) is the set of all interior points of \( h \),  
            and is denoted by \( h^\circ \), \( \overline{AB}^\circ \), or \( \text{Int } \overrightarrow{AB} \).
        \item \textbf{Definition (Opposite rays)}: Two rays with the same endpoint whose union is a line are called \textbf{opposite rays}
        \item \textbf{Notation:} Denote the ray opposite to ray $h$ by $h^{\prime}$. So, $\overrightarrow{AB}^{\prime}$ means the ray opposite $\overrightarrow{AB} $
        \item \textbf{Definition}: Let $H,K$ be opposite halfplanes with edge $m$. Two points in the same halfplane are said to be on the \textbf{same side} of $m$. 
        \item \textbf{Definition}: $A^{*} $ is called the \textbf{antipode} of $A$

    \end{itemize}

    \pagebreak 
    \subsection{Theorems}
    \begin{itemize}
        \item \textbf{Theorem 6.1 (Symmetry of betweenness)}. For a general plane $\mathbb{P}$ with points, lines, distance, and satisfy the seven axioms, $A-B-C \iff C-B-A$
        \item \textbf{Theorem 6.2 (UMT)}: If $A-B-C$ then $B-A-C$ and $A-C-B$ are false.
        \item \textbf{Theorem 7.6}: For any point $A$ on a line $\ell$ there exists a point $C$ not on $\ell$ with $0 < AC <\omega$ 
    \item \textbf{Triangle inequality for the line}: If $A,B,C$ are any three distinct, collinear points, then 
        \begin{align*}
            AB + BC \geq AC 
        \end{align*}
    \item \textbf{Rule of insertion}: 
        \begin{itemize}
            \item If $ A\text{-}B\text{-}C$ and $ A\text{-}X\text{-}B$, then $ A\text{-}X\text{-}B\text{-}C $
            \item If $ A\text{-}B\text{-}C$ and $ B\text{-}X\text{-}C$, then $ A\text{-}B\text{-}X\text{-}C $
        \end{itemize}
        \item \textbf{Theorem 8.1}: If $\omega = \infty$, then $\mathbb{D} = [0,\infty$); if $\omega < \infty$, then $\mathbb{D} = [0,\omega] $
        \item \textbf{Theorem 8.2} Each segment, ray, and line has infinitely many points.
        \item \textbf{Theorem 8.3}. If $X \ne Y$ are points different from $A$ on ray $\overrightarrow{AB}$, then one of $ A\text{-}X\text{-}Y$ or $ A\text{-}Y\text{-}X$ is true.
        \item \textbf{Theorem 8.4}. If $C$ is any point on ray $ \overrightarrow{AB}$ with $ 0 < AC < \omega$, then $ \overrightarrow{AC} = \overrightarrow{AB} $
        \item \textbf{Theorem 8.6 (UDR)} For any ray $ \overrightarrow{AB}$ and any real number $s$ with $0 \leq s \leq \omega$, there is a \textbf{unique} point $X$ on $\overrightarrow{AB}$ with $AX = s$. $X$ is in $\overline{AB}$ if and only if $s \leq   AB $
        \item \textbf{Theorem 9.1 (Antipode on line theorem)}: Let $A$ be a point on a line $m$ (in a plane with the 11 axioms). Assume that $\omega < \infty$. Then, there exists a unique point $A^{*}_{m}$ on $m$ such that $AA_{m}^{*} = \omega$. Further, if $X$ is any other point on $m$, then $ A\text{-}X\text{-}A^{*}_{m} $
        \item \textbf{Theorem 9.2 (Almost-uniqueness for Quadrichotomy)}:  
            Suppose that \( A, B, C, X \) are distinct points on a line \( m \),  
            and that \( A - B - C \). Then \textbf{\textit{exactly one}} of the following holds:  
            \[
                X - A - B, \quad A - X - B, \quad B - X - C, \quad B - C - X
            \]
            with the \textbf{\textit{only exception}} that both \( X - A - B \) and \( B - C - X \) are true  
            when \( \omega < \infty \) and \( X = B_m^* \).
            \bigbreak \noindent 
            (Note that \( B_m^* - A - B \) and \( B - C - B_m^* \) \textbf{\textit{are both true}} by Thm. 9.1)
        \item \textbf{Theorem 9.4}.
            If \( h \) is a ray with two endpoints \( A \) and \( P \),  
            then \( \omega < \infty \) and \( P = A_m^* \), where \( m \) is the carrier of \( h \) (\( h \subseteq m \)).
        \item \textbf{Theorem 9.6 (Opposite ray theorem)}: If $ B\text{-}A\text{-}C$, then $\overrightarrow{AB}$ and $\overrightarrow{AC}$ are opposite rays
            \bigbreak \noindent 
            Also, for $m = \overleftrightarrow{AB}$
            \begin{align*}
                \overrightarrow{AB} \cap \overrightarrow{AC} = 
                \begin{cases}
                    \{A\}     & \text{ if } \omega = \infty \\
                    \{A, A_{m}^{*}\}     & \text{ if } \omega<\infty
                \end{cases}
            \end{align*}
        \item \textbf{Corollary 9.7}: Each ray has a unique opposite ray.
        \item \textbf{Corollary 9.8}: Let $A,B$ be points on line $m$ with $0 <AB<\omega <\infty$. Then $\overrightarrow{AB}^{\prime} = \overrightarrow{AB_{m}^{*}} $
        \item \textbf{Corollary 9.9}: Let $A,B$ be points on line $m$ with $ 0 < AB < \omega < \infty$. Then, $ m = \overline{AB} \cup \overline{BA_{m}^{*}} \cup \overline{A_{m}^{*}B_{m}^{*}} \cup \overline{B_{m}^{*}A}$, with the interiors of these segments being disjoint.
        \item \textbf{Theorem 9.10}: Let $A,B$ be points on line $m$ with $0 < AB < \omega < \infty$ . Let $C \ne A,B,A_{m}^{*}, B_{m}^{*} $ be another point on $m$. Then there is no betweenness relation for $A,B,C$ if and only if $C \in \overline{A_{m}^{*}B_{m}^{*}}^{0}$
        \item \textbf{Definition}. A subset $S$ of $\mathbb{P}$ is \textbf{convex} if for each pair of points $X \ne Y$ in $S$ with $XY < \omega$, $\overline{XY} \subseteq S$ holds.
        \item \textbf{Theorem 10.1}: If $S_{1}$ and $S_{2}$ are convex sets in $\mathbb{P}$, then so is $S_{1} \cap S_{2}$
        \item \textbf{Theorem 10.2}: Segments, rays, and lines are convex.
        \item \textbf{Definition}: A pair of sets $H,K$ in $\mathbb{P}$ is called \textbf{opposed around a line $m$} if 
            \begin{itemize}
                \item $H,K \ne \varnothing $
                \item $H,K$ are convex
                \item $H \cap K = \varnothing $
                \item $H \cup K = \mathbb{P} - m$
            \end{itemize}
        \item \textbf{Theorem 10.3} Let $H,K$ be sets opposed around a line $m$ in $\mathbb{P}$. Suppose that $A,C$ are points so that $C \in m$, $A \in H$, $AC < \omega$. Then, $\text{Int}\overrightarrow{CA} \subseteq H$, and $\text{Int}\overrightarrow{CA}^{\prime} \subseteq K $
        \item \textbf{Corollary 10.4}: let $H,K$ be sets opposed around a line $m$, let $A,B$ be points not on $m$, with $ A\text{-}X\text{-}B$ for some point $X \in m$. Then, $A,B$ lie one in each of $H$ and $K$, in some order.
        \item \textbf{Definition}: Let $m$ be a line. Sets $H,K$ are called \textbf{opposite halfplanes with edge $m$} if:
            \bigbreak \noindent 
            \begin{align*}
                &H,K \text{ are opposed around $m$, and whenever } X \in H, Y \in K \text{ and } XY < \omega, \\ &\text{ then, } \overline{XY} \cap m \ne \varnothing
            \end{align*}
        \item \textbf{Theorem 10.5}: Suppose that $m$ is a line  so that there exists a pair $H,K$ of opposite half planes with edge $m$. Suppose also that $\omega < \infty$ and $A$ is a point on $m$. If $B$ is any point in $\mathbb{P}$ with $AB = \omega$, then $B \in m$ (so $B = A_{m}^{*}$, and there is only one point $B$ in all of $\mathbb{P}$ with $AB = \omega$)
            \bigbreak \noindent 
            In other words, let $H,K$ be opposite halfplanes with edge a line $m$, let $A \in m$, $\omega < \infty$. If $B \in \mathbb{P}$, $AB = \omega$, then $B \in m$, and $B$ unique in $\mathbb{P}$
        \item \textbf{Theorem 10.6}: Suppose that there is a pair $H,K$ of opposite halfplanes with edge $m$. Let $A \ne B$ be points not on $m$. Then, 
            \begin{align*}
                A,B \text{ lie one in each of $H,K$ } \iff \text{ there is a point $X$ on $m$ such that $ A\text{-}X\text{-}B $}
            \end{align*}
        \item \textbf{Corollary 10.7 (Needs proof)}: Suppose that there is a pair $H,K$ of opposite halfplanes with edge a line $m$. Then, $H,K$ is the only pair of sets opposed around $m$.
        \item \textbf{Theorem 10.8}: Suppose that $\omega < \infty$. For each point $ A$, there is exactly one point $A^{*}$ in $\mathbb{P}$ with $AA^{*} = \omega$. Also, every line through $A$ goes through $A^{*}$ as well.
        \item \textbf{Corollary 10.9}: Suppose that $\omega < \infty$. For any line $m$ and point $P$, there are just two possibilities:
            \begin{align*}
               \begin{cases}
                   P,P^{*} &\text{ both on $m$}     \\
                   P, P^{*} &\text{on opposite sides of $m$}
               \end{cases}
            \end{align*}

        \item \textbf{Theorem 10.10 (Pasch's Axioms) (needs proof)}: Let $A,B,C$ be three noncollinear points. Let $X$ be a point with $ B\text{-}X\text{-}C $, and $m$ a line through $X$ but not through $A,B,$ or $C$. Then, exactly one of
            \begin{enumerate}
                \item $m$ contains a point $Y$ with $ A\text{-}Y\text{-}C$
                \item $m$ contains a point $Z$ with $ A\text{-}Z\text{-}B $
            \end{enumerate}
        \item \textbf{Theorem 10.11}: Assume that $\omega < \infty$. Then, any two distinct lines must have a point (in fact, a pair of antipodes) in common.


    \end{itemize}

    \pagebreak 
    \subsection{Propositions}
    \begin{itemize}
        \item \textbf{Proposition 6.3}
            \begin{enumerate}[label=(\alph*)]
                \item $\overline{AB}$ lies in one line, the line $\overleftrightarrow{AB} $
                \item $\overline{AB} = \overline{BA} $
                \item If $x\in \overline{AB}$, with $X \ne B$, then $AX < AB $
            \end{enumerate}
        \item \textbf{Proposition 6.4}: Let $A,B,C,D$ be collinear points with $0 < AB < \omega$, $0< CD<\omega$, and $\overline{AB} = \overline{CD}$, then
        \begin{enumerate}[label=(\alph*)]
                \item Either $\{A,B\} = \{C,D\}$ or $\{A,B\} \cap \{C,D\} = \varnothing$
                \item $AB = CD$
            \end{enumerate}
        \item \textbf{Proposition 7.1}: If $A\text{-}B\text{-}C$ and $A\text{-}C\text{-}D$, then $A,B,C,D$ are distinct and collinear 
        \item \textbf{Proposition 7.2} If $A\text{-}B\text{-}C\text{-}D$, then $A,B,C,D$ are distinct and collinear, and $D\text{-}C\text{-}B\text{-}A $
        \item \textbf{Proposition 7.5}: If $X \ne Y$ are points distinct from $A$ or ray $\overrightarrow{AB}$, then at least one of $ A\text{-}X\text{-}Y$ or $ A\text{-}Y\text{-}X$ or $X,Y$ in $ \overline{AB}$ is true.
        \item \textbf{Important fact}:  Suppose $X$ is a point on a ray $\overrightarrow{AB}$ in a general plane.
            \begin{enumerate}
                \item If $ A\text{-}X\text{-}B$ then $AX < AB $
                \item If $ A\text{-}B\text{-}X$ then $AX > AB $
                \item IF $X = B$ then $AX = AB$
            \end{enumerate}
        \item \textbf{Proposition 8.11} Let $A,B$ be any two points on line $m$, with $0 < AB <\omega$. Then, there exists a point $C$ on $m$ with $ C\text{-}A\text{-}B$ and $ CB < \omega$.
        \item \textbf{Proposition 8.5}: A ray has at most two endpoints
        \item \textbf{Proposition 8.7}: Let $\overline{AB}$ be a segment and $X,Y \in \overline{AB}$. Then, $XY \leq AB$, and if $XY = AB$, then $\{X,Y\} = \{A,B\}$
        \item \textbf{Proposition 8.8} If $\overline{AB} = \overline{CD}$, then $\{A,B\}  = \{C,D\}$
        \item \textbf{Proposition 8.9}: In each segment $\overline{AB}$ there is a unique point $M$, called the \textbf{midpoint} of $\overline{AB} $, with the property that $AM = \frac{1}{2}AB$. Further, $AM = MB $
        \item \textbf{Proposition 9.3}: Assume \( \omega < \infty \). Let \( A, B \) be points on line \( m \)  
            with \( 0 < AB < \omega \). Then  
            \begin{enumerate}
                \item[(a)] \( \overrightarrow{AB} = \overline{AB} \cup \overline{BA_m^*} \) and \( \overline{AB}^{\circ} \cap \overline{BA_m^*}^{\circ} = \varnothing \).
                \item[(b)] \( \overrightarrow{AB} = \overrightarrow{A_m^* B} \), so that if \( A \) is an endpoint of a ray  
                    with carrier \( m \), then so is \( A_m^* \).
            \end{enumerate}
        \item \textbf{Proposition between} Let $\overrightarrow{AB}$ and $\overrightarrow{AC}$ be opposite rays, and points $X \in \text{Int}\overrightarrow{AB}$, $Y \in \text{Int}\overrightarrow{AC} $ with $AX + AY \leq \omega$, then $ X\text{-}A\text{-}Y$
        \item \textbf{Proposition Noncollinear}: If $A,B,C$ are three noncollinear points (not all on the same line), then $AB, AC,BC$ all less than $\omega$.

    \end{itemize}

    \pagebreak 
    \unsect{Part 2}
    \bigbreak \noindent 
    \subsection{Axioms}
    \begin{itemize}
        \item \textbf{Measure axioms}:
            \begin{enumerate}
                \item [M1]: For all coterminal rays $p,q$, $0 \leq pq \leq 180$
                \item [M2]: $pq = 0 \iff p=q$
                \item [M3]: $pq = qp$
                \item [M4]: $pq = 180 \iff q=p^{\prime} $
            \end{enumerate}
                \item \textbf{Betweenness of rays axiom (Ax.BR)}: If $a,b,c$ are distinct, coterminal rays, and if $ab+bc \leq 180$, then there exists a betweenness relation among $a,b,c$
                    \bigbreak \noindent 
                    Thus, if no betweenness relation exists, then
                    \begin{align*}
                        ab + bc > 180 \\
                        ac + cb > 180 \\
                        ba + ac > 180
                    \end{align*}
                \item \textbf{Quadrichotomy of Rays Axiom (Ax.QR)}: If $a,b,c,x$ are distinct, coterminal rays, and if $ a\text{-}b\text{-}c$, then at least one of the following must hold
                    \begin{align*}
                        x\text{-}a\text{-}b \quad a\text{-}x\text{-}b \quad b\text{-}x\text{-}c \quad b\text{-}c\text{-}x
                    \end{align*}
                    \bigbreak \noindent 
                    So, Ax.QR says that whenever $ a\text{-}b\text{-}c$ (say in pencil $P$), then any other ray in $P$ is in either fan $\overrightarrow{ba}$ or fan $\overrightarrow{bc} $ (so $P = \overrightarrow{ba} \cup \overrightarrow{bc} $)
                \item \textbf{Real fan axiom (Ax.RF)}: For any fan $\overrightarrow{ab} $ and for any real number $t$ with $ 0 \leq t \leq 180$, there is a ray $r$ in $\overrightarrow{ab} $ with $ar = t $
                    \bigbreak \noindent 
                    Ax.RF says every real number from 0 to 180 produces at least one ray in the fan
                    \bigbreak \noindent 
                    \textbf{Note:} Ax.RF is one version of what is sometimes called the \textbf{Protractor Axiom}
                \item \textbf{Compatibility Axiom (Ax.C)}: Let $A,B,C$ be points on line $m$, and $X$ a point not on $m$. If $ A\text{-}B\text{-}C$, then $ \overrightarrow{XA}\text{-}\overrightarrow{XB}\text{-}\overrightarrow{XC} $
        \item \textbf{Side-angle-side axiom (Ax.SAS)}: If under the correspondence $ABC \leftrightarrow XYZ$ between the vertices of $ \triangle ABC$ and those of $ \triangle XYZ$, two sides of $ \triangle ABC$ are congruent to the corresponding two sides of $\triangle XYZ$, and the angle included between these two sides of $ \triangle ABC$ is congruent to the corresponding angle of $\triangle XYZ$, then $\triangle ABC \cong \triangle XYZ$

    \end{itemize}

    \pagebreak 
    \subsection{Definitions}
    \begin{itemize}
        \item \textbf{definition: \textit{Coterminal rays}}: Rays with the same endpoint
        \item \textbf{Definition: \textit{Angle}}: $ab = a \cup b $, where $a,b$ are coterminal rays
        \item \textbf{Definition: \textit{Pencil of rays at point $A$}}: The set of all rays with endpoint $A$: denote by $P_{A}$ or just $P$
            \bigbreak \noindent 
            When $\omega < \infty$, each ray $h = \overrightarrow{AB} = \overrightarrow{A^{*}B}$, so $P_{A} = P_{A^{*}} $. $h^{\prime} $ is the opposite ray to $h$, as before
        \item \textbf{Undefined Term \textit{Angle distance function, or angle measure}}: A function $\mu$ from all pairs $(p,q) $ of coterminal rays to $\mathbb{R}$
            \bigbreak \noindent 
            We abbreviate the angular distance between rays $p,q$, or the angle measure of the angle $pq$, $\mu(p,q)$ as $pq$ 
        \item \textbf{Angular distance in $\mathbb{E}$, $\hat{\mathbb{E}}$, $\mathbb{M} $}: The usual measure in degrees (0 to 180)
            \begin{align*}
                pq = \cos^{-1}{\left(\frac{1+mn}{\sqrt{1+m^{2}}\sqrt{1+n^{2}}}\right)}
            \end{align*}
        \item \textbf{Angular distance in $\mathbb{H}$}:
            \begin{align*}
                \mu_{\mathbb{H}}(p,q) = \cos^{-1}{\left(\frac{1+mn-bc}{\sqrt{1+m^{2}-b^{2}}\sqrt{1+n^{2}-c^{2}}}\right)} 
            \end{align*}
        \item \textbf{Definition \textit{(betweenness for rays)}}: Ray $b$ lies \textbf{between} rays $a$ and $c$ ($ a\text{-}b\text{-}c $) provided that
            \begin{enumerate}[label=(\alph*)]
                \item $a,b,c$ are different, coterminal
                \item $ab + bc = ac $
            \end{enumerate}

        \item \textbf{Definition \textit{(Wedge, fan)}}: Let $p,q$ be coterminal rays with $0<pq<180$.
            \begin{itemize}
                \item \textbf{Wedge $\overline{pq} = \{p,q\} \cup \{r: p\text{-}r\text{-}q\}$}
                \item \textbf{Fan $\overrightarrow{pq} = \{p,q\} \cup \{r: p\text{-}r\text{-}q\} \cup \{r: p\text{-}q\text{-}r\}$}
            \end{itemize}
        \item \textbf{Definition \textit{(quad betweenness)}}: $ a\text{-}b\text{-}c\text{-}d $ means that all four of 
            \begin{align*}
                a\text{-}b\text{-}c \quad a\text{-}b\text{-}d \quad a\text{-}c\text{-}d \quad b\text{-}c\text{-}d
            \end{align*}
            are true
        \item \textbf{Notation and terminology}: Recall that $\hcancel{pq}$ means $p \cup q$, then union of the rays. Measure of $\hcancel{pq} $ means the angular distance $pq$
            \bigbreak \noindent 
            Suppose $p = \overrightarrow{BA}$, $ q = \overrightarrow{BC}$. Then, write
            \begin{align*}
                \hcancel{pq} = \underline{\angle ABC} = \underline{\angle CBA}
            \end{align*}
            Or just $\underline{\angle B}$ when clear, and
            \begin{align*}
                pq = \angle ABC = \angle CBA
            \end{align*}
            or just $\angle B$.
        \item \textbf{Definition}: 
            \begin{itemize}
                \item \textbf{Zero angle:} $\underline{pq}$ is a \textbf{zero angle} if $pq = 0 $ ($\iff p = q$)
                \item \textbf{Straight angle}: If $pq = 180 (\iff p = q^{\prime}) $
                \item \textbf{Proper angle:} if $0 < pq < 180 $
                \item \textbf{acute angle}: if $ 0 < pq < 90$
                \item \textbf{right angle}: if $ pq = 90$
                \item \textbf{obtuse angle}: if $ 90 < pq < 180$
            \end{itemize}
        \item \textbf{Definition}: The ray $b$ from the midpoint proposition is called the \textbf{bisector} of angle $\underline{pq}$ 
        \item \textbf{Definition: Congruence}: Two segments $\overline{AB}$ and $ \overline{XY}$ are \textbf{congruent} $(\cong)$ if they have the same length: $\overline{AB} \cong \overline{XY} $ means $AB = XY$
            \bigbreak \noindent 
            Two angles $\angle CAB$ and $\angle ZXY$ are congruent if they have the same angle measure
            \bigbreak \noindent 
            Two triangles $\triangle ABC$ and $\triangle ZXY$ are congruent under the correspondence $A\leftrightarrow X$, $B \leftrightarrow Y, C\leftrightarrow Z$ (Write as $ABC \leftrightarrow XYZ $) if 
            \begin{align*}
                \overline{AB} \cong \overline{XY},\quad \overline{BC} \cong\overline{YZ} ,\quad \overline{AC} \cong \overline{XZ}
            .\end{align*}
            and 
            \begin{align*}
                \angle ABC \cong \angle XYZ, \quad \angle CAB \cong \angle ZXY,\quad \angle BCA \cong \angle YZX
            .\end{align*}
            denote this by $\triangle ABC \cong \triangle XYZ $
        \item \textbf{Definition: Absolute plane}: An \textbf{absolute plane} $\mathbb{P}$ is a set of points $\mathbb{P}$ with lines, distance, and angular distance (all undefined terms), such that all 21 axioms are true. The three planes above are absolute planes
        \item \textbf{Definition: types of triangles}
            \begin{itemize}
                \item A triangle is \textbf{isosceles} if two sides have the same length
                \item \textbf{Equilateral} if all three sides have the same length
                \item \textbf{Equiangular} if all three angles have the same measure  
            \end{itemize}
            \textbf{Note:} A triangle can be called \textbf{scalene} if all all three sides have different lengths and all three angles have different measures



    \end{itemize}

    \pagebreak 
    \subsection{Theorems}
    \begin{itemize}
        \item \textbf{Theorem 11.1 \textit{(symmetry of betweenness)}}: $ a\text{-}b\text{-}c \iff c\text{-}b\text{-}a$
        \item \textbf{Theorem 11.3 \textit{UMT}}: If $ a\text{-}b\text{-}c$, then $ b\text{-}a\text{-}c$ and $ a\text{-}c\text{-}b$ are false.
        \item \textbf{Theorem 11.2 (non-triviality)}: For any ray $p$ there is a coterminal ray $q$ so that $0 < pq < 180$
        \item \textbf{Theorem \textit{(Triangle inequality for rays)}}: If $a,b,c$ are three distinct, coterminal rays, then $ab + bc \geq ac$
        \item \textbf{Theorem 11.5 \textit{(Rule of insertion for rays)}}:
            \begin{enumerate}[label=(\alph*)]
                \item If $ a\text{-}b\text{-}c$ and $ a\text{-}r\text{-}b$, then $ a\text{-}r\text{-}b\text{-}c $
                \item If $ a\text{-}b\text{-}c $ and $ b\text{-}r\text{-}c $, then $ a\text{-}b\text{-}r\text{-}c $
            \end{enumerate}
        \item \textbf{Theorem 11.6 (Unique angular distance for fans)}: For any fan $\overrightarrow{pq}$ and any real number $t$ with $0 \leq t \leq 180$, there is a unique ray $r$ in $\overrightarrow{pq}$ with $pr = t$. $r$ is in $\overline{pq} $ if and only if $t \leq  pq$
        \item \textbf{Theorem 11.8}: If ray $a$ lies in pencil $P$, then $ a\text{-}r\text{-}a^{\prime} $ for every other ray $r$ in $P$
        \item \textbf{Theorem 11.9 (Almost uniqueness of quadrichotomy for rays)}: Suppose that $a,b,c,r$ are distinct rays in a pencil $P$, and that $ a\text{-}b\text{-}c$. Then, \textbf{exactly} one of 
            \begin{align*}
                r\text{-}a\text{-}b \quad a\text{-}r\text{-}b \quad b\text{-}r\text{-}c \quad b\text{-}c\text{-}r
            \end{align*}
            With the exception that both $ r\text{-}a\text{-}b $ and $ b\text{-}c\text{-}r$ are true when $r = b^{\prime} $
        \item \textbf{Theorem 11.10 (Opposite fan theorem)}: Let $p,q,r$ be rays in pencil $P$ such that $ q\text{-}p\text{-}r$. Then, $ \overrightarrow{pq} \cup \overrightarrow{pr} = P$, and $ \overrightarrow{pq} \cap \overrightarrow{pr} = \{p,p^{\prime}\} $
        \item \textbf{Corollary 11.11}: If $p,q$ are rays in pencil $P$ with $0 < pq < 180$, then $P = \overrightarrow{pq} \cup \overrightarrow{pq^{\prime}} $ and $\overrightarrow{pq} \cap \overrightarrow{pq^{\prime}} = \{p,p^{\prime}\}$
        \item \textbf{Theorem 12.2 (Fan: halfplane)}: Let $H,K$ be opposite halfplanes with edge line $\ell$, point $B \in H$. Let $X,A$ be points on $\ell$ with $0 < AX < \omega$. Let $h = \overrightarrow{XA}$, $k = \overrightarrow{XB}$. Then, $H $ consists of all points on all rays of the fan $\overrightarrow{hk}$, except for the points of $\ell$
            \bigbreak \noindent 
            That is, $P \in H \iff P \in j^{0}$, where $j^{0}$ is the interior of some ray $j \in \overrightarrow{hk}$, $j \ne h$ or $h^{\prime}$
        \item \textbf{Corollary 12.3}: Let $z$ by any number with $0 < z < 180$. For any ray $\overrightarrow{AB}$ there are exactly two rays $h,k$ in $P_{A}$ such that $\overrightarrow{AB}h = z = \overrightarrow{AB}k$. Furthermore, $h^{0}$ and $k^{0}$ lie in opposite halfplanes with edge $\overleftrightarrow{AB} $
        \item \textbf{Theorem 12.4 (The Crossbar Theorem)}: If $\underline{hk}$ is a proper angle with vertex (common endpoint) $X$, if $A \in h^{0}$ (so $h = \overrightarrow{XA}$), $C \in k^{0}$ (so $k = \overrightarrow{XC}$), and $ h\text{-}j\text{-}k$, then there is an interior point $B$ of $j$ with $ A\text{-}B\text{-}C$
        \item \textbf{Theorem 13.1 (ASA)}: If under the correspondence $ABC \leftrightarrow XYZ$, two angles and the included side of $\triangle ABC$ are congruent, respectively, to the corresponding two angles and included side of $\triangle XYZ$, then $\triangle ABC \cong \triangle XYZ $
        \item \textbf{Theorem 13.2 (pons asinorum ("Bride of asses"))} In any $\triangle ABC$, 
            $$ AB = AC \iff \angle ACB = \angle ABC $$
        \item \textbf{Corollary 13.3}: A triangle is equilateral if and only if it is equiangular
        \item \textbf{Theorem 13.4 (SSS)}: If in $\triangle ABC$ and $\triangle XYZ$, $\overline{AB} \cong \overline{XY}$, $ \overline{BC} \cong \overline{YZ}$ and $\overline{CA} \cong \overline{ZX}$, then 
            \begin{align*}
                \triangle ABC \cong \triangle XYZ
            .\end{align*}

    \end{itemize}

    \pagebreak 
    \subsection{Propositions}
    \begin{itemize}
        \item \textbf{Proposition 11.14}
            \begin{enumerate}[label=(\alph*)]
                \item If $\omega < \infty$, then $\angle ABC = \angle AB^{*}C$
                \item If $P \in \overrightarrow{BA}^{0} $ and $Q \in \overrightarrow{BC}^{0} $, then $\angle ABC = \angle PBQ$
                    \bigbreak \noindent 
            \end{enumerate}
        \item \textbf{Proposition 11.15 (Midpoint)}: If $\underline{pq}$ is a proper angle, then there is exactly one ray $b$ in the wedge $\overline{pq}$ so that $pb  = \frac{1}{2}pq $
      
    \end{itemize}



    \pagebreak 
    \subsection{Duals of results from chapters 8 and 9}
    \bigbreak \noindent 
    \subsubsection{Theorems (14)}
    \begin{itemize}
        \item \textbf{Theorem 8.1D}: The set of angle measures $\mathbb{D} = [0,180]$
        \item \textbf{Theorem 8.2D}: All wedges, fans, pencils have infinitely many rays
        \item \textbf{Theorem 8.3D}: Let $x\ne y$ be distinct from $a$ on fan $\overrightarrow{ab}$. Then, exactly one of 
            \begin{align*}
                a\text{-}x\text{-}y \quad \text{ or } \quad a\text{-}y\text{-}x
            .\end{align*}
        \item \textbf{Theorem 8.4D}: Let $\overrightarrow{ab}$ be a fan. If $c \in \overrightarrow{ab}$, $0 < c < 180$, then $\overrightarrow{ab} = \overrightarrow{ac}$
        \item \textbf{Theorem 8.6D}: Stated in theorem 11.6
        \item \textbf{Theorem 9.1D}: Let ray $a$ be in pencil $P$, there exists a unique fan $a^{\prime} \in P$ such that $aa^{\prime} = 180$. For all other rays $x\in P$, $ a\text{-}x\text{-}a^{\prime} $
        \item \textbf{Theorem 9.2D}: Stated in theorem 11.8
        \item \textbf{Theorem 9.4D}: If $ap = 180$ in some fan $h$, then $p = a^{\prime}$.
        \item \textbf{Theorem 9.6D}: Stated in theorem 11.9
        \item \textbf{Theorem 9.7D}: Each fan has a unique opposite fan.
        \item \textbf{Theorem 9.8D}: Let rays $a,b \in P$, if $0 < ab < 180$, then fan $\overrightarrow{ab}^{\prime} = \overrightarrow{ab^{\prime}} $
        \item \textbf{Theorem 9.9D}: Let rays $a,b \in P$, if $0 < ab < 180$, then $P = \overline{ab} \cup \overline{ab^{\prime}} \cup \overline{ba^{\prime}} \cup \overline{b^{\prime}a^{\prime}}$, where the interiors of these wedges are disjoint.
        \item \textbf{Theorem 9.10D}: Let rays $a,b \in P$, if $0 < ab < 180$, and $c$ is some other ray in $P$, then there exists no betweenness relation among $a,b,c$ if and only if $c \in \overline{a^{\prime}b^{\prime}} $
    \end{itemize}

    \bigbreak \noindent 
    \subsubsection{Propositions}
    \begin{itemize}
        \item \textbf{Proposition 8.11D}: Let $a,b \in P$, $0 < ab < 180$, there exists $c\in P$ such that $ c\text{-}a\text{-}b$, $cb < 180$
        \item \textbf{Proposition 8.5D}: A fan has at most two terminal rays 
        \item \textbf{Proposition 8.7D}: Let $\overline{ab}$ be a wedge, for all $x,y \in \overline{ab}$, $xy \leq ab$, if $xy = ab$, then $\{x,y\} = \{a,b\}$
        \item \textbf{Proposition 8.8D}: If $\overline{ab}  = \overline{cd}$, then $\{a,b\} = \{c,d\}$
        \item \textbf{Proposition 8.9D}: Stated in proposition 11.15
        \item \textbf{Proposition 9.3D}: Let $a,b \in P$ such that $0 < ab < 180$. Then,
            \begin{itemize}
                \item Fan $\overrightarrow{ab} = \overline{ab} \cup \overline{ba^{\prime}}$, with $\overline{ab} \cap \overline{ba^{\prime}}  = \varnothing$
                \item Fan $\overrightarrow{ab} = \overrightarrow{a^{\prime}b} $
            \end{itemize}
    \end{itemize}







    
\end{document}
