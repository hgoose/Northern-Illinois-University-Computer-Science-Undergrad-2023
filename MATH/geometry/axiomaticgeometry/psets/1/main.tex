 \documentclass{report}
 
 \input{~/dev/latex/template/preamble.tex}
 \input{~/dev/latex/template/macros.tex}
 
 \title{\Huge{}}
 \author{\huge{Nathan Warner}}
 \date{\huge{}}
 \fancyhf{}
 \rhead{}
 \fancyhead[R]{\itshape Warner} % Left header: Section name
 \fancyhead[L]{\itshape\leftmark}  % Right header: Page number
 \cfoot{\thepage}
 \renewcommand{\headrulewidth}{0pt} % Optional: Removes the header line
 %\pagestyle{fancy}
 %\fancyhf{}
 %\lhead{Warner \thepage}
 %\rhead{}
 % \lhead{\leftmark}
 %\cfoot{\thepage}
 %\setborder
 % \usepackage[default]{sourcecodepro}
 % \usepackage[T1]{fontenc}
 
 % Change the title
 \hypersetup{
     pdftitle={}
 }

 \geometry{
  left=1.5in,
  right=1.5in,
  top=1in,
  bottom=1in
}
 
 \begin{document}
     % \maketitle
     %     \begin{titlepage}
     %    \begin{center}
     %        \vspace*{1cm}
     % 
     %        \textbf{}
     % 
     %        \vspace{0.5cm}
     %         
     %             
     %        \vspace{1.5cm}
     % 
     %        \textbf{Nathan Warner}
     % 
     %        \vfill
     %             
     %             
     %        \vspace{0.8cm}
     %      
     %        \includegraphics[width=0.4\textwidth]{~/niu/seal.png}
     %             
     %        Computer Science \\
     %        Northern Illinois University\\
     %        United States\\
     %        
     %             
     %    \end{center}
     % \end{titlepage}
     % \tableofcontents
    \pagebreak \bigbreak \noindent
    Nate Warner \ \quad \quad \quad \quad \quad \quad \quad \quad \quad \quad \quad \quad  MATH 353 \quad  \quad \quad \quad \quad \quad \quad \quad \quad \ \ \quad \quad Spring 2025
    \begin{center}
        \textbf{Problem set 1 - Due: Fri, Jan 24}
    \end{center}
    \bigbreak \noindent 
    \begin{mdframed}
        1. \begin{enumerate}[label=(\alph*)]
            \item In $E$, find the distance between $A\left(-\frac{1}{3}, 0\right)$ and $B\left(0, \frac{1}{3}\right)$.
            \item In $M$, find the distance between $A\left(-\frac{1}{3}, 0\right)$ and $B\left(0, \frac{1}{3}\right)$.
            \item In $H$, find the distance between $A\left(-\frac{1}{3}, 0\right)$ and $B\left(0, \frac{1}{3}\right)$.
            \item In $S$ (radius $r = 1$), find the distance between $C\left(0, 0, 1\right)$ and $D\left(0, -\frac{1}{\sqrt{2}}, \frac{1}{\sqrt{2}}\right)$.
            \item In $S$ (radius $r = \frac{1}{2}$), find the distance between 
                $P\left(\frac{1}{4}, \frac{\sqrt{2}}{4}, -\frac{1}{4}\right)$ and $Q\left(\frac{1}{6}, -\frac{1}{3}, \frac{1}{3}\right)$.
            \item In $G$, find the distance between $A(-2, -3)$ and $B(4, 6)$.
        \end{enumerate}
    \end{mdframed}
    \bigbreak \noindent 
    \begin{remark}
        The \textit{Euclidean distance} $e(AB)$ between $A$ and $B$ satisfies 
        \begin{align*}
            e(AB) = \sqrt{(x_{2} - x_{1})^{2} + (y_{2} - y_{1})^{2}}
        \end{align*}
        Let $\mathbb{M}$ denote the Minkowski plane. If $A(x_{1}, y_{1})$ and $B(x_{2}, y_{2}) $ are on the line $y=mx+b$, then the \textit{Minkowski distance} $d_{\mathbb{M}}(AB) = \abs{x_{1} - x_{2}}(1+\abs{m})$
        \bigbreak \noindent 
        Let $\mathbb{S}(r)$ denote the spherical plane, which describes the surface of a sphere with radius $r$. For any two points $A,B$ that lie on this plane, the distance between them is the arc length given $d_{\mathbb{S}} = r\theta$. An explicit formula using the points coordinates is $d_{\mathbb{S}} = r\cos^{-1}{\left(\frac{ax+by+cz}{r^{2}}\right)}$
        \bigbreak \noindent 
        Let $\mathbb{G}$ denote the gap plane. For points $A,B$ in $\mathbb{G}$, we define $d_{\mathbb{G}}(AB)$ as
        \begin{align*}
            d_{\mathbb{G}}(AB) = \begin{cases}
                e(AB) & \text{ for $A,B$ on the same side of the gap} \\     
                e(AB) - e(CD) & \text{ for $A,B$ on the opposite sides of the gap} 
            \end{cases}
        \end{align*}
        Let $\mathbb{H}$ denote the Hyperbolic plane. For two points $A,B$ that lie on this plane, then $M,N$ are the points where the chord $AB $ meets the unit circle. The Hyperbolic distance $d_{\mathbb{H}}(AB)$ is given by
        \begin{align*}
            d_{\mathbb{H}} &= \ln{\left(\frac{e(AN)e(BM)}{e(AM)e(BN)}\right)}
        \end{align*}
        Where $e(AN),e(BM),...$ denotes the Euclidean distance.
        \bigbreak \noindent 
        \textbf{Note:} $d_{H}(AA)$ is defined to be one. That is, $d_{\mathbb{H}}(AA) = 1$.
        \bigbreak \noindent 
        $\qed$
    \end{remark}
    \bigbreak \noindent 
    a.) If points on the Euclidean plane $A,B$ are given by coordinates $\left(-\frac{1}{3},0\right)$, and $\left(0,\frac{1}{3}\right) $ respectively, then the Euclidean distance $e(AB)$ is given by
    \begin{align*}
        e(AB) &= \sqrt{\left(0-\left(-\frac{1}{3}\right)\right)^{2} + \left(\frac{1}{3}-0\right)^{2}} \\
              &= \sqrt{\left(\frac{1}{3}\right)^{2} + \left(\frac{1}{3}\right)^{2}} = \sqrt{\frac{2}{9}} = \frac{\sqrt{2}}{3}
    \end{align*}
    b.) If points on the Minkowski plane $A,B$ are given by coordinates $\left(-\frac{1}{3},0\right)$, and $\left(0,\frac{1}{3}\right) $ respectively, then the Minkowski distance $d_{\mathbb{M}}$ is given by
    \begin{align*}
        d_{\mathbb{M}} &= \bigg\lvert 0-\left(-\frac{1}{3}\right) \bigg\rvert + \bigg\lvert \frac{1}{3}-0 \bigg\rvert \\
                       &= \bigg\lvert \frac{1}{3} \bigg\rvert + \bigg\lvert \frac{1}{3} \bigg\rvert = \frac{1}{3} + \frac{1}{3} = \frac{2}{3}
    \end{align*}
    c.) If points on the Hyperbolic plane $A,B$ are given by coordinates $\left(-\frac{1}{3},0\right)$, and $\left(0,\frac{1}{3}\right) $ respectively, then the Hyperbolic distance $d_{\mathbb{H}}$ is given by
    \begin{align*}
        d_{\mathbb{H}} &= \ln{\left(\frac{e(AN)e(BM)}{e(AM)e(BN)}\right)}
    \end{align*}
    Thus, we first find points $M,N$. 
    \bigbreak \noindent 
    \begin{figure}[ht]
        \centering
        \incfig{hyper1}
        \label{fig:hyper1}
    \end{figure}
    \bigbreak \noindent 
    If the line $\ell$ that passes through $A,B$ has slope $m= \frac{\frac{1}{3}}{\frac{1}{3}} =1 $. Then, the equation of the line is given by
    \begin{align*}
        y-0 &= 1\left(x-\left(-\frac{1}{3}\right)\right) \\
        \implies y &= x  + \frac{1}{3}
    \end{align*}
    Since the circle is given by $x^{2} + y^{2} = 1$, or $y = \pm \sqrt{ 1- x^{2}}$. The line $\ell$ meets this circle at 
    \begin{align*}
        x + \frac{1}{3} &= \sqrt{1-x^{2}} \\
        \implies 3x + 1 &= 3\sqrt{1-x^{2}} \\
        \implies (3x+1)^{2} &= 9(1-x^{2}) = 9-9x^{2} \\
        \implies 9x^{2} + 6x + 1 - 9 + 9x^{2} &= 0 \\
        \implies 18x^{2} + 6x - 8 &=0
    \end{align*}
    Thus,
    \begin{align*}
        x &= \frac{-6 \pm \sqrt{6^{2} - 4(18)(-8)}}{2(18)} \\
          &= -\frac{1}{6}\pm \frac{\sqrt{17}}{6}
    \end{align*}
    Let $\xi(x) = x+\frac{1}{3}$. Then, $M = \left(-\frac{1}{6} - \frac{\sqrt{17}}{6}, \xi\left(-\frac{1}{6} - \frac{\sqrt{17}}{6}\right)\right)$, and $N = \left(-\frac{1}{6} + \frac{\sqrt{17}}{6}, \xi\left(-\frac{1}{6} + \frac{\sqrt{17}}{6}\right)\right) $. Since
    \begin{align*}
        \xi\left(-\frac{1}{6} - \frac{\sqrt{17}}{6}\right) &= \left(-\frac{1}{6} - \frac{\sqrt{17}}{6}\right) + \frac{1}{3} = \frac{1}{6} - \frac{\sqrt{17}}{6} \\
        \xi\left(-\frac{1}{6} + \frac{\sqrt{17}}{6}\right) &= \left(-\frac{1}{6} + \frac{\sqrt{17}}{6}\right) + \frac{1}{3} = \frac{1}{6} + \frac{\sqrt{17}}{6} \\
    \end{align*}
    $M = \left(-\frac{1}{6} - \frac{\sqrt{17}}{6}, \frac{1}{6} - \frac{\sqrt{17}}{6}\right)$, and $N = \left(-\frac{1}{6} + \frac{\sqrt{17}}{6}, \frac{1}{6} + \frac{\sqrt{17}}{6}\right) $. We can now find the Euclidean distances required to compute $d_{\mathbb{H}}(AB)$. We use the Euclidean distance formula $e(CD) = \bigg\lvert x_{1} - x_{2} \bigg\rvert \sqrt{1 + m^{2}}$. Since $m = 1$, let $\lambda = \sqrt{1 + m^{2}} = \sqrt{1 + 1^{2}} = \sqrt{2} $, then $e(CD) = \bigg\lvert x_{1} - x_{2} \bigg\rvert \cdot \sqrt{2} $, for all $C(x_{1},y_{1}), D(x_{2}, y_{2})$ on the line $\ell$ through $A,B$.
    \begin{align*}
        e(AN) &= \bigg\lvert -\frac{1}{3}-\left(-\frac{1}{6}+\frac{\sqrt{17}}{6}\right) \bigg\rvert \sqrt{2} = \frac{\sqrt{2} + \sqrt{34}}{6}\\ 
        e(BM) &= \bigg\lvert 0-\left(-\frac{1}{6}-\frac{\sqrt{17}}{6}\right) \bigg\rvert \sqrt{2} = \frac{\sqrt{2} + \sqrt{34}}{6}\\
        e(AM) &= \bigg\lvert -\frac{1}{3} - \left(-\frac{1}{6}-\frac{\sqrt{17}}{6}\right) \bigg\rvert \sqrt{2} = \frac{-\sqrt{2} + \sqrt{34}}{6}\\
        e(BN) &= \bigg\lvert 0 - \left(-\frac{1}{6} + \frac{\sqrt{17}}{6}\right) \bigg\rvert \sqrt{2} = \frac{-\sqrt{2} + \sqrt{34}}{6}
    \end{align*}
    Thus,
    \begin{align*}
        d_{\mathbb{H}} &= \ln{\left(\frac{\left(\frac{\sqrt{2} + \sqrt{34}}{6}\right)^{2}}{\left(\frac{-\sqrt{2} + \sqrt{34}}{6}\right)^{2}}\right)} \approx 0.9899
    \end{align*}
    d.) If points $C,D$ on the spherical plane $\mathbb{S}(1)$ are given by coordinates $\left(0,0,1\right)$, and $\left(0,-\frac{1}{\sqrt{2}}, \frac{1}{\sqrt{2}}\right) $ respectively, then the distance $d_{\mathbb{S}}$ is given by
    \begin{align*}
        d_{\mathbb{S}} &= r\cos^{-1}{\left(\frac{c_{1}d_{1} + c_{2}d_{2} + c_{3}d_{3}}{r^{2}}\right)} \\
                       &= 1\cos^{-1}{\left(\frac{0(0) +0\left(-\frac{1}{\sqrt{2}}\right) + 1\left(\frac{1}{\sqrt{2}}\right)}{1^{2}}\right)} \\
                       &= cos^{-1}\left(\frac{1}{\sqrt{2}}\right) = \frac{\pi}{4}
    \end{align*}
    e.) Next, consider $\mathbb{S}\left(\frac{1}{2}\right)$, with $P\left(\frac{1}{4}, \frac{\sqrt{2}}{4}, -\frac{1}{4}\right)$, and $Q\left(\frac{1}{6}, -\frac{1}{3}, \frac{1}{3}\right) $. Then,
    \begin{align*}
        d_{\mathbb{S}} &= \frac{1}{2}\cos^{-1}{\left(\frac{\frac{1}{4}\left(\frac{1}{6}\right) + \frac{\sqrt{2}}{4}\left(-\frac{1}{3}\right) -\frac{1}{4}\left(\frac{1}{3}\right) }{\left(\frac{1}{2}\right)}\right)^{2}} \\
        &= \frac{1}{2}\cos^{-1}{\left(\frac{ \frac{1}{24} - \frac{\sqrt{2}}{24} -\frac{1}{12}  }{\frac{1}{4}}\right)} \approx 1.1314
    \end{align*}
    f.) Let $\mathbb{G}$ denote the gapped plane. If points $A(-2,-3), B(4,6)$ lie on $\mathbb{G}$, then their distance $d_{\mathbb{G}}$ is given by
    \begin{align*}
        \begin{cases}
            e(AB) &\text{ if $A,B$ lie on the same side}      \\
            e(AB) - e(CD) & \text{ otherwise}
        \end{cases}
    \end{align*}
    If the line $\ell$ that passes through $A,B$ has slope $m= \frac{6+3}{4+2} = \frac{3}{2} $, then the equation of the line is given by
    \begin{align*}
        y  +3 &= \frac{3}{2}(x+2) \\
        \implies  y &= \frac{3}{2}x
    \end{align*}
    When $x=0$, $y=0$. When $x=1, y=\frac{3}{2}$. Thus, $C = (0,0)$, and $D = \left(1,\frac{3}{2}\right) $. Thus,
    \begin{align*}
        d_{\mathbb{G}} &= e(AB) - e(CD) \\
                       &= \bigg\lvert 4+2 \bigg\rvert\sqrt{1 + \left(\frac{3}{2}\right)^{2}} - \bigg\lvert 1-0 \bigg\rvert\sqrt{1+\left(\frac{3}{2}\right)^{2}} \\
                       &= 6\sqrt{1+\frac{9}{4}} - \sqrt{1+\frac{9}{4}} \\
                       &= 6\sqrt{\frac{13}{4}} - \sqrt{\frac{13}{4}} \\
                       &= \frac{6\sqrt{13}}{2} - \frac{\sqrt{13}}{2} \approx 9.0139
    \end{align*}





    

    \begin{mdframed}
        \textbf{2.} Find two points $A, B$ in $H$ such that $d_H(AB) > 13$. Show the calculation that justifies your answer.
    \end{mdframed}
    \bigbreak \noindent 
    Let $A,B$ lie on the $x$-axis. Then $M = (-1,0)$, and $N = (1,0)$. Thus, we require $A(\alpha,0 ), B(\beta,0)$ such that
    \begin{align*}
        d_{\mathbb{H}}(AB) &= \ln{\left(\frac{e(AN)e(BM)}{e(AM)e(BN)}\right)} > 13
    \end{align*}
    Let's try $A(-0.99,0)$, $B(0.99,0)$. Since the chord of the circle passes through $(-1,0)$, and $(1,0)$, it has slope $m=0$, and equation $y=0$. Thus, $e(PQ) = \abs{q_{1}-p_{1}} \sqrt{1+0^{2}} = \abs{q_{1} - p_{1}}$ for all points $P(p_{1}, p_{2}), Q(q_{1}, q_{2})$
    \begin{align*}
        e(AN) &= \left\lvert 1+0.99 \right\rvert = 1.99 \\
        e(BM) &= \abs{-1-0.99} = 1.99 \\
        e(AM) &= \abs{-1+0.99} = 0.01 \\
        e(BN) &+ \abs{1-0.99} = 0.01
    \end{align*}
    Thus,
    \begin{align*}
        d_{\mathbb{H}} = \ln{\left(\frac{1.99^{2}}{0.01^{2}}\right)} \approx 10.59
    \end{align*}
    Not quite, let's instead try $A(-0.9999,0), B(0.9999,0)$. Which has distance
    \begin{align*}
        d_{\mathbb{H}} &= \ln{\frac{1.9999^{2}}{0.0001^{2}}} \approx 19.81 > 13
    \end{align*}
    Thus, $A = (-0.9999,0 )$, and $B = (0.9999,0 )$


    \begin{mdframed}
        \textbf{3.} Prove Proposition 2.1.
    \end{mdframed}
    \bigbreak \noindent 
    \textbf{Proposition 1.1} If $A(x_{1}, y_{1})$ and $B(x_{2}, y_{2}) $ are on the line $y=mx+b$, then $e(AB) = \abs{x_{1} - x_{2}}\sqrt{m^{2} + 1}$
    \bigbreak \noindent 
    \textbf{\textit{Proof.}} Assume $A(x_{1}, y_{1})$ and $B(x_{2}, y_{2}) $ are on the line $y=mx+b$, and $e(AB) = \sqrt{(x_{2} - x_{1})^{2} + (y_{2} - y_{1})^{2}}$. Observe that the slope $m$ of the line is given by
    \begin{align*}
        m = \frac{y_{2} - y_{1}}{x_{2} - x_{1}}
    \end{align*}
    Which implies 
    \begin{align*}
        y_{2} - y_{1} = m(x_{2} - x_{1})
    \end{align*}
    Plugging this expression for $y_{2} -y_{1}$ into $e(AB)$ yields
    \begin{align*}
        e(AB) &= \sqrt{(x_{2} - x_{1})^{2} + (m(x_{2}-x_{1}))^{2}} \\
              &= \sqrt{(x_{2} - x_{1})^{2} + (m^{2}(x_{2}-x_{1})^{2})} \\
              &= \sqrt{(x_{2}-x_{1})^{2}[1 + m^{2}]} \\
              &= \sqrt{(x_{2} - x_{1})^{2}} \cdot \sqrt{m^{2} + 1} \\
              &= \abs{x_{2} - x_{1}} \sqrt{m^{2} + 1}  \\
              &= \abs{-(x_{1} - x_{2})} \sqrt{m^{2} + 1}  \\
              &= \abs{x_{1} - x_{2}} \sqrt{m^{2} + 1}  \\
    \end{align*}
    As desired \hspace*{\fill}$\blacksquare$ 


    \begin{mdframed}
        \textbf{4.} Prove Proposition 2.2.
    \end{mdframed}
    \bigbreak \noindent 
    \textbf{Proposition 1.2} If $A(x_{1}, y_{1})$ and $B(x_{2}, y_{2}) $ are on the line $y=mx+b$, then $d_{\mathbb{M}}(AB) = \abs{x_{1} - x_{2}}(1+\abs{m})$
    \bigbreak \noindent 
    \textbf{\textit{Proof.}} Assume $A(x_{1}, y_{1})$ and $B(x_{2}, y_{2}) $ are on the line $y=mx+b$, and $d_{\mathbb{M}} = \abs{x_2-x_1} + \abs{y_{2} - y_{1}}$. Observe that the slope $m$ of the line is given by
    \begin{align*}
        m = \frac{y_{2}-y_{1}}{x_{2} - x_{1}}
    \end{align*}
    Which implies 
    \begin{align*}
        y_{2} - y_{1} = m(x_{2} - x_{1})
    \end{align*}
    Plugging this expression for $y_{2} -y_{1}$ into $d_{\mathbb{M}}$ yields
    \begin{align*}
        d_{\mathbb{M}} &= \abs{x_{2} - x_{1}} + \abs{y_{2} - y_{1}} \\
                       &= \abs{x_{2} - x_{1}} + \abs{m(x_{2} - x_{1})} \\
                       &= \abs{x_{2} - x_{1}} + \abs{m}\abs{x_{2} - x_{1}} \\
                       &= \abs{x_{2} - x_{1}}(1 + \abs{m}) \\
                       &= \abs{-(x_{1} - x_{2})} (1 + \abs{m}) \\
                       &= \abs{-1}\abs{x_{1} - x_{2}} (1 + \abs{m}) \\
                       &= \abs{x_{1} - x_{2}} (1 + \abs{m}) \\
    \end{align*}
    As desired \hspace*{\fill} $\blacksquare$

    \begin{mdframed}
        \textbf{5.} Let $A(x_1, y_1)$ and $B(x_2, y_2)$ be two points on opposite sides of the gap in $G$ and on the line $l : y = mx + b$. Derive a formula for $d_G(AB)$ in terms of $x_1, x_2$, and $m$.
    \end{mdframed}
    \bigbreak \noindent 
    If $A = (x_{1}, y_{1})$, and $B = (x_{2}, y_{2})$, then the gapped distance $d_{\mathbb{G}}$ is given by
    \begin{align*}
        d_{\mathbb{G}} &= e(AB) - e(CD)
    \end{align*}
    Where $C = (0, b)$, and $D = (1,m + b)$. Using $e(AB) = \abs{x_{1} - x_{2}}\sqrt{1+m^{2}}$, we get
    \begin{align*}
        d_{\mathbb{G}} &= \left\lvert x_{1} - x_{2} \right\rvert\sqrt{1+m^{2}} - \abs{1-0}\sqrt{1+m^{2}} \\
                       &= \abs{x_{1} - x_{2}} \sqrt{1+m^{2}} - \sqrt{1+m^{2}} \\
                       &= \sqrt{1+m^{2}} \left(\left(\abs{x_{1} - x_{2}}\right) - 1\right)
    \end{align*}






 \end{document} % (:
