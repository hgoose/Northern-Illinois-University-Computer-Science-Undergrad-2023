 \documentclass{report}
 
 \input{~/dev/latex/template/preamble.tex}
 \input{~/dev/latex/template/macros.tex}
 
 \title{\Huge{}}
 \author{\huge{Nathan Warner}}
 \date{\huge{}}
 \fancyhf{}
 \rhead{}
 \fancyhead[R]{\itshape Warner} % Left header: Section name
 \fancyhead[L]{\itshape\leftmark}  % Right header: Page number
 \cfoot{}
 \renewcommand{\headrulewidth}{0pt} % Optional: Removes the header line
 %\pagestyle{fancy}
 %\fancyhf{}
 %\lhead{Warner \thepage}
 %\rhead{}
 % \lhead{\leftmark}
 %\cfoot{\thepage}
 %\setborder
 % \usepackage[default]{sourcecodepro}
 % \usepackage[T1]{fontenc}
 
 % Change the title
 \hypersetup{
     pdftitle={}
 }

 \geometry{
  left=1in,
  right=1in,
  top=1in,
  bottom=1in
}
 
 \begin{document}
     % \maketitle
     %     \begin{titlepage}
     %    \begin{center}
     %        \vspace*{1cm}
     % 
     %        \textbf{}
     % 
     %        \vspace{0.5cm}
     %         
     %             
     %        \vspace{1.5cm}
     % 
     %        \textbf{Nathan Warner}
     % 
     %        \vfill
     %             
     %             
     %        \vspace{0.8cm}
     %      
     %        \includegraphics[width=0.4\textwidth]{~/niu/seal.png}
     %             
     %        Computer Science \\
     %        Northern Illinois University\\
     %        United States\\
     %        
     %             
     %    \end{center}
     % \end{titlepage}
     % \tableofcontents
    \pagebreak \bigbreak \noindent
    \begin{mdframed}
        6a. Show that if $\omega = \infty$, $\overrightarrow{AB} \cup \overrightarrow{BA} = \overleftrightarrow{AB} $
    \end{mdframed}
    \bigbreak \noindent 
    \textbf{\textit{Proof.}} Assume $A,B$ are distinct collinear, and $\omega = \infty$. Since $\omega = \infty$, $AB < \omega$. Thus, $ \overrightarrow{AB}$ and $\overrightarrow{BA}$ are well defined. Further, $A,B$ are together in a unique line. Namely, the line $\overleftrightarrow{AB}$.
    \bigbreak \noindent 
    Let $X$ exist on the line $\overleftrightarrow{AB}$. If $X = A$, then $X \in \overrightarrow{AB}$ and $X \in \overrightarrow{BA}$ by definition of $\overrightarrow{AB}$ and $\overrightarrow{BA}$. Similarly, if $X = B$, then $X \in \overrightarrow{AB}$ and $X \in \overrightarrow{BA}$. For the following argument, we can therefore  assume that $X \ne A$ or $B$.
    \bigbreak \noindent 
    Since $\omega = \infty$, it is guaranteed that $AB  +BX \leq \omega = \infty$. Thus, by Ax.BP, there exists a betweenness relation among $A,B,X$, and exactly one of the following must be satisfied
    \begin{align*}
        A\text{-}X\text{-}B \tag{1} \\
        A\text{-}B\text{-}X \tag{2} \\
        B\text{-}A\text{-}X \tag{3}
    \end{align*}
    \bigbreak \noindent 
    We examine these cases separately. If $ A\text{-}X\text{-}B$, then $X\in \overrightarrow{AB}$ and $X\in \overrightarrow{BA}$. If $ A\text{-}B\text{-}X$, then $X\in \overrightarrow{AB}$. Lastly, if $ B\text{-}A\text{-}X$, then $X \in \overrightarrow{BA}$
    \bigbreak \noindent 
    In any case, $X \in \overleftrightarrow{AB}$ implies $X$ is either in $\overrightarrow{AB}$ or $\overrightarrow{BA}$ or both.
    \bigbreak \noindent 
    Therefore, $\overleftrightarrow{AB} = \overrightarrow{AB} \cup \overrightarrow{BA}$ \endpf

    \bigbreak \noindent 
    \begin{mdframed}
        12. Suppose that $A,B,C$ are three distinct, collinear points such that $AC \leq \frac{1}{2}AB$ and $BC \leq \frac{1}{2} AB $. Prove that $ A\text{-}C\text{-}B$ and $AC = BC = \frac{1}{2}AB$
    \end{mdframed}
    \bigbreak \noindent 
    \textbf{\textit{Proof.}} Assume $ A,B,C$ are three distinct, collinear points such that $AC \leq \frac{1}{2}AB$, and $BC \leq \frac{1}{2}AB$
    \bigbreak \noindent 
    By the definition of $\omega$, $AC, BC, AB \leq \omega$. Since $ AC \leq \frac{1}{2}AB$, and $BC \leq\frac{1}{2}AB$, we have
    \begin{align*}
        AC + BC \leq \frac{1}{2} AB + \frac{1}{2} AB \leq AB \leq \omega
    \end{align*}
    Observe that since $AC + BC \leq AB$, both $AC$ and $BC = CB$ must be less than $AB$. That is, $AC,BC < AB$ since by Ax.D2, $AC, BC, AB \ne 0$.
    \bigbreak \noindent 
    Since $AC + BC \leq \omega$, by Ax.BP, there is a betweenness relation among $A,B,C$. One of the following must hold
    \begin{align*}
        A\text{-}B\text{-}C \\
        B\text{-}A\text{-}C \\
        A\text{-}C\text{-}B
    \end{align*}
    Assume the relation is $ A\text{-}B\text{-}C $. Then, we have $AB + BC = AC$ which implies $AB,BC < AC $. But this contradicts the fact that $AC < AB$. Thus, the relation is not $ A\text{-}B\text{-}C$. 
    \bigbreak \noindent 
    Assume the relation is $ B\text{-}A\text{-}C$. Then, $BA + AC  = BC$, or equivalently $AB + AC = BC $. This contradicts the fact that $BC < AB$. Thus, this must also not be the relation.
    \bigbreak \noindent 
    Therefore, the relation we have is $ A\text{-}C\text{-}B$.
    \bigbreak \noindent 
    Next, we aim to show that $AC = BC = \frac{1}{2} AB$. Since $ A\text{-}C\text{-}B$ was established, we have $AC + BC = AB$, solving for $BC$, we get 
    \begin{align*}
        BC = AB - AC
    \end{align*}
    Since $AC$ is bounded above by $\frac{1}{2}AB$. That is, $AC \leq \frac{1}{2} AB$, it must be that $AB - AC \geq AB - \frac{1}{2} AB$. Thus,
    \begin{align*}
        BC = AB - AC &\geq AB - \frac{1}{2} AB \\
        \therefore BC &\geq \frac{1}{2}AB
    \end{align*}
    But, since we know that $BC \leq \frac{1}{2}AB$, the only way both $ BC \leq \frac{1}{2}AB$ and $BC \geq \frac{1}{2} AB$ can be satisfied is if $BC = \frac{1}{2}AB$. Now, since $BC =\frac{1}{2}AB$, we have
    \begin{align*}
        AC + BC &= AB \implies AC + \frac{1}{2}AB = AB \\
        \therefore AC &= AB - \frac{1}{2}AB = \frac{1}{2}AB
    \end{align*}
    \bigbreak \noindent 
    Thus, we conclude that $AC = BC = \frac{1}{2}AB $ \endpf



\end{document}
