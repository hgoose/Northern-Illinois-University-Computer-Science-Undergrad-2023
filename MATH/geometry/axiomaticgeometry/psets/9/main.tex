 \documentclass{report}
 
 \input{~/dev/latex/template/preamble.tex}
 \input{~/dev/latex/template/macros.tex}
 
 \title{\Huge{}}
 \author{\huge{Nathan Warner}}
 \date{\huge{}}
 \fancyhf{}
 \rhead{}
 \fancyhead[R]{\itshape Warner} % Left header: Section name
 \fancyhead[L]{\itshape\leftmark}  % Right header: Page number
 \cfoot{\thepage}
 \renewcommand{\headrulewidth}{0pt} % Optional: Removes the header line
 %\pagestyle{fancy}
 %\fancyhf{}
 %\lhead{Warner \thepage}
 %\rhead{}
 % \lhead{\leftmark}
 %\cfoot{\thepage}
 %\setborder
 % \usepackage[default]{sourcecodepro}
 % \usepackage[T1]{fontenc}
 
 % Change the title
 \hypersetup{
     pdftitle={}
 }

 \geometry{
  left=1in,
  right=1in,
  top=1in,
  bottom=1in
}
 
 \begin{document}
     % \maketitle
     %     \begin{titlepage}
     %    \begin{center}
     %        \vspace*{1cm}
     % 
     %        \textbf{}
     % 
     %        \vspace{0.5cm}
     %         
     %             
     %        \vspace{1.5cm}
     % 
     %        \textbf{Nathan Warner}
     % 
     %        \vfill
     %             
     %             
     %        \vspace{0.8cm}
     %      
     %        \includegraphics[width=0.4\textwidth]{~/niu/seal.png}
     %             
     %        Computer Science \\
     %        Northern Illinois University\\
     %        United States\\
     %        
     %             
     %    \end{center}
     % \end{titlepage}
     % \tableofcontents
    \pagebreak \bigbreak \noindent
    Nate Warner \ \quad \quad \quad \quad \quad \quad \quad \quad \quad \quad \quad \quad  MATH 353 \quad  \quad \quad \quad \quad \quad \quad \quad \quad \ \ \quad \quad Spring 2025
    \begin{center}
        \textbf{Problem set 11 - Due: Wednesday, April 2}
    \end{center}
    \bigbreak \noindent 
    \begin{mdframed}
        1. Assume that $AC < \omega$, $ A\text{-}B\text{-}C$, and $D$ is a point not on $\overleftrightarrow{AC}$. Let $h$ be a ray with endpoint $C$ such that $h$ meets $\overline{BD}^{0} $. Prove that $h$ meets $\overline{AD} $
    \end{mdframed}
    \bigbreak \noindent 
    \textbf{\textit{Proof.}} By Ax.C, point $D\not\in \overleftrightarrow{AC}$  and $ A\text{-}B\text{-}C$ yields $ \overrightarrow{DA}\text{-}\overrightarrow{DB}\text{-}\overrightarrow{DC}$. Let $E \in \overline{DA}^{0}$. By the Crossbar theorem, there exists a point $F \in \overline{DB}^{0}$ such that $ E\text{-}F\text{-}C$, let $h$ be the ray with endpoint $C$ that contains points $E,F,C$. Note that point $F$ is where $h$ meets $ \overline{BD^{0}} $
    \bigbreak \noindent 
    Thus, $ h$ meets $\overline{AD}$ at point $E$ \endpf

    \bigbreak \noindent 
    \begin{mdframed}
        2. Suppose that $A,P$ and $R$ are noncollinear, $ A\text{-}X\text{-}P$, $ A\text{-}Z\text{-}R$ and $ P\text{-}Q\text{-}R $
        \begin{enumerate}[label=(\alph*)]
            \item Prove there is a point $Y$ on $\overrightarrow{AQ}$ so that $ X\text{-}Y\text{-}Z$
            \item Prove further that $ A\text{-}Y\text{-}Q$
        \end{enumerate}
    \end{mdframed}
    \bigbreak \noindent 
    \textbf{\textit{Proof.}} Observe that since $ P\text{-}Q\text{-}R$, $P,Q,R$ are collinear. Let $\overleftrightarrow{PR}$ be the line that contains these three points. Since $A,P,R$ noncollinear, $A \not\in \overleftrightarrow{PR} $. By Ax.C $ \overrightarrow{AP}\text{-}\overrightarrow{AQ}\text{-}\overrightarrow{AR}$. Note that $X \in \overrightarrow{AP}^{0}, Z \in \overrightarrow{AR}^{0}$. Thus, by the crossbar theorem, there exits a point $ Y \in \overrightarrow{AQ}^{0}$ such that $ X\text{-}Y\text{-}Z $
    \bigbreak \noindent 
    Since $Y \in \overrightarrow{AQ}$, one of $ A\text{-}Y\text{-}Q$ or $ A\text{-}Q\text{-}Y$ must be true. Assume for the sake of contradiction that $ A\text{-}Q\text{-}Y$.
    \bigbreak \noindent 
    Consider the line $\overleftrightarrow{XZ}$, note that $Y \in \overleftrightarrow{XZ}$. By Ax.S, $\overleftrightarrow{XZ}$ splits the plane into a pair of opposite halfplanes with edge $\overleftrightarrow{XZ}$, call this pair $H,K$. By $ A\text{-}Q\text{-}Y = Y\text{-}Q\text{-}A$ and theorem 10.3, $\overrightarrow{YQ}^{0} \subseteq $ one of the halfplanes, let's say its $H$. Since $Q \in \overrightarrow{YQ}^{0}$, $Q \in H$. $ Y\text{-}Q\text{-}A$ implies $ A \in \overrightarrow{YQ}^{0}$, thus $A \in H$. So, $A,Q$ in the same halfplane ($H$).
    \bigbreak \noindent 
    Next, we consider $ A\text{-}Z\text{-}R$, which implies by theorem 10.6 that $A,R$ in opposite halfplanes (since $Z \in \overleftrightarrow{XZ}) $. 
    \bigbreak \noindent 
    Similarly, since $X \in \overleftrightarrow{XZ}$, and $ A\text{-}X\text{-}P$, $A,P$ in opposite halfplanes by Thm 10.6. 
    \bigbreak \noindent 
    Observe that $ R\text{-}Q\text{-}P$ implies $Q \in \overline{PR}$, and since $R,P$ not in the halfplane with $A$, they must be in the same halfplane. Namely, the halfplane $K$ since $ A \in H$. Thus, since $K$ convex (by definition of $\frac{1}{2}$planes), $\overline{PR} \in K$, and since $Q \in \overline{PR}$, $Q \in K$, which is a contradiction, since $Q \in H$ implies $Q \not\in K$. Thus, $ A\text{-}Q\text{-}Y$ is not a valid assumption and must be thrown out. However, we know that $y \in \overrightarrow{AQ}$, which means the only possibility left  is that $ A\text{-}Y\text{-}Q$ \endpf

    \bigbreak \noindent 
    \begin{mdframed}
        3. Let $P = (0.8,0)$, $Q = (0.9, 0)$, $R = (0.9,0.1)$. Compute both the $\mathbb{E}$-measure and the $\mathbb{H}$-measure of $\underline{\angle QPR}$. Repeat for $P=(0.98,0)$, $Q = (0.99,0), R=(0.99,0.1)$.
    \end{mdframed}
    \bigbreak \noindent 
    \begin{remark}
        The $\mathbb{E}$-measure for $\underline{\angle QPR}$, if $\overleftrightarrow{PR}$ given by $y=mx+b$, $\overleftrightarrow{PQ}$ by $y=nx+c$ is
        \begin{align*}
            \angle QPR = \cos^{-1}{\left(\frac{1+mn}{\sqrt{1+m^{2}}\sqrt{1+n^{2}}}\right)}
        .\end{align*}
    \end{remark}
    \bigbreak \noindent 
    Let $P = (0.8,0)$, $Q = (0.9, 0)$, $R = (0.9,0.1)$. Then,
    \begin{align*}
        m &= \frac{0.1 - 0}{0.9-0.8} = 1, \\
        n &= \frac{0 - 0}{0.9 -0.8} = 0
    .\end{align*}
    Thus,
    \begin{align*}
        \angle QPR = \cos^{-1}{\left(\frac{1+1(0)}{\sqrt{1+1^{2}}\sqrt{1+0^{2}}}\right)} = \cos^{-1}{\left(\frac{1}{\sqrt{2}}\right)} = \frac{\pi}{4}
    .\end{align*}
    For $P=(0.98,0), Q=(0.99,0), R=(0.99,0.1)$, we have
    \begin{align*}
        m &= \frac{0.1 - 0}{0.99-0.98} = \frac{0.1}{0.01} = 10,\\
        n&= 0
    .\end{align*}
    Thus,
    \begin{align*}
        \underline{\angle QPR} = \cos^{-1}{\left(\frac{1+1(0)}{\sqrt{1+10^{2}}\sqrt{1+0^{2}}}\right)} = \cos^{-1}{\left(\frac{1}{\sqrt{101}}\right)} \approx  1.47
    .\end{align*}
    \bigbreak \noindent 
    \begin{remark}
        The $\mathbb{H}$-measure for $\underline{\angle QPR}$ is given by 
        \begin{align*}
            \mu_{\mathbb{H}}(\overrightarrow{PQ}, \overrightarrow{PR}) = \cos^{-1}{\left(\frac{1+mn-bc}{\sqrt{1+m^{2}-b^{2}}\sqrt{1+n^{2}-c^{2}}}\right)} 
        \end{align*}
        provided $\overleftrightarrow{PR}$ given by $y=mx+b$, $\overleftrightarrow{PQ}$ given by $y = nx+c$.
    \end{remark}
    \bigbreak \noindent 
    For $P=(0.8,0), Q=(0.9,0), R=(0.9,0.1)$, we have
    \begin{align*}
        &\overleftrightarrow{PQ}:\ y=0, \quad n = 0, c = 0\\
        &\overleftrightarrow{PR}:\ y - 0 = 1(x-0.8) \implies y = x-0.8, \quad m=1, b=0.8
    .\end{align*}
    Thus,
    \begin{align*}
        \mu_{\mathbb{H}}(\overrightarrow{PQ}, \overrightarrow{PR}) = \cos^{-1}{\left(\frac{1+1(0)-0.8(0)}{\sqrt{1+1^{2}-0.8^{2}}\sqrt{1+0^{2}-0^{2}}}\right)} = \cos^{-1}{\left(\frac{1}{\sqrt{2-0.8^{2}}}\right)} \approx 0.9078
    .\end{align*}
    For $P=(0.98,0), Q=(0.99,0), R=(0.99,0.1)$, we have
    \begin{align*}
        &\overleftrightarrow{PQ}:\ y=0, \quad n = 0, c = 0\\
        &\overleftrightarrow{PR}:\ y - 0 = 1(x-0.98) \implies y = x-0.98, \quad m=1, b=0.98
    .\end{align*}
    Thus,
    \begin{align*}
            \mu_{\mathbb{H}}(\overrightarrow{PQ}, \overrightarrow{PR}) = \cos^{-1}{\left(\frac{1+1(0)-0.98(0)}{\sqrt{1+1^{2}-0.98^{2}}\sqrt{1+0^{2}-0^{2}}}\right)} = \cos^{-1}{\left(\frac{1}{\sqrt{2-0.98^{2}}}\right)} \approx 0.9506
    .\end{align*}

    
    


\end{document}
