 \documentclass{report}
 
 \input{~/dev/latex/template/preamble.tex}
 \input{~/dev/latex/template/macros.tex}
 
 \title{\Huge{}}
 \author{\huge{Nathan Warner}}
 \date{\huge{}}
 \fancyhf{}
 \rhead{}
 \fancyhead[R]{\itshape Warner} % Left header: Section name
 \fancyhead[L]{\itshape\leftmark}  % Right header: Page number
 \cfoot{\thepage}
 \renewcommand{\headrulewidth}{0pt} % Optional: Removes the header line
 %\pagestyle{fancy}
 %\fancyhf{}
 %\lhead{Warner \thepage}
 %\rhead{}
 % \lhead{\leftmark}
 %\cfoot{\thepage}
 %\setborder
 % \usepackage[default]{sourcecodepro}
 % \usepackage[T1]{fontenc}
 
 % Change the title
 \hypersetup{
     pdftitle={}
 }

 \geometry{
  left=1in,
  right=1in,
  top=1in,
  bottom=1in
}
 
 \begin{document}
     % \maketitle
     %     \begin{titlepage}
     %    \begin{center}
     %        \vspace*{1cm}
     % 
     %        \textbf{}
     % 
     %        \vspace{0.5cm}
     %         
     %             
     %        \vspace{1.5cm}
     % 
     %        \textbf{Nathan Warner}
     % 
     %        \vfill
     %             
     %             
     %        \vspace{0.8cm}
     %      
     %        \includegraphics[width=0.4\textwidth]{~/niu/seal.png}
     %             
     %        Computer Science \\
     %        Northern Illinois University\\
     %        United States\\
     %        
     %             
     %    \end{center}
     % \end{titlepage}
     % \tableofcontents
    \pagebreak \bigbreak \noindent
    Nate Warner \ \quad \quad \quad \quad \quad \quad \quad \quad \quad \quad \quad \quad \quad \quad \quad \quad  MATH 353 \quad  \quad \quad \quad \quad \quad \quad \quad \quad \ \ \quad \quad \quad \quad \ \quad Spring 2025
    \begin{center}
        \textbf{Problem set 13 - Due: Monday, April 21}
    \end{center}
    \bigbreak \noindent 
    \begin{mdframed}
        5. Given a proper angle $\underline{hk}$, $h^{\prime}$ the ray opposite $h$, that ray $r$ bisects $\underline{hk}$ and ray $s$ bisects $\underline{h^{\prime}k}$, show that $rs = 90$ (Hint: use insertion to show that $ r\text{-}k\text{-}s$)
    \end{mdframed}
    \bigbreak \noindent 
    \textbf{\textit{Proof.}} Assume a proper angle $\underline{hk}$, with a ray $r$ that bisects $ \underline{hk}$, and a ray $s$ that bisects $ \underline{h^{\prime}k} $. Since $r$ bisects $\underline{hk}$, $ h\text{-}r\text{-}k$, with $ hr = rk = \frac{1}{2}hk$. Also, $s$ bisects $ \underline{kh^{\prime}}$, $ k\text{-}s\text{-}h^{\prime}$ and $ks = sh^{\prime} = \frac{1}{2}kh^{\prime}$
    \bigbreak \noindent 
    Consider the fan $\overrightarrow{hk}$, by the dual of Proposition 9.3, $\overrightarrow{hk} = \overline{hk} \cup \overline{kh^{\prime}}$. So, since $ k\text{-}s\text{-}h^{\prime}$, $ s \in \overline{kh^{\prime}}$. Thus, $s\in \overrightarrow{hk}$, specifically $ h\text{-}k\text{-}s$.
    \bigbreak \noindent 
    So, $ h\text{-}r\text{-}k $ with $ h\text{-}k\text{-}s$ and the ROI implies $ h\text{-}r\text{-}k\text{-}s$, which implies $ r\text{-}k\text{-}s$. We have
    \begin{align*}
        rk + ks = rs \\
        \implies \frac{1}{2}hk + \frac{1}{2}kh^{\prime} = rs \\
        \implies kh + kh^{\prime} = 2rs
    .\end{align*}
    By Theorem 14.1, $kh$ and $kh^{\prime}$ are supplementary, so $ hk + kh^{\prime} = 180$. Thus,
    \begin{align*}
        kh + kh^{\prime} &= 2rs \\
        \implies 180 &= 2rs \\
        \implies rs &= 90
    .\end{align*}
    Therefore, $ rs = 90$ \endpf

    \bigbreak \noindent 
    \begin{mdframed}
        12. Prove Theorem 14.9
    \end{mdframed}
    \bigbreak \noindent 
    \begin{remark}
        \textit{(Theorem 14.9)}: Every point of the perpendicular bisector of a segment is equidistant from the endpoints of the segment: $AX = BX$ for all $X$ on the perpendicular bisector
    \end{remark}
    \bigbreak \noindent 
    \textbf{\textit{Proof.}} Let $ \overleftrightarrow{AB}$ be a line, at $m$ be the perpendicular bisector of $ \overleftrightarrow{AB}$ at the midpoint $M$ of $\overline{AB}$. Let $X \in m$.
    \bigbreak \noindent 
    If $X \in \overleftrightarrow{AB}$, then $X = M$, and $AX = BX$ by definition of the midpoint $M$ of $\overline{AB}$. So, assume that $X \not\in \overleftrightarrow{AB}$.
    \bigbreak \noindent 
    Since $X \in m$, $X \not\in \overleftrightarrow{AB}$, $A,M,X,B$ are noncollinear, so we have $ \triangle AXM$ and $\triangle BXM$. By definition of perpendicular, $ \angle AMX = \angle BMX = 90$.
    \pagebreak \bigbreak \noindent 
    \begin{figure}[ht]
        \centering
        \incfig{tri}
        \label{fig:tri}
    \end{figure}
    \bigbreak \noindent 
    Consider the correspondence $AXM \leftrightarrow BXM$ between the vertices of triangles $\triangle AXM$ and $ \triangle BXM$. We have $\overline{MX} \cong \overline{MX}$, $\underline{\angle  AMX} \cong \underline{\angle BMX}$, and $ \overline{AM} \cong \overline{MB}$ (by definition of of the midpoint $M$ of segment $\overline{AB}$), so $ \triangle AXM \cong \triangle BXM$ by AX.SAS, which gives 
    \begin{align*}
        \overline{AX} \cong \overline{BX}
    .\end{align*}
    Therefore, $AX = BX$ \endpf


    \bigbreak \noindent 
    \begin{mdframed}
        13. Prove Theorem 14.10
    \end{mdframed}
    \bigbreak \noindent 
    \begin{remark}
        \textit{Theorem 14.10 (converse of 14.9)}: Let $m = \overleftrightarrow{AB}$, suppose that line $n\ne m$ meets $m$ at the midpoint $M$ of $\overline{AB}$. Suppose that there is some point $X$ on $n$, not on $m$, so that $AX = BX$. Then, $n \perp n$ at $M$
    \end{remark}
    \bigbreak \noindent 
    \textbf{\textit{Proof.}} Assume that $ m = \overleftrightarrow{AB}$, and that $ n \ne m$ meets $m$ at the midpoint $M$ of $\overline{AB}$. Further assume that there is a point $X$ on $n$, but not on $m$ such that $AX = BX$
    \bigbreak \noindent 
    Since $X \in n, X \not\in m$, $ A,M,B,X$ are noncollinear, so we have triangles $ \triangle AMX$, and $\triangle BMX$.
    \bigbreak \noindent 
    \begin{figure}[ht]
        \centering
        \incfig{tri2}
        \label{fig:tri2}
    \end{figure}
    \bigbreak \noindent 
    We consider the correspondence $AMX \leftrightarrow BMX$ between the vertices of $\triangle AMX$ and $ \triangle BMX $.
    \bigbreak \noindent 
    By Pons Asinorum, $\angle MAX = \angle MBX$, so $\underline{\angle MAX} \cong \underline{\angle MBX}$. Also, since $AX = BX$, we have $\overline{AX} \cong \overline{BX}$, and $\overline{AM} \cong \overline{MB}$ by definition of the midpoint $M$ of $\overline{AB}$. So, by AX.SAS, $\triangle AMX \cong \triangle BMX$, and thus
    \begin{align*}
        \angle AMX = \angle BMX
    .\end{align*}
    By definition of the midpoint $M$ of segment $\overline{AB} $, $ A\text{-}M\text{-}B$, so $ \overrightarrow{MA}$ opposite to $ \overrightarrow{MB}$, and $\overrightarrow{MA}\overrightarrow{MB} = \angle AMB =180 $
    \bigbreak \noindent 
    Consider the ray $\overrightarrow{MX}$. By Theorem 11.8, $ \overrightarrow{MA}\text{-}\overrightarrow{MX}\text{-}\overrightarrow{MB} $, so $ \overrightarrow{MA}\overrightarrow{MX} + \overrightarrow{MX}\overrightarrow{MB} = \overrightarrow{MA}\overrightarrow{MB} = 180  $, or equivalently, $ \angle AMX + \angle BMX = 180$. Thus, 
    \begin{align*}
        2\angle AMX &= 180 \\
        \implies \angle AMX &= \angle BMX = 90
    .\end{align*}
    Next, we consider the ray opposite to $\overrightarrow{MX}$, ray $\overrightarrow{MX}^{\prime}$. By Ax.RR, there exists a point $C\in \overrightarrow{MX}^{\prime}$ such that $MC < \omega$. So, by theorem 8.4, $\overrightarrow{MX}^{\prime} = \overrightarrow{MC}$. By Theorem 11.8, we have that
    \begin{align*}
        \overrightarrow{MX}\text{-}\overrightarrow{MB}\text{-}\overrightarrow{MC}
    \end{align*}
    and
    \begin{align*}
        \overrightarrow{MX}\text{-}\overrightarrow{MA}\text{-}\overrightarrow{MC}
    .\end{align*}
    So, $ \overrightarrow{MX}\overrightarrow{MB} + \overrightarrow{MB}\overrightarrow{MC} = 180 \implies \angle BMX + \angle BMC = 180$, and $ \overrightarrow{MX}\overrightarrow{MA} + \overrightarrow{MA}\overrightarrow{MC} = 180 \implies \angle AMX + \angle AMC = 180$. Since $ \angle BMX = \angle AMX =90$, we have
    \begin{align*}
        90 + \angle BMC &= 180\\
        \implies \angle BMC &= 90, 
    \end{align*}
    and 
    \begin{align*}
        90 + \angle AMC &= 180 \\
        \implies \angle AMC &= 90
    .\end{align*}
    \bigbreak \noindent 
    \begin{figure}[ht]
        \centering
        \incfig{tri3}
        \label{fig:tri3}
    \end{figure}
    \bigbreak \noindent 
    So, the four angles determined by the intersection of $n$ with $m$ are $90$, thus $n \perp m$ at $M$ by definition of perpendicular. \endpf

    \bigbreak \noindent 
    \begin{mdframed}
        16. Suppose that $A,B$ and $C$ are noncollinear points such that $AB = AC = \frac{\omega}{2}$ ($\omega < \infty$). Prove that $A$ is a pole for $ \overleftrightarrow{BC}$ (Hint: Consider $\triangle ABC$ and $\triangle A^{*}BC$)
    \end{mdframed}
    \bigbreak \noindent 
    \textbf{\textit{Proof.}} Assume that $A,B,C$ are noncollinear points such that $AB = AC = \frac{\omega}{2}$. 
    \bigbreak \noindent 
    We first note that $A^{*},B,C$ are also noncollinear points, so we have $\triangle ABC,\ \triangle A^{*}BC $. Consider the correspondence $ABC \leftrightarrow A^{*}BC$ between the vertices of triangles $\triangle ABC$ and $ \triangle A^{*}BC$
    \bigbreak \noindent 
    \begin{figure}[ht]
        \centering
        \incfig{tri4}
        \label{fig:tri4}
    \end{figure}
    \bigbreak \noindent 
    We have that $\overline{BC} \cong \overline{BC}$. Since $AB = \frac{\omega}{2}$, and $ A\text{-}B\text{-}A^{*}$ by Theorem 9.1, we have that
    \begin{align*}
        AB + BA^{*} &= AA^{*} = \omega \\
        \implies BA^{*} &= \omega - AB \\
        \implies BA^{*} &= \omega - \frac{\omega}{2} = \frac{\omega}{2}
    .\end{align*}
    Similarly, $ A\text{-}C\text{-}A^{*}$, and $AC = \frac{\omega}{2}$, so
    \begin{align*}
        AC + CA^{*} &= AA^{*} = \omega \\
        \implies CA^{*} &= \omega - AC \\
        \implies CA^{*} &= \omega-\frac{\omega}{2} = \frac{\omega}{2}
    .\end{align*} 
    So, $\overline{AB} \cong \overline{A^{*}B}$, and $\overline{AC} \cong \overline{A^{*}C}$. By Theorem 13.4 (SSS), we have that $\triangle ABC \cong \triangle A^{*}BC$. Thus, $ \underline{\angle ABC} \cong \underline{\angle A^{*}BC}$, and $\underline{\angle ACB} \cong\underline{\angle A^{*}CB} $ , which implies $\angle ABC = \angle A^{*}BC $, and $ \angle ACB = \angle A^{*}CB$.
    \bigbreak \noindent 
    Consider the rays $\overrightarrow{BA},\overrightarrow{BA^{*}}, \overrightarrow{BC}$. By Theorem 9.6, rays $\overrightarrow{BA}$ and $\overrightarrow{BA^{*}}$ are opposite, and Theorem 11.8 implies
    \begin{align*}
        \overrightarrow{BA}\text{-}\overrightarrow{BC}\text{-}\overrightarrow{BA^{*}}
    .\end{align*}
    Thus,
    \begin{align*}
        \overrightarrow{BA}\overrightarrow{BC} + \overrightarrow{BC}\overrightarrow{BA^{*}} = \overrightarrow{BA}\overrightarrow{BA^{*}} = 180 \\
        \implies \angle ABC + \angle A^{*}BC = 180 \\
        \implies 2\angle ABC = 180 \\
        \implies \angle ABC  = \angle A^{*}BC= 90
    .\end{align*}
    Also, it can be easily shown by that the angle supplementary to $\angle ABC$ is 90, and the angle supplementary to $\angle A^{*}BC$ is 90. Thus, the four angles determined by the intersection of $\overleftrightarrow{AB}$ with $\overleftrightarrow{BC}$ are all 90, so $\overleftrightarrow{AB} \perp \overleftrightarrow{BC}$ at $B$, and $\overleftrightarrow{AB}$ meets $ \overleftrightarrow{BC}$ at a point ($B$) distance $\frac{\omega}{2}$ from $A$, so by the definition of a pole, $A$ is a pole for $ \overleftrightarrow{BC}$ \endpf.











\end{document}
