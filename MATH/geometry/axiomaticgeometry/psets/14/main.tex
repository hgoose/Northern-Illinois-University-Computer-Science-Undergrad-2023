 \documentclass{report}
 
 \input{~/dev/latex/template/preamble.tex}
 \input{~/dev/latex/template/macros.tex}
 
 \title{\Huge{}}
 \author{\huge{Nathan Warner}}
 \date{\huge{}}
 \fancyhf{}
 \rhead{}
 \fancyhead[R]{\itshape Warner} % Left header: Section name
 \fancyhead[L]{\itshape\leftmark}  % Right header: Page number
 \cfoot{\thepage}
 \renewcommand{\headrulewidth}{0pt} % Optional: Removes the header line
 %\pagestyle{fancy}
 %\fancyhf{}
 %\lhead{Warner \thepage}
 %\rhead{}
 % \lhead{\leftmark}
 %\cfoot{\thepage}
 %\setborder
 % \usepackage[default]{sourcecodepro}
 % \usepackage[T1]{fontenc}
 
 % Change the title
 \hypersetup{
     pdftitle={}
 }

 \geometry{
  left=1in,
  right=1in,
  top=1in,
  bottom=1in
}
 
 \begin{document}
     % \maketitle
     %     \begin{titlepage}
     %    \begin{center}
     %        \vspace*{1cm}
     % 
     %        \textbf{}
     % 
     %        \vspace{0.5cm}
     %         
     %             
     %        \vspace{1.5cm}
     % 
     %        \textbf{Nathan Warner}
     % 
     %        \vfill
     %             
     %             
     %        \vspace{0.8cm}
     %      
     %        \includegraphics[width=0.4\textwidth]{~/niu/seal.png}
     %             
     %        Computer Science \\
     %        Northern Illinois University\\
     %        United States\\
     %        
     %             
     %    \end{center}
     % \end{titlepage}
     % \tableofcontents
    \pagebreak \bigbreak \noindent
    Nate Warner \ \quad \quad \quad \quad \quad \quad \quad \quad \quad \quad \quad \quad \quad \quad \quad \quad  MATH 353 \quad  \quad \quad \quad \quad \quad \quad \quad \quad \ \ \quad \quad \quad \quad \ \quad Spring 2025
    \begin{center}
        \textbf{Problem set 14 - Due: Monday, April 28}
    \end{center}
    \bigbreak \noindent 
    \begin{mdframed}
        15.3. Prove Corollary 15.4 
    \end{mdframed}
    \bigbreak \noindent 
    \begin{remark}
        \textit{Corollary 15.4}. The nonright angles of a small right triangle are accute
    \end{remark}
    \bigbreak \noindent 
    \textbf{\textit{Proof}}. Assume a small triangle $\triangle ABC$, with $\angle ABC = 90$.
    \bigbreak \noindent 
    \begin{figure}[ht]
        \centering
        \incfig{smallright}
        \label{fig:smallright}
    \end{figure}
    \bigbreak \noindent 
    Extend the segment $\overline{BC}$ to form exterior angle $\underline{\angle DBA}$. Since $\underline{\angle ABC}$ and $\underline{\angle DBA}$ are supplementary, $\angle DBA  = 180 - \angle ABC = 180 - 90 = 90$.
    \bigbreak \noindent 
    \begin{figure}[ht]
        \centering
        \incfig{fig2}
        \label{fig:fig2}
    \end{figure}
    \bigbreak \noindent 
    By theorem 15.3, $\angle DBA > \angle ACB$ and $ \angle BAC$, which implies $ \angle BAC, \angle ACB < 90$. Since $\underline{\angle ACB}$ and $ \underline{\angle BAC} $ are the nonright angles and they have angle measure less than $90$, the nonright angles of a small right triangle are therefore acute. \endpf

    \bigbreak \noindent 
    \begin{mdframed}
        15.4. Prove Corollary 15.5 (Hint: Show that if $M$ is the midpoint of the base $\overline{BC}$ of isosceles triangle $\triangle ABC$, then $ \triangle ABM$  and $\triangle ACM$ are both small right triangles)
    \end{mdframed}
    \bigbreak \noindent 
    \begin{remark}
        \textit{Corollary 15.5}. The base angles of an isosceles triangle whose congruent sides are $< \frac{\omega}{2}$ are acute.
    \end{remark}
    \bigbreak \noindent 
    \textbf{\textit{Proof.}} Assume an isosceles triangle $ \triangle ABC$, with congruent sides $< \frac{\omega}{2}$, let $AB, AC$ be the congruent sides, so $\overline{AB} \cong \overline{AC} \implies AB = AC$. 
    \bigbreak \noindent 
    Let $M$ be the midpoint of $\overline{BC}$, call the line that contains $A,M$ $ \overleftrightarrow{AM}$. Since $A \in \overleftrightarrow{AM},\ A \not\in \overleftrightarrow{BC}$, and $BA = BC$, $ \overleftrightarrow{AM} \perp \overleftrightarrow{BC}$ at $M$, so $\angle AMB = \angle AMC = 90$.
    \bigbreak \noindent 
    \begin{figure}[ht]
        \centering
        \incfig{iso}
        \label{fig:iso}
    \end{figure}
    \bigbreak \noindent 
    Since $AB, AC < \frac{\omega}{2}$, and $ B\text{-}M\text{-}C$, Theorem 15.1 implies $ AM < \frac{\omega}{2}$. Since $ B\text{-}M\text{-}C$, we have $ BM + MC = BC$. Since $A,B,C$ noncollinear, $BC < \omega$. By definition of the midpoint $M$ of the segment $ \overline{BC}$, $ BM = MC = \frac{1}{2}BC$, so $BC = 2BM = 2BC$. So,
    \begin{align*}
        BC < \omega \\
        \implies 2BM < \omega \\
        \implies BM < \frac{\omega}{2}
    .\end{align*}
    And,
    \begin{align*}
        BC &< \omega \\
        \implies 2MC &< \omega \\
        \implies MC &< \frac{\omega}{2}
    .\end{align*}
    So, $\triangle AMB$ and $ \triangle AMC$ are both small.
    \bigbreak \noindent 
    Therefore, by Corollary 15.4, $\angle ABM < 90$, and $\angle ACM  < 90$. \endpf


    \pagebreak \bigbreak \noindent 
    \begin{mdframed}
        15.7. Show that if $\omega < \infty$, then for any triangle $\triangle ABC$, 
        \begin{align*}
            AB + BC  + CA < 2\omega
        .\end{align*}
        Hint: Apply the Triangle Inequality  to $\triangle BCA^{*} $
    \end{mdframed}
    \bigbreak \noindent 
    \textbf{\textit{Proof.}} Assume $\omega < \infty$, and the existence of triangle $\triangle ABC$.
    \bigbreak \noindent 
    Consider the triangle $\triangle BCA^{*}$, by the Triangle Inequality, we have
    \begin{align*}
       BA^{*} + CA^{*} > BC 
    .\end{align*}
    Note that by Theorem 9.1 $ A\text{-}B\text{-}A^{*} \implies AB + BA^{*} = AA^{*} = \omega \implies BA^{*} = \omega - AB$, and similarly $ CA^{*} = \omega - CA$. So,
    \begin{align*}
        BA^{*} + CA^{*} &> BC \\
        \implies \omega - BA + \omega - CA &> BC \\
        \therefore AB + BC + CA &< 2\omega
    .\end{align*}
    As desired. \endpf

    \bigbreak \noindent 
    \begin{mdframed}
        16.2. Suppose that $\triangle ABC$ and $\triangle XYZ$ are two small triangles with $\angle A = \angle X$, $AB = XY$, and $\angle B < \angle Y$. Prove that $ \angle C > \angle Z $
    \end{mdframed}
    \bigbreak \noindent 
    \textbf{\textit{Proof}}. Assume that $\triangle ABC$ and $\triangle XYZ$ are two small triangles with $\angle A = \angle X$, $AB = XY$, and $\angle B  < \angle Y$.
    \bigbreak \noindent 
    \begin{figure}[ht]
        \centering
        \incfig{tri7}
        \label{fig:tri7}
    \end{figure}
    \bigbreak \noindent 
    Consider the rays, $\overrightarrow{YZ}, \overrightarrow{YX}$ and the fan $\overrightarrow{\overrightarrow{YZ}\overrightarrow{YX}} $ (which exists since $X,Y,Z$ noncollinear). By Theorem 11.6, there exists a ray $j\in \overrightarrow{\overrightarrow{YZ}\overrightarrow{YX}}$ such that $\overrightarrow{YX}j = \angle B$, $\overrightarrow{YX}j $ must be in the wedge $\overline{\overrightarrow{YZ}\overrightarrow{YX}}$ since $\angle B < \angle Y = \overrightarrow{YZ}\overrightarrow{YX} $.
    \bigbreak \noindent 
    By the Crossbar Theorem, there exists a point $E \in j^{0}$ such that $ X\text{-}E\text{-}Z$.
    \pagebreak \bigbreak \noindent 
    \begin{figure}[ht]
        \centering
        \incfig{tri11}
        \label{fig:tri11}
    \end{figure}
    \bigbreak \noindent 
    Notice that by Theorem 13.1 (ASA), we have congruence of triangles, specifically $\triangle ABC \cong \triangle XYE$ under the correspondence $ABC \leftrightarrow XYE$
    \bigbreak \noindent 
    Thus, $\angle C = \angle XEY$. Observe that $\angle XEY = \angle C$ is exterior to $\triangle EYZ$, thus $\angle XEY = \angle C > \angle EZY = \angle Z$. Thus, $\angle C > \angle Z$. \endpf

    \bigbreak \noindent 
    \begin{mdframed}
        16.5. Prove Corollary 16.2
    \end{mdframed}
    \bigbreak \noindent 
    \begin{remark}
        \textit{Corollary 16.2}. The hypotenuse of a small right triangle is its longest side
    \end{remark}
    \bigbreak \noindent 
    \textbf{\textit{Proof.}} Assume a small right triangle, call this triangle $\triangle ABC$, where $\angle B = 90$.
    \bigbreak \noindent 
    \begin{figure}[ht]
        \centering
        \incfig{tri}
        \label{fig:tri}
    \end{figure}
    \bigbreak \noindent 
    By Theorem 15.4, $\angle A$ and $\angle C < 90$, so $\angle B > \angle A $, and $\angle B > \angle  C$. Thus, by Theorem 16.1 (comparison), $CB < CA$ and $AB <  AC$. Therefore, the hypotenuse  $\overline{AC}$ is the longest side. \endpf
    










\end{document}
