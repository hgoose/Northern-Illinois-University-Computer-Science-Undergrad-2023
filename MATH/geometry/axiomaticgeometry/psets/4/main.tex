 \documentclass{report}
 
 \input{~/dev/latex/template/preamble.tex}
 \input{~/dev/latex/template/macros.tex}
 
 \title{\Huge{}}
 \author{\huge{Nathan Warner}}
 \date{\huge{}}
 \fancyhf{}
 \rhead{}
 \fancyhead[R]{\itshape Warner} % Left header: Section name
 \fancyhead[L]{\itshape\leftmark}  % Right header: Page number
 \cfoot{\thepage}
 \renewcommand{\headrulewidth}{0pt} % Optional: Removes the header line
 %\pagestyle{fancy}
 %\fancyhf{}
 %\lhead{Warner \thepage}
 %\rhead{}
 % \lhead{\leftmark}
 %\cfoot{\thepage}
 %\setborder
 % \usepackage[default]{sourcecodepro}
 % \usepackage[T1]{fontenc}
 
 % Change the title
 \hypersetup{
     pdftitle={}
 }

 \geometry{
  left=1in,
  right=1in,
  top=1in,
  bottom=1in
}
 
 \begin{document}
     % \maketitle
     %     \begin{titlepage}
     %    \begin{center}
     %        \vspace*{1cm}
     % 
     %        \textbf{}
     % 
     %        \vspace{0.5cm}
     %         
     %             
     %        \vspace{1.5cm}
     % 
     %        \textbf{Nathan Warner}
     % 
     %        \vfill
     %             
     %             
     %        \vspace{0.8cm}
     %      
     %        \includegraphics[width=0.4\textwidth]{~/niu/seal.png}
     %             
     %        Computer Science \\
     %        Northern Illinois University\\
     %        United States\\
     %        
     %             
     %    \end{center}
     % \end{titlepage}
     % \tableofcontents
    \pagebreak \bigbreak \noindent
    Nate Warner \ \quad \quad \quad \quad \quad \quad \quad \quad \quad \quad \quad \quad  MATH 353 \quad  \quad \quad \quad \quad \quad \quad \quad \quad \ \ \quad \quad Spring 2025
    \begin{center}
        \textbf{Problem set 4 - Due: Friday, Feb 7}
    \end{center}
    \bigbreak \noindent 
    \begin{mdframed}
        1. For each set, either find the supremum (l.u.b) or explain why none exists
        \begin{enumerate}[label=(\alph*)]
            \item $\{-2,-\frac{1}{2}, 0,\frac{4}{5},\frac{3}{2}\} $
            \item $\{x:\ x\in \mathbb{R} \text{ and } 5x^{2} < 45\} $
            \item $\{.6,.66,.666,.6666,...\} $
            \item $\{x^{2}:\ x\in\mathbb{R} \text{ and } x<2\} $
            \item $\{x^{3}:\ x\in\mathbb{R} \text{ and } x<2\} $
            \item $\{\frac{x}{3+x}:\ x\in \mathbb{R} \text{ and } x>0\} $
            \item $\{x:\ x=d_{\mathbb{S}}(PQ) \text{ for some  points $P,Q$ in $\mathbb{S} $(radius 1) }\} $
            \item $\{x:\ x=d_{\mathbb{M}}(PQ) \text{ for some  points $P,Q$ in $\mathbb{M} $}\} $
            \item $\{x:\ x=d_{\mathbb{G}}(PQ) \text{ for some  points $P,Q$ in $\mathbb{G} $}\} $
        \end{enumerate}
    \end{mdframed}
    \bigbreak \noindent 
    a.) The supremum is $\frac{3}{2}$
    \bigbreak \noindent 
    b.) We have
    \begin{align*}
        5x^{2} &< 45 \\
        x^{2} &< 9 \\
        \abs{x} &< 3 \\
        -3 &< x < 3
    \end{align*}
    Thus, $ \{x:\ x\in \mathbb{R} \text{ and } 5x^{2} < 45\} $ is precisely the open interval $(-3,3)$ and the supremem is therefore $3$

    \bigbreak \noindent 
    c.) The supremum is $\frac{2}{3} = 0.6666666667$

    \bigbreak \noindent 
    d.) The set $\{x^{2}:\ x\in \mathbb{R} \text{ and } x<2\} $ is the open interval $[0,4)$. Thus, the supremum is 4
    \bigbreak \noindent 
    e.) The set $\{x^{3}:\ x\in \mathbb{R} \text{ and } x<2\} $ is the open interval $(-\infty,8) $. Thus, the supremum is 8
    \bigbreak \noindent 
    f.) We find the limit as $x\to \infty $
    \begin{align*}
        \lim\limits_{x \to \infty}{\frac{x}{3+x}} &= \lim\limits_{x \to \infty}{\frac{\frac{x}{x}}{\frac{3}{x} + \frac{x}{x}}} \\
        &=\lim\limits_{x \to \infty}{\frac{1}{\frac{3}{x} + 1}} = \frac{1}{0+1} = 1
    \end{align*}
    Therefore, the set has supremum one.
    \bigbreak \noindent 
    g.) The set of distances $\mathbb{D}$ for the spherical plane with radius $r$ is bounded above by $\pi r $. Thus, the supremum is $\pi(1)=  \pi $
    \bigbreak \noindent 
    h.) The set of distances $\mathbb{D}$ on the Minkowski plane is unbounded. Distance is given by
    \begin{align*}
        \abs{x_{1} - x_{2}} + \abs{y_{1} -y_{2}}
    \end{align*}
    For $P(x_{1}, y_{1}), Q(x_{2}, y_{2})$ which grows arbitrarily large as $Q$ gets further from $P$. Therefoer, there is no supremum
    \bigbreak \noindent 
    i.) The set of distances $\mathbb{D}$ in the gap plane is also unbounded. Therefore there is no supremum 

    \bigbreak \noindent 
    \begin{mdframed}
        2. Prove proposition 4.1
    \end{mdframed}
    \bigbreak \noindent 
    \textbf{Proposition.} Let $S$ be a nonempty set of real numbers that has a least upper bound $b \in \mathbb{R}$. Let $t \in \mathbb{R}$ such that $t < b$. Then, there exists some $s \in S$ such that $t < s \leq b$.
    \bigbreak \noindent 
    \textbf{\textit{Proof.}} Assume $S$ is a nonempty subset of the real numbers with a least upper bound $b$. Let $t\in \mathbb{R}$ such that $t<b$. Since $b$ is a least upper bound of $S$, we have
    \begin{align*}
        \forall \ s \in S,\ s \leq b 
    \end{align*}
    Since $t< b$, $t$ cannot be an upper bound for $S$. If it were, then that would contradict $b$ being the least upper bound. Since $t$ is not an upper bound of $S$, then this implies the existence of some $s\in S$ such that $ t< s$. If this were not the case, then the negation which states, for all $s\in S$, $t \geq s$ would be true. Since the negation implies that $t$ is an upper bound, which we know can't be the case, there must exist some $s\in S$ such that $t < s$. 
    \bigbreak \noindent 
    Since $s \leq b$ for all $s\in S$, and we know that there exists some $s \in S$ such that $t < s$, there must be at least one $s$ that satisfies
    \begin{align*}
        t < s \leq b
    \end{align*}
    \hspace*{\fill} $\blacksquare$

    \bigbreak \noindent 
    \begin{mdframed}
        3. Show that in the $\mathbb{H}$ model, $\mathbb{D} = [0,\infty)$. (Hints: Compute $d_{\mathbb{H}}(AB)$ (in terms of $x$) for $A=(0,0)$ and $B=(x,0)$, $0<x<1$. Then use the fact that $\ln$ sends the interval $(1,\infty)$ onto $(0,\infty)$ )
    \end{mdframed}
    \bigbreak \noindent 
    Fix $A$ at $(0,0)$, let $B = (x,0)$ for $ 0 < x < 1$. Thus, $M = (-1,0)$, and $N = (1,0)$. If the hyperbolic distance is given by
    \begin{align*}
        \ln{\left(\frac{e(AN)e(BM)}{e(AM)e(BN)}\right)}
    \end{align*}
    Where $e(PQ)$ is the Euclidean distance $e(PQ) = \abs{x_{1} -x_{2}}\sqrt{1+m^{2}}$ for all points $P(x_{1}, y_{1}), Q(x_{2}, y_{2})$ on the line $y=mx+b$, then we have $e(PQ) = \left\lvert x_{1} - x_{2} \right\rvert \sqrt{1+0^{2}} = \left\lvert x_{1} - x_{2} \right\rvert$, which implies $e(AN) = \left\lvert 0 - 1 \right\rvert =1$, and $e(AM) = \left\lvert 0 - (-1)\right\rvert  = 1$.
    \bigbreak \noindent 
    Also, 
    \begin{align*}
        E(BM) &= \left\lvert x-(-1) \right\rvert = \left\lvert x+1 \right\rvert \\
        E(BN) &= \left\lvert x -1 \right\rvert
    \end{align*}
    \bigbreak \noindent 
    Therefore,
    \begin{align*}
        d_{\mathbb{H}} = \ln{\left(\frac{\left\lvert x+1 \right\rvert}{\left\lvert x-1 \right\rvert}\right)}
    \end{align*}
    Analyzing the input function of the natural log, we see the domain is $(-\infty, 1) \cup (1,\infty)$. We have
    \begin{align*}
        \lim\limits_{x \to \infty}{\frac{\sqrt{(x+1)^{2}}}{\sqrt{(x-1)^{2}}}} &= \sqrt{\lim\limits_{x \to \infty}{\left(\frac{x+1}{x-1}\right)^{2}}} \\
                                                                              &= \sqrt{\lim\limits_{x \to \infty}{\frac{x^{2} + 2x + 1}{x^{2} -2x +1}}} \\
                                                                              &= \sqrt{\lim\limits_{x \to \infty}{\frac{1 + \frac{2}{x} + \frac{1}{x^{2}}}{1 - \frac{2}{x} + \frac{1}{x^{2}}}}} \\
                                                                              &= \sqrt{\frac{1 + 0 + 0 }{1 - 0+ 0}} = 1
    \end{align*}
    Similarly, $\lim\limits_{x \to -\infty}{\frac{\left\lvert x+1 \right\rvert}{\left\lvert x-1 \right\rvert}} =1 $. Further, $\lim\limits_{x \to 1}{\left\lvert x-1 \right\rvert} = 0$. Since the denominator tends to zero, we have $\lim\limits_{x \to 1}{\frac{\left\lvert x+1 \right\rvert}{\left\lvert x-1 \right\rvert}} = \infty$. Thus, the range of $f(x) = \frac{\left\lvert x+1 \right\rvert}{\left\lvert x-1 \right\rvert} $ is $(1,\infty) $. Therefore, the domain of $\ln{\left(\frac{\left\lvert x+1 \right\rvert}{\left\lvert x-1 \right\rvert}\right)} $ is $(1,\infty)$. Since we know $\ln:\ (1,\infty) \to [0,\infty)$, the set $\mathbb{D}$ is therefore $\{x:\ x \geq 0\} = [0,\infty)$

    \bigbreak \noindent 
    \begin{mdframed}
        4. Let $\mathbb{P} = \{1,2,3\}, \mathbb{L} = \{\{1\}, \{1,2\}, \{2,3\}\}$. Define distance by 
        \begin{align*}
            d(PQ) = P - Q
        \end{align*}
        for all $P,Q$ in $\mathbb{P}$ (equal or not).
        \begin{enumerate}[label=(\alph*)]
            \item Tell which of the seven axioms fail to hold in this example, and explain why
            \item Find $\mathbb{D}$, the set of all distances (ie the image of $d$), and find $\omega$, the supremum of $\mathbb{D}$
        \end{enumerate}
    \end{mdframed}
    \bigbreak \noindent 
    \begin{remark}
        \textbf{Distance axioms}: For all points $P,Q$
        \begin{enumerate}
            \item $PQ \geq 0 $
            \item $PQ = 0 \iff P = Q $
            \item $PQ = QP $
        \end{enumerate}
        \textbf{Incidence axioms}:
        \begin{enumerate}
            \item At least two lines
            \item Each line contains at least two different points
            \item Each pair of points are together in at least one line
            \item Each pair of points with $PQ < \omega$ are together in at most one line
        \end{enumerate}
       \qed 
    \end{remark}
    \bigbreak \noindent 
    We have the distances
    \begin{align*}
        d(1,1) &= 0 \\
        d(1,2) &= -1 \\
        d(1,3) &= -2 \\
        d(2,1) &= 1 \\
        d(2,2) &= 0 \\
        d(2,3) &= -1 \\
        d(3,3) &= 0 \\
        d(3,2) &= 1 \\
        d(3,1) &= 2
    \end{align*}
    Thus, the distance function $d:\ \mathbb{P} \times \mathbb{P} \to \mathbb{R}$ is given by 
    \begin{align*}
        \begin{array}{c|ccc}
           &1&2&3 \\
           \hline
            1 & 0 & -1 &  -2\\
            2 & 1 & 0 & - 1\\
            3& 2 & 1 & 0
        \end{array}
    \end{align*}
    \bigbreak \noindent 
    a.) The first and third distance axioms do not hold. Observe that $d(12) = -1$ and $d(12) = -1$ while $d(21) =1$
    \bigbreak \noindent 
    Moreover, the second and third incidence axioms do not hold. Observe that the line $\{1\}$ contains only one point, and the pair of points $\{1,3\} $ are in no line.
    \bigbreak \noindent 
    b.) The set of distances is $\mathbb{D} = \{-2,-1,0,1,2\}$, which has $\omega = \text{sup}\ \mathbb{D} = 2$

    

    \bigbreak \noindent 
    \begin{mdframed}
        5. Give an example of a plane (which satisfies the first seven axioms) in which all the points are collinear
    \end{mdframed}
    \bigbreak \noindent 
    The following plane satisfies the first seven axioms. Let $\mathbb{P} = \{A,B,C,D\}, \mathbb{L} = \{\{A,B,C,D\}, \{A,D\}\}$, with distance function
    \begin{align*}
        \begin{array}{c|cccc}
          &A&B&C&D \\
            A&0&1&3&4 \\
            B& 1&0&2&3\\
            C& 3&2&0&1\\
            D& 4&3&1&0 
       \end{array}
    \end{align*}
    Satisfies all three distance axioms and the four incidence axioms. Note that line $\{A,D\}$ is contained within $\{A,B,C,D\} $
    \pagebreak \bigbreak \noindent
    \begin{figure}[ht]
        \centering
        \incfig{line2}
        \label{fig:line2}
    \end{figure}



\end{document}
