 \documentclass{report}
 
 \input{~/dev/latex/template/preamble.tex}
 \input{~/dev/latex/template/macros.tex}
 
 \title{\Huge{}}
 \author{\huge{Nathan Warner}}
 \date{\huge{}}
 \fancyhf{}
 \rhead{}
 \fancyhead[R]{\itshape Warner} % Left header: Section name
 \fancyhead[L]{\itshape\leftmark}  % Right header: Page number
 \cfoot{\thepage}
 \renewcommand{\headrulewidth}{0pt} % Optional: Removes the header line
 %\pagestyle{fancy}
 %\fancyhf{}
 %\lhead{Warner \thepage}
 %\rhead{}
 % \lhead{\leftmark}
 %\cfoot{\thepage}
 %\setborder
 % \usepackage[default]{sourcecodepro}
 % \usepackage[T1]{fontenc}
 
 % Change the title
 \hypersetup{
     pdftitle={}
 }

 \geometry{
  left=1in,
  right=1in,
  top=1in,
  bottom=1in
}
 
 \begin{document}
     % \maketitle
     %     \begin{titlepage}
     %    \begin{center}
     %        \vspace*{1cm}
     % 
     %        \textbf{}
     % 
     %        \vspace{0.5cm}
     %         
     %             
     %        \vspace{1.5cm}
     % 
     %        \textbf{Nathan Warner}
     % 
     %        \vfill
     %             
     %             
     %        \vspace{0.8cm}
     %      
     %        \includegraphics[width=0.4\textwidth]{~/niu/seal.png}
     %             
     %        Computer Science \\
     %        Northern Illinois University\\
     %        United States\\
     %        
     %             
     %    \end{center}
     % \end{titlepage}
     % \tableofcontents
    \pagebreak \bigbreak \noindent
    Nate Warner \ \quad \quad \quad \quad \quad \quad \quad \quad \quad \quad \quad \quad  MATH 353 \quad  \quad \quad \quad \quad \quad \quad \quad \quad \ \ \quad \quad Spring 2025
    \begin{center}
        \textbf{Problem set 6 - Due: Wednesday, March 5}
    \end{center}
    \bigbreak \noindent 
    \begin{mdframed}
        1. Prove that if $0 < AB < \omega$, $X \ne Y$, $ A\text{-}X\text{-}B$ and $ A\text{-}Y\text{-}B$, then either $ A\text{-}X\text{-}Y\text{-}B$ or $ A\text{-}Y\text{-}X\text{-}B$. Does the same conclusion follow if $AB = \omega $?
    \end{mdframed}
    \bigbreak \noindent 
    \textbf{\textit{Proof.}} Assume $0 < AB < \omega$, $X \ne Y$, $ A\text{-}X\text{-}B$ and $ A\text{-}Y\text{-}B$. Since $0 < AB < \omega$, $\overrightarrow{AB}$ is defined.
    \bigbreak \noindent 
    $ A\text{-}X\text{-}B$ and $ A\text{-}Y\text{-}B$ together with $X \ne Y$ implies $A,B,X,Y$ are distinct, collinear. Thus, we can apply theorem 8.3, which says one of the following relations must hold.
    \begin{align*}
       A\text{-}X\text{-}Y \quad \text{ or } \quad A\text{-}Y\text{-}X 
    \end{align*}
    Assume $ A\text{-}X\text{-}Y$, then $ A\text{-}Y\text{-}B$ with the ROI yields $ A\text{-}X\text{-}Y\text{-}B $
    \bigbreak \noindent 
    Assume $ A\text{-}Y\text{-}X$, then $ A\text{-}X\text{-}B$ with the ROI yields $ A\text{-}Y\text{-}X\text{-}B$ \endpf
    \bigbreak \noindent 
    \textbf{Note:} If $AB = \omega$, then ray $\overrightarrow{AB}$ would not be defined, and we would not be able to invoke theorem 8.3. Thus, the same conclusion would not follow.

    \bigbreak \noindent 
    \begin{mdframed}
        2. Prove proposition 8.11
    \end{mdframed}
    \bigbreak \noindent 
    \textbf{Proposition 8.11} Let $A,B$ be any two points on line $m$, with $0 < AB <\omega$. Then, there exists a point $C$ on $m$ with $ C\text{-}A\text{-}B$ and $ CB < \omega$.
    \bigbreak \noindent 
    \textbf{\textit{Proof.}} Assume $A,B$ are any two points on line $m$, with $0 < AB < \omega$. Thus, $\overrightarrow{AB}$ and $\overrightarrow{BA}$ are defined. We let our choice of $C$ be on the ray $\overrightarrow{BA}$ but not in $\overline{BA}$. We note that there are infinitely many choices for $C$. By Ax.RR, there exists a choice of $C$ on the ray $\overrightarrow{BA}$ such that $BC = CB < \omega$.
    \bigbreak \noindent 
    Since $C$ on $\overrightarrow{BA}$ but not in $\overline{BA}$, $ C\text{-}A\text{-}B$ by definition of the ray $\overrightarrow{BA}$. \endpf
    \bigbreak \noindent 

    
    \pagebreak \bigbreak \noindent 
    \begin{mdframed}
        3. Show via the following steps that $\mathbb{H}$ satisfies axiom RR.
        \bigbreak \noindent 
        Let $M = (r, mr+b)$ and $N = (t, mt+b)$ be the points of intersection of the line $ y=mx+b$ with the unit circle, $r < t$. So $l = \{(x,mx+b):\ r < x < t\} $ is a (typical nonvertical) line in $\mathbb{H}$. Let $A = (a,ma+b), C = (c,mc+b)$ be two points on $l$. We will assume $a<c$, so that $r<a<c<t$
        \begin{enumerate}[label=(\alph*)]
            \item Show that $X = (x,y)$ is on $\overrightarrow{AC} $ (in $\mathbb{H}$) if and only if $a \leq x < t$ and $y=mx+b$
            \item For $X = (x,y) \ne A$ on $\overrightarrow{AC}$, show 
                \begin{align*}
                    AX = \ln{\left(\frac{(t-a)(x-r)}{(a-r)(t-x)}\right)}
                \end{align*}
            \item For any real number $s > 0 $, show that there exists an $x$ with $ a < x < t$ such that
                \begin{align*}
                    \ln{\left(\frac{(t-a)(x-r)}{(a-r)(t-x)}\right)} = s
                \end{align*}
                and hence $AX  = s$ for $X = (x,mx+b) \in \overrightarrow{AC}$
        \end{enumerate}
    \end{mdframed}
    \bigbreak \noindent 
    a.) We show part (a) in two parts. First (1) that $X = (x,y) \in \overrightarrow{AC}  \implies a \leq x < t$, and $y = mx+b$. Then (2), $a \leq x < t$ and $y = mx+b  \implies X = (x,y) \in \overrightarrow{AC}$
    \bigbreak \noindent 
    (1) assume $X = (x,y)$ exists on the ray $\overrightarrow{AC}$. $X = A$, $X = C$, or one of $ A\text{-}X\text{-}C$, $ A\text{-}C\text{-}X$. If $X = A$, then $(x,y) = (a, ma+b)$ implies $x =a$, and $y = ma +b = mx + b$. Thus, $a \leq x < t$ and $y = mx+b$ are satisfied. If $X = C$, then $(x,y) = (c, mc+b)$ implies $x = c$, and $y = mc + b = mx + b$. Since $t > c > a$, $a \leq x < t$ and $y=mx+b$ are satisfied.
    \bigbreak \noindent 
    Assume $X \ne A$ or $C$. Then, $X \in \overrightarrow{AC}$ implies one of $ A\text{-}X\text{-}C$ or $ A\text{-}C\text{-}X$. Assume $ A\text{-}X\text{-}C$. Since $ t> c > a$ $x$ must live somewhere between $a$ and $c$ for $ A\text{-}X\text{-}C $ to be satisfied on the hyperbolic plane. Thus, $ a < x < c$ satisfies $a \leq x < t$. Next, since $ A\text{-}X\text{-}C$, $X$ is collinear with $A$ and $C$, which implies $y = mx+b$ is satisfied.
    \bigbreak \noindent 
    Assume $ A\text{-}C\text{-}X$. Again, since $t > c > a$, $ A\text{-}C\text{-}X$ implies $ c < x < t$, which satisfies $a \leq x < t$. Also, $X$ is collinear with $A$ and $C$, thus $y = mx+b$ is also satisfied.
    \bigbreak \noindent 
    (2) Assume $a \leq x < t$ and $y=mx+b$. Since $y = mx+b$, $X$ is collinear with $A$ and $C$. Thus, $X$ exists somewhere on the line $\overleftrightarrow{AC}$. But, since $a \leq x < t$, $X$ must be somewhere between $A$ and $N$. Notice that this is precisely the definition of the ray $\overrightarrow{AC} = \{A,C\} \cup \{X: A\text{-}X\text{-}C\} \cup \{X: A\text{-}C\text{-}X\}$ on the hyperbolic line $\overleftrightarrow{AC}$. Thus, for $a \leq x < t$ to be satisfied, one of $ A\text{-}X\text{-}C$ or $ A\text{-}C\text{-}X$ must be true.
    \bigbreak \noindent 
    Note that $x = a$ or $x=c$ implies $X = A$ or $X = C$, both of which make $X$ be on the ray $\overrightarrow{AC}$, as desired.
    \bigbreak \noindent 
    b.) Assume $X = (x,y) \ne A$ on $\overrightarrow{AC}$, then 
    \begin{align*}
        AX &= \ln{\left(\frac{e(AN)e(XM)}{e(AM)e(XN)}\right)} = \ln{\left(\frac{\left\lvert a-t \right\rvert \cdot \left\lvert x-r \right\rvert}{\left\lvert a-r \right\rvert \cdot \left\lvert x-t \right\rvert}\right)}
    \end{align*}
    But, since $r < a \leq x <  c < t$, we have
    \begin{align*}
        AX = \ln{\left(\frac{(t-a)(x-r)}{(a-r)(t-x)}\right)}
    \end{align*}
    As desired.
    \bigbreak \noindent 
    c.) We first note that as $x \to t$, $(t-x) \to 0 \implies \frac{(t-a)(x-r)}{(a-r)(t-x)} \to +\infty \implies \ln{\left(\frac{(t-a)(x-r)}{(a-r)(t-x)}\right)} \to +\infty$. Next, we note that as $x \to a$, $\frac{(t-a)(x-r)}{(a-r)(t-x)} \to 1$, since $ \frac{(t-a)(x-r)}{(a-r)(t-x)}  = \frac{(t-a)(a-r)}{(a-r)(t-a)} = 1$ when $x = a$. Therefore the domain of $AX = \ln{\left(\frac{(t-a)(x-r)}{(a-r)(t-x)}\right)} $ is $(1,\infty) $
    \bigbreak \noindent 
    Since the natural log function is continuous over $[1,\infty)$ and maps $[1,\infty) \to (0, \infty) $, any $s >0$ has an $x$ satisfying $a < x < t$ such that 
    \begin{align*}
        AX = \ln{\left(\frac{(t-a)(x-r)}{(a-r)(t-x)}\right)} = s
    \end{align*}
    \endpf


    \pagebreak \bigbreak \noindent 
    \begin{mdframed}
        4. Let $A = (0,0)$ and $B = (.8,0)$ in $\mathbb{H}$, and compute the midpoint of $\overline{AB}$ (It's not (.4,0))
    \end{mdframed}
    \bigbreak \noindent 
    We require a point $K$ such that $AK = KB = \frac{1}{2}AB$. First, we compute $d_{\mathbb{H}}(AB)$. If $A = (0,0) $, and $B = (0.8,0)$, then $M = (-1,0)$ and $N = (1,0)$. Further, distance is given by $e(PQ) = \left\lvert x_{1} - x_{2} \right\rvert $ for all points on the line $y = 0$ collinear with $A$ and $B$
    \begin{align*}
        d_{\mathbb{H}}(AB) &= \ln{\left(\frac{e(AN)e(BM)}{e(AM)e(BN)}\right)} = \ln{\left(\frac{1\left\lvert 0.8-(-1) \right\rvert}{1\left\lvert 0.8-1 \right\rvert}\right)} \\
        &= \ln{\left(\frac{1.8}{0.2}\right)} = \ln{\left(9\right)}
    \end{align*}
    Thus, we require $K$ such that $AK = KB = \frac{1}{2}AB = \frac{1}{2}\ln{\left(9\right)}  = \ln{\left(9^{\frac{1}{2}}\right)} = \ln{\left(3\right)}$. That is,
    \begin{align*}
        d_{\mathbb{H}}(AK) &= \ln{\left(\frac{e(AN)e(KM)}{e(AM)e(KN)}\right)} = \ln{\left(3\right)} \\
        \implies \frac{e(AN)e(KM)}{e(AM)e(KN)} &= 3
    \end{align*}
    Note that $e(AN) = e(AM) = 1$. Thus, if $K = (x,0)$ for $ 0 < x < 0.8 $
    \begin{align*}
        \frac{e(KM)}{e(KN)} &= 3 \\
        \implies \frac{\left\lvert x+1 \right\rvert}{\left\lvert x-1 \right\rvert} &= 3 \\
        \implies \frac{\sqrt{(x+1)^{2}}}{\sqrt{x-1}^{2}} &= 3 \\
        \implies \sqrt{(x+1)^{2}} &= 3\sqrt{(x-1)^{2}} \\
        \implies (x+1)^{2} &= 9(x-1)^{2} \\
        \implies x^{2} + 2x + 1 &= 9x^{2} -18x + 9 \\
        \implies 8x^{2} - 20x + 8 &= 0
    \end{align*}
    By the quadratic formula
    \begin{align*}
        x &= \frac{20 \pm \sqrt{20^{2} - 4(8)(8)}}{2(8)} \\
          &= \frac{20\pm 12}{16}
    \end{align*}
    Thus, $x = \frac{1}{2}, 2$. Observe that since $2 > 1 > 0.8$, it cannot be a solution. Thus, $x = \frac{1}{2}, d_{\mathbb{H}}(AK) = \ln{\left(3\right)}$ and the midpoint is therefore $K = (0.5,0)$. We quickly verify that $d_{\mathbb{H}}(KB) = \ln{\left(3\right)}$
    \begin{align*}
        d_{\mathbb{H}}(KB) &= \ln{\left(\frac{e(KN)e(BM)}{e(KM)e(BN)}\right)} = \ln{\left(\frac{\left\lvert 0.5-1 \right\rvert \cdot \left\lvert 0.8 + 1 \right\rvert}{\left\lvert 0.5 + 1 \right\rvert\cdot  \left\lvert 0.8 - 1 \right\rvert} \right)} = \ln{\left(3\right)}
    \end{align*}
    Thus, $d_{\mathbb{H}}(AK) = d_{\mathbb{H}}(KB) = \frac{1}{2}AB = \ln{\left(3\right)} $, and $K = (0.5,0)$ is the midpoint of $\overline{AB} $





\end{document}
