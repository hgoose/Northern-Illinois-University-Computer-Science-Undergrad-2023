 \documentclass{report}
 
 \input{~/dev/latex/template/preamble.tex}
 \input{~/dev/latex/template/macros.tex}
 
 \title{\Huge{}}
 \author{\huge{Nathan Warner}}
 \date{\huge{}}
 \fancyhf{}
 \rhead{}
 \fancyhead[R]{\itshape Warner} % Left header: Section name
 \fancyhead[L]{\itshape\leftmark}  % Right header: Page number
 \cfoot{\thepage}
 \renewcommand{\headrulewidth}{0pt} % Optional: Removes the header line
 %\pagestyle{fancy}
 %\fancyhf{}
 %\lhead{Warner \thepage}
 %\rhead{}
 % \lhead{\leftmark}
 %\cfoot{\thepage}
 %\setborder
 % \usepackage[default]{sourcecodepro}
 % \usepackage[T1]{fontenc}
 
 % Change the title
 \hypersetup{
     pdftitle={}
 }

 \geometry{
  left=1in,
  right=1in,
  top=1in,
  bottom=1in
}
 
 \begin{document}
     % \maketitle
     %     \begin{titlepage}
     %    \begin{center}
     %        \vspace*{1cm}
     % 
     %        \textbf{}
     % 
     %        \vspace{0.5cm}
     %         
     %             
     %        \vspace{1.5cm}
     % 
     %        \textbf{Nathan Warner}
     % 
     %        \vfill
     %             
     %             
     %        \vspace{0.8cm}
     %      
     %        \includegraphics[width=0.4\textwidth]{~/niu/seal.png}
     %             
     %        Computer Science \\
     %        Northern Illinois University\\
     %        United States\\
     %        
     %             
     %    \end{center}
     % \end{titlepage}
     % \tableofcontents
    \pagebreak \bigbreak \noindent
    Nate Warner \ \quad \quad \quad \quad \quad \quad \quad \quad \quad \quad \quad \quad  MATH 353 \quad  \quad \quad \quad \quad \quad \quad \quad \quad \ \ \quad \quad Spring 2025
    \begin{center}
        \textbf{Problem set 9 - Due: Monday, March 24}
    \end{center}
    \bigbreak \noindent 
    \begin{mdframed}
        1. Assume $\omega < \infty $. Show that if $A^{*} $ is the antipode of $A$ and $B$ is any other point, then $ A\text{-}B\text{-}A^{*} $ and $BA^{*} = \omega - AB $
    \end{mdframed}
    \bigbreak \noindent 
    \textbf{\textit{Proof.}} Assume $\omega < \infty$. Let $A^{*}$ be the antipode of $A$ in $\mathbb{P}$. Let $B$ be any other point.
    \bigbreak \noindent 
    Let $m$ be the line that contains $B$ and $A$. By theorem 10.8, every line through $A$ goes through $A^{*}$ as well. Thus, $A^{*}\in m$. So, $A,A^{*}, B$ are distinct, collinear, and by theorem 9.1, $A\text{-}B\text{-}A^{*}$
    \bigbreak \noindent 
    Since $ A\text{-}B\text{-}A^{*}$, we have $AB + BA^{*} = AA^{*} = \omega$. Thus, $ BA^{*} = \omega - AB$ \endpf

    \bigbreak \noindent 
    \begin{mdframed}
        5. Suppose $P$ is a point not on a line $m = \overleftrightarrow{AB}$, and suppose $X$ and $Y$ are points with $ A\text{-}X\text{-}P$ nd $ P\text{-}B\text{-}Y $. Show that $XY < \omega$ and that $\overline{XY} $ meets $m$
    \end{mdframed}
    \bigbreak \noindent 
    \textbf{\textit{Proof.}} Assume $P$ be a point not on a line $m = \overleftrightarrow{AB}$, and let $X,Y$ be points with $ A\text{-}X\text{-}P$ and $ P\text{-}B\text{-}Y$. 
    \bigbreak \noindent 
    By Ax.S, line $m$ there exists a pair of opposite halfplanes with edge $m$, call them $H,K$. Let $H$ be the halfplane that contains $P$. Let $\overleftrightarrow{AP}$ be the line through $A,$ and $P$. Note that since $ A\text{-}X\text{-}P$, $A,X \in \overleftrightarrow{AP}$. Observe that $A,B,P$ are three noncollinear points. Thus, By proposition noncollinear, each of $AB,AP,BP < \omega$. Thus, $\overleftrightarrow{AP}$ is the unique line through $A$ and $X$. Hence, $x \not\in m$. 
    \bigbreak \noindent 
    Notice that since $ P\text{-}B\text{-}Y$, $\overrightarrow{BP}, \overrightarrow{BY}$ are opposite rays by Thm 9.6. Since $P \in H$, $B \in m$ Thm 10.3 tell us that $\text{Int}\overrightarrow{BP} \subseteq H$, $\text{Int}\overrightarrow{BY} \subseteq K$. Since $Y \in \text{Int}\overrightarrow{BY}, Y \in K$.
    \bigbreak \noindent 
    Further, note that $X \in \overrightarrow{AP}$ by definition of $ A\text{-}X\text{-}P $. Since $A \in m$, $P\in H$, Thm 10.3 suggests $\text{Int}\overrightarrow{AP} \subseteq H$. Since $X \in \text{Int}\overrightarrow{AP}$, $X$ is therefore a member of $H$. 
    \bigbreak \noindent 
    Thus, we have $X \in H$, $Y \in K$. We noted previously that $X \in \overleftrightarrow{AP}$, which is the unique line through $A,X$ and hence the only line that contains $A,X$. But what about $Y$?
    \bigbreak \noindent 
    First, since $ P\text{-}B\text{-}Y$, $Y \in \overrightarrow{PB}$. Call the carrier of this ray $\overleftrightarrow{PB}$. We saw above that by proposition noncollinear, $PB < \omega$. Thus, $\overleftrightarrow{PB}$ is the unique line through $P,B$, and hence the unique line through $P,Y$. Thus, $Y$ is contained only in this line.
    Since $ \overleftrightarrow{AP} \ne \overleftrightarrow{PB}$, $X,P,Y$ are three noncollinear points, and by proposition noncollinear, $XY < \omega$. 
    \bigbreak \noindent 
    Since $X \in H$, $Y \in K$, $XY < \omega$. By the definition of opposite halfplanes with edge a line $m$, $\overline{XY} \cap m \ne \varnothing$. Thus, $\overline{XY}$ meets $m$ \endpf



    \pagebreak \bigbreak \noindent 
    \begin{mdframed}
        6. Let $m$ be a line and $P,Q$ points such that $P \not\in m$, $PQ = 1$, and $PX \geq 2$ for all $X$ on $m$. Prove that $P$ and $Q$ lie on the same side of $m$.
    \end{mdframed}
    \bigbreak \noindent 
    \textbf{\textit{Proof.}} Assume $m$ is a line, and $P,Q$ are points such that $P\not\in m$, $PQ = 1$, and $PX \geq 2$ for all $X$ on $m$
    \bigbreak \noindent 
    By Ax.S, there exists a pair of opposite halfplanes with edge $m$, call them $H,K$. Let $H$ be the halfplane that contains $P$. That is, $P \in H$.
    \bigbreak \noindent 
    Assume for the sake of contradiction that $Q \in m$. Since $Q \in m$, $PQ \geq 2$, which contradicts $PQ = 1$. Thus, $Q \not\in m$.
    \bigbreak \noindent 
    Further assume that $Q \in K$. That is, $P,Q$ on opposite sides of $m$. Then, by theorem 10.6, there exists an $X \in m$ such that $ P\text{-}X\text{-}Q$, which implies $PX + XQ = PQ$, and thus $PX < PQ$, which again is a contradiction since $PX \geq 2$ and $PQ  =1$. 
    \bigbreak \noindent 
    Thus, $Q$ must also lie in $H$, and $P,Q$ are therefore both on the same side of $m$ \endpf

    \bigbreak \noindent 
    \begin{mdframed}
        8. Prove Theorem 10.10        
    \end{mdframed}
    \bigbreak \noindent 
    \begin{remark}
        \textit{(Theorem 10.10 (Pasch's theorem))} :
        Let $A,B,C$ be three noncollinear points. Let $X$ be a point with $ B\text{-}X\text{-}C $, and $m$ a line through $X$ but not through $A,B,$ or $C$. Then, exactly one of
        \begin{enumerate}
            \item $m$ contains a point $Y$ with $ A\text{-}Y\text{-}C$
            \item $m$ contains a point $Z$ with $ A\text{-}Z\text{-}B $
        \end{enumerate}

    \end{remark}
    \bigbreak \noindent 
    \textbf{\textit{Proof.}} Let $A,B,C$ be three noncollinear points. Let $X$ be a point with $ B\text{-}X\text{-}C$, and $m$ a line through $X$ but not through $A,B,$ or $C$.
    \bigbreak \noindent 
    First, we observe that since $A,B,C$ are three noncollinear points, each of $AB,AC,BC < \omega$
    \bigbreak \noindent 
    By Ax.S, $m$ there exists a pair of opposite halfplanes $H,K$ with edge $m$. Since $m$ does not go through $A$, $A$ must lie in one of the halfplanes. Without loss of generality, assume $A \in H$
    \bigbreak \noindent 
    Consider $B,C$, since $ B\text{-}X\text{-}C$, $X\in m$ we conclude by theorem 10.6 that $B,C$ lie in opposite sides of $m$. Thus, either $B$ with $A$ in $H$, or $C$ with $A$ in $H$. 
    \bigbreak \noindent 
    First, consider $B$ with $A$ in $H$. Thus, $C \in K$. Since $A \in H$, $C \in K$, $AC < \omega$, we have by the definition of opposite halfplanes, $\overline{AC} \cap m \ne \varnothing$. Thus, $ \overline{AC}$ intersects $m$, call the point of intersection $Y$. By the definition of the intersection, $Y \in \overline{AC} \cap m$. Thus, $Y \in \overline{AC}$. By the definition of the segment $\overline{AC}$, $ A\text{-}Y\text{-}C$.  
    \bigbreak \noindent 
    Note that since $B \in H$ with $A$, this argument does not hold for the segment $\overline{AB}$, and we can generate no such point $Z$ such that $ A\text{-}Z\text{-}B$
    \bigbreak \noindent 
    The same argument but with $A,C \in H$, $B \not\in H \implies B \in K$ generates a point $Z$ such that $ A\text{-}Z\text{-}B$.  \endpf
    





\end{document}
