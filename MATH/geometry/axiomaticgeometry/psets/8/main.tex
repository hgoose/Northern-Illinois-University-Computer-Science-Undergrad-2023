 \documentclass{report}
 
 \input{~/dev/latex/template/preamble.tex}
 \input{~/dev/latex/template/macros.tex}
 
 \title{\Huge{}}
 \author{\huge{Nathan Warner}}
 \date{\huge{}}
 \fancyhf{}
 \rhead{}
 \fancyhead[R]{\itshape Warner} % Left header: Section name
 \fancyhead[L]{\itshape\leftmark}  % Right header: Page number
 \cfoot{\thepage}
 \renewcommand{\headrulewidth}{0pt} % Optional: Removes the header line
 %\pagestyle{fancy}
 %\fancyhf{}
 %\lhead{Warner \thepage}
 %\rhead{}
 % \lhead{\leftmark}
 %\cfoot{\thepage}
 %\setborder
 % \usepackage[default]{sourcecodepro}
 % \usepackage[T1]{fontenc}
 
 % Change the title
 \hypersetup{
     pdftitle={}
 }

 \geometry{
  left=1in,
  right=1in,
  top=1in,
  bottom=1in
}
 
 \begin{document}
     % \maketitle
     %     \begin{titlepage}
     %    \begin{center}
     %        \vspace*{1cm}
     % 
     %        \textbf{}
     % 
     %        \vspace{0.5cm}
     %         
     %             
     %        \vspace{1.5cm}
     % 
     %        \textbf{Nathan Warner}
     % 
     %        \vfill
     %             
     %             
     %        \vspace{0.8cm}
     %      
     %        \includegraphics[width=0.4\textwidth]{~/niu/seal.png}
     %             
     %        Computer Science \\
     %        Northern Illinois University\\
     %        United States\\
     %        
     %             
     %    \end{center}
     % \end{titlepage}
     % \tableofcontents
    \pagebreak \bigbreak \noindent
    Nate Warner \ \quad \quad \quad \quad \quad \quad \quad \quad \quad \quad \quad \quad  MATH 353 \quad  \quad \quad \quad \quad \quad \quad \quad \quad \ \ \quad \quad Spring 2025
    \begin{center}
        \textbf{Problem set 10 - Due: Friday, March 28}
    \end{center}
    \bigbreak \noindent 
    \begin{mdframed}
        1. Assume $\omega < \infty$. Suppose that $A,B,C$ are points with $AC < \omega $ and $ A\text{-}B\text{-}C$. Let $X$ be any point not on $\overleftrightarrow{AC}$ and let $A^{*}$ be the antipode of $A$. Prove that $ \overrightarrow{XB}\text{-}\overrightarrow{XC}\text{-}\overrightarrow{XA^{*}} $ 
    \end{mdframed}
    \bigbreak \noindent 
    \textbf{\textit{Proof.}} Assume $\omega < \infty$, $A,B,C$ are points with $AC < \omega$, and $ A\text{-}B\text{-}C$. Let $X$ be any point not on $\overleftrightarrow{AC}$, and let $A^{*}$ be the antipode of $A$.
    \bigbreak \noindent 
    First, let $\overrightarrow{XA} = n, \overrightarrow{XB} = j, \overrightarrow{XC} = \ell, \overrightarrow{XA^{*}} = k$. Thus, we aim to show that $ j\text{-}\ell\text{-}k$. We first note that by Ax.C, we have
    \begin{align*}
        n\text{-}j\text{-}\ell
    .\end{align*}
    Next, we observe that by theorem 9.1, $ A\text{-}C\text{-}A^{*}$, and by Ax.C, $ \overrightarrow{XA}\text{-}\overrightarrow{XC}\text{-}\overrightarrow{XA^{*}} $, or $ n\text{-}\ell\text{-}k$. Thus, we have
    \begin{align*}
        n\text{-}j\text{-}\ell \quad \text{and} \quad n\text{-}\ell\text{-}k
    .\end{align*}
    Which by the rule of insertion, gives us
    \begin{align*}
        n\text{-}j\text{-}\ell\text{-}k
    .\end{align*}
    Which yields $ j\text{-}\ell\text{-}k = \overrightarrow{XB}\text{-}\overrightarrow{XC}\text{-}\overrightarrow{XA^{*}}$ as desired \endpf

    \bigbreak \noindent 
    \begin{mdframed}
        2. Prove Theorem 11.9
    \end{mdframed}
    \bigbreak \noindent 
    \begin{remark}
        (\textit{Theorem 11.9 Almost uniqueness of quadrichotomy for rays}): Suppose that $a,b,c,r$ are distinct rays in a pencil $P$, and that $ a\text{-}b\text{-}c$. Then, \textbf{exactly} one of 
        \begin{align*}
            r\text{-}a\text{-}b \quad a\text{-}r\text{-}b \quad b\text{-}r\text{-}c \quad b\text{-}c\text{-}r
        \end{align*}
        With the exception that both $ r\text{-}a\text{-}b $ and $ b\text{-}c\text{-}r$ are true when $r = b^{\prime} $
        \bigbreak \noindent 
        (\textit{Dual of Theorem 8.3}): Let $x \ne y$ by rays distinct from ray $a$ on the fan $ \overrightarrow{ab}$. Then, exactly one of the following relations must hold.
        \begin{align*}
            a\text{-}x\text{-}y \quad \text{or} \quad a\text{-}y\text{-}x
        .\end{align*}
    \end{remark}
    \bigbreak \noindent 
    \textbf{\textit{Proof}}: We proceed by dualizing the proof of theorem 9.2.
    \bigbreak \noindent 
    By Axiom.QR, at least one of 
    \begin{align*}
        r\text{-}a\text{-}b \quad a\text{-}r\text{-}b \quad b\text{-}r\text{-}c \quad b\text{-}c\text{-}r
    .\end{align*}
    Suppose we have $ a\text{-}r\text{-}b$. Then, $ a\text{-}b\text{-}c$ and the rule of insertion yields $ a\text{-}r\text{-}b\text{-}c $
    \bigbreak \noindent 
    So, $ a\text{-}r\text{-}b $ and $ r\text{-}b\text{-}c$ are true. Which, by the UMT guarantees that both $ b\text{-}r\text{-}c $ and $ b\text{-}c\text{-}r$ are false. 
    \bigbreak \noindent 
    Next, suppose that $ b\text{-}r\text{-}c$ is true. Then, $ a\text{-}b\text{-}c$ and the rule of insertion yields $ a\text{-}b\text{-}r\text{-}c$. So, $ a\text{-}b\text{-}r$ and $ b\text{-}r\text{-}c$ are true, and by the UMT, all three of  $ r\text{-}a\text{-}b$, $ a\text{-}r\text{-}b$, $ b\text{-}c\text{-}r$ are false. Thus, none of the other three relations hold.
    \bigbreak \noindent 
    So, if more than one of $ r\text{-}a\text{-}b, a\text{-}r\text{-}b, b\text{-}r\text{-}c, b\text{-}c\text{-}r$ holds, they must be exactly $ r\text{-}a\text{-}b$ and $ b\text{-}c\text{-}r$
    \bigbreak \noindent 
    Assume that $ r\text{-}a\text{-}b$ and $ b\text{-}c\text{-}r$ are true. Suppose toward a contradiction that $br < 180$. Then, fan $\overrightarrow{br}$ is defined, and $ r\text{-}a\text{-}b, b\text{-}c\text{-}r$ implies $a,c$ are in $ \overrightarrow{br}$. By the dual of theorem 8.3 (stated above), one of 
    \begin{align*}
        b\text{-}a\text{-}c \quad \text{or} \quad b\text{-}c\text{-}a
    \end{align*}
    is true. But, this contradicts $ a\text{-}b\text{-}c$ by the UMT. 
    \bigbreak \noindent 
    Therefore, $br = 180$, hence $ r = b^{\prime}$. \endpf
    

    \bigbreak \noindent 
    \begin{mdframed}
        3. Prove Theorem 11.10
    \end{mdframed}
    \bigbreak \noindent 
    \begin{remark}
       \textit{(Theorem 11.10: Opposite Fan Theorem)}. Let $p,q,r$ be rays in pencil $P$ such that $ q\text{-}p\text{-}r$. Then, $ \overrightarrow{pq} \cup \overrightarrow{pr} = P$, and $ \overrightarrow{pq} \cap \overrightarrow{pr} = \{p,p^{\prime}\} $
    \end{remark}
    \bigbreak \noindent 
    \textbf{\textit{Proof.}} $ p,q,r$ are together in the unique pencil $P$. Further, $ q\text{-}p\text{-}r$ implies $pq, pr< qr \leq 180 $, so fans $\overrightarrow{pq}, \overrightarrow{pr}$ are defined.
    \bigbreak \noindent 
    If $x \ne p,q,r$ is in pencil $P$, then ax.QR says one of 
    \begin{align*}
        x\text{-}q\text{-}p \quad q\text{-}x\text{-}p \quad p\text{-}x\text{-}r \quad p\text{-}r\text{-}x
    \end{align*}
    must be satisfied. In other words, $x$ is in $\overrightarrow{pq}$ or $\overrightarrow{pr}$. So, $P \subseteq \overrightarrow{pq} \cup \overrightarrow{pr} $. Hence, $P = \overrightarrow{pq} \cup \overrightarrow{pr} $
    \bigbreak \noindent 
    Since $\overrightarrow{pq}$ and $\overrightarrow{pr}$ have the same endpoint, and $ \overrightarrow{pq} \cup \overrightarrow{pr} = P$, $\overrightarrow{pq}$ and $\overrightarrow{pr}$ are opposite rays
    \bigbreak \noindent 
    What about $\overrightarrow{pq} \cap \overrightarrow{pr}$? $ q\text{-}p\text{-}r$ implies not $ p\text{-}q\text{-}r$ or $ p\text{-}r\text{-}q $, so $ q \not\in \overrightarrow{pr}$, and $ r \not\in \overrightarrow{pq}$. So, neither $q$ nor $r$ is in $\overrightarrow{pq} \cap \overrightarrow{pr}$
    \bigbreak \noindent 
    Let $x$ be any ray $\ne p,q,r$ in $P$. Suppose $X \in \overrightarrow{pq} \cap \overrightarrow{pr}$
    \begin{align*}
        &x \in \overrightarrow{pq} \implies x\text{-}q\text{-}p \text{ or } q\text{-}x\text{-}p \\
        &x \in \overrightarrow{pr} \implies p\text{-}x\text{-}r \text{ or } p\text{-}r\text{-}x
    .\end{align*}
    So two are true. Theorem 11.10 applied to $ q\text{-}p\text{-}r$ and ray $x$ implies it must be $ q\text{-}x\text{-}p$ and $ p\text{-}r\text{-}x$, with $ x = p^{\prime}$. Thus, $\overrightarrow{pq} \cap \overrightarrow{pr} = \{p,p^{\prime}\} $ \endpf
    


\end{document}
