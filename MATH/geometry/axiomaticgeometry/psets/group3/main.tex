\documentclass{report}

\input{~/dev/latex/template/preamble.tex}
\input{~/dev/latex/template/macros.tex}

\title{\Huge{}}
\author{\huge{Nathan Warner}}
\date{\huge{}}
\fancyhf{}
\rhead{}
\fancyhead[R]{\itshape Warner} % Left header: Section name
\fancyhead[L]{\itshape\leftmark}  % Right header: Page number
\cfoot{\thepage}
\renewcommand{\headrulewidth}{0pt} % Optional: Removes the header line
%\pagestyle{fancy}
%\fancyhf{}
%\lhead{Warner \thepage}
%\rhead{}
% \lhead{\leftmark}
%\cfoot{\thepage}
%\setborder
% \usepackage[default]{sourcecodepro}
% \usepackage[T1]{fontenc}

% Change the title
\hypersetup{
    pdftitle={G2}
}

\geometry{
  left=1in,
  right=1in,
  top=.5in,
  bottom=1in
}


\begin{document}
    % \maketitle
    %     \begin{titlepage}
    %    \begin{center}
    %        \vspace*{1cm}
    % 
    %        \textbf{G2}
    % 
    %        \vspace{0.5cm}
    %         
    %             
    %        \vspace{1.5cm}
    % 
    %        \textbf{Nathan Warner}
    % 
    %        \vfill
    %             
    %             
    %        \vspace{0.8cm}
    %      
    %        \includegraphics[width=0.4\textwidth]{~/niu/seal.png}
    %             
    %        Computer Science \\
    %        Northern Illinois University\\
    %        United States\\
    %        
    %             
    %    \end{center}
    % \end{titlepage}
    % \tableofcontents
    \begin{mdframed}
        8. Suppose that $A,B,C$ are three noncollinear points. Prove that there exsits a fourth point $D$ not on $\overleftrightarrow{AB}$ so that $\triangle ABC \cong \triangle ABD $
    \end{mdframed}
    \bigbreak \noindent 
    \textbf{\textit{Proof.}} Suppose that $A,B,C$ are three noncollinear points. If $D = C$, then $ \triangle ABC \cong \triangle ABD$ under the correspondence $ ABC \leftrightarrow ABD $ trivially, so we may assume that $ C \ne D$.
    \bigbreak \noindent 
    By Ax.S, $\overleftrightarrow{AB}$ generates a pair of opposite halfplanes $H,K$ with edge $\overleftrightarrow{AB}$. Since $A,B,C$ noncollinear, $C\not\in\overleftrightarrow{AB}$. Let $K$ be the halfplane that contains $C$
    \bigbreak \noindent 
    \begin{figure}[ht]
        \centering
        \incfig{f1}
        \label{fig:f1}
    \end{figure}
    \bigbreak \noindent 
    Consider the angle measure $\overrightarrow{AB}\overrightarrow{AC} = \angle BAC$. By theorem 12.3, there are two rays $j,k$ with endpoint $A$ and angle measure $\overrightarrow{AB}j = \overrightarrow{AB}k = \overrightarrow{AB}\overrightarrow{AC}$. Let $k$ be one the side that contains $C$ ($K$), by Theorem 11.6, $k = \overrightarrow{AC}$. Since $k^{0} \subseteq K, j^{0} \subseteq H$.
    \bigbreak \noindent 
    \begin{figure}[ht]
        \centering
        \incfig{f2}
        \label{fig:f2}
    \end{figure}
    \bigbreak \noindent 
    Consider the distance $AC$, by Theorem 8.6, each ray with endpoint $A$ has a unique point $X$ such that $AX = AC$, on ray $\overrightarrow{AC}$, $X = C$. Call the point in ray $j$ $D$
    \pagebreak \bigbreak \noindent 
    \begin{figure}[ht]
        \centering
        \incfig{f3}
        \label{fig:f3}
    \end{figure}
    \bigbreak \noindent 
    Since $D \in H$ implies $D \not\in \overleftrightarrow{AB}$, $A,B,D$ are three noncollinear points, and a triangle $ \triangle ABD$ is formed.
    \bigbreak \noindent 
    \begin{figure}[ht]
        \centering
        \incfig{f4}
        \label{fig:f4}
    \end{figure}
    \bigbreak \noindent 
    Consider the correspondence $ABC \leftrightarrow ABD$ between the vertices of triangles $\triangle ABC$ and $\triangle ABD$
    \bigbreak \noindent 
    Since $AD = AC$, $AB = AB$, and $\angle BAD =  \angle BAC$, we have $\overline{AD} \cong \overline{AC}, \overline{AB} \cong \overline{AB}, \underline{\angle BAC} \cong \underline{\angle BAD}$, and Ax.SAS implies that $ \triangle ABC \cong \triangle ABD $. \endpf

    \bigbreak \noindent 
    \begin{mdframed}
        9. Suppose that $\omega < \infty$, that $P,Q,R$ are noncollinear points with $P^{*} = $ antipode of $P$, and that $\angle PQR = 30$ and $\angle PRQ = 150$. Prove that $\triangle P^{*}QR \cong \triangle PRQ $ 
    \end{mdframed}
    \bigbreak \noindent 
    \textbf{\textit{Proof.}} Suppose that $\omega < \infty$, that $P,Q,R$ are noncollinear points with $P^{*} = $ antipode of $P$, and that $\angle PQR  = 30$, and $\angle PRQ = 150$
    \bigbreak \noindent 
    First, we note that  $QR = RQ$, which implies $ \overline{QR} = \overline{RQ} \cong \overline{RQ}$
    \bigbreak \noindent 
    Next, by Theorem 11.8, we have that $ \overrightarrow{RP}\text{-}\overrightarrow{RQ}\text{-} \overrightarrow{RP}^{\prime}$. By Coroll. 9.8, $\overrightarrow{RP}^{\prime} = \overrightarrow{RP^{*}}$, so $ \overrightarrow{RP}\text{-}\overrightarrow{RQ}\text{-}\overrightarrow{RP^{*}}$. Thus,
    \begin{align*}
        \overrightarrow{RP}\overrightarrow{RQ} + \overrightarrow{RQ}\overrightarrow{RP^{*}} = \overrightarrow{RP}\overrightarrow{RP^{*} } = 180
    .\end{align*}
    Since $ \overrightarrow{RP}\overrightarrow{RQ}= \angle PRQ$, and $ \overrightarrow{RQ}\overrightarrow{RP^{*}} = \angle QRP^{*}$, we have
    \begin{align*}
        \angle PRQ + \angle QRP^{*} &= 180 \\
        \implies \angle QRP^{*} &= 180 - \angle PRQ = 180 - 150 = 30
    .\end{align*}
    So, $ \angle PQR = \angle P^{*}RQ$, and thus $ \underline{\angle PQR} \cong \underline{\angle P^{*}RQ} $
    \bigbreak \noindent 
    Similarly $ \overrightarrow{QP}\text{-}\overrightarrow{QR}\text{-}\overrightarrow{QP^{*}}$ by Theorem 11.8 and Coroll.9.8,  so
    \begin{align*}
        \overrightarrow{QP}\overrightarrow{QR} + \overrightarrow{QR}\overrightarrow{QP^{*}} = \overrightarrow{QP}\overrightarrow{QP^{*}} = 180
    .\end{align*}
    Since $ \overrightarrow{QP}\overrightarrow{QR} = \angle PQR$, and $ \overrightarrow{QR}\overrightarrow{QP^{*}} = \angle RQP^{*}$, we have that
    \begin{align*}
        \angle PQR + \angle RQP^{*} &= 180  \\
        \implies \angle RQP^{*} &= 180 - \angle PQR = 180 - 30 = 150
    .\end{align*}
    Thus, $\angle P^{*}QR = 150$, which means $\angle P^{*}QR = \angle PRQ = 150$, and therefore $\underline{\angle P^{*}QR} \cong \underline{\angle PRQ} $
    
    \bigbreak \noindent 
    So, under the correspondence $PRQ \leftrightarrow P^{*}QR$ between the vertices of triangles $ \triangle PRQ$ and $ \triangle P^{*}QR$, by Theorem 13.1 (ASA), we have that 
    \begin{align*}
        \triangle PRQ \cong \triangle P^{*}QR
    .\end{align*}
    \endpf


    
    
\end{document}
