 \documentclass{report}
 
 \input{~/dev/latex/template/preamble.tex}
 \input{~/dev/latex/template/macros.tex}
 
 \title{\Huge{}}
 \author{\huge{Nathan Warner}}
 \date{\huge{}}
 \fancyhf{}
 \rhead{}
 \fancyhead[R]{\itshape Warner} % Left header: Section name
 \fancyhead[L]{\itshape\leftmark}  % Right header: Page number
 \cfoot{\thepage}
 \renewcommand{\headrulewidth}{0pt} % Optional: Removes the header line
 %\pagestyle{fancy}
 %\fancyhf{}
 %\lhead{Warner \thepage}
 %\rhead{}
 % \lhead{\leftmark}
 %\cfoot{\thepage}
 %\setborder
 % \usepackage[default]{sourcecodepro}
 % \usepackage[T1]{fontenc}
 
 % Change the title
 \hypersetup{
     pdftitle={}
 }

 \geometry{
  left=1in,
  right=1in,
  top=1in,
  bottom=1in
}
 
 \begin{document}
     % \maketitle
     %     \begin{titlepage}
     %    \begin{center}
     %        \vspace*{1cm}
     % 
     %        \textbf{}
     % 
     %        \vspace{0.5cm}
     %         
     %             
     %        \vspace{1.5cm}
     % 
     %        \textbf{Nathan Warner}
     % 
     %        \vfill
     %             
     %             
     %        \vspace{0.8cm}
     %      
     %        \includegraphics[width=0.4\textwidth]{~/niu/seal.png}
     %             
     %        Computer Science \\
     %        Northern Illinois University\\
     %        United States\\
     %        
     %             
     %    \end{center}
     % \end{titlepage}
     % \tableofcontents
    \pagebreak \bigbreak \noindent
    Nate Warner \ \quad \quad \quad \quad \quad \quad \quad \quad \quad \quad \quad \quad  MATH 353 \quad  \quad \quad \quad \quad \quad \quad \quad \quad \ \ \quad \quad Spring 2025
    \begin{center}
        \textbf{Problem set 5 - Due: Friday, Feb 14}
    \end{center}
    \bigbreak \noindent 
    \begin{mdframed}
        1. Show that for any three points $A,B,C$ on any line in $\mathbb{H}$,
        \begin{align*}
            \text{$A$-$B$-$C$ in } $\mathbb{E}$\  \iff\  \text{$A$-$B$-$C$ in } \mathbb{H}
        \end{align*}
    \end{mdframed}
    \bigbreak \noindent 
    We prove in two parts
    \begin{enumerate}[label=(\alph*)]
        \item $A\text{-}B\text{-}C \in \mathbb{E} \implies A\text{-}B\text{-}C \in \mathbb{H}$
        \item $A\text{-}B\text{-}C \in \mathbb{H} \implies A\text{-}B\text{-}C \in \mathbb{E}$
    \end{enumerate}
    \bigbreak \noindent 
    \textbf{\textit{Proof}} We begin by proving part (a). Assume $A\text{-}B\text{-}C$ is true in $\mathbb{E}$ for three distinct collinear points $A,B,C$. Thus,
    \begin{align*}
        AB + BC + AC
    \end{align*}

    \bigbreak \noindent 
    For $A\text{-}B\text{-}C$ (B between A and C) in the hypebolic plane (Poincare model), We require $d_{\mathbb{H}}(AB) + d_{\mathbb{H}}(BC) = d_{\mathbb{H}}(AC)$. That is,
    \begin{align*}
        \ln{\left(\frac{e(AN)e(BM)}{e(AM)e(BN)}\right)} + \ln{\left(\frac{e(BN)e(CM)}{e(BM)e(CN)}\right)} = \ln{\left(\frac{e(AN)e(CM)}{e(AM)e(CN)}\right)}
    \end{align*}
    We have 
    \begin{align*}
        &\ln{\left(\frac{e(AN)e(BM)}{e(AM)e(BN)}\right)} + \ln{\left(\frac{e(BN)e(CM)}{e(BM)e(CN)}\right)} \\
        &= \ln{\left(e(AN)\right)} + \ln{\left(e(BM)\right)} - \ln{\left(e(AM)\right)} - \ln{\left(e(BN)\right)} \\
        &+ \ln{\left(e(BN)\right)} + \ln{\left(e(CM)\right)} - \ln{\left(BM\right)} - \ln{\left(CN\right)} \\
        &=\ln{\left(e(AN)\right)} - \ln{\left(e(AM)\right)} + \ln{\left(e(CM)\right)} - \ln{\left(e(CN)\right)} \\
        &=\ln{\left(e(AN)\right)}+ \ln{\left(e(CM)\right)} - \ln{\left(e(AM)\right)}  - \ln{\left(e(CN)\right)} \\
        &= \ln{\left(\frac{e(AN)e(CM)}{e(AM)e(CN)}\right)} = d_{\mathbb{H}}(AC)
    \end{align*}
    \bigbreak \noindent 
    Thus, $A\text{-}B\text{-}C$ in $\mathbb{E}$ implies $A\text{-}B\text{-}C$ in $\mathbb{H}$. Similarly, $B\text{-}A\text{-}C$ in $\mathbb{E}$ implies $B\text{-}A\text{-}C $ in $\mathbb{H} $, and $A\text{-}C\text{-}B$ in $\mathbb{E}$ implies $A\text{-}C\text{-}B $ in $\mathbb{H}$
    \bigbreak \noindent 
    By the UMT, since $A\text{-}B\text{-}C$ occurs in $\mathbb{E}$, both $B\text{-}A \text{-}C$ and $A\text{-}C\text{-}B$ will not occur. Exactly one of them will occur, and each relation in $\mathbb{E}$ implies the same relation happens in $\mathbb{H} $
    \bigbreak \noindent 
    (b) If $A\text{-}B\text{-}C$ happens in $\mathbb{H}$, then by the UMT the other two do not. But since $A,B,C$ are distinct and collinear, one of them must occur in $\mathbb{E}$, so only $A\text{-}B\text{-}C$ will be true in $\mathbb{E}$ by the UMT \hspace*{\fill} $\blacksquare $

    \pagebreak \bigbreak \noindent 
    \begin{mdframed}
        2. Show that in example 6.1, the relations $A\text{-}C\text{-}B, A\text{-}D\text{-}B, C\text{-}A\text{-}D$, and $C\text{-}B\text{-}D$ hold
    \end{mdframed}
    \bigbreak \noindent 
    We have distances
    \begin{align*}
        \begin{array}{c|ccccc}
           &A&B&C&D&E \\
            A&0 & 3 & 1 & 2 & 4 \\
            B& 3 & 0 & 2 & 1 & 4 \\
            C& 1 & 2 & 0 & 3 & 4 \\
            D& 2 & 1 &3 & 0 & 4\\
            E& 4 & 4 & 4 &4   & 0 \\
        \end{array}
    \end{align*}
    \bigbreak \noindent 
    We have
    \begin{align*}
        AC + CB &= 1 + 2 = 3 = AB \implies A\text{-}C\text{-}B \\
        AD + DB &= 2 + 1 = 3 = AB \implies A\text{-}D\text{-}B \\
        CA + AD &=1 + 2 =3 = CD \implies C\text{-}A\text{-}D \\
        CB + BD &= 2 + 1 = 3 = CD \implies C\text{-}B\text{-}D 
    \end{align*}
    \endpf

    \bigbreak \noindent 
    \begin{mdframed}
        3. Assume the first seven axioms. Suppose that $A,B,X,Y$ are distinct, collinear points such that the distance between any two of them is less than $\omega$ and such that $Y \in \overline{AB} $, $X \in \overrightarrow{AB}, X\not\in \overline{AB}$, and $B \in \overline{XY} $. Prove that $Y \in \overline{AX} $
    \end{mdframed}
    \bigbreak \noindent 
    \textbf{\textit{Proof.}} Assume $A,B,X,Y$ are distinct, collinear points such that the distance between two of them is less than $\omega$. Further, assume that $Y \in \overline{AB}$, $X \in \overrightarrow{AB}$, $X \not\in \overline{AB}$, and $B \in \overline{XY}$. We aim to show that $Y \in \overline{AX}$. More specifically, that $A\text{-}Y\text{-}X$, or $AY + YX + AX$
    \bigbreak \noindent 
    Since the distance between any two of the given points is less than $\omega$, all rays and segments involving any pair of points are well defined. Using the given information, we have
    \begin{align*}
        Y \in \overline{AB} &\implies A\text{-}Y\text{-}B \implies AY + YB = AB \tag{1} \\
        X \in \overrightarrow{AB} &\implies A\text{-}X\text{-}B \text{ or } A\text{-}B\text{-}X \\
        X \not\in \overline{AB} &\implies \neg\left(A\text{-}X\text{-}B\right) \implies A\text{-}B\text{-}X \implies AB + BX = AX \tag{2} \\
        B \in \overline{XY} &\implies X\text{-}B\text{-}Y \implies XB + BY = XY \tag{3}
    \end{align*}
    Observe that since $AY + YB = AB$, and $AB + BX = AX$, we have $AY + YB + BX = AX$. Next, notice that $XB + BY = XY \implies BX + YB = YX$ by distance axiom 3. Since these distances are just real numbers, we can rearrange the expression as $YB = YX - BX$. We can then plug this expression into $AY + YB + BX = AX $ to get
    \begin{align*}
        AY + YB + BX &= AX \\
        \implies AY + YX - BX + BX &= AX \\
        \implies AY + YX &= AX
    \end{align*}
    Which, by the definition of betweenness, $A\text{-}Y\text{-}X$. Which, by the definition of the segment $\overline{AX} = \{P: A\text{-}P\text{-}X\}$, means that $y \in \overline{AX}$ \endpf


    \bigbreak \noindent 
    \begin{mdframed}
        4. Construct an example of a plane $\mathbb{P}$ that satisfies the first seven axioms, with a ray $\overrightarrow{AB}$ and points $X\ne  Y $ in $\overrightarrow{AB} $ such that $AX = AY $
    \end{mdframed}
    \bigbreak \noindent 
    Let $\mathbb{P} = \{A,B,X,Y\}$, $\mathbb{L} = \{A,B,X,Y\}, \{X,Y\}$, with distances
    \begin{align*}
        \begin{array}{c|cccc}
           &A&B&X&Y \\
            A & 0 & 2 & 1 & 1  \\ 
            B &  2  & 0 & 1 & 1 \\
            X &  1 & 1& 0 & 3 \\
            Y & 1 & 1& 3 & 0
        \end{array}
    \end{align*}
    Which satisfies distance axioms
    \begin{enumerate}
        \item $PQ \geq 0 $
        \item $PQ = 0 \iff P = Q $
        \item $PQ = QP $
    \end{enumerate}
    And incidence axioms
    \begin{enumerate}[label=(\alph*)]
        \item At least two lines
        \item Each line contains at least two different points
        \item Each pair of points are together in at least one line
        \item Each pair of points $P,Q$ with $PQ < \omega$ are together in at most one line
    \end{enumerate}






\end{document}
