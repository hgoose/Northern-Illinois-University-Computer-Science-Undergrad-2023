\documentclass{report}

\input{~/dev/latex/template/preamble.tex}
\input{~/dev/latex/template/macros.tex}

\title{\Huge{}}
\author{\huge{Nathan Warner}}
\date{\huge{}}
\fancyhf{}
\rhead{}
\fancyhead[R]{\itshape Warner} % Left header: Section name
\fancyhead[L]{\itshape\leftmark}  % Right header: Page number
\cfoot{\thepage}
\renewcommand{\headrulewidth}{0pt} % Optional: Removes the header line
%\pagestyle{fancy}
%\fancyhf{}
%\lhead{Warner \thepage}
%\rhead{}
% \lhead{\leftmark}
%\cfoot{\thepage}
%\setborder
% \usepackage[default]{sourcecodepro}
% \usepackage[T1]{fontenc}

% Change the title
\hypersetup{
    pdftitle={G2}
}

\geometry{
  left=1in,
  right=1in,
  top=.5in,
  bottom=1in
}


\begin{document}
    % \maketitle
    %     \begin{titlepage}
    %    \begin{center}
    %        \vspace*{1cm}
    % 
    %        \textbf{G2}
    % 
    %        \vspace{0.5cm}
    %         
    %             
    %        \vspace{1.5cm}
    % 
    %        \textbf{Nathan Warner}
    % 
    %        \vfill
    %             
    %             
    %        \vspace{0.8cm}
    %      
    %        \includegraphics[width=0.4\textwidth]{~/niu/seal.png}
    %             
    %        Computer Science \\
    %        Northern Illinois University\\
    %        United States\\
    %        
    %             
    %    \end{center}
    % \end{titlepage}
    % \tableofcontents
    \pagebreak \bigbreak \noindent 
    Nate Warner, Jacyn Beelow, Rachel Keske\ \quad \quad \quad \quad  MATH 353 \quad  \quad \quad \quad \quad \quad \ \ \quad \quad \quad \quad \quad \quad \quad Spring 2025
    \begin{center}
        \textbf{Group pset 2 - Due: Wednesday, March 19}
    \end{center}
    \bigbreak \noindent 
    \begin{mdframed}
        1. Let $\overrightarrow{AB}$ be a ray with carrier $m$, and $C$ a point in $\overrightarrow{AB}^{0} $. Prove that if $\omega < \infty$, then $C_{m}^{*} \not\in \overrightarrow{AB} $
    \end{mdframed}
    \bigbreak \noindent 
    \textbf{\textit{Proof.}} Assume ray $\overrightarrow{AB}$  with carrier $m$. Let $C$ be a point in $\text{Int}\overrightarrow{AB}$, and $\omega < \infty$.
    \bigbreak \noindent 
    Since $C \in \text{Int}\overrightarrow{AB}$, $C \ne A$ by the definition of the interior of a ray. Further, by the definition of $\overrightarrow{AB}$, one of $ A\text{-}C\text{-}B$ or $ A\text{-}B\text{-}C$ are true.
    \bigbreak \noindent 
    Suppose for the sake of contradiction that $C^{*}_{m} \in \overrightarrow{AB}$. Then, one of $ A\text{-}C^{*}_{m}\text{-}B$, $ A\text{-}B\text{-}C_{m}^{*}$. We consider four cases.
    \begin{enumerate}
        \item $ A\text{-}C\text{-}B $ and $ A\text{-}C_{m}^{*}\text{-} B$
        \item $ A\text{-}C\text{-}B$ and $ A\text{-}B\text{-}C_{m}^{*} $
        \item $ A\text{-}B\text{-}C $ and $ A\text{-}C^{*}_{m}\text{-}B $
        \item $ A\text{-}B\text{-}C$ and $ A\text{-}B\text{-}C_{m}^{*} $
    \end{enumerate}
    We first remark that since $\overrightarrow{AB}  $ defined, $AB < \omega$. Also, $C \in \text{Int}\overrightarrow{AB}  $ implies $C \ne A $
    \bigbreak \noindent 
    Assume (1) is true. Thus, we have $ A\text{-}C\text{-}B$ and $  A\text{-}C_{m}^{*}\text{-}B$. Since $ AB < \omega$ and $ AC_{m}^{*} + C_{m}^{*}B = AB$, $AC_{m}^{*} < AB < \omega$, and by theorem 8.4, $\overrightarrow{AB} = \overrightarrow{AC_{m}^{*}} $. Next, observe that since $C \in \overrightarrow{AB} = \overrightarrow{AC_{m}^{*}} $, one of
    \begin{align*}
        A\text{-}C\text{-}C_{m}^{*} \quad A\text{-}C_{m}^{*}\text{-}C
    \end{align*}
    Assume $ A\text{-}C\text{-}C_{m}^{*}$. In this case, $AC + CC_{m}^{*} = AC_{m}^{*}$, which implies $CC_{m}^{*} < AC_{m}^{*}$. But, with $A \ne C$ and Theorem 9.1, $CC_{m}^{*} = \omega$, and $AC_{m}^{*} < \omega$. Thus, $CC_{m}^{*} < AC_{m}^{*} \implies \omega < \omega$, a contradiction.
    \bigbreak \noindent 
    Next, assume $ A\text{-}C_{m}^{*}\text{-}C$, which implies $ AC_{m}^{*} + C_{m}^{*}C = AC$, and $C_{m}^{*}C < AC$. But, since $ A\text{-}C\text{-}B$, and $AB < \omega$, we have $AC < AB$. Thus, $C_{m}^{*}C < AC < AB < \omega$ is a contradiction, since $CC_{m}^{*} = \omega $. Thus, not ($ A\text{-}C\text{-}B$ and $ A\text{-}C_{m}^{*}\text{-}B $)
    \bigbreak \noindent 
    Assume (2) is true, then $ A\text{-}C\text{-}B$ and $ A\text{-}B\text{-}C_{m}^{*}$. In this case, ROI yields $ A\text{-}C\text{-}B\text{-}C_{m}^{*}$, which yields $ A\text{-}C\text{-}C_{m}^{*}$. This new relation gives $AC + CC_{m}^{*} = AC_{m}^{*}$, which again implies $CC_{m}^{*} < AC_{m}^{*} < \omega$, a contradiction by theorem 9.1. Thus, not ($ A\text{-}C\text{-}B$ and $A\text{-}B\text{-}C_{m}^{*}$)
    \bigbreak \noindent 
    Assume (3) is true, in a similar fashion to the previous case, from $ A\text{-}B\text{-}C$, $ A\text{-}C_{m}^{*}\text{-}B$ and the ROI, we get $ A\text{-}C_{m}^{*}\text{-}B\text{-}C $, which gives $ A\text{-}C_{m}^{*}\text{-}C$. From this, $AC_{m}^{*} +CC_{m}^{*} = AC$. Which means we have $ CC_{m}^{*} < AC < \omega$, which is a contradiction by theorem 9.1 ($CC_{m}^{*} = \omega$). Thus, not ($ A\text{-}B\text{-}C $ and $ A\text{-}C_{m}^{*}\text{-}B $)
    \bigbreak \noindent 
    Lastly, assume (4). Thus, $ A\text{-}B\text{-}C$ and $ A\text{-}B\text{-}C_{m}^{*}$. In this case, $ A\text{-}B\text{-}C_{m}^{*} $ gives $ AB + BC = AC_{m}^{*} $, but $ A\text{-}B\text{-}C$ tells us that $A \ne C$, and by theorem 9.1, $AC_{m}^{*} < \omega$. Thus, by theorem 8.4, $\overrightarrow{AB} = \overrightarrow{AC_{m}^{*}}$. This means one of 
    \begin{align*}
        A\text{-}C\text{-}C_{m}^{*} \quad A\text{-}C_{m}^{*}\text{-}C
    \end{align*}
    Which we saw in case (1) both give contradictions. So, not ($ A\text{-}B\text{-}C$ and $ A\text{-}B\text{-}C_{m}^{*}$ )
    \bigbreak \noindent 
    Therefore, $C \not\in \overrightarrow{AB} $ \endpf

    \pagebreak \bigbreak \noindent 
    \begin{mdframed}
        6. Prove Theorem 9.10
    \end{mdframed}
    \bigbreak \noindent 
    \textbf{\textit{Proof.}} Let $A,B$ be points on line $m$ with $0 < AB < \omega < \infty$. Let $C \ne A,B,A^{*}_{m},B^{*}_{m} $ be another point on $m$.
    \bigbreak \noindent 
    First, assume $C \in \overrightarrow{AB} \cup \overrightarrow{BA}$, which equals $\overline{AB} \cup \overline{BA^{*}_{m}}  \cup \overline{AB_{m}^{*}}$ By proposition 9.3. If $C \in \overrightarrow{AB} \cup \overrightarrow{BA}$, then one of 
    \begin{align*}
        A\text{-}C\text{-}B \quad A\text{-}B\text{-}C \quad B\text{-}C\text{-}A \quad B\text{-}A\text{-}C
    \end{align*}
    By definition of a ray. Observe that in any case, there is a betweenness relation among $A,B,C$. 
    \bigbreak \noindent 
    By corollary 9.9, the only segment left to examine is $ \overline{A_{m}^{*}B_{m}^{*}}$. Thus, assume $C \in \text{Int}\overline{A_{m}^{*}B_{m}^{*}} $ (since $C \ne A^{*}_{m}$ or $B_{m}^{*}$), which implies $ A^{*}_{m}\text{-}C\text{-} B_{m}^{*}$
    \bigbreak \noindent 
    Assume for the sake of contradiction that there does exist a betweenness relation among $A,B,C$. Then, one of 
    \begin{align*}
        A\text{-}B\text{-}C \quad A\text{-}C\text{-}B \quad B\text{-}A\text{-}C
    \end{align*}
    \bigbreak \noindent 
    Assume $ A\text{-}B\text{-}C $, then $C \in \overrightarrow{AB}$ by the definition of a ray. But, by proposition 9.3, $\overline{A_{m}^{*}B^{*}_{m}}^{0} $ is not included in $\overrightarrow{AB}$. Thus, a contradiction. Similarly, $ B\text{-}A\text{-}C $ implies $ C \in \overrightarrow{BA}$, another contradiction.
    \bigbreak \noindent 
    lastly, assume $ A\text{-}C\text{-}B $, then $C \in \overline{AB}^{0}$. But, by prop 9.9, $ \overline{AB}^{0}\cap \overline{A^{*}_{m}B^{*}_{m}}^{0} = \varnothing$. Thus, a contradiction.
    \bigbreak \noindent 
    Therefore, there is no betweenness relation among $A,B,C$ if and only if $C \in \overline{A^{*}_{m}B^{*}_{m}}^{0}$ \endpf

    
    
\end{document}
