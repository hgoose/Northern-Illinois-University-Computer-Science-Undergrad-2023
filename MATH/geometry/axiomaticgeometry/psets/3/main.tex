 \documentclass{report}
 
 \input{~/dev/latex/template/preamble.tex}
 \input{~/dev/latex/template/macros.tex}
 
 \title{\Huge{}}
 \author{\huge{Nathan Warner}}
 \date{\huge{}}
 \fancyhf{}
 \rhead{}
 \fancyhead[R]{\itshape Warner} % Left header: Section name
 \fancyhead[L]{\itshape\leftmark}  % Right header: Page number
 \cfoot{\thepage}
 \renewcommand{\headrulewidth}{0pt} % Optional: Removes the header line
 %\pagestyle{fancy}
 %\fancyhf{}
 %\lhead{Warner \thepage}
 %\rhead{}
 % \lhead{\leftmark}
 %\cfoot{\thepage}
 %\setborder
 % \usepackage[default]{sourcecodepro}
 % \usepackage[T1]{fontenc}
 
 % Change the title
 \hypersetup{
     pdftitle={}
 }

 \geometry{
  left=1in,
  right=1in,
  top=1in,
  bottom=1in
}
 
 \begin{document}
     % \maketitle
     %     \begin{titlepage}
     %    \begin{center}
     %        \vspace*{1cm}
     % 
     %        \textbf{}
     % 
     %        \vspace{0.5cm}
     %         
     %             
     %        \vspace{1.5cm}
     % 
     %        \textbf{Nathan Warner}
     % 
     %        \vfill
     %             
     %             
     %        \vspace{0.8cm}
     %      
     %        \includegraphics[width=0.4\textwidth]{~/niu/seal.png}
     %             
     %        Computer Science \\
     %        Northern Illinois University\\
     %        United States\\
     %        
     %             
     %    \end{center}
     % \end{titlepage}
     % \tableofcontents
    \pagebreak \bigbreak \noindent
    Nate Warner \ \quad \quad \quad \quad \quad \quad \quad \quad \quad \quad \quad \quad  MATH 353 \quad  \quad \quad \quad \quad \quad \quad \quad \quad \ \ \quad \quad Spring 2025
    \begin{center}
        \textbf{Problem set 3 - Due: Monday, Feb 3}
    \end{center}
    \bigbreak \noindent 
    \begin{mdframed}
        1. The figure below, where $a=3,b=4$ presents an implicit proof of the Pythagorean Theorem. Make this proof explicit through
        \begin{enumerate}[label=(\alph*)]
            \item Note that each of the four $a\times b$ rectangles in the picture is split into two congruent right triangles by a diagonal whose length is denoted $d$
            \item Show that the quadrilateral $PQRS$ is a square (that is, each vertex angle is a right angle). You may assume that the three angles of a triangle add to $180^{\circ}$
            \item Use the decomposition of $PQRS$ into four triangles and a square to find the area of $PQRS$ in terms of $a$ and $b$
            \item Conclude that $d^{2}= a^{2} + b^{2}$
        \end{enumerate}
    \end{mdframed}
    \begin{figure}[ht]
        \centering
        \incfig{fig3}
        \label{fig:fig3}
    \end{figure}
    \fc{}
    \bigbreak \noindent 
    a.) We begin by noting that each of the four $a\times b$ rectangles is split into two congruent right triangles by a diagonal whose length is denoted $d$
    \pagebreak \bigbreak \noindent 
    b.) Let's first denote each angle $\alpha, \beta$
    \bigbreak \noindent 
    \begin{figure}[ht]
        \centering
        \incfig{alpha}
        \label{fig:alpha}
    \end{figure}
    \fc{}
    \bigbreak \noindent 
    Next, we examine the rectangle that is split into two congruent triangles
    \bigbreak \noindent 
    \begin{figure}[ht]
        \centering
        \incfig{triangle}
        \label{fig:triangle}
    \end{figure}
    \fc{}
    \bigbreak \noindent 
    We see that since the measure opposite to $d$ is a right angle, we must have $\alpha + \beta = 90^{\circ}$, because the sum of a triangles angles must add up to $180^{\circ}$.
    \bigbreak \noindent 
    Observe from figure two that $\angle P  = \angle Q = \angle R = \angle S = \alpha + \beta = 90^{\circ}$. Thus, $\angle P + \angle Q + \angle R + \angle S = 90 + 90 + 90 + 90 = 360$. Thus, we conclude that the quadrilateral $PQRS$ is a square.

    \bigbreak \noindent 
    c.) First, observe that since $PQRS$ forms a square, and its side lengths are $d$, its area is $d^{2}$. Next, we find its area in terms of its four triangles plus its inner square. Observe that each triangle has area $\frac{1}{2}(bh) = \frac{1}{2}(ab)$. Since there are four triangles, the triangles contribute $\frac{1}{2}(ab)(4) = 2ab$ to the area of $PQRS$. Further, observe that each side of the inner square has length $b-a$. Thus, the inner square has area $(b-a)^{2}$, and the total area is given by
    \begin{align*}
        \text{area } &= 2ab + (b-a)^{2} \\
        &= 2ab +b^{2}-2ab + a^{2} = a^{2} + b^{2}
    \end{align*}
    \bigbreak \noindent 
    d.) Notice we have two expressions of the area of $PQRS$, $d^{2}$ and $a^{2} + b^{2}$. Thus, we conclude $d^{2} = a^{2} + b^{2}$  \hspace*{\fill} $\blacksquare $

    \pagebreak \bigbreak \noindent 
    \begin{mdframed}
        2. Euclid proved the Exterior Angle Inequality, which says that an exterior angle of a triangle is larger than either remote interior angle, without using the Fifth Postulate. Use the Exterior Angle Inequality to show that if line $\ell$ crosses line $m$ and $n$ so that the interior angles on one side add to two right angles (see figure \thefigtitle), then $m$ and $n$ are parallel. (Hint: Suppose that $m$ and $n$ meet and find a contradiction.) Do not assume that the three angles of a triangle add to 180.
    \end{mdframed}
    \bigbreak \noindent 
    \begin{figure}[ht]
        \centering
        \incfig{fig4}
        \label{fig:fig4}
    \end{figure}
    \fc{}
    \bigbreak \noindent 
    \textbf{Lemma 1 (Alternate interior angle equality)}. Consider the transversal configuration depicted below
    \bigbreak \noindent 
    \begin{figure}[ht]
        \centering
        \incfig{tc}
        \label{fig:tc}
    \end{figure}
    \bigbreak \noindent 
    Suppose $a + b = 180$, then $b = d$, and $c=a$.
    \bigbreak \noindent 
    \textbf{\textit{Proof.}} Consider the transversal configuration shown above. Assume $a+b = 180$, then $a=180-b$. Since vertical angles are equal, we have $d=h$. But since $a,h$ are supplementary, we have $a + h = 180$, which implies $h = 180 -a$. Thus,
    \begin{align*}
        d = h = 180 - a
    \end{align*}
    Since $a+b = 180$ implies $ b = 180-a$, we have
    \begin{align*}
        d = h = 180 - a = b
    \end{align*}
    \bigbreak \noindent 
    Thus, $d=b$. Next, we show that $c=a$. Since $c$ and $f$ are vertical, we have $c = f$. Further, since $a + b =180$, we have $a = 180 -b$. Notice that $b$ and $f$ are supplementary, which implies $b + f = 180$, or $f = 180 - b $. So, since $c=f = 180 -b$, and $a = 180-b$, we have $c = f = 180 - b = a$. Thus, $c=a$
    \bigbreak \noindent 
    Therefore, we conclude that if $a+ b =180$, $b = d$ and $c = a $ \hspace*{\fill}$\blacksquare$

    \bigbreak \noindent 
    2.) Assume for the sake of contradiction that lines $m$ and $n$ meet on the right side of figure $4$. Call the point where they meet $C$
    \bigbreak \noindent 
    \begin{figure}[ht]
        \centering
        \incfig{fig6}
        \label{fig:fig6}
    \end{figure}
    \fc{}
    \bigbreak \noindent 
    Since we have three noncollinear points, $\triangle ABC$ is formed. Next, extend $BA$ through $A$ to point $D$, exterior angle $\angle CAD$ is formed. Call this angle $\gamma$
    \bigbreak \noindent 
    \begin{figure}[ht]
        \centering
        \incfig{fig7}
        \label{fig:fig7}
    \end{figure}
    \bigbreak \noindent 
    Notice $\alpha$ and $\gamma$ are supplementary. Thus, $\alpha + \gamma = 180$, which implies $\gamma = 180 - \alpha $. By the Exterior Angle Inequality, we have $\gamma > \beta$. Since our hypothesis suggests $\alpha + \beta = 180$, we have $\beta = 180 -\alpha$. Thus, we have $\gamma = \beta$, but by the EAI, gamma must be strictly greater than beta. So, $\gamma > \beta$, and $\gamma = \beta$... Contradiction.
    \bigbreak \noindent 
    Therefore, we must throw out our assumption that lines $m,n$ meet on that side.
    \bigbreak \noindent 
    For the other (left) side of figure 4, denote the two interior angles $\gamma$, $\delta$.
    \pagebreak \bigbreak \noindent 
    \begin{figure}[ht]
        \centering
        \incfig{delta}
        \label{fig:delta}
    \end{figure}
    \bigbreak \noindent 
    But, by lemma 1, since $\alpha + \beta = 180$, we have $\beta = \delta$, and $\alpha = \gamma$. Thus, we have
    \bigbreak \noindent 
    \begin{figure}[ht]
        \centering
        \incfig{gamma}
        \label{fig:gamma}
    \end{figure}
    \bigbreak \noindent 
    Since the sum of the interior angles on both sides of the transversal are the same (180), the proof above also implies that the lines $m$ and $n$ will not meet on the left side. 
    \bigbreak \noindent 
    Therefore, since $m,n$ will not meet at either side, the two lines must be parallel. \hspace*{\fill} $\blacksquare $


    \pagebreak \bigbreak \noindent 
    \begin{mdframed}
        3. Show that Playfair's Postulate implies the Fifth Postulate (Hint: Use problem 2.)
    \end{mdframed}
    \bigbreak \noindent 
    \textbf{Lemma 2}. Suppose the transversal $\ell$ intersects lines $m$ and $n$ such that the consecutive interior angle sum on one of the sides is greater than $180$. Then, the sum of the consecutive interior angles on the opposite side is less than 180.
    \bigbreak \noindent 
    \textbf{\textit{Proof.}} Assume the transversal $\ell$ intersects lines $m$, and $n$ such that the sum of consecutive interior angles on one of the sides is greater than $180$. Call these two angles $\alpha,\beta$. Then, we have $\alpha + \beta > 180$. Let the angle supplementary to $\alpha$ be $d$, and the angle supplementary to $\beta$ be $c$. We have
    \begin{align*}
        \alpha + d &= 180 \implies d = 180 - \alpha\\
        \beta + c &= 180 \implies c = 180 - \beta
    \end{align*}
    Thus,
    \begin{align*}
        d + c &= 180 -a  +180 - b \\
        &= 360 - (\alpha + b)
    \end{align*}
    Since $\alpha + \beta > 180$, 
    \begin{align*}
        \alpha + \beta > 180 \\
        -(\alpha + \beta) < -180 \\
        -(\alpha + \beta) < 180 - 360 \\
        360 - (\alpha + \beta) < 180
    \end{align*}
    Thus, $d+c < 180$ \hspace*{\fill} $\blacksquare$

        \bigbreak \noindent 
    \begin{remark}
        (Playfair's Postulate). "In a plane, given a line and a point not on it, at most one line parallel to the given line can be drawn through the point"
        \bigbreak \noindent 
        (Euclids Fifth Postulate). "If a line segment intersects two straight lines forming two interior angles on the same side that are less than two right angles, then the two lines, if extended indefinitely, meet on that side on which the angles sum to less than two right angles."
    \end{remark}
    \qed
    \bigbreak \noindent 
    Assume Playfair's postulate. Thus, given a line and a point not on the line, there is a unique line through the given point parallel with the given line. Draw the line parallel to the given line through the given point, and the transversal that intersects both lines. By question two, we know that this unique parallel line must have consecutive interior angles that sum to 180 on both sides of the transversal. Since this parallel line is unique, all other lines through the given point must not be parallel to the given line. This suggests that the sum of the consecutive interior angles on either side must not be 180. Further, from this we know that either the left or right side of the transversal configuration must have a consecutive interior angle sum of less than 180 (by lemma 2). To assert that it is the side of the configuration with consecutive interior angle sum less than 180, we must show that the side greater than 180 will not meet on that side. From there, since we know that since they cannot be parallel, and they will not meet on the side greater than 180, they must meet at the side less than 180.
    \bigbreak \noindent 
    \textbf{Proposition.} Consider two lines $m,n$ and a transversal $\ell$ intersecting $m,n$. If the consecutive interior angles $\alpha,\beta$ sum greater than 180, they do not meet on that side.
    \bigbreak \noindent 
    \textbf{\textit{Proof.}} Assume for the sake of contradiction that they do in fact meet on that side
    \pagebreak \bigbreak \noindent 
    \begin{figure}[ht]
        \centering
        \incfig{tr2}
        \label{fig:tr2}
    \end{figure}
    \bigbreak \noindent 
    Call the point where they meet $C$, since we have three noncollinear points $A,B,C$, $\triangle ABC$ is formed.
    \bigbreak \noindent 
    Define $\angle CBD$ as the exterior angle for $\triangle ABC$, call it measure $\gamma$
    \bigbreak \noindent 
    \begin{figure}[ht]
        \centering
        \incfig{figp4}
        \label{fig:figp4}
    \end{figure}
    \bigbreak \noindent 
    \bigbreak \noindent 
    $\beta$ and $\gamma$ are supplementary, so $\beta + \gamma = 180^{\circ}$. Thus, $\gamma = 180^{\circ} - \beta$. By the EAI, $\gamma > \alpha$, which means $180^{\circ} - \beta > \alpha$. Thus, we have $180^{\circ} > \alpha + \beta$. But, we stated that $\alpha + \beta > 180^{\circ}$, which is a contradiction. 
    \bigbreak \noindent 
    Therefore, by contradiction, are assumption that $m,n$ meet on that side is false, and therefore $m,n$ must not meet on that side.  \hspace*{\fill} $\blacksquare$
    \bigbreak \noindent
    We have therefore established that since any line must not be parallel, and therefore must have a consecutive angle sum of less than 180 on one of the sides, and the opposite side (with sum greater than 180) must not be the side where they meet, they must meet on the side with sum less than 180. \hspace*{\fill} $\blacksquare$




    


    \pagebreak \bigbreak \noindent 
    \begin{mdframed}
        5. Show that if the Fifth Postulate holds, then the angle sum of any triangle equals two right angles ($180^{\circ}$) (Hint: Consider the line through a vertex of the triangle that is parallel to the opposite side, as in figure \thefigtitle, this pair of parallel lines is crossed by each of the other two sidelines of the triangle. What can you say about the interior angles in these configurations?)
    \end{mdframed}
    \bigbreak \noindent 
    \begin{figure}[ht]
        \centering
        \incfig{fig5}
        \label{fig:fig5}
    \end{figure}
    \fc{}
    \bigbreak \noindent 
    We have
    \bigbreak \noindent 
\begin{figure}[ht]
    \centering
    \incfig{lmane1}
    \label{fig:lmane1}
\end{figure}
\bigbreak \noindent 
    Observe that since $m$ and $n$ are parallel, $\ell_{1}$ and $\ell_{2}$ are transversal. This implies
    \begin{align*}
        \alpha + \beta + \delta &= 180          \tag{1}\\
        \gamma + \delta &= 180 \tag{2}
    \end{align*}
    Since $\varphi$ and $\gamma$ are supplementary, we also have $\varphi + \gamma =180$. Thus, we have $ \gamma = 180 -\varphi $, and $\gamma +\delta = 180 $, which implies $\gamma = 180 - \delta $ (by 2). Thus,
    \begin{align*}
        180 - \varphi &=      180 - \delta \\
        \implies -\varphi &= -\delta \\
        \implies \varphi &= \delta
    \end{align*}
    Thus, since $\alpha + \beta + \delta = 180$ (1), we have $\alpha + \beta + \varphi =180$, which is precisely the sum of the triangles angles. \hspace*{\fill} $\blacksquare$





\end{document}
