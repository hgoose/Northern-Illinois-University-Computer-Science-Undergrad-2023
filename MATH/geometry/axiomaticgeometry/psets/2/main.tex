 \documentclass{report}
 
 \input{~/dev/latex/template/preamble.tex}
 \input{~/dev/latex/template/macros.tex}
 
 \title{\Huge{}}
 \author{\huge{Nathan Warner}}
 \date{\huge{}}
 \fancyhf{}
 \rhead{}
 \fancyhead[R]{\itshape Warner} % Left header: Section name
 \fancyhead[L]{\itshape\leftmark}  % Right header: Page number
 \cfoot{\thepage}
 \renewcommand{\headrulewidth}{0pt} % Optional: Removes the header line
 %\pagestyle{fancy}
 %\fancyhf{}
 %\lhead{Warner \thepage}
 %\rhead{}
 % \lhead{\leftmark}
 %\cfoot{\thepage}
 %\setborder
 % \usepackage[default]{sourcecodepro}
 % \usepackage[T1]{fontenc}
 
 % Change the title
 \hypersetup{
     pdftitle={}
 }

 \geometry{
  left=1.5in,
  right=1.5in,
  top=1in,
  bottom=1in
}
 
 \begin{document}
     % \maketitle
     %     \begin{titlepage}
     %    \begin{center}
     %        \vspace*{1cm}
     % 
     %        \textbf{}
     % 
     %        \vspace{0.5cm}
     %         
     %             
     %        \vspace{1.5cm}
     % 
     %        \textbf{Nathan Warner}
     % 
     %        \vfill
     %             
     %             
     %        \vspace{0.8cm}
     %      
     %        \includegraphics[width=0.4\textwidth]{~/niu/seal.png}
     %             
     %        Computer Science \\
     %        Northern Illinois University\\
     %        United States\\
     %        
     %             
     %    \end{center}
     % \end{titlepage}
     % \tableofcontents
    \pagebreak \bigbreak \noindent
    Nate Warner \ \quad \quad \quad \quad \quad \quad \quad \quad \quad \quad \quad \quad  MATH 353 \quad  \quad \quad \quad \quad \quad \quad \quad \quad \ \ \quad \quad Spring 2025
    \begin{center}
        \textbf{Problem set 2 - Due: Wednesday, Jan 24}
    \end{center}
    \bigbreak \noindent 
    \begin{mdframed}
        1. Find all points of the form $P(x,0)$ in $\mathbb{H}$ whose distance from $O(0,0)$ is $\ln(2)$
    \end{mdframed}
    \bigbreak \noindent 
    If $O(0,0)$, and $P(x,0)$, then $M(-1,0)$, and $N(1,0)$. If distance between two points $A,B$ on the Hyperbolic plane is given by
    \begin{align*}
        d_{\mathbb{H}} = \ln{\left(\frac{e(AN)e(BM)}{e(AM)e(BN)}\right)}
    \end{align*}
    Where $e(PQ)$ denotes the Euclidean distance, we require
    \begin{align*}
       \frac{e(ON)e(PM)}{e(OM)e(PN)}  = 2
    \end{align*}
    Since the line on which these points lie is the line $y= 0$, then for all points $P(x_{1}, y_{1}), Q(x_{2}, y_{2})$ on this line, we have $e(PQ) = \abs{x_{1} -x_{2}}\sqrt{1+0^{2}} = \abs{x_{1} - x_{2}}$. We note that $e(ON) = \abs{0-1} =1$, and $e(OM) = \abs{0 - (-1)} = 1 $. Thus, we require
    \begin{align*}
        \frac{1(e(PM))}{1(e(PN))} = 2
    \end{align*}
    Observe that 
    \begin{align*}
        e(PM) &= \abs{x-(-1)} = \abs{x+1} = \sqrt{(x+1)^{2}}\\
        e(PN) &= \abs{x - 1} = \sqrt{(x-1)^{2}}
    \end{align*}
    Thus,
    \begin{align*}
        \frac{\sqrt{(x+1)^{2}}}{\sqrt{(x-1)^{2}}} &= 2 \\
        \implies \sqrt{(x+1)^{2}} &= 2\left(\sqrt{(x-1)^{2}}\right) \\
        \implies (x+1)^{2} &= 4(x-1)^{2} \\
        \implies x^{2} + 2x + 1 &= 4(x^{2} - 2x +1) \\
        \implies x^{2} + 2x + 1 &= 4x^{2} -8x + 4 \\
        \implies x^{2} - 4x^{2} +2x + 8x + 1 -4 &= 0 \\
        \implies -3x^{2} + 10x -3 &= 0
    \end{align*}
    By the quadratic formula, we have
    \begin{align*}
        x &= \frac{-10\pm \sqrt{100 - 4(-3)(-3)}}{2(-3)} \\
          &= \frac{-10 \pm 8}{-6}
    \end{align*}
    Thus, $x=\frac{1}{3}$, and $x=3$. Rather, $x\in \{\frac{1}{3},3\}$. We can verify that these two points $P\left(\frac{1}{3}, 0\right), P\left(3,0\right)$ infact give $d_{\mathbb{H}}(OP) = \ln(2)$. First, let $x=\frac{1}{3}$, we see
    \begin{align*}
        d_{\mathbb{H}}\left((0,0), \left(\frac{1}{3}, 0\right)\right) &= \ln\left(\frac{\left\lvert\right\frac{1}{3} + 1\rvert}{\left\lvert \frac{1}{3}-1 \right\rvert}\right) = \ln\left(\frac{\frac{4}{3}}{\frac{2}{3}}\right) = \ln\left(\frac{4(3)}{2(3)}\right) = \ln\left(2\right)
    \end{align*}
    Next, when $x=3$, we have
    \begin{align*}
        d_{\mathbb{H}}\left((0,0), (3,0)\right) &=\ln\left(\frac{\left\lvert 3 + 1 \right\rvert}{\left\lvert 3-1 \right\rvert}\right) = \ln\left(\frac{4}{2}\right) = \ln(2)
    \end{align*}
    \bigbreak \noindent 
    Therefore, the points $P(x,0)$ that give Hyperbolic distance $\ln(2)$ are $P\left(\frac{1}{3}, 0 \right) $ and $P(3,0)$

    \bigbreak \noindent 
    \begin{mdframed}
        2. Let $P(a,b,c)$ and $Q(x,y,z)$ be points on the sphere $\mathbb{S}$ or radius $r$ centered at $O(0,0,0)$. Let $d$ be the Euclidean distance $PQ$ and $\theta$ be the radian measure of $\angle POQ$
        \begin{enumerate}[label=(\alph*)]
            \item Recall the Law of Cosines for the triangle $POQ$  and use it to show that $\cos{\left(\theta \right)}  = \frac{2r^{2}-d^{2}}{2r^{2}}$
            \item Recall the Euclidean dastance formula for points in three dimensional space and use it and part (a) to show that $\cos{\left(\theta \right)}  = \frac{ax+by+cz}{r^{2}}$
            \item Use (b) to derive that $d_{\mathbb{S}}(PQ) = r\cos^{-1}{\left(\frac{ax+by+cz}{r^{2}}\right)}$
        \end{enumerate}
    \end{mdframed}
    \bigbreak \noindent 
    \begin{figure}[ht]
        \centering
        \incfig{fig1}
        \label{fig:fig1}
    \end{figure}
    \bigbreak \noindent 
    a.) For a triangle with sides $a$, $b$, and $c$, opposite respective angles $\alpha$, $\beta$, and $\gamma$ (see Fig.~1), the law of cosines states:
    \begin{align*}
        a^{2} = b^{2} + c^{2} - 2bc\cos{\left(\alpha\right)}
    \end{align*}
    Thus, for the triangle depicted above, we have 
    \begin{align*}
        d^{2} &= r^{2} + r^{2} -2(r)(r)\cos{\left(\theta \right)} \\
        \implies \theta &= \cos^{-1}{\left(\frac{d^{2}-2r^{2}}{-2r^{2}}\right)} \\
                        &=\cos^{-1}{\left(\frac{-(2r^{2}-d^{2})}{-2r^{2}}\right)} \\
                        &=\cos^{-1}{\left(\frac{2r^{2}-d^{2}}{2r^{2}}\right)}
    \end{align*}
    b.) If the Euclidean distance $d$ is given by $\sqrt{(x-a)^{2} + (y-b)^{2} + (z-c)^{2}}$. Then,
    \begin{align*}
        \theta &= \cos^{-1}{\left(\frac{2r^{2}-\left(\sqrt{(x-a)^{2} + (y-b)^{2} +(z-c)^{2}}\right)^{2}}{2r^{2}}\right)} \\
               &= \cos^{-1}{\left(\frac{2r^{2} - \left((x-a)^{2} + (y-b)^{2} + (z-c)^{2}\right)}{2r^{2}}\right)} \\
               &= \cos^{-1}{\left(\frac{2r^{2} - (x^{2} -2ax + a^{2} +y^{2} - 2by + b^{2} + z^{2} - 2cz + c^{2})}{2r^{2}}\right)} \\
               &= \cos^{-1}{\left(\frac{2r^{2} -x^{2} + 2ax - a^{2}  -y^{2} +2by -b^{2} -z^{2} + 2cz - c^{2}}{2r^{2}}\right)}
    \end{align*}
    Since $P(a,b,c)$ and $Q(x,y,z)$ both lie on the sphere or radius $r$, we have the conditions
    \begin{align*}
        a^{2} + b^{2} + c^{2} &= r^{2} \\
        x^{2} + y^{2} + z^{2} &=r^{2}
    \end{align*}
    Thus,
    \begin{align*}
       \theta  &= \cos^{-1}{\left(\frac{r^{2} + r^{2} -x^{2} + 2ax - a^{2}  -y^{2} +2by -b^{2} -z^{2} + 2cz - c^{2}}{2r^{2}}\right)} \\
               &= \cos^{-1}{\left(\frac{a^{2} + b^{2} +c^{2} + x^{2} + y^{2} + z^{2} -x^{2} + 2ax - a^{2}  -y^{2} +2by -b^{2} -z^{2} + 2cz - c^{2}}{2r^{2}}\right)} \\
               &= \cos^{-1}{\left(\frac{2ax + 2by + 2cz}{2r^{2}}\right)} \\
               &= \cos^{-1}{\left(\frac{ax+by+cz}{r^{2}}\right)}
    \end{align*}
    c.) Finally, let $d_{\mathbb{S}}(PQ)$ denote the arc length of $PQ $. If $d_{\mathbb{S}}(PQ) = r\theta$, and $\theta  = \cos^{-1}{\left(\frac{ax+by+cz}{r^{2}}\right)} $, then
    \begin{align*}
        d_{\mathbb{S}}(PQ) = r\cos^{-1}{\left(\frac{ax+by+cz}{r^{2}}\right)}
    \end{align*}
    As desired \endpf

    \bigbreak \noindent 
    \begin{mdframed}
        3.  Form the \textit{contrapositive} and \textit{converse} of each of the following
        \begin{enumerate}[label=(\alph*)]
            \item If a course is worthwhile, then it requires effort
            \item If Carl breaks the world record, then he wins a gold medal
        \end{enumerate}
    \end{mdframed}
    \bigbreak \noindent 
    \begin{remark}
        If a statement is of the form $P\implies Q$, then the converse is $Q\implies P$, and the contrapositive is $\neg Q \implies \neg P $
        \qed
    \end{remark}
    \bigbreak \noindent 
    a.) 
    \begin{itemize}
        \item \textbf{Converse}: If a course requires effort then it is worthwhile
        \item \textbf{Contrapositive}: If a course doesn't require effort, then it is not worthwhile
    \end{itemize}
    \bigbreak \noindent 
    b.)
    \begin{itemize}
        \item \textbf{Converse}: If Carl wins a gold medal, then he breaks the world record
        \item \textbf{Contrapositive}: If Carl does not win a gold medal, then he did not break the world record
    \end{itemize}

    \bigbreak \noindent 
    \begin{mdframed}
        4. Form the \textit{negation} of each of the following
        \begin{enumerate}[label=(\alph*)]
            \item She sells sea shells and sheep bells
            \item If Sam finished his homework, then he enjoyed the weekend
            \item The heat wave breaks or we go swimming
            \item Every student has a point beyond which he cannot be forced to work
            \item All candidates are in debt or some voters are undecided
            \item If a man answers, then the caller hangs up
            \item Every quadrilateral has the sum of its interior angles equal to $360^{\circ}$
            \item Time and tide wait for no person
        \end{enumerate}
    \end{mdframed}
    \bigbreak \noindent 
    \begin{remark}
        The negation of an implication $P \implies Q$ is $P \land \neg Q $. By De Morgan's laws, $\neg(P \land Q) \equiv \neg P \lor \neg Q$, and $\neg(P\lor Q)  \equiv \neg P \land \neg Q$
        \bigbreak \noindent 
        The negation of a universal (for all) statement is a existential (there exists) statement, the negation of an existential statement is a universal statement.
    \end{remark}
    \bigbreak \noindent 
    a.) 
    \begin{quote}
       $\neg$(She sells sea shells and sheep bells) $\equiv$ She doesn't sell sea shells or doesn't sell sheep bells
    \end{quote}
    \bigbreak \noindent 
    b.) 
    \begin{quote}
       $\neg$(If Sam finished his homework, then he enjoyed the weekend) $\equiv$ Sam finished his homework and he didn't enjoy the weekend
    \end{quote}
    \bigbreak \noindent 
    c.) 
    \begin{quote}
       $\neg$(The heat wave breaks or we go swimming) $\equiv$ The heat wave didn't break and we didn't go swimming 
    \end{quote}
    \bigbreak \noindent 
    d.) 
    \begin{quote}
       $\neg $(Every student has a point beyond which he cannot be forced to work) $\equiv$ There exists some student that does not have a point beyond which he cannot be forced to work. 
    \end{quote}
    \bigbreak \noindent 
    e.)
    \begin{quote}
      $\neg$(All candidates are in debt or some voters are undecided)  $\equiv$ There exists some candidate that is not in debt and all voters are undecided
    \end{quote}
    \bigbreak \noindent 
    f.)
    \begin{quote}
       $\neg$(If a man answers, then the caller hangs up) $\equiv$ A man answers and the caller doesn't hang up 
    \end{quote}
    \bigbreak \noindent 
    g.)
    \begin{quote}
       $\neg$(Every quadrilateral has the sum of its interior angles equal to $360^{\circ}$) $\equiv$ There exists a quadrilateral where the sum of its interior angles does not equal $360^{\circ}$
    \end{quote}
    \bigbreak \noindent 
    h.) 
    \begin{quote}
       $\neg$(Time and tide wait for no person) $\equiv$ Either time waits fora person or tide waits for a person or both. 
    \end{quote}
    
    
    
    
    
    
    
    

    









 \end{document} % (:
