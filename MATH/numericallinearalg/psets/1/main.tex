 \documentclass{report}
 
 \input{~/dev/latex/template/preamble.tex}
 \input{~/dev/latex/template/macros.tex}
 
 \title{\Huge{}}
 \author{\huge{Nathan Warner}}
 \date{\huge{}}
 \fancyhf{}
 \rhead{}
 \fancyhead[R]{\itshape Warner} % Left header: Section name
 \fancyhead[L]{\itshape\leftmark}  % Right header: Page number
 \cfoot{\thepage}
 \renewcommand{\headrulewidth}{0pt} % Optional: Removes the header line
 %\pagestyle{fancy}
 %\fancyhf{}
 %\lhead{Warner \thepage}
 %\rhead{}
 % \lhead{\leftmark}
 %\cfoot{\thepage}
 %\setborder
 % \usepackage[default]{sourcecodepro}
 % \usepackage[T1]{fontenc}
 
 % Change the title
 \hypersetup{
     pdftitle={}
 }

 \geometry{
  left=1in,
  right=1in,
  top=1in,
  bottom=1in
}
 
 \begin{document}
     % \maketitle
     %     \begin{titlepage}
     %    \begin{center}
     %        \vspace*{1cm}
     % 
     %        \textbf{}
     % 
     %        \vspace{0.5cm}
     %         
     %             
     %        \vspace{1.5cm}
     % 
     %        \textbf{Nathan Warner}
     % 
     %        \vfill
     %             
     %             
     %        \vspace{0.8cm}
     %      
     %        \includegraphics[width=0.4\textwidth]{~/niu/seal.png}
     %             
     %        Computer Science \\
     %        Northern Illinois University\\
     %        United States\\
     %        
     %             
     %    \end{center}
     % \end{titlepage}
     % \tableofcontents
    \pagebreak \bigbreak \noindent
    Nate Warner \ \quad \quad \quad \quad \quad \quad \quad \quad \quad \quad \quad \quad  MATH 434 \quad  \quad \quad \quad \quad \quad \quad \quad \quad \ \ \quad \quad Fall 2025
    \begin{center}
        \textbf{Problem set 1 - Due: Sunday, September 28}
    \end{center}
    \bigbreak \noindent 
    \begin{mdframed}
        1.2.4. Prove that if $A^{-1}$ exists, then there can be no nonzero $y$ for which $Ay = 0$
    \end{mdframed}

    \bigbreak \noindent 
    \begin{mdframed}
        1.2.5. Prove that if $A^{-1}$ exists, then $\det(A) \ne 0$.
    \end{mdframed}

    \bigbreak \noindent 
    \begin{mdframed}
        1.2.11. Check that the equations in Example 1.2.10 are correct. Check that the coefficient matrix of the system is nonsingular
    \end{mdframed}

    \pagebreak \bigbreak \noindent 
    \begin{mdframed}
        1.3.4. Use pencil and paper to solve the system
        \[
            \begin{bmatrix}
                2 & 0 & 0 & 0 \\
                -1 & 2 & 0 & 0 \\
                3 & 1 & -1 & 0 \\
                4 & 1 & -3 & 3
            \end{bmatrix}
            \begin{bmatrix}
                y_{1} \\
                y_{2} \\
                y_{3} \\
                y_{4}
            \end{bmatrix}
            =
            \begin{bmatrix}
                2 \\
                3 \\
                2 \\
                9
            \end{bmatrix}
        \]
        by forward substitution
    \end{mdframed}

    \pagebreak \bigbreak \noindent 
    \begin{mdframed}
        1.3.11. Use column-oriented forward substitution to solve the system from Exercise 1.3.4.
    \end{mdframed}

    \pagebreak \bigbreak \noindent 
    \begin{mdframed}
        1.3.15. Develop the row-oriented version of back substitution. Write pseudocode in the spirit of (1.3.5) and (1.3.13).
    \end{mdframed}

    \pagebreak \bigbreak \noindent 
    \begin{mdframed}
        1.3.16. Develop the column-oriented version of back substitution Write pseudocode in the spirit of (1.3.5) and (1.3.13).
    \end{mdframed}

    \pagebreak \bigbreak \noindent 
    \begin{mdframed}
        1.3.17. Solve the upper-triangular system
        \[
            \begin{bmatrix}
                3 & 2 & 1 & 0 \\
                0 & 1 & 2 & 3 \\
                0 & 0 & -2 & 1 \\
                0 & 0 & 0 & 4
            \end{bmatrix}
            \begin{bmatrix}
                x_{1} \\
                x_{2} \\
                x_{3} \\
                x_{4}
            \end{bmatrix}
            =
            \begin{bmatrix}
                -10 \\
                10 \\
                1 \\
                12
            \end{bmatrix}
        \]
        (a) by row-oriented back substitution, (b) by column-oriented back substitution
    \end{mdframed}

    \pagebreak \bigbreak \noindent 
    \begin{mdframed}
        1.4.21.  Let
        \[
            A =
            \begin{bmatrix}
                16 & 4 & 8 & 4 \\
                4 & 10 & 8 & 4 \\
                8 & 8 & 12 & 10 \\
                4 & 4 & 10 & 12
            \end{bmatrix},
            \qquad
            b =
            \begin{bmatrix}
                32 \\
                26 \\
                38 \\
                30
            \end{bmatrix}.
        \]
        Notice that $A$ is symmetric, (a) Use the inner-product formulation of Cholesky's method to show that $A$ is positive definite and compute its Cholesky factor, (b) Use forward and back substitution to solve the linear system $Ax =b$. 
    \end{mdframed}

    \pagebreak \bigbreak \noindent 
    \begin{mdframed}
        1.4.31. Use the outer-product form to work part (a) of Exercise 1.4.21. 
    \end{mdframed}

    \pagebreak \bigbreak \noindent 
    \begin{mdframed}
        1.4.33. Write a nonrecursive algorithm that implements the outer-product formulation of Cholesky's algorithm (1.4.28). Your algorithm should exploit the symmetry of $A$ by referencing only the main diagonal and upper part of $A$, and it should store $R$ over $A$. Be sure to put in the necessary check before taking the square root.
    \end{mdframed}

    \pagebreak \bigbreak \noindent 
    \begin{mdframed}
        1.4.40. Use the bordered form to work part (a) of Exercise 1.4.21. 
    \end{mdframed}

    \pagebreak \bigbreak \noindent 
    \begin{mdframed}
        1.4.54. Prove Proposition 1.4.53. 
        \bigbreak \noindent 
        As in the previous exercise, do not use the Cholesky decomposition in your proof; use the fact that $x^{\top} A x > 0$ for all nonzero $x$.
    \end{mdframed}
    \bigbreak \noindent 
    \textbf{Proposition 1.4.53}. Let $A $ be positive definite, and consider a partition
    \begin{align*}
        A = \begin{bmatrix} A_{11} & A_{12} \\ A_{21} & A_{22} \end{bmatrix}

    \end{align*}
    in which $A_{11} $ and $A_{22} $ are square. Then $A_{11}$ and $A_{22}$ are positive definite

    \pagebreak  \bigbreak \noindent 
    \begin{mdframed}
        1.4.56. Prove Proposition 1.4.55. 
    \end{mdframed}
    \bigbreak \noindent 
    \textbf{Proposition 1.4.55}: If $A$ and $X$ are $n\times n$, $A$ is positive definite, and $X$ is nonsingular,
    then the matrix $B = X^{\top} A X$ is also positive definite.
    \bigbreak \noindent 
    Considering the special case $A = I$ (which is clearly positive definite), we see that
    this proposition is a generalization of Theorem 1.4.4.

    \pagebreak \bigbreak \noindent 
    \begin{mdframed}
        1.4.58*.
        Let 
        \[
            A = 
            \begin{bmatrix}
                A_{11} & A_{12} \\
                A_{21} & A_{22}
            \end{bmatrix}
        \]
        be positive definite, and suppose $A_{11}$ is $j \times j$ and 
        $A_{22}$ is $k \times k$. By Proposition 1.4.53, $A_{11}$ is positive definite. 
        Let $R_{11}$ be the Cholesky factor of $A_{11}$, let $R_{12} = R_{11}^{-T} A_{12}$, 
        and let $\tilde{A}_{22} = A_{22} - R_{12}^{T} R_{12}$. The matrix $\tilde{A}_{22}$ is 
        called the \emph{Schur complement} of $A_{11}$ in $A$.
        \begin{enumerate}[(a)]
            \item Show that 
                \[
                    \tilde{A}_{22} = A_{22} - A_{21} A_{11}^{-1} A_{12}.
                \]
            \item Establish a decomposition of $A$ that is similar to (1.4.57) and involves $\tilde{A}_{22}$.
            \item Prove that $\tilde{A}_{22}$ is positive definite.
        \end{enumerate}
    \end{mdframed}

    \pagebreak \bigbreak \noindent 
    \begin{mdframed}
        1.4.62. Prove that if $A$ is positive definite, then $\det(A) > 0$
    \end{mdframed}


\end{document}
