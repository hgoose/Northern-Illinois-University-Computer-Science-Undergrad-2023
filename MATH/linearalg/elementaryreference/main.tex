\documentclass{report}

\input{~/dev/latex/template/preamble.tex}
\input{~/dev/latex/template/macros.tex}

\title{\Huge{}}
\author{\huge{Nathan Warner}}
\date{\huge{}}
\fancyhf{}
\rhead{}
\fancyhead[R]{\itshape Warner} % Left header: Section name
\fancyhead[L]{\itshape\leftmark}  % Right header: Page number
\cfoot{\thepage}
\renewcommand{\headrulewidth}{0pt} % Optional: Removes the header line
%\pagestyle{fancy}
%\fancyhf{}
%\lhead{Warner \thepage}
%\rhead{}
% \lhead{\leftmark}
%\cfoot{\thepage}
%\setborder
% \usepackage[default]{sourcecodepro}
% \usepackage[T1]{fontenc}

% Change the title
\hypersetup{
    pdftitle={Elementary Linear Algebra Reference}
}

\begin{document}
    % \maketitle
        \begin{titlepage}
       \begin{center}
           \vspace*{1cm}
    
           \textbf{Elementary Linear Algebra Reference}
    
           \vspace{0.5cm}
                
                
           \vspace{1.5cm}
    
           \textbf{Nathan Warner}
    
           \vfill
                
                
           \vspace{0.8cm}
         
           \includegraphics[width=0.4\textwidth]{~/niu/seal.png}
                
           Computer Science \\
           Northern Illinois University\\
           United States\\
           
                
       \end{center}
    \end{titlepage}
    \tableofcontents
    \pagebreak 
    \unsect{Solutions to linear systems}
    \begin{itemize}
        \item \textbf{Possible solutions to a linear system of two unknowns}: The linear system can have a \textbf{unique solution, no solution, or infinitely many solutions}.
            \item \textbf{Does the solution set form a line, plane, hyperplane, or something else?}: The formation of the solution set depends on the number of free variables,
                \begin{itemize}
                    \item \textbf{No free variables (one unique solution)}: Intersects at a point
                    \item \textbf{One free variable (Uncountable solutions)}: Solution set is a line (1-dimensional subspace)
                    \item \textbf{Two free variable (Uncountable solutions)}: Solution set forms a plane (2-dimensional subspace)
                    \item \textbf{Three free variable (Uncountable solutions)}: Solution set is a three dimensional subspace (In $\mathbb{R}^{3}$ it would be the whole space)
                    \item \textbf{$k$ free variables}: Solution set is a $k$-dimensional subspace in $\mathbb{R}^{n} $
                        \bigbreak \noindent 
                        \textbf{Note:} A \( k \)-dimensional subspace in \( \mathbb{R}^n \) means that the solution set spans a \( k \)-dimensional space within the \( n \)-dimensional ambient space \( \mathbb{R}^n \).

                \end{itemize}

        \item \textbf{Determine if three planes intersect at a unique point}: For this, we find all three normal vectors $\vec{\mathbf{n}}_{1}, \vec{\mathbf{n}}_{2}$, and $\vec{\mathbf{n}}_{3}$. Then we find the triple scalar product, that is
            \begin{align*}
                \vec{\mathbf{n}}_{1} \cdot (\vec{\mathbf{n}}_{2} \times \vec{\mathbf{n}}_{3})
            .\end{align*}
            If this value is non-zero, we have intersection at a unique point. If the value is zero, we either have no intersection, or intersection at a line.

    \end{itemize}

    \pagebreak 
    \unsect{Linearity}
    \begin{itemize}
        \item \textbf{The properties of linear equations}:
            A function \( f: \mathbb{R}^n \rightarrow \mathbb{R} \) representing a linear equation is linear, meaning it satisfies the following properties for all vectors \( \mathbf{x}, \mathbf{y} \in \mathbb{R}^n \) and all scalars \( c \in \mathbb{R} \):
            \begin{itemize}
                \item \textbf{Additivity:} \( f(\mathbf{x} + \mathbf{y}) = f(\mathbf{x}) + f(\mathbf{y}) \)
                \item \textbf{Homogeneity of Degree 1:} \( f(c\mathbf{x}) = cf(\mathbf{x}) \)
                    \bigbreak \noindent 
                    It follows from this that $f(c\mathbf{x})$, when $c=0$ implies $f(0 \mathbf{x}) = 0 f(\mathbf{x}) = 0$. Thus, we add the property
                \item \textbf{Scale by zero}: $f(0) = 0$
            \end{itemize}
            These properties define a linear function and imply that the graph of a linear equation is a straight line (in 2D) or a plane (in 3D).

    \end{itemize}

    \pagebreak 
    \unsect{Matrix algebra}

    \pagebreak 
    \unsect{Transpose}
    \begin{itemize}
        
    \end{itemize}





    
\end{document}
