\documentclass{report}

\input{~/dev/latex/template/preamble.tex}
\input{~/dev/latex/template/macros.tex}

\title{\Huge{}}
\author{\huge{Nathan Warner}}
\date{\huge{}}
\fancyhf{}
\rhead{}
\fancyhead[R]{\itshape Warner} % Left header: Section name
\fancyhead[L]{\itshape\leftmark}  % Right header: Page number
\cfoot{\thepage}
\renewcommand{\headrulewidth}{0pt} % Optional: Removes the header line
%\pagestyle{fancy}
%\fancyhf{}
%\lhead{Warner \thepage}
%\rhead{}
% \lhead{\leftmark}
%\cfoot{\thepage}
%\setborder
% \usepackage[default]{sourcecodepro}
% \usepackage[T1]{fontenc}

% Change the title
\hypersetup{
    pdftitle={Vectors in Space}
}

\begin{document}
    % \maketitle
        \begin{titlepage}
       \begin{center}
           \vspace*{1cm}
    
           \textbf{Chapter II} \\
           Vectors in Space
    
           \vspace{0.5cm}
            
                
           \vspace{1.5cm}
    
           \textbf{Nathan Warner}
    
           \vfill
                
                
           \vspace{0.8cm}
         
           \includegraphics[width=0.4\textwidth]{~/niu/seal.png}
                
           Computer Science \\
           Northern Illinois University\\
           February 30, 2023 \\
           United States\\
           
                
       \end{center}
    \end{titlepage}
    \tableofcontents
    \pagebreak \bigbreak \noindent 
    \beginch{2}{Vectors in Space}

    \bigbreak \noindent 
    \unsect{2.1 Vectors in the Plane}
    \bigbreak \noindent 
    \subsubsection{Vector Representation}
    \bigbreak \noindent 
    A vector in a plane is represented by a directed line segment (an arrow). The endpoints of the segment are called the initial point and the terminal point of the vector. An arrow from the initial point to the terminal point indicates the direction of the vector. The length of the line segment represents its magnitude. We use the notation $\lVert \vec{v} \rVert$ to denote the magnitude of the vector $\vec{v}$. A vector with an initial point and terminal point that are the same is called the zero vector, denoted $\vec{0}$. The zero vector is the only vector without a direction, and by convention can be considered to have any direction convenient to the problem at hand.
    Vectors with the same magnitude and direction are called equivalent vectors. We treat equivalent vectors as equal, even if they have different initial points. Thus, if $\vec{v}$ and $\vec{w}$ are equivalent, we write
    \[
    \vec{v} = \vec{w}.
    \]
    \bigbreak \noindent 
    \begin{dfn}
        Vectors are said to be \textbf{equivalent} vectors if they have the same magnitude and direction. 
    \end{dfn}

    \pagebreak \bigbreak \noindent 
    \subsection{Combining Vectors}
    \bigbreak \noindent 
    \begin{dfn}[Scalars]
        The product $k\vec{v}$ of a vector $\vec{v}$ and a scalar $k$ is a vector with a magnitude that is $|k|$ times the magnitude of $\vec{v}$, and with a direction that is the same as the direction of $\vec{v}$ if $k > 0$, and opposite the direction of $\vec{v}$ if $k < 0$. This is called scalar multiplication. If $k = 0$ or $\vec{v} = \vec{0}$, then $k\vec{v} = \vec{0}$.
    \end{dfn}
    \bigbreak \noindent 
    \begin{dfn}[Vector Addition]
        The sum of two vectors $\mathbf{v}$ and $\mathbf{w}$ can be constructed graphically by placing the initial point of $\mathbf{w}$ at the terminal point of $\mathbf{v}$. Then, the vector sum, $\mathbf{v} + \mathbf{w}$, is the vector with an initial point that coincides with the initial point of $\mathbf{v}$ and has a terminal point that coincides with the terminal point of $\mathbf{w}$. This operation is known as vector addition.
        \bigbreak \noindent 
        \fig{.8}{./figures/1.jpeg}
    \end{dfn}
    \bigbreak \noindent 
    It is also appropriate here to discuss vector subtraction. We define $\vec{v} - \vec{w}$ as $\vec{v} + (-\vec{w}) = \vec{v} + (-1)\vec{w}$. The vector $\vec{v} - \vec{w}$ is called the vector difference. Graphically, the vector $\vec{v} - \vec{w}$ is depicted by drawing a vector from the terminal point of $\vec{w}$ to the terminal point of $\vec{v}$.
    \bigbreak \noindent 
    \fig{.8}{./figures/2.jpeg}
    \bigbreak \noindent 
    The initial point of $\vec{v} + \vec{w}$ is the initial point of $\vec{v}$. The terminal point of $\vec{v} + \vec{w}$ is the terminal point of $\vec{w}$. These three vectors form the sides of a triangle. It follows that the length of any one side is less than the sum of the lengths of the remaining sides. So we have
    \[
        \lVert \vec{v} + \vec{w} \rVert \leq \lVert \vec{v} \rVert + \lVert \vec{w} \rVert.
    \]
    This is known more generally as the triangle inequality. There is one case, however, when the resultant vector $\vec{u} + \vec{v}$ has the same magnitude as the sum of the magnitudes of $\vec{u}$ and $\vec{v}$. This happens only when $\vec{u}$ and $\vec{v}$ have the same direction.
    \bigbreak \noindent 
    \pagebreak 
    \subsection{Vector Components}
    \bigbreak \noindent 
    We have seen how to plot a vector when we are given an initial point and a terminal point. However, because a vector can be placed anywhere in a plane, it may be easier to perform calculations with a vector when its initial point coincides with the origin. We call a vector with its initial point at the origin a standard-position vector. Because the initial point of any vector in standard position is known to be $(0,0)$, we can describe the vector by looking at the coordinates of its terminal point. Thus, if vector $\vec{v}$ has its initial point at the origin and its terminal point at $(x,y)$, we write the vector in component form as
    \[
        \vec{v} = \langle x, y \rangle.
    \]
    When a vector is written in component form like this, the scalars $x$ and $y$ are called the components of $\vec{v}$.
    \bigbreak \noindent
    \begin{dfn}
        The vector with initial point $(0,0)$ and terminal point $(x,y)$ can be written in component form as
        \[
            \vec{v} = \langle x, y \rangle.
        \]
        The scalars $x$ and $y$ are called the components of $\mathbf{v}$.
    \end{dfn}
    \bigbreak \noindent 
    Recall that vectors are named with lowercase letters in bold type or by drawing an arrow over their name. We have also learned that we can name a vector by its component form, with the coordinates of its terminal point in angle brackets. However, when writing the component form of a vector, it is important to distinguish between $\langle x, y \rangle$ and $(x, y)$. The first ordered pair uses angle brackets to describe a vector, whereas the second uses parentheses to describe a point in a plane. The initial point of $\langle x, y \rangle$ is $(0,0)$; the terminal point of $\langle x, y \rangle$ is $(x, y)$.
    \bigbreak \noindent 
    When we have a vector not already in standard position, we can determine its component form in one of two ways. We can use a geometric approach, in which we sketch the vector in the coordinate plane, and then sketch an equivalent standard-position vector. Alternatively, we can find it algebraically, using the coordinates of the initial point and the terminal point. To find it algebraically, we subtract the $x$-coordinate of the initial point from the $x$-coordinate of the terminal point to get the $x$ component, and we subtract the $y$-coordinate of the initial point from the $y$-coordinate of the terminal point to get the $y$ component.
    \begin{thrmm}[Component form of a vector not in standard position]
        Let $\vec{v}$ be a vector with initial point $(x_i, y_i)$ and terminal point $(x_t, y_t)$. Then we can express $\vec{v}$ in component form as $\vec{v} = \langle x_t - x_i, y_t - y_i \rangle."
    \end{thrmm}
    \bigbreak \noindent 
    To find the magnitude of a vector, we calculate the distance between its initial point and its terminal point. The magnitude of vector $\vec{v} = \langle x, y \rangle$ is denoted $\lVert \vec{v} \rVert$, or $|\vec{v}|$, and can be computed using the formula
    \[
        \lVert \vec{v} \rVert = \sqrt{x^2 + y^2}.
    \]
    Note that because this vector is written in component form, it is equivalent to a vector in standard position, with its initial point at the origin and terminal point $(x, y)$. Thus, it suffices to calculate the magnitude of the vector in standard position. Using the distance formula to calculate the distance between initial point $(0,0)$ and terminal point $(x, y)$, we have
    \[
        \lVert \vec{v} \rVert = \sqrt{(x - 0)^2 + (y - 0)^2} = \sqrt{x^2 + y^2}.
    \]
    Based on this formula, it is clear that for any vector $\vec{v}$, $\lVert \vec{v} \rVert \geq 0$, and $\lVert \vec{v} \rVert = 0$ if and only if $\vec{v} = \vec{0}$.
    \bigbreak \noindent 
    The magnitude of a vector can also be derived using the Pythagorean theorem.
    \pagebreak \bigbreak \noindent 
    \begin{dfn}
        Let $\vec{v} = \langle x_1, y_1 \rangle$ and $\vec{w} = \langle x_2, y_2 \rangle$ be vectors, and let $k$ be a scalar.
        \bigbreak \noindent 
        Scalar multiplication: $k\vec{v} = \langle kx_1, ky_1 \rangle$
        \bigbreak \noindent 
        Vector addition: $\vec{v} + \vec{w} = \langle x_1, y_1 \rangle + \langle x_2, y_2 \rangle = \langle x_1 + x_2, y_1 + y_2 \rangle$
    \end{dfn}
    \bigbreak \noindent 
    \subsection{Unit Vectors}
    \bigbreak \noindent 
    A unit vector is a vector with magnitude $1$. For any nonzero vector $\vec{v}$, we can use scalar multiplication to find a unit vector $\vec{u}$ that has the same direction as $\vec{v}$. To do this, we multiply the vector by the reciprocal of its magnitude:
    \[
    \vec{u} = \frac{1}{\lVert \vec{v} \rVert} \vec{v}.
    \]
    Recall that when we defined scalar multiplication, we noted that $\lVert k\vec{v} \rVert = |k| \cdot \lVert \vec{v} \rVert$. For $\vec{u} = \frac{1}{\lVert \vec{v} \rVert} \vec{v}$, it follows that $\lVert \vec{u} \rVert = \frac{1}{\lVert \vec{v} \rVert} (\lVert \vec{v} \rVert) = 1$. We say that $\vec{u}$ is the unit vector in the direction of $\vec{v}$ (Figure 2.17). The process of using scalar multiplication to find a unit vector with a given direction is called normalization.

    




    




    
    
    





    
    

\end{document}
