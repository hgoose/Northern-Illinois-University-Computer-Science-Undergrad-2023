\documentclass{report}

\input{~/dev/latex/template/preamble.tex}
\input{~/dev/latex/template/macros.tex}

\title{\Huge{}}
\author{\huge{Nathan Warner}}
\date{\huge{}}
\fancyhf{}
\rhead{}
\fancyhead[R]{\itshape Warner} % Left header: Section name
\fancyhead[L]{\itshape\leftmark}  % Right header: Page number
\cfoot{\thepage}
\renewcommand{\headrulewidth}{0pt} % Optional: Removes the header line
%\pagestyle{fancy}
%\fancyhf{}
%\lhead{Warner \thepage}
%\rhead{}
% \lhead{\leftmark}
%\cfoot{\thepage}
%\setborder
% \usepackage[default]{sourcecodepro}
% \usepackage[T1]{fontenc}

% Change the title
\hypersetup{
    pdftitle={Vectors in Space}
}

\begin{document}
    % \maketitle
        \begin{titlepage}
       \begin{center}
           \vspace*{1cm}
    
           \textbf{Chapter II} \\
           Vectors in Space
    
           \vspace{0.5cm}
            
                
           \vspace{1.5cm}
    
           \textbf{Nathan Warner}
    
           \vfill
                
           \vspace{0.8cm}
         
           \includegraphics[width=0.4\textwidth]{~/niu/seal.png}
                
           Computer Science \\
           Northern Illinois University\\
           February 30, 2023 \\
           United States\\
           
                
       \end{center}
    \end{titlepage}
    \tableofcontents
    \pagebreak \bigbreak \noindent 
    \beginch{2}{Vectors in Space}

    \bigbreak \noindent 
    \unsect{2.1 Vectors in the Plane}
    \bigbreak \noindent 
    \subsubsection{Vector Representation}
    \bigbreak \noindent 
    A vector in a plane is represented by a directed line segment (an arrow). The endpoints of the segment are called the initial point and the terminal point of the vector. An arrow from the initial point to the terminal point indicates the direction of the vector. The length of the line segment represents its magnitude. We use the notation $\lVert \vec{v} \rVert$ to denote the magnitude of the vector $\vec{v}$. A vector with an initial point and terminal point that are the same is called the zero vector, denoted $\vec{0}$. The zero vector is the only vector without a direction, and by convention can be considered to have any direction convenient to the problem at hand.
    Vectors with the same magnitude and direction are called equivalent vectors. We treat equivalent vectors as equal, even if they have different initial points. Thus, if $\vec{v}$ and $\vec{w}$ are equivalent, we write
    \[
    \vec{v} = \vec{w}.
    \]
    \bigbreak \noindent 
    \begin{dfn}
        Vectors are said to be \textbf{equivalent} vectors if they have the same magnitude and direction. 
    \end{dfn}

    \pagebreak \bigbreak \noindent 
    \subsection{Combining Vectors}
    \bigbreak \noindent 
    \begin{dfn}[Scalars]
        The product $k\vec{v}$ of a vector $\vec{v}$ and a scalar $k$ is a vector with a magnitude that is $|k|$ times the magnitude of $\vec{v}$, and with a direction that is the same as the direction of $\vec{v}$ if $k > 0$, and opposite the direction of $\vec{v}$ if $k < 0$. This is called scalar multiplication. If $k = 0$ or $\vec{v} = \vec{0}$, then $k\vec{v} = \vec{0}$.
    \end{dfn}
    \bigbreak \noindent 
    \begin{dfn}[Vector Addition]
        The sum of two vectors $\mathbf{v}$ and $\mathbf{w}$ can be constructed graphically by placing the initial point of $\mathbf{w}$ at the terminal point of $\mathbf{v}$. Then, the vector sum, $\mathbf{v} + \mathbf{w}$, is the vector with an initial point that coincides with the initial point of $\mathbf{v}$ and has a terminal point that coincides with the terminal point of $\mathbf{w}$. This operation is known as vector addition.
        \bigbreak \noindent 
        \fig{.8}{./figures/1.jpeg}
    \end{dfn}
    \bigbreak \noindent 
    It is also appropriate here to discuss vector subtraction. We define $\vec{v} - \vec{w}$ as $\vec{v} + (-\vec{w}) = \vec{v} + (-1)\vec{w}$. The vector $\vec{v} - \vec{w}$ is called the vector difference. Graphically, the vector $\vec{v} - \vec{w}$ is depicted by drawing a vector from the terminal point of $\vec{w}$ to the terminal point of $\vec{v}$.
    \bigbreak \noindent 
    \fig{.8}{./figures/2.jpeg}
    \bigbreak \noindent 
    The initial point of $\vec{v} + \vec{w}$ is the initial point of $\vec{v}$. The terminal point of $\vec{v} + \vec{w}$ is the terminal point of $\vec{w}$. These three vectors form the sides of a triangle. It follows that the length of any one side is less than the sum of the lengths of the remaining sides. So we have
    \[
        \lVert \vec{v} + \vec{w} \rVert \leq \lVert \vec{v} \rVert + \lVert \vec{w} \rVert.
    \]
    This is known more generally as the triangle inequality. There is one case, however, when the resultant vector $\vec{u} + \vec{v}$ has the same magnitude as the sum of the magnitudes of $\vec{u}$ and $\vec{v}$. This happens only when $\vec{u}$ and $\vec{v}$ have the same direction.
    \bigbreak \noindent 
    \pagebreak 
    \subsection{Vector Components}
    \bigbreak \noindent 
    We have seen how to plot a vector when we are given an initial point and a terminal point. However, because a vector can be placed anywhere in a plane, it may be easier to perform calculations with a vector when its initial point coincides with the origin. We call a vector with its initial point at the origin a standard-position vector. Because the initial point of any vector in standard position is known to be $(0,0)$, we can describe the vector by looking at the coordinates of its terminal point. Thus, if vector $\vec{v}$ has its initial point at the origin and its terminal point at $(x,y)$, we write the vector in component form as
    \[
        \vec{v} = \langle x, y \rangle.
    \]
    When a vector is written in component form like this, the scalars $x$ and $y$ are called the components of $\vec{v}$.
    \bigbreak \noindent
    \begin{dfn}
        The vector with initial point $(0,0)$ and terminal point $(x,y)$ can be written in component form as
        \[
            \vec{v} = \langle x, y \rangle.
        \]
        The scalars $x$ and $y$ are called the components of $\mathbf{v}$.
    \end{dfn}
    \bigbreak \noindent 
    Recall that vectors are named with lowercase letters in bold type or by drawing an arrow over their name. We have also learned that we can name a vector by its component form, with the coordinates of its terminal point in angle brackets. However, when writing the component form of a vector, it is important to distinguish between $\langle x, y \rangle$ and $(x, y)$. The first ordered pair uses angle brackets to describe a vector, whereas the second uses parentheses to describe a point in a plane. The initial point of $\langle x, y \rangle$ is $(0,0)$; the terminal point of $\langle x, y \rangle$ is $(x, y)$.
    \bigbreak \noindent 
    When we have a vector not already in standard position, we can determine its component form in one of two ways. We can use a geometric approach, in which we sketch the vector in the coordinate plane, and then sketch an equivalent standard-position vector. Alternatively, we can find it algebraically, using the coordinates of the initial point and the terminal point. To find it algebraically, we subtract the $x$-coordinate of the initial point from the $x$-coordinate of the terminal point to get the $x$ component, and we subtract the $y$-coordinate of the initial point from the $y$-coordinate of the terminal point to get the $y$ component.
    \begin{thrmm}[Component form of a vector not in standard position]
        Let $\vec{v}$ be a vector with initial point $(x_i, y_i)$ and terminal point $(x_t, y_t)$. Then we can express $\vec{v}$ in component form as $\vec{v} = \langle x_t - x_i, y_t - y_i \rangle.$
    \end{thrmm}
    \bigbreak \noindent 
    To find the magnitude of a vector, we calculate the distance between its initial point and its terminal point. The magnitude of vector $\vec{v} = \langle x, y \rangle$ is denoted $\lVert \vec{v} \rVert$, or $|\vec{v}|$, and can be computed using the formula
    \[
        \lVert \vec{v} \rVert = \sqrt{x^2 + y^2}.
    \]
    Note that because this vector is written in component form, it is equivalent to a vector in standard position, with its initial point at the origin and terminal point $(x, y)$. Thus, it suffices to calculate the magnitude of the vector in standard position. Using the distance formula to calculate the distance between initial point $(0,0)$ and terminal point $(x, y)$, we have
    \[
        \lVert \vec{v} \rVert = \sqrt{(x - 0)^2 + (y - 0)^2} = \sqrt{x^2 + y^2}.
    \]
    Based on this formula, it is clear that for any vector $\vec{v}$, $\lVert \vec{v} \rVert \geq 0$, and $\lVert \vec{v} \rVert = 0$ if and only if $\vec{v} = \vec{0}$.
    \bigbreak \noindent 
    The magnitude of a vector can also be derived using the Pythagorean theorem.
    \pagebreak \bigbreak \noindent 
    \begin{dfn}
        Let $\vec{v} = \langle x_1, y_1 \rangle$ and $\vec{w} = \langle x_2, y_2 \rangle$ be vectors, and let $k$ be a scalar.
        \bigbreak \noindent 
        Scalar multiplication: $k\vec{v} = \langle kx_1, ky_1 \rangle$
        \bigbreak \noindent 
        Vector addition: $\vec{v} + \vec{w} = \langle x_1, y_1 \rangle + \langle x_2, y_2 \rangle = \langle x_1 + x_2, y_1 + y_2 \rangle$
    \end{dfn}
    \bigbreak \noindent 
    \subsection{Unit Vectors}
    \bigbreak \noindent 
    A unit vector is a vector with magnitude $1$. For any nonzero vector $\vec{v}$, we can use scalar multiplication to find a unit vector $\vec{u}$ that has the same direction as $\vec{v}$. To do this, we multiply the vector by the reciprocal of its magnitude:
    \[
    \vec{u} = \frac{1}{\lVert \vec{v} \rVert} \vec{v}.
    \]
    Recall that when we defined scalar multiplication, we noted that $\lVert k\vec{v} \rVert = |k| \cdot \lVert \vec{v} \rVert$. For $\vec{u} = \frac{1}{\lVert \vec{v} \rVert} \vec{v}$, it follows that $\lVert \vec{u} \rVert = \frac{1}{\lVert \vec{v} \rVert} (\lVert \vec{v} \rVert) = 1$. We say that $\vec{u}$ is the unit vector in the direction of $\vec{v}$. The process of using scalar multiplication to find a unit vector with a given direction is called normalization.

    \pagebreak 
    \unsect{2.2 Vectors in Three Dimensions}
    \bigbreak \noindent 

    \subsection{Three-Dimensional Coordinate Systems}
    \bigbreak \noindent 
    As we have learned, the two-dimensional rectangular coordinate system contains two perpendicular axes: the horizontal x-axis and the vertical y-axis. We can add a third dimension, the z-axis, which is perpendicular to both the x-axis and the y-axis. We call this system the three-dimensional rectangular coordinate system. It represents the three dimensions we encounter in real life.
    \bigbreak \noindent 
    \begin{dfn}
        The three-dimensional rectangular coordinate system consists of three perpendicular axes: the x-axis, the y-axis, the z-axis, and an origin at the point of intersection (0) of the axes. Because each axis is a number line representing all real numbers in  $\mathbb{R} $, the three-dimensional system is often denoted by  $\mathbb{R^{3}} $.
    \end{dfn}
    \bigbreak \noindent 
    \fig{.8}{./figures/3.jpeg}
    \bigbreak \noindent 
    In two dimensions, we describe a point in the plane with the coordinates $(x,y)$. Each coordinate describes how the point aligns with the corresponding axis. In three dimensions, a new coordinate, $z$, is appended to indicate alignment with the $z$-axis: $(x,y,z)$. A point in space is identified by all three coordinates. To plot the point $(x,y,z)$, go $x$ units along the $x$-axis, then $y$ units in the direction of the $y$-axis, then $z$ units in the direction of the $z$-axis.
    \bigbreak \noindent 
    \fig{.8}{./figures/4.jpeg}
    \bigbreak \noindent 
    In two-dimensional space, the coordinate plane is defined by a pair of perpendicular axes. These axes allow us to name any location within the plane. In three dimensions, we define coordinate planes by the coordinate axes, just as in two dimensions. There are three axes now, so there are three intersecting pairs of axes. Each pair of axes forms a coordinate plane: the $xy$-plane, the $xz$-plane, and the $yz$-plane. We define the $xy$-plane formally as the following set: $\{(x,y,0) : x,y \in \mathbb{R}\}$.
    Similarly, the $xz$-plane and the $yz$-plane are defined as $\{(x,0,z) : x,z \in \mathbb{R}\}$
    and $\{(0,y,z) : y,z \in \mathbb{R}\}$,
    respectively.
    \bigbreak \noindent 
    To visualize this, imagine you’re building a house and are standing in a room with only two of the four walls finished. (Assume the two finished walls are adjacent to each other.) If you stand with your back to the corner where the two finished walls meet, facing out into the room, the floor is the $xy$-plane, the wall to your right is the $xz$-plane, and the wall to your left is the $yz$-plane.
    \bigbreak \noindent 
    \fig{.8}{./figures/5.jpeg}
    \bigbreak \noindent 
    \begin{thrmm}
       The distance $d$ between points $(x_1, y_1, z_1)$ and $(x_2, y_2, z_2)$ is given by the formula
       \[
           d = \sqrt{(x_2 - x_1)^2 + (y_2 - y_1)^2 + (z_2 - z_1)^2}.
       \]
    \end{thrmm}

    \bigbreak \noindent 
    \subsection{Writing Equations in $\mathbb{R}^{3}$}
    \bigbreak \noindent 
    Now that we can represent points in space and find the distance between them, we can learn how to write equations of geometric objects such as lines, planes, and curved surfaces in $\mathbb{R}^3$. First, we start with a simple equation. Compare the graphs of the equation $x=0$ in $\mathbb{R}$, $\mathbb{R}^2$, and $\mathbb{R}^3$. From these graphs, we can see the same equation can describe a point, a line, or a plane.
    \bigbreak \noindent 
    \fig{.8}{./figures/6.jpeg}
    \bigbreak \noindent 
    In space, the equation $x=0$ describes all points $(0,y,z)$. This equation defines the $yz$-plane. Similarly, the $xy$-plane contains all points of the form $(x,y,0)$. The equation $z=0$ defines the $xy$-plane and the equation $y=0$ describes the $xz$-plane. 
    \bigbreak \noindent 
    \fig{.8}{./figures/7.jpeg}
    \bigbreak \noindent 
    Understanding the equations of the coordinate planes allows us to write an equation for any plane that is parallel to one of the coordinate planes. When a plane is parallel to the xy-plane, for example, the z-coordinate of each point in the plane has the same constant value. Only the x- and y-coordinates of points in that plane vary from point to point.
    \bigbreak \noindent 
    \begin{thrmm}
        
        \begin{enumerate}
            \item The plane in space that is parallel to the xy-plane and contains point  (a,b,c) can be represented by the equation $z=c$.
            \item The plane in space that is parallel to the xz-plane and contains point  (a,b,c) can be represented by the equation  $y=b$.
            \item The plane in space that is parallel to the yz-plane and contains point  (a,b,c) can be represented by the equation  $x=a$
        \end{enumerate}


    \end{thrmm}

    \pagebreak \bigbreak \noindent 
    \begin{dfn}
        A sphere is the set of all points in space equidistant from a fixed point, the center of the sphere, 
        just as the set of all points in a plane that are equidistant from the center represents a circle. 
        In a sphere, as in a circle, the distance from the center to a point on the sphere is called the \textit{radius}.
    \end{dfn}
    \fig{.8}{./figures/8.jpeg}
    \bigbreak \noindent 
    \begin{thrmm}[Equation of a sphere]
        The sphere with center $(a,b,c)$ and radius $r$ can be represented by the equation
        \[
            (x - a)^2 + (y - b)^2 + (z - c)^2 = r^2.
        \]
        This equation is known as the \textbf{standard equation of a sphere}.
    \end{thrmm}

    \bigbreak \noindent 
    \begin{exm}[Graphing Other Equations in Three Dimensions]
        Describe the set of points that satisfies  $(x-4)(z-2) = 0$ and graph the set.
        
    \end{exm}
    \bigbreak \noindent 
    \textcolor{red}{\textit{Solution.}}
    \bigbreak \noindent 
    We must have either $x - 4 = 0$ or $z - 2 = 0$, so the set of points forms the two planes $x = 4$ and $z = 2$.
    \bigbreak \noindent 
    \fig{.8}{./figures/9.jpeg}

    \pagebreak 
    \unsect{2.3 The Dot Product}
    \bigbreak \noindent 
    \subsection{The Dot Product and Its Properties}
    \bigbreak \noindent 
    We have already learned how to add and subtract vectors. In this chapter, we investigate two types of vector multiplication. The first type of vector multiplication is called the \textbf{dot product}, based on the notation we use for it, and it is defined as follows:
    \bigbreak \noindent 
    \begin{dfn}[The dot Product]
        The dot product of vectors $\vec{u} = \langle u_1, u_2, u_3 \rangle$ and $\vec{v} = \langle v_1, v_2, v_3 \rangle$ is given by the sum of the products of the components
        \[
            \vec{u} \cdot \vec{v} = u_1v_1 + u_2v_2 + u_3v_3.
        \]
    \end{dfn}
    \bigbreak \noindent 
    \nt{When two vectors are combined under addition or subtraction, the result is a vector. When two vectors are combined using the dot product, the result is a scalar. For this reason, the dot product is often called the scalar product. It may also be called the inner product.}
    \bigbreak \noindent 
    \begin{dfn}[Properties of the dot product]
        Let $\vec{u}$, $\vec{v}$, and $\vec{w}$ be vectors, and let $c$ be a scalar.
        \begin{enumerate}
            \item Commutative property: $\vec{u} \cdot \vec{v} = \vec{v} \cdot \vec{u}$
            \item Distributive property: $\vec{u} \cdot (\vec{v} + \vec{w}) = \vec{u} \cdot \vec{v} + \vec{u} \cdot \vec{w}$
            \item Associative property of scalar multiplication: $(c\vec{u} \cdot \vec{v}) = (c\vec{u}) \cdot \vec{v} = \vec{u} \cdot (c\vec{v})$
            \item Property of magnitude: $\vec{v} \cdot \vec{v} = \|\vec{v}\|^2$
        \end{enumerate}
    \end{dfn}

    \pagebreak 
    \subsection{Using the Dot Product to Find the Angle between Two Vectors}
    \bigbreak \noindent 
    When two nonzero vectors are placed in standard position, whether in two dimensions or three dimensions, they form an angle between them. The dot product provides a way to find the measure of this angle. This property is a result of the fact that we can express the dot product in terms of the cosine of the angle formed by two vectors.
    \bigbreak \noindent 
    \fig{.8}{./figures/10.jpeg}
    \bigbreak \noindent 
    \begin{thrmm}[Evaluating a Dot Product]
        The dot product of two vectors is the product of the magnitude of each vector and the cosine of the angle between them:
        \begin{align*}
            \vec{u} \cdot \vec{v} = \norm{u} \cdot \norm{v} \cdot \cos{\theta }
        .\end{align*}
    \end{thrmm}
    \bigbreak \noindent 
    \pf{Proof}{
    Place vectors $\vec{u}$ and $\vec{v}$ in standard position and consider the vector $\vec{v} - \vec{u}$. These three vectors form a triangle with side lengths $\|\vec{u}\|$, $\|\vec{v}\|$, and $\|\vec{v} - \vec{u}\|$.
    \bigbreak \noindent 
    \fig{.8}{./figures/11.jpeg}
    \bigbreak \noindent 
    Recall from trigonometry that the law of cosines describes the relationship among the side lengths of the triangle and the angle $\theta$. Applying the law of cosines here gives
    \[
        \|\vec{v} - \vec{u}\|^2 = \|\vec{u}\|^2 + \|\vec{v}\|^2 - 2\|\vec{u}\|\|\vec{v}\|\cos\theta.
    \]
    The dot product provides a way to rewrite the left side of this equation:
    \begin{align*}
        &\|\vec{v} - \vec{u}\|^2 = (\vec{v} - \vec{u}) \cdot (\vec{v} - \vec{u}) \\
        &= (\vec{v} - \vec{u}) \cdot \vec{v} - (\vec{v} - \vec{u}) \cdot \vec{u} \\
        &= \vec{v} \cdot \vec{v} - \vec{u} \cdot \vec{v} - \vec{v} \cdot \vec{u} + \vec{u} \cdot \vec{u} \\
        &= \|\vec{v}\|^2 - 2\vec{u} \cdot \vec{v} + \|\vec{u}\|^2.
    .\end{align*}
    Substituting into the law of cosines yields
    \begin{align*}
        &\norm{\vec{v} - \vec{u}}^{2} = \norm{\vec{u}}^{2} + \norm{\vec{v}}^{2} -2\norm{\vec{u}}\norm{\vec{v}}\cos{\theta } \\
        &\norm{\vec{v}}^{2}-2\vec{u} \cdot \vec{v} + \norm{\vec{u}}^{2} = \norm{\vec{u}}^{2} + \norm{\vec{v}}^{2} -2 \norm{\vec{u}}\norm{\vec{v}}\cos{\theta } \\
        &-2\vec{u} \cdot \vec{v}  = -2 \norm{\vec{u}}\norm{\vec{v}}\cos{\theta } \\
        &\vec{u} \cdot \vec{v} = \norm{\vec{u}}\norm{\vec{v}} \cos{\theta }
    .\end{align*}
}

    \pagebreak \bigbreak \noindent 
    We can use this form of the dot product to find the measure of the angle between two nonzero vectors. The following equation rearranges Equation 2.3 to solve for the cosine of the angle:
    \begin{align*}
        \cos{\theta } = \frac{\vec{u} \cdot \vec{v}}{\norm{\vec{u}}\norm{\vec{v}}}
    .\end{align*}
    \bigbreak \noindent 
    Using this equation, we can find the cosine of the angle between two nonzero vectors. Since we are considering the smallest angle between the vectors, we assume $0^\circ \leq \theta \leq 180^\circ$ (or $0 \leq \theta \leq \pi$ if we are working in radians). The inverse cosine is unique over this range, so we are then able to determine the measure of the angle $\theta$.
    \bigbreak \noindent 
    The angle between two vectors can be acute $(0 < \cos\theta < 1)$, obtuse $(-1 < \cos\theta < 0)$, or straight $(\cos\theta = -1)$. If $\cos\theta = 1$, then both vectors have the same direction. If $\cos\theta = 0$, then the vectors, when placed in standard position, form a right angle. We can formalize this result into a theorem regarding orthogonal (perpendicular) vectors.
    \bigbreak \noindent 
    \fig{.8}{./figures/12.jpeg}
    \bigbreak \noindent 
    \begin{thrmm}[Orthogonal Vectors]
        The nonzero vectors  $\vec{u}$ and  $\vec{v}$ are \textbf{orthogonal vectors} if and only if  $\vec{u} \cdot \vec{v} = 0 $
    \end{thrmm}
    \bigbreak \noindent 
    The angle a vector makes with each of the coordinate axes, called a direction angle, is very important in practical computations, especially in a field such as engineering. For example, in astronautical engineering, the angle at which a rocket is launched must be determined very precisely. A very small error in the angle can lead to the rocket going hundreds of miles off course. Direction angles are often calculated by using the dot product and the cosines of the angles, called the direction cosines. Therefore, we define both these angles and their cosines.
    \pagebreak 
    \begin{dfn}
        The angles formed by a nonzero vector and the coordinate axes are called the \textbf{direction angles} for the vector. The cosines for these angles are called the \textbf{direction cosines}.
    \end{dfn}
    \bigbreak \noindent 
    \fig{.8}{./figures/13.jpeg}

    \bigbreak \noindent 
    \subsection{Projections}
    \bigbreak \noindent 
    As we have seen, addition combines two vectors to create a resultant vector. But what if we are given a vector and we need to find its component parts? We use vector projections to perform the opposite process; they can break down a vector into its components. The magnitude of a vector projection is a scalar projection. For example, if a child is pulling the handle of a wagon at a 55° angle, we can use projections to determine how much of the force on the handle is actually moving the wagon forward. We return to this example and learn how to solve it after we see how to calculate projections.
    \bigbreak \noindent 
    \fig{.8}{./figures/14.jpeg}
    \bigbreak \noindent 
    \pagebreak 
    \begin{dfn}
        The vector projection of $\vec{v}$ onto $\vec{u}$ is the vector labeled $\text{proj}_{\vec{u}}\vec{v}$ in Figure 2.50. It has the same initial point as $\vec{u}$ and $\vec{v}$ and the same direction as $\vec{u}$, and represents the component of $\vec{v}$ that acts in the direction of $\vec{u}$. If $\theta$ represents the angle between $\vec{u}$ and $\vec{v}$, then, by properties of triangles, we know the length of $\text{proj}_{\vec{u}}\vec{v}$ is $\|\text{proj}_{\vec{u}}\vec{v}\| = \|\vec{v}\|\cos\theta$. Note that when the angle $\theta$ between $\vec{u}$ and $\vec{v}$ is an obtuse angle, the projection will be in the opposite direction of $\vec{u}$. When expressing $\cos\theta$ in terms of the dot product, this becomes
        \begin{align*}
            \|\text{proj}_{\vec{u}}\vec{v}\| &= \|\vec{v}\|\cos\theta \\
            &= \|\vec{v}\|\left(\frac{|\vec{u} \cdot \vec{v}|}{\|\vec{u}\|\|\vec{v}\|}\right) \\
            &= \frac{|\vec{u} \cdot \vec{v}|}{\|\vec{u}\|}.
        .\end{align*}
        We now multiply by a unit vector in the direction of $\vec{u}$ to get $\text{proj}_{\vec{u}}\vec{v}$:
        \begin{align*}
            \text{proj}_{\vec{u}}\vec{v} &=\frac{\vec{u}\cdot \vec{v}}{\norm{\vec{u}}}\left(\frac{1}{\norm{\vec{u}}}\vec{u}\right) \\
                                         &=\frac{\vec{u}\cdot \vec{v}}{\norm{\vec{u}}^{2}}\vec{u}
        .\end{align*}
        The length of this vector is also known as the scalar projection of $\vec{v}$ onto $\vec{u}$ and is denoted by
        \[
            \|\text{proj}_{\vec{u}}\vec{v}\| = \text{comp}_{\vec{u}}\vec{v} = \frac{\vec{u} \cdot \vec{v}}{\|\vec{u}\|}.
        \]
    \end{dfn}
    \bigbreak \noindent 
    \fig{.8}{./figures/15.jpeg}

    \bigbreak \noindent 
    \subsection{The resolution of a vector into components}
    \bigbreak \noindent 
    Sometimes it is useful to decompose vectors---that is, to break a vector apart into a sum. This process is called the resolution of a vector into components. Projections allow us to identify two orthogonal vectors having a desired sum. For example, let $\vec{v} = \langle 6, -4 \rangle$ and let $\vec{u} = \langle 3, 1 \rangle$. We want to decompose the vector $\vec{v}$ into orthogonal components such that one of the component vectors has the same direction as $\vec{u}$.
    \bigbreak \noindent 
    We first find the component that has the same direction as $\vec{u}$ by projecting $\vec{v}$ onto $\vec{u}$. Let $\vec{p} = \text{proj}_{\vec{u}}\vec{v}$. Then, we have
    \bigbreak \noindent 
    \begin{align*}
        &\vec{p} = \frac{\vec{u} \cdot \vec{v}}{\|\vec{u}\|^2} \vec{u}  \\
        &=\frac{18-4}{9+1}\vec{u} \\
        &=\frac{7}{5}\vec{u} = \frac{7}{5}\langle 3,1\rangle = \langle \frac{21}{5}, \frac{7}{5}\rangle
    .\end{align*}
    Now consider the vector $\vec{q} = \vec{v} - \vec{p}$. We have
    \begin{align*}
        &\vec{q} = \vec{v} - \vec{p} \\
        &=\langle 6,-4 \rangle - \left\langle \frac{21}{5}, \frac{7}{5} \right\rangle \\
        &=\left\langle \frac{9}{5}, -\frac{27}{5} \right\rangle
    .\end{align*}
    \bigbreak \noindent 
    Clearly, by the way we defined $q$, we have $v = q + p$, and
    \begin{align*}
         &\vec{q} \cdot \vec{p} = \langle \frac{9}{5}, -\frac{27}{5} \rangle \cdot \langle \frac{21}{5}, \frac{7}{5} \rangle \\
         &= \frac{9(21)}{25} + \frac{-27(7)}{25} \\
         &= \frac{\frac{189}{25}}{-\frac{189}{25}} = 0. 
    .\end{align*}
    \bigbreak \noindent 
    Therefore, $\vec{q}$ and $\vec{p}$ are orthogonal.
    \bigbreak \noindent 
    \begin{exm}[Scalar Projection of Velocity]
        A container ship leaves port traveling  $15^{\circ}$
  north of east. Its engine generates a speed of 20 knots along that path (see the following figure). In addition, the ocean current moves the ship northeast at a speed of 2 knots. Considering both the engine and the current, how fast is the ship moving in the direction  $15^{\circ} $
  north of east? Round the answer to two decimal places.
  \bigbreak \noindent 
  \fig{.8}{./figures/16.jpeg}
        
    \end{exm}
    \bigbreak \noindent 
    \textcolor{red}{\textit{Solution.}}
    Let $\vec{v}$ be the velocity vector generated by the engine, and let $\vec{w}$ be the velocity vector of the current. We already know $\|\vec{v}\| = 20$ along the desired route. We just need to add in the scalar projection of $\vec{w}$ onto $\vec{v}$. We get
    \begin{align*}
        &\text{comp}_{\vec{v}}\vec{w} = \frac{\vec{v} \cdot \vec{w}}{\norm{\vec{v}}} \\
        &\frac{\norm{\vec{v}}\norm{\vec{w}}\cos{30^{\circ}}}{\norm{\vec{v}}} \\
        &=\norm{\vec{w}}\cos{(30^{\circ})} \\
        &=2 \frac{\sqrt{3}}{2} = \sqrt{3} \approx  1.73\ \text{knots}
    .\end{align*}
    \bigbreak \noindent 
    The ship is moving at 21.73 knots in the direction $15^{\circ}$ north of east

    \bigbreak \noindent 
    \subsection{Work}
    \bigbreak \noindent 
    Now that we understand dot products, we can see how to apply them to real-life situations. The most common application of the dot product of two vectors is in the calculation of work.
    \bigbreak \noindent 
    From physics, we know that work is done when an object is moved by a force. When the force is constant and applied in the same direction the object moves, then we define the work done as the product of the force and the distance the object travels: $W = Fd$.
    We saw several examples of this type in earlier chapters. Now imagine the direction of the force is different from the direction of motion, as with the example of a child pulling a wagon. To find the work done, we need to multiply the component of the force that acts in the direction of the motion by the magnitude of the displacement. The dot product allows us to do just that. If we represent an applied force by a vector $\mathbf{F}$ and the displacement of an object by a vector $\mathbf{s}$, then the work done by the force is the dot product of $\mathbf{F}$ and $\mathbf{s}$.
    \bigbreak \noindent 
    \begin{dfn}[Work]
        When a constant force is applied to an object so the object moves in a straight line from point $P$ to point $Q$, the work $W$ done by the force $\mathbf{F}$, acting at an angle $\theta$ from the line of motion, is given by
        \[ W = \vec{F} \cdot \overrightarrow{PQ} = \|\vec{F}\| \|\overrightarrow{PQ}\| \cos \theta. \]
    \end{dfn}
    \bigbreak \noindent 
    Let’s revisit the problem of the child’s wagon introduced earlier. Suppose a child is pulling a wagon with a force having a magnitude of 8 lb on the handle at an angle of $55^{\circ}$. If the child pulls the wagon 50 ft, find the work done by the force. 
    \bigbreak \noindent 
    \fig{.8}{./figures/17.jpeg}
    \bigbreak \noindent 
    We have 
    \begin{align*}
        W &= \norm{\vec{F}} \norm{\vec{PQ}}\cos{\theta } \\
       &=8(50)(\cos{(55^{\circ})}) \\
       &\approx 229\ \text{ft} \cdot \text{lb}
    .\end{align*}


    


    



    


    


 

    

\end{document}
