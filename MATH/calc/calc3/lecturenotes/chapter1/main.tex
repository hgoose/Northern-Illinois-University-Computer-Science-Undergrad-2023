\documentclass{report}

\input{~/dev/latex/template/preamble.tex}
\input{~/dev/latex/template/macros.tex}

\title{\Huge{}}
\author{\huge{Nathan Warner}}
\date{\huge{}}
\fancyhf{}
\rhead{}
\fancyhead[R]{\itshape Warner} % Left header: Section name
\fancyhead[L]{\itshape\leftmark}  % Right header: Page number
\cfoot{\thepage}
\renewcommand{\headrulewidth}{0pt} % Optional: Removes the header line
%\pagestyle{fancy}
%\fancyhf{}
%\lhead{Warner \thepage}
%\rhead{}
% \lhead{\leftmark}
%\cfoot{\thepage}
%\setborder
% \usepackage[default]{sourcecodepro}
% \usepackage[T1]{fontenc}

% Change the title
\hypersetup{
    pdftitle={Parametric Equations and Polar Coordinates}
}

\begin{document}
    % \maketitle
        \begin{titlepage}
       \begin{center}
           \vspace*{1cm}
    
           \textbf{Chapter I} \\
           Parametric Equations and Polar Coordinates
    
           \vspace{0.5cm}
            
                
           \vspace{1.5cm}
    
           \textbf{Nathan Warner}
    
           \vfill
                
                
           \vspace{0.8cm}
         
           \includegraphics[width=0.4\textwidth]{~/niu/seal.png}
                
           Computer Science \\
           Northern Illinois University\\
           February 16, 2023 \\
           United States\\
           
                
       \end{center}
    \end{titlepage}
    \tableofcontents
    \pagebreak \bigbreak \noindent 
    \beginch{1}{Parametric Equations and Polor Coordinates}

    \bigbreak \noindent 
    \unsect{1.1 Parametric Equations}
    \bigbreak \noindent 
    In the two-dimensional coordinate system, parametric equations are useful for describing curves that are not necessarily functions. The parameter is an independent variable that both \( x \) and \( y \) depend on, and as the parameter increases, the values of \( x \) and \( y \) trace out a path along a plane curve. For example, if the parameter is \( t \) (a common choice), then \( t \) might represent time. Then \( x \) and \( y \) are defined as functions of time, and \( (x(t), y(t)) \) can describe the position in the plane of a given object as it moves along a curved path.
    \bigbreak \noindent 
    Consider the orbit of Earth around the Sun. Our year lasts approximately 365.25 days, but for this discussion we will use 365 days. On January 1 of each year, the physical location of Earth with respect to the Sun is nearly the same, except for leap years, when the lag introduced by the extra \( \frac{1}{4} \) day of orbiting time is built into the calendar. We call January 1 ``day 1'' of the year. Then, for example, day 31 is January 31, day 59 is February 28, and so on.
    \bigbreak \noindent 
    \begin{minipage}[b]{0.47\textwidth}
        The number of the day in a year can be considered a variable that determines Earth’s position in its orbit. As Earth revolves around the Sun, its physical location changes relative to the Sun. After one full year, we are back where we started, and a new year begins. According to Kepler’s laws of planetary motion, the shape of the orbit is elliptical, with the Sun at one focus of the ellipse1. 
    \end{minipage}
    \hspace{.1in} 
    \begin{minipage}[t]{0.47\textwidth}
        \fig{.8}{./figures/1.jpeg}
    \end{minipage}

    \bigbreak \noindent 
    \begin{minipage}[]{0.47\textwidth}
        Figure 1 depicts Earth’s orbit around the Sun during one year. The point labeled \( F_2 \) is one of the foci of the ellipse; the other focus is occupied by the Sun. If we superimpose coordinate axes over this graph, then we can assign ordered pairs to each point on the ellipse (Figure 2). Then each \( x \) value on the graph is a value of position as a function of time, and each \( y \) value is also a value of position as a function of time. Therefore, each point on the graph corresponds to a value of Earth’s position as a function of time.
    \end{minipage}
    \hspace{.1in}
    \begin{minipage}[]{0.47\textwidth}
        \fig{.8}{./figures/2.jpeg}
    \end{minipage}
    \bigbreak \noindent 
    \begin{dfn}
        If \( x \) and \( y \) are continuous functions of \( t \) on an interval \( I \), then the equations
        \[ x = x(t) \quad \text{and} \quad y = y(t) \]
        are called \textbf{parametric equations} and \( t \) is called the \textbf{parameter}. The set of points \( (x, y) \) obtained as \( t \) varies over the interval \( I \) is called the graph of the parametric equations. The graph of parametric equations is called a \textbf{parametric curve} or plane curve, and is denoted by \( C \).
    \end{dfn}
    \bigbreak \noindent 
    \nt{Notice in this definition that \( x \) and \( y \) are used in two ways. The first is as functions of the independent variable \( t \). As \( t \) varies over the interval \( I \), the functions \( x(t) \) and \( y(t) \) generate a set of ordered pairs \( (x, y) \). This set of ordered pairs generates the graph of the parametric equations. In this second usage, to designate the ordered pairs, \( x \) and \( y \) are variables. It is important to distinguish the variables \( x \) and \( y \) from the functions \( x(t) \) and \( y(t) \).}
    \pagebreak 
    \begin{exm}[Graphing a Parametrically Defined Curve]
        Sketch the curves described by the following parametric equations:
        \begin{equation}
            \begin{alignedat}{3}
                % \text{(a)}\quad x(t) &= t - 1, \quad & y(t) &= 2t + 4, \quad & -3 &\leq t \leq 2 \\
                \text{(a)}\quad &x:\ \mathbb{R} \to \mathbb{R}:\ t \mapsto t - 1, \quad &&y:\ \mathbb{R} \to \mathbb{R}:\ t\mapsto 2t+4, \quad &&-3 \leq t \leq 2 \\
                \text{(b)}\quad &x:\ \mathbb{R} \to \mathbb{R}:\ t\mapsto t^{2}-3, \quad &&y:\ \mathbb{R} \to \mathbb{R}:\ t\mapsto 2t+1 , \quad &&-2 \leq t \leq 3 \\
                \text{(c)}\quad &x:\ \mathbb{R} \to \mathbb{R}:\ t\mapsto 4\cos{t}, \quad &&y:\ \mathbb{R} \to \mathbb{R}:\ t\mapsto 4\sin{t}, \quad &&0 \leq t \leq 2\pi
            \end{alignedat}
        \end{equation}
    \end{exm}
    \bigbreak \noindent 
    \textcolor{red}{\textit{Solution a.}}
    To create a graph of this curve, first set up a table of values. Since the independent variable in both \( x(t) \) and \( y(t) \) is \( t \), let \( t \) appear in the first column. Then \( x(t) \) and \( y(t) \) will appear in the second and third columns of the table.
    \bigbreak \noindent 
    \begin{tabularx}{\textwidth}{|X|X|X|}
        \hline
        \( t \) & \( x(t) \) & \( y(t) \) \\
        \hline
        -3 & -4 & -2 \\
        -2 & -3 & 0 \\
        -1 & -2 & 2 \\
        0 & -1 & 4 \\
        1 & 0 & 6 \\
        2 & 1 & 8 \\
        \hline
    \end{tabularx}
    \bigbreak \noindent 
    \begin{minipage}[b]{0.47\textwidth}
        The second and third columns in this table provide a set of points to be plotted. The graph of these points appears in Figure 3. The arrows on the graph indicate the orientation of the graph, that is, the direction that a point moves on the graph as t varies from −3 to 2.
    \end{minipage}
    \hspace{.1in}
    \begin{minipage}[]{0.47\textwidth}
        \fig{.8}{./figures/3.jpeg}
    \end{minipage}
    \pagebreak \bigbreak \noindent 
    \textcolor{red}{\textit{Solution b.}}
    To create a graph of this curve, again set up a table of values.
    \bigbreak \noindent 
    \begin{tabularx}{\textwidth}{|X|X|X|}
        \hline
        $t$ & $x(t)$ & $y(t)$ \\
        \hline
        -2 & 1 & -3 \\
        -1 &-2 & -1 \\
        0 & -3  & 1 \\
        1 & -2 & 3 \\
        2 & 1 & 5 \\
        3 & 6 & 7  \\
        \hline
    \end{tabularx}
    \bigbreak \noindent 
    As $t$ progresses from −2 to 3, the point on the curve travels along a parabola. The direction the point moves is again called the orientation and is indicated on the graph.
    \bigbreak \noindent 
    \fig{.8}{./figures/4.jpeg} 
    \bigbreak \noindent 
    \textcolor{red}{\textit{Solution c.}}
    In this case, use multiples of  $\frac{\pi}{6}$ for $t$ and create another table of values
    \bigbreak \noindent 
    \begin{minipage}[]{0.47\textwidth}
        \begin{tabularx}{\textwidth}{|X|X|X|}
            \hline
            $t$ & $x(t)$ & y(t) \\
            \hline
            0 & 4 &0 \\
            $\frac{\pi}{6}$ & $2\sqrt{3} \approx 3.5 $ & 2 \\
            $\frac{\pi}{3}$ & 2 & $2\sqrt{3}$ \\
            $\frac{\pi}{2}$ & 0 & 4 \\
            $\frac{2\pi}{3}$ & -2 & $2\sqrt{3}$ \\
            $\frac{5\pi}{6}$ & $-2\sqrt{3} $ & 2 \\
            $\pi$ & $-4$ & 0 \\
            $\frac{7\pi}{6}$ & $-2\sqrt{3} $ & 2 \\
            $\frac{4\pi}{3}$   & -2 & $-2\sqrt{3} $ \\
            $\frac{3\pi}{2}$  & 0 & 4 \\
            $\frac{5\pi}{3}$  & 2 & $-2\sqrt{3} $ \\
            $\frac{11\pi}{6}$ & $2\sqrt{3}$ & 2\\
            $2\pi$ & 4 & 0  \\
            \hline
        \end{tabularx}
    \end{minipage}
    \hspace{.1in}
    \begin{minipage}[]{0.47\textwidth}
        \fig{.8}{./figures/5.jpeg}
    \end{minipage}

    \pagebreak \bigbreak \noindent 
    \subsection{Eliminating the Parameter}
    \bigbreak \noindent 
    To better understand the graph of a curve represented parametrically, it is useful to rewrite the two equations as a single equation relating the variables \( x \) and \( y \). Then we can apply any previous knowledge of equations of curves in the plane to identify the curve. For example, the equations describing the plane curve in Example 1.1b. are
    \[
        x(t) = t^2 - 3, \quad y(t) = 2t + 1, \quad -2 \leq t \leq 3.
    \]
    Solving the second equation for \( t \) gives
    \[
        t = \frac{y - 1}{2}.
    \]
    This can be substituted into the first equation:
    \begin{align*}
        &x = \left(\frac{y-1}{2}\right)^{2} - 3 \\
        &x = \frac{(y-1)^{2}}{4} - 3 \\
        &x = \frac{y^{2}-2y+1}{4}-3 \\
        &x = \frac{y^{2}-2y-11}{4}
    .\end{align*}
    This equation describes \( x \) as a function of \( y \). These steps give an example of \textit{eliminating the parameter}. The graph of this function is a parabola opening to the right. Recall that the plane curve started at \( (1, -3) \) and ended at \( (6, 7) \). These terminations were due to the restriction on the parameter \( t \).
    \bigbreak \noindent 
    Sometimes it is necessary to be a bit creative in eliminating the parameter. The parametric equations for this example are
    \begin{align*}
        x(t) = 4\cos{t}, \quad y(t) = 3\sin{t}
    .\end{align*}
    \bigbreak \noindent 
    Solving either equation for $t$ directly is not advisable because sine and cosine are not one-to-one functions. However, dividing the first equation by 4 and the second equation by 3 (and suppressing the $t$) gives us
    \begin{align*}
        \cos{t} = \frac{x}{4}, \quad \sin{t} = \frac{y}{3}
    .\end{align*}
    \bigbreak \noindent 
    Now use the Pythagorean identity  $\cos^{2}{t} + \sin^{2}{t} = 1$ and replace the expressions for $\sin{t}$ and $\cos{t}$ with the equivalent expressions in terms of $x$ and $y$. This gives
    \begin{align*}
        &\left(\frac{x}{4}\right)^{2} +  \left(\frac{y}{3}\right)^{2} = 1 \\
        &\frac{x^{2}}{16}  + \frac{y^{2}}{9} = 1
    .\end{align*}
    \pagebreak \bigbreak \noindent 
    \begin{minipage}[b]{0.47\textwidth}
        This is the equation of a horizontal ellipse centered at the origin, with semimajor axis 4 and semiminor axis 3 as shown in the following graph. 
    \end{minipage}
    \hspace{.1in}
    \begin{minipage}[]{0.47\textwidth}
        \fig{.8}{./figures/6.jpeg} 
    \end{minipage}
    \bigbreak \noindent 
    So far we have seen the method of eliminating the parameter, assuming we know a set of parametric equations that describe a plane curve. What if we would like to start with the equation of a curve and determine a pair of parametric equations for that curve? This is certainly possible, and in fact it is possible to do so in many different ways for a given curve. The process is known as \textbf{parameterization of a curve}.
    \bigbreak \noindent 
    \begin{exm}[Parameterizing a Curve]
        Find two different pairs of parametric equations to represent the graph of  $y=2x^{2}-3.$
    \end{exm}
    \bigbreak \noindent 
    \textcolor{red}{\textit{Solution.}}
    First, it is always possible to parameterize a curve by defining \( x(t) = t \), then replacing \( x \) with \( t \) in the equation for \( y(t) \). This gives the parameterization
    \[
        x(t) = t, \quad y(t) = 2t^2 - 3.
    \]
    \bigbreak \noindent 
    We have complete freedom in the choice for the second parameterization. For example, we can choose \( x(t) = 3t - 2 \). The only thing we need to check is that there are no restrictions imposed on \( x \); that is, the range of \( x(t) \) is all real numbers. This is the case for \( x(t) = 3t - 2 \). Now since \( y = 2x^2 - 3 \), we can substitute \( x(t) = 3t - 2 \) for \( x \). This gives
    \begin{align*}
        &y(t) = 2(3t - 2)^2 - 3 \\
        &= 2(9t^2 - 12t + 4) - 3 \\
        &= 18t^2 - 24t + 8 - 3 \\
        &= 18t^2 - 24t + 5
    .\end{align*}
    Therefore, a second parameterization of the curve can be written as
    \[
        x(t) = 3t - 2 \quad \text{and} \quad y(t) = 18t^2 - 24t + 5.
    \]

    \pagebreak 
    \subsection{Cycloids and Other Parametric Curves}
    \bigbreak \noindent 
    Imagine going on a bicycle ride through the country. The tires stay in contact with the road and rotate in a predictable pattern. Now suppose a very determined ant is tired after a long day and wants to get home. So he hangs onto the side of the tire and gets a free ride. The path that this ant travels down a straight road is called a \textbf{cycloid} (Figure 7). A cycloid generated by a circle (or bicycle wheel) of radius $a$ is given by the parametric equations
    \begin{align*}
        x(t) = a(t-\sin{t}), \quad y(t) = a(1-\cos{t})
    .\end{align*}
    \bigbreak \noindent 
    To see why this is true, consider the path that the center of the wheel takes. The center moves along the x-axis at a constant height equal to the radius of the wheel. If the radius is \( a \), then the coordinates of the center can be given by the equations
    \[
        x(t) = at, \quad y(t) = a
    \]
    for any value of \( t \). Next, consider the ant, which rotates around the center along a circular path. If the bicycle is moving from left to right then the wheels are rotating in a clockwise direction. A possible parameterization of the circular motion of the ant (relative to the center of the wheel) is given by
    \[
        x(t) = -a\sin t, \quad y(t) = -a\cos t.
    \]
    (The negative sign is needed to reverse the orientation of the curve. If the negative sign were not there, we would have to imagine the wheel rotating counterclockwise.) Adding these equations together gives the equations for the cycloid.
    \[
        x(t) = a(t - \sin t), \quad y(t) = a(1 - \cos t).
    \]
    \bigbreak \noindent 
    \fig{.8}{./figures/7.jpeg}
    \bigbreak \noindent 
    \begin{minipage}[b]{0.47\textwidth}
        Now suppose that the bicycle wheel doesn’t travel along a straight road but instead moves along the inside of a larger wheel, as in Figure 8. In this graph, the green circle is traveling around the blue circle in a counterclockwise direction. A point on the edge of the green circle traces out the red graph, which is called a hypocycloid.
    \end{minipage}
    \hspace{.1in}
    \begin{minipage}[]{0.47\textwidth}
        \fig{.8}{./figures/8.jpeg}
    \end{minipage}
    \bigbreak \noindent 
    The general parametric equations for a hypocycloid are
    \begin{align*}
        &x(t) = (a-b)\ \cos{t} + b \cos{\left(\frac{a-b}{b}\right)}t \\
        &y(t) = (a-b)\ \sin{t} + b \sin{\left(\frac{a-b}{b}\right)}t \\
    .\end{align*}

    \bigbreak \noindent 
    These equations are a bit more complicated, but the derivation is somewhat similar to the equations for the cycloid. In this case we assume the radius of the larger circle is \( a \) and the radius of the smaller circle is \( b \). Then the center of the wheel travels along a circle of radius \( a - b \). This fact explains the first term in each equation above. The period of the second trigonometric function in both \( x(t) \) and \( y(t) \) is equal to \( \frac{2\pi b}{a - b} \).
    \bigbreak \noindent 
    \begin{minipage}[]{0.47\textwidth}
        The ratio \( \frac{a}{b} \) is related to the number of cusps on the graph (cusps are the corners or pointed ends of the graph), as illustrated in Figure 9. This ratio can lead to some very interesting graphs, depending on whether or not the ratio is rational. Figure 1.10 corresponds to \( a = 4 \) and \( b = 1 \). The result is a hypocycloid with four cusps. Figure 1.11 shows some other possibilities. The last two hypocycloids have irrational values for \( \frac{a}{b} \). In these cases the hypocycloids have an infinite number of cusps, so they never return to their starting point. These are examples of what are known as space-filling curves.
    \end{minipage}
    \hspace{.1in}
    \begin{minipage}[]{0.47\textwidth}
        \fig{0.5}{./figures/9.jpeg}
    \end{minipage}

    \pagebreak 
    \unsect{1.2 Calculus of Parametric Curves}
    \bigbreak \noindent 
    Now that we have introduced the concept of a parameterized curve, our next step is to learn how to work with this concept in the context of calculus.
    \bigbreak \noindent 
    \subsection{Derivatives of Parametric Equations}
    \bigbreak \noindent 
    \begin{thrmm}[Derivative of Parametric Equations]
        Consider the plane curve defined by the parametric equations \( x = x(t) \) and \( y = y(t) \). Suppose that \( x'(t) \) and \( y'(t) \) exist, and assume that \( x'(t) \neq 0 \). Then the derivative \( \frac{dy}{dx} \) is given by
        \[
            \frac{dy}{dx} = \frac{\frac{dy}{dt}}{\frac{dx}{dt}} = \frac{y'(t)}{x'(t)}.
        \]
    \end{thrmm}
    \bigbreak \noindent 
    \pf{Proof}{
        This theorem can be proven using the Chain Rule. In particular, assume that the parameter \( t \) can be eliminated, yielding a differentiable function \( y = F(x) \). Then \( y(t) = F(x(t)) \). Differentiating both sides of this equation using the Chain Rule yields
        \[
            y'(t) = F'(x(t)) \cdot x'(t),
        \]
        so
        \[
            F'(x(t)) = \frac{y'(t)}{x'(t)}.
        \]
        But \( F'(x(t)) = \frac{dy}{dx} \), which proves the theorem.
        \(\blacksquare\)
    }
    \bigbreak \noindent 
    \nt{This theorem can be used to calculate derivatives of plane curves, as well as critical points. Recall that a critical point of a differentiable function \( y = f(x) \) is any point \( x = x_0 \) such that either \( f'(x_0) = 0 \) or \( f'(x_0) \) does not exist. Equation 1.1 gives a formula for the slope of a tangent line to a curve defined parametrically regardless of whether the curve can be described by a function \( y = f(x) \) or not.}

    \pagebreak 
    \subsection{Second-Order Derivatives}
    \bigbreak \noindent 
    Our next goal is to see how to take the second derivative of a function defined parametrically. The second derivative of a function \( y = f(x) \) is defined to be the derivative of the first derivative; that is,
    \[
        \frac{d^2y}{dx^2} = \frac{d}{dx}\left[\frac{dy}{dx}\right].
    \]
    Since \( \frac{dy}{dx} = \frac{dy/dt}{dx/dt} \), we can replace the \( y \) on both sides of this equation with \( \frac{dy}{dx} \). This gives us
    \[
        \frac{d^2y}{dx^2} = \frac{d}{dx}\left(\frac{dy}{dx}\right) = \left(\frac{d}{dt}\right)\left(\frac{dy}{dx}\right)\frac{dx}{dt}.
    \]
    If we know \( \frac{dy}{dx} \) as a function of \( t \), then this formula is straightforward to apply.

    \bigbreak \noindent 
    \begin{exm}[second order derivative]
        Calculate the second derivative \(\frac{d^2y}{dx^2}\) for the plane curve defined by the parametric equations \( x(t) = t^2 - 3 \), \( y(t) = 2t - 1 \), for \( -3 \leq t \leq 4 \).
    \end{exm}
    \bigbreak \noindent 
    \textcolor{red}{\textit{Solution.}}
    \begin{align*}
        &\frac{dy}{dx} = \frac{y^{\prime}(t)}{x^{\prime}(t)} = \frac{2}{2t} =\frac{1}{t} \\
        &\implies \frac{d^{2}y}{dx^{2}} = \frac{\left(\frac{d}{dt}\right)\left(\frac{dy}{dx}\right)}{\frac{dx}{dt}} = \frac{\frac{d}{dt}\left(\frac{1}{t}\right)}{2t} \\
        &=-\frac{\frac{1}{t^{2}}}{2t} \\
        &=-\frac{1}{2t^{3}}
    .\end{align*}

    \pagebreak 
    \subsection{Integrals Involving Parametric Equations}
    \bigbreak \noindent 
    Suppose we have a parametric curve defined by the set of equations
    \begin{align*}
        &x(t) = f(t) \\
        &y(t) = g(t)
    .\end{align*}
    \bigbreak \noindent 
    For $c \leq t \leq d$. Assuming elimination of the parameter is possible, we have
    \begin{align*}
        y = f(x)
    .\end{align*}
    \bigbreak \noindent 
    Given what we know about the riemann integral $\int_{a}^{b}\ f(x)\ dx $, we can use our assumption to get $\int_{a}^{b}\ y\ dx $. Solving for $dx$ we get
    \begin{align*}
        &\frac{dx}{dt} = f^{\prime}(t) \\
        &dx = f^{\prime}(t)dt
    .\end{align*}
    Thus, now we have the integral 
    \begin{align*}
        \int_{a}^{b}\ yf^{\prime}(t)\ dt
    .\end{align*}
    Since we know $y=g(t)$, and adjusting the bounds for values of $t$, we arrive at 
    \begin{align*}
        &\int_{c}^{d}\ g(t)f^{\prime}(t)\ dt \\
        &\text{Or simply}\ \int_{c}^{d}\ y(t)x^{\prime}(t)\ dt
    .\end{align*}
    \bigbreak \noindent 
    This leads to the following theorem
    \bigbreak \noindent 
    \begin{thrmm}[Integral Involving Parametric Equations]
        Consider the non-self-intersecting plane curve defined by the parametric equations
        \[
            x = x(t), \quad y = y(t), \quad a \leq t \leq b
        \]
        and assume that \( x(t) \) is differentiable. The area under this curve is given by
        \[
            A = \int_{a}^{b} y(t) x'(t) \, dt.
        \]
    \end{thrmm}

    \pagebreak 
    \subsection{Arc Length of a Parametric Curve}
    \bigbreak \noindent 
    \begin{minipage}[]{0.47\textwidth}
        In addition to finding the area under a parametric curve, we sometimes need to find the arc length of a parametric curve. In the case of a line segment, arc length is the same as the distance between the endpoints. If a particle travels from point A to point B along a curve, then the distance that particle travels is the arc length. To develop a formula for arc length, we start with an approximation by line segments as shown in the following graph.
    \end{minipage}
    \hspace{.1in}
    \begin{minipage}[]{0.47\textwidth}
        \fig{.8}{./figures/10.jpeg}
    \end{minipage}
    \bigbreak \noindent 
    Given a plane curve defined by the functions \( x = x(t) \), \( y = y(t) \), \( a \leq t \leq b \), we start by partitioning the interval \([a, b]\) into \( n \) equal subintervals: \( t_0 = a < t_1 < t_2 < \cdots < t_n = b \). The width of each subinterval is given by \( \Delta t = \frac{b - a}{n} \). We can calculate the length of each line segment:
    \[
        d_1 = \sqrt{(x(t_1) - x(t_0))^2 + (y(t_1) - y(t_0))^2}
    \]
    \[
        d_2 = \sqrt{(x(t_2) - x(t_1))^2 + (y(t_2) - y(t_1))^2}
    \]
    \bigbreak \noindent 
    Then add these up. We let \( s \) denote the exact arc length and \( s_n \) denote the approximation by \( n \) line segments:
    \[
        s \approx \sum_{k=1}^{n} s_k = \sum_{k=1}^{n} \sqrt{(x(t_k) - x(t_{k-1}))^2 + (y(t_k) - y(t_{k-1}))^2}.
    \]
    If we assume that \( x(t) \) and \( y(t) \) are differentiable functions of \( t \), then the Mean Value Theorem (Introduction to the Applications of Derivatives) applies, so in each subinterval \([t_{k-1}, t_k]\) there exist \( \hat{t}_k \) and \( \tilde{t}_k \) such that
    \[
        x(t_k) - x(t_{k-1}) = x'(\hat{t}_k)(t_k - t_{k-1}) = x'(\hat{t}_k) \Delta t
    \]
    \[
        y(t_k) - y(t_{k-1}) = y'(\tilde{t}_k)(t_k - t_{k-1}) = y'(\tilde{t}_k) \Delta t.
    \]
    \bigbreak \noindent 
    Therefore the equation becomes
    \begin{align*}
        &s \approx \sum_{k=1}^{n} s_k \\
        &= \sum_{k=1}^{n} \sqrt{(x'(\hat{t}_k) \Delta t)^2 + (y'(\tilde{t}_k) \Delta t)^2} \\
        &= \sum_{k=1}^{n} \sqrt{(x'(\hat{t}_k))^2 (\Delta t)^2 + (y'(\tilde{t}_k))^2 (\Delta t)^2}  \\
        &= \left( \sum_{k=1}^{n} \sqrt{(x'(\hat{t}_k))^2 + (y'(\tilde{t}_k))^2} \right) \Delta t
    .\end{align*}
    \pagebreak \bigbreak \noindent 
    This is a Riemann sum that approximates the arc length over a partition of the interval  [a,b]. If we further assume that the derivatives are continuous and let the number of points in the partition increase without bound, the approximation approaches the exact arc length. This gives
    \begin{align*}
        &s = \lim_{n \to \infty} \sum_{k=1}^{n} s_k  \\
        &= \lim_{n \to \infty} \left( \sum_{k=1}^{n} \sqrt{(x'(\hat{t}_k))^2 + (y'(\tilde{t}_k))^2} \right) \Delta t  \\
        &= \int_{a}^{b} \sqrt{(x'(t))^2 + (y'(t))^2} \, dt
    .\end{align*}
    \bigbreak \noindent 
    \nt{When taking the limit, the values of \( \hat{t}_k \) and \( \tilde{t}_k \) are both contained within the same ever-shrinking interval of width \( \Delta t \), so they must converge to the same value.}

    \bigbreak \noindent 
    \begin{thrmm}[Arc Length of a Parametric Curve]
        Consider the plane curve defined by the parametric equations
        \[
            x = x(t), \quad y = y(t), \quad t_1 \leq t \leq t_2
        \]
        and assume that \( x(t) \) and \( y(t) \) are differentiable functions of \( t \). Then the arc length of this curve is given by
        \[
            s = \int_{t_1}^{t_2} \sqrt{\left(\frac{dx}{dt}\right)^2 + \left(\frac{dy}{dt}\right)^2} \, dt.
        \]
    \end{thrmm}
    \bigbreak \noindent 
    A side derivation can be made. Suppose the parameter can be eliminated, leading to a function \( y = F(x) \). Then \( y(t) = F(x(t)) \) and the Chain Rule gives \( y'(t) = F'(x(t)) \cdot x'(t) \). Substituting into the theorem above we get 
    \begin{align*}
        &s = \int_{t_1}^{t_2} \sqrt{\left(\frac{dx}{dt}\right)^2 + \left(\frac{dy}{dt}\right)^2} \, dt  \\
        &= \int_{t_1}^{t_2} \sqrt{\left(\frac{dx}{dt}\right)^2 + \left(F'(x) \frac{dx}{dt}\right)^2} \, dt  \\
        &= \int_{t_1}^{t_2} \sqrt{\left(\frac{dx}{dt}\right)^2 \left(1 + (F'(x))^2\right)} \, dt  \\
        &= \int_{t_1}^{t_2} x'(t) \sqrt{1 + \left(\frac{dy}{dx}\right)^2} \, dt. 
    .\end{align*}
    \bigbreak \noindent 
    Here we have assumed that \( x'(t) > 0 \), which is a reasonable assumption. The Chain Rule gives \( dx = x'(t) \, dt \), and letting \( a = x(t_1) \) and \( b = x(t_2) \) we obtain the formula
    \[
    s = \int_{a}^{b} \sqrt{1 + \left(\frac{dy}{dx}\right)^2} \, dx,
    \]

    \pagebreak 
    \subsection{Surface Area Generated by a Parametric Curve}
    \bigbreak \noindent 
    \begin{thrmm}[Surface area for a parametric curve]
        The analogous formula for a parametrically defined curve is
        \[
            S = 2\pi \int_{a}^{b} y(t) \sqrt{(x'(t))^2 + (y'(t))^2} \, dt
        \]
        provided that \( y(t) \) is not negative on \([a, b]\).
    \end{thrmm}

    \begin{align*}
        \frac{-1-\sqrt{1+4x}}{2} < 0 
    .\end{align*}
    

    \pagebreak 
    \unsect{1.3 Polar Coordinates}
    \bigbreak \noindent 
    \subsection{Defining Polar Coordinates}
    \bigbreak \noindent 
    \begin{minipage}[]{0.47\textwidth}
        To find the coordinates of a point in the polar coordinate system, consider Figure 11. The point \( P \) has Cartesian coordinates \((x, y)\). The line segment connecting the origin to the point \( P \) measures the distance from the origin to \( P \) and has length \( r \). The angle between the positive \( x \)-axis and the line segment has measure \( \theta \). This observation suggests a natural correspondence between the coordinate pair \((x, y)\) and the values \( r \) and \( \theta \). This correspondence is the basis of the polar coordinate system. Note that every point in the Cartesian plane has two values (hence the term ordered pair) associated with it. In the polar coordinate system, each point also has two values associated with it: \( r \) and \( \theta \).
    \end{minipage}
    \hspace{.1in}
    \begin{minipage}[]{0.47\textwidth}
        \fig{.8}{./figures/11.jpeg}
    \end{minipage}
    \bigbreak \noindent 
    Using right-triangle trigonometry, the following equations are true for the point \( P \):
    \[
        \cos \theta = \frac{x}{r} \quad \text{so} \quad x = r \cos \theta
    \]
    \[
        \sin \theta = \frac{y}{r} \quad \text{so} \quad y = r \sin \theta.
    \]
    Furthermore,
    \[
        r^2 = x^2 + y^2 \quad \text{and} \quad \tan \theta = \frac{y}{x}.
    \]

    \bigbreak \noindent 
    \begin{thrmm}[Converting Points between Coordinate Systems]
        Given a point \( P \) in the plane with Cartesian coordinates \((x, y)\) and polar coordinates \((r, \theta)\), the following conversion formulas hold true:
        \begin{align*}
            x = r \cos \theta \quad \text{and} \quad y = r \sin \theta,
        .\end{align*}
        \begin{align*}
            r^2 = x^2 + y^2 \quad \text{and} \quad \tan \theta = \frac{y}{x}.
        .\end{align*}
        These formulas can be used to convert from rectangular to polar or from polar to rectangular coordinates.
    \end{thrmm}

    \pagebreak \bigbreak \noindent 
    The polar representation of a point is not unique. For example, the polar coordinates \((2, \frac{\pi}{3})\) and \((2, \frac{7\pi}{3})\) both represent the point \((1, \sqrt{3})\) in the rectangular system. Also, the value of \( r \) can be negative. Therefore, the point with polar coordinates \((-2, \frac{4\pi}{3})\) also represents the point \((1, \sqrt{3})\) in the rectangular system, as we can see by using Equation 1.8:
    \begin{align*}
        x &= r \cos \theta \\
        &= -2 \cos\left(\frac{4\pi}{3}\right) \\
        &= -2 \left(-\frac{1}{2}\right) \\
        &= 1
    .\end{align*}
    and
    \begin{align*}
        y &= r \sin \theta \\
        &= -2 \sin\left(\frac{4\pi}{3}\right) \\
        &= -2 \left(-\frac{\sqrt{3}}{2}\right) \\
        &= \sqrt{3}.
    .\end{align*}
    Every point in the plane has an infinite number of representations in polar coordinates. However, each point in the plane has only one representation in the rectangular coordinate system.
    \bigbreak \noindent 
    \begin{minipage}[]{0.47\textwidth}
        Note that the polar representation of a point in the plane also has a visual interpretation. In particular, \( r \) is the directed distance that the point lies from the origin, and \( \theta \) measures the angle that the line segment from the origin to the point makes with the positive \( x \)-axis. Positive angles are measured in a counterclockwise direction and negative angles are measured in a clockwise direction. The polar coordinate system appears in the following figure.
    \end{minipage}
    \hspace{.1in}
    \begin{minipage}[]{0.47\textwidth}
        \fig{.7}{./figures/12.jpeg} 
    \end{minipage}

    \pagebreak 
    \subsection{Polar Curves}
    \bigbreak \noindent 
    Now that we know how to plot points in the polar coordinate system, we can discuss how to plot curves. In the rectangular coordinate system, we can graph a function \( y = f(x) \) and create a curve in the Cartesian plane. In a similar fashion, we can graph a curve that is generated by a function \( r = f(\theta) \).
    \bigbreak \noindent 
    The general idea behind graphing a function in polar coordinates is the same as graphing a function in rectangular coordinates. Start with a list of values for the independent variable \( \theta \) in this case) and calculate the corresponding values of the dependent variable \( r \). This process generates a list of ordered pairs, which can be plotted in the polar coordinate system. Finally, connect the points, and take advantage of any patterns that may appear. The function may be periodic, for example, which indicates that only a limited number of values for the independent variable are needed.
    \bigbreak \noindent 
    \subsubsection{Plotting a Curve in Polar Coordinates}
    \begin{enumerate}
        \item Create a table with two columns. The first column is for \( \theta \), and the second column is for \( r \).
        \item Create a list of values for \( \theta \).
        \item Calculate the corresponding \( r \) values for each \( \theta \).
        \item Plot each ordered pair \( (r, \theta) \) on the coordinate axes.
        \item Connect the points and look for a pattern.
    \end{enumerate}

    \bigbreak \noindent 
    \begin{exm}[Graphing a polar equation]
        Graph the curve defined by the function $r=4\sin{\theta}$. Identify the curve and rewrite the equation in rectangular coordinates.
        
    \end{exm}
    \bigbreak \noindent 
    \textcolor{red}{\textit{Solution.}}
    Because the function is a multiple of a sine function, it is periodic with period  $2\pi$, so use values for $\theta$ between 0 and  $2\pi$
    \bigbreak \noindent 
    \begin{center}
        \begin{tabular}{cc|cc}
            \toprule
            \( \theta \) & \( r = 4\sin\theta \) & \( \theta \) & \( r = 4\sin\theta \) \\
            \midrule
            0 & 0 & \( \pi \) & 0 \\
            \( \frac{\pi}{6} \) & 2 & \( \frac{7\pi}{6} \) & -2 \\
            \( \frac{\pi}{4} \) & \( 2\sqrt{2} \approx 2.8 \) & \( \frac{5\pi}{4} \) & \( -2\sqrt{2} \approx -2.8 \) \\
            \( \frac{\pi}{3} \) & \( 2\sqrt{3} \approx 3.4 \) & \( \frac{4\pi}{3} \) & \( -2\sqrt{3} \approx -3.4 \) \\
            \( \frac{\pi}{2} \) & 4 & \( \frac{3\pi}{2} \) & -4 \\
            \( \frac{2\pi}{3} \) & \( 2\sqrt{3} \approx 3.4 \) & \( \frac{5\pi}{3} \) & \( -2\sqrt{3} \approx -3.4 \) \\
            \( \frac{3\pi}{4} \) & \( 2\sqrt{2} \approx 2.8 \) & \( \frac{7\pi}{4} \) & \( -2\sqrt{2} \approx -2.8 \) \\
            \( \frac{5\pi}{6} \) & 2 & \( \frac{11\pi}{6} \) & -2 \\
            \( 2\pi \) & 0 \\
            \bottomrule
        \end{tabular}
     \end{center}
     \bigbreak \noindent 
     Plotting these points gives the graph
     \pagebreak 
     \fig{1}{./figures/13.jpeg}
     \bigbreak \noindent 
     This is the graph of a circle. The equation \( r = 4\sin\theta \) can be converted into rectangular coordinates by first multiplying both sides by \( r \). This gives the equation \( r^2 = 4r\sin\theta \). Next use the facts that \( r^2 = x^2 + y^2 \) and \( y = r\sin\theta \). This gives \( x^2 + y^2 = 4y \). To put this equation into standard form, subtract \( 4y \) from both sides of the equation and complete the square:
     \[
         x^2 + (y^2 - 4y) = x^2 + (y^2 - 4y + 4) = x^2 + (y - 2)^2 = 0 + 4.
     \]
     This is the equation of a circle with radius 2 and center \( (0, 2) \) in the rectangular coordinate system.
     \bigbreak \noindent 
     The equation of the circle can be transformed into rectangular coordinates using the coordinate transformation formulas
     \begin{align*}
         &r^{2} = x^{2} + y^{2}
         &x = r\cos{\theta} \\
         &y = r\sin{\theta}
     .\end{align*}
     \pagebreak 
     \subsection{Types of polar curves}
     \bigbreak \noindent 
     \begin{tabularx}{\textwidth}{|X|X|X|}
        \hline
        Name & Equation & Example \\
        \hline
        Line passing through the pole with slope $\tan{K}$ & $\theta =K$ & \fig{.5}{./figures/14.png}\\
        \hline
        Circle & $r=a\cos{\theta} + b\sin{\theta} $ & \fig{.5}{./figures/16.png}\\
        \hline
        Spiral& $r=a+b\theta  $&\fig{.5}{./figures/15.png} \\
        \hline
     \end{tabularx}

     \pagebreak \bigbreak \noindent 
      \begin{tabularx}{\textwidth}{|X|X|X|}
        \hline
        Name & Equation & Example \\
        \hline
        Cardioid & \fig{.5}{./figures/20.png} & \fig{.5}{./figures/17.png}\\
        \hline
        Limacon & \fig{.5}{./figures/21.png} & \fig{.5}{./figures/18.png}\\
        \hline
        Rose & \fig{.5}{./figures/21.png}&\fig{.5}{./figures/19.png} \\
        \hline
     \end{tabularx}
     \bigbreak \noindent 
     For the rose, If the coefficient of $\theta$
     is even, the graph has twice as many petals as the coefficient. If the coefficient of $\theta$
     is odd, then the number of petals equals the coefficient.

     \pagebreak 
     \unsect{1.4 Area and Arc Length in Polar Coordinates}
     \bigbreak \noindent 
     \subsection{Areas of Regions Bounded by Polar Curves}
     \bigbreak \noindent 
     \begin{thrmm}
         Suppose \( f \) is continuous and nonnegative on the interval \( \alpha \leq \theta \leq \beta \) with \( 0 < \beta - \alpha \leq 2\pi \). The area of the region bounded by the graph of \( r = f(\theta) \) between the radial lines \( \theta = \alpha \) and \( \theta = \beta \) is
        \[ A = \frac{1}{2} \int_{\alpha}^{\beta} [f(\theta)]^2 \, d\theta = \frac{1}{2} \int_{\alpha}^{\beta} r^2 \, d\theta. \]
     \end{thrmm}

     \bigbreak \noindent 
     \subsection{Arc Length in Polar Curves}
     \bigbreak \noindent 
     Here we derive a formula for the arc length of a curve defined in polar coordinates.
     \bigbreak \noindent 
     In rectangular coordinates, the arc length of a parameterized curve \((x(t), y(t))\) for \(a \leq t \leq b\) is given by
     \[ L = \int_{a}^{b} \sqrt{\left(\frac{dx}{dt}\right)^2 + \left(\frac{dy}{dt}\right)^2} \, dt. \]
     In polar coordinates, we define the curve by the equation \( r = f(\theta) \), where \( \alpha \leq \theta \leq \beta \). In order to adapt the arc length formula for a polar curve, we use the equations
     \[ x = r \cos \theta = f(\theta) \cos \theta \quad \text{and} \quad y = r \sin \theta = f(\theta) \sin \theta, \]
     and we replace the parameter \( t \) by \( \theta \). Then
     \[ \frac{dx}{d\theta} = f'(\theta) \cos \theta - f(\theta) \sin \theta \quad \text{and} \quad \frac{dy}{d\theta} = f'(\theta) \sin \theta + f(\theta) \cos \theta.  \]
     We replace \( dt \) by \( d\theta \), and the lower and upper limits of integration are \( \alpha \) and \( \beta \), respectively. Then the arc length formula becomes
     \begin{align*}
      &L = \int_{a}^{b} \sqrt{\left(\frac{dx}{dt}\right)^2 + \left(\frac{dy}{dt}\right)^2} \, dt  \\
      &= \int_{\alpha}^{\beta} \sqrt{\left(\frac{dx}{d\theta}\right)^2 + \left(\frac{dy}{d\theta}\right)^2} \, d\theta \\
      &= \int_{\alpha}^{\beta} \sqrt{\left(f'(\theta) \cos \theta - f(\theta) \sin \theta\right)^2 + \left(f'(\theta) \sin \theta + f(\theta) \cos \theta\right)^2} \, d\theta \\
      &= \int_{\alpha}^{\beta} \sqrt{(f'(\theta))^2(\cos^2 \theta + \sin^2 \theta) + (f(\theta))^2(\cos^2 \theta + \sin^2 \theta)} \, d\theta \\
      &= \int_{\alpha}^{\beta} \sqrt{(f'(\theta))^2 + (f(\theta))^2} \, d\theta \\
      &= \int_{\alpha}^{\beta} \sqrt{r^2 + \left(\frac{dr}{d\theta}\right)^2} \, d\theta. 
  .\end{align*}
     This gives us the following theorem.

     \bigbreak \noindent 
     \begin{thrmm}[Arc Length of a Curve Defined by a Polar Function]
         Let \( f \) be a function whose derivative is continuous on an interval \( \alpha \leq \theta \leq \beta \). The length of the graph of \( r = f(\theta) \) from \( \theta = \alpha \) to \( \theta = \beta \) is
         \begin{align*}
              &L = \int_{\alpha}^{\beta} \sqrt{[f(\theta)]^2 + [f'(\theta)]^2} \, d\theta \\
              &= \int_{\alpha}^{\beta} \sqrt{r^2 + \left(\frac{dr}{d\theta}\right)^2} \, d\theta. 
         .\end{align*}
     \end{thrmm}

     \bigbreak \noindent 
     \begin{exm}[Finding the Arc Length of a Polar Curve]
         Find the arc length of the polar curve 
         \begin{align*}
             r = 2 + 2\cos{\theta }
         .\end{align*}
         
     \end{exm}
     \bigbreak \noindent 
     \textcolor{red}{\textit{Solution.}}
     When \( \theta = 0 \), \( r = 2 + 2\cos(0) = 4 \). Furthermore, as \( \theta \) goes from \( 0 \) to \( 2\pi \), the cardioid is traced out exactly once. Therefore, these are the limits of integration. Using \( f(\theta) = 2 + 2\cos(\theta) \), \( \alpha = 0 \), and \( \beta = 2\pi \). Thus we have
     \begin{align*}
          &L = \int_{\alpha}^{\beta} \sqrt{[f(\theta)]^2 + [f'(\theta)]^2} \, d\theta  \\
          &= \int_{0}^{2\pi} \sqrt{[2 + 2\cos(\theta)]^2 + [-2\sin(\theta)]^2} \, d\theta  \\
          &= \int_{0}^{2\pi} \sqrt{4 + 8\cos(\theta) + 4\cos^2(\theta) + 4\sin^2(\theta)} \, d\theta  \\
          &= \int_{0}^{2\pi} \sqrt{4 + 8\cos(\theta) + 4(\cos^2(\theta) + \sin^2(\theta))} \, d\theta  \\
          &= \int_{0}^{2\pi} \sqrt{8 + 8\cos(\theta)} \, d\theta  \\
          &= 2 \int_{0}^{2\pi} \sqrt{2 + 2\cos(\theta)} \, d\theta.
     .\end{align*}
     Next, using the identity \( \cos(2\alpha) = 2\cos^2(\alpha) - 1 \), add 1 to both sides and multiply by 2. This gives \( 2 + 2\cos(2\alpha) = 4\cos^2(\alpha) \). Substituting \( \alpha = \frac{\theta}{2} \) gives \( 2 + 2\cos(\theta) = 4\cos^2\left(\frac{\theta}{2}\right) \), so the integral becomes
     \begin{align*}
          &L = 2 \int_{0}^{2\pi} \sqrt{2 + 2\cos(\theta)} \, d\theta  \\
          &= 2 \int_{0}^{2\pi} \sqrt{4\cos^2\left(\frac{\theta}{2}\right)} \, d\theta  \\
          &= 2 \int_{0}^{2\pi} 2 \left| \cos\left(\frac{\theta}{2}\right) \right| \, d\theta. 
     .\end{align*}
     The absolute value is necessary because the cosine is negative for some values in its domain. To resolve this issue, change the limits from \( 0 \) to \( \pi \) and double the answer. This strategy works because cosine is positive between \( 0 \) and \( \frac{\pi}{2} \). Thus,
     \begin{align*}
          &L = 4 \int_{0}^{2\pi} \left| \cos\left(\frac{\theta}{2}\right) \right| \, d\theta  \\
          &= 8 \int_{0}^{\pi} \cos\left(\frac{\theta}{2}\right) \, d\theta  \\
          &= 8 \left[ 2\sin\left(\frac{\theta}{2}\right) \right]_{0}^{\pi}  \\
          &= 16. 
     .\end{align*}






     
     




   


    
    

\end{document}
