 \documentclass{report}
 
 \input{~/dev/latex/template/preamble.tex}
 \input{~/dev/latex/template/macros.tex}
 
 \title{\Huge{}}
 \author{\huge{Nathan Warner}}
 \date{\huge{}}
 \fancyhf{}
 \rhead{}
 \fancyhead[R]{\itshape Warner} % Left header: Section name
 \fancyhead[L]{\itshape\leftmark}  % Right header: Page number
 \cfoot{\thepage}
 \renewcommand{\headrulewidth}{0pt} % Optional: Removes the header line
 %\pagestyle{fancy}
 %\fancyhf{}
 %\lhead{Warner \thepage}
 %\rhead{}
 % \lhead{\leftmark}
 %\cfoot{\thepage}
 %\setborder
 % \usepackage[default]{sourcecodepro}
 % \usepackage[T1]{fontenc}
 
 % Change the title
 \hypersetup{
     pdftitle={}
 }

 \geometry{
  left=1.5in,
  right=1.5in,
  top=1in,
  bottom=1in
}
 
 \begin{document}
     % \maketitle
     %     \begin{titlepage}
     %    \begin{center}
     %        \vspace*{1cm}
     % 
     %        \textbf{}
     % 
     %        \vspace{0.5cm}
     %         
     %             
     %        \vspace{1.5cm}
     % 
     %        \textbf{Nathan Warner}
     % 
     %        \vfill
     %             
     %             
     %        \vspace{0.8cm}
     %      
     %        \includegraphics[width=0.4\textwidth]{~/niu/seal.png}
     %             
     %        Computer Science \\
     %        Northern Illinois University\\
     %        United States\\
     %        
     %             
     %    \end{center}
     % \end{titlepage}
     % \tableofcontents
    \pagebreak \bigbreak \noindent
    Nate Warner \ \quad \quad \quad \quad \quad \quad \quad \quad \quad \quad \quad \quad  MATH 232 \quad  \quad \quad \quad \quad \quad \quad \quad \quad \ \ \quad \quad Spring 2024
    \begin{center}
        \textbf{Homework/Worksheet 11 - Due: Saturday, May 2}
    \end{center}
    \bigbreak \noindent 
    \begin{mdframed}
        1. Evaluate $\int_C xy^4 \, ds$, where \(C\) is the right half of the circle \(x^2 + y^2 = 16\).
    \end{mdframed}
    \bigbreak \noindent 
    \begin{remark}
        Let \(f\) be a continuous function with a domain that includes the smooth curve \(C\) with parameterization \(\mathbf{r}(t)\), \(a \leq t \leq b\). Then
        \[
            \int_C f\,ds = \int_a^b f(\mathbf{r}(t)) \|\mathbf{r}'(t)\|\,dt.
        \]
    \end{remark}
    \bigbreak \noindent 
    First, we find $\mathbf{r}(t)$, if $C$ is the right half the circle with radius 4 centered at the origin, then $\mathbf{r}(t)$i is given by
    \begin{align*}
        r(t) = \left\langle 4\cos{\left(\theta \right)}, 4\sin{\left(\theta \right)} \right\rangle \quad \text{for } -\frac{\pi}{2} \leq \theta  \leq\frac{\pi}{2}
    .\end{align*}
    \bigbreak \noindent 
    This implies 
    \begin{align*}
        f(\mathbf{r}(t)) &= 4\cos{\left(\theta \right)}4^{4}\sin^{4}{\left(\theta \right)} \\
        \text{and } \norm{\mathbf{r}(t)} &= \sqrt{16\sin^{2}{\left(\theta \right)} + 16\cos^{2}{\left(\theta \right)}} = 4
    .\end{align*}
    Thus, we have the integral
    \begin{align*}
        \int_C fds &= \int_{-\frac{\pi}{2}}^{\frac{\pi}{2}}4^{6}\cos{\left(\theta    \right)}\sin{\left(\theta \right)}\, d\theta  \\
                   &=\frac{4^{6}}{5}\bigg[\sin^{5}{\left(\theta \right)}\bigg|_{-\frac{\pi}{2}}^{\frac{\pi}{2}} \\
                   &= \frac{4^{6} \cdot 2}{5}
    .\end{align*}

    

    \bigbreak \noindent 
    \begin{mdframed}
        2. Evaluate $\int_C \mathbf{F} \cdot d\mathbf{r}$, where \(\mathbf{F}(x, y) = \langle -1, 0 \rangle\), and \(C\) is the part of the graph \(y = \frac{1}{2}x^3 - x\) from \((2, 2)\) to \((-2, -2)\).
    \end{mdframed}
    \bigbreak \noindent 
    We can define $\mathbf{r}(t)$ is this case to be 
    \begin{align*}
        \mathbf{r}(t) &= \left\langle t, \frac{1}{2}t^{3} -t  \right\rangle \\
        \implies \mathbf{r}^{\prime}(t) &= \left\langle 1, \frac{3}{2}t^{2} - 1 \right\rangle
        \implies d\mathbf{r} = \left\langle 1, \frac{3}{2}t^{2}-1 \right\rangle \,dt
    .\end{align*}
    \bigbreak \noindent 
    With $t$ ranging from $2$ to -2. Thus, we have
    \begin{align*}
        \int_C \mathbf{F} \cdot d\mathbf{r} &= \int_{2}^{-2}\left\langle -1,0 \right\rangle \cdot \left\langle 1,\frac{3}{2}t^{2}-1 \right\rangle \,dt \\
        &=\int_{2}^{-2}  -1\, dt \\
        &=-1(-2-2) \\
        &=4
    .\end{align*}

    \bigbreak \noindent 
    \begin{mdframed}
        3. Evaluate the integral $\int_C (2x - y) \, dx + (x + 3y) \, dy$, where \(C\) lies along the x-axis from \(x = 0\) to \(x = 5\).
    \end{mdframed}
    \bigbreak \noindent 
    We define the parameterization of the curve $C$ to be
    \begin{align*}
        x &= 5t \quad y = 0 \quad \text{for } 0 \leq t \leq 1 \\
        \implies \mathbf{r}(t) &= \left\langle 5t, 0 \right\rangle
    .\end{align*}
    \bigbreak \noindent 
    We then find the differentials $dx$, $dy$. They are given by 
    \begin{align*}
        dx &= x^{\prime}(t)dt = 5dt\\
        dy &= y^{\prime}(t)dt = 0
    .\end{align*}
    \bigbreak \noindent 
    Thus, we have the integral
    \begin{align*}
        &\int_{0}^{1} (2(5t) - 0 )5dt + \int_{0}^{1} (5t +3(0))0 \, dt \\
        &=\int_{0}^{1} 50t \, dt \\
        &=25\bigg[t^{2}\bigg|_{0}1 \\
        &=25(1-0)  = 25
    .\end{align*}

    \bigbreak \noindent 
    \begin{mdframed}
        4. Determine whether the vector field is conservative and, if it is, find the potential function.
        \begin{enumerate}
            \item[(a)] \(\mathbf{F}(x, y) = \langle -y + e^x \sin x, (x + 2)e^x \cos y \rangle\)
            \item[(b)] \(\mathbf{F}(x, y) = \langle 2x \cos y - y \cos x, -x^2 \sin y - \sin x \rangle\)
            \item[(c)] \(\mathbf{F}(x, y) = \langle 2xye^{x^2y}, 6x^2 e^{x^2y} \rangle\)
        \end{enumerate}
    \end{mdframed}
    \bigbreak \noindent 
    \textbf{Problem 4a.} If we call $\mathbf{F} = \left\langle P(x,y), Q(x,y) \right\rangle $, then we can identify if the vector field is conservative if $P_{y} = Q_{x}$
    \begin{align*}
        P_{y} &= -1  \\
        Q_{x} &= xe^{x} + e^{x} + 2e^{x}\cos{\left(y\right)}
    .\end{align*}
    \bigbreak \noindent 
    Since $P_{y} \ne Q_{x}$, the given vector field is \textbf{not} conservative.
    \bigbreak \noindent 
    \textbf{Problem 4b.} Again, we check the partials
    \begin{align*}
        P_{y} &= -2x\sin{\left(y\right)} - \cos{\left(x\right)} \\
        Q_{x} &=-2x\sin{\left(y\right)} - \cos{\left(x\right)}
    .\end{align*}
    \bigbreak \noindent 
    Since $P_{y} = Q_{x}$, the given vector field is conservative. We the find the potential function $f$
    \begin{align*}
        g(x,y) &= \int P(x,y)dx = x^{2}\cos{\left(y\right)} + h(y) \\
        g_{y} &= -x^{2}\sin{\left((y)\right)} + h^{\prime}(y) \\
        g_{y} &= Q(x,y) \implies -x^{2}\sin{\left(y\right)} + h^{\prime}(y) = -x^{2}\sin{\left((y)\right)} -\sin{\left(x\right)} \\
              &\implies h^{\prime}(y) = -\sin{\left(x\right)} \\
              &\implies \int h^{\prime}(y)dy = \int -\sin{\left(x\right)}dy \\
              &\implies h(y) = -y\sin{\left(x\right)} + C \\
        \therefore f(x,y) &= -x^{2}\cos{\left(y\right)} -y\sin{\left(x\right)} + C
    .\end{align*}
    \bigbreak \noindent 
    \textbf{Problem 4c.}
    \begin{align*}
        P_{y} &= 2xe^{x^{2}} \\
        Q_{x} &= e^{y}(6x^{2}(2xe^{x^{2}}) + 12xe^{x^{2}})
    .\end{align*}
    \bigbreak \noindent 
    Since $P_{y} \ne Q_{x}$, the given vector field is not conservative

    \bigbreak \noindent 
    \begin{mdframed}
        5. Evaluate the integral $\int_C \nabla f \cdot d\mathbf{r}$, where \(f(x, y) = x^2y - x\) and \(C\) is any path in the plane from \((1, 2)\) to \((3, 2)\).
    \end{mdframed}
    \bigbreak \noindent 
    \begin{remark}
        Let \( C \) be a piecewise smooth curve with parameterization \(\mathbf{r}(t), a \leq t \leq b\).
        \bigbreak \noindent 
        Let \( f \) be a function of two or three variables with first-order partial derivatives that exist and are continuous on \( C \). Then,
        \[
            \int_C \nabla f \cdot d\mathbf{r} = f(\mathbf{r}(b)) - f(\mathbf{r}(a)).
        \]
    \end{remark}
    \bigbreak \noindent 
    Since $\mathbf{r}(b) = (3,2)$, and $\mathbf{r}(a) = (1,2)$, finding the parameterization is unnecessary. To evaluate the given integral we find $f(3,2) - f(1,2) $
    \begin{align*}
        f(3,2) - f(1,2) &= 3^{2}(2)-3 - (1^{2}(2)-1) \\
        &=14
    .\end{align*}

    \bigbreak \noindent 
    \begin{mdframed}
        6. Evaluate the following line integrals by applying Green's theorem:
        \begin{enumerate}
            \item[(a)] \(\int_C xy \, dx + (x + y) \, dy\), where \(C\) is the boundary of the region lying between the graphs of \(x^2 + y^2 = 1\) and \(x^2 + y^2 = 9\) oriented in the counterclockwise direction.
            \item[(b)] \(\int_C (1 - y^3) \, dx + (x^3 + e^{y^2}) \, dy\), where \(C\) is the circle \(x^2 + y^2 = 4\) oriented in the counterclockwise direction.
        \end{enumerate}
    \end{mdframed}
    \bigbreak \noindent 
    \begin{remark}
        Let \( D \) be an open, simply connected region with a boundary curve \( C \) that is a piecewise smooth, simple closed curve oriented counterclockwise. Let \( \mathbf{F} = \langle P, Q \rangle \) be a vector field with component functions that have continuous partial derivatives on \( D \). Then,
        \[
            \oint_C \mathbf{F} \cdot d\mathbf{r} = \oint_C P \, dx + Q \, dy = \iint_D \left( \frac{\partial Q}{\partial x} - \frac{\partial P}{\partial y} \right) \, dA.
        \]
        \bigbreak \noindent 
        \textbf{Note:} Greens thoerem can only be used for a two-dimensional vector field
    \end{remark}
    \bigbreak \noindent 
    \textbf{Problem 6a.}
    \bigbreak \noindent 
    We know that an integral of the form $\int_C \mathbf{F} \cdot \mathbf{T}ds$ can be written as $\int_C P(x,y)dx + Q(x,y)dy$, where $\mathbf{F} = \left\langle P(x,y), Q(x,y)\right\rangle $. By this fact, we identify the following
    \begin{align*}
        P(x,y) &= xy \\
        Q(x,y) &= x+y \\
        \implies Q_{x} &= 1 \\
        \implies P_{y} &= x
    .\end{align*}
    \bigbreak \noindent 
    Using greens theorem for circulation, we have
    \begin{align*}
        \iint_R Q_{x} - P_{y}dA
    .\end{align*}
    \bigbreak \noindent 
    We must next identify our region $D$, since we are dealing with two circles, we shall represent the region in polar form.
    \begin{align*}
        D = \{(r,\theta ):\ 1 \leq r \leq 3,\ 0 \leq \theta  \leq 2\pi\} 
    .\end{align*}
    \bigbreak \noindent 
    Thus, we have the integral
    \begin{align*}
        &\iint_D_{xy} (Q_{x} - P_{y})dA = \iint_D_{xy} (1-x)dA \\
        &=\iint_{D_{r\theta }} (1-r\cos{\left(\theta \right)})dA \\
        &=\int_{0}^{2\pi }\int_{1}^{3} (1-r\cos{\left(\theta \right)})r\,drd\theta \\
        &=\int_{0}^{2\pi }\frac{1}{2}r^{2}-\frac{1}{3}r^{3}\cos{\left(\theta \right)}\bigg|_{1}^{3}  \, d\theta  \\
        &=\int_{0}^{2\pi }\frac{9}{2}-9\cos{\left(\theta \right)} \, d\theta  \\
        &=\frac{9}{2}\theta -9\sin{\left(\theta \right)}\bigg|_{0}^{2\pi} \\
        &=9\pi
    .\end{align*}
    \bigbreak \noindent 
    \textbf{Problem 6b.}
    Again, we start by identifying $P(x,y)$ and $Q(x,y)$ from the given integral
    \begin{align*}
        P(x,y) &= 1-y^{3} \\
        Q(x,y) &= x^{3} + e^{y^{2}} \\
        \implies Q_{x} &= 3x^{2} \\
        \implies P_{y} &= -3y^{2}
    .\end{align*}
    \bigbreak \noindent 
    We identify the region D as the polar region 
    \begin{align*}
        D = \{(r,\theta ):\ r \leq 2\, 0 \leq \theta  \leq 2\pi\}
    .\end{align*}
    \bigbreak \noindent 
    Thus, we have the integral
    \begin{align*}
        &\iint_{D_{xy}} (Q_{x} - P_{y})dA = \iint_{D_{xy}}(3x^{2} + 3y^{2})dA \\
        &=\iint_{D_{r\theta}}3r^{2}dA_{r\theta } \\
        &=\int_{0}^{2\pi }\int_{0}^{2}  3r^{3}\, drd\theta   \\
        &=\int_{0}^{2\pi}  \frac{3}{4}\bigg[r^{4}\bigg|_{0}^{2}\, d\theta  \\
        &=\int_{0}^{2\pi}  12\, d\theta  \\
        &=12(2\pi - 0) = 24\pi
    .\end{align*}




 \end{document} % (:
