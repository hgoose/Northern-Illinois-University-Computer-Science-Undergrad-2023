 \documentclass{report}
 
 \input{~/dev/latex/template/preamble.tex}
 \input{~/dev/latex/template/macros.tex}
 
 \title{\Huge{}}
 \author{\huge{Nathan Warner}}
 \date{\huge{}}
 \fancyhf{}
 \rhead{}
 \fancyhead[R]{\itshape Warner} % Left header: Section name
 \fancyhead[L]{\itshape\leftmark}  % Right header: Page number
 \cfoot{\thepage}
 \renewcommand{\headrulewidth}{0pt} % Optional: Removes the header line
 %\pagestyle{fancy}
 %\fancyhf{}
 %\lhead{Warner \thepage}
 %\rhead{}
 % \lhead{\leftmark}
 %\cfoot{\thepage}
 %\setborder
 % \usepackage[default]{sourcecodepro}
 % \usepackage[T1]{fontenc}
 
 % Change the title
 \hypersetup{
     pdftitle={}
 }

 \geometry{
  left=1.5in,
  right=1.5in,
  top=1in,
  bottom=1in
}
 
 \begin{document}
     % \maketitle
     %     \begin{titlepage}
     %    \begin{center}
     %        \vspace*{1cm}
     % 
     %        \textbf{}
     % 
     %        \vspace{0.5cm}
     %         
     %             
     %        \vspace{1.5cm}
     % 
     %        \textbf{Nathan Warner}
     % 
     %        \vfill
     %             
     %             
     %        \vspace{0.8cm}
     %      
     %        \includegraphics[width=0.4\textwidth]{~/niu/seal.png}
     %             
     %        Computer Science \\
     %        Northern Illinois University\\
     %        United States\\
     %        
     %             
     %    \end{center}
     % \end{titlepage}
     % \tableofcontents
    \pagebreak \bigbreak \noindent
    Nate Warner \ \quad \quad \quad \quad \quad \quad \quad \quad \quad \quad \quad \quad  MATH 232 \quad  \quad \quad \quad \quad \quad \quad \quad \quad \ \ \quad \quad Spring 2024
    \begin{center}
        \textbf{Homework/Worksheet 7 - Due: Sunday, March 31}
    \end{center}
    \bigbreak \noindent 
    \begin{mdframed}
        1. Find the directional derivative of the function $f(x,y) = e^{x}\cos{\left(y\right)}$ at $P\left(0,\frac{\pi}{2}\right)$ in the direction of $\mathbf{u} = \left\langle 0,1 \right\rangle $
    \end{mdframed}
    \bigbreak \noindent 
    \begin{remark}
                Let $z=f(x,y)$ be a function of two variables $x$ and $y$, and assume that $f_x$ and $f_y$ exist and $f(x, y)$ is differentiable everywhere. Then the directional derivative of $f$ in the direction of $\mathbf{u}=\langle u_{x},u_{y}\rangle$ is given by
        \[
            D_{\mathbf{u}}f(x,y) = f_x(x,y)u_{x} + f_y(x,y)u_{y}
        \]
    \end{remark}
    \bigbreak \noindent 
    We start by finding the gradient vector $\nabla f(x,y) = \left\langle f_{x}, f_{y} \right\rangle$
    \begin{align*}
       \nabla f(x,y) = \left\langle e^{x}\cos{\left(y\right)}, -e^{x}\sin{\left(y\right)} \right\rangle 
    .\end{align*}
    We then compute the directional derivative $D_{u}f(x,y)$ as the dot product between the gradient and the unit vector
    \begin{align*}
        D_{u}f(x,y) &= \nabla f(x,y) \cdot \vec{\mathbf{u}} = e^{x}\cos{\left(y\right)}(0)  -e^{x}\sin{\left(y\right)}(1) \\
        &=-e^{x}\sin{\left(y\right)}
    .\end{align*}
    With this, we can compute $D_{u}f\left(0,\frac{\pi}{2}\right) $
    \begin{align*}
        D_{u}f\left(0,\frac{\pi}{2}\right) &= -e^{0}\sin{\left(\frac{\pi}{2}\right)} \\
        &=-1(1) = -1
    .\end{align*}

    \bigbreak \noindent 
    \begin{mdframed}
        2. Find the directional derivative of the function $f(x, y) = x^2 + 2y^2$ in the direction of $\mathbf{v} = \langle \cos \theta, \sin \theta \rangle$, where $\theta = \frac{\pi}{6}$.
    \end{mdframed}
    \bigbreak \noindent 
    Again, we start by finding the gradient vector $\nabla f(x,y)$
    \begin{align*}
        \nabla f(x,y) = \left\langle f_{x}, f_{y} \right\rangle = \left\langle 2x, 4y \right\rangle
    .\end{align*}
    \bigbreak \noindent 
    We then compute the dot product against the given unit vector $\vec{\mathbf{v}} = \left\langle \cos{\left(\frac{\pi}{6}\right)}, \sin{\left(\frac{\pi}{6}\right)} \right\rangle$
    \begin{align*}
        \nabla f(x,y) \cdot \vec{\mathbf{v}} &= 2x\cos{\left(\frac{\pi}{6}\right)} + 4y\sin{\left(\frac{\pi}{6}\right)} \\
        &=2x\left(\frac{\sqrt{3}}{2}\right) + 4y\left(\frac{1}{2}\right) \\
        &=\sqrt{3}x + 2y
    .\end{align*}
    \bigbreak \noindent 
    Thus, the directional derivative $D_{u}f(x,y)$, with $f(x,y) = x^{2} + 2y^{2}$ and $\vec{\mathbf{v}} = \cos{\left(\frac{\pi}{6}\right)}, \sin{\left(\frac{\pi}{6}\right)}$ is given by 
    \begin{align*}
        D_{v} f(x,y) = \sqrt{3}x + 2y
    .\end{align*}

    \pagebreak 
    \begin{mdframed}
        3. Find the directional derivative of the function $f(x, y) = \ln(5x + 4y)$ at $P(3, 9)$ in the direction of $\mathbf{v} = \langle 6, 8 \rangle$.
    \end{mdframed}
    \bigbreak \noindent 
    Since the given vector $\vec{\mathbf{v}}$ is of $\norm{\vec{\mathbf{v}}} \neq 1$, we must first divide by the norm, this gives us a unit vector in the direction of $\vec{\mathbf{v}}$ with norm one. We get $\hat{\mathbf{v}} = \frac{1}{10}\left\langle 6,8 \right\rangle $. We now find the gradient vector $\nabla f(x,y)$
    \begin{align*}
        \nabla f(x,y) = \left\langle \frac{5}{5x+4y} , \frac{4}{5x+4y} \right\rangle
    .\end{align*}
    \bigbreak \noindent 
    From here we compute the directional derivative $D_{\hat{\mathbf{v}}} f(x,y)$
    \begin{align*}
        D_{\hat{\mathbf{v}}}f(x,y) &= \nabla f(x,y) \cdot \hat{\mathbf{v}} \\
                                   &=\frac{1}{10}\left[\frac{5(6)}{5x+4y} + \frac{4(8)}{5x+4y}\right]
    .\end{align*}
    Thus, we have
    \begin{align*}
        D_{\hat{\mathbf{v}}}(3,9) &= \frac{1}{10}\left[\frac{5(6) + 4(8)}{5(3) + 4(9)}\right] \\
        &=\frac{31}{255}
    .\end{align*}
    
    \bigbreak \noindent 
    \begin{mdframed}
        4. Find the gradient vector of $f(x, y) = xe^{y} - \ln(x)$ at $P(-3, 0)$.
    \end{mdframed}
    \bigbreak \noindent 
    The gradient of $f(x,y)$ at $P(-3,0)$ is given by
    \begin{align*}
        \nabla f(x,y) &= \left\langle e^{y} - \frac{1}{x}, \, xe^{y} \right\rangle \\
        \nabla f(-3,0) &= \left\langle e^{0} - \frac{1}{-3}, \, -3e^{0} \right\rangle \\
                       &= \left\langle \frac{4}{3}, -3 \right\rangle
    .\end{align*}

    \bigbreak \noindent 
    \begin{mdframed}
        5. Find the maximum rate of change of $f(x, y) = \cos(3x + 2y)$ at $\left(\frac{\pi}{6}, -\frac{\pi}{8}\right)$ and the direction in which it occurs.
    \end{mdframed}
    \bigbreak \noindent 
    \begin{remark}
         If $\nabla f(x_0,y_0) \neq 0$, then $D_{\mathbf{u}}f(x_0,y_0)$ is maximized when $\mathbf{u}$ points in the same direction as $\nabla f(x_0,y_0)$. The maximum value of $D_{\mathbf{u}}f(x_0,y_0)$ is $\|\nabla f(x_0,y_0)\|$.
    \end{remark}
    \bigbreak \noindent 
    Thus, we start by computing the gradient vector $\nabla f(x,y)$ and evaluating it at the point $\left(\frac{\pi}{6}, -\frac{\pi}{8}\right) $
    \begin{align*}
        \nabla f(x,y) &= \left\langle -3\sin{\left(3x+2y\right)}, -2\sin{\left(3x+2y\right)} \right\rangle \\
        \nabla f\left(\frac{\pi}{6}, -\frac{\pi}{8}\right) &= \left\langle -3\sin{\left(\frac{\pi}{4}\right)}, -2\sin{\left(\frac{\pi}{4}\right)}\right\rangle \\
                                                           &= \left\langle \frac{-3\sqrt{2}}{2}, -\sqrt{2} \right\rangle
    .\end{align*}
    \bigbreak \noindent 
    From this, we see that the maximum rate of change is at 
    \begin{align*}
        \norm{\nabla f\left(\frac{\pi}{6}, -\frac{\pi}{8}\right)} &= \sqrt{\left(\frac{-3\sqrt{2}}{2}\right)^{2} + \left(-\sqrt{2}\right)^{2}} \\
        &=\frac{\sqrt{26}}{2} \approx 2.5495
    .\end{align*}
    \bigbreak \noindent 
    Which occurs in the direction of $\nabla f\left(\frac{\pi}{6}, -\frac{\pi}{8}\right) = \left\langle -\frac{3\sqrt{2}}{2}, -\sqrt{2} \right\rangle$

    \bigbreak \noindent 
    \begin{mdframed}
        \noindent For the functions below, use the second derivative test to identify any critical points and determine whether each critical point is a maximum, minimum, saddle point, or none of these.
        \begin{enumerate}
            \item[(a)] $f(x, y) = x^2 - 6x + y^2 + 4y - 8$
            \item[(b)] $f(x, y) = y^2 + xy + 3y + 2x + 3$
        \end{enumerate}       
    \end{mdframed}
    \bigbreak \noindent 
    \begin{remark}
        Let $z = f(x, y)$ be a function of two variables that is defined on an open set containing the point $(x_0, y_0)$. The point $(x_0, y_0)$ is called a critical point of a function of two variables $f$ if one of the two following conditions holds:
        \begin{enumerate}
            \item $f_x(x_0, y_0) = f_y(x_0, y_0) = 0$
            \item Either $f_x(x_0, y_0)$ or $f_y(x_0, y_0)$ does not exist.
        \end{enumerate}
        \bigbreak \noindent 
        Let $z = f(x, y)$ be a function of two variables for which the first- and second-order partial derivatives are continuous on some disk containing the point $(x_0, y_0)$. Suppose $f_x(x_0, y_0) = 0$ and $f_y(x_0, y_0) = 0$. Define the quantity
        \[ D = f_{xx}(x_0, y_0)f_{yy}(x_0, y_0) - (f_{xy}(x_0, y_0))^2. \]
        \begin{enumerate}
            \item[I.] If $D > 0$ and $f_{xx}(x_0, y_0) > 0$, then $f$ has a local minimum at $(x_0, y_0)$.
            \item[II.] If $D > 0$ and $f_{xx}(x_0, y_0) < 0$, then $f$ has a local maximum at $(x_0, y_0)$.
            \item[III.] If $D < 0$, then $f$ has a saddle point at $(x_0, y_0)$.
            \item[IV.] If $D = 0$, then the test is inconclusive.
        \end{enumerate}
    \end{remark}
    \bigbreak \noindent 
    Thus, we find $\nabla f(x,y)$ and solve $\nabla f(x,y) = \vec{\mathbf{0}}$ to find the critical points
    \begin{align*}
        \nabla f(x,y) &= \left\langle 2x-6, 2y+4 \right\rangle \\
        \left\langle 2x-6, 2y+4 \right\rangle &= \vec{\mathbf{0}} \\ 
        \implies x&=3 \\
        \implies y&=-2
    .\end{align*}
    We see we have a critical point at $C_{1}(3,-2)$, calculating the discriminant we find
    \begin{align*}
        D &= 2(2) - 0^{2} = 4 > 0 
    .\end{align*}
    Since the discriminant is positive, we know this point must either be a local min or a local max. Since $f_{xx} >0$, we know that the critical point $(3,-2)$ is a local min. 
    \pagebreak \bigbreak \noindent 
    For part b, we again start by finding the gradient vector
    \begin{align*}
        \nabla f(x,y) = \left\langle y+2, 2y+x+3 \right\rangle
    .\end{align*}
    This gives the following system of linear equations
       \begin{equation}
            \begin{cases}
                 y+2&=0  \\
                 2y+x+3&=0 
            \end{cases}
        \end{equation}
        \bigbreak \noindent 
        Solving the first equation for $y$ gives $y =-2$, plugging this result into the second equation yields the $x$ value
        \begin{align*}
            2(-2) + x +3 &= 0 \\
            \implies x &=1
        .\end{align*}
        \bigbreak \noindent 
        Thus, we have a critical point at $(1,-2)$. To further examine this point, we again calculate the discriminant
        \begin{align*}
            D = 0(2) - (1)^{2} < 0 
        .\end{align*}
        \bigbreak \noindent 
        Since the discriminant is negative, we conclude that the critical point (1,-2) is a saddle point by the second derivative test.

    \bigbreak \noindent 
    \begin{mdframed}
       7. Use the method of Lagrange multipliers to find the maximum and minimum values of the function $f(x, y) = xy$ subject to the given constraint $4x^2 + 8y^2 = 16$. 
    \end{mdframed}
    \bigbreak \noindent 
    \begin{remark}
        Let $f$ and $g$ be functions of two variables with continuous partial derivatives at every point of some open set containing the smooth curve $g(x,y)=0$. Suppose that $f$, when restricted to points on the curve $g(x,y)=0$, has a local extremum at the point $(x_0,y_0)$ and that $\nabla g(x_0,y_0) \neq 0$. Then there is a number $\lambda$ called a Lagrange multiplier, for which
        \[
            \nabla f(x_0,y_0) = \lambda \nabla g(x_0,y_0).
        \]
        \bigbreak \noindent 
        To find the absolute extrema, we must solve the system given by
           \begin{equation}
                \begin{cases}
                     \nabla f(x, y)&=\lambda \nabla g(x,y) \\ 
                     g(x,y)&=0
                \end{cases}
            \end{equation}
            \bigbreak \noindent 
            After finding all points that satisfy this system, we plug into $f(x,y)$. The smallest value will be the absolute minimum, and the largest will be the absolute maxmimum
    \end{remark}
    \bigbreak \noindent 
    With this, we begin by finding $\nabla f(x,y) $ and $\nabla g(x,y) $
    \begin{align*}
        \nabla f(x,y) &= \left\langle y,x \right\rangle \\
        \nabla g(x,y) &= \left\langle 8x, 16y \right\rangle
    .\end{align*}
    With this, we have the system
       \begin{equation}
            \begin{cases}
                \left\langle y,x \right\rangle  =\lambda \left\langle 8x,16y \right\rangle \\
                4x^{2} + 8y^{2} -16 =0
            \end{cases}
        \end{equation}
        \bigbreak \noindent 
        Which implies the system
           \begin{equation}
                \begin{cases}
                    y = \lambda 8x \\
                    x = \lambda 16y \\
                    4x^{2} + 8y^{2} -16 =0
                \end{cases}
            \end{equation}
            \bigbreak \noindent 
            Solving one and two for lambda gives
            \begin{align*}
                \lambda = \frac{y}{8x} &= \frac{x}{16y} \\
                \implies 8x^{2} &= 16y^{2} \\
                \implies x &= \pm\sqrt{2}y
            .\end{align*}
            Plugging the positive version of $x$ into the third equation gives 
            \begin{align*}
                4(\sqrt{2}y)^{2} +8y^{2} - 16 &= 0 \\
                \implies 8y^{2} + 8y^{2} - 16 &= 0 \\
                \implies 16y^{2} - 16 &= 0 \\
                \implies y&=\pm1
            .\end{align*}
            \bigbreak \noindent 
            Thus, we $x=+\sqrt{2}y $, $y=\pm1$, which gives the solutions, $(-\sqrt{2}, -1),\, (\sqrt{2},1) $
            \bigbreak \noindent 
            Similarly, we use the negative version of $x$ to get the remaining solution.
            \begin{align*}
                4(-\sqrt{2}y)^{2} + 8y^{2} - 16 &= 0 \\
                \implies y= \pm1
            .\end{align*}
            So, when $x=-\sqrt{2}y$, we find that $y$ is also $\pm1$, this gives the remaining solutions $(\sqrt{2}y,-1),\, (-\sqrt{2}y, 1)$. To find the absolute min and max, we need to evaluate $f$ at these points
            \begin{align*}
                f(-\sqrt{2},-1) &= \sqrt{2}\\
                f(\sqrt{2},1) &= \sqrt{2}\\
                f(\sqrt{2},-1) &= -\sqrt{2}\\
                f(-\sqrt{2},1) &= -\sqrt{2}
            .\end{align*}
            \bigbreak \noindent 
            Thus, we see that we have an absolute maximum at $f(-\sqrt{2}, -1) = f(\sqrt{2}, 1) = \sqrt{2}$, and an absolute minimum at $(f(\sqrt{2},-1)) = f(-\sqrt{2}, 1) = -\sqrt{2} $
    


    


 \end{document} 
