 \documentclass{report}
 
 \input{~/dev/latex/template/preamble.tex}
 \input{~/dev/latex/template/macros.tex}
 
 \title{\Huge{}}
 \author{\huge{Nathan Warner}}
 \date{\huge{}}
 \fancyhf{}
 \rhead{}
 \fancyhead[R]{\itshape Warner} % Left header: Section name
 \fancyhead[L]{\itshape\leftmark}  % Right header: Page number
 \cfoot{\thepage}
 \renewcommand{\headrulewidth}{0pt} % Optional: Removes the header line
 %\pagestyle{fancy}
 %\fancyhf{}
 %\lhead{Warner \thepage}
 %\rhead{}
 % \lhead{\leftmark}
 %\cfoot{\thepage}
 %\setborder
 % \usepackage[default]{sourcecodepro}
 % \usepackage[T1]{fontenc}
 
 % Change the title
 \hypersetup{
     pdftitle={}
 }

 \geometry{
  left=1.5in,
  right=1.5in,
  top=1in,
  bottom=1in
}
 
 \begin{document}
     % \maketitle
     %     \begin{titlepage}
     %    \begin{center}
     %        \vspace*{1cm}
     % 
     %        \textbf{}
     % 
     %        \vspace{0.5cm}
     %         
     %             
     %        \vspace{1.5cm}
     % 
     %        \textbf{Nathan Warner}
     % 
     %        \vfill
     %             
     %             
     %        \vspace{0.8cm}
     %      
     %        \includegraphics[width=0.4\textwidth]{~/niu/seal.png}
     %             
     %        Computer Science \\
     %        Northern Illinois University\\
     %        United States\\
     %        
     %             
     %    \end{center}
     % \end{titlepage}
     % \tableofcontents
    \pagebreak \bigbreak \noindent
    Nate Warner \ \quad \quad \quad \quad \quad \quad \quad \quad \quad \quad \quad \quad  MATH 232 \quad  \quad \quad \quad \quad \quad \quad \quad \quad \ \ \quad \quad Spring 2024
    \begin{center}
        \textbf{Homework/Worksheet 9 - Due: Saturday, April 13}
    \end{center}
    \bigbreak \noindent 
    \begin{mdframed}
        1. For the double integrals below, convert the integrals to polar coordinates and evaluate them.
        \begin{enumerate}[label=(\alph*)]
            \item $\int_{0}^{3}\int_{0}^{\sqrt{9-y^{2}}}  (x^{2} + y^{2})\,  dxdy$
            \item $\int_{0}^{1}\int_{0}^{\sqrt{1-x^{2}}}  (x+y)\,  dydx$
        \end{enumerate}
    \end{mdframed}
    \bigbreak \noindent 
    \textbf{Problem 1a.} We identify the given region as
    \begin{align*}
        D = \{(x,y):\ 0 \leq y \leq 3,\ 0 \leq x \leq \sqrt{9-y^2}\}
    .\end{align*}
    To transform this region to one of the form $S = \{(r,\theta ):\ \alpha \leq \theta \leq \beta,\ h_{1}(\theta ) \leq r \leq h_{2}(\theta) \}$, we first notice that the region $D$ describes a quarter circle in the first quadrant of the $xy$-plane. Thus, our new region $S$ is given by
    \begin{align*}
        S = \{(r,\theta ):\ 0 \leq \theta  \leq \frac{\pi}{2},\ 0 \leq r \leq 3\}
    .\end{align*}
    \bigbreak \noindent 
    So we have the integral
    \begin{align*}
        &\int_{0}^{\frac{\pi}{2} }\int_{0}^{3} (r^{2})r \, drd\theta  \\
        &=\int_{0}^{\frac{\pi}{2}}  \frac{1}{4}\bigg[r^{4}\bigg]_{0}^{3}\, d\theta  \\
        &=\int_{0}^{\frac{\pi}{2}}  \frac{81}{4}\, d\theta  \\
        &=\frac{81}{4}(\frac{\pi}{2}-0) \\
        &=\frac{81\pi}{8}
    .\end{align*}
    \bigbreak \noindent 
    \textbf{Problem 1b.} Again, we identify the given region and then transform it to polar
    \begin{align*}
        D &= \{(x,y):\ 0 \leq x \leq 1,\ 0 \leq y \leq \sqrt{1-x^{2}}\} \\
        \implies  S &= \{(r,\theta ):\ 0 \leq r \leq 1,\ 0 \leq \theta  \leq \frac{\pi}{2}\}
    .\end{align*}
    \bigbreak \noindent 
    Thus, we have the integral
    \begin{align*}
        &\int_{0}^{\frac{\pi}{2}}\int_{0}^{1} (r\cos{\left(\theta \right)}+r\sin{\left(\theta \right)})\, rdrd\theta  \\
        &=\int_{0}^{\frac{\pi}{2}}\int_{0}^{1}r^{2}\cos{\left(\theta \right)} +r^{2}\sin{\left(\theta \right)} \,drd\theta  \\
        &=\int_{0}^{\frac{\pi}{2}}\bigg[\cos{\left(\theta \right)}\int_{0}^{1}r^{2}dr+\sin{\left(\theta \right)}\int_{0}^{1} r^{2} \, dr\bigg]  \, d\theta  \\
        &=\int_{0}^{\frac{\pi}{2}}  \frac{1}{3}\cos{\left(\theta \right)} + \frac{1}{3}\sin{\left(\theta \right)}\, d\theta  \\
        &=\frac{1}{3}\bigg[\sin{\left(\theta \right)}-\cos{\left(\theta \right)}\bigg]_{0}^{\frac{\pi}{2}} \\
        &=\frac{1}{3}(1-0-(0-1)) = \frac{2}{3}
    .\end{align*}


    \bigbreak \noindent 
    \begin{mdframed}
        2. Find the volume of the solid bounded by the paraboloid $z=2-9x^{2} -9y^{2}$ and the plane $z = 1$.
    \end{mdframed}
    \bigbreak \noindent 
    We first define this region as 
    \begin{align*}
        E = \{(x,y,z): (x,y) \in D,\ 1 \leq z \leq 2-9x^{2}-9y^{2}\}
    .\end{align*}
    \bigbreak \noindent 
    We must next deduce $D$, which is the projection of $E$ onto the $xy$-plane. To accomplish this, we can set $z=1$, We set $z=1$ instead of $z=0$ because the floor of our region is defined by the plane $z=1$.
    \begin{align*}
        &2-9x^{2} -9y^{2} = 1 \\
        \implies &x^{2} + y^{2} = \frac{1}{9} 
    .\end{align*}
    Thus, the projection of $E$ onto the $xy$-plane is a disk centered at the origin of radius $\frac{1}{3}$, with this information we determine that $D=\{(x,y):\ -\frac{1}{3} \leq x \leq \frac{1}{3},\ - \sqrt{\frac{1}{9}-x^{2}} \leq y \leq \sqrt{\frac{1}{9}-x^{2}}\} $
    \bigbreak \noindent 
    Thus, region $E$ is given by
    \begin{align*}
        E = \{(x,y,z):\ -\frac{1}{3} \leq x \leq \frac{1}{3},\ - \sqrt{\frac{1}{9}-x^{2}} \leq y \leq \sqrt{\frac{1}{9}-x^{2}},\ 1 \leq z \leq 9-x^{2}-y^{2}\}
    .\end{align*}
    \bigbreak \noindent 
    We then move this region to cylindrical coordinates, of the form $G = \{(r,\theta ,z):\ g_{1}(\theta ) \leq r \leq g_{2}(\theta ) \alpha \leq \theta  \leq \beta,\ u_{1}(r,\theta ) \leq z \leq u_{2}(r,\theta )\} $, thus region $G$ in cylindrical coordinates is given by 
    \begin{align*}
        G = \{(r,\theta,z):\ 0 \leq r \leq \frac{1}{3},\ 0 \leq \theta  \leq 2\pi,\ 1 \leq z \leq 2-9r^{2}\}
    .\end{align*}
    \bigbreak \noindent 
    Then the volume is given by the triple integral over the cylindrical region
    \begin{align*}
        &\int_{0}^{2\pi }\int_{0}^{\frac{1}{3}}\int_{1}^{2-9r^{2}}\, rdzdrd\theta \\
        &=\int_{0}^{2\pi }r\int_{0}^{\frac{1}{3}}2-9r^{2} - 1  \, drd\theta  \\
        &=\int_{0}^{2\pi}\int_{0}^{\frac{1}{3}}r-9r^{3}  \, drd\theta  \\
        &=\int_{0}^{2\pi} \frac{1}{2}r^{2}-\frac{9}{4}r^{4} \, d\theta  \\
        &=\int_{0}^{2\pi}  \frac{1}{2}\left(\frac{1}{3}\right)^{2}-\frac{9}{4}\left(\frac{1}{3}\right)^{4}\, d\theta \\
        &=\int_{0}^{2\pi}  \frac{1}{36}\, d\theta  \\
        &=\frac{1}{36}(2\pi-0) = \frac{\pi}{18}
    .\end{align*}

    \pagebreak \bigbreak \noindent 
    \begin{mdframed}
        3. The solid \(E\) is bounded by \(y^2 + z^2 = 9\), \(z = 0\), $x=0$, and \(x = 5\) (see picture in problem 5.4.211). Evaluate the triple integral \(\iiint_E z \, dV\) by integrating first with respect to \(z\), then \(y\), and then \(x\).
    \end{mdframed}
    \bigbreak \noindent 
    We see that the solid $E$ describes a cylinder with radius $3$ centered along the $x$-axis. Thus, we have the integral
    \begin{align*}
        &\int_{0}^{5}\int_{-3}^{3}\int_{0}^{3}\, dzdydx   \\
        &=\int_{0}^{5}\int_{-3}^{3}3\, dydx \\
        &=\int_{0}^{5} 3y\bigg|_{-3}^{3} \, dx \\
        &=\int_{0}^{5}9+9  \, dx \\
        &=\int_{0}^{5}18  \, dx \\
        &=18(5) \\
        &=90
    .\end{align*}

    \bigbreak \noindent 
    \begin{mdframed}
        4.  The solid \(E\) is bounded by \(y = \sqrt{x}\), \(x = 4\), \(y = 0\), and \(z = 1\) (see picture in problem 5.4.212). Evaluate the triple integral \(\iiint_E z \, dV\) by integrating first with respect to \(x\), then \(y\), and then \(z\).
    \end{mdframed}
    \bigbreak \noindent 
    We describe the given region as
    \begin{align*}
        E &= \{(x,y,z):\ (x,y) \in D,\ 0 \leq z \leq 1\} \\
        \text{With}\ D_{\text{t2}} &= \{0 \leq x \leq 4,\ 0 \leq y \leq \sqrt{x}\}
    .\end{align*}
    \bigbreak \noindent 
    However, since the problem states to integrate first with respect to $x$, then $y$, our integral would be
    \begin{align*}
        \int_{0}^{1}\int_{0}^{\sqrt{x}}\int_{0}^{4} z \, dxdydz
    .\end{align*}
    This is a problem because the upper bound for the middle intgeral involve an $x$ variable. Thus, we must change the region $D$ from type 1 to type 2.
    \begin{align*}
        D_{t1} = \{(x,y):\ y^{2} \leq x \leq 4,\ 0 \leq y \leq 2\}
    .\end{align*}
    The intgeral then becomes
    \begin{align*}
        &\int_{0}^{1}\int_{0}^{2}\int_{y^{2}}^{4}  z\, dxdydz \\
        &=\int_{0}^{1}z\int_{0}^{2} 4- y^{2}\, dydz =\int_{0}^{1} z \bigg[4y-\frac{1}{3}y^{3}\bigg]_{0}^{2}\, dz \\
        &=\int_{0}^{1} z\left(8-\frac{8}{3}\right) \, dz  = \int_{0}^{1} \frac{16}{3}z \, dz\\ 
        &= \frac{8}{3}(1-0) = \frac{8}{3}
    .\end{align*}


    \pagebreak 
    \begin{mdframed}
        5. Find the volume of the solid \(E\) bounded by \(z = 10 - 2x - y\) and situated in the first octant (see picture in problem 5.4.231).
    \end{mdframed}
    \bigbreak \noindent 
    Since we know we are only focused on the first octant, we have $x=0$, $y=0$, and $z=0$. This deduction gives the lower bounds for each integral. Furthermore, we can find the upper bounds for the $x$ and $y$ integral by first setting $z=0$, solving for $y$, then setting $y=0$ and solving for x
    \begin{align*}
        0&=10 - 2x - y  \\
        \implies y &= 10-2x \\
        \implies x&=5
    .\end{align*}
    \bigbreak \noindent 
    Thus we have the region 
    \begin{align*}
        E = \{(x,y,z):\ 0 \leq x \leq 5,\ 0 \leq y \leq 10-2x,\ 0 \leq z \leq 10-2x-y\}
    .\end{align*}
    And the integral is given by
    \begin{align*}
        &\int_{0}^{5}\int_{0}^{10-2x}\int_{0}^{10-2x-y}  \, dzdydx \\
        &=\int_{0}^{5}\int_{0}^{10-2x}10-2x-y  \, dydx \\
        &=\int_{0}^{5} 10y-2xy-\frac{1}{2}y^{2}\bigg|_{0}^{10-2x} \, dx \\
        &=\int_{0}^{5} 10(10-2x)-2x(10-2x)-\frac{1}{2}(10-2x)^{2} \, dx \\
        &=\int_{0}^{5} 2x^{2}-20x+50 \, dx\\
        &=\frac{2}{3}x^{3}-10x^{2}+50x\bigg|^{5}_{0} \\
        &=\frac{250}{3}
    .\end{align*}

    \bigbreak \noindent 
    \begin{mdframed}
        6. Let \(f(x, y, z) = x^2 + y^2\) and \(E = \{(x, y, z) : 0 \leq x^2 + y^2 \leq 4, y \geq 0, 0 \leq z \leq 3 - x\}\). Convert the integral \(\iiint_E f(x, y, z) \, dV\) into cylindrical coordinates and evaluate it.
    \end{mdframed}
    \bigbreak \noindent 
    Since the region $D$ on the $xy$-plane is a circle, we can easily change the given region to cylindrical coordinates. We note that because $y \geq 0$, we only consider the top half of the circle, this gives the new region as
    \begin{align*}
        E = \{(r,\theta ,z):\ 0 \leq r \leq 2,\ 0 \leq \theta \leq \pi,\ 0 \leq z \leq 3-r\cos{\left(\theta \right)} \}
    .\end{align*}
    Hence, the volume is given by the integral
    \begin{align*}
        &\int_{0}^{\pi }\int_{0}^{2}\int_{0}^{3-r\cos{\left(\theta \right)}}  r^{2}r\, dzdrd\theta  \\
        &=\int_{0}^{\pi }\int_{0}^{2} r^{3}(3-r\cos{\left(\theta \right)}) \, drd\theta  = \int_{0}^{\pi}\int_{0}^{2} 3r^{3}-r^{4}\cos{\left(\theta \right)} \, drd\theta  \\
        &=\int_{0}^{\pi } \frac{3}{4}r^{4} - \frac{1}{5}r^{5}\cos{\left(\theta \right)}\bigg|_{0}^{2} \, d\theta = \int_{0}^{\pi}  12 - \frac{32}{5}\cos{\left(\theta \right)}\, d\theta  \\
        &= 12(\pi - 0) - \frac{32}{5}(\sin{\left(\pi\right)} - \sin{\left(0\right)} = 12\pi
    .\end{align*}
    

    \bigbreak \noindent 
    \begin{mdframed}
        7. Convert the integral
        \[
            \int_0^1 \int_{-\sqrt{1-y^2}}^{\sqrt{1-y^2}} \int_{\sqrt{x^2 + y^2}}^{x^2 + y^2} xz \, dz \, dx \, dy
        \]
        into an integral in cylindrical coordinates.
    \end{mdframed}
    \bigbreak \noindent 
    We express the given integral as the region 
    \begin{align*}
    E=\{(x,y,z):\ 0 \leq y \leq 1,\ -\sqrt{1-y^{2}} \leq x \leq\sqrt{1-y^{2}},\ x^{2} + y^{2} \leq z \leq \sqrt{x^{2}+y^{2}}\}
    .\end{align*}
    \bigbreak \noindent 
    We notice that the bounds for $x$ and $y$ define a circle on the $xy$-plane centered at the origin with radius 1. Using this and converting the bounds of $z$ and the integrand to polar form, we get the cylindrical region
    \begin{align*}
        E = \{(r,\theta ,z):\ 0 \leq \theta \leq 2\pi,\ 0 \leq r \leq 1,\ r^{2} \leq z \leq r \}
    .\end{align*}
    Thus, the integral becomes
    \begin{align*}
        &\int_{0 }^{2\pi }\int_{0}^{1}\int_{r^{2}}^{r}r^{2}\cos{\left(\theta \right)}z  \, dzdrd\theta  \\
        &=\int_{0}^{2\pi }\int_{0}^{1} \frac{1}{2}r^{2}\cos{\left(\theta \right)}\big[z^{2}\big]_{r^{2}}^{r},\ drd\theta   \\
        &=\int_{0}^{2\pi}\int_{0}^{1} \frac{1}{2}r^{5}\cos{\left(\theta \right)} - \frac{1}{2}r^{6}\cos{\left(\theta \right)} \, drd\theta  \\
        &=\int_{0}^{2\pi} \frac{1}{10}r^{5}\cos{\left(\theta \right)} -\frac{1}{14}r^{7}\cos{\left(\theta \right)}\bigg|^{1}_{0}\, d\theta  \\
        &= \frac{1}{35}\int_{0}^{2\pi}  \cos{\left(\theta \right)}\, d\theta  \\
        &=-\frac{1}{35}\big[\sin{\left(\theta \right)}\big]_{0}^{2\pi} \\
        &= 0
    .\end{align*}






 \end{document} % (:
