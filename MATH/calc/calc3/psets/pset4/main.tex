 \documentclass{report}
 
 \input{~/dev/latex/template/preamble.tex}
 \input{~/dev/latex/template/macros.tex}
 
 \title{\Huge{}}
 \author{\huge{Nathan Warner}}
 \date{\huge{}}
 \fancyhf{}
 \rhead{}
 \fancyhead[R]{\itshape Warner} % Left header: Section name
 \fancyhead[L]{\itshape\leftmark}  % Right header: Page number
 \cfoot{\thepage}
 \renewcommand{\headrulewidth}{0pt} % Optional: Removes the header line
 %\pagestyle{fancy}
 %\fancyhf{}
 %\lhead{Warner \thepage}
 %\rhead{}
 % \lhead{\leftmark}
 %\cfoot{\thepage}
 %\setborder
 % \usepackage[default]{sourcecodepro}
 % \usepackage[T1]{fontenc}
 
 % Change the title
 \hypersetup{
     pdftitle={}
 }

 \geometry{
  left=1.5in,
  right=1.5in,
  top=1in,
  bottom=1in
}
 
 \begin{document}
     % \maketitle
     %     \begin{titlepage}
     %    \begin{center}
     %        \vspace*{1cm}
     % 
     %        \textbf{}
     % 
     %        \vspace{0.5cm}
     %         
     %             
     %        \vspace{1.5cm}
     % 
     %        \textbf{Nathan Warner}
     % 
     %        \vfill
     %             
     %             
     %        \vspace{0.8cm}
     %      
     %        \includegraphics[width=0.4\textwidth]{~/niu/seal.png}
     %             
     %        Computer Science \\
     %        Northern Illinois University\\
     %        United States\\
     %        
     %             
     %    \end{center}
     % \end{titlepage}
     % \tableofcontents
    \pagebreak \bigbreak \noindent
    Nate Warner \ \quad \quad \quad \quad \quad \quad \quad \quad \quad \quad \quad \quad  MATH 232 \quad  \quad \quad \quad \quad \quad \quad \quad \quad \ \ \quad \quad Spring 2024
    \begin{center}
        \textbf{Homework/Worksheet 4 - Due: Friday, February 23}
    \end{center}
    \bigbreak \noindent 
    \begin{mdframed}
        1. Find the domain of the vector function $\mathbf{r}(t) = t^2\ \mathbf{i} + \sqrt{t - 3}\ \mathbf{j} + \frac{3}{2t+1}\ \mathbf{k}$.
    \end{mdframed}
    \bigbreak \noindent 
    First, we define
    \begin{align*}
        x(t) &= t^{2} \\
        y(t) &= \sqrt{t-3} \\
        z(t) &= \frac{3}{2t+1}
    .\end{align*}
    Next, next define the domains of each function. The intersection of the domains will be the domain of $\vec{\mathbf{r}}(t)$
    \begin{align*}
        d(x(t)) &=  \mathbb{R} \\
        d(y(t)) &= t-3 \geq 0 \implies t \geq 3 \\
        d(z(t)) &=2t+1 \neq 0 \implies t \neq -\frac{1}{2}
    .\end{align*}
    \cc{From this, we have the overall domain}
    \begin{align*}
        d(\vec{\mathbf{r}}(t)):\ \{t \mid t \geq 3\}
    .\end{align*}

    \bigbreak \noindent 
    \begin{mdframed}
        2. Evaluate the limit $\lim_{t \to 1} \left( \frac{t^2 - 1}{t - 1} \mathbf{i} + \sqrt{t + 3} \mathbf{j} + \frac{\sin(\pi t)}{\ln t} \mathbf{k} \right)$. 
    \end{mdframed}
    \bigbreak \noindent 
    \begin{remark}
        Let \(f\), \(g\), and \(h\) be functions of \(t\). 
        The limit of the vector-valued function  \(\mathbf{r}(t) = f(t)\mathbf{i} + g(t)\mathbf{j} + h(t)\mathbf{k}\) as \(t\) approaches \(a\) is given by
        \[
            \lim_{t \to a} \mathbf{r}(t) = \left[ \lim_{t \to a} f(t) \right]\mathbf{i} + \left[ \lim_{t \to a} g(t) \right]\mathbf{j} + \left[ \lim_{t \to a} h(t) \right]\mathbf{k},
        \]
        provided the limits  \(\lim_{t \to a} f(t)\), \(\lim_{t \to a} g(t)\), and \(\lim_{t \to a} h(t)\) exist.
    \end{remark}
    \bigbreak \noindent 
    Thus, if we define 
    \begin{align*}
        f(t) &= \frac{t^{2}-1}{t-1} \\
        g(t) &= \sqrt{t+3} \\
        h(t) &= \frac{\sin{\left(\pi t\right)}}{\ln{(t)}}
    .\end{align*}
    \bigbreak \noindent 
    We can find the limit of each function as $t \to 1$
    \begin{align*}
        \lim\limits_{t \to 1}{f(t)} &= \lim\limits_{t \to 1}{\frac{t^{2}-1}{t-1}} = \lim\limits_{t \to 1}{\frac{(t-1)(t+1)}{t-1}} = \lim\limits_{t \to 1}{t+1} = 2 \\
        \lim\limits_{t \to 1}{g(t)} &= \lim\limits_{t \to 1}{\sqrt{t+3}} = \sqrt{4} = 2\\
        \lim\limits_{t \to 1}{h(t)} &= \lim\limits_{t \to 1}{\frac{\sin{\left(\pi t\right)}}{\ln{(t)}}} \Heq \lim\limits_{t \to 1}{\frac{\pi\cos{\left(\pi t\right)}}{\frac{1}{t}}} = \lim\limits_{t \to 1}{\pi t \cos{\left(\pi t\right)}} = -\pi
    .\end{align*}
    \cc{Thus, we have the limit for the vector function}
    \begin{align*}
        \lim\limits_{t \to 1}{\vec{\mathbf{r}}(t)} = 2\ \hat{\mathbf{i}} + 2\ \hat{\mathbf{j}} -\pi\ \hat{\mathbf{k}}
    .\end{align*}
    

    \bigbreak \noindent 
    \begin{mdframed}
        3. Find the derivative of the vector function $\mathbf{r}(t) = t e^t \mathbf{i} + t \ln t \mathbf{j} + \sin(3t) \mathbf{k}$.
    \end{mdframed}
    \bigbreak \noindent 
    \begin{remark}
        Let $f$, $g$, and $h$ be differentiable functions of $t$. If $\mathbf{r}(t) = f(t)\mathbf{i} + g(t)\mathbf{j} + h(t)\mathbf{k}$, then $\mathbf{r}'(t) = f'(t)\mathbf{i} + g'(t)\mathbf{j} + h'(t)\mathbf{k}$.
    \end{remark}
    \bigbreak \noindent 
    Thus, we define
    \begin{align*}
        f(t) &= te^{t} \\
        g(t) &= t\ln{(t)} \\
        h(t) &= \sin{\left(3t\right)}
    .\end{align*}
    Next, we find the derivative with respect to $t$ of each function
    \begin{align*}
        f^{\prime}(t) &= te^{t} + e^{t} \\
        g^{\prime}(t) &= t \cdot \frac{1}{t} + \ln{(t)} = 1 + \ln{(t)} \\
        h^{\prime}(t) &=3\cos{\left(3t\right)}
    .\end{align*}
    \bigbreak \noindent 
    \cc{Therefore we have}
        \begin{align*}
            \vec{\mathbf{r}}^{\prime}(t) = (te^{t} + e^{t})\ \hat{\mathbf{i}} + (1+\ln{(t)})\ \hat{\mathbf{j}} + (3\cos{\left(3t\right)})\ \hat{\mathbf{k}}
        .\end{align*}

    \bigbreak \noindent 
    \begin{mdframed}
        4. For the vector-valued functions below, find a tangent parametric equations for the tangent line to the curve at the given point.
        \begin{enumerate}[label=(\alph*)]
            \item $\mathbf{r}(t) = \cos 2t \mathbf{i} + 2 \sin t \mathbf{j} + t^2 \mathbf{k}$; $t = \frac{\pi}{2}$
            \item $\mathbf{r}(t) = \ln(t + 1) \mathbf{i} + t \cos 2t \mathbf{j} + 2^{t} \mathbf{k}$; $t = 0$
        \end{enumerate}
    \end{mdframed}
    \bigbreak \noindent 
    \begin{remark}
       To find the parametric equation of a line, we need a direction vector and a point. To find the parametric equations for a line tangent to a curve at some point $t$, we use $\vec{\mathbf{r}}^{\prime}(t)$ as the direction vector. This is due to the fact that for any position on our curve given by $\vec{\mathbf{r}}(t)$, the derivative $\vec{\mathbf{r}}^{\prime}(t)$ will be a vector tangent to that point. To find a point suitable for the parametric equations, we use $\vec{\mathbf{r}}(t)$.
    \end{remark}
    \bigbreak \noindent 
    \textbffProblem 4a.} First, we find our point at $\vec{\mathbf{r}}(t)$. In this case, $\vec{\mathbf{r}}\left(\frac{\pi}{2}\right)$
    \begin{align*}
        \vec{\mathbf{r}}\left(\frac{\pi}{2}\right) &= \cos{\left(\pi\right)}\ \hat{\mathbf{i}} + 2\sin{\left(\frac{\pi}{2}\right)}\ \hat{\mathbf{j}} + \left(\frac{\pi}{2}\right)^{2}\ \hat{\mathbf{k}} \\
        &= -1\ \hat{\mathbf{i}} + 2\ \hat{\mathbf{j}} + \left(\frac{\pi^{2}}{4}\right)\ \hat{\mathbf{k}}
    .\end{align*}
    Thus, we have the point $P\left(-1,2,\frac{\pi^{2}}{4}\right) $. Next, we find the direction vector by finding $\vec{\mathbf{r}}^{\prime}\left(\frac{\pi}{2}\right)$
    \begin{align*}
        \vec{\mathbf{r}}^{\prime}(t) &= -2\sin{\left(2t\right)}\ \hat{\mathbf{i}}  + 2\cos{\left(t\right)}\ \hat{\mathbf{j}} + 2t\ \hat{\mathbf{k}} \\
        \vec{\mathbf{r}}\left(\frac{\pi}{2}\right) &= -2\sin{\left(\pi\right)}\ \hat{\mathbf{i}} + 2\cos{\left(\frac{\pi}{2}\right)}\ \hat{\mathbf{j}} + \pi\ \hat{\mathbf{k}} \\
                                                   &=\pi \hat{\mathbf{k}}
    .\end{align*}
    Thus, we have the direction vector $\langle0,0,\pi\rangle $
    \bigbreak \noindent 
    \cc{The tangent line to the curve at the point $t = \frac{\pi}{2} $ is given by the parametric equations}
    \begin{align*}
        x(\tau) &= -1 \\
        y(\tau) &= 2 \\
        z(\tau) &= \frac{\pi^{2}}{4} + \pi \tau
    .\end{align*}

    \bigbreak \noindent 
    \textbf{Problem 4b.} Again, we find $\vec{\mathbf{r}}(t)$ and $\vec{\mathbf{r}}^{\prime}(t)$
    \begin{align*}
        \vec{\mathbf{r}}\left(0\right) &= 0\ \hat{\mathbf{i}} + 0\ \hat{\mathbf{j}} + \hat{\mathbf{k}} \\
        \vec{\mathbf{r}}^{\prime}(t) &= \frac{1}{t+1}\ \hat{\mathbf{i}} -2t\sin{\left(2t\right)} + \cos{\left(2t\right)}\ \hat{\mathbf{j}} + 2^{t}\ln{2}\ \hat{\mathbf{k}} \\
        \vec{\mathbf{r}}^{\prime}(0) &= 1\ \hat{\mathbf{i}} + \ln{2}\ \hat{\mathbf{k}}
    .\end{align*}
    \cc{Thus, we have the point (0,0,1) and the direction vector $\left\langle 1,0,\ln{2} \right\rangle $, which gives the parametric equations}
    \begin{align*}
        x(\tau) &= t \\
        y(\tau) &=0 \\
        z(\tau) &=1 +\ln{(2)}\tau
    .\end{align*}


    \bigbreak \noindent 
    \begin{mdframed}
        5. Suppose that the acceleration function, initial velocity, and initial position of a particle are $\mathbf{a}(t) = -5 \cos t \mathbf{i} - 5 \sin t \mathbf{j}$, $\mathbf{v}(0) = 9 \mathbf{i} + 2 \mathbf{j}$, and $\mathbf{r}(0) = 5\mathbf{i}$, respectively. Find $\mathbf{v}(t)$ and $\mathbf{r}(t)$.
    \end{mdframed}
    \bigbreak \noindent 
    To find the velocity vector, we integrate the acceleration vector
    \begin{align*}
        \vec{\mathbf{v}}(t) = \int \vec{\mathbf{a}}(t) &= -5\int \cos{\left(t\right)}\ dt\ \hat{\mathbf{i}} -5 \int \sin{\left(t\right)}\ dt \ \hat{\mathbf{j}} \\
                                 &= -5\sin{\left(t\right)} + C_{1}\ \hat{\mathbf{i}} + 5\cos{\left(t\right)}\ + C_{2}\ \hat{\mathbf{j}}
    .\end{align*}
    We use the fact that $\vec{\mathbf{v}}(0) = 9\ \hat{\mathbf{i}} + 2\ \hat{\mathbf{j}}$ to find the constants of integration
    \begin{align*}
        -5\sin{\left(0\right)} + C_{1}\ \hat{\mathbf{i}} &= 9\ \hat{\mathbf{i}} \implies C_{1} = 9 \\
        5\cos{\left(0\right)} + C_{2}\ \hat{\mathbf{j}} &=2\ \hat{\mathbf{j}} \implies C_{2} = -3
    .\end{align*}
    Thus, the velocity vector is given by 
    \begin{align*}
        \vec{\mathbf{v}}(t) = -5\sin{\left(t\right)} + 9\ \hat{\mathbf{i}} + 5\cos{\left(t\right)} -3\ \hat{\mathbf{j}}
    .\end{align*}
    We then integrate the velocity vector to find the position vector $\vec{\mathbf{r}}(t)$
    \begin{align*}
        \vec{\mathbf{r}}(t) &= \int \vec{\mathbf{v(t)}}\ dt = -5\int \sin{\left(t\right)}\ dt\ \hat{\mathbf{i}} + 5\int \cos{\left(t\right)}\ dt\ \hat{\mathbf{j}} \\
        &=5\cos{\left(t\right)} + C_{1}\ \hat{\mathbf{i}} + 5\sin{\left(t\right)}\ + C_{2}\ \hat{\mathbf{j}}
    .\end{align*}
    We then use the fact that $r(0) = 5\ \hat{\mathbf{i}}$ to find the constants of integration
    \begin{align*}
        5\cos{\left(0\right)} + C_{1}\ \hat{\mathbf{i}} &= 5\ \hat{\mathbf{i}} \implies C_{1} = 0 \\
        5\sin{\left(0\right)} + C_{2}\ \hat{\mathbf{j}} &= 0\ \hat{\mathbf{j}} \implies C_{2} = 0
    .\end{align*}
    \bigbreak \noindent 
    \cc{Thus, we have}
    \begin{align*}
        \vec{\mathbf{v}}(t) &= -5\sin{\left(t\right)} + 9\ \hat{\mathbf{i}} + 5\cos{\left(t\right)} -3\ \hat{\mathbf{j}}\\
        \vec{\mathbf{r}}(t) &= 5\cos{\left(t\right)}\ \hat{\mathbf{i}} + 5\sin{\left(t\right)}\ \hat{\mathbf{j}}
    .\end{align*}
    



     
 \end{document} 
