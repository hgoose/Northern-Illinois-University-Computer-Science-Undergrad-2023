 \documentclass{report}
 
 \input{~/dev/latex/template/preamble.tex}
 \input{~/dev/latex/template/macros.tex}
 
 \title{\Huge{}}
 \author{\huge{Nathan Warner}}
 \date{\huge{}}
 \fancyhf{}
 \rhead{}
 \fancyhead[R]{\itshape Warner} % Left header: Section name
 \fancyhead[L]{\itshape\leftmark}  % Right header: Page number
 \cfoot{\thepage}
 \renewcommand{\headrulewidth}{0pt} % Optional: Removes the header line
 %\pagestyle{fancy}
 %\fancyhf{}
 %\lhead{Warner \thepage}
 %\rhead{}
 % \lhead{\leftmark}
 %\cfoot{\thepage}
 %\setborder
 % \usepackage[default]{sourcecodepro}
 % \usepackage[T1]{fontenc}
 
 % Change the title
 \hypersetup{
     pdftitle={}
 }

 \geometry{
  left=1.5in,
  right=1.5in,
  top=1in,
  bottom=1in
}
 
 \begin{document}
     % \maketitle
     %     \begin{titlepage}
     %    \begin{center}
     %        \vspace*{1cm}
     % 
     %        \textbf{}
     % 
     %        \vspace{0.5cm}
     %         
     %             
     %        \vspace{1.5cm}
     % 
     %        \textbf{Nathan Warner}
     % 
     %        \vfill
     %             
     %             
     %        \vspace{0.8cm}
     %      
     %        \includegraphics[width=0.4\textwidth]{~/niu/seal.png}
     %             
     %        Computer Science \\
     %        Northern Illinois University\\
     %        United States\\
     %        
     %             
     %    \end{center}
     % \end{titlepage}
     % \tableofcontents
    \pagebreak \bigbreak \noindent
    Nate Warner \ \quad \quad \quad \quad \quad \quad \quad \quad \quad \quad \quad \quad  MATH 232 \quad  \quad \quad \quad \quad \quad \quad \quad \quad \ \ \quad \quad Spring 2024
    \begin{center}
        \textbf{Homework/Worksheet 2 - Due: Wednesday, January 31}
    \end{center}
    \bigbreak \noindent 
    \begin{mdframed}
        1. Convert the rectangular equation $y^{2} = 4x$ to polar form and sketch its graph.
    \end{mdframed}
    \bigbreak \noindent 
    \begin{remark}
        Given a point $P$ with cartesian coordinates $(x,y)$, and polar coordinates $(r, \theta)$ the following conversion formulas are true.
       \begin{align*}
           &x = r\cos{\theta} \\
           &y= r\sin{\theta} \\
           &x^{2} +y^{2} = r^{2} \\
           &\tan{\theta} = \frac{y}{x}
       .\end{align*} 
    \end{remark}
    \bigbreak \noindent 
    With the formulas mentioned above, we can convert $y^{2} = 4x$ to polar form.
    \begin{align*}
        &y^{2} = 4x \\
        &y= 4\cdot \frac{x}{y} \\
        &r\sin{\theta } = 4\cot{\theta} \\
        &r = 4\cot{\theta}\csc{\theta}
    .\end{align*}
    To graph this equation, we first make a table of points
    \bigbreak \noindent 
    \begin{tabularx}{\textwidth}{|X|X|}
        \hline
        $\theta$ & r \\
        \hline
        0 & undefined \\
        $\frac{\pi}{2}$ & 0 \\
        $\pi$ & undefined\\
        $\frac{3\pi}{2}$ & 0 \\
        $2\pi$ & undefined\\
        \hline
    \end{tabularx}
    \bigbreak \noindent 
    Additionally, we know $\csc{\theta }$ has period $2\pi$, and $\cot{\theta }$ has period $\pi$. Thus, $4\cot{\theta }\csc{\theta }$ will have period $\pi$. Since both functions have vertical asymptotes at $x=k\pi, k\in\mathbb{R}$, we know the graph of $4\cot{\theta }\csc{\theta }$ will also have these asymptotes. Moreover, we can find the zeros by setting the equation equal to zero and solving for $\theta  $
    \begin{align*}
        &4\cot{\theta }\csc{\theta } = 0 \\
        &\frac{\cos{\theta }}{\sin^{2}{\theta }} = 0 \\
        &\cos{\theta } = 0 \\
        &\theta = \cos^{-1}{0} \\
        &\theta  = \frac{\pi}{2} + k\pi,\ k\in\mathbb{R}
    .\end{align*}
    Now we need to determine the behavior of the graph as $\theta$ approaches $0$ and $\pi$
    \begin{align*}
        &\lim\limits_{\theta  \to 0}{4 \frac{\cos{\theta }}{\sin^{2}{0}}} = \infty \\
        &\lim\limits_{\theta  \to \pi}{4 \frac{\cos{\theta }}{\sin^{2}{0}}} = -\infty
    .\end{align*}
    From this information, the polar curve can be sketched
    \pagebreak 
    \begin{figure}[ht]
        \centering
        \incfig{p2}
        \label{fig:p2}
    \end{figure}

    \pagebreak \bigbreak \noindent
    \begin{mdframed}
        2. Convert the polar equation $r=6\cos{\theta}$ to rectangular form and sketch its graph.
    \end{mdframed}
    Converting to rectangular form we get.
    \begin{align*}
        &r^{2} = 6r\cos{\theta } \\
        &x^{2} +y^{2} = 6x \\
        &x^{2} -6x + y^{2} = 0 \\
        &(x-3)^{2} + y^{2} = 3^{2}
    .\end{align*}
    We see that this is the equation of a circle, with center (3,0) and radius $r=3$. Thus we have
    \bigbreak \noindent 
\begin{figure}[ht]
    \centering
    \incfig{circle2}
    \label{fig:circle2}
\end{figure}

    \pagebreak \bigbreak \noindent 
    \begin{mdframed}
        3. Sketch the curve $r=3-2\cos{\theta}$ by first sketching the graph of $r$ as a function of $\theta$ in Cartesian coordinates.
    \end{mdframed}
    \bigbreak \noindent 
    Sketching this curve in the rectangular system,
    \bigbreak \noindent 
    \begin{figure}[ht]
        \centering
        \incfig{maneaan}
        \label{fig:maneaan}
    \end{figure}

    \bigbreak \noindent 
    From this we can sketch the polar curve (Very Rough sketch)
    \bigbreak \noindent 
\begin{figure}[ht]
    \centering
    \incfig{pkill2}
    \label{fig:pkill2}
\end{figure}




     
 \end{document} 
