 \documentclass{report}
 
 \input{~/dev/latex/template/preamble.tex}
 \input{~/dev/latex/template/macros.tex}
 
 \title{\Huge{}}
 \author{\huge{Nathan Warner}}
 \date{\huge{}}
 \fancyhf{}
 \rhead{}
 \fancyhead[R]{\itshape Warner} % Left header: Section name
 \fancyhead[L]{\itshape\leftmark}  % Right header: Page number
 \cfoot{\thepage}
 \renewcommand{\headrulewidth}{0pt} % Optional: Removes the header line
 %\pagestyle{fancy}
 %\fancyhf{}
 %\lhead{Warner \thepage}
 %\rhead{}
 % \lhead{\leftmark}
 %\cfoot{\thepage}
 %\setborder
 % \usepackage[default]{sourcecodepro}
 % \usepackage[T1]{fontenc}
 
 % Change the title
 \hypersetup{
     pdftitle={}
 }

 \geometry{
  left=1.5in,
  right=1.5in,
  top=1in,
  bottom=1in
}
 
 \begin{document}
     % \maketitle
     %     \begin{titlepage}
     %    \begin{center}
     %        \vspace*{1cm}
     % 
     %        \textbf{}
     % 
     %        \vspace{0.5cm}
     %         
     %             
     %        \vspace{1.5cm}
     % 
     %        \textbf{Nathan Warner}
     % 
     %        \vfill
     %             
     %             
     %        \vspace{0.8cm}
     %      
     %        \includegraphics[width=0.4\textwidth]{~/niu/seal.png}
     %             
     %        Computer Science \\
     %        Northern Illinois University\\
     %        United States\\
     %        
     %             
     %    \end{center}
     % \end{titlepage}
     % \tableofcontents
    \pagebreak \bigbreak \noindent
    Nate Warner \ \quad \quad \quad \quad \quad \quad \quad \quad \quad \quad \quad \quad  MATH 232 \quad  \quad \quad \quad \quad \quad \quad \quad \quad \ \ \quad \quad Spring 2024
    \begin{center}
        \textbf{Homework/Worksheet 2 - Due: Wednesday, January 31}
    \end{center}
    \bigbreak \noindent 
    \begin{mdframed}
        1. Convert the rectangular equation $y^{2} = 4x$ to polar form and sketch its graph.
    \end{mdframed}
    \bigbreak \noindent 
    \begin{remark}
        Given a point $P$ with cartesian coordinates $(x,y)$, and polar coordinates $(r, \theta)$ the following conversion formulas are true.
       \begin{align*}
           &x = r\cos{\theta} \\
           &y= r\sin{\theta} \\
           &x^{2} +y^{2} = r^{2} \\
           &\tan{\theta} = \frac{y}{x}
       .\end{align*} 
    \end{remark}
    \bigbreak \noindent 
    With the formulas mentioned above, we can convert $y^{2} = 4x$ to polar form.
    \begin{align*}
        &y^{2} = 4x \\
        &y= 4\cdot \frac{x}{y} \\
        &r\sin{\theta } = 4\cot{\theta} \\
        &r = 4\cot{\theta}\csc{\theta}
    .\end{align*}
    To graph this equation, we first make a table of points
    \bigbreak \noindent 
    \begin{tabularx}{\textwidth}{|X|X|}
        \hline
        $\theta$ & r \\
        \hline
        0 & undefined \\
        $\frac{\pi}{2}$ & 0 \\
        $\pi$ & undefined\\
        $\frac{3\pi}{2}$ & 0 \\
        $2\pi$ & undefined\\
        \hline
    \end{tabularx}
    \bigbreak \noindent 
    Additionally, we know $\csc{\theta }$ has period $2\pi$, and $\cot{\theta }$ has period $\pi$. Thus, $4\cot{\theta }\csc{\theta }$ will have period $\pi$. Since both functions have vertical asymptotes at $x=k\pi, k\in\mathbb{R}$, we know the graph of $4\cot{\theta }\csc{\theta }$ will also have these asymptotes. Moreover, we can find the zeros by setting the equation equal to zero and solving for $\theta  $
    \begin{align*}
        &4\cot{\theta }\csc{\theta } = 0 \\
        &\frac{\cos{\theta }}{\sin^{2}{\theta }} = 0 \\
        &\cos{\theta } = 0 \\
        &\theta = \cos^{-1}{0} \\
        &\theta  = \frac{\pi}{2} + k\pi,\ k\in\mathbb{R}
    .\end{align*}
    Now we need to determine the behavior of the graph as $\theta$ approaches $0$ and $\pi$
    \begin{align*}
        &\lim\limits_{\theta  \to 0}{4 \frac{\cos{\theta }}{\sin^{2}{0}}} = \infty \\
        &\lim\limits_{\theta  \to \pi}{4 \frac{\cos{\theta }}{\sin^{2}{0}}} = -\infty
    .\end{align*}
    From this information, the polar curve can be sketched
    \pagebreak 
    \begin{figure}[ht]
        \centering
        \incfig{p2}
        \label{fig:p2}
    \end{figure}

    \pagebreak \bigbreak \noindent
    \begin{mdframed}
        2. Convert the polar equation $r=6\cos{\theta}$ to rectangular form and sketch its graph.
    \end{mdframed}
    Converting to rectangular form we get.
    \begin{align*}
        &r^{2} = 6r\cos{\theta } \\
        &x^{2} +y^{2} = 6x \\
        &x^{2} -6x + y^{2} = 0 \\
        &(x-3)^{2} + y^{2} = 3^{2}
    .\end{align*}
    We see that this is the equation of a circle, with center (3,0) and radius $r=3$. Thus we have
    \bigbreak \noindent 
\begin{figure}[ht]
    \centering
    \incfig{circle2}
    \label{fig:circle2}
\end{figure}

    \pagebreak \bigbreak \noindent 
    \begin{mdframed}
        3. Sketch the curve $r=3-2\cos{\theta}$ by first sketching the graph of $r$ as a function of $\theta$ in Cartesian coordinates.
    \end{mdframed}
    \bigbreak \noindent 
    Sketching this curve in the rectangular system,
    \bigbreak \noindent 
    \begin{figure}[ht]
        \centering
        \incfig{maneaan}
        \label{fig:maneaan}
    \end{figure}

    \bigbreak \noindent 
    From this we can sketch the polar curve (Very Rough sketch)
    \bigbreak \noindent 
\begin{figure}[ht]
    \centering
    \incfig{pkill2}
    \label{fig:pkill2}
\end{figure}

    \pagebreak \bigbreak \noindent 
    \begin{mdframed}
        4. Determine a definite integral that represents the area of the region in the first quadrant enclosed by $r = 2-\cos{\theta}$
    \end{mdframed}
    \bigbreak \noindent 
    \begin{remark}
       Suppose $f$ is continuous and nonnegative on the interval $\alpha \leq \theta \leq \beta$ with $0 < \alpha - \beta \leq 2\pi $. The area of the region bounded by the graph of $r=f(\theta)$ between the radial lines $\theta =\alpha$ and $\theta =\beta$ is given by
       \begin{align*}
           A = \frac{1}{2}\int_{\alpha}^{\beta}\ r^{2}\ d\theta 
       .\end{align*}
    \end{remark}
    
    \bigbreak \noindent 
    The first quadrant of this polar curve lies between the radial lines $\theta = 0$ and $\theta = \frac{\pi}{2}$. Thus, we have the integral
    \begin{align*}
        \frac{1}{2}\int_{0}^{\frac{\pi}{2}}\ (2-\cos{\theta })^{2}\ d\theta 
    .\end{align*}

    \bigbreak \noindent 
    \begin{mdframed}
        5. Sketch the curve $r=4\cos{3\theta }$ and find the area enclosed by one petal.
    \end{mdframed}
    \bigbreak \noindent 
    We first sketch the graph in the rectangular system, and then we tranlate over to polar.
    \bigbreak \noindent 
    \begin{minipage}[]{0.47\textwidth}
        \incfig{grapher}
    \end{minipage}
    \begin{minipage}[]{0.47\textwidth}
        \incfig{figher2}
    \end{minipage}
    \bigbreak \noindent 
    Because this curve has exactly three petals, where each petal is symmetric covering a total range of $2\pi$, we divide $2\pi$ by 3 to get the range of just one petal. Thus, each petal covers an angle of $\frac{2\pi}{3}$. To find the area of one of the petals, we can compute the integral 
    \begin{align*}
        &\frac{1}{2}\int_{0}^{\frac{2\pi}{3}}\ r^{2}\ d\theta  \\
        &=\frac{1}{2}\int_{0}^{\frac{2\pi}{3}}\ (4\cos{3\theta })^{2}\ d\theta  \\
        &=\frac{1}{2}\int_{0}^{\frac{2\pi}{3}}\ 16\cos^{2}{3\theta}\ d\theta  \\
        &=8\int_{0}^{\frac{2\pi}{3}}\ \cos^{2}{3\theta}\ d\theta 
    .\end{align*}
    \bigbreak \noindent 
    From this, we can use the double angle formula $\cos{2\alpha} = 2\cos^{2}{\alpha}-1$ to solve for $\cos^{2}{3\theta}$
    \begin{align*}
        &\cos{2\alpha } = 2\cos^{2}{\alpha} -1 \\
        &\cos^{2}{\alpha} =\frac{1}{2} + \frac{1}{2}\cos{2\alpha}
    .\end{align*}
    \bigbreak \noindent 
    Now if we let $\alpha = 3\theta$, we can get a form that perfectly matches our integrand.
    \begin{align*}
        \cos^{2}{3\theta} = \frac{1}{2} + \frac{1}{2}\cos{6\theta}
    .\end{align*}
    Thus, we have the integral
    \begin{align*}
        &8\int_{0}^{\frac{2\pi}{3}}\ \frac{1}{2}+\frac{1}{2}\cos{6\theta }\ d\theta  \\
        &=8\bigg[\frac{1}{2}\theta +\frac{1}{12}\sin{6\theta }\bigg|_0^{\frac{2\pi}{3}} \\
        &=8\bigg[\left(\frac{\pi}{3}- 0\right) - \left(\frac{1}{12}\sin{\left(4\pi\right)}-\sin{0}\right)\bigg] \\
        &=\frac{8\pi}{3}
    .\end{align*}

    \pagebreak \bigbreak \noindent 
    \begin{mdframed}
        6. Find the length of the curve $r=2\cos{\theta} $ on the interval $0 \leq \theta \leq 2\pi $
    \end{mdframed}
    \bigbreak \noindent 
    \begin{remark}
        
        Let $f$ be a function whose derivative is continous on the interval $\alpha \leq \theta \leq \beta$. The length of the graph $r=f(\theta )$ from $\theta =\alpha$ to $\theta =\beta$ is 
        \begin{align*}
            \int_{\alpha}^{\beta}\ \sqrt{r^{2} + \left(\frac{dr}{d\theta }\right)^{2}}\ d\theta 
        .\end{align*}
    \end{remark}
    \bigbreak \noindent 
    With this, we can find the length of the curve 
    \begin{align*}
        &\int_{0}^{2\pi}\ \sqrt{4\cos^{2}{\theta} + 4\sin^{2}{\theta }}\ d\theta  \\
        &=\int_{0}^{2\pi}\ \sqrt{4(\cos^{2}{\theta } + \sin^{2}{\theta })}\ d\theta \\
        &=\int_{0}^{2\pi}\ \sqrt{4}\ d\theta  \\
        &=2\bigg[\theta \bigg|_0^{2\pi} \\
        &=4\pi
    .\end{align*}

    \bigbreak \noindent 
    \begin{mdframed}
        7. Find the slope of the tangent line to the polar curve $r=3\cos{\theta}$ at the point $\theta =\frac{\pi}{3} $.
    \end{mdframed}
    \bigbreak \noindent 
    First, we find $r$
    \begin{align*}
        &r = 3\cos{\left(\frac{\pi}{3}\right)} \\
        &=\frac{3}{2}
    .\end{align*}
    From here we must find $\frac{dy}{dx}$ at $\theta =\frac{\pi}{3}$ to get the slope
    \begin{align*}
        &\frac{dy}{dx} = \frac{\frac{dy}{d\theta }}{\frac{dx}{d\theta }} = \frac{\frac{dr}{d\theta}\sin{\theta } + r\cos{\theta }}{\frac{dr}{d\theta }\cos{\theta } - r\sin{\theta }}\\
        &\frac{dy}{dx} = \frac{-3\sin^{2}{\theta}+\frac{3}{2}\cos{\theta }}{-3\sin{\theta }\cos{\theta } - \frac{3}{2}\sin{\theta }}  \\
        &\frac{dy}{dx} = \frac{-3\sin^{2}{\left(\frac{\pi}{3}\right)}+\frac{3}{2}\cos{\left(\frac{\pi}{3}\right) }}{-3\sin{\left(\frac{\pi}{3}\right) }\cos{\left(\frac{\pi}{3}\right) } - \frac{3}{2}\sin{\left(\frac{\pi}{3}\right) }}  \\
        &=\frac{-3\left(\frac{\sqrt{3}}{2}\right)^{2}+\frac{3}{2}\left(\frac{1}{2}\right)}{-3\left(\frac{\sqrt{3}}{2}\right)\left(\frac{1}{2}\right)-\frac{3}{2}\left(\frac{\sqrt{3}}{2}\right)} \\
        &=\frac{\sqrt{3}}{3}
    .\end{align*}

    \pagebreak \bigbreak \noindent 
    \begin{mdframed}
        8. Find the area of the region that lies inside $r = 1+\cos{\theta}$ and outside $r=\cos{\theta}$
    \end{mdframed}
    \bigbreak \noindent 
    First, we find the area of $1+\cos{\theta}$
    \begin{align*}
        &\frac{1}{2}\int_{0}^{2\pi}\ (1+\cos{\theta })^{2}\ d\theta  \\
        &\frac{1}{2}\int_{0}^{2\pi}\ 1+2\cos{\theta } + \cos^{2}{\theta }\ d\theta  \\
        &\frac{1}{2}\int_{0}^{2\pi}d\theta  + \frac{1}{2}\int_{0}^{2\pi}\ 2\cos{\theta }\ d\theta  + \frac{1}{2}\int_{0}^{2\pi}\ \frac{1}{2} + \frac{1}{2}\cos{2\theta }\ d\theta  \\
        &=\frac{1}{2}\bigg[\theta \bigg|_0^{2\pi} + \bigg[\sin{\theta }\bigg|_0^{2\pi} + \frac{1}{4}\bigg[\theta +\frac{1}{2} \sin{2\theta }\bigg|_0^{2\pi} \\
        &= \pi + \left(0-0\right) + \left(\frac{1}{4}\bigg[2\pi + \frac{1}{2}\sin{4\pi} - \left(0 + \frac{1}{2}\sin{(0)}\right)\bigg]\right) \\
        &=\pi + \left(\frac{\pi}{2}+0 - 0\right) \\
        &=\pi+\frac{\pi}{2}\\
        &=\frac{3\pi}{2}
    .\end{align*}
    \bigbreak \noindent 
    Then we can find the area of $r = \cos{\theta}$
    \begin{align*}
        &\frac{1}{2}\int_{0}^{\pi}\ \cos^{2}{\theta } \ d\theta  \\
        &\frac{1}{2}\int_{0}^{2\pi}\ \frac{1}{2} + \frac{1}{2}\cos{2\theta }\ d\theta  \\
        &=\frac{1}{2}\bigg[\frac{1}{2}\theta +\frac{1}{4}\sin{2\theta}\bigg|_0^{2\pi} \\
        &=\frac{1}{2}\left[\pi + 0 - 0 + 0 \right] \\
        &=\frac{\pi}{2}
    .\end{align*}

    \bigbreak \noindent 
    Thus we have 
    \begin{align*}
        &A_{1} - A_{2} \\
        &\frac{3\pi}{2} - \frac{\pi}{2} \\
        &= \pi
    .\end{align*}
    

    



     
 \end{document} 
