 \documentclass{report}
 
 \input{~/dev/latex/template/preamble.tex}
 \input{~/dev/latex/template/macros.tex}
 
 \title{\Huge{}}
 \author{\huge{Nathan Warner}}
 \date{\huge{}}
 \fancyhf{}
 \rhead{}
 \fancyhead[R]{\itshape Warner} % Left header: Section name
 \fancyhead[L]{\itshape\leftmark}  % Right header: Page number
 \cfoot{\thepage}
 \renewcommand{\headrulewidth}{0pt} % Optional: Removes the header line
 %\pagestyle{fancy}
 %\fancyhf{}
 %\lhead{Warner \thepage}
 %\rhead{}
 % \lhead{\leftmark}
 %\cfoot{\thepage}
 %\setborder
 % \usepackage[default]{sourcecodepro}
 % \usepackage[T1]{fontenc}
 
 % Change the title
 \hypersetup{
     pdftitle={}
 }

 \geometry{
  left=1.5in,
  right=1.5in,
  top=1in,
  bottom=1in
}
 
 \begin{document}
     % \maketitle
     %     \begin{titlepage}
     %    \begin{center}
     %        \vspace*{1cm}
     % 
     %        \textbf{}
     % 
     %        \vspace{0.5cm}
     %         
     %             
     %        \vspace{1.5cm}
     % 
     %        \textbf{Nathan Warner}
     % 
     %        \vfill
     %             
     %             
     %        \vspace{0.8cm}
     %      
     %        \includegraphics[width=0.4\textwidth]{~/niu/seal.png}
     %             
     %        Computer Science \\
     %        Northern Illinois University\\
     %        United States\\
     %        
     %             
     %    \end{center}
     % \end{titlepage}
     % \tableofcontents
    \pagebreak \bigbreak \noindent
    Nate Warner \ \quad \quad \quad \quad \quad \quad \quad \quad \quad \quad \quad \quad  MATH 232 \quad  \quad \quad \quad \quad \quad \quad \quad \quad \ \ \quad \quad Spring 2024
    \begin{center}
        \textbf{Homework/Worksheet 8 - Due: Sunday, April 7}
    \end{center}
    \bigbreak \noindent 

    \begin{mdframed}
        1. Evaluate the integrals below:
        \begin{enumerate}[label=(\alph*)]
            \item \relax $\int_{\ln{(2)}}^{\ln{(3)}}\left(\int_{0}^{1}\ e^{x+y}\ dy\right)\ dx$
            \item  $\int_{1}^{e}\ \int_{1}^{e}\ \frac{\sin{\left(\ln{(x)})\cos{\left(\ln{(y)}\right)}\right)}}{xy}\ dx\ dy$
            \item  $\int_{1}^{2}\ \int_{0}^{1}\ xe^{x-y} dy\, dx$
        \end{enumerate}
    \end{mdframed}
    \bigbreak \noindent 
    \textbf{Problem 1a.}
    \begin{align*}
        &\int_{\ln{(2)}}^{\ln{(3)}}\left(\int_{0}^{1}\ e^{x+y}\ dy\right)\ dx \\
        &=\int_{\ln{2}}^{\ln{3}}\ e^{x}\int_{0}^{1}\ e^{y}\ dy\ dx \\
        &=\int_{\ln{2}}^{\ln{3}}\ e^{x}[e^{y}]_0^{1}\ dx \\
        &=\int_{\ln{2}}^{\ln{3}}\ e^{x}(e-1)\ dx \\
        &=e\int_{\ln{2}}^{\ln{3}}\ e^{x}\ dx - \int_{\ln{2}}^{\ln{3}}\ e^{x}\ dx \\
        &=e[e^{x}]_{\ln{2}}^{\ln{3}} - e^{x}\bigg|_{\ln{2}}^{\ln{3}} \\
        &=e(3-2) - (3-2) \\
        &=e-1
    .\end{align*}

    \bigbreak \noindent 
    \textbf{Problem 1b.}
    \begin{align*}
         &\int_{1}^{e}\ \int_{1}^{e}\ \frac{\sin{\left(\ln{(x)})\cos{\left(\ln{(y)}\right)}\right)}}{xy}\ dx\ dy \\
         &=\int_{1}^{e}\ \frac{\cos{\left(\ln{(y)}\right)}}{y}\ \int_{1}^{e}\ \frac{\sin{\left(\ln{(x)}\right)}}{x}\ dxdy \\
         &=\int_{1}^{e}\ \frac{\cos{\left(\ln{(y)}\right)}}{y}\ \int_{0}^{1}\ \sin{\left(u\right)}\ du dy \\
         &=\int_{e}^{1}\ \frac{\cos{\left(\ln{(y)}\right)}}{y}\ [\cos{\left(u\right)}]_{0}^{1}dy \\
         &=\int_{e}^{1}\ \frac{\cos{\left(\ln{(y)}\right)}}{y}\ [\cos{\left(1\right)} -1]dy \\
         &=\cos{\left(1\right)}\int_{1}^{0}\ \cos{\left(u\right)}\ du - \int_{1}^{0}\ \cos{\left(u\right)}\ du \\
         &=\cos{\left(1\right)}(-\sin{\left(1\right)}) - (-\sin{\left(1\right)}) \\
         &=\sin{\left(1\right)}(-\cos{\left(1\right)} + 1)
    .\end{align*}

    \pagebreak 
    \textbf{Problem 1c.}
    \begin{align*}
        &\int_{1}^{2}\ \int_{0}^{1}\ xe^{x-y} dy\, dx \\
        &=\int_{0}^{1}\ xe^{x}\int_{1}^{2}\ e^{y}\ dy\ dx \\
        &=\int_{0}^{1}\ xe^{x}[e^{y}]_{1}^{2}\ dx \\
        &=\int_{0}^{1}\ xe^{x}(e^{2}-e)\ dx \\
        &=e^{2}\int_{0}^{1}\ xe^{x}\ dx - e\int_{0}^{1}\ xe^{x}\ dx \\
        &=e^{2}\bigg[xe^{x}-\int_{0}^{1}\ e^{x}\ dx\bigg] - e\bigg[xe^{x} -\int_{0}^{1}\ e^{x}\ dx\bigg] \\
        &=e^{2}\bigg[e-e -(-1)\bigg] - e\bigg[e-e -(-1)\bigg] \\
        &= e^{2} - e
    .\end{align*}

    \bigbreak \noindent 
    \begin{mdframed}
        2. Find the volume of the solid under the surface $z = 2x + y^2$ and above the region bounded by $y = x^{5}$ and $y = x$. 
    \end{mdframed}
    \bigbreak \noindent 
    We find the volume of the solid under the surface $z=2x+y^2$ by integrating over the region $D = \{(x,y):\ 0 \leq x \leq 1,\ x^{5} \leq y \leq x\} $. The bounds of $x$ was found by finding the points of interception between the two curves $x^{5}$ and $x$. That is, $x^{5} -x = 0 \implies x = 0,\ x= 1$. The bounds of $y$ were found by examining the two curves and seeing that $x^{5} \leq x\ \forall x \in [0,1] $. 
    \bigbreak \noindent 
    The volume is then given by
    \begin{align*}
        V &=\iint_{D}(2x+y^{2})dA \\
        &=\int_{0}^{1}\ \int_{x^{5}}^{x}\ 2x+y^{2}\ dy\ dx \\
        &=\int_{0}^{1}\ 2xy + \frac{1}{3}y^{3}\bigg|^{x}_{x^{5}}\ dx \\
        &=\int_{0}^{1}\ 2x^{2} + \frac{1}{3}x^{3} - 2x^{6} - \frac{1}{3}x^{15}\ dx \\
        &=\frac{2}{3}x^{3} + \frac{1}{12}x^{4} - \frac{2}{7} - \frac{1}{48}x^{16}\bigg|^{1}_{0} \\
        &=\frac{2}{3} + \frac{1}{12} - \frac{2}{7} - \frac{1}{48} \\
        &=\frac{149}{336}
    .\end{align*}

    \pagebreak 
    \begin{mdframed}
        3. Find the volume of the solid under the plane $z = 3x + y$ and above the region determined by $y = x^{7}$ and $y = x$.
    \end{mdframed}
    \bigbreak \noindent 
    The region $D$ for this integral is similar to the last. It is $D = \{(x,y):\ 0 \leq x \leq 1,\ x^{7} \leq y \leq x\} $. Thus the volume is given by the integral
    \begin{align*}
        V &= \iint_{D}(3x+y)\, dA \\
          &= \int_{0}^{1}\ \int_{x^{7}}^{x}\ 3x+y\ dy\ dx \\
          &= \int_{0}^{1}\ 3xy + \frac{1}{2}y^{2}\bigg|^{x}_{x^{7}}\ dx \\
          &=\int_{0}^{1}\ 3x^{2}+\frac{1}{2}x^{2}-3x^{8}-\frac{1}{2}x^{14}\ dx \\
          &=x^{3}+\frac{1}{6}x^{3} -\frac{1}{3}x^{9}-\frac{1}{30}x^{15}\bigg|^{1}_{0} \\
          &=1+\frac{1}{6}-\frac{1}{3}-\frac{1}{30} \\
          &=\frac{4}{5}
    .\end{align*}

    \bigbreak \noindent 
    \begin{mdframed}
        4. Find the volume of the solid bounded by the planes $x + y = 1$, $x - y = 1$, $x = 0$, $z = 0$, and $z = 10$.
    \end{mdframed}
    \bigbreak \noindent 
    To find the volume of the solid under the given parameters, we identify the three dimensional region  and integrate over it.
    \begin{align*}
        &(1): \quad x+y =1 \implies y = x-1 \\
        &(2): \quad x-y = 1 \implies y = 1-x \\
        &(3): \quad x = 0 
    .\end{align*}
    \bigbreak \noindent 
    We see functions these define our region on the $xy$-plane. We can equate 1 and 2 to find the right bound of $x$ values
    \begin{align*}
        x -1  &= 1- x \\
              \implies x &= 1
    .\end{align*}
    \bigbreak \noindent 
    Thus, our region $E$ is defined by
    \begin{align*}
        E = \{(x,y,z):\ 0 \leq x \leq 1,\ x-1 \leq y \leq 1-x,\ 0 \leq z \leq 10\}
    .\end{align*}
    And the volume of this region is given by
    \begin{align*}
        V &= \iiint_{E}\, dV \\
          &=\int_{0}^{1}\int_{x-1}^{1-x}\int_{0}^{10}  \, dzdydx \\
          &=\int_{0}^{1}\int_{x-1}^{1-x} z\bigg|^{10}_{0} \, dydx \\
          &=\int_{0}^{1}10\int_{x-1}^{1-x}\, dydx \\
          &=\int_{0}^{1} 10\bigg[y\bigg|_{x-1}^{1-x} \, dx \\
          &=10\int_{0}^{1} 1-x -(x-1) \, dx \\
          &=10\int_{0}^{1} -2x +2 \, dx \\
          &=-20\int_{0}^{1} x -1 \, dx \\
          &=-20\bigg[\frac{1}{2}x^{2}-x\bigg|_{0}^{1} \\
          &=-20\left(\frac{1}{2}-1\right) \\
          &=10
    .\end{align*}

    \bigbreak \noindent 
    \begin{mdframed}
        5. Evaluate the following integrals by changing the order of integration.
        \begin{enumerate}[label=(\alph*)]
            \item $\int_{-1}^{\frac{\pi}{2}}\int_{0}^{x+1}  \sin{\left(x\right)}\,dydx  $
            \item $\int_{-1}^{0}\int_{-\sqrt{y+1}}^{\sqrt{y+1}}  y^{2}\,dxdy $
        \end{enumerate}
    \end{mdframed}
    \bigbreak \noindent 
    \textbf{Problem 5a.} We see that the current domain of integration is given by
    \begin{align*}
        D = \{(x,y):\ -1 \leq x \leq\frac{\pi}{2},\ 0 \leq y \leq x+1\}
    .\end{align*}
    We can then convert this type 1 region to a region of type 2.
    \begin{align*}
        \therefore D = \{(x,y):\ y-1 \leq x \leq \frac{\pi}{2},\ 0 \leq y \leq \frac{\pi}{2} + 1\}
    .\end{align*}
    From this it follows that our integral becomes
    \begin{align*}
        &\int_{0}^{\frac{\pi}{2}+1}\int_{y-1}^{\frac{\pi}{2}}  \sin{\left(x\right)}\, dxdy \\
        &=-\int_{0}^{\frac{\pi}{2}+1}  \cos{\left(x\right)}\bigg|^{y-1}_{\frac{\pi}{2}}\, dy \\
        &=-\int_{0}^{\frac{\pi}{2}+1}  \cos{\left(\frac{\pi}{2} \right)} - \cos{\left(y-1\right)}\, dy \\
        &=\int_{0}^{\frac{\pi}{2}+1}  \cos{\left(y-1\right)}\, dy \\
        &=\sin{\left(y-1\right)}\bigg|^{0}_{\frac{\pi}{2}+1} \\
        &=\sin{\left(\frac{\pi}{2}+1\right)} - \sin{\left(-1\right)} \\
        &=\sin{\left(\frac{\pi}{2}\right)}\cos{\left(1\right)}+\cos{\left(\frac{\pi}{2}\right)}\sin{\left(1\right)} + \sin{\left(1\right)} \\
        &=1 + \sin{\left(1\right)}
    .\end{align*}

    \bigbreak \noindent 
    \textbf{Problem 5b.} We see that our current region is of type 2 and is given by
    \begin{align*}
        D = \{(x,y):\ -1 \leq y \leq 0,\ -\sqrt{y+1} \leq x \leq\sqrt{y+1}\}
    .\end{align*}
    We can then change this region to type 1.
    \begin{align*}
        D = \{(x,y):\ -1 \leq x \leq 1,\ x^{2}-1 \leq y \leq 0\}
    .\end{align*}
    Thus, we have the integral
    \begin{align*}
        &\int_{-1}^{1}\int_{x^{2}-1}^{0} y^{2}\, dydx \\
        &=\int_{-1}^{1}  \frac{1}{3}\bigg[y^{3}\bigg]_{x^{2}-1}^{0}\, dx \\
        &=-\frac{1}{3}\int_{-1}^{1} (x^{2}-1)^{3} \, dx \\
        &=-\frac{1}{3}\int_{-1}^{1} x^{6} -3x^{4}+3x^{2}-1 \, dx \\
        &=-\frac{1}{3}\bigg[\frac{1}{7}x^{7}-\frac{3}{5}x^{5}+x^{3}-x\bigg]_{-1}^{1} \\
        &=\frac{32}{105}
    .\end{align*}






 \end{document} % (:
