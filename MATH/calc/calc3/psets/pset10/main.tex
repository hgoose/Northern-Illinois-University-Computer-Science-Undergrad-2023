 \documentclass{report}
 
 \input{~/dev/latex/template/preamble.tex}
 \input{~/dev/latex/template/macros.tex}
 
 \title{\Huge{}}
 \author{\huge{Nathan Warner}}
 \date{\huge{}}
 \fancyhf{}
 \rhead{}
 \fancyhead[R]{\itshape Warner} % Left header: Section name
 \fancyhead[L]{\itshape\leftmark}  % Right header: Page number
 \cfoot{\thepage}
 \renewcommand{\headrulewidth}{0pt} % Optional: Removes the header line
 %\pagestyle{fancy}
 %\fancyhf{}
 %\lhead{Warner \thepage}
 %\rhead{}
 % \lhead{\leftmark}
 %\cfoot{\thepage}
 %\setborder
 % \usepackage[default]{sourcecodepro}
 % \usepackage[T1]{fontenc}
 
 % Change the title
 \hypersetup{
     pdftitle={}
 }

 \geometry{
  left=1.5in,
  right=1.5in,
  top=1in,
  bottom=1in
}
 
 \begin{document}
     % \maketitle
     %     \begin{titlepage}
     %    \begin{center}
     %        \vspace*{1cm}
     % 
     %        \textbf{}
     % 
     %        \vspace{0.5cm}
     %         
     %             
     %        \vspace{1.5cm}
     % 
     %        \textbf{Nathan Warner}
     % 
     %        \vfill
     %             
     %             
     %        \vspace{0.8cm}
     %      
     %        \includegraphics[width=0.4\textwidth]{~/niu/seal.png}
     %             
     %        Computer Science \\
     %        Northern Illinois University\\
     %        United States\\
     %        
     %             
     %    \end{center}
     % \end{titlepage}
     % \tableofcontents
    \pagebreak \bigbreak \noindent
    Nate Warner \ \quad \quad \quad \quad \quad \quad \quad \quad \quad \quad \quad \quad  MATH 232 \quad  \quad \quad \quad \quad \quad \quad \quad \quad \ \ \quad \quad Spring 2024
    \begin{center}
        \textbf{Homework/Worksheet 10 - Due: Saturday, April 27}
    \end{center}
    \bigbreak \noindent 
    \begin{mdframed}
        1. Use spherical coordinates to find the volume of the ball $\rho \leq 3$ that is situated between the cones $\varphi = \frac{\pi}{4}$ and $\varphi = \frac{\pi}{3}$.
    \end{mdframed}
    \bigbreak \noindent 
    To find the volume of the ball, we first determine the spherical region $E$, which we see is given by
    \begin{align*}
        E = \{(\rho, \theta, \varphi):\ 0 \leq \theta  \leq 2\pi,\ \frac{\pi}{4} \leq \varphi \leq \frac{\pi}{3},\ 0 \leq \rho \leq 3\}
    .\end{align*}
    \bigbreak \noindent 
    And the integral for the volume of the ball is given by
    \begin{align*}
        &\int_{0}^{2\pi}\int_{\frac{\pi}{4}}^{\frac{\pi}{3}} \int_{0}^{3} \rho^{2}\sin{\left(\varphi\right)}d\rho d\varphi d\theta  \\
        &=\int_{0}^{2\pi}\int_{\frac{\pi}{4}}^{\frac{\pi}{3}} \frac{1}{3}\bigg[\rho^{3}\bigg|_0^{3}\sin{\left(\varphi\right)}d\varphi d\theta  \\
        &=\int_{0}^{2\pi}\int_{\frac{\pi}{4}}^{\frac{\pi}{3}} 9\sin{\left(\varphi\right)}d\varphi d\theta  \\ 
        &=\int_{0}^{2\pi} -9\bigg[\cos{\left(\varphi\right)}\bigg|_{\frac{\pi}{4}}^{\frac{\pi}{3}}d\theta  \\ 
        &=\int_{0}^{2\pi} -9\left(\frac{1}{2}-\frac{\sqrt{2}}{2}\right)d\theta  \\ 
        &=\int_{0}^{2\pi} -\frac{9}{2}+\frac{9\sqrt{2}}{2}d\theta  \\ 
        &=-\frac{9}{2}(2\pi - 0) + \frac{9\sqrt{2}}{2}(2\pi-0) \\
        &=-9\pi + 9\pi\sqrt{2}
    .\end{align*}

    \bigbreak \noindent 
    \begin{mdframed}
        2. Convert the integral
        \[
            \int_{-4}^{4} \int_{-\sqrt{16-y^2}}^{\sqrt{16-y^2}} \int_{-\sqrt{16-x^2-y^2}}^{\sqrt{16-x^2-y^2}} \left( x^2 + y^2 + z^2 \right) \,dz\,dx\,dy
        \]
        into an integral in spherical coordinates.
    \end{mdframed}
    \bigbreak \noindent 
    We clearly see that the region described by the given integral is a sphere of radius 4, with the bounds of integration for the first two integrals being a projection of this sphere onto the $xy$-plane. Thus, this is an elementary region regarding spherical coordinates, the conversion is trivial
    \begin{align*}
        E = \{(\rho, \theta, \varphi):\ 0 \leq \theta  \leq 2\pi,\ 0 \leq \varphi \leq \pi,\ 0 \leq \rho \leq 4\}
    .\end{align*}
    \bigbreak \noindent 
    Before we construct the new integral, we must transform the integrand using rectangular to spherical conversion formulas. We remark $\rho^{2} = x^{2} + y^{2} + z^{2}$. Hence, we have the integral
    \begin{align*}
        \int_{0}^{2\pi }\int_{0}^{\pi }\int_{0}^{3} \rho^{2}\rho^{2}\sin{\left(\varphi\right)} \, d\rho d\varphi d\theta 
    .\end{align*}
    \bigbreak \noindent 
    Where $\rho^{2}\sin{\left(\varphi\right)} $ is the Jacobian of the rectangular $\rightarrow$ spherical transformation.
    \begin{align*}
        \implies&\int_{0}^{2\pi }\int_{0}^{\pi }\int_{0}^{4} \rho^{4}\sin{\left(\varphi\right)} \, d\rho d\varphi d\theta  \\
                &=\int_{0}^{2\pi }\int_{0}^{\pi }\frac{1}{5}\bigg[\rho^{5}\bigg|_0^{4}\sin{\left(\varphi\right)} \,d\varphi d\theta  \\
                &=\int_{0}^{2\pi }\int_{0}^{\pi }\frac{1024}{5}\sin{\left(\varphi\right)} \,d\varphi d\theta  \\
                &=\int_{0}^{2\pi }-\frac{1024}{5}\bigg[\cos{\left(\varphi\right)}\bigg|_{0}^{\pi}\,d\theta  \\
                &=\int_{0}^{2\pi }-\frac{1024}{5}\bigg[\cos{\left(\varphi\right)}\bigg|_{0}^{\pi}\,d\theta  \\
                &=\frac{1024 \cdot  2}{5}(2\pi - 0) \\
                &=\frac{4096\pi}{5}
    .\end{align*}


    \bigbreak \noindent 
    \begin{mdframed}
        3. Convert the integral
        \[
        \int_{0}^{4} \int_{0}^{\sqrt{16-x^2}} \int_{-\sqrt{16-x^2-y^2}}^{\sqrt{16-x^2-y^2}} \left( x^2 + y^2 + z^2 \right)^2 \,dz\,dy\,dx
        \]
        into an integral in spherical coordinates.
    \end{mdframed}
    \bigbreak \noindent 
    This region is almost the same as the previous problem, but the region $D$ in the $xy$-plane is reduced to the quarter circle (in the first quadrant). This reduces the bounds for $\theta$ to $0 \leq \theta  \leq\frac{\pi}{2}$. The other bounds remain the same. Furthermore, the integrand becomes $\rho^{4}$ instead of $\rho^{2}$
    \begin{align*}
        \int_{0}^{\frac{\pi}{2 }}\int_{0}^{\pi }\int_{0}^{4} r^{6}\sin{\left(\varphi\right)} \, d\rho d\varphi d\theta  \\
    .\end{align*}
    This is an elementary integral resulting in the quantity $\frac{16384\pi}{7} $


    \bigbreak \noindent 
    \begin{mdframed}
        4. Express the volume of the solid inside the sphere \(x^2 + y^2 + z^2 = 16\) and outside the cylinder \(x^2 + y^2 = 4\) as triple integrals in cylindrical coordinates and spherical coordinates, respectively.
    \end{mdframed}
    \bigbreak \noindent 
    We can represent this region in cylindrical coordinates as
    \begin{align*}
        E = \{(r,\theta ,z):\ 0 \leq \theta \leq 2\pi,\ 2 \leq r \leq 4,\ -\sqrt{16-r^{2}} \leq z \leq \sqrt{16-r^{2}}\}
    .\end{align*}
    Which gives the integral
    \begin{align*}
        \int_{0}^{2\pi }\int_{2}^{4}\int_{-\sqrt{16-r^{2}}}^{\sqrt{16-r^{2}}}  z\,dzdrd\theta  
    .\end{align*}
    \bigbreak \noindent 
    Converting the given region to spherical takes some work. Unlike the cylindrical region, which was easily visualized and transformed, the transformation to spherical will be made analytically.
    \bigbreak \noindent 
    First, we note that the bounds for $\theta$ are quite clear, theta will range from $0$ to $2\pi$. To find the remaining bounds, we play around with the conversion forumlas and see what we can find out. The conversion formulas for rectangular $\rightarrow$ spherical are $x=\rho\sin{\left(\varphi\right)}\cos{\left(\theta \right)} $, $y=\rho\sin{\left(\varphi\right)}\sin{\left(\theta \right)} $, $z=\rho\cos{\left(\varphi\right)} $, and $\rho^{2} = x^{2} + y^{2} + z^{2}$
    \begin{align*}
        &x^{2} + y^{2} + z^{2} = 16 \implies \rho = 4\\
        &x^{2} + y^{2} = 4 \\
        \implies &4 + z^{2} = 16  \\
        \implies &z = \pm \sqrt{12} = \pm 2\sqrt{3}
    .\end{align*}
    \bigbreak \noindent 
    Since the region is bounded above by the sphere, which has radius 4, we know that $\rho = 4$ must be the upper bound. The lowerbound is going to take some more work. To find the lower bound, we must examine the equation of the cylinder 
    \begin{align*}
        &x^{2} + y^{2} = 4 \\
        \implies & \rho^{2}\sin^{2}{\left(\varphi\right)}\cos^{2}{\left(\theta \right)} + \rho^{2}\sin^{2}{\left(\varphi\right)}\sin^{2}{\left(\theta \right)} = 4 \\
        \implies & \rho^{2}\sin^{2}{\left(\varphi\right)} =4 \\
        \implies & \rho = \frac{2}{\sin{\left(\varphi\right)}}
    .\end{align*}
    This must be the lower bound for the integral regarding $\rho$. Lastly, to find the bounds for $\varphi$, we again turn our attention to the intersection.
    \begin{align*}
        z&=\pm2\sqrt{3} \\
        \text{with } z &= \rho\cos{\left(\varphi\right)} \implies \pm2\sqrt{3} = \rho \cos{\left(\varphi\right)} \\
        \implies  \varphi &=\cos^{-1}{\left(\frac{\pm 2\sqrt{3}}{4}\right)} \\
                          &=\frac{\pi}{6},\, \frac{5\pi}{6}
    .\end{align*}
    Thus, we have the integral
    \begin{align*}
        \int_{0}^{2 \pi }\int_{\frac{\pi}{6}}^{\frac{5\pi}{6}}\int_{\frac{2}{\sin{\left(\varphi\right)} }}^{4} \rho^{2}\sin{\left(\varphi\right)} \, d\rho d\varphi d\theta 
    .\end{align*}

    \bigbreak \noindent 
    \begin{mdframed}
        5. Describe the vector field \(F(x, y) = \langle y, \sin(x) \rangle\) by drawing some of its vectors.
    \end{mdframed}
    \bigbreak \noindent 
    To draw this vector field, we create a table with a few coordinates, then find the corresponding vector for each coordinate. We then plot the vectors
    \bigbreak \noindent 
    \begin{center}
        \begin{tabular}{ccc}
            \hline
            \(x\) & \(y\) & Vector \(\langle y, \sin(x) \rangle\) \\ \hline
            0 & 0 & \(\langle 0, 0 \rangle\) \\
            \(\frac{\pi}{2}\) & 1 & \(\langle 1, 1 \rangle\) \\
            \(\pi\) & 2 & \(\langle 2, 0 \rangle\) \\
            \(\frac{3\pi}{2}\) & -1 & \(\langle -1, -1 \rangle\) \\
            \(2\pi\) & 0 & \(\langle 0, 0 \rangle\) \\
            \(\frac{\pi}{4}\) & 2 & \(\langle 2, \frac{\sqrt{2}}{2} \rangle\) \\
            \(\frac{3\pi}{4}\) & -2 & \(\langle -2, -\frac{\sqrt{2}}{2} \rangle\) \\ \hline
        \end{tabular}
    \end{center}
    \bigbreak \noindent 
    \fig{.5}{~/Downloads/ec571962-1864-49f7-be4c-fed29f9f98bf.png}

    \bigbreak \noindent 
    \begin{mdframed}
        6. Find the gradient vector field of the function \(f(x, y) = x \sin y + \cos y\).
    \end{mdframed}
    \bigbreak \noindent 
    To find the gradient vector field of the given function, we simply find the gradient vector $\nabla f(x,y) = \left\langle f_{x}(x,y), f_{y}(x,y) \right\rangle $
    \begin{align*}
        \nabla f(x,y) = \left\langle \sin{\left(y\right)},x-\sin{\left(y\right)} \right\rangle
    .\end{align*}


 \end{document} % (:
