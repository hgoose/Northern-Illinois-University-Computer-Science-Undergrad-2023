 \documentclass{report}
 
 \input{~/dev/latex/template/preamble.tex}
 \input{~/dev/latex/template/macros.tex}
 
 \title{\Huge{}}
 \author{\huge{Nathan Warner}}
 \date{\huge{}}
 \fancyhf{}
 \rhead{}
 \fancyhead[R]{\itshape Warner} % Left header: Section name
 \fancyhead[L]{\itshape\leftmark}  % Right header: Page number
 \cfoot{\thepage}
 \renewcommand{\headrulewidth}{0pt} % Optional: Removes the header line
 %\pagestyle{fancy}
 %\fancyhf{}
 %\lhead{Warner \thepage}
 %\rhead{}
 % \lhead{\leftmark}
 %\cfoot{\thepage}
 %\setborder
 % \usepackage[default]{sourcecodepro}
 % \usepackage[T1]{fontenc}
 
 % Change the title
 \hypersetup{
     pdftitle={}
 }

 \geometry{
  left=1.5in,
  right=1.5in,
  top=1in,
  bottom=1in
}
 
 \begin{document}
     % \maketitle
     %     \begin{titlepage}
     %    \begin{center}
     %        \vspace*{1cm}
     % 
     %        \textbf{}
     % 
     %        \vspace{0.5cm}
     %         
     %             
     %        \vspace{1.5cm}
     % 
     %        \textbf{Nathan Warner}
     % 
     %        \vfill
     %             
     %             
     %        \vspace{0.8cm}
     %      
     %        \includegraphics[width=0.4\textwidth]{~/niu/seal.png}
     %             
     %        Computer Science \\
     %        Northern Illinois University\\
     %        United States\\
     %        
     %             
     %    \end{center}
     % \end{titlepage}
     % \tableofcontents
    \pagebreak \bigbreak \noindent
    Nate Warner \ \quad \quad \quad \quad \quad \quad \quad \quad \quad \quad \quad \quad  MATH 232 \quad  \quad \quad \quad \quad \quad \quad \quad \quad \ \ \quad \quad Spring 2024
    \begin{center}
        \textbf{Homework/Worksheet 5 - Due: Friday, February 28}
    \end{center}
    \bigbreak \noindent 

    \begin{mdframed}
        1. Find the arc length of the curve $r(t) = \left\langle 2\sin{\left(t\right)}, 5t,2\cos{\left(t\right)} \right\rangle $, $0 \leq t \leq \pi $
    \end{mdframed}
    \bigbreak \noindent 
    \begin{remark}
        Let $\mathbf{r}(t)$ describe a smooth curve for $t \geq a$. Then the arc-length function is given by
        \begin{equation}
            s(t) = \int_{a}^{t} \|\mathbf{r}'(u)\| \, du
        \end{equation}
    \end{remark}
    \bigbreak \noindent 
    First, we find $\vec{\mathbf{r}}^{\prime}(t)$
    \begin{align*}
        \vec{\mathbf{r}}^{\prime}(t) = \left\langle 2\cos{\left(t\right)}, 5, -2\sin{\left(t\right)} \right\rangle
    .\end{align*}
    \bigbreak \noindent 
    Then, we define the arc length of the curve for $0 \leq t \leq \pi $ as 
    \begin{align*}
        &\int_{0}^{\pi}\ \norm{\vec{\mathbf{r}}^{\prime}(t)}\ dt \\
        &=\int_{0}^{\pi}\ \sqrt{4\cos^{2}{\left(t\right)} + 25 + 4\sin^{2}{\left(t\right)}}\ dt \\
        &=\int_{0}^{\pi}\ \sqrt{29}\ dt \\
        &=\pi\sqrt{29}
    .\end{align*}

    \bigbreak \noindent 
    \begin{mdframed}
        2. A thin plate made of iron is located in the $xy$-plane. The temperature $T$ in degrees Celsius at a point $P(x, y)$ is inversely proportional to the square of its distance from the origin. Express $T$ as a function of $x $ and $y$
    \end{mdframed}
    \bigbreak \noindent 
    First, let's draw our plane
    \bigbreak \noindent 
    \begin{figure}[ht]
        \centering
        \incfig{drawplane}
        \label{fig:drawplane}
    \end{figure}
    \bigbreak \noindent 
    We note that the distance between any two points is given by the distance formula
    \begin{align*}
        d = \sqrt{(x_{2} - x_{1})^{2} + (y_{2} - y_{2})^{2}}
    .\end{align*}
    Which implies the distance between any point $P$ and the origin $O(0,0)$ is given by 
    \begin{align*}
        d = \sqrt{x^{2} + y^{2}}
    .\end{align*}
    If the temperature $T$ at a point $P(x,y)$ is inversely proportional to the square of its distance from the origin. That is, $T \propto \frac{1}{d^{2}}$, then our function $T$ of $(x,y)$ is given by 
    \begin{align*}
        T(x,y) = \frac{k}{x^{2} + y^{2}}
    .\end{align*}
    Where $k$ is the proportionality constant

    \bigbreak \noindent 
    \begin{mdframed}
        3. Find the domain of the functions below:
        \begin{enumerate}[label=(\alph*)]
            \item f(x,y) = $\sqrt{x^{2} + y^{2} -4}$
            \item f(x,y) = $4\ln{(y^{2} -x)}$
            \item f(x,y) = $\sqrt{4-x^{2}-4y^{2}}$
            \item f(x,y) = $\frac{1}{\ln{(xy-6)}}$
        \end{enumerate}
    \end{mdframed}
    \bigbreak \noindent 
    \textbf{Problem 3a.} First, we determine any restrictions. We find
    \begin{align*}
        &x^{2} + y^{2} - 4 \geq 0
    .\end{align*}
    From this, change the inequality to equality so we can construct a graph of the restriction
    \begin{align*}
        x^{2} + y^{2} = 4
    .\end{align*}
    We identify this as a circle with radius 4, we note that this will be a closed set because of the non-strict inequality. We then use point (0,0) to test points inside the circle, and (5,0) to test points outside the circle. We find the domain to be all points outside the circle of radius 2 centered at the origin. Thus
    \begin{align*}
        D:\ \{(x,y) \in \mathbb{R}^{2}:\ x^{2} + y^{2} - 4 \geq 0\}
    .\end{align*}

    \bigbreak \noindent 
    \textbf{Problem 3b.} We identify the restriction to be
    \begin{align*}
        y^{2} -x > 0
    .\end{align*}
    leading to the graph 
    \begin{align*}
       y = \pm \sqrt{x} 
    .\end{align*}
    which gives us a parabola opening to the right (open set). Testing points $T_{1}(-1,0)$ for points outside the parabola and $T_{2}(1,0)$ for points inside the parabola we see that our domain is the set of all points outside of the parabola. That is,
    \begin{align*}
        D:\ \{(x,y) \in \mathbb{R}^{2}:\ y^{2}-x>0\}
    .\end{align*}

    \pagebreak \bigbreak \noindent 
    \textbf{Problem 3c.} Similar to part \texttt{a}, we find the restriction to be
    \begin{align*}
        4-x^{2}-4y^{2} \geq 0
    .\end{align*}
    Thus we have the domain
    \begin{align*}
        D:\ \{(x,y) \in \mathbb{R}^{2}:\ 4-x^{2}-4y^{2} \geq 0 \}
    .\end{align*}

    \bigbreak \noindent 
    \textbf{Problem 3d.} Here we have the restrictions
    \begin{align*}
        &xy-6 > 0 \\
        &xy-6 \neq 1;
    .\end{align*}
    Thus we have the domain
    \begin{align*}
        D:\ \{(x,y) \in \mathbb{R}^{2}:\ xy-6 > 0\ \text{and } xy-6\neq 1\}
    .\end{align*}

    \bigbreak \noindent 
    \begin{mdframed}
        4. Evaluate the limit
        \begin{align*}
            \lim\limits_{(x,y) \to (0,0)}{\frac{x^{3}-y^{3}}{x-y}}
        .\end{align*}
    \end{mdframed}
    \bigbreak \noindent 
    First, we notice that we can write the numerator as a difference of cubes
    \begin{align*}
        &\lim\limits_{(x,y) \to (0,0)}{\frac{(x-y)(x^{2}+xy+y^{2})}{x-y}} \\
        &=\lim\limits_{(x,y) \to (0,0)}{x^{2}+xy+y^{2}}
    .\end{align*}
    By properties of limits, we see
    \begin{align*}
        \lim\limits_{(x,y) \to (0,0)}{x^{2}+xy+y^{2}} &= 0+(0)(0)+0 \\
        &=0
    .\end{align*}

    \bigbreak \noindent 
    \begin{mdframed}
        5. Evaluate the limit
        \begin{align*}
            \lim\limits_{(x,y) \to (0,0)}{\frac{xy + y^{3}}{x^{2} + y^{2}}}
        .\end{align*}
    \end{mdframed}
    \bigbreak \noindent 
    Since evaluating this limit directly yields an indeterminate form, we show that the limit does not exist by showing that the limit differs among different paths.
    \bigbreak \noindent 
    For the path $y=0$ (along the $x$-axis), we have
    \begin{align*}
        \lim\limits_{x \to 0}{\frac{0}{x^{2}}} = 0
    .\end{align*}
    \bigbreak \noindent 
        For the path $x=0$ (along the $y$-axis), we have
        \begin{align*}
            \lim\limits_{y \to 0}{\frac{y^{3}}{y^{2}}} &= \lim\limits_{y \to 0}{\frac{\frac{y^{3}}{y^{2}}}{\frac{y^{2}}{y^{2}}}} \\
            &=\lim\limits_{y \to 0}{y} = 0
        .\end{align*}
        For the path $y=2x$, we have
        \begin{align*}
            &\lim\limits_{(x,2x) \to (0,0)}{\frac{2x^{2} + 8x^{3}}{x^{2} + 4x^{2}}} \\
            &=\lim\limits_{(x,2x) \to (0,0)}{\frac{2x^{2} + 8x^{3}}{5x^{2}}} \\
            &=\lim\limits_{(x,2x) \to (0,0)}{\frac{2x^{2}(1+4x)}{5x^{2}}} \\
            &=\lim\limits_{(x,2x) \to (0,0)}{\frac{2(1+4x)}{5}}  \\
            &=\frac{2(1+4(0))}{5} \\
            &=\frac{2}{5}
        .\end{align*}
        \bigbreak \noindent 
        Thus, since $0 \neq \frac{2}{5}$, it has been shown that the limit does not exist at $(0,0)$
        \bigbreak \noindent 
        \begin{mdframed}
            6.  Use polar coordinates to evaluate the limit
            \begin{align*}
                \lim_{(x,y) \to (0,0)} (x^2 + y^2) \ln(x^2 + y^2)
            .\end{align*}
        \end{mdframed}
        \bigbreak \noindent 
        First, we note that we have the conversion $x^{2} + y^{2} = r^{2}$ this leads to the new limit
        \begin{align*}
            \lim\limits_{r \to 0}{r^{2}\ln{(r^{2})}} = \lim\limits_{r \to 0}{\frac{\ln{(r^{2})}}{\frac{1}{r^{2}}}} \Heq \lim\limits_{r \to 0}{-r^{2}} = 0
        .\end{align*}

        \bigbreak \noindent 
        \begin{mdframed}
            7. Determine whether 
            \begin{align*}
                g(x,y) = \frac{x^{2}-y^{2}}{x^{2} + y^{2}}
            .\end{align*}
        \end{mdframed}
        \bigbreak \noindent 
        We remark that for a function to be continuous at a point, it must be clearly defined at that point. In this case, it is quite clear that it is not defined at the point (0,0). Thus, it is not continuous at the point $(0,0)$



    




 \end{document} 
