 \documentclass{report}
 
 \input{~/dev/latex/template/preamble.tex}
 \input{~/dev/latex/template/macros.tex}
 
 \title{\Huge{}}
 \author{\huge{Nathan Warner}}
 \date{\huge{}}
 \fancyhf{}
 \rhead{}
 \fancyhead[R]{\itshape Warner} % Left header: Section name
 \fancyhead[L]{\itshape\leftmark}  % Right header: Page number
 \cfoot{\thepage}
 \renewcommand{\headrulewidth}{0pt} % Optional: Removes the header line
 %\pagestyle{fancy}
 %\fancyhf{}
 %\lhead{Warner \thepage}
 %\rhead{}
 % \lhead{\leftmark}
 %\cfoot{\thepage}
 %\setborder
 % \usepackage[default]{sourcecodepro}
 % \usepackage[T1]{fontenc}
 
 % Change the title
 \hypersetup{
     pdftitle={}
 }

 \geometry{
  left=1.5in,
  right=1.5in,
  top=1in,
  bottom=1in
}
 
 \begin{document}
     % \maketitle
     %     \begin{titlepage}
     %    \begin{center}
     %        \vspace*{1cm}
     % 
     %        \textbf{}
     % 
     %        \vspace{0.5cm}
     %         
     %             
     %        \vspace{1.5cm}
     % 
     %        \textbf{Nathan Warner}
     % 
     %        \vfill
     %             
     %             
     %        \vspace{0.8cm}
     %      
     %        \includegraphics[width=0.4\textwidth]{~/niu/seal.png}
     %             
     %        Computer Science \\
     %        Northern Illinois University\\
     %        United States\\
     %        
     %             
     %    \end{center}
     % \end{titlepage}
     % \tableofcontents
    \pagebreak \bigbreak \noindent
    Nate Warner \ \quad \quad \quad \quad \quad \quad \quad \quad \quad \quad \quad \quad  MATH 232 \quad  \quad \quad \quad \quad \quad \quad \quad \quad \ \ \quad \quad Spring 2024
    \begin{center}
        \textbf{Homework/Worksheet 1 - Due: Wednesday, January 24}
    \end{center}
    \bigbreak \noindent 
    \begin{mdframed}
        1. For the parametric equations below, sketch the curve by eliminating the parameter.
        \begin{enumerate}[label=(\alph*)]
            \item $x=t^{2} + t$,\ $y=t^{2}-1 $ 
            \item $x=4+2\cos{\theta}$,\ $y=-1+\sin{\theta } $
            \item $x=t^{2}$,\ $y=t^{3} $
            \item $x=3\cos{t}$,\ $y=3\sin{t}$
        \end{enumerate}
    \end{mdframed}

    \bigbreak \noindent 
    \textbf{Problem 1a.} First we solve for $x(t)$ for $t$
    \begin{align*}
        &x = t^{2}+t \\
        \implies &t^{2}+t-x = 0
    .\end{align*}
    Since this is a quadratic equation, we can solve for $t$ using the quadratic formula. Thus,
    \bigbreak \noindent 
    \begin{align*}
        t = \frac{-1\pm\sqrt{1+4x}}{2}
    .\end{align*}
    \bigbreak \noindent 
    \begin{remark}
        Examing the domain of the radical we see
        \begin{align*}
            &1+4x \geq 0 \\
            &x \geq -\frac{1}{4}
        .\end{align*}
        By this fact, we notice $\forall\ x \in [-\frac{1}{4}, \infty), \frac{-1-\sqrt{1+4x}}{2} < 0$. Since are parameter refers to values of time, we can exclude the negative version. 
    \end{remark}
    \bigbreak \noindent 
    We can now plug in our equation for $t$ into $y$.
    \begin{align*}
        y &= \left(\frac{-1+\sqrt{1+4x}}{2}\right)^{2} - 1 \\
          &= \left(-\frac{1}{2}+\frac{1}{2}\sqrt{1+4x}\right)^{2} - 1 \\
          &=\left(\frac{1}{4}-\frac{1}{4}\sqrt{4x+1}-\frac{1}{4}\sqrt{4x+1}+\left(\frac{1}{2}\sqrt{4x+1}\right)^{2}\right) - 1 \\
          &=\frac{1}{4}-\frac{1}{2}\sqrt{4x+1}+\frac{1}{4}(4x+1)-1 \\
          &=x-\frac{1}{2}-\frac{1}{2}\sqrt{4x+1}
    .\end{align*}
    \pagebreak \bigbreak \noindent 
    We can now sketch the graph by identifying some key points. When $x=-\frac{1}{4}$, $y=-\frac{3}{4}$. So we have an left endpoint at $\left(-\frac{1}{4}, -\frac{3}{4}\right)$. When $y = f(x) = 0$, we have
    \begin{align*}
        &x-\frac{1}{2}-\frac{1}{2}\sqrt{4x+1} = 0 \\
        &-\frac{1}{2}\sqrt{4x+1} = \frac{1}{2}-x \\
        &-\sqrt{4x+1} = 1-2x \\
        &\sqrt{4x+1} = 2x-1 \\
        &4x+1 = (2x-1)^{2} \\
        &4x+1= 4x^{2}-4x+1 \\
        &4x = 4x^{2}-4x \\
        &x = x^{2}-x \\
        &x^{2}-2x = 0 \\
        &x(x-2) = 0 \\
        \implies &x=0,\ x=2
    .\end{align*}
    \bigbreak \noindent 
    \nt{Only $x=2$ is a valid solution. When $x=0$, 
        \begin{align*}
            &0 - \frac{1}{2} - \frac{1}{2}\sqrt{4(0) + 1} = 0 \\
            &-\frac{1}{2}-\frac{1}{2} = 0 \\
            &-1\neq 0
        .\end{align*}
    }
    \bigbreak \noindent 
    Thus, we have an x-intercept at $(2,f(2))$, or $(2,0)$. Similarly, we have a y-intercept at $f(0) = 0 - \frac{1}{2} - \frac{1}{2}\sqrt{4(0)+1} = -1$, or $(0,-1)$. With these points, we can create a rough sketch of the parametric curve.
    \begin{figure}[ht]
        \centering
        \incfig{fig123}
        \label{fig:fig123}
    \end{figure}

    \pagebreak \bigbreak \noindent 
    \textbf{Problem 1b.} Using properties of trig functions, we can relate $x$ and $y$ using the Pythagorean identity $\sin^{2}{x} + \cos^{2}{x} = 1$, We can determine
    \begin{align*}
       &\cos{\theta} = \frac{x-4}{2} \\
       &\sin{\theta} = y+1
    .\end{align*}
    Thus, 
    \begin{align*}
        &\left(\frac{x-4}{2}\right)^{2} + (y-1) = 1
    .\end{align*}
    \bigbreak \noindent 
    We notice that this is the form of an ellipse, with a horizontal major axis, and a vertical minor axis. Furthermore, we see that the center is at $(4,-1)$, and we have $a=2$, $b=1$. Thus we have a major axis of length $2a = 2(2) = 4$, and minor axis length $2(b) = 2(1) = 2$. With this information, we can sketch our ellipse.
    \bigbreak \noindent
\begin{figure}[ht]
    \centering
    \incfig{fig1234}
    \label{fig:fig1234}
\end{figure}

    \pagebreak \bigbreak \noindent 
    \textbf{Problem 1c.} We start by solving $x$ for $t$
    \begin{align*}
        &x = t^{2} \\
        \implies &t = \pm x^{\frac{1}{2}}
    .\end{align*}
    Since we are only concerned with positive values of $t$, we only need the positive version. With this, we can plug into $y$ 
    \begin{align*}
        y = x^{\frac{3}{2}}
    .\end{align*}
    Now we can simply sketch the graph

    \bigbreak \noindent 
\begin{figure}[ht]
    \centering
    \incfig{figur}
    \label{fig:figur}
\end{figure}

    \pagebreak \bigbreak \noindent 
    \textbf{Problem 1d.} First, we solve for $\sin{t}$ and $\cos{t} $
    \begin{align*}
        &\frac{x}{3} = \cos{t} \\
        &\frac{y}{3} = \sin{t}
    .\end{align*}
    Now by the Pythagorean identity we can relate $x$ and $y$
    \begin{align*}
        \frac{x^{2}}{3^{2}} + \frac{y^{2}}{3^{2}} = 1
    .\end{align*}
    \bigbreak \noindent 
    Again, we have an ellipse with $a = 3$, $b=3$, and center $(0,0)$. Thus, we can sketch our ellipse with the center at the orgin and $r=3$ for both axis.
    \bigbreak \noindent 
\begin{figure}[ht]
    \centering
    \incfig{figmane3}
    \label{fig:figmane3}
\end{figure}

    \pagebreak \bigbreak \noindent 
    \begin{mdframed}
        2. Find an equation of the tangent line to the curve at the given point.
        \begin{enumerate}[label=(\alph*)]
            \item $x=3\sin{t}$, $y=3\cos{t}$;\ $t=\frac{\pi}{4}$ 
            \item $x = t\ln{t}$, $y=\sin^{2}{t}$;\ $t=\frac{\pi}{4}$
        \end{enumerate}
    \end{mdframed}
    \bigbreak \noindent 
    \begin{remark}
        Consider the plane curve defined by the parametric equations \( x = x(t) \) and \( y = y(t) \). Suppose \( x'(t) \) and \( y'(t) \) exist, and assume that \( x'(t) \neq 0 \). Then the derivative \( \frac{dy}{dx} \) is given by
        \[
            \frac{dy}{dx} = \frac{\frac{dy}{dt}}{\frac{dx}{dt}} = \frac{y'(t)}{x'(t)}.
        \]
    \end{remark}
    
    \bigbreak \noindent 
    \textbf{Problem 2a.} We start by finding the derivative of the parametric curve
    \begin{align*}
        \frac{dy}{dx} = \frac{-3\sin{t}}{3\cos{t}} = -\tan{t}
    .\end{align*}
    We find the slope at $t=\frac{\pi}{4}$ by plugging into the derivative function 
    \begin{align*}
        &-\tan{\left(\frac{\pi}{4}\right)} = -1 \\
        &\therefore m = -1
    .\end{align*}
    Now that we have the slope, we need to find the point when $t=\frac{\pi}{4}$
    \begin{align*}
        &x\left(\frac{\pi}{4}\right) = 3\sin{\left(\frac{\pi}{4}\right)} = \frac{3\sqrt{2}}{2} \\
        &y\left(\frac{\pi}{4}\right) = 3\cos{\left(\frac{\pi}{4}\right)} = \frac{3\sqrt{2}}{2} \\
        &\therefore P\left(\frac{3\sqrt{2}}{2}, \frac{3\sqrt{2}}{2}\right)
    .\end{align*}
    Thus, the equation of the tangent line to the curve at $t=\frac{\pi}{4} $ is given by 
    \begin{align*}
        &y - \frac{3\sqrt{2}}{2} = -1\left(x-\frac{3\sqrt{2}}{2}\right) \\
        &y=-x+3\sqrt{2}
    .\end{align*}

    \pagebreak \bigbreak \noindent 
    \textbf{Problem 2b.} Again, we start by finding the derivate
    \begin{align*}
        &\frac{dy}{dx} = \frac{2\sin{t}\cos{t}}{1+\ln{t}} \\
        &\frac{dy}{dx} = \frac{\sin{(2t)}}{1+\ln{t}} \\
    .\end{align*}
    Then we find the slope by plugging $t=\frac{\pi}{4}$ into the equation
    \begin{align*}
        &\frac{\sin{\left(\frac{\pi}{2}\right)}}{1+\ln{\left(\frac{\pi}{4}\right)}} \\
        &=\frac{1}{1+\ln{\left(\frac{\pi}{4}\right)}}
    .\end{align*}
    Now we can find the point $P$ that corresponds to $t=\frac{\pi}{4} $
    \begin{align*}
        &x\left(\frac{\pi}{4}\right) = \frac{\pi}{4}\ln{\left(\frac{\pi}{4}\right)} \\
        &y\left(\frac{\pi}{4}\right) = \sin^{2}{\left(\frac{\pi}{4}\right)} = \frac{1}{2} \\
        &\therefore\ P\left(\frac{\pi}{4}\ln{\left(\frac{\pi}{4}\right)},\frac{1}{2}\right)
    .\end{align*}
    Thus, the equation of the tangent line to the curve at the point $t=\frac{\pi}{4} $ is given by
    \begin{align*}
        y - \frac{1}{2} = \frac{1}{1+\ln{\left(\frac{\pi}{4}\right)}}\left(x-\frac{\pi}{4}\ln{\left(\frac{\pi}{4}\right)}\right)
    .\end{align*}

    \pagebreak \bigbreak \noindent 
    \begin{mdframed}
        3. Let \( x = 3t^2 \), \( y = t^3 - t \). Find \(\frac{d^2y}{dx^2}\) and the intervals on which the curve is concave up as well as concave down.
    \end{mdframed}
    \bigbreak \noindent 
    \begin{remark}
        \frac{d^{2}y}{dx^{2}} = \frac{d}{dx}\left(\frac{dy}{dx}\right) = \frac{\left(\frac{d}{dt}\right)\left(\frac{dy}{dx}\right)}{\frac{dx}{dt}}
    \end{remark}
    
    \bigbreak \noindent 
    We start by finding the first derivative
    \begin{align*}
        \frac{dy}{dx} = \frac{3t^{2}-1}{6t}
    .\end{align*}
    \bigbreak \noindent 
    Now we can find the second derivative
    \begin{align*}
        &\frac{d^{2}y}{dx^{2}} = \frac{\frac{d}{dt}\left(\frac{3t^{2}-1}{6t}\right)}{6t} \\
        &=\frac{\frac{36t^{2}-((3t^{2}-1)6)}{36t^{2}}}{6t} \\
        &=\frac{\frac{36t^{2}-(18t^{2}-6)}{36t^{2}}}{6t} \\
        &=\frac{\frac{18t^{2}+6}{36t^{2}}}{6t} \\
        &=\frac{3t^{2}+1}{36t^{3}}
    .\end{align*}
    \bigbreak \noindent 
    Now we find inflection points by determining where this function equals zero and where the function does not exist. Examining the numerator, we see that there are no real solutions to $3t^{2} + 1 = 0$. So our only inflection point is going to be when the function is undefined. This is when 
    \begin{align*}
        &36t^{3} = 0 \\
        &t = 0
    .\end{align*}
    Testing points to the left and right of $t = 0$, we see that for $(-\infty, 0)$, the curve is concave down. For $(0,\infty)$, the curve is concave up. We conclude that we have an inflection point at $t=0$

    \pagebreak \bigbreak \noindent 
    \begin{mdframed}
        4. Find the exact arc length of the curves below.
        \begin{enumerate}[label=(\alph*)]
            \item $x=\cos{(2t)}$, $y=\sin{2t} $, $0 \leq t \leq \frac{\pi}{2} $
            \item $x= e^{t}\cos{t}$, $y=e^{t}\sin{t}$, $0 \leq t \leq\frac{\pi}{2} $
        \end{enumerate}
    \end{mdframed}
    \bigbreak \noindent 
    \begin{remark}
       The formula for arc len of a parametric curve is given by 
       \begin{align*}
        s = \int_{a}^{b}\ \sqrt{\left(\frac{dx}{dt}\right)^{2} + \left(\frac{dy}{dt}\right)}\ dt
       .\end{align*}
    \end{remark}
    \bigbreak \noindent 
    \textbf{Problem 4a.} Using the forumla we find
    \begin{align*}
       &s = \int_{0}^{\frac{\pi}{2}}\ \sqrt{\left(-2\sin{2t}\right)^{2} + \left(2\cos{2t}\right)^{2}}\ dt  \\
       &=\int_{0}^{\frac{\pi}{2}}\ \sqrt{4\sin^{2}{2t} + 4\cos^{2}{2t}}\ dt\\
       &=2\int_{0}^{\frac{\pi}{2}}\ dt \\
       &=2(\frac{\pi}{2}) \\
       &=\pi
    .\end{align*}

    \bigbreak \noindent 
    \textbf{Problem 4b.} Using the formula we find
    \begin{align*}
        s &= \int_{0}^{\frac{\pi}{2}}\ \sqrt{\left(-e^{t}(\sin{t}-\cos{t})\right)^{2} + \left(e^{t}(\cos{t} + \sin{t})\right)^{2}}\ dt \\
       &=\int_{0}^{\frac{\pi}{2}}\ \sqrt{e^{2t}(\sin{t}-\cos{t})^{2} + e^{2t}(\cos{t}+\sin{t})^{2}}\ dt \\
       &=\int_{0}^{\frac{\pi}{2}}\ \sqrt{e^{2t}\left[(\sin{t}-\cos{t})^{2}+(\cos{t}+\sin{t})^{2}\right]}\ dt \\
       &=\int_{0}^{\frac{\pi}{2}}\ \sqrt{e^{2t}\left[\sin^{2}{t}-2\sin{t}\cos{t}+\cos^{2}{t}+\cos^{2}{t}+2\sin{t}\cos{t}+\sin^{2}{t}\right]}\ dt\\
       &=\int_{0}^{\frac{\pi}{2}}\ \sqrt{e^{2t}(2\sin^{2}{t} + 2\cos^{2}{t})}\ dt \\
       &=\int_{0}^{\frac{\pi}{2}}\ \sqrt{2e^{2t}}\ dt \\
       &=\int_{0}^{\frac{\pi}{2}}\ \sqrt{2}e^{t}\ dt \\
       &=\sqrt{2}\bigg[e^{t}\bigg|^{\frac{\pi}{2}}_0 \\
       &=\sqrt{2}\left(e^{\frac{\pi}{2}} -1\right)
    .\end{align*}
    








    
    


     
 \end{document} 

    
