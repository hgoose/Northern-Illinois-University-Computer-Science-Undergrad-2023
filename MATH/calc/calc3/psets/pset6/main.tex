 \documentclass{report}
 
 \input{~/dev/latex/template/preamble.tex}
 \input{~/dev/latex/template/macros.tex}
 
 \title{\Huge{}}
 \author{\huge{Nathan Warner}}
 \date{\huge{}}
 \fancyhf{}
 \rhead{}
 \fancyhead[R]{\itshape Warner} % Left header: Section name
 \fancyhead[L]{\itshape\leftmark}  % Right header: Page number
 \cfoot{\thepage}
 \renewcommand{\headrulewidth}{0pt} % Optional: Removes the header line
 %\pagestyle{fancy}
 %\fancyhf{}
 %\lhead{Warner \thepage}
 %\rhead{}
 % \lhead{\leftmark}
 %\cfoot{\thepage}
 %\setborder
 % \usepackage[default]{sourcecodepro}
 % \usepackage[T1]{fontenc}
 
 % Change the title
 \hypersetup{
     pdftitle={}
 }

 \geometry{
  left=1.5in,
  right=1.5in,
  top=1in,
  bottom=1in
}
 
 \begin{document}
     % \maketitle
     %     \begin{titlepage}
     %    \begin{center}
     %        \vspace*{1cm}
     % 
     %        \textbf{}
     % 
     %        \vspace{0.5cm}
     %         
     %             
     %        \vspace{1.5cm}
     % 
     %        \textbf{Nathan Warner}
     % 
     %        \vfill
     %             
     %             
     %        \vspace{0.8cm}
     %      
     %        \includegraphics[width=0.4\textwidth]{~/niu/seal.png}
     %             
     %        Computer Science \\
     %        Northern Illinois University\\
     %        United States\\
     %        
     %             
     %    \end{center}
     % \end{titlepage}
     % \tableofcontents
    \pagebreak \bigbreak \noindent
    Nate Warner \ \quad \quad \quad \quad \quad \quad \quad \quad \quad \quad \quad \quad  MATH 232 \quad  \quad \quad \quad \quad \quad \quad \quad \quad \ \ \quad \quad Spring 2024
    \begin{center}
        \textbf{Homework/Worksheet 6 - Due: Wednesday, March 6}
    \end{center}
    \bigbreak \noindent 

    \begin{mdframed}
        1. Let $f(x, y)=e^{x y} \cos x \sin y$. Find $f_x(x, y)$ and $f_y(x, y)$.
    \end{mdframed}
    \bigbreak \noindent 
    By the product rule, treating $y$ as a constant, we find
    \begin{align*}
        f_{x}(x,y) &= \sin{\left(y\right)}\frac{\delta f}{\delta x}e^{xy}\cos{\left(x\right)} \\
        &=\sin{\left(y\right)}(-e^{xy}\sin{\left(x\right)} + ye^{xy}\cos{\left(x\right)})
    .\end{align*}
    \bigbreak \noindent 
    \bigbreak \noindent 
    Treating $x$ as a constant, we find
    \begin{align*}
        f_{y}(x,y) &= \cos{\left(x\right)} \frac{\delta f}{\delta y}e^{xy}\sin{\left(y\right)} \\
        &=\cos{\left(x\right)}(e^{xy}\cos{\left(y\right)} + xe^{xy}\sin{\left(y\right)})
    .\end{align*}

    \begin{mdframed}
        2. Let $f(x, y)=\frac{x y}{x-y}$. Find $f_x(2,-2)$ and $f_y(2,-2)$. Interpret these results as slopes.
    \end{mdframed}
    \bigbreak \noindent 
    First, we find $\frac{\delta }{\delta x}$ and $\frac{\delta }{\delta y}$. To do this, we use the chain rule.
    \begin{align*}
        \frac{\delta }{\delta x} &=  \frac{xy - (y(x-y))}{(x-y)^{2}} = \frac{-y^{2}}{(x-y)^{2}} \\
        \frac{\delta }{\delta y}&= \frac{x(x-y)-(-xy)}{(x-y)^{2}} = \frac{x^{2}}{(x-y)^{2}}
    .\end{align*}
    \bigbreak \noindent 
    Next, we evaluate at the point $P(2,-2)$
    \begin{align*}
        f_{x}(2,-2) &= \frac{-(-2)^{2}}{(2+2)^{2}} = -\frac{1}{4} \\
        f_{y}(2,-2)&=  \frac{2^{2}}{(2+2)^{2}} = \frac{1}{4}
    .\end{align*}
    \bigbreak \noindent 
    This implies that the slope in the x-direction is $-\frac{1}{4}$, while the slope in the y-direction is $\frac{1}{4}$. 

    \bigbreak \noindent 
    \begin{mdframed}
        3. Let $f(x, y)=\ln (x-y)$. Find $f_{x x}(x, y), f_{y y}(x, y)$, and $f_{x y}(x, y)$.
    \end{mdframed}
    \bigbreak \noindent 
    Using properties of differentation we find
    \begin{align*}
        f_{x}(x,y) &= \frac{1}{x+y} \\
        f_{xx}(x,y) &= -\frac{1}{(x+y)^{2}} \\
        f_{y}(x,y) &= \frac{1}{x+y} \\
        f_{yy} &= -\frac{1}{(x + y)^{2}} \\
        f_{xy} &= \frac{-1}{(x + y)^{2}}
  .\end{align*}

    \bigbreak \noindent 
    \begin{mdframed}
        4. Show that $f(x, y)=\ln \left(x^2+y^2\right)$ solves Laplace's equation $\frac{\partial^2 z}{\partial x^2}+\frac{\partial^2 z}{\partial y^2}=0$.
    \end{mdframed}
    \bigbreak \noindent 
    First, we find the partial derivatives
    \begin{align*}
        f_{x}(x,y) &= \frac{2x}{x^{2}+y^{2}} \\
        f_{xx}(x,y) &= \frac{2(x^{2} + y^{2}) - 4x^{2}}{(x^{2}+y^{2})^{2}}  = \frac{-2x^{2}+2y^{2}}{(x^{2}+y^{2})^{2}} \\
        f_{y}(x,y) &= \frac{2y}{x^{2}+y^{2}} \\
                   f_{yy}(x,y)&=\frac{2(x^{2}+y^{2}-4y^{2})}{(x^{2}+y^{2})^{2}} = \frac{-2y^{2}+2x^{2}}{(x^{2}+y^{2})^{2}}
    .\end{align*}
    \bigbreak \noindent 
    With these results, we can compute $f_{xx} + f_{yy}$ and confirm that it equates to zero
    \begin{align*}
        f_{xx} + f_{yy} &= \frac{-2x^{2}+2y^{2}}{(x^{2}+y^{2})^{2}} + \frac{-2y^{2}+2x^{2}}{(x^{2}+y^{2})^{2}} \\
        &=\frac{-2x^{2}+2y^{2}-2y^{2}+2x^{2}}{(x^{2}+y^{2})^{2}} \\
        &=\frac{0}{(x^{2}+y^{2})^{2}} \\
        &=0
    .\end{align*}

    \bigbreak \noindent 
    \begin{mdframed}
        5. Find an equation of the tangent plane to the surface $f(x, y)=\ln \left(10 x^2+2 y^2+1\right)$ at $P(0,0,0)$.
    \end{mdframed}
    \bigbreak \noindent 
    \begin{remark}
        Let $P_0=(x_0,y_0,z_0)$ be a point on a surface $S$, and let $C$ be any curve passing through $P_0$ and lying entirely in $S$. If the tangent lines to all such curves $C$ at $P_0$ lie in the same plane, then this plane is called the tangent plane to $S$ at $P_0$.
        \bigbreak \noindent 
        The equation for a tangent plane at a point is given by
        \begin{align*}
            z = f(x_{0}, y_{0}) + f_{x}(x_{0}, y_{0})(x-x_{0}) + f_{y}(x_{0}, y_{0})(y-y_{0})
        .\end{align*}
    \end{remark}
    \bigbreak \noindent 
    With this, we start by computing $f(0,0)$
    \begin{align*}
        f(0,0) &= \ln{10(0)^{2} + 2(0)^{2} + 1} \\
        &=\ln{1} = 0
    .\end{align*}
    \bigbreak \noindent 
    Next, we find the partial derivatives
    \begin{align*}
        f_{x}(x,y) &= \frac{20x}{10x^{2} + 2y^{2} +1} \\
        f_{y}(x,y) &= \frac{4y}{10x^{2} + 2y^{2}  +1}
    .\end{align*}
    \bigbreak \noindent 
    We now evaluate at our point
    \begin{align*}
        f_{x}(0,0) &= 0 \\
        f_{y}(0,0) &= 0
    .\end{align*}
    \bigbreak \noindent 
    This gives the tangent plane at the point $P(0,0,0)$  as $z=0$. Thus, the plane is flat and parallel to the $xy$-plane
    

    \bigbreak \noindent 
    \begin{mdframed}
        6. Let $f(x, y)=\ln \left(\sqrt{x^2+y^2}\right)$.
        \begin{enumerate}[label=(\alph*)]
            \item Find an equation of the tangent plane to the surface $f(x, y)$ at $(3,4, \ln 5)$.
            \item Find the linearization $L(x, y)$ of the function $f(x, y)$ at $(3,4)$.
            \item Use the linear approximation of $f(x, y)$ at $(3,4)$ to approximate $f(2.99,4.01)$.
        \end{enumerate}
    \end{mdframed}
    \bigbreak \noindent 
    To find the equation of the tangent plane at the point $(3,4,\ln{(5)})$, we first need to find the partial derivatives
    \begin{align*}
        f_{x} &= \frac{1}{(x^{2}+y^{2})^{\frac{1}{2}}} \cdot \frac{1}{2}(x^{2}+y^{2})^{-\frac{1}{2}}\cdot 2x = \frac{2x}{2(x^{2}+y^{2})} \\
        f_{y}&=\frac{2y}{2(x^{2}+y^{2})}
    .\end{align*}
    \bigbreak \noindent 
    Next, we evaluate at the point $(3,4)$
    \begin{align*}
        f_{x}(3,4) = \frac{2(3)}{2(3^{2}+4^{2})} = \frac{6}{50} = \frac{3}{25} \\
        f_{y}(3,4) = \frac{2(4)}{50} = \frac{4}{25}
    .\end{align*}
    \bigbreak \noindent 
    Now, by the equation of a tangent plane, we have
    \begin{align*}
        z = \ln{(5)} + \frac{3}{25}(x-3)+\frac{4}{25}(y-4)
    .\end{align*}
    \bigbreak \noindent 
    The equation above is also the linearization $L(x,y)$ at $(3,4)$. We can now use it to approximate $f(2.99,4.01)$, we have
    \begin{align*}
        L(2.99,4.01) &= \ln{(5)} + \frac{3}{25}(2.99-3) + \frac{4}{25}(4.01-4) \\
        &\approx 1.60983
    .\end{align*}

    \bigbreak \noindent 
    \begin{mdframed}
        7. Find the linear approximation of the function $f(x, y)=e^x \cos y$ at $P(0,0)$ and use it to approximate $f(0.01,-0.02)$.
    \end{mdframed}
    \bigbreak \noindent 
    First, we find the partial derivatives and evaluate them at the point $P(0,0)$
    \begin{align*}
        p_{x} &= e^{x}\cos{\left(y\right)} \\
        p_{x}(0,0) = e^{0}\cos{\left(0\right)} = 1 \\
        p_{y} &= -e^{x}\sin{\left(y\right)}
        p_{y}(0,0) = -e^{0}\sin{\left(0\right)} =  0
    .\end{align*}
    \bigbreak \noindent 
    Now, the linearization is given by
    \begin{align*}
        L(x,y) &= 1+ 1(x-0) + 0(y-0) \\
        &=1 + x
    .\end{align*}
    \bigbreak \noindent 
    We then use $L(x,y)$ to approximate $f(0.01, -0.02)$
    \begin{align*}
        L(0.01, -0.02) &= 1 + 0.01 \\
        &=1.01
    .\end{align*}


    \bigbreak \noindent 
    \begin{mdframed}
        8. Let $f(x, y)=x^4, x=t, y=t$. Use the chain rule to find $d f / d t$.
    \end{mdframed}
    \bigbreak \noindent 
    \begin{remark}
        Suppose that \(x = g(t)\) and \(y = h(t)\) are differentiable functions of \(t\) and \(z = f(x,y)\) is a differentiable function of \(x\) and \(y\). Then \(z = f(x(t), y(t))\) is a differentiable function of \(t\) and
        \[
            \frac{dz}{dt} = \frac{\partial z}{\partial x} \cdot \frac{dx}{dt} + \frac{\partial z}{\partial y} \cdot \frac{dy}{dt},
        \]
        where the ordinary derivatives are evaluated at \(t\) and the partial derivatives are evaluated at \((x,y)\).
    \end{remark}
    \bigbreak \noindent 
    By this, we find
    \begin{align*}
        \frac{df}{dt} &= 4x^{3} \\
                      &= 4t^{3}
    .\end{align*}
    \not\in
    \begin{align*}
        \int_{0}^{k}\ x^{2}+2x\ dx =  0 \not\in k\neq 0 
    .\end{align*}
    

    \bigbreak \noindent 
    \begin{mdframed}
        9. Let $z=e^{x^2 y}$, where $x=\sqrt{u v}$ and $y=1 / v$. Find $\frac{\partial z}{\partial u}$ and $\frac{\partial z}{\partial v}$.
    \end{mdframed}
    \bigbreak \noindent 
    After constructing a tree diagram, we find
    \begin{align*}
        \frac{\delta z}{\delta u} &= \frac{\delta z}{\delta x} \cdot \frac{\delta x}{\delta u}  \\
        \frac{\delta z}{\delta v} &= \frac{\delta z}{\delta x} \cdot \frac{\delta x}{\delta v} + \frac{\delta z}{\delta y}\cdot \frac{\delta y}{\delta v}
    .\end{align*}
    Thus,
    \begin{align*}
        &\frac{\delta z}{\delta u} = 2xye^{x^{2}y} \cdot \frac{1}{2}(uv)^{-\frac{1}{2}}(v) = \frac{2xye^{x^{2}y}v}{2(uv)^{\frac{1}{2}}} = \frac{2(uv)^{\frac{1}{2}}v^{-1}e^{(uv)^{\frac{1}{2}}^{2}v^{-1}v}}{2(uv)^{\frac{1}{2}}} = e^{u} \\
        &\frac{\delta z}{\delta v} = 2xye^{x^{2}y} \cdot \frac{1}{2}(uv)^{-\frac{1}{2}}(u) + x^{2}e^{x^{2}y}\cdot -\frac{1}{v^{2}} = \frac{2xye^{x^{2}y}u}{2(uv)^{\frac{1}{2}}} - \frac{x^{2}e^{x^{2}y}}{v^{2}} \\
        &=\frac{(uv)^{1/2}\cdot \frac{1}{v}e^{uv\cdot \frac{1}{v}}u}{(uv)^{\frac{1}{2}}} - \frac{uve^{uv\cdot \frac{1}{v}}}{v^{2}} \\
        &=\frac{ue^{u}}{v}-\frac{ue^{v}}{v} \\
        &=0
    .\end{align*}



 \end{document} 
