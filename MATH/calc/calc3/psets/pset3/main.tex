 \documentclass{report}
 
 \input{~/dev/latex/template/preamble.tex}
 \input{~/dev/latex/template/macros.tex}
 
 \title{\Huge{}}
 \author{\huge{Nathan Warner}}
 \date{\huge{}}
 \fancyhf{}
 \rhead{}
 \fancyhead[R]{\itshape Warner} % Left header: Section name
 \fancyhead[L]{\itshape\leftmark}  % Right header: Page number
 \cfoot{\thepage}
 \renewcommand{\headrulewidth}{0pt} % Optional: Removes the header line
 %\pagestyle{fancy}
 %\fancyhf{}
 %\lhead{Warner \thepage}
 %\rhead{}
 % \lhead{\leftmark}
 %\cfoot{\thepage}
 %\setborder
 % \usepackage[default]{sourcecodepro}
 % \usepackage[T1]{fontenc}
 
 % Change the title
 \hypersetup{
     pdftitle={}
 }

 \geometry{
  left=1.5in,
  right=1.5in,
  top=1in,
  bottom=1in
}
 
 \begin{document}
     % \maketitle
     %     \begin{titlepage}
     %    \begin{center}
     %        \vspace*{1cm}
     % 
     %        \textbf{}
     % 
     %        \vspace{0.5cm}
     %         
     %             
     %        \vspace{1.5cm}
     % 
     %        \textbf{Nathan Warner}
     % 
     %        \vfill
     %             
     %             
     %        \vspace{0.8cm}
     %      
     %        \includegraphics[width=0.4\textwidth]{~/niu/seal.png}
     %             
     %        Computer Science \\
     %        Northern Illinois University\\
     %        United States\\
     %        
     %             
     %    \end{center}
     % \end{titlepage}
     % \tableofcontents
    \pagebreak \bigbreak \noindent
    Nate Warner \ \quad \quad \quad \quad \quad \quad \quad \quad \quad \quad \quad \quad  MATH 232 \quad  \quad \quad \quad \quad \quad \quad \quad \quad \ \ \quad \quad Spring 2024
    \begin{center}
        \textbf{Homework/Worksheet 3 - Due: Wednesday, February 7}
    \end{center}
    \bigbreak \noindent 
    \begin{mdframed}
        1. A football thrown by a quarterback has an initial speed of 70 mph and an angle of elevation of $60^{\circ}$. Determine the velocity vector in mph and express it in component form. (Round to two decimal places.)
    \end{mdframed}
    \bigbreak \noindent 
    First, lets construct a figure 
    \bigbreak \noindent 
    \begin{figure}[ht]
        \centering
        \incfig{ink2}
        \label{fig:ink2}
    \end{figure}
    \bigbreak \noindent 
    Since we know $\norm{\vec{V}} = 70mph$, we can use properties of right triangles to find $\vec{V}_{x}$ and $\vec{V}_{y}$
    \begin{align*}
        &\cos{\theta } = \frac{\text{opp}}{\text{hyp}} \implies \text{hyp}\cos{\theta} = \text{opp} \\
        &\sin{\theta } = \frac{\text{adj}}{\text{hyp}} \implies \text{hyp}\sin{\theta } = \text{adj}
    .\end{align*}
    \bigbreak \noindent 
    With these findings, we can find the components of our velocity vector.
    \begin{align*}
        &\vec{V}_{x} = \norm{\vec{V}}\cos{\theta } = 70mph\cos{60^{\circ}} = 35mph \\
        &\vec{V}_{y} = \norm{\vec{V}}\sin{\theta } = 70mph\sin{60^{\circ}} = 60.62mph
    .\end{align*}
    \bigbreak \noindent 
    \textbf{Conclusion.} Thus, the components for the velocity vector are
    \begin{align*}
        \vec{V} = (35\hat{i} + 60.62\hat{j})\ \text{mph}
    .\end{align*}
    Where $\hat{i}$ is the unit vector along the positive x-axis, and $\hat{j}$ is the unit vector along the positive y-axis.

    \pagebreak \bigbreak \noindent 
    \begin{mdframed}
        2. Let $\mathbf{u} = \langle 1, 1, 0 \rangle$, $\mathbf{v} = \langle 0, 1, -1 \rangle$. Find the magnitude of the vectors $\mathbf{u} - \mathbf{v}$ and $-2\mathbf{v}$.
    \end{mdframed}
    \bigbreak \noindent 
    To find $\vec{u}$  - $\vec{v}$, we simply subtract their components.
    \begin{align*}
        &\vec{u} - \vec{v} = \langle 1-0, 1-1, 0-(-1)\rangle \\
        &=\langle 1,0,1\rangle
    .\end{align*}
    From this, we can find $\norm{\vec{u} - \vec{v}}$
    \begin{align*}
        &\norm{\vec{u} - \vec{v}} = \sqrt{1^{2} + 0^{2} + 1^{2}} \\
        &=\sqrt{2}
    .\end{align*}
    \bigbreak \noindent 
    Next, we find the vector corresponding to $-2\vec{v}$
    \bigbreak \noindent 
    By this, we have 
    \begin{align*}
        -3\vec{v} &= \left\langle -3(0), -3(1), -3(-1) \right\rangle  \\
       &=\left\langle 0,-3,3 \right\rangle \\
       &\implies \norm{\left\langle 0,-3,3 \right\rangle} = \sqrt{0^{2} + (-3)^{2} + 3^{2}} \\
       &=\sqrt{18} = 3\sqrt{2}
    .\end{align*}

    \bigbreak \noindent 
    \begin{mdframed}
        3. Find a vector $\vec{u}$ in the same direction of the vector $\vec{v} = \left\langle 2,4,1 \right\rangle$, whose magnitude is 15, that
        is, $\norm{\vec{u}}$= 15
    \end{mdframed}
    \bigbreak \noindent 
    First, we find $\norm{\vec{v}}$
    \begin{align*}
        \norm{\vec{v}} &= \sqrt{2^{2} + 4^{2} + 1^{2}} \\
        &=\sqrt{21}
    .\end{align*}
    \bigbreak \noindent 
    Next, we find $\hat{u}$ in the direction of $\vec{v}$
    \begin{align*}
        \hat{u} &= \frac{1}{\norm{\vec{v}}}\vec{v} \\
        &=\frac{1}{\sqrt{21}}\left\langle 2,4,1 \right\rangle \\
        &=\left\langle  \frac{2}{\sqrt{21}}, \frac{4}{\sqrt{21}}, \frac{1}{\sqrt{21}} \right\rangle
    .\end{align*}
    \bigbreak \noindent 
    This gives us a vector $\hat{u}$ in the direction of $\vec{v}$ with magnitude 1. When then manipulate $\hat{u}$ s.t the magnitude becomes 15. To do this, we multiply by a scalar of 15.
    \begin{align*}
        15\hat{u} &= 15 \left\langle \frac{2}{\sqrt{21}}, \frac{4}{\sqrt{21}}, \frac{1}{\sqrt{21}} \right\rangle \\
        &= \left\langle \frac{30}{\sqrt{21}}, \frac{60}{\sqrt{21}}, \frac{15}{\sqrt{21}} \right\rangle
    .\end{align*}
    \bigbreak \noindent 
    \textbf{Conclusion.} The vector $\vec{u}$ in the direction of $\vec{v}$ with magnitude $15$ is
    \begin{align*}
        \left\langle \frac{30}{\sqrt{21}}, \frac{60}{\sqrt{21}}, \frac{15}{\sqrt{21}} \right\rangle
    .\end{align*}
    \bigbreak \noindent 
    \begin{mdframed}
        4. Find the angle between the vectors $\vec{a} = \left\langle 0,-1,-3\right\rangle $ and $\vec{b} = \left\langle 2,3,-1 \right\rangle $
    \end{mdframed}
    \bigbreak \noindent 
    For this, we use the formula 
    \begin{align*}
        \cos{\theta } = \frac{\vec{u} \cdot \vec{v}}{\norm{\vec{u}}\norm{\vec{v}}}
    .\end{align*}
    \bigbreak \noindent 
    Which gives
    \begin{align*}
        \cos{\theta} &= \frac{\vec{a} \cdot \vec{b}}{\norm{\vec{a}}\norm{\vec{b}}} \\
                     \cos{\theta } &= \frac{-3 + 3}{\norm{\vec{a}}\norm{\vec{b}}} \\
                     \theta &= \cos^{-1}{0} \\
                            &=\frac{\pi}{2} = 90^{\circ}
    .\end{align*}
    \textbf{Conclusion.} Thus, the angle between these two vectors is $\frac{\pi}{2} = 90^{\circ}$
    \bigbreak \noindent 
    \begin{mdframed}
        5. Let $\vec{u} = \left\langle 2,4,0 \right\rangle $ and $\vec{v}  = \left\langle 0,4,2 \right\rangle$
        \begin{enumerate}[label=(\alph*)]
            \item Find the component form of vector $w=\text{proj}_{\vec{u}}\vec{v}$ that represents the projection of $\vec{v}$ onto $\vec{u}$.
            \item Write the decomposition $\vec{v} = \vec{w} + \vec{q}$, where $\vec{w}$ is the projection of $\vec{v}$ onto $\vec{u}$ and $\vec{q}$ is a vector orthogonal to the direction of $\vec{u}$
        \end{enumerate}
    \end{mdframed}
    \bigbreak \noindent 
    First, we find $\vec{w} = \text{proj}_{\vec{u}}\vec{v} $
    \begin{align*}
        \vec{w} &= \text{proj}_{\vec{u}} \vec{v} = \frac{\vec{u} \cdot \vec{v}}{\norm{\vec{u}}^{2}}\vec{u} \\
        &=\frac{16}{20}\left\langle 2,4,0 \right\rangle \\
        &=\left\langle \frac{32}{20}, \frac{64}{20}, 0 \right\rangle \\
        &= \left\langle \frac{8}{5}, \frac{16}{5},0 \right\rangle
    .\end{align*}
    \bigbreak \noindent 
    Next, we define $\vec{q} = \vec{v} - \vec{w}$. Where $\vec{q}$ is the vector orthogonal to the direction of $\vec{u}$
    \begin{align*}
        \vec{q} &= \left\langle 0-\frac{8}{5}, 4-\frac{16}{5}, 2-0 \right\rangle \\
        &=\left\langle -\frac{8}{5}, \frac{4}{5}, 2 \right\rangle
    .\end{align*}
    \bigbreak \noindent 
    \textbf{Conclusion.} The projection of $\vec{v}$ onto $\vec{u}$ is given by $\vec{w} = \left\langle \frac{8}{5}, \frac{16}{5}, 0\right\rangle $, and the vector orthogonal to the direction of $\vec{u}$ is given by $\vec{q} = \left\langle -\frac{8}{4}, \frac{4}{5}, 2 \right\rangle $

    \pagebreak \bigbreak \noindent 
    \begin{mdframed}
        6. Let $A(2, -3, 4)$, $B(0, 1, 2)$, $C(-1, 2, 0)$
        \begin{enumerate}[label=(\alph*)]
            \item Find a vector orthogonal to both $\vec{u} = \vec{AB}$ and $\vec{v} = \vec{AC} $
            \item Find the area of parallelogram $ABCD$ with adjacent sides $\vec{AB}$ and $\vec{AC} $
            \item Find the area of the triangle $ABC$.
        \end{enumerate}
    \end{mdframed}
    \bigbreak \noindent 
    First, we find the vectors $\vec{u} = \vec{AB}$ and $\vec{v} = \vec{AC}$
    \begin{align*}
        \vec{u} &= \left\langle 0-2,1-(-3),2-4 \right\rangle \\
        &= \left\langle -2, 4, -2  \right\rangle \\
        \vec{v} &= \left\langle -1-2,2-(-3),0-4 \right\rangle \\
                &= \left\langle -3,5,-4 \right\rangle
    .\end{align*}
    \bigbreak \noindent 
    To find a vector orthogonal to both $\vec{u}$ and $\vec{v}$, we compute the cross product $\vec{u} \times \vec{v}$
    \begin{align*}
        \vec{u} \times \vec{v} &= \left\langle u_{y}v_{z} - u_{z}v_{y}, u_{x}v_{z}-u_{z}v_{x}, u_{x}v_{y} - u_{y}v_{x} \right\rangle \\
        &= \left\langle 4(-4)-(-2)(5), (-2)(-4) - (-2)(-3), (-2)(5)-4(-3) \right\rangle \\
        &= \left\langle -16+10, 8-6, -10+12 \right\rangle \\
        &=\left\langle -6,2,2 \right\rangle
    .\end{align*}
    \bigbreak \noindent 
    Next, we find the area of parallelogram $ABCD$ with adjacent sides $\vec{AB}$ and $\vec{AC} $ by finding the magnitude of the cross product.
    \begin{align*}
        \norm{\vec{u} \times \vec{v}} = \sqrt{(-6)^{2} + 2^{2} + 2^{2}} \\
        &=\sqrt{44}
    .\end{align*}
    \bigbreak \noindent 
    Finally, the area of a triangle formed by the two vectors is half the area of the parallelogram formed by the same vectors. Thus we have
    \begin{align*}
        \frac{1}{2}\sqrt{44} = \frac{\sqrt{44}}{2}
    .\end{align*}
    
    



     
 \end{document} 
