\documentclass{report}

\input{~/dev/latex/template/preamble.tex}
\input{~/dev/latex/template/macros.tex}

\usepackage{adjustbox}

\title{\Huge{Math 170: Quiz 3}}
\author{\huge{Nathan Warner}}
\date{\huge{April 3, 2023}}
\pagestyle{fancy}
\fancyhf{}
\rhead{QUIZ SOLUTIONS}
\lhead{\leftmark}
\cfoot{\thepage}
% \usepackage[default]{sourcecodepro}
% \usepackage[T1]{fontenc}

\pgfpagesdeclarelayout{boxed}
{
  \edef\pgfpageoptionborder{0pt}
}
{
  \pgfpagesphysicalpageoptions
  {%
    logical pages=1,%
  }
  \pgfpageslogicalpageoptions{1}
  {
    border code=\pgfsetlinewidth{1.5pt}\pgfstroke,%
    border shrink=\pgfpageoptionborder,%
    resized width=.95\pgfphysicalwidth,%
    resized height=.95\pgfphysicalheight,%
    center=\pgfpoint{.5\pgfphysicalwidth}{.5\pgfphysicalheight}%
  }%
}

\pgfpagesuselayout{boxed}

\begin{document}
    \maketitle
    \bigbreak \noindent 
    \begin{mdframed}
      \textbf{1. Using the graph of the function f , find the following. Circle your final answers.}
      \bigbreak \noindent 
      \hspace{\parindent} \textbf{1.a:} Local Min: \circled{$f(2) = 3$}
      \bigbreak \noindent 
      \hspace{\parindent} \textbf{1.b:} Local Max \circled{$f(4) = 5$}
      \bigbreak \noindent 
      \hspace{\parindent} \textbf{1.c:} Absolute Min: \circled{None}
      \bigbreak \noindent 
      \hspace{\parindent} \textbf{1.d:} Absolute Max: \circled{$f(0) = 5 $ and $f(4) = 5 $}
    \end{mdframed}

    \bigbreak \noindent 
    \begin{mdframed}
      \textbf{3. The height in feet of a given body about the surface of the earth at time t seconds is given.}
      \begin{align*}
        y = 1000 + 160t -16t^{2},\ 3\leq t \leq 6
      .\end{align*}
      \bigbreak \noindent 
      \textbf{Find:}
      \bigbreak \noindent 
      \textbf{a.) The maximum and minimum height of the body}
      \bigbreak \noindent 
      \textbf{b.) The maximum and minimum velocity v of the body, and}
      \bigbreak \noindent 
      \textbf{c.) The maximum and minimum speed s of the body}
      \bigbreak \noindent 
      \textbf{During the given time interval.}

      \bigbreak \noindent 
      \textit{Closed Interval: [3,6]}

      \bigbreak \noindent 
      \hspace{\parindent} \textbf{3.a: }
      \bigbreak \noindent 
      \textit{$y^{\prime}$}:
      \begin{align*}
        y^{\prime} = 160t-32t
      .\end{align*}

      \bigbreak \noindent 
      \textit{Set $y^{\prime}=0$}
      \begin{align*}
        160-32t = 0 \\
        -32t = -160 \\
        t = \frac{-160}{-32} \\
        t = 5 
      .\end{align*}
      \begin{center}
        Since $5 \in [3,6]$, 5 is a critical value
      \end{center}

      \bigbreak \noindent 
      \textit{$y^{\prime}$ DNE}
      \begin{center}
        There are no values of t that make this function undefined
      \end{center}

      \bigbreak \noindent 
      \textit{Plug critical values into y:}
      \begin{align*}
        y(5) = 1000 + 160(5) -16(5)^{2} \\
        \boxed{=1400}
      .\end{align*}

      \bigbreak \noindent 
      \textit{Find $y(a)$ and $y(b)$:}
      \begin{align*}
        y(3) = 1000+160(3)-16(3)^{2} \\
        \boxed{=1336}
      .\end{align*}
      \begin{align*}
        y(6) = 1000+160(6)-16(6)^{2} \\
        \boxed{=1384}
      .\end{align*}

      \bigbreak \noindent 
      \textbf{\textit{\underline{Therefore:}}}
      \begin{align*}
        \circled{Max\ Height:\ f(5) = 1400\ ft}\ \ \  
        \circled{Min\ Height:\ f(3) = 1336\ ft}
      .\end{align*}

      \bigbreak \noindent 
      \hspace{\parindent} \textbf{3.b:}
      \bigbreak \noindent 
      \textit{Find v(t):}
      \begin{align*}
        v(t) = 160-32t
      .\end{align*}

      \bigbreak \noindent 
      \textit{$v^{\prime}(t)$}:
      \begin{align*}
        v^{\prime}(t) = -32
      .\end{align*}

      \bigbreak \noindent 
      \textit{Set $v^{\prime}(t) =0 $}:
      \begin{align*}
        -32 = 0
      .\end{align*}
          \begin{center}
      Since $-32 \notin [3,6]$, -32 is not a critical value
    \end{center}

    \bigbreak \noindent 
    \textit{Find $v(a)\ and\ v(b)$}:
    \begin{align*}
      v(3) = 160-32(3) \\
      \boxed{=64\ ft \diagdown s}
    .\end{align*}
    \begin{align*}
      v(6) = 160-32(6) \\
      \boxed{=-32\ ft \diagdown s}
    .\end{align*}

    \bigbreak \noindent 
    \textbf{\textit{\underline{Therefore:}}}
    \bigbreak \noindent 
    \begin{center}
      \circled{Max\ Velocity:\ v(3) = 64\ ft$\diagdown$ s}\ \ \ 
      \circled{Min\ Velocity:\ v(6) = -32\ ft$\diagdown$s}
    \end{center}

    \bigbreak \noindent 
    \hspace{\parindent} \textbf{3.c:}
    \bigbreak \noindent 
    \textit{Since $ s(t) = |v(t)|$, we can take the absolute value of our values from 3.b and retrieve that maximum and minimum speeds:}
    \begin{align*}
      s(3) = \abs{160-32(3)} \\
      \boxed{=64\ ft \diagdown s}
    .\end{align*}
    \begin{align*}
      s(6) = \abs{160 -32(6)} \\
      \boxed{=32\ ft \diagdown s}
    .\end{align*}

    \bigbreak \noindent 
    \textbf{\textit{\underline{Therefore:}}}
    \begin{center}
      \circled{Max Speed: s(3) = 64 $ft \diagdown s$}\ \ \ 
      \circled{Min Speed: s(6) = 32 $ft \diagdown s$}
    \end{center}
    \end{mdframed}

    \pagebreak \bigbreak \noindent
    \begin{mdframed}
      \textbf{4. Find the limit using l’Hospital’s Rule, if the rule is appropriate.}
      \bigbreak \noindent 
      \hspace{\parindent} \textbf{4.a: }
      \begin{align*}
        \lim_{x \to 1}{\frac{3x^{2}+2x-5}{x^{4}+3x^{2}-4}}
      .\end{align*}
      \begin{align*}
        \lim_{x \to 1}{3x^{2}+2x-5} = 0\ \ and\ \ \lim_{x \to 1}{x^{4}+3x^{2}-4} =0
      .\end{align*}
      \bigbreak \noindent 
      \textit{Since we have an indeterminate form of the type $\frac{0}{0} $}, we can use L'Hospital's Rule to evaluate the limit.
      \bigbreak \noindent
      \textit{So:}
      \begin{align*}
        L'H = \lim_{x \to 1}{\frac{6x+2}{4x^{3}+6x}} \\
        = \frac{8}{10} \\
      \end{align*}
      \begin{center}
        \begin{large}
          \circled{=$\frac{4}{5}$}
        \end{large}
      \end{center}

      \bigbreak \noindent 
      \hspace{\parindent} \textbf{4.b:}
      \begin{align*}
        \lim_{x \to \infty}{\frac{e^{2+\ln{x}}}{3x+4}}
      .\end{align*}
      \begin{align*}
        \lim_{x \to \infty}{e^{2+\ln{x}}} = \infty\ and\ \lim_{x \to \infty}{3x+4} = \infty
      .\end{align*}
      \bigbreak \noindent 
      \textit{Since we have an indeterminate form of the type $\frac{\infty}{\infty}$}, We must use L'Hospital's Rule.
      \begin{align*}
        L'H = \lim_{x \to \infty}{\frac{e^{2+\ln{x}}\cdot \frac{1}{x}}{3}} \\
        L'H = \lim_{x \to \infty}{\frac{\frac{e^{2+\ln{x}}}{x}}{3}} \\
        L'H = \lim_{x \to \infty}{\frac{e^{2+\ln{x}}}{3x}} \\
      .\end{align*}

      \bigbreak \noindent 
      \textit{At it's current state, we are stuck in a loop of applying L'Hospital's Rule to no avail, therefore we must try and simpliy the equation}
      \begin{align*}
        L'H = \lim_{x \to \infty}{\frac{e^{2+\ln{x}}}{3x}} \\
        L'H = \lim_{x \to \infty}{\frac{e^{2}\cdot e^{\ln{x}}}{3x}} \\
        L'H = \lim_{x \to \infty}{\frac{e^{2}\cdot x}{3x}} \\
        L'H = \lim_{x \to \infty}{\frac{e^{2}}{3}} \\
      .\end{align*}
      \begin{center}
        \begin{Large}
          \textbf{}
          \circled{= $\frac{e^{2}}{3} $}
        \end{Large}
      \end{center}

      \bigbreak \noindent 
      \hspace{\parindent} \textbf{4.c:}
      \begin{align*}
        \lim_{x \to \pi}{\frac{\sin^{2}{x}}{(x-\pi)^{2}}}
      .\end{align*}
      \begin{align*}
        \lim_{x \to \pi}{\sin^{2}{x}} = 0\ and\ \lim_{x \to \pi}{(x-\pi)^{2}} = 0
      .\end{align*}
      \bigbreak \noindent 
      \textit{Since we have an indeterminate form of the type $\frac{0}{0}$}, we can use L'Hospital's Rule
      \begin{align*}
        L'H= \lim_{x \to \pi}{\frac{2\sin{x}\cos{x}}{2(x-\pi)(1)}} \\
        L'H = \lim_{x \to \pi}{\frac{2\sin{x}\cos{x}}{2x-2\pi)}} \\
      .\end{align*}
      \begin{align*}
        \lim_{x \to \pi}{2\sin{(x)}\cos{(x)}} = 0\ and\ \lim_{x \to \pi}{2x-2\pi} = 0
      .\end{align*}
      \bigbreak \noindent 
      \textit{We still have the indeterminate form of the type $\frac{0}{0}$}, so once again, we must use L'Hospital's Rule
      \begin{align*}
        L'H = \lim_{x \to \pi}{\frac{2[(\sin{x})(-\sin{x}) + (\cos{x})(\cos{x})]}{2}} \\
        L'H = \lim_{x \to \pi}{\frac{2[-\sin^{2}{x}+\cos^{2}{x}]}{2}} \\
        L'H = \lim_{x \to \pi}{\frac{-2\sin^{2}{x}+2\cos^{2}{x}}{2}}
      .\end{align*}

      \bigbreak \noindent 
      \textit{From here if we evaluate:}
      \begin{align*}
        \lim_{x \to \pi}{-2\sin^{2}{x}+\cos^{2}{x}}
      .\end{align*}

      \bigbreak \noindent 
      \textit{We get:}
      \begin{align*}
        -2\sin^{2}{\pi} + 2\cos^{2}{\pi} \\
        = -2(0)^{2} + 2(-1)^{2} \\
        = 2
      .\end{align*}

      \bigbreak \noindent 
      \textit{Therefore we have}
      \begin{align*}
        \frac{2}{2} \\
      .\end{align*}
      \begin{center}
        \begin{Large}
          \circled{1}
        \end{Large}
      \end{center}
    \end{mdframed}

\end{document}
