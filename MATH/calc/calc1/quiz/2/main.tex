\documentclass{report}

\input{~/dev/latex/template/preamble.tex}
\input{~/dev/latex/template/macros.tex}

\title{\Huge{Quiz 2 Solutions}}
\author{\huge{Nathan Warner}}
\date{\huge{March 1, 2023}}
\pagestyle{fancy}
\fancyhf{}
\rhead{QUIZ NO.2}
\lhead{\leftmark}
\cfoot{\thepage}

\pgfpagesdeclarelayout{boxed}
{
  \edef\pgfpageoptionborder{0pt}
}
{
  \pgfpagesphysicalpageoptions
  {%
    logical pages=1,%
  }
  \pgfpageslogicalpageoptions{1}
  {
    border code=\pgfsetlinewidth{1.5pt}\pgfstroke,%
    border shrink=\pgfpageoptionborder,%
    resized width=.95\pgfphysicalwidth,%
    resized height=.95\pgfphysicalheight,%
    center=\pgfpoint{.5\pgfphysicalwidth}{.5\pgfphysicalheight}%
  }%
}

\pgfpagesuselayout{boxed}

\begin{document}
    \maketitle
    \noindent
    \begin{mdframed}
        \textbf{1.a: Differentiate \textbf{\textit{\underline{without}}} using the product or quotient rule}
        \begin{align*}
            f(x) = \frac{x^{3}-3x}{2x^{2}}
        .\end{align*}
    \bigbreak \noindent 
    \pf{Solution: In order to bypass using the quotient rule we will simplify the function}{}
    \bigbreak \noindent
    \textit{So:}
    \begin{align*}
        \frac{x^{3}- 3x}{2x^{2}} \longrightarrow \frac{x^{3}}{2x^{2}} - \frac{3x}{2x^{2}} \\
        = \frac{x}{2} - \frac{3}{2x} \\ 
        = \frac{1}{2}x - \frac{3}{2}x^{-1}
    \end{align*}
    \bigbreak \noindent
    \textit{Now this is the function we will differentiate:}
    \begin{align*}
        f^{\prime}(x) = \frac{1}{2}(1) - (-1)\frac{3}{2}x^{-1-1}  \\
        = \frac{1}{2} + \frac{3}{2}x^{-2} \\
        = \frac{1}{2}+\frac{3}{2x^{2}}
    \end{align*}
    \textit{Now we find a common denominator:} 
    \begin{align*}
        f^{\prime}(x) = \frac{3}{2x^{2}} + \frac{1}{2}(\frac{x^{2}}{x^{2}}) \\
        =\frac{3}{2x^{2}} + \frac{x^{2}}{2x^{2}} \\
        = \frac{3+x^{2}}{2x^{2}}
    .\end{align*}
    \bigbreak \noindent 
    \textit{Therefore our answer is:}
    \begin{align*}
      \boxed{f^{\prime}(x) = \frac{3+x^{2}}{2x^{2}}}
    .\end{align*}
    \end{mdframed}

    \pagebreak \bigbreak \noindent
    \begin{mdframed}
        \textbf{1.b}
        \begin{align*}
            G(x) = \frac{3x}{x^{2}+1}
        .\end{align*}
        \bigbreak \noindent 
        \pf{Solution: To find the deriviative we will use the quotient rule:}{}
        \bigbreak \noindent 
        \textit{If:}
        \begin{align*}
            \frac{d}{dx}\bigg[ \frac{f(x)}{g(x)}\bigg] = \frac{g(x) \frac{d}{dx}[f(x)] - f(x) \frac{d}{dx}[g(x)]}{[g(x)]^2}
        .\end{align*}
        \bigbreak \noindent 
        \textit{And:}
        \begin{align*}
            f(x) = 3x \\
            f^{\prime}(x) = 3
        .\end{align*}
        \begin{align*}
            g(x) = x^{2} + 1 \\
            g^{\prime}(x) = 2x
        .\end{align*}
        \bigbreak \noindent 
        \textit{Then:}
        \begin{align*}
            G^{\prime}(x) = \frac{(x^{2}+1)(3) - (3x)(2x)}{(x^{2}+1)^{2}} \\
            =\frac{(x^{2}+1)(3)-6x^{2}}{(x^{2}+1)^{2}} \\ 
            = \frac{3x^{2}+3-6x^{2}}{(x^{2}+1)^{2}} \\
            = \boxed{\frac{-3x^{2}+3}{(x^{2}+1)^{2}}}
        .\end{align*}
    \end{mdframed}

    \bigbreak \noindent 
    \begin{mdframed}
        \textbf{1.c}
        \begin{align*}
            f(x) = -x + \tan{x}
        .\end{align*}
        \pf{Solution: Derive}{}
        \begin{align*}
            f^{\prime}(x) = -1 + \sec^{2}{x} \\ 
            = \sec^{2}{x}  - 1
        .\end{align*}
        \bigbreak \noindent 
        \textit{By the pathagorean identity:}
        \begin{align*}
            \sec^{2}{x} - 1 = \tan^{2}{x}
        .\end{align*}
        \textit{Then:}
        \begin{align*}
          \boxed{f^{\prime}(x) = \tan^{2}{x}}
        .\end{align*}
    \end{mdframed}

    \pagebreak \bigbreak \noindent
    \begin{mdframed}
        \textbf{1.d}
        \begin{align*}
            g(t) = (1+\sin{t})^{3}
        .\end{align*}
        \bigbreak \noindent 
        \pf{Solution: Using the chain rule we can differentiate:}{}
        \begin{align*}
            g^{\prime}(t) = 3(1+\sin{t})^{2} \cdot (0+\cos{t}) \\
            = \boxed{3\cos{t}(1+\sin{t})^{2}}
        .\end{align*}
    \end{mdframed}

    \bigbreak \noindent 
    \begin{mdframed}
        \textbf{Question 2: Find $\frac{dy}{dx}$}:
        \begin{align*}
            2x^{2}+xy+3y^{2} = 0
        .\end{align*}
        \bigbreak \noindent 
        \pf{Solution: By using implicit differentiate we can solve:}{}
        \begin{align*}
            4x+x\frac{dy}{dx} + y + 6y \frac{dy}{dx} = 0 \\ 
            = x \frac{dy}{dx} + 6y \frac{dy}{dx} = -4x-y \\
            = \frac{dy}{dx}(x+6y) = -4x -y \\
            = \frac{dy}{dx} = \frac{-4x-y}{x+6y} \\
            = \frac{dy}{dx} = \frac{-(4x+y)}{x+6y} \\
            = \boxed{\frac{dy}{dx} = -\frac{4x+y}{x+6y}} \\
        .\end{align*}
    \end{mdframed}

    \pagebreak \bigbreak \noindent
    \begin{mdframed}
        \textbf{Question 3: Use logarithmic differentation to find the deriviative:}
        \begin{align*}
            y = \frac{\sqrt{(x^{2}+1)^{3}}}{\sqrt[3]{(x^{3}+1)^{7}}}
        .\end{align*}
        \bigbreak \noindent 
        \pf{Solution: \textit{We will begin by rewriting the equation}}{}
        \begin{align*}
            y = \frac{\big[(x^{2}+1)^{3}\big]^{\frac{1}{2}}}{\big[(x^{3}+1)^{7}\big]^{\frac{1}{3}}} \\
            = \frac{(x^{2}+1)^{\frac{3}{2}}}{(x^{3}+1)^{\frac{7}{3}}}
        .\end{align*}
        \bigbreak \noindent 
        \textit{Now we can differentiate by using logarithmic differentiate:}
        \begin{align*}
            \ln{y} = \ln{\frac{(x^{2}+1)^{\frac{3}{2}}}{(x^{3}+1)^{\frac{7}{3}}}}
        .\end{align*}
        \bigbreak \noindent 
        \textit{From here we can use properties of natural logarithms to rewrite the equation:}
        \bigbreak \noindent 
        \textit{If:}
        \begin{align*}
            \ln{\frac{x}{y}} = \ln{x} - \ln{y}
        .\end{align*}
        \bigbreak \noindent 
        \textit{Then we can rewrite the equation as:}
        \begin{align*}
            \ln{y} = \ln{(x^{2}+1)^{\frac{3}{2}}} - \ln{(x^{3}+1)^{\frac{7}{3}}}
        .\end{align*}
        \bigbreak \noindent 
        \textit{Now we can the power rule of natural logarithms:}
        \bigbreak \noindent 
        \textit{If:}
        \begin{align*}
            \ln{x^{y}} = y \cdot \ln{x}
        .\end{align*}
        \bigbreak \noindent 
        \textit{Then we can rewrite as:}
        \begin{align*}
            \ln{y} = \frac{3}{2}\ln{(x^{2}+1)} - \frac{7}{3}\ln{(x^{3}+1)}
        .\end{align*}
        \textit{Now we can differentiate:}
        \begin{align*}
            \frac{1}{y}\frac{dy}{dx} = \frac{3}{2} \cdot \frac{1}{x^{2}+1} \cdot 2x - \frac{7}{3}\cdot \frac{1}{x^{3}+1} \cdot 3x^{2} \\
            = \frac{6x}{2}\cdot \frac{1}{x^{2}+1} - \frac{21x^{2}}{3}\cdot \frac{1}{x^{3}+1} \\
            = 3x \cdot \frac{1}{x^{2}+1} - 7x^{2}\cdot \frac{1}{x^{3}+1} \\
            = \frac{3x}{x^{2}+1} - \frac{7x^{2}}{x^{3}+1}
        .\end{align*}
    \end{mdframed}

    \pagebreak \bigbreak \noindent
    \begin{mdframed}
        \textit{From here we find a common denominator:}
        \begin{align*}
            \frac{1}{y}\frac{dy}{dx} = \frac{3x}{x^{2}+1}\cdot \frac{x^{3}+1}{x^{3}+1} - \frac{7x^{2}}{x^{3}+1} \cdot \frac{x^{2}+1}{x^{2}+1} \\
            = \frac{3x(x^{3}+1)}{(x^{2}+1)(x^{3}+1)} - \frac{7x^{2}(x^{2}+1)}{(x^{3}+1)(x^{2}+1)} \\
            = \frac{3x(x^{3}+1)-7x^{2}(x^{2}+1)}{(x^{2}+1)(x^{3}+1)} \\
            = \frac{3x^{4}+3x -(7x^{4}+7x^{2})}{(x^{2}+1)(x^{3}+1)} \\
            = \frac{3x^{4}+3x -7x^{4}-7x^{2}}{(x^{2}+1)(x^{3}+1)} \\
            = \frac{-4x^{4}-7x^{2}+3x}{(x^{2}+1)(x^{3}+1)} \\
        .\end{align*}
        \bigbreak \noindent 
        \textit{Now we can mulitply both sides by y to get $\frac{dy}{dx}$ alone and then plug our original equation into y and solve:}
        \begin{align*}
           y \cdot \frac{1}{y} \frac{dy}{dx} = \frac{-4x^{4}-7x^{2}+3x}{(x^{2}+1)(x^{3}+1)} \cdot y \\
           = \frac{dy}{dx} = \frac{-4x^{4}-7x^{2}+3x}{(x^{2}+1)(x^{3}+1)} \cdot y \\
           = \frac{dy}{dx} = \frac{x(-4x^{3}-7x+3)}{(x^{2}+1)(x^{3}+1)} \cdot y \\
           = \frac{dy}{dx} = \frac{x(-4x^{3}-7x+3)}{(x^{2}+1)(x^{3}+1)} \cdot \frac{(x^{2}+1)^{\frac{3}{2}}}{(x^{3}+1)^{\frac{7}{4}}} \\
           = \frac{dy}{dx} = \frac{x(-4x^{3}-7x+3)(x^{2}+1)^{\frac{3}{2}}}{(x^{2}+1)(x^{3}+1)(x^{3}+1)^{\frac{7}{4}}}  \\
           = \frac{dy}{dx} = \frac{x(-4x^{3}-7x+3)(x^{2}+1)^{\frac{3}{2}}}{(x^{2}+1)(x^{3}+1)(x^{3}+1)^{\frac{7}{4}}} \\
           = \frac{dy}{dx} = \frac{x(-4x^{3}-7x+3)(x^{2}+1)^{\frac{1}{2}}}{(x^{3}+1)(x^{3}+1)^{\frac{7}{4}}} \\
       .\end{align*}
       \bigbreak \noindent 
       \textit{Lastly we can rewrite $(x^{3}+1)$ as a \textbf{\textit{\underline{sum of cubes}}}, because $(x^{3}+1)$ is the same as $(x^{3}+1^{3})$},
       \bigbreak \noindent 
       \textit{If:}
       \begin{align*}
           (a^{3}+b^{3}) = (a+b)(a^{2}-ab+b^{2})
       .\end{align*}
       \textit{Where:}
       \begin{align*}
           a = x \\
           b = 1
       .\end{align*}
    \end{mdframed}
    \pagebreak \bigbreak \noindent
    \begin{mdframed}
        \textbf{\textbf{\textit{\underline{Therefore:}}}}
        \begin{align*}
          \boxed{\frac{dy}{dx} = \frac{x(-4x^{3}-7x+3)(x^{2}+1)^{\frac{1}{2}}}{(x+1)(x^{2}-x-x^{2})(x^{3}+1)^{\frac{7}{4}}}} \\
        .\end{align*}
    \end{mdframed}
\end{document}
