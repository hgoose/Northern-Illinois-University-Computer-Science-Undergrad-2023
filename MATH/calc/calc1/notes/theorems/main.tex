\documentclass{report}

\input{~/dev/latex/template/preamble.tex}
\input{~/dev/latex/template/macros.tex}

\title{\Huge{Calc 1 Chapters 2-5 All Theorems}}
\author{\huge{Nathan Warner}}
\date{\huge{May 4, 2023}}
\pagestyle{fancy}
\fancyhf{}
\lhead{Warner \thepage}
\rhead{THEOREMS}
% \lhead{\leftmark}
\cfoot{\thepage}
% \usepackage[default]{sourcecodepro}
% \usepackage[T1]{fontenc}

\pgfpagesdeclarelayout{boxed}
{
  \edef\pgfpageoptionborder{0pt}
}
{
  \pgfpagesphysicalpageoptions
  {%
    logical pages=1,%
  }
  \pgfpageslogicalpageoptions{1}
  {
    border code=\pgfsetlinewidth{1.5pt}\pgfstroke,%
    border shrink=\pgfpageoptionborder,%
    resized width=.95\pgfphysicalwidth,%
    resized height=.95\pgfphysicalheight,%
    center=\pgfpoint{.5\pgfphysicalwidth}{.5\pgfphysicalheight}%
  }%
}

\pgfpagesuselayout{boxed}

\begin{document}
    \maketitle
    \bigbreak \noindent \bigbreak \noindent
    \begin{large}
        \textbf{Secant Lines}
    \end{large}
    \begin{align*}
          m_{pq} = \frac{y_{2}-y_{1}}{x_{2}-x_{1}}\ rather,\ \frac{P_{y} - Q_{y}}{P_{x}-Q_{x}}
    .\end{align*}
    \bigbreak \noindent \bigbreak \noindent
    And, later we learned that the  slope of the secant line is defined by:
    \begin{align*}
      m_{PQ} = \frac{f(x+h)-f(x)}{h} \\
      And:
      m_{PQ} = \frac{f(x)-f(a)}{x-a}
    .\end{align*}
    \bigbreak \noindent \bigbreak \noindent 
    \begin{large}
        \textbf{Tangent lines}
    \end{large}
    \bigbreak \noindent 
    Approximation of slope of tangent line
    \begin{align*}
        \lim\limits_{Q \to P}{m_{PQ} = m}
    .\end{align*}
    \bigbreak \noindent \bigbreak \noindent
    And, later we learned that the  slope of the secant line is defined by:
    \begin{align*}
      m_{tan} = \lim\limits_{h \to 0}{\frac{f(x+h)-f(x)}{h}} \\
      And:
      m_{tan} = \lim\limits_{x \to a}{\frac{f(x)-f(a)}{x-a}}
    .\end{align*}


    \bigbreak \noindent \bigbreak \noindent 
    \begin{large}
        \textbf{Point-slope form}
    \end{large}
    \begin{align*}
        y-y_{1} = m(x-x_{1})
    .\end{align*}

    \bigbreak \noindent \bigbreak \noindent 
    \begin{large}
        \textbf{Limits}
    \end{large}
    \begin{align*}
      \lim\limits_{x \to a}{f(x) = L}\\
      \lim\limits_{x \to a}{f(x) = l}\ \Leftrightarrow \lim\limits_{x \to a-}{f(x) = l}\ \wedge \lim\limits_{x \to a+}{f(x) = l}
    .\end{align*}

    \bigbreak \noindent \bigbreak \noindent 
    \begin{large}
        \textbf{Asymptotes with limits}
    \end{large}
    \bigbreak \noindent 
    Vertical if:
    \begin{align*}
        \lim\limits_{x \to a-}{f(x) = \infty\ or\ -\infty} \\
        \lim\limits_{x \to a+}{f(x) = \infty\ or\ -\infty} \\
        \lim\limits_{x \to a}{f(x) = \infty\ or\ -\infty}
    .\end{align*}
    \bigbreak \noindent \bigbreak \noindent
    Horizontal if:
    \begin{align*}
        \lim\limits_{x \to \infty }{f(x) = L}          \\
        Or \lim\limits_{x \to -\infty}{f(x) = l}
    .\end{align*}

      \pagebreak \bigbreak \noindent
      \begin{large}
          \textbf{Continuity}
      \end{large}
      \bigbreak \noindent 
    For a function to have continuity $at a$, 3 things must be true:
      \begin{enumerate}
        \item f(x) is defined at $a$
        \item $\lim\limits_{x \to a}{f(x)}$ exists
        \item $\lim\limits_{x \to a}{f(x) = f(a)} $
      \end{enumerate}
      If no. 3 is true, the function is automatically continous at $a$

      \bigbreak \noindent \bigbreak \noindent 
      \begin{large}
          \textbf{One-sided continuity}
      \end{large}      \begin{itemize}
        \item Continuity from the right:
          \begin{align*}
            \lim\limits_{x \to a+}{f(x)=f(a)}
          .\end{align*}
        \item Continuity from the left:
          \begin{align*}
            \lim\limits_{x \to a-}{f(x)=f(a)}
          .\end{align*}
      \end{itemize}
      \bigbreak \noindent \bigbreak \noindent
      If $f$ and $g $ are continuous at $a $, then:
      \begin{itemize}
        \item $f+g $
        \item $f-g $
        \item $fg $
        \item $\frac{f}{g} $
        \item $cf $
      \end{itemize}
      \bigbreak \noindent 
      Are all continuous at $a $
      \bigbreak \noindent \bigbreak \noindent
      Also:
      \begin{itemize}
        \item Any polynomial is continous on its domain ($ \mathbb{R} $)
        \item Any rational function is continous on its domain
      \end{itemize}
      \bigbreak \noindent \bigbreak \noindent
      \nt{If $ \lim\limits_{x \to a}{f(x)} $ exists, Then you don't need to worry about which side the continuity is coming from}

      \pagebreak \bigbreak \noindent
      \begin{large}
          \textbf{Intermediate value theorem}
      \end{large}
            \bigbreak \noindent \bigbreak \noindent
      Suppose f is continuous on [a,b], Let N be any number between f(a) and f(b), where $f(a) \neq f(b)$, then:
      \begin{align*}
        \exists\ c \in (a,b)\ |\ f(c) = N
      .\end{align*}

      \bigbreak \noindent \bigbreak \noindent 
      \begin{large}
          \textbf{Derivatives and rates of change}
      \end{large}
          \begin{itemize}
      \item Slope of Secant Line:
        \begin{align*}
          m_{PQ} = \frac{f(x)- f(a)}{x-a} \\
          or: 
          \frac{f(a+h)- f(a)}{h}
        .\end{align*}
      \item Slope of Tangent Line:
        \begin{align*}
          m_{tan} = \lim\limits_{x \to a}{\frac{f(x)- f(a)}{x-a}} \\
          or: 
          \lim\limits_{h \to 0}{\frac{f(a+h)- f(a)}{h}}
        .\end{align*}
      \item Average Velocity:
        \begin{align*}
          v_{ave} = \frac{f(x)- f(x)}{x-a}
        .\end{align*}
      \item Instantaneous Velocity:
        \begin{align*}
          v_{inst} = \lim\limits_{x \to a}{\frac{f(x)- f(a)}{x-a}} \\
          or: 
          \lim\limits_{h \to 0}{\frac{f(a+h)- f(a)}{h}}
        .\end{align*}
      \item Speed:
        \begin{align*}
          Speed\ = \abs{Velocity}
        .\end{align*}
      \item Derivatives Definition:
        \begin{align*}
          f^{\prime}(x) = \lim\limits_{x \to a}{\frac{f(x)-f(a)}{x-a}} \\
          Or:
          f^{\prime}(x) = \lim\limits_{h \to 0}{\frac{f(x+h)-f(x)}{h}}
        .\end{align*}
      \item Know that:
    \begin{align*}
      s(t) = postition\ function \\
      v(t) = s^{\prime}(t) = velocity\ function \\
      a(t) = v^{\prime}(t) = acceleration\ function
    .\end{align*}
    \end{itemize}
    \bigbreak \noindent \bigbreak \noindent 
    \nt{If f(x) is differentiable at $a$, then it is continous at $a$, the converse is not true}

    \bigbreak \noindent \bigbreak \noindent 
    \begin{large}
      \textbf{Derivatives of common functions}
    \end{large}
    \bigbreak \noindent 
    Exponential Functions:
    \begin{itemize}
      \item $\frac{d}{dx}e^{x} = e^{x} \cdot \frac{d}{dx}x$
      \item $\frac{d}{dx}a^{x} = a^{x} \cdot \ln{a} \cdot \frac{d}{dx}x$
      \item $\frac{d}{dx}\ln{x}  = \frac{1}{x} \cdot \frac{d}{dx}x$
      \item $\frac{d}{dx}\log_a{x} = \frac{1}{x \cdot \ln{a}} \cdot \frac{d}{dx}x$
    \end{itemize}
    \bigbreak \noindent \bigbreak \noindent
    Trig Functions:
    \begin{itemize}
      \item $\frac{d}{dx}\sin{x}  = \cos{x}$
      \item $\frac{d}{dx}\cos{x}  = -\sin{x}$
      \item $\frac{d}{dx}\tan{x} = \sec^{2}{x} $
      \item $\frac{d}{dx}\csc{x} = -\csc{x}\cot{x} $
      \item $\frac{d}{dx}\sec{x} = \sec{x}\tan{x} $
      \item $\frac{d}{dx}\cot{x} = -\csc^{2}{x} $
    \end{itemize}
    \bigbreak \noindent \bigbreak \noindent
    Inverse Trig:
    \begin{itemize}
      \item $\frac{d}{dx}$arcsin(x) = $\frac{1}{\sqrt{1-x^{2}}}$
      \item $\frac{d}{dx}$arccos(x) = -$\frac{1}{\sqrt{1-x^{2}}}$
      \item $\frac{d}{dx}$arctan(x) = $\frac{1}{x^{2}+1}$
      \item $\frac{d}{dx}$arccsc(x) = $-\frac{1}{x\sqrt{x^{2}-1}}$
      \item $\frac{d}{dx}$arcsec(x) = $-\frac{1}{x\sqrt{x^{2}-1}}$
      \item $\frac{d}{dx}$arccot(x) = $-\frac{1}{x^{2}+1}$
    \end{itemize}

    \pagebreak \bigbreak \noindent
    \bigbreak \noindent \bigbreak \noindent
    Hyperbolic Trig
    \begin{itemize}
      \item $\frac{d}{dx}\sinh{x}  = \cosh{x}$
      \item $\frac{d}{dx}\cosh{x} = \sinh{x}$
      \item $\frac{d}{dx}\tan{x} = sech^{2}x$
      \item $\frac{d}{dx}csch{x} = -csch{x}cothx$
      \item $\frac{d}{dx}sech{x} = -sechxtanhx$
      \item $\frac{d}{dx}coth{x} = -csc^{2}x$
    \end{itemize}

    \bigbreak \noindent \bigbreak \noindent 
    \begin{large}
        \textbf{Normal Line}
    \end{large}
    \begin{align*}
        m_{tan} \cdot m_{normal } = -1
    .\end{align*}

    \bigbreak \noindent \bigbreak \noindent 
    \begin{large}
        \textbf{Product and quotient rule}
    \end{large}
    \bigbreak \noindent \bigbreak \noindent 
    Product rule:
    \begin{align*}
       \frac{d}{dx}[f(x) \cdot g(x)] = f(x) \frac{d}{dx}[g(x)] + g(x) \frac{d}{dx}[f(x)] 
    .\end{align*}
    \bigbreak \noindent \bigbreak \noindent 
    Quotient Rule:
    \begin{align*}
        \frac{d}{dx}\bigg[ \frac{f(x)}{g(x)}\bigg] = \frac{g(x) \frac{d}{dx}[f(x)] - f(x) \frac{d}{dx}[g(x)]}{[g(x)]^2}
    .\end{align*}

    \bigbreak \noindent \bigbreak \noindent 
    \begin{large}
        \textbf{Chain rule}
    \end{large}
      \begin{itemize}
          \item If:
        \begin{align*}
          F(x) = f(g(x))
        .\end{align*}
        \item Then:
        \begin{align*}
          F^{\prime}(x) = f^{\prime}(g(x)) \cdot g^{\prime}(x)
        .\end{align*}
      \end{itemize}


    \bigbreak \noindent \bigbreak \noindent 
    \begin{large}
        \textbf{Exponential growth and decay}
    \end{large}
    \begin{align*}
        y = Ce^{kt}
    .\end{align*}

    \bigbreak \noindent \bigbreak \noindent 
    \begin{large}
        \textbf{Newton's law of cooling}
    \end{large}
    \begin{align*}
        T(t) = t_{s} + Ce^{kt} \\
        C = t_{0} - t_{s}
    .\end{align*}

    \bigbreak \noindent \bigbreak \noindent 
    \begin{large}
        \textbf{Linear Approximation}
    \end{large}
    \begin{align*}
        f(x) \approx L(x) = f(a) - f^{\prime}(a)(x-a)
    .\end{align*}

    \pagebreak \bigbreak \noindent
    \begin{large}
        \textbf{Differentials}
    \end{large}
    \begin{align*}
      dy = f^{\prime}(x)dx \\
      \Delta x = dx \\
      \Delta y = (f(x+\Delta x) -f(x))
    .\end{align*}
    \bigbreak \noindent 
    \nt{$\Delta y $, can sometimes be difficult to find so we can use $dy \approx \Delta y $}


    \bigbreak \noindent \bigbreak \noindent 
    \begin{large}
        \textbf{Extreme value theorem}
    \end{large}
        \begin{itemize}
      \item If $f$ is continuous on a closed interval [a,b], then $f$ attains an
      obsulute maximum value $f(c)$ and an absolute minimum value $f(d)$ where $c,d \in [a,b]$
    \end{itemize}

    \bigbreak \noindent \bigbreak \noindent 
    \begin{large}
        \textbf{Fermats theorem}
    \end{large}
    \begin{itemize}
      \item If $f$ has a local minimum or maximum at $c$, and if $f^{\prime}(c)$ exists, 
      then $f^{\prime}(c)=0$
    \end{itemize}

    \bigbreak \noindent \bigbreak \noindent 
    \begin{large}
        \textbf{Critical number theorem}
    \end{large}
        \begin{itemize}
      \item $c$ in the domain of $f(x)$ is a critical number if $f^{\prime}(c)=0$ or if $f^{\prime}(c)$ does not exist. \\
      Note: If $f$ has a local max or min at $c$, then $c$ is a critical number of $f$ 
      \item Critical number has to obey restriction
    \end{itemize}

    \bigbreak \noindent \bigbreak \noindent 
    \begin{large}
        \textbf{Rolles Theorem}
    \end{large}
         \bigbreak \noindent 
     If f(x) satisfies the following:
     \begin{enumerate}
       \item continuous on [a,b]
        \item differentiable on (a,b)
        \item f(a) = f(b)
     \end{enumerate}
     \smallbreak \noindent
     Then there is a $c \in (a,b)$ such that $f^{\prime}(c) = 0$ 

     \bigbreak \noindent 
     Notes:
     \begin{itemize}
       \item If rolle's theorem can be applied, just set $f^{\prime}(x) = 0$ to find c, remember you are finding all c in the \textbf{\textit{\underline{open interval}}}, so if c does not obey this interval, it is not a solution
     \end{itemize}

     \pagebreak \bigbreak \noindent
     \begin{large}
         \textbf{The mean value theorem }
     \end{large}
           \bigbreak \noindent 
     if f(x) satisfies the following:
     \begin{enumerate}
       \item continuous on [a,b]
        \item differentiable on (a,b)
     \end{enumerate}
     \smallbreak \noindent
     then there is a number $c \in (a,b)$ such that
     \begin{align*}
       f^{\prime}(c) = \frac{f(b) - f(a)}{b -a}
     .\end{align*}
     \bigbreak \noindent 
     In other words,
     \begin{align*}
       m_{tan} = m_{sec}
     .\end{align*}
     \bigbreak \noindent 
     Notes:
     \begin{itemize}
       \item If rational function, find where function is undefined, if that number is not an element of the interval, then it is continous on the closed interval
        \item If $f^{\prime}(x)$ is defined on the open interval, then it is differentiable on the open interval
        \item use the theorem, then set $f^{\prime}(c) = c$
     \end{itemize}


     \bigbreak \noindent \bigbreak \noindent 
     \begin{large}
         \textbf{ L'Hospital's Rule}
     \end{large}
     \begin{align*}
         \lim\limits_{x \to a}{\frac{f(x)}{g(x)}} = \frac{f^{\prime}(x)}{g^{\prime}(x)}
     .\end{align*}

     \bigbreak \noindent \bigbreak \noindent 
     \begin{large}
         \textbf{Newton's Method}
     \end{large}
     \begin{align*}
         x_{n+1} = x_{n} - \frac{f(x_{n})}{f^{\prime}(x_{n})}
     .\end{align*}

     \pagebreak \bigbreak \noindent
     \begin{large}
         \textbf{Common Antiderivatives}
     \end{large}
     \bigbreak \noindent \bigbreak \noindent
     \begin{itemize}
      \item Exponential      
    \begin{itemize}
      \item $x^{n} = \frac{x^{n+1}}{n+1} + C$
      \item $\frac{1}{x} = \ln{\abs{x}} + C$
      \item $a^{x} = \frac{a^{x}}{\ln{a}} + C$
      \item $\ln{x} = x\ln{x} - x + C$
      \item $e^{x} = e^{x} + C$
    \end{itemize}
    \bigbreak \noindent \bigbreak \noindent
  \item Trig:
    \begin{itemize}
      \item $\sin{x} = -\cos{x} +  C$
      \item $\cos{x} = \sin{x} + C$
      \item $\tan{x} = \ln{\abs{\sec{x}}} + C$
      \item $\csc{x} = \ln{\abs{\csc{x}-\cot{x}}} + C $
      \item $\sec{x}  = \ln{\abs{\sec{x}+\tan{x}}} + C$
      \item $\cot{x} = \ln{\abs{\sin{x}}} + C$
      \item $\sin^{2}{x} = \frac{1}{2}x-\frac{1}{4}\sin{2x} + C$
      \item $\cos^{2}{x}= \frac{1}{2}x+\frac{1}{4}\sin{2x} + C$
      \item $\tan^{2}{x}= -x + \tan{x} + C $
      \item $\csc^{2}{x}= -\cot{x} + C$
      \item $\sec^{2}{x} =\tan{x} + C$
      \item $\cot^{2}{x} =-x  - \cot{x} + C$
    \end{itemize}
    \bigbreak \noindent \bigbreak \noindent
  \item Hyperbolic Trig
    \begin{itemize}
      \item $\sinh{x} = \cosh{x} + C$
      \item $\cosh{x} = \sinh{x} + C$ 
      \item $\tanh{x} = \ln{\abs{\cos{x}}} + C$
      \item $csch{x} =  \ln{\abs{\tan{h}(\frac{1}{2}x)}} + C$ 
      \item $sech{x} = \tan^{-1}{(\sinh{(x)})} + C$
      \item $coth{x}= \ln{\abs{\sinh{x}}} + C $
      \item $csch^{2}{x} = -\coth{x}+ C$
      \item $sech^{2}{x} = \tanh{x} + C$
    \end{itemize}
    \end{itemize}

      \pagebreak \bigbreak \noindent
      \begin{large}
          \textbf{Riemann sum}
      \end{large}
      \begin{align*}
          \summation{n}{i=1}\ \Delta x(f(x_{i}))\ 
      .\end{align*}

      \bigbreak \noindent \bigbreak \noindent 
      \begin{large}
          \textbf{Definition of definite integrals}
      \end{large}
      \begin{align*}
          \int_{a}^{b}\ f(x)\ dx = \lim\limits_{n \to \infty}{\summation{n}{i=1}\ \Delta xf(x_{i}^{*})\ } \\
          \Delta x = \frac{b-a}{n} \\
          x_{i} = a + i\Delta x
      .\end{align*}

          \bigbreak \noindent \bigbreak \noindent
          \begin{large}
              \textbf{The fundemental theorem of calculus}
          \end{large}
          \begin{align*}
            Part\ 1:\ \frac{d}{dx} \int_{a}^{x}\ f(x)\ dt  = f(x), a \leq x \leq b         \\
            Part\ 2:\ \int_{a}^{b}\ f(x)\ dx = F(b) - F(a)\ where\ F^{\prime} = f. \\
            and \\
            \int_{a}^{b}\ f(x)\ dx = -\int_{b}^{a}\ f(x)\ dx
          .\end{align*}
          \bigbreak \noindent \bigbreak \noindent
          \begin{large}
            \textbf{The Substitution Rule (u-sub)}
          \end{large}
          \bigbreak \noindent \bigbreak \noindent
          If $u=g(x)$ is differentiable and its range $\in I $ and $f $ is continuous on $I$, then
          \begin{align*}
            \int f(g(x))g^{\prime}(x)\ dx = \int f(u)\ du
          .\end{align*}
          \bigbreak \noindent \bigbreak \noindent


          \pagebreak \bigbreak \noindent
          \thmcon{\textbf{Laws of Limits, Derivatives, Summations} 
            \bigbreak \noindent \bigbreak \noindent
            \textbf{\textit{\underline{Limits:}}}
            \begin{itemize}
              \item $\lim\limits_{x \to a}{f(x) + g(x)} = \lim\limits_{x \to a}{f(x)} + \lim\limits_{x \to a}{g(x)}$ \ \ \ \  \ \ \ \ \ \ \ \ \ \ \ \ \ \ \ \ \ \ \ \ \ \ \ \ \ \ \ \ \ \ \  \textbullet $\lim\limits_{x \to a}{f(x)- g(x)} = \lim\limits_{x \to a}{f(x)} - \lim\limits_{x \to a}{g(x)} $
              \item $\lim\limits_{x \to a}{cf(x)} = c \cdot \lim\limits_{x \to a}{f(x)} $ \ \ \ \ \ \ \ \ \ \ \ \ \ \ \ \ \ \ \ \ \ \ \ \ \ \ \ \ \ \ \ \ \ \ \ \ \ \ \ \ \ \ \ \ \ \ \ \ \ \ \ \ \ \  \textbullet $\lim\limits_{x \to a}{f(x)\cdot g(x)} = \lim\limits_{x \to a}{f(x) } \cdot \lim\limits_{x \to a}{g(x)} $
              \item $\lim\limits_{x \to a}{\frac{f(x)}{g(x)}} = \frac{\lim\limits_{x \to a}{f(x)}}{\lim\limits_{x \to a}{g(x)}},\ \ \text{if g(x) $\neq0$} $  \ \  \ \ \ \ \  \ \ \ \ \ \ \ \ \ \ \ \ \textbullet $\lim\limits_{x \to a}{\bigg(f(x)\bigg)^{n}}  = \bigg[\lim\limits_{x \to a}{f(x)}\bigg]^{n}$, where n is a positive integer
              \item $\lim\limits_{x \to a}{c} = c $  \ \ \ \ \ \ \ \ \ \ \ \ \ \ \ \ \ \ \ \ \ \ \ \  \ \ \ \ \ \ \ \ \ \ \ \ \ \ \ \ \ \ \ \ \ \ \ \ \ \ \ \  \ \ \ \ \  \ \ \ \ \ \ \    \  \ \ \ \ \ \ \  \ \ \textbullet $\lim\limits_{x \to a}{x} = a $
              \item $\lim\limits_{x \to a}{x^{n}} = a^{n}$, where n is a positive integer \ \ \ \ \ \  \ \ \ \ \ \ \  \ \ \ \ \ \ \ \ \  \ \ \textbullet $\lim\limits_{x \to a}{\sqrt[n]{x}} = \sqrt[n]{a} $, where n is a positive integer
              \item $\lim\limits_{x \to a}{\sqrt[n]{f(x)}} = \sqrt[n]{\lim\limits_{x \to a}{f(x)}} $, where n is a positive integer
            \end{itemize}
            \bigbreak \noindent \bigbreak \noindent
            \textbf{\textit{\underline{Derivatives:}}}
            \begin{itemize}
              \item $\frac{d}{dx}c  = 0$
              \item $\frac{d}{dx}x  = 1$
              \item $ \frac{d}{dx}(x^n) = n \cdot x^{n-1} \rightarrow$ \text{\textbf{\textit{Power Rule}}}
              \item $ \frac{d}{dx}[c \cdot f(x)] = c \cdot \frac{d}{dx}[f(x)]$
              \item $ \frac{d}{dx}[f(x) \pm g(x)] = \frac{d}{dx}f(x)\pm \frac{d}{dx}g(x)$
            \end{itemize}
            \bigbreak \noindent \bigbreak \noindent 
            \textbf{\textit{\underline{Summation:}}}
            \begin{itemize}
                 \item $\summation{n}{i=m}c \cdot a_i = c\summation{n}{i=m} a_{i}$, where c is a constant
                \item $\summation{n}{i=m} (a_{i} + b_{i})= \summation{n}{i=m} a_{i} + \summation{n}{i=m} b_{i} $
                \item $\summation{n}{i=m} (a_{i} - b_{i})= \summation{n}{i=m} a_{i} - \summation{n}{i=m} b_{i} $
                \item $\summation{n}{i=1} 1=n $
                \item $\summation{n}{i=1} c=c \cdot n $, where c is a constant
                \item $\summation{n}{i=1} i= \frac{n(n+1)}{2} $
                \item $\summation{n}{i=1}i^{2} = \frac{n(n+1)(2n+1)}{6} $
                \item $\summation{n}{i=1}i^{3}=\bigg[\frac{n(n+1)}{2}\bigg]^{2} $
            \end{itemize}
            % \vspace{10em}
          }
          \pagebreak \bigbreak \noindent
          \thmcon{\textbf{Properties of Integrals:}
            \bigbreak \noindent 
          \begin{itemize}
            \item $\int_{a}^{b} cdx = c(b-a) $ 
            \item $\int_{a}^{b}cf(x)dx = c \cdot \int_{a}^{b}f(x)dx $
            \item $\int_{a}^{b}[f(x) + g(x)]dx = \int_{a}^{b}f(x)dx + \int_{a}^{b}g(x)dx$
            \item $\int_{a}^{b}[f(x) - g(x)]dx = \int_{a}^{b}f(x)dx - \int_{a}^{b}g(x)dx$
            \item $\int_{a}^{c}f(x)dx = \int_{a}^{b}f(x)dx + \int_{b}^{c}f(x)dx $
            \item if $f(x) \geq 0$ for all $a \leq x \leq b$, then $\int_{a}^{b}f(x)dx \geq  0 $
            \item if $f(x) \geq g(x) $ for all $a \leq x \leq b$, then $\int_{a}^{b}f(x)dx \geq \int_{a}^{b}g(x)dx $
            \item if $m \leq f(x) \leq M $ for $a \leq x \leq b$, then $m(b-a) \leq \int_{a}^{b}f(x)dx \leq M(b-a)$
          \end{itemize}
          }




\end{document}
