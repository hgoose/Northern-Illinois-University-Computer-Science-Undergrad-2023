\documentclass{report}

\input{~/dev/latex/template/preamble.tex}
\input{~/dev/latex/template/macros.tex}

\title{\Huge{2.6 Hw Solutions}}
\author{\huge{Nathan Warner}}
\date{\huge{Feb 02, 2023}}

\begin{document}
    \maketitle
    \begin{Large}
        \noindent \textbf{Question 1:}
    \end{Large}

    \bigbreak \noindent \bigbreak \noindent 
    \pf{Solution}{\textbf{See Hw}}
    \bigbreak \noindent 

    \bigbreak \noindent \bigbreak \noindent  
    \begin{Large}
        \noindent \textbf{Question 2:}
    \end{Large}

    \bigbreak \noindent \bigbreak \noindent 
    \pf{Solution}{}
    \bigbreak \noindent 
    \textbf{1.) We want to divide each term in the numerator and denominator by $x^2$}

    \bigbreak \noindent 
    \begin{align*}
        \lim\limits_{x \to \infty}{ \frac{ \frac{8x^2}{x^2} - \frac{3}{x^2}}{ \frac{7x^2}{x^2} + \frac{x^2}{x^2} - \frac{3}{x^2}}}
    .\end{align*}

    \bigbreak \noindent 
    Which simplifys to:

    \begin{align*}
        \lim\limits_{x \to \infty}{ \frac{8 - \frac{3}{x^2}}{7 + \frac{1}{x} - \frac{3}{x^2}}}
    .\end{align*}

    \bigbreak \noindent 
    and if we take the limit of each of these terms, we get:

    \begin{align*}
        \frac{8-0}{7+0-0} \\
        = \frac{8}{7}
    .\end{align*}

    \bigbreak \noindent \bigbreak \noindent 
    \nt{Remember from pre-calc, to find the H.A of a rational function when the degree of the 
        denominator \textbf{\textit{Equals}} the degree of the numerator. The answer is found by \textbf{\textit{Dividing the coefficients} from the terms with the highest exponents}
    }

    \bigbreak \noindent \bigbreak \noindent  
    \begin{Large}
        \noindent \textbf{Question 3:}
    \end{Large}

    \bigbreak \noindent \bigbreak \noindent 
    \pf{Solution}{Same concept as Question 2.}
    \bigbreak \noindent 

    \bigbreak \noindent \bigbreak \noindent  
    \begin{Large}
        \noindent \textbf{Question 4:}
    \end{Large}

    \bigbreak \noindent \bigbreak \noindent 
    \pf{Solution}{}
    \bigbreak \noindent 
    \textbf{1.) If we divide each of terms by the degree of the denominator, \textbf{\textit{x}}
        then we get:
    }

    \begin{align*}
        \lim\limits_{x \to \infty}{ \frac{ \frac{-7}{x}}{2 + \frac{5}{x}}}
    .\end{align*}

    \bigbreak \noindent 
    \textbf{2.) and if we take the limit of each of the terms in both the numerator and denominator. We get:}

    \begin{align*}
        \frac{0}{2+0} \\
        = 0
    .\end{align*}

    \bigbreak \noindent 
    \nt{Remember from pre-calc, that if the degree of the denominator is higher than the degree
        in the numerator, the H.A is \textbf{\textit{y = 0}}
    }

    \bigbreak \noindent \bigbreak \noindent 
    \begin{Large}
        \textbf{Question 5:}
    \end{Large}

    \bigbreak \noindent  \bigbreak \noindent 
    \pf{Solution}{}
    \bigbreak \noindent 
    \textbf{1.) Reverse factor the equation and get:}

    \begin{align*}
        \lim\limits_{x \to - \infty}{ \frac{7u^4 + 6u^2 - 1}{u^4 + 18u^2 + 81}}
    .\end{align*}

    \bigbreak \noindent 
    Now divide both sides by the highest term in the denominator: \textbf{\textit{$x^4$}}

    \begin{align*}
        \lim\limits_{x \to - \infty}{ \frac{ \frac{7u^4}{u^4} + \frac{6u^2}{u^4} - \frac{1}{u^4}}{1 + \frac{18u^2}{u^4} + \frac{81}{u^4}}}
    .\end{align*}

    \bigbreak \noindent 
    Simplify and get:

    \begin{align*}
        \lim\limits_{x \to - \infty}{ \frac{7 + \frac{6}{u^2} - \frac{1}{u^4}}{1 + \frac{18}{u^2} + \frac{81}{u^4}}}
    .\end{align*}

    \bigbreak \noindent 
    Take the limit of each term and we are left with:

    \begin{align*}
        \frac{7+0-0}{1+0+0} \\ 
        = 7
    .\end{align*}

    \bigbreak \noindent \bigbreak \noindent 
    \begin{Large}
        \textbf{Question 6:}
    \end{Large}

    \pagebreak \bigbreak \noindent
    \pf{Solution}{}
    \bigbreak \noindent 
    \textbf{1.) Divide each side by $x^4$}

    \begin{align*}
        \lim\limits_{x \to - \infty}{ \frac{ \frac{4x^5}{x^4} - \frac{x}{x^4}}{ \frac{x^4}{x^4} + \frac{5}{x^4}}}
    .\end{align*}

    \bigbreak \noindent 
    Which simplifies to:

    \begin{align*}
        \lim\limits_{x \to - \infty}{ \frac{4x - \frac{1}{x^3}}{1+ \frac{5}{x^4}}}
    .\end{align*}

    \bigbreak \noindent 
    Now take the limit of each term:

    \begin{align*}
        \frac{4 \left(- \infty\right) - 0}{1 + 0} \\
        = \frac{- \infty}{1} \\ 
        = - \infty
    .\end{align*}

    \bigbreak \noindent \bigbreak \noindent 
    \begin{Large}
        \textbf{Question 7:}
    \end{Large}

    \bigbreak \noindent  \bigbreak \noindent 
    \pf{Solution}{}
    \bigbreak \noindent 
    DNE

     \bigbreak \noindent \bigbreak \noindent 
    \begin{Large}
        \textbf{Question 8:}
    \end{Large}

    \bigbreak \noindent  \bigbreak \noindent 
    \pf{Solution}{}
    \bigbreak \noindent 
    As x approaches 0 from the right, ln(x) decreases without bound. 

    \bigbreak \noindent 
    So if take:

    \begin{align*}
        \lim\limits_{x \to - \infty}{\arctan(x)}
    .\end{align*}

    \bigbreak \noindent 
    We get $-\frac{\pi}{2}$, this is because of arctans asymptotes.

    \bigbreak \noindent 
    Then if we multiply that by 3, we get:

    \begin{align*}
        3 \cdot - \frac{\pi}{2} \\ 
        = - \frac{3\pi}{2}
    .\end{align*}

    \pagebreak \bigbreak \noindent
    \begin{Large}
        \textbf{Question 9:}
    \end{Large}

    \bigbreak \noindent 
    \pf{Solution}{}
    \bigbreak \noindent 
    For H.A, divide both halfs by $x^2$ and simplify:

    \begin{align*}
        \lim\limits_{x \to \infty}{ \frac{2 + \frac{3}{x^2}}{3 + \frac{14}{x} - \frac{5}{x^2}}}
    .\end{align*}

    \bigbreak \noindent 
    Take the limits of each term:

    \begin{align*}
        \frac{2+ 0}{3 + 0 - 0} \\ 
        = \frac{2}{3}
    .\end{align*}

    \bigbreak \noindent 
    Notice it would be the same for $\lim\limits_{x \to - \infty}{}$, Therefore our only H.A is y= $ \frac{1}{3}$

    \bigbreak \noindent 
    For V.A, Factor the equation and find the zeros of the denominator. Notice we cannot factor the top, but the bottom factors into:

    \begin{align*}
        \left(3x - 1\right) \left(x + 5\right)
    .\end{align*}

    \bigbreak \noindent 
    So our V.A's are, x = -5, $ \frac{1}{3}$

    \bigbreak \noindent \bigbreak \noindent 
    \begin{Large}
        \textbf{Question 10:}
    \end{Large}

    \bigbreak \noindent 
    \pf{Solution}{}
    \bigbreak \noindent 
    \textbf{Part 1.)}
    \bigbreak \noindent 
    Set equation = 0, solve for x:

    \begin{align*}
        e^x - 3 = 0 \\ 
        e^x = 3 \\ 
        x = ln(3)
    .\end{align*}

    \bigbreak \noindent 
    \textbf{Part 3.)}
    \bigbreak \noindent 
    \textbf{Find: $\lim\limits_{x \to - \infty}{ \frac{7e^x}{e^x-3}}$}

    \bigbreak \noindent 
    Since $e^{-\infty}$ = 0, our equation is: 

    \begin{align*}
        \frac{7 \left(0\right)}{0 - 3} \\ 
        = \frac{0}{-3} \\ 
        = 0
    .\end{align*}

    \bigbreak \noindent 
    \textbf{Find: $\lim\limits_{x \to \infty}{ \frac{7e^x}{e^x-3}}$}

    \bigbreak \noindent 
    Divide both halfs by $e^x$

    \begin{align*}
        \lim\limits_{x \to \infty}{ \frac{7}{1- \frac{3}{e^x}}} \\ 
        = \frac{7}{1} \\
        = 7
    .\end{align*}

    \bigbreak \noindent 
    So our V.A is the answer we found in part 1, and our H.A is what we just found in part 3 

    \bigbreak \noindent \bigbreak \noindent \bigbreak \noindent 
    \begin{Large}
        \textbf{Question 11:}
    \end{Large}
    \bigbreak \noindent 
    \pf{Solution}{}
    \bigbreak \noindent 
    The first condition, \textbf{\textit{ $\lim\limits_{x \to \pm \infty}{f(x)} = 0$}}, 
    Tells us that y = 0 is a H.A, this means that the degree of the numerator is less than 
    the degree of the denominator.

    \bigbreak \noindent 
    The condition to the right tells us that x = 0 is a V.A, so we need $-x^2$ in the denominator. \textbf{\textit{Negative because of - $\infty$}}

    \bigbreak \noindent 
    The next condition tells us that there is a factor of (x-4) in the numerator.

    \bigbreak \noindent 
    The last 2 conditions tells us that x = 5 is a V.A and therefore (x-5) belongs in the denominator.

    \bigbreak \noindent 
    So our equation is:
    \begin{align*}
        \frac{x-4}{-x^2 \left(x-5\right)}
    .\end{align*}

    \bigbreak \noindent \bigbreak \noindent \bigbreak \noindent 
    \begin{Large}
        \textbf{Question 11:}
    \end{Large}

    \bigbreak \noindent 
    \pf{Solution}{}
    \bigbreak \noindent 
    \begin{align*}
        20 \cdot 25 = 500
    .\end{align*}
    \bigbreak \noindent 
    So:
    \begin{align*}
        \frac{500t}{6000+25t} \\ 
        = \frac{25 \left(20t\right)}{6000 + 25t} \\ 
        = \frac{25 \left(20t\right)}{25 \left(240 + t\right)} \\ 
        = \frac{20t}{240+t}
    .\end{align*}

    \bigbreak \noindent 
    \textbf{Part b.)}
    \bigbreak \noindent 
    \begin{align*}
        \lim\limits_{t \to \infty}{ \frac{20t}{240 + t}}
    .\end{align*}

    \bigbreak \noindent 
    Divide each term by t:

    \begin{align*}
        \lim\limits_{x \to \infty}{ \frac{ \frac{20t}{t}}{ \frac{240}{t} + \frac{t}{t}}}    
    .\end{align*}

    \bigbreak \noindent 
    Simplify:

    \begin{align*}
        \lim\limits_{x \to \infty}{ \frac{20}{ \frac{240}{t} + 1}}
    .\end{align*}

    \bigbreak \noindent 
    Take limit of each term:

    \begin{align*}
        \frac{20}{0 + 1} \\ 
        = 20
    .\end{align*}

\end{document}
