\documentclass{report}

\input{~/dev/latex/template/preamble.tex}
\input{~/dev/latex/template/macros.tex}

\title{\Huge{Section 2.2 Homework}}
\author{\huge{Nathan Warner}}
\date{\huge{Jan 24, 2023}}

\begin{document}
    \maketitle
    \section{\Large{2.2 HW Answers}}
    
    \bigbreak \noindent \bigbreak \noindent 
    \begin{large}
       \noindent \textbf{Question 1: See HW } 
    \end{large}

    \bigbreak \noindent \bigbreak \noindent  
    \begin{large}
        \noindent \textbf{Question 2: See HW}
    \end{large}

    \bigbreak \noindent \bigbreak \noindent  
    \begin{large}
       \textbf{Question 3: Use the graph of the function f to state the value of each limit, if it exists. (If an answer does not exist, enter DNE.)} 
    \end{large}

    \bigbreak \noindent 
    \begin{large}
        \begin{align*}
            f \left(x\right) = \frac{e^{1/x}-5}{e^{1/x}+1}
        .\end{align*}
    \end{large}
    
    
    \bigbreak \noindent 
    \pf{Solution}{}
        
    \bigbreak \noindent 
    \begin{large}
        \textbf{a.)} $\lim\limits_{x \to a-}{f \left(x\right)}$
    \end{large}
    
    \bigbreak plug in a value below 0 thats close to 0, say \textbf{-0.25}, this returns the value
    -4.89, so we can see f(x) is getting closer to \textbf{5}

    \bigbreak \noindent \bigbreak \noindent  
    \begin{large}
        \textbf{b.)} $\lim\limits_{x \to 0+}{f \left(x\right)}$ 
    \end{large}

    \bigbreak \noindent 
    Plug in a value above 0 thats close to 0, say \textbf{0.25}, this returns 0.89, so we
    can see that f(x) is getting closer to \textbf{1}
    
    \bigbreak \noindent \bigbreak \noindent 
    a and b were not the same values so \textbf{c} is DNE
     
    \bigbreak \noindent \bigbreak \noindent 
    \begin{large}
       \textbf{Question 4: just plug in the values to fill in the table and then deduce the limit} 
    \end{large}
    
\end{document}
