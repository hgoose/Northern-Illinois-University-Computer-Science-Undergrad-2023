\documentclass{report}

\input{~/dev/latex/template/preamble.tex}
\input{~/dev/latex/template/macros.tex}

\title{\Huge{2.8 Hw Solutions}}
\author{\huge{Nathan Warner}}
\date{\huge{Feb 5, 2023}}

\pgfpagesdeclarelayout{boxed}
{
  \edef\pgfpageoptionborder{0pt}
}
{
  \pgfpagesphysicalpageoptions
  {%
    logical pages=1,%
  }
  \pgfpageslogicalpageoptions{1}
  {
    border code=\pgfsetlinewidth{1.5pt}\pgfstroke,%
    border shrink=\pgfpageoptionborder,%
    resized width=.95\pgfphysicalwidth,%
    resized height=.95\pgfphysicalheight,%
    center=\pgfpoint{.5\pgfphysicalwidth}{.5\pgfphysicalheight}%
  }%
}

\pgfpagesuselayout{boxed}

\begin{document}
    \maketitle
    \begin{Large}
        \noindent \textbf{Question 1:}
    \end{Large}
    \bigbreak \noindent 
    \pf{Solution}{}
    \bigbreak \noindent 
    \textbf{See hw:}

    \bigbreak \noindent \bigbreak \noindent \bigbreak \noindent 
    \begin{Large}
        \textbf{Question 2:}
    \end{Large}
    \bigbreak \noindent 
    \pf{Solution}{}
    \bigbreak \noindent 
    \textbf{See hw:}

    \bigbreak \noindent \bigbreak \noindent \bigbreak \noindent 
    \begin{Large}
        \textbf{Question 3:}
    \end{Large}
    \bigbreak \noindent 
    \pf{Solution}{}
    \bigbreak \noindent 
    \textit{If:}
    \begin{align*}
      f\prime(x)= \lim\limits_{h \to 0}{ \frac{f(x+h) - f(x)}{h}} 
    .\end{align*}
    \bigbreak \noindent 
    \textit{Then:}
    \begin{align*}
      f\prime(t) = \lim\limits_{h \to 0}{ \frac{5.5(t+h)^2+7(t+h) - (5.5t^2+7t)}{h}} \\
      = \lim\limits_{h \to 0}{ \frac{5.5t^2+5.5h^2+11th+7t+7h -5.5t^2-7t}{h}} \\
      = \lim\limits_{h \to 0}{ \frac{5.5h^2+11th+7h}{h}} \\
      = \lim\limits_{h \to 0}{ \frac{h(5.5h+11t+7)}{h}} \\ 
      = \lim\limits_{h \to 0}{5.5h+11t+7} \\ 
      = 5.5(0) + 11t + 7  \\
      = 11t+7
    .\end{align*}
    \bigbreak \noindent 
    \textit{Domain for both is $\mathbb{R}$}

    \bigbreak \noindent \bigbreak \noindent \bigbreak \noindent 
    \begin{Large}
        \textbf{Question 4:}
    \end{Large}
    \bigbreak \noindent 
    \pf{Solution}{}
    \bigbreak \noindent 
    \textit{If:}
    \begin{align*}
      f\prime(x) = \lim\limits_{h \to 0}{ \frac{f(x+h) - f(a)}{h}}
    .\end{align*}
    \bigbreak \noindent 
    \textit{Then:}
    \begin{align*}
      f\prime(x) = \lim\limits_{h \to 0}{ \frac{ \frac{1}{(x+h)^2-36} -  \frac{1}{x^2-36}}{h}} \\ 
      = \lim\limits_{h \to 0}{ \frac{(x^2-36) - ((x+h^2) - 36)}{h(x^2-36)((x+h)^2-36)}} \\ 
      = \lim\limits_{h \to 0}{ \frac{ x^2-36 - (x^2+h^2+2xh-36)}{h(x^2-36)((x+h)^2-36)}} \\ 
      = \lim\limits_{h \to 0}{ \frac{x^2-36-x^2-h^2-2xh+36}{h(x^2-36)((x+h)^2-36)}} \\ 
      = \lim\limits_{h \to 0}{ \frac{-h^2-2xh}{h(x^2-36)((x+h)^2-36)}} \\ 
      = \lim\limits_{h \to 0}{ \frac{h(-h-2x)}{h(x^2-36)((x+h)^2-36)}} \\
      = \lim\limits_{h \to 0}{ \frac{-h-2x}{(x^2-36)((x+h)^2-36)}} \\ 
      = \frac{-(0) - 2x}{(x^2-36)(x+0)^2-36} \\ 
      = \frac{-2x}{(x^2-36)^2}
    .\end{align*}
    \bigbreak \noindent 
    \textit{\underline{Domain: f(x)}}  
    \begin{align*}
      x^2-36=0 \\ 
       x^2=36 \\
      x = \pm 6 \\ 
      (- \infty, -6) \cup (-6,6) \cup (6, \infty)
    .\end{align*}
    \bigbreak \noindent 
    \textit{\underline{Domain: $f\prime(x)$}}
    \begin{align*}
      (x^2-36)^2 = 0 \\
      x^2-36 = 0
      x^2=36 \\
      x = \pm 6 \\ 
      (- \infty, -6) \cup (-6,6) \cup (6, \infty)
    .\end{align*}

    \bigbreak \noindent \bigbreak \noindent \bigbreak \noindent 
    \begin{Large}
        \textbf{Question 5:}
    \end{Large}
    \bigbreak \noindent 
    \pf{Solution}{}
    \bigbreak \noindent 
    \textit{If:}
    \begin{align*}
      f\prime(x) = \lim\limits_{h \to 0}{ \frac{f(x+h) - f(a)}{h}}
    .\end{align*}
    \bigbreak \noindent 
    \textit{Then:}
    \begin{align*}
      f\prime(x) = \lim\limits_{h \to 0}{ \frac{ \frac{1}{9 + \sqrt{x+h}} - \frac{1}{9+ \sqrt{x}}}{h}} \\ 
      = \lim\limits_{h \to 0}{ \frac{(9+ \sqrt{x})-(9+ \sqrt{x+h})}{h(9+ \sqrt{x})(9+ \sqrt{x+h})}} \\ 
      = \lim\limits_{h \to 0}{ \frac{9+ \sqrt{x} -9- \sqrt{x+h}}{h(9+ \sqrt{x})(9+ \sqrt{x+h})}} \\ 
      = \lim\limits_{h \to 0}{ \frac{ \sqrt{x} - \sqrt{x+h}}{h(9+ \sqrt{x})(9+ \sqrt{x+h})}} \\ 
    .\end{align*}
    \textit{Multiply both halfs by the conjugate: $ \sqrt{x} + \sqrt{x+h}$} 
    \begin{align*}
      \lim\limits_{h \to 0}{ \frac{x-(x+h)}{h(9+ \sqrt{x})(9+ \sqrt{x+h})( \sqrt{x} + \sqrt{x+h})}} \\ 
      = \lim\limits_{h \to 0}{ \frac{x-x-h}{h(9+ \sqrt{x})(9+ \sqrt{x+h})( \sqrt{x} + \sqrt{x+h})}} \\ 
      = \lim\limits_{h \to 0}{ \frac{-h}{h(9+ \sqrt{x})(9+ \sqrt{x+h})( \sqrt{x} + \sqrt{x+h})}} \\
      = \lim\limits_{h \to 0}{ \frac{-1}{(9+ \sqrt{x})(9+ \sqrt{x+h})( \sqrt{x} + \sqrt{x+h})}} \\
      = \frac{-1}{(9+ \sqrt{x})(9+ \sqrt{x+0})( \sqrt{x} + \sqrt{x+0})} \\
      = \frac{-1}{(9+ \sqrt{x})(9+ \sqrt{x})( \sqrt{x} + \sqrt{x})} \\
      = -\frac{1}{2\sqrt{x}(9+ \sqrt{x})^2} \\
    .\end{align*}
    \bigbreak \noindent 
    \textit{Domain: f(x)}
    \begin{align*}
        x \geq 0 \\ 
        and\ \\ 
        9 + \sqrt{x} = 0 \\
        \sqrt{x} = -9 \\
        x = \sqrt{-9} = undefined \\
        = [0, \infty) 
    .\end{align*}
    \bigbreak \noindent 
    \textit{Domain: $f\prime(x)$}
    \begin{align*}
      x \geq 0 \\
      and\ \\ 
      2\sqrt{x} = 0 \\
       \sqrt{x} = 0 \\
       x = \sqrt{0} \\ 
      = 0\\ therefore\ x\ cannot\ be\ zero \\
      = (0, \infty)
    .\end{align*}
    \bigbreak \noindent \bigbreak \noindent \bigbreak \noindent 
    \begin{Large}
        \textbf{Question 6:}
    \end{Large}
    \bigbreak \noindent 
    \pf{Solution}{}
    \bigbreak \noindent 
    \textbf{\textit{F cannot be differentiable at:}}
    \begin{itemize}
      \item a corner (-4)
      \item a discontinuity (0)
      \item a vertical tangent (2)
    \end{itemize}

    \bigbreak \noindent \bigbreak \noindent \bigbreak \noindent 
    \begin{Large}
        \textbf{Question 7:}
    \end{Large}
    \bigbreak \noindent 
    \pf{Solution}{}
    \bigbreak \noindent 

    \bigbreak \noindent \bigbreak \noindent \bigbreak \noindent 
    \begin{Large}
        \textbf{Question 9:}
    \end{Large}
    \bigbreak \noindent 
    \pf{Solution}{}
    \bigbreak \noindent 

    \bigbreak \noindent \bigbreak \noindent \bigbreak \noindent 
    \begin{Large}
        \textbf{Question 9:}
    \end{Large}
    \bigbreak \noindent 
    \pf{Solution}{}
    \bigbreak \noindent 
    
\end{document}
