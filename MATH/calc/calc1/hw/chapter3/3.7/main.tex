\documentclass{report}

\input{~/dev/latex/template/preamble.tex}
\input{~/dev/latex/template/macros.tex}

\title{\Huge{3.7 Hw Solutions}}
\author{\huge{Nathan Warner}}
\date{\huge{March 7, 2023}}
\pagestyle{fancy}
\fancyhf{}
\rhead{HW SOLUTIONS}
\lhead{\leftmark}
\cfoot{\thepage}

\pgfpagesdeclarelayout{boxed}
{
  \edef\pgfpageoptionborder{0pt}
}
{
  \pgfpagesphysicalpageoptions
  {%
    logical pages=1,%
  }
  \pgfpageslogicalpageoptions{1}
  {
    border code=\pgfsetlinewidth{1.5pt}\pgfstroke,%
    border shrink=\pgfpageoptionborder,%
    resized width=.95\pgfphysicalwidth,%
    resized height=.95\pgfphysicalheight,%
    center=\pgfpoint{.5\pgfphysicalwidth}{.5\pgfphysicalheight}%
  }%
}

\pgfpagesuselayout{boxed}

\begin{document}
    \maketitle
    \begin{Large}
        \noindent \textbf{Question 1:}
    \end{Large}
    \bigbreak \noindent 
    \pf{Solution}{}
    \bigbreak \noindent 
    \textbf{Part 1.) Find the derivative of h:}
    \begin{align*}
      v(t) = h^{\prime} = 25.5 -9.8t
    .\end{align*}

    \bigbreak \noindent 
    \textbf{Part 2.) Plug 2 and 4 into velocity function:}
    \begin{align*}
      v(2) = 25.5-9.8(2) \\
      =5.9
    .\end{align*}
    \begin{align*}
      v(4) = 25.5 - 9.8(4) \\
      =-13.7
    .\end{align*}

    \bigbreak \noindent 
    \textbf{Part 3.) Set v(t) = 0 and solve for t}
    \begin{align*}
      No\ work 
    .\end{align*}

    \bigbreak \noindent 
    \textbf{Part 4.) Plug answer from part 3 into h for t}
    \begin{align*}
      No\ Work
    .\end{align*}

    \bigbreak \noindent 
    \textbf{part 5.) Set h = 0 and solve using the quadratic formula}
    \begin{align*}
      no\ work
    .\end{align*}

    \bigbreak \noindent 
    \textbf{Part 6.) Plug answer from part 5 into v(t)}

    \bigbreak \noindent \bigbreak \noindent 
    \begin{Large}
        \textbf{Question 2:}
    \end{Large}
    \bigbreak \noindent 
    \pf{Solution}{}
    \bigbreak \noindent 
    \textbf{a.) find v(t), set equal to zero and then plug answer into h}
    \begin{align*}
      no\ work
    .\end{align*}

    \bigbreak \noindent 
    \textbf{b.) set h = 384 and solve using the quadratic fomula, then plug answers into v(t)}
    \begin{align*}
      no\ work
    .\end{align*}

    \bigbreak \noindent \bigbreak \noindent 
    \begin{Large}
        \textbf{Question 3:}
    \end{Large}
    \bigbreak \noindent 
    \pf{Solution}{}
    \bigbreak \noindent 
    \textbf{Part a.) Find v(t), set equal to 25, factor and solve for t}
    \bigbreak \noindent 
    \nt{we don't care about negative time}

    \bigbreak \noindent 
    \textbf{Part b.) Find a(t) and set = 0 }
    \bigbreak \noindent 
    \nt{Again, don't care about negative time}

    \bigbreak \noindent \bigbreak \noindent 
    \begin{Large}
        \textbf{Question 4:}
    \end{Large}
    \bigbreak \noindent 
    \pf{Solution}{}
    \bigbreak \noindent 
    \begin{align*}
      A = \pi r^{2}
    .\end{align*}
    \bigbreak \noindent 
    \textbf{a.i)}
    \begin{align*}
        \frac{a(5)-  a(4)}{5-4} \\
        = \frac{\pi (5)^{2} - \pi(4)^{2}}{5-4} \\ 
        = \frac{25\pi - 16\pi}{1} \\
        = \boxed{9\pi}
    .\end{align*}

    \bigbreak \noindent 
    \textbf{Part b.)}
    \begin{align*}
      A(r) = \pi r^{2} \\
      A^{\prime}(r) = 2\pi r \\
      A^{\prime}(4) = 2 \pi(4) \\ 
      = \boxed{8\pi}
    .\end{align*}

    \bigbreak \noindent \bigbreak \noindent 
    \begin{Large}
        \textbf{Question 5:}
    \end{Large}
    \bigbreak \noindent 
    \pf{Solution}{}
    \bigbreak \noindent 
    \begin{align*}
      S = 4\pi r^{2} \\ 
      S^{\prime} = 8\pi r 
    .\end{align*}

    \bigbreak \noindent 
    \textbf{a.)}
    \begin{align*}
      S(4) = 8\pi(4) \\ 
      = 32\pi
    .\end{align*}

    \bigbreak \noindent 
    \textbf{b.)}
    \begin{align*}
      S(5) = 8\pi (5) \\
      = 40\pi
    .\end{align*}

    \bigbreak \noindent \bigbreak \noindent 
    \begin{Large}
        \textbf{Question 6:}
    \end{Large}
    \bigbreak \noindent 
    \pf{Solution}{}
    \bigbreak \noindent 
    \textit{If:}
    \begin{align*}
      V = 5500(1-\frac{1}{50}t)^{2}
    .\end{align*}
    \bigbreak \noindent 
    \textit{Then:}
    \begin{align*}
      V^{\prime} = 11000(1-\frac{1}{50}t) \cdot -\frac{1}{50}
    .\end{align*}
    \bigbreak \noindent 
    \textbf{5 min:}
    \begin{align*}
      V^{\prime}(5) = 11000(1-\frac{1}{50}(5)) \cdot -\frac{1}{50} \\ 
      = -198
    .\end{align*}
    \bigbreak \noindent 
    \textbf{Flowing the fastest:}
    \begin{align*}
      0\ min
    .\end{align*}
    \bigbreak \noindent 
    \textbf{Flowing the slowest:}
    \begin{align*}
      50\ min
    .\end{align*}
    \bigbreak \noindent 
    \nt{remember the interval is $0\leq t\leq 50$, water does not start flowing out at 5 minutes, it starts at 0}

    \bigbreak \noindent \bigbreak \noindent 
    \begin{Large}
        \textbf{Question 7:}
    \end{Large}
    \bigbreak \noindent 
    \pf{Solution}{}
    \bigbreak \noindent 
    \begin{align*}
      D(t) = 7+5\cos{[0.503(t-6.75)]}
    .\end{align*}
    \begin{align*}
      D^{\prime}(t) =  0 + 5(-\sin{(0.503(t-6.75)))} \cdot (0.503(1 - 0)) \\ 
      =  5(-\sin{(0.503(t-6.75)))} \cdot 0.503\\ 
      =  5(-\sin{(0.503t-3.39525))} \cdot 0.503\\ 
      =  5(-0.503\sin{(0.503t-3.39525))} \\ 
      =  -2.515\sin{(0.503t-3.39525)}\\ 
    .\end{align*}

    \bigbreak \noindent 
    \textbf{a-d.)}
    \begin{align*}
      D(2) =  -2.515\sin{(0.503(2)-3.39525)}\\ 
      =  1.72
    .\end{align*}

    \bigbreak \noindent \bigbreak \noindent 
    \begin{Large}
        \textbf{Question 8:}
    \end{Large}
    \bigbreak \noindent 
    \pf{Solution}{}
    \bigbreak \noindent 
    \begin{align*}
      n(t) = 600 \cdot 3^{t}
    .\end{align*}
    \bigbreak \noindent 
    \textbf{b.) Find $n^{\prime}(t)$}
    \bigbreak \noindent 
    \textit{If:}
    \begin{align*}
      \frac{d}{dx} a^{x} = a^{x}\cdot \ln{x}
    .\end{align*}
    \bigbreak \noindent 
    \textit{And:}
    \begin{align*}
      n(t) = 600\cdot 3^{t}
    .\end{align*}
    \bigbreak \noindent 
    \textit{Then:}
    \begin{align*}
      n^{\prime}(t) = 600 \cdot 3^{t} \cdot \ln{3}
    .\end{align*}
    \bigbreak \noindent
    \textit{So:}
    \begin{align*}
      n^{\prime}(1.5) = 600 \cdot 3^{1.5} \cdot \ln{3} \\ 
      =3425 
    \end{align*}

    \bigbreak \noindent \bigbreak \noindent 
    \begin{Large}
        \textbf{Question 9:}
    \end{Large}
    \bigbreak \noindent 
    \pf{Solution}{}
    \bigbreak \noindent 
    \begin{align*}
      no\ work
    .\end{align*}

    \bigbreak \noindent \bigbreak \noindent 
    \begin{Large}
        \textbf{Question 10:}
    \end{Large}
    \bigbreak \noindent 
    \pf{Solution}{}
    \bigbreak \noindent 
    \begin{align*}
      no\ work
    .\end{align*}


    
\end{document}
