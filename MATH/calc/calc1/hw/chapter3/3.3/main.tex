\documentclass{report}

\input{~/dev/latex/template/preamble.tex}
\input{~/dev/latex/template/macros.tex}

\title{\Huge{3.3 Hw Solutions:}}
\author{\huge{Nathan Warner}}
\date{\huge{Feb 21, 2023}}

\pgfpagesdeclarelayout{boxed}
{
  \edef\pgfpageoptionborder{0pt}
}
{
  \pgfpagesphysicalpageoptions
  {%
    logical pages=1,%
  }
  \pgfpageslogicalpageoptions{1}
  {
    border code=\pgfsetlinewidth{1.5pt}\pgfstroke,%
    border shrink=\pgfpageoptionborder,%
    resized width=.95\pgfphysicalwidth,%
    resized height=.95\pgfphysicalheight,%
    center=\pgfpoint{.5\pgfphysicalwidth}{.5\pgfphysicalheight}%
  }%
}

\pgfpagesuselayout{boxed}

\begin{document}
    \maketitle
    \begin{Large}
        \noindent \textbf{Question 1:}
    \end{Large}
    \bigbreak \noindent 
    \pf{Solution}{}
    \bigbreak \noindent 

    \bigbreak \noindent \bigbreak \noindent 
    \begin{Large}
        \textbf{Question 2:}
    \end{Large}
    \bigbreak \noindent 
    \pf{Solution}{}
    \bigbreak \noindent 

    \bigbreak \noindent \bigbreak \noindent 
    \begin{Large}
        \textbf{Question 3:}
    \end{Large}
    \bigbreak \noindent 
    \pf{Solution}{}
    \bigbreak \noindent 


    \bigbreak \noindent \bigbreak \noindent 
    \begin{Large}
        \textbf{Question 4:}
    \end{Large}
    \bigbreak \noindent 
    \pf{Solution}{}
    \bigbreak \noindent 

    \bigbreak \noindent \bigbreak \noindent 
    \begin{Large}
        \textbf{Question 5:}
    \end{Large}
    \bigbreak \noindent 
    \pf{Solution}{}
    \bigbreak \noindent 

    \bigbreak \noindent \bigbreak \noindent 
    \begin{Large}
        \textbf{Question 6:}
    \end{Large}
    \bigbreak \noindent 
    \pf{Solution}{}
    \bigbreak \noindent 
    \begin{align*}
        f(x)= 2x \\
        f^{\prime}(x) = 2
    .\end{align*}
    \begin{align*}
        g(x) = 7-tan(x) \\
        g^{\prime}(x) = -\sec^2{x}
    .\end{align*}
    \bigbreak \noindent 
    \textit{Now:}
    \begin{align*}
        F^{\prime}(x) = \frac{(7-\tan{x})(2) - (2x)(-\sec^{2}{x})}{(7-\tan{x})^2} \\
        = \frac{14-2\tan{x}+2x\sec{x}}{(7-\tan{x})^2} \\ 
        = \frac{2(7-\tan{x}+x\sec^{2}{x})}{(7-\tan{x})^2}
    .\end{align*}
    

    \pagebreak \bigbreak \noindent
    \begin{Large}
        \textbf{Question 7:}
    \end{Large}
    \bigbreak \noindent 
    \pf{Solution}{}
    \bigbreak \noindent 
    \begin{align*}
        F(x) = \frac{1+\sec{w}}{1-\sec{w}}
    .\end{align*}
    \bigbreak \noindent 
    \textit{If:}
    \begin{align*}
        f(w) = 1+\sec{w} \\
        f^{\prime}(x) = \sec{w}\tan{w}
    .\end{align*}
    \begin{align*}
        g(w) = 1-\sec{w} \\ 
        g^{\prime}(x) = -\sec{w}\tan{w}
    .\end{align*}
    \bigbreak \noindent 
    \textit{Then:}
    \begin{align*}
        F^{\prime}(w) = \frac{(1-\sec{w})(\sec{w}\tan{w})-(1+\sec{w})(-\sec{w}\tan{w})}{(1-\sec{w})^2} \\
        = \frac{\sec{w}\tan{w}[(1-\sec{w}) - (1+\sec{w})(-1)]}{(1-\sec{w})^2} \\
        = \frac{\sec{w}\tan{w}[(1-\sec{w})+( - 1-\sec{w})(-1)]}{(1-\sec{w})^2} \\
        = \frac{\sec{w}\tan{w}[1-\sec{w}+1+\sec{w}]}{(1-\sec{w})^2} \\
        = \frac{\sec{w}\tan{w}[2]}{(1-\sec{w})^2} \\
        = \frac{2\sec{w}\tan{w}}{(1-\sec{w})^2} \\
    .\end{align*}

    \bigbreak \noindent \bigbreak \noindent 
    \begin{Large}
        \textbf{Question 8:}
    \end{Large}
    \bigbreak \noindent 
    \pf{Solution}{}
    \bigbreak \noindent 
    \textbf{\textit{\underline{Part 1:}}}
    \begin{align*}
        f(t) = t \\
        f^{\prime}(t) = 1
    .\end{align*}
    \begin{align*}
        g(t) = \sin{t} \\
        g^{\prime}(t) = \cos{t}
    .\end{align*}
    \bigbreak \noindent 
    \textit{So:}
    \begin{align*}
        F^{\prime}(t) = t \cdot \cos{t} + \sin{t} \cdot 1 \\
        = t\cos{t} + \sin{t}
    .\end{align*}
    \bigbreak \noindent 
    \textbf{\textit{\underline{Part 2:}}}
    \begin{align*}
        G^{\prime}(t) = 0 + 1 \\
        =1
    .\end{align*}
    \bigbreak \noindent 
    \textbf{\textit{\underline{Part 3:}}}
    \begin{align*}
        F^{\prime}(t) = \frac{(1+t)(t\cos{t}+\sin{t})-t\sin{t}}{(1+t)^2} \\
        = \frac{(1+t)(t\cos{t}+\sin{t})-t\sin{t}}{(1+t)^2} \\
        = \frac{t\cos{t}+\sin{t}+t^2\cos{t}+t\sin{t}-t\sin{t}}{(1+t)^2} \\
        = \frac{t\cos{t}+\sin{t}+t^2\cos{t}}{(1+t)^2} \\
    .\end{align*}

    \bigbreak \noindent \bigbreak \noindent 
    \begin{Large}
        \textbf{Question 9:}
    \end{Large}
    \bigbreak \noindent 
    \pf{Solution}{}
    \bigbreak \noindent 
    \begin{align*}
        \frac{d}{dx}\csc{x} = \frac{d}{dx}\frac{1}{\sin{x}} \\
        = \frac{\sin{x} \cdot 0 - 1 \cdot \cos{x}}{\sin^{2}{x}} \\
        = \frac{-\cos{x}}{\sin^{2}{x}} \\
        = \frac{1}{\sin{x}} \cdot \frac{\cos{x}}{\sin{x}} \\
        = -\csc{x}\cot{x}
    .\end{align*}

    \bigbreak \noindent \bigbreak \noindent 
    \begin{Large}
        \textbf{Question 10:}
    \end{Large}
    \bigbreak \noindent 
    \pf{Solution}{}
    \bigbreak \noindent
    \begin{align*}
        f(x) = 12x \\
        f^{\prime}(x) = 12
    .\end{align*}
    \begin{align*}
        g(x) = \sin{x} \\
        g^{\prime}(x) = \cos{x}
    .\end{align*}
    \bigbreak \noindent
    \textit{Now:}
    \begin{align*}
      y^{\prime} = 12x \cdot \cos{x} + \sin{x} \cdot 12 \\
      = 12x\cos{x} + 12\sin{x} \\
      = 12(x\cos{x}+\sin{x})
    \end{align*}
    \bigbreak \noindent 
    \textbf{\textit{\underline{Second part of a.)}}}
    \begin{align*}
        12(\frac{\pi}{2}\cos{\frac{\pi}{2}+\sin{\frac{\pi}{2}}}) \\
        = 12(\frac{\pi}{2}(0)+1) \\
        =12(1) \\
        =12
    .\end{align*}
    \bigbreak \noindent 
    \textit{With this:}
    \begin{align*}
        y-y_1 = m\left(x-x_1\right) \\
        so\ \\
        y - 6\pi = 12(x-\frac{\pi}{2}) \\
        y - 6\pi = 12x - 6\pi \\
        = y = 12x
    .\end{align*}

    \bigbreak \noindent \bigbreak \noindent 
    \begin{Large}
        \textbf{Question 11:}
    \end{Large}
    \bigbreak \noindent 
    \pf{Solution}{}
    \bigbreak \noindent 

    \bigbreak \noindent \bigbreak \noindent 
    \begin{Large}
        \textbf{Question 12:}
    \end{Large}
    \bigbreak \noindent 
    \pf{Solution}{}
    \bigbreak \noindent 

    \bigbreak \noindent \bigbreak \noindent 
    \begin{Large}
        \textbf{Question 13:}
    \end{Large}
    \bigbreak \noindent 
    \pf{Solution}{}
    \bigbreak \noindent 

    \bigbreak \noindent \bigbreak \noindent 
    \begin{Large}
        \textbf{Question 14:}
    \end{Large}
    \bigbreak \noindent 
    \pf{Solution}{}
    \bigbreak \noindent 
    
\end{document}
