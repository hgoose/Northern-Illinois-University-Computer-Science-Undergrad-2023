\documentclass{report}

\input{~/dev/latex/template/preamble.tex}
\input{~/dev/latex/template/macros.tex}

\title{\Huge{}}
\author{\huge{Nathan Warner}}
\date{\huge{}}
\pagestyle{fancy}
\fancyhf{}
\lhead{Warner \thepage}
\rhead{}
% \lhead{\leftmark}
\cfoot{\thepage}
%\setborder
% \usepackage[default]{sourcecodepro}
% \usepackage[T1]{fontenc}

\begin{document}
    % \maketitle
        \begin{titlepage}
       \begin{center}
           \vspace*{1cm}
    
           \textbf{Calculus 2} \\
           Chapter 3
    
           \vspace{0.5cm}
            
                
           \vspace{1.5cm}
    
           \textbf{Nathan Warner}
    
           \vfill
                
                
           \vspace{0.8cm}
         
           \includegraphics[width=0.4\textwidth]{}
                
           Computer Science \\
           Northern Illinois University\\
           September 27, 2023 \\
           United States\\
           
                
       \end{center}
    \end{titlepage}
    \tableofcontents
    \pagebreak \bigbreak \noindent
    \vspace{2in} \\
    \begin{Huge}
       \textbf{Techniques of Integration} 
    \end{Huge}
    \bigbreak \noindent 
    \line(1,0){490}
    \bigbreak \noindent 
    
    \phantomsection
    \addcontentsline{toc}{section}{\Large 3.1 Integration by Parts}
    \section*{\Large 3.1 Integration by Parts}
    \bigbreak \noindent 
    \smallbreak \noindent
    \begin{definition}
        Many students want to know whether there is a product rule for integration. There isn’t, but there is a technique based on the product rule for differentiation that allows us to exchange one integral for another. We call this technique \textbf{integration by parts.}
    \end{definition}

    \bigbreak \noindent 
    \phantomsection
    \addcontentsline{toc}{subsection}{The Integration-by-Parts Formula}
    \subsection*{The Integration-by-Parts Formula}
    \bigbreak \noindent 
    If, \( h(x) = f(x)g(x) \),
then by using the product rule, we obtain \( h'(x) = f'(x)g(x) + g'(x)f(x) \).
Although at first it may seem counterproductive, let’s now integrate both sides of this equation:
\[ \int h'(x) \, dx = \int \left( g(x)f'(x) + f(x)g'(x) \right) \, dx. \]
\bigbreak \noindent 
This gives us
\[ h(x) = f(x)g(x) = \int g(x)f'(x) \, dx + \int f(x)g'(x) \, dx. \]
Now we solve for \( \int f(x)g'(x) \, dx \):
\[ \int f(x)g'(x) \, dx = f(x)g(x) - \int g(x)f'(x) \, dx. \]
By making the substitutions \( u = f(x) \)
and \( v = g(x) \),
which in turn make \( du = f'(x) \, dx \)
and \( dv = g'(x) \, dx \),
we have the more compact form
\[ \int u \, dv = uv - \int v \, du. \]


    \pagebreak \bigbreak \noindent 
    \begin{thrm}[Integration by Parts]
        Let \( u = f(x) \)  and \( v = g(x) \)  be functions with continuous derivatives. Then, the integration-by-parts formula for the integral involving these two functions is: \[ \int u \, dv = uv - \int v \, du. \]
    \end{thrm}

    \bigbreak \noindent \bigbreak \noindent 
    \begin{eg}[Using Integration by Parts]
        Use integration by parts with \( u = x \)
and \( dv = \sin x \, dx \)
to evaluate 
\[ \int x \sin x \, dx. \]
        
    \end{eg}
    \bigbreak \noindent 
    \pf{Solution}{
        So to use the formula:
        \begin{align*}
            \int_{}^{}\ u\ dv = uv - \int v\ du
        .\end{align*}
        \bigbreak \noindent 
        We need:
        \begin{align*}
            u = x \quad du = dx \\
            dv = \sin{x}dx \quad v = -\cos{x}
        .\end{align*}
        \bigbreak \noindent 
        Thus:
        \begin{align*}
            \int x\sin{x}\ dx &= -x\cos{x} - \int -\cos{x}\ dx \\
                              &=-x\cos{x} + \sin{x} + C
        .\end{align*}
    

    }
    \bigbreak \noindent 
    The natural question to ask at this point is: How do we know how to choose  u
  and  dv?
  Sometimes it is a matter of trial and error; however, the acronym \textbf{LIATE} can often help to take some of the guesswork out of our choices. This acronym stands for 
  \begin{itemize}
      \item \textbf{L}ogarithmic Functions
      \item \textbf{I}nverse Trigonometric Functions
      \item \textbf{A}lgebraic Functions
      \item \textbf{T}rigonometric Functions
        \item \textbf{E}ponential Functions
  \end{itemize}
    This mnemonic serves as an aid in determining an appropriate choice for  u.
    
    



























\end{document}
