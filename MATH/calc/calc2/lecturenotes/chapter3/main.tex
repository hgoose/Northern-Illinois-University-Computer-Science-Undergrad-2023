\documentclass{report}

\input{~/dev/latex/template/preamble.tex}
\input{~/dev/latex/template/macros.tex}

\title{\Huge{}}
\author{\huge{Nathan Warner}}
\date{\huge{}}
\pagestyle{fancy}
\fancyhf{}
\lhead{Warner \thepage}
\rhead{}
% \lhead{\leftmark}
\cfoot{\thepage}
%\setborder
% \usepackage[default]{sourcecodepro}
% \usepackage[T1]{fontenc}

\begin{document}
    % \maketitle
        \begin{titlepage}
       \begin{center}
           \vspace*{1cm}
    
           \textbf{Calculus 2} \\
           Chapter 3
    
           \vspace{0.5cm}
            
                
           \vspace{1.5cm}
    
           \textbf{Nathan Warner}
    
           \vfill
                
                
           \vspace{0.8cm}
         
           \includegraphics[width=0.4\textwidth]{~/niu/seal.png}
                
           Computer Science \\
           Northern Illinois University\\
           September 27, 2023 \\
           United States\\
           
                
       \end{center}
    \end{titlepage}
    \tableofcontents
    \pagebreak \bigbreak \noindent
    \vspace{2in} \\
    \begin{Huge}
       \textbf{Techniques of Integration} 
    \end{Huge}
    \bigbreak \noindent 
    \line(1,0){490}
    \bigbreak \noindent 
    
    \phantomsection
    \addcontentsline{toc}{section}{\Large 3.1 Integration by Parts}
    \section*{\Large 3.1 Integration by Parts}
    \bigbreak \noindent 
    \smallbreak \noindent
    \begin{definition}
        Many students want to know whether there is a product rule for integration. There isn’t, but there is a technique based on the product rule for differentiation that allows us to exchange one integral for another. We call this technique \textbf{integration by parts.}
    \end{definition}

    \bigbreak \noindent 
    \phantomsection
    \addcontentsline{toc}{subsection}{The Integration-by-Parts Formula}
    \subsection*{The Integration-by-Parts Formula}
    \bigbreak \noindent 
    If, \( h(x) = f(x)g(x) \),
then by using the product rule, we obtain \( h'(x) = f'(x)g(x) + g'(x)f(x) \).
Although at first it may seem counterproductive, let’s now integrate both sides of this equation:
\[ \int h'(x) \, dx = \int \left( g(x)f'(x) + f(x)g'(x) \right) \, dx. \]
\bigbreak \noindent 
This gives us
\[ h(x) = f(x)g(x) = \int g(x)f'(x) \, dx + \int f(x)g'(x) \, dx. \]
Now we solve for \( \int f(x)g'(x) \, dx \):
\[ \int f(x)g'(x) \, dx = f(x)g(x) - \int g(x)f'(x) \, dx. \]
By making the substitutions \( u = f(x) \)
and \( v = g(x) \),
which in turn make \( du = f'(x) \, dx \)
and \( dv = g'(x) \, dx \),
we have the more compact form
\[ \int u \, dv = uv - \int v \, du. \]


    \pagebreak \bigbreak \noindent 
    \begin{thrm}[Integration by Parts]
        Let \( u = f(x) \)  and \( v = g(x) \)  be functions with continuous derivatives. Then, the integration-by-parts formula for the integral involving these two functions is: \[ \int u \, dv = uv - \int v \, du. \]
    \end{thrm}

    \bigbreak \noindent \bigbreak \noindent 
    \begin{eg}[Using Integration by Parts]
        Use integration by parts with \( u = x \)
and \( dv = \sin x \, dx \)
to evaluate 
\[ \int x \sin x \, dx. \]
        
    \end{eg}
    \bigbreak \noindent 
    \pf{Solution}{
        So to use the formula:
        \begin{align*}
            \int_{}^{}\ u\ dv = uv - \int v\ du
        .\end{align*}
        \bigbreak \noindent 
        We need:
        \begin{align*}
            u = x \quad du = dx \\
            dv = \sin{x}dx \quad v = -\cos{x}
        .\end{align*}
        \bigbreak \noindent 
        Thus:
        \begin{align*}
            \int x\sin{x}\ dx &= -x\cos{x} - \int -\cos{x}\ dx \\
                              &=-x\cos{x} + \sin{x} + C
        .\end{align*}
    

    }
    \bigbreak \noindent 
    The natural question to ask at this point is: How do we know how to choose  u
  and  dv?
  Sometimes it is a matter of trial and error; however, the acronym \textbf{LIATE} can often help to take some of the guesswork out of our choices. This acronym stands for 
  \begin{itemize}
      \item \textbf{L}ogarithmic Functions
      \item \textbf{I}nverse Trigonometric Functions
      \item \textbf{A}lgebraic Functions
      \item \textbf{T}rigonometric Functions
        \item \textbf{E}ponential Functions
  \end{itemize}
    This mnemonic serves as an aid in determining an appropriate choice for  u.

    \pagebreak 
    \phantomsection
    \addcontentsline{toc}{subsection}{Applying integration by parts more than once}
    \subsection*{Applying integration by parts more than once}
    \bigbreak \noindent 
    \begin{eg}[Evaluate]
       \begin{align*}
            \int x^{2}e^{3x}\ dx
       .\end{align*}
    \end{eg}
    \bigbreak \noindent 
    \pf{Solution}{
        By \textbf{LIATE}, we let $u=x^{2}$, and $dv= e^{3x}$. Thus, we get:
        \begin{align*}
            &u=x^{2} \quad dv = e^{3x} \\
            &du=2xdx \quad v = \frac{1}{3}e^{3x}
        .\end{align*}
        \bigbreak \noindent 
        Then by theorem 1, we get:
        \begin{align*}
            \int udv &= uv - \int vdu \\
            &=\int x^{2}e^{3x}\ dx = x^{2}\frac{1}{3}e^{3x} - \int \frac{1}{3}e^{3x}2x\ dx \\
            &=\int x^{2}e^{3x}\ dx = x^{2}\frac{1}{3}e^{3x} - \int \frac{2}{3}e^{3x}x\ dx \\
        .\end{align*}
        \bigbreak \noindent 
        At this point, we will notice that we still cannot evaluate the integral $\int \frac{2}{3}e^{3x}x\ dx $. Thus, we must apply the theorem once more.
        \bigbreak \noindent 
        \begin{minipage}[]{0.47\textwidth}
            \begin{align*}
            &\int \frac{2}{3}e^{3x}x\ dx  \\
            &u = x \quad dv = \frac{2}{3}e^{3x} \\
            &du = dx \quad v = \frac{2}{9}e^{3x}
        .\end{align*}
        \end{minipage}
        \begin{minipage}[]{0.47\textwidth}
            Thus:
            \begin{align*}
                \int \frac{2}{3}e^{3x}\ dx &= \frac{2}{9}e^{3x}x - \int \frac{2}{9}e^{3x}\ dx \\
                                       &=\frac{2}{9}xe^{3x} - \frac{2}{27}e^{3x}
            .\end{align*}
        \end{minipage}
        \bigbreak \noindent 
        In full we have:
        \begin{align*}
            &\int x^{2}e^{3x}\ dx = \frac{1}{3}x^{2}e^{3x} - \left(\frac{2}{9}xe^{3x}-\frac{2}{27}e^{3x}\right) \\
            &=\frac{1}{3}e^{3x}x^{2}-\frac{2}{9}xe^{3x}+\frac{2}{27}e^{3x} + C
        .\end{align*}
    
    }

    \pagebreak 
    \phantomsection
    \addcontentsline{toc}{subsection}{Applying Integration by Parts When LIATE Doesn’t Quite Work}
    \subsection*{Applying Integration by Parts When LIATE Doesn’t Quite Work}
    \bigbreak \noindent 
    \begin{eg}[Evaluate]
        \begin{align*}
            \int t^{3}e^{t^{2}}\ dt
        .\end{align*}
    \end{eg}
    \bigbreak \noindent 
    \pf{Solution}{
        If we use a strict interpretation of the mnemonic \textbf{LIATE} to make our choice of  $u$, we end up with  $u = t^3$  and  $dv = e^{t^2} dt$. 
        Unfortunately, this choice won’t work because we are unable to evaluate  $\int e^{t^2}\ dt$.  However, since we can evaluate  $\int t e^{t^2}\ dt$, we can try choosing  $u = t^2$ and  $dv = t e^{t^2}\ dt$. With these choices we have
        \begin{align*}
            &u = t^{2} \quad dv = te^{t^{2}} \\
            &du = 2t\ dt \quad v = \frac{1}{2}e^{t^{2}}
        .\end{align*}
        \bigbreak \noindent 
        Thus, we obtain:
        \begin{align*}
            \int t^{3}e^{t^{2}}\ dt &= \frac{1}{2}t^{2}e^{t^{2}} - \int \frac{1}{2}e^{t^{2}}2t\\
            &= \frac{1}{2}t^{2}e^{t^{2}} - \int e^{t^{2}}t\ dt \\
            &= \frac{1}{2}t^{2}e^{t^{2}} - \frac{1}{2}e^{t^{2}} + C 
        .\end{align*}

        }

        \bigbreak \noindent 
        \begin{eg}[Evaluate]
            \begin{align*}
                \int \sin{(\ln{(x)})}\ dx
            .\end{align*}
            \bigbreak \noindent 
        \end{eg}
        \pf{Solution}{
            Here, we let $u = \sin{(\ln{(x)})}$ and $dv = 1dx$, so we have:
            \begin{align*}
                &u = \sin{(\ln{(x)})} \quad dv = dx \\
                &du = \frac{1}{x}\cos{(\ln{(x)})}\ dx \quad v = x
            .\end{align*}
            \bigbreak \noindent 
            Which gives us:
            \begin{align*}
                \int \sin{(\ln{(x)})}\ dx = x\sin{(\ln{(x)})} - \int \cos{(\ln{(x)})}\ dx
            .\end{align*}
            \bigbreak \noindent 
            Which leaves us in no better shape than the original integral, so we apply the theorem once more:
            \bigbreak \noindent 
            \begin{minipage}[]{0.47\textwidth}
                \begin{align*}
                    &\int \cos{(\ln{(x)})} \\
                    &u = \cos{(\ln{(x)})} \quad dv = dx \\
                    &du = -\frac{1}{x}\sin{(\ln{(x)})}\ dx \quad v = x
                .\end{align*}
            \end{minipage}
            \begin{minipage}[]{0.47\textwidth}
                Thus we have:
                \begin{align*}
                    \int \cos{(\ln{(x)})} &= x\cos{(\ln{(x)})} - \int -\sin{(\ln{(x)})}\ dx
                .\end{align*}
            \end{minipage}
            \bigbreak \noindent 
            At this point, we have:
            \begin{align*}
                \int \sin{(\ln{(x)})}\ dx &= x\sin{(\ln{(x)})} - \left(x\cos{(\ln{(x)})}- \int -\sin{(ln(x))}\ dx\right) \\
                &= x\sin{(\ln{(x)})} - \left(x\cos{(\ln{(x)})}+ \int \sin{(ln(x))}\ dx\right) \\
                &= x\sin{(\ln{(x)})} - x\cos{(\ln{(x)})}- \int \sin{(ln(x))}\ dx \\
            .\end{align*}
            \bigbreak \noindent 
            The last integral is now the same as the original. It may seem that we have simply gone in a circle, but now we can actually evaluate the integral. To see how to do this more clearly, substitute:
            \begin{align*}
                I = \int \sin{(\ln{(x)})}\ dx
            .\end{align*}
            \bigbreak \noindent 
            Thus, the equation becomes:
            \begin{align*}
                I &= x\sin{(\ln{(x)})} - x\cos{(\ln{(x)})} - I \\
                  2I&=x\sin{(\ln{(x)})} - x\cos{(\ln{(x)})}  \\
                    I&= \frac{1}{2}x\sin{(\ln{(x)})} - \frac{1}{2}x\cos{(\ln{(x)})}
            .\end{align*}
            \bigbreak \noindent 
            Substituting back in for $I$ we get:
            \begin{align*}
                \int \sin{(\ln{(x)})}\ dx  = \frac{1}{2}x\sin{(\ln{(x)})} - \frac{1}{2}x\cos{(\ln{(x)})} + C
            .\end{align*}
            \bigbreak \noindent 
        }

        \bigbreak \noindent 
        \phantomsection
        \addcontentsline{toc}{subsection}{Integration by parts for definite integrals}
        \subsection*{Integration by parts for definite integrals}
        \bigbreak \noindent 
        Now that we have used integration by parts successfully to evaluate indefinite integrals, we turn our attention to definite integrals. The integration technique is really the same, only we add a step to evaluate the integral at the upper and lower limits of integration.
        \bigbreak \noindent 
        \begin{thrm}[Integration by Parts for Definite Integrals]
           Let  $u = f(x)$  and  $v = g(x)$ be functions with continuous derivatives on $[a,b]$. Then:
           \begin{align*}
               \int_{a}^{b} u \, dv &= uv\big|_{a}^{b} - \int_{a}^{b} v \, du
           .\end{align*}
        \end{thrm}

        \pagebreak 
        \phantomsection
        \addcontentsline{toc}{section}{3.2 Trigonometric Integrals}
        \section*{3.2 Trigonometric Integrals}
        \bigbreak \noindent 
        In this section we look at how to integrate a variety of products of trigonometric functions. These integrals are called trigonometric integrals. They are an important part of the integration technique called trigonometric substitution, which is featured in Trigonometric Substitution. This technique allows us to convert algebraic expressions that we may not be able to integrate into expressions involving trigonometric functions, which we may be able to integrate using the techniques described in this section. In addition, these types of integrals appear frequently when we study polar, cylindrical, and spherical coordinate systems later. Let’s begin our study with products of $\sin{x}$ and $\cos{x}$.

        \bigbreak \noindent 
        \phantomsection
        \addcontentsline{toc}{subsection}{Integrating Products and Powers of sinx and cosx}
        \subsection*{Integrating $\cos^{j}{x}\sin{x}$}
        \bigbreak \noindent 
        In this case, we can preform a simple u-substation, where we let $u=\cos{x}$, and from there we can evaluate.
        \bigbreak \noindent 
        \begin{eg}[Evaluate]
           \begin{align*}
               &\int \cos^{5}{x}\sin{x}\ dx \\
               &=- \int u^{5}\ du \\
               &-\frac{1}{6}u^{6} + C \\
               &= -\frac{1}{6}\cos^{6}{x} + C
           .\end{align*} 
        \end{eg}
        
        \bigbreak \noindent 
        \phantomsection
        \addcontentsline{toc}{subsection}{Integrating $\cos^{j}{x}\sin^{k}{x}$ when $k$ is odd}
        \subsection*{Integrating $\cos^{j}{x}\sin^{k}{x}$ when $k$ is odd}
        \bigbreak \noindent 
        In this case, we can use the trigonometric identity: $\sin^{2}{x} = 1-\cos^{2}{x}$ to rewrite the expression such that using a u-substation will work. In general:
        \begin{align*}
            &\int \cos^{j}{x}\sin^{k}{x}\ dx\quad \text{s.t}\ $k = 2l+1\ l \in \mathbb{Z}$ \\
            &=\int \cos^{j}{x}(1-\cos^{2}{k})^{\frac{k-1}{2}}\sin{(x)}
        .\end{align*}
        \bigbreak \noindent 
        \begin{eg}[Evaluate]
            \begin{align*}
                &\int \cos^{2}{x}\sin^{5}{x}\ dx \\
                &=\int\cos^{2}{x}(1-\cos^{2}{x})^{\frac{5-1}{2}}\sin{x}\ dx \\
                &=\int \cos^{2}{x}(1-\cos^{2}{x})^{2}\sin{x}\ dx \\
                &=etc...
            .\end{align*}
        \end{eg}
        \bigbreak \noindent 
        \nt{This fact also holds for $\int \sin^{j}{x}\cos^{k}{x}\, for\ k = 2l+1,\ l \in \mathbb{Z}$}

        \pagebreak 
        \phantomsection
        \addcontentsline{toc}{subsection}{Integrating even powers of $\sin{x}$}
        \subsection*{Integrating even powers of $\sin{x}$}
        \bigbreak \noindent 
        In the next example, we see the strategy that must be applied when there are only even powers of \( \sin(x) \) and \( \cos(x) \). For integrals of this type, the identities
        \[
        \sin^2(x) = \frac{1}{2} - \frac{1}{2}\cos(2x) = \frac{1 - \cos(2x)}{2}
        \]
        and
        \[
        \cos^2(x) = \frac{1}{2} + \frac{1}{2}\cos(2x) = \frac{1 + \cos(2x)}{2}
        \]
        are invaluable. These identities are sometimes known as power-reducing identities and they may be derived from the double-angle identity  $\cos(2x) = \cos^2(x) - \sin^2(x)$ and the Pythagorean identity  $\cos^2(x) + \sin^2(x) = 1$

        \bigbreak \noindent 
        \begin{eg}[Evaluate]
           \begin{align*}
               &\int \sin^{2}{x}\ dx
           .\end{align*} 
           \bigbreak \noindent 
           By the identity described above, we can derive:
           \begin{align*}
               \cos{2x} &= 1-2\sin^{2}{x} \\
               \sin^{2}{x} &= \frac{1}{2}-\frac{1}{2}\cos{2x} 
           .\end{align*}
           \bigbreak \noindent 
           Thus we have:
           \begin{align*}
               &\int \frac{1}{2}-\frac{1}{2}\cos{2x}\ dx \\
               &=\frac{1}{2}x - \frac{1}{4}\sin{2x} + C
           .\end{align*}
        \end{eg}

        \pagebreak 
        \phantomsection
        \addcontentsline{toc}{subsection}{Problem-Solving Strategy: Integrating Products and Powers of sin x and cos x}
        \subsection*{Problem-Solving Strategy: Integrating Products and Powers of sin x and cos x}
        \bigbreak \noindent 
        To integrate 
\[
\int \cos^j(x) \sin^k(x) \, dx
\]
use the following strategies:

\begin{enumerate}
    \item If \( k \) is odd, rewrite \( \sin^k(x) \) as \( \sin^{k-1}(x) \sin(x) \) and use the identity \( \sin^2(x) = 1 - \cos^2(x) \) to rewrite \( \sin^{k-1}(x) \) in terms of \( \cos(x) \). Integrate using the substitution \( u = \cos(x) \). This substitution makes \( du = -\sin(x) \, dx \).
    
    \item If \( j \) is odd, rewrite \( \cos^j(x) \) as \( \cos^{j-1}(x) \cos(x) \) and use the identity \( \cos^2(x) = 1 - \sin^2(x) \) to rewrite \( \cos^{j-1}(x) \) in terms of \( \sin(x) \). Integrate using the substitution \( u = \sin(x) \). This substitution makes \( du = \cos(x) \, dx \). (Note: If both \( j \) and \( k \) are odd, either strategy 1 or strategy 2 may be used.)
    
    \item If both \( j \) and \( k \) are even, use 
    \[
    \sin^2(x) = \frac{1}{2} - \frac{1}{2} \cos(2x)
    \]
    and 
    \[
    \cos^2(x) = \frac{1}{2} + \frac{1}{2} \cos(2x).
    \]
    After applying these formulas, simplify and reapply strategies 1 through 3 as appropriate.
\end{enumerate}


        

        


        
        

    
    
    






\end{document}
