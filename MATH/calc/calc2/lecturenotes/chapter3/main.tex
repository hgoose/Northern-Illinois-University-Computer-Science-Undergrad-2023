\documentclass{report}

\input{~/dev/latex/template/preamble.tex}
\input{~/dev/latex/template/macros.tex}

\title{\Huge{}}
\author{\huge{Nathan Warner}}
\date{\huge{}}
\pagestyle{fancy}
\fancyhf{}
\lhead{Warner \thepage}
\rhead{}
% \lhead{\leftmark}
\cfoot{\thepage}
%\setborder
% \usepackage[default]{sourcecodepro}
% \usepackage[T1]{fontenc}

\begin{document}
    % \maketitle
        \begin{titlepage}
       \begin{center}
           \vspace*{1cm}
    
           \textbf{Calculus 2} \\
           Chapter 3
    
           \vspace{0.5cm}
            
                
           \vspace{1.5cm}
    
           \textbf{Nathan Warner}
    
           \vfill
                
                
           \vspace{0.8cm}
         
           \includegraphics[width=0.4\textwidth]{~/niu/seal.png}
                
           Computer Science \\
           Northern Illinois University\\
           September 27, 2023 \\
           United States\\
           
                
       \end{center}
    \end{titlepage}
    \tableofcontents
    \pagebreak \bigbreak \noindent
    \vspace{2in} \\
    \begin{Huge}
       \textbf{Techniques of Integration} 
    \end{Huge}
    \bigbreak \noindent 
    \line(1,0){490}
    \bigbreak \noindent 
    
    \phantomsection
    \addcontentsline{toc}{section}{\Large 3.1 Integration by Parts}
    \section*{\Large 3.1 Integration by Parts}
    \bigbreak \noindent 
    \smallbreak \noindent
    \begin{definition}
        Many students want to know whether there is a product rule for integration. There isn’t, but there is a technique based on the product rule for differentiation that allows us to exchange one integral for another. We call this technique \textbf{integration by parts.}
    \end{definition}

    \bigbreak \noindent 
    \phantomsection
    \addcontentsline{toc}{subsection}{The Integration-by-Parts Formula}
    \subsection*{The Integration-by-Parts Formula}
    \bigbreak \noindent 
    If, \( h(x) = f(x)g(x) \),
then by using the product rule, we obtain \( h'(x) = f'(x)g(x) + g'(x)f(x) \).
Although at first it may seem counterproductive, let’s now integrate both sides of this equation:
\[ \int h'(x) \, dx = \int \left( g(x)f'(x) + f(x)g'(x) \right) \, dx. \]
\bigbreak \noindent 
This gives us
\[ h(x) = f(x)g(x) = \int g(x)f'(x) \, dx + \int f(x)g'(x) \, dx. \]
Now we solve for \( \int f(x)g'(x) \, dx \):
\[ \int f(x)g'(x) \, dx = f(x)g(x) - \int g(x)f'(x) \, dx. \]
By making the substitutions \( u = f(x) \)
and \( v = g(x) \),
which in turn make \( du = f'(x) \, dx \)
and \( dv = g'(x) \, dx \),
we have the more compact form
\[ \int u \, dv = uv - \int v \, du. \]


    \pagebreak \bigbreak \noindent 
    \begin{thrm}[Integration by Parts]
        Let \( u = f(x) \)  and \( v = g(x) \)  be functions with continuous derivatives. Then, the integration-by-parts formula for the integral involving these two functions is: \[ \int u \, dv = uv - \int v \, du. \]
    \end{thrm}

    \bigbreak \noindent \bigbreak \noindent 
    \begin{eg}[Using Integration by Parts]
        Use integration by parts with \( u = x \)
and \( dv = \sin x \, dx \)
to evaluate 
\[ \int x \sin x \, dx. \]
        
    \end{eg}
    \bigbreak \noindent 
    \pf{Solution}{
        So to use the formula:
        \begin{align*}
            \int_{}^{}\ u\ dv = uv - \int v\ du
        .\end{align*}
        \bigbreak \noindent 
        We need:
        \begin{align*}
            u = x \quad du = dx \\
            dv = \sin{x}dx \quad v = -\cos{x}
        .\end{align*}
        \bigbreak \noindent 
        Thus:
        \begin{align*}
            \int x\sin{x}\ dx &= -x\cos{x} - \int -\cos{x}\ dx \\
                              &=-x\cos{x} + \sin{x} + C
        .\end{align*}
    

    }
    \bigbreak \noindent 
    The natural question to ask at this point is: How do we know how to choose  u
  and  dv?
  Sometimes it is a matter of trial and error; however, the acronym \textbf{LIATE} can often help to take some of the guesswork out of our choices. This acronym stands for 
  \begin{itemize}
      \item \textbf{L}ogarithmic Functions
      \item \textbf{I}nverse Trigonometric Functions
      \item \textbf{A}lgebraic Functions
      \item \textbf{T}rigonometric Functions
        \item \textbf{E}ponential Functions
  \end{itemize}
    This mnemonic serves as an aid in determining an appropriate choice for  u.
    \bigbreak \noindent 
    \nt{A better version might be LIAET, where exponential and trig functions are swapped}

    \pagebreak 
    \phantomsection
    \addcontentsline{toc}{subsection}{Applying integration by parts more than once}
    \subsection*{Applying integration by parts more than once}
    \bigbreak \noindent 
    \begin{eg}[Evaluate]
       \begin{align*}
            \int x^{2}e^{3x}\ dx
       .\end{align*}
    \end{eg}
    \bigbreak \noindent 
    \pf{Solution}{
        By \textbf{LIATE}, we let $u=x^{2}$, and $dv= e^{3x}$. Thus, we get:
        \begin{align*}
            &u=x^{2} \quad dv = e^{3x} \\
            &du=2xdx \quad v = \frac{1}{3}e^{3x}
        .\end{align*}
        \bigbreak \noindent 
        Then by theorem 1, we get:
        \begin{align*}
            \int udv &= uv - \int vdu \\
            &=\int x^{2}e^{3x}\ dx = x^{2}\frac{1}{3}e^{3x} - \int \frac{1}{3}e^{3x}2x\ dx \\
            &=\int x^{2}e^{3x}\ dx = x^{2}\frac{1}{3}e^{3x} - \int \frac{2}{3}e^{3x}x\ dx \\
        .\end{align*}
        \bigbreak \noindent 
        At this point, we will notice that we still cannot evaluate the integral $\int \frac{2}{3}e^{3x}x\ dx $. Thus, we must apply the theorem once more.
        \bigbreak \noindent 
        \begin{minipage}[]{0.47\textwidth}
            \begin{align*}
            &\int \frac{2}{3}e^{3x}x\ dx  \\
            &u = x \quad dv = \frac{2}{3}e^{3x} \\
            &du = dx \quad v = \frac{2}{9}e^{3x}
        .\end{align*}
        \end{minipage}
        \begin{minipage}[]{0.47\textwidth}
            Thus:
            \begin{align*}
                \int \frac{2}{3}e^{3x}\ dx &= \frac{2}{9}e^{3x}x - \int \frac{2}{9}e^{3x}\ dx \\
                                       &=\frac{2}{9}xe^{3x} - \frac{2}{27}e^{3x}
            .\end{align*}
        \end{minipage}
        \bigbreak \noindent 
        In full we have:
        \begin{align*}
            &\int x^{2}e^{3x}\ dx = \frac{1}{3}x^{2}e^{3x} - \left(\frac{2}{9}xe^{3x}-\frac{2}{27}e^{3x}\right) \\
            &=\frac{1}{3}e^{3x}x^{2}-\frac{2}{9}xe^{3x}+\frac{2}{27}e^{3x} + C
        .\end{align*}
    
    }

    \pagebreak 
    \phantomsection
    \addcontentsline{toc}{subsection}{Applying Integration by Parts When LIATE Doesn’t Quite Work}
    \subsection*{Applying Integration by Parts When LIATE Doesn’t Quite Work}
    \bigbreak \noindent 
    \begin{eg}[Evaluate]
        \begin{align*}
            \int t^{3}e^{t^{2}}\ dt
        .\end{align*}
    \end{eg}
    \bigbreak \noindent 
    \pf{Solution}{
        If we use a strict interpretation of the mnemonic \textbf{LIATE} to make our choice of  $u$, we end up with  $u = t^3$  and  $dv = e^{t^2} dt$. 
        Unfortunately, this choice won’t work because we are unable to evaluate  $\int e^{t^2}\ dt$.  However, since we can evaluate  $\int t e^{t^2}\ dt$, we can try choosing  $u = t^2$ and  $dv = t e^{t^2}\ dt$. With these choices we have
        \begin{align*}
            &u = t^{2} \quad dv = te^{t^{2}} \\
            &du = 2t\ dt \quad v = \frac{1}{2}e^{t^{2}}
        .\end{align*}
        \bigbreak \noindent 
        Thus, we obtain:
        \begin{align*}
            \int t^{3}e^{t^{2}}\ dt &= \frac{1}{2}t^{2}e^{t^{2}} - \int \frac{1}{2}e^{t^{2}}2t\\
            &= \frac{1}{2}t^{2}e^{t^{2}} - \int e^{t^{2}}t\ dt \\
            &= \frac{1}{2}t^{2}e^{t^{2}} - \frac{1}{2}e^{t^{2}} + C 
        .\end{align*}

        }

        \bigbreak \noindent 
        \begin{eg}[Evaluate]
            \begin{align*}
                \int \sin{(\ln{(x)})}\ dx
            .\end{align*}
            \bigbreak \noindent 
        \end{eg}
        \pf{Solution}{
            Here, we let $u = \sin{(\ln{(x)})}$ and $dv = 1dx$, so we have:
            \begin{align*}
                &u = \sin{(\ln{(x)})} \quad dv = dx \\
                &du = \frac{1}{x}\cos{(\ln{(x)})}\ dx \quad v = x
            .\end{align*}
            \bigbreak \noindent 
            Which gives us:
            \begin{align*}
                \int \sin{(\ln{(x)})}\ dx = x\sin{(\ln{(x)})} - \int \cos{(\ln{(x)})}\ dx
            .\end{align*}
            \bigbreak \noindent 
            Which leaves us in no better shape than the original integral, so we apply the theorem once more:
            \bigbreak \noindent 
            \begin{minipage}[]{0.47\textwidth}
                \begin{align*}
                    &\int \cos{(\ln{(x)})} \\
                    &u = \cos{(\ln{(x)})} \quad dv = dx \\
                    &du = -\frac{1}{x}\sin{(\ln{(x)})}\ dx \quad v = x
                .\end{align*}
            \end{minipage}
            \begin{minipage}[]{0.47\textwidth}
                Thus we have:
                \begin{align*}
                    \int \cos{(\ln{(x)})} &= x\cos{(\ln{(x)})} - \int -\sin{(\ln{(x)})}\ dx
                .\end{align*}
            \end{minipage}
            \bigbreak \noindent 
            At this point, we have:
            \begin{align*}
                \int \sin{(\ln{(x)})}\ dx &= x\sin{(\ln{(x)})} - \left(x\cos{(\ln{(x)})}- \int -\sin{(ln(x))}\ dx\right) \\
                &= x\sin{(\ln{(x)})} - \left(x\cos{(\ln{(x)})}+ \int \sin{(ln(x))}\ dx\right) \\
                &= x\sin{(\ln{(x)})} - x\cos{(\ln{(x)})}- \int \sin{(ln(x))}\ dx \\
            .\end{align*}
            \bigbreak \noindent 
            The last integral is now the same as the original. It may seem that we have simply gone in a circle, but now we can actually evaluate the integral. To see how to do this more clearly, substitute:
            \begin{align*}
                I = \int \sin{(\ln{(x)})}\ dx
            .\end{align*}
            \bigbreak \noindent 
            Thus, the equation becomes:
            \begin{align*}
                I &= x\sin{(\ln{(x)})} - x\cos{(\ln{(x)})} - I \\
                  2I&=x\sin{(\ln{(x)})} - x\cos{(\ln{(x)})}  \\
                    I&= \frac{1}{2}x\sin{(\ln{(x)})} - \frac{1}{2}x\cos{(\ln{(x)})}
            .\end{align*}
            \bigbreak \noindent 
            Substituting back in for $I$ we get:
            \begin{align*}
                \int \sin{(\ln{(x)})}\ dx  = \frac{1}{2}x\sin{(\ln{(x)})} - \frac{1}{2}x\cos{(\ln{(x)})} + C
            .\end{align*}
            \bigbreak \noindent 
        }

        \bigbreak \noindent 
        \phantomsection
        \addcontentsline{toc}{subsection}{Integration by parts for definite integrals}
        \subsection*{Integration by parts for definite integrals}
        \bigbreak \noindent 
        Now that we have used integration by parts successfully to evaluate indefinite integrals, we turn our attention to definite integrals. The integration technique is really the same, only we add a step to evaluate the integral at the upper and lower limits of integration.
        \bigbreak \noindent 
        \begin{thrm}[Integration by Parts for Definite Integrals]
           Let  $u = f(x)$  and  $v = g(x)$ be functions with continuous derivatives on $[a,b]$. Then:
           \begin{align*}
               \int_{a}^{b} u \, dv &= uv\big|_{a}^{b} - \int_{a}^{b} v \, du
           .\end{align*}
        \end{thrm}

        \pagebreak 
        \phantomsection
        \addcontentsline{toc}{section}{3.2 Trigonometric Integrals}
        \section*{3.2 Trigonometric Integrals}
        \bigbreak \noindent 
        In this section we look at how to integrate a variety of products of trigonometric functions. These integrals are called trigonometric integrals. They are an important part of the integration technique called trigonometric substitution, which is featured in Trigonometric Substitution. This technique allows us to convert algebraic expressions that we may not be able to integrate into expressions involving trigonometric functions, which we may be able to integrate using the techniques described in this section. In addition, these types of integrals appear frequently when we study polar, cylindrical, and spherical coordinate systems later. Let’s begin our study with products of $\sin{x}$ and $\cos{x}$.

        \bigbreak \noindent 
        \begin{large}
            \textbf{Integrating $\cos^{j}{x}\sin{x}$}
        \end{large}
        \bigbreak \noindent 
        In this case, we can preform a simple u-substation, where we let $u=\cos{x}$, and from there we can evaluate.
        \bigbreak \noindent 
        \begin{eg}[Evaluate]
           \begin{align*}
               &\int \cos^{5}{x}\sin{x}\ dx \\
               &=- \int u^{5}\ du \\
               &-\frac{1}{6}u^{6} + C \\
               &= -\frac{1}{6}\cos^{6}{x} + C
           .\end{align*} 
        \end{eg}
        
        \bigbreak \noindent 
        \begin{large}
            \textbf{Integrating $\cos^{j}{x}\sin^{k}{x}$ when $k$ is odd}
        \end{large}
        \bigbreak \noindent 
        In this case, we can use the trigonometric identity: $\sin^{2}{x} = 1-\cos^{2}{x}$ to rewrite the expression such that using a u-substation will work. In general:
        \begin{align*}
            &\int \cos^{j}{x}\sin^{k}{x}\ dx\quad \text{s.t}\ k = 2l+1\ l \in \mathbb{Z} \\
            &=\int \cos^{j}{x}(1-\cos^{2}{k})^{\frac{k-1}{2}}\sin{(x)}
        .\end{align*}
        \bigbreak \noindent 
        \begin{eg}[Evaluate]
            \begin{align*}
                &\int \cos^{2}{x}\sin^{5}{x}\ dx \\
                &=\int\cos^{2}{x}(1-\cos^{2}{x})^{\frac{5-1}{2}}\sin{x}\ dx \\
                &=\int \cos^{2}{x}(1-\cos^{2}{x})^{2}\sin{x}\ dx \\
                &=etc...
            .\end{align*}
        \end{eg}
        \bigbreak \noindent 
        \nt{This fact also holds for $\int \sin^{j}{x}\cos^{k}{x}\, for\ k = 2l+1,\ l \in \mathbb{Z}$}

        \pagebreak 
        \phantomsection
        \addcontentsline{toc}{subsection}{Integrating even powers of sin(x)}
        \subsection*{Integrating even powers of $\sin{x}$}
        \bigbreak \noindent 
        In the next example, we see the strategy that must be applied when there are only even powers of \( \sin(x) \) and \( \cos(x) \). For integrals of this type, the identities
        \[
        \sin^2(x) = \frac{1}{2} - \frac{1}{2}\cos(2x) = \frac{1 - \cos(2x)}{2}
        \]
        and
        \[
        \cos^2(x) = \frac{1}{2} + \frac{1}{2}\cos(2x) = \frac{1 + \cos(2x)}{2}
        \]
        are invaluable. These identities are sometimes known as power-reducing identities and they may be derived from the double-angle identity  $\cos(2x) = \cos^2(x) - \sin^2(x)$ and the Pythagorean identity  $\cos^2(x) + \sin^2(x) = 1$

        \bigbreak \noindent 
        \begin{eg}[Evaluate]
           \begin{align*}
               &\int \sin^{2}{x}\ dx
           .\end{align*} 
           \bigbreak \noindent 
           By the identity described above, we can derive:
           \begin{align*}
               \cos{2x} &= 1-2\sin^{2}{x} \\
               \sin^{2}{x} &= \frac{1}{2}-\frac{1}{2}\cos{2x} 
           .\end{align*}
           \bigbreak \noindent 
           Thus we have:
           \begin{align*}
               &\int \frac{1}{2}-\frac{1}{2}\cos{2x}\ dx \\
               &=\frac{1}{2}x - \frac{1}{4}\sin{2x} + C
           .\end{align*}
        \end{eg}

        \pagebreak 
        \phantomsection
        \addcontentsline{toc}{subsection}{Problem-Solving Strategy: Integrating Products and Powers of sin x and cos x}
        \subsection*{Problem-Solving Strategy: Integrating Products and Powers of sin x and cos x}
        \bigbreak \noindent 
        To integrate 
\[
\int \cos^j(x) \sin^k(x) \, dx
\]
use the following strategies:

\begin{enumerate}
    \item If \( k \) is odd, rewrite \( \sin^k(x) \) as \( \sin^{k-1}(x) \sin(x) \) and use the identity \( \sin^2(x) = 1 - \cos^2(x) \) to rewrite \( \sin^{k-1}(x) \) in terms of \( \cos(x) \). Integrate using the substitution \( u = \cos(x) \). This substitution makes \( du = -\sin(x) \, dx \).
    
    \item If \( j \) is odd, rewrite \( \cos^j(x) \) as \( \cos^{j-1}(x) \cos(x) \) and use the identity \( \cos^2(x) = 1 - \sin^2(x) \) to rewrite \( \cos^{j-1}(x) \) in terms of \( \sin(x) \). Integrate using the substitution \( u = \sin(x) \). This substitution makes \( du = \cos(x) \, dx \). (Note: If both \( j \) and \( k \) are odd, either strategy 1 or strategy 2 may be used.)
    
    \item If both \( j \) and \( k \) are even, use 
    \[
    \sin^2(x) = \frac{1}{2} - \frac{1}{2} \cos(2x)
    \]
    and 
    \[
    \cos^2(x) = \frac{1}{2} + \frac{1}{2} \cos(2x).
    \]
    After applying these formulas, simplify and reapply strategies 1 through 3 as appropriate.
\end{enumerate}

    \bigbreak \noindent 
    \phantomsection
    \addcontentsline{toc}{subsection}{Power reduction formulas}
    \subsection*{Power reduction formulas}
    \bigbreak \noindent 
    \begin{thrm}[Power reduction formulas]
        \begin{itemize}
            \item \textbf{Power Reduction Formula (sine)}
                \begin{align*}
                    \int \sin^{n}{x}\ dx = -\frac{1}{n}\sin^{n-1}{x}\cos{x} + \frac{n-1}{n}\int \sin^{n-2}{x}\ dx
                .\end{align*}
            \item \textbf{Power Reduction Formula (cosine)}
                \begin{align*}
                    \int \cos^{n}{x}\ dx = \frac{1}{n}\cos^{n-1}{x}\sin{x} + \frac{n-1}{n}\int \cos^{n-2}{x}\ dx
                .\end{align*}
            \item \textbf{Power Reduction Formula (secant)}
            \begin{align*}
                \int \sec^{n}{x}\ dx &= \frac{1}{n-1}\sec^{n-1}{x}\sin{x}+\frac{n-2}{n-1}\int \sec^{n-2}{x}\ dx \\
                \int \sec^{n}{x}\ dx &= \frac{1}{n-1}\sec^{n-2}{x}\tan{x}+\frac{n-2}{n-1}\int \sec^{n-2}{x}\ dx
            \end{align*}
            \item \textbf{Power Reduction Formula (Tangent)}
                \begin{align*}
                    \int \tan^{n}{x}\ dx &= \frac{1}{n-1}\tan^{n-1}{x} - \int \tan^{n-2}{x}\ dx
                .\end{align*}
        \end{itemize}
    \end{thrm}
    

    \pagebreak 
    \phantomsection
    \addcontentsline{toc}{subsection}{Integrating products of sines and cosines of different angles}
    \subsection*{Integrating products of sines and cosines of different angles}
    \bigbreak \noindent 
    \begin{thrm}
       To integrate products involving  sin(ax), sin(bx), cos(ax), and  cos(bx), use the substitutions:
       \begin{itemize}
           \item \textbf{Sine Products}
            \begin{align*}
                \sin(ax) \sin(bx) &= \frac{1}{2} \cos((a-b)x) - \frac{1}{2} \cos((a+b)x)
            \end{align*}

            \item \textbf{Sine and Cosine Products}
            \begin{align*}
                \sin(ax) \cos(bx) &= \frac{1}{2} \sin((a-b)x) + \frac{1}{2} \sin((a+b)x)
            \end{align*}

            \item \textbf{Cosine Products}
            \begin{align*}
                \cos(ax) \cos(bx) &= \frac{1}{2} \cos((a-b)x) + \frac{1}{2} \cos((a+b)x)
            \end{align*}
       \end{itemize}
       \bigbreak \noindent 
       Which are trivial if you know the trigonometric \textbf{product to sum} identities
    \end{thrm}

    \pagebreak 
    \phantomsection
    \addcontentsline{toc}{section}{3.3 Trigonometric Substitution}
    \section*{3.3 Trigonometric Substitution}
    \bigbreak \noindent 
    In this section, we explore integrals containing expressions of the form  $\sqrt{a^2 - x^2}$, $\sqrt{a^2 + x^2}$ and $\sqrt{x^2 - a^2}$
    where the values of \( a \) are positive. We have already encountered and evaluated integrals containing some expressions of this type, but many still remain inaccessible. The technique of trigonometric substitution comes in very handy when evaluating these integrals. This technique uses substitution to rewrite these integrals as trigonometric integrals.

    \bigbreak \noindent 
    \begin{large}
        \textbf{Integrals involving $\sqrt{a^{2} -x^{2}}$}
    \end{large}
    \bigbreak \noindent 
    \begin{minipage}[]{0.47\textwidth}
    Consider the integral:
    \begin{align*}
        \int \sqrt{9-x^{2}}\ dx
    .\end{align*}
    \bigbreak \noindent 
    The first thing we can deduce when we see an integral of this form ($\sqrt{a^{2} - x^{2}}$), is that the integrand looks awfully like it could be written as \textit{Pythagoreans theorem} $a^{2} + b^{2} = c^{2}$. So, let's draw a triangle and see what we can figure out. But first, let's gather some information...
    \begin{align*}
        &If:\ a^{2} + b^{2} = c^{2} \\
        &a = \sqrt{b^{2} - c^{2}}\quad \text{Possibility I} \\
        &b = \sqrt{c^{2} - a^{2}}\quad \text{Possibility II}
    .\end{align*}
    Thus, we know our full equation $\sqrt{3^{2}-x^{2}}$, must be either side $a$ or $b$, and the terms inside the square root must the hypotenuse, and the remaining side. When it comes to choosing which is which, we will base our reasoning on what makes things easiest...
    \end{minipage}
    \hspace{0.5in}
    \begin{minipage}[]{0.47\textwidth}
        \incfig{mytri}
        \bigbreak \noindent 
         So, we let $a=\sqrt{3^{2}-x^{2}}$, $b=x$, and $c=3$. The reason we choose our sides this way is because now, we can define the angle $\theta$ as $\sin{\theta} = \frac{opp}{hyp} = \frac{x}{3}$. Thus, $x = 3\sin{\theta}$
         \bigbreak \noindent 
         Side note: This \textbf{reference triangle} will come in handy later.
    \end{minipage}
    \bigbreak \noindent 
    Now, since we have deduced $x=3\sin{\theta}$, we can rewrite our integral as:
    \begin{align*}
        \int \sqrt{9-(3\sin{\theta })^{2}}\ dx
    .\end{align*}
    \bigbreak \noindent 
    However, we still need to account for our $dx$. Since we know $x=3\sin{\theta}$, $dx$ must be $3\cos{\theta}$. So our integral becomes:
    \begin{align*}
        &\int \sqrt{9-9\sin^{2}{\theta }}\ 3\cos{\theta}d\theta  \\
        &=\int \sqrt{9(1-\sin^{2}{\theta })}\ 3\cos{\theta}d\theta  \\
        &=\int 3\sqrt{1-\sin^{2}{\theta }}\ 3\cos{\theta}d\theta  \\
        &= 9\int \sqrt{\cos^{2}{\theta }}\ \cos{\theta }d\theta  \\
        &=9 \int \cos^{2}{\theta }\ d\theta 
    .\end{align*}
    \bigbreak \noindent 
   To understand why we can write $ \sqrt{\cos^2{\theta}}$  as  $\cos{\theta}$, consider the integral involving $\sqrt{a^2 - x^2}$. For the integral to be real-valued, $x \text{ must lie in the interval } [-a, a]$. When we use the substitution $x = 3\sin{\theta}$,  this interval corresponds to a $ \theta \text{ domain of } [-\pi/2, \pi/2]$. However, if the problem context ensures that  $x$ is positive (for instance, due to the domain of integration or other constraints), then our  $\theta \text{ domain narrows to } [0, \pi/2]$. In this interval, the cosine function is always non-negative, so $ \sqrt{\cos^2{\theta}} \text{ simplifies to } \cos{\theta}$.

   \pagebreak \bigbreak \noindent 
   Now we can evaluate the integral:
   \begin{align*}
       &9\int \cos^{2}{\theta }d\theta  \\
       &=9 \left[\frac{1}{2}\theta +\frac{1}{4}\sin{2\theta }\right] + C \\
       &=\frac{9}{2}\theta +\frac{9}{4}\sin{2\theta } + C
   .\end{align*}
   \bigbreak \noindent 
   \begin{minipage}[]{0.47\textwidth}
       From here, we must convert back to x's, to do this, we must revisit our reference triangle. We know:
       \begin{align*}
           \sin{\theta } = \frac{1}{3}x
       .\end{align*}
       \bigbreak \noindent 
       Thus:
       \begin{align*}
           &\theta  = \sin^{-1}{\frac{1}{3}x} \\
           &\sin{2\theta } = \sin{\left(2\sin^{-1}{\frac{1}{3}x}\right)}
       .\end{align*}
       \bigbreak \noindent 
       Therefore, our final answer is:
       \begin{align*}
           \frac{9}{2}\sin^{-1}{\left(\frac{1}{3}x\right)}+\frac{9}{4}\sin{\left(2\sin^{-1}{\left(\frac{1}{3}x\right)}\right)} + C
       .\end{align*}
   \end{minipage}
   \hspace{.5in}
   \begin{minipage}[]{0.47\textwidth}
       \incfig{mytri}
   \end{minipage}

   \bigbreak \noindent 
   \begin{large}
       \textbf{Other Forms}
   \end{large}
   \bigbreak \noindent 
   Now that we have walked through the process for integrals of the form $\sqrt{a^{2} - x^{2}}$, let's take a look at the process for the other forms, specifically $\sqrt{x^{2} - a^{2}}$ or $\sqrt{a^{2} + x^{2}}$
   \bigbreak \noindent 
   When we have an integral in the form of $\sqrt{x^{2} - a^{2}}$, we use the substitution $x=a\sec{\theta }$ by restricting the domain to $\bigg(0,\frac{\pi}{2}\bigg)\cup \bigg[\pi, \frac{3\pi}{2}\bigg) $. We use the identity  $\sec^{2}{\theta } - 1 = \tan^{2}{\theta } $ to simplify the integrand
   \bigbreak \noindent 
   When we have  an integral in the form of $\sqrt{a^{2} + x^{2}}$, we use the substitution $x=a\tan{\theta }$ by restricting the domain to $\left(-\frac{\pi}{2}, \frac{\pi}{2}\right) $. We then use the identity $1+\tan^{2}{\theta } = \sec^{2}{\theta }$ to simplify the integrand
   
   \bigbreak \noindent 
   \nt{These trigonometric substition forms do not rely on the square root. This means we can still make the substations even if there is not a square root}

   \pagebreak 
   \phantomsection
   \addcontentsline{toc}{subsection}{Reference Triangles}
   \subsection*{Reference Triangles}
   \bigbreak \noindent 
   It is a good idea to be familiar with all three versions of the reference triangles, this way you don't need to expend effort deducing the sides.
   \bigbreak \noindent 
   \begin{minipage}[]{0.47\textwidth}
       Form 1: $\sqrt{a^{2} - x^{2}}$ where $x=a\sin{x}$
       \bigbreak \noindent 
        \incfig{tri2}
   \end{minipage}
   \begin{minipage}[]{0.47\textwidth}
        Form 2: $\sqrt{x^{2} - a^{2}}$ where $x=a\sec{x}$
        \bigbreak \noindent 
        \incfig{tri3}
   \end{minipage}
   \bigbreak \noindent 
   Form 3: $\sqrt{a^{2} + x^{2}}$ where $x=a\tan{x}$
   \bigbreak \noindent 
\begin{figure}[ht]
    \centering
    \incfig{tri4}
    \label{fig:tri4}
\end{figure}

    \pagebreak \bigbreak \noindent 
    \phantomsection
    \addcontentsline{toc}{section}{3.4 Partial Fractions}
    \section*{3.4 Partial Fractions}
    \bigbreak \noindent 
    In this section, we examine the method of partial fraction decomposition, which allows us to decompose rational functions into sums of simpler, more easily integrated rational functions. Using this method, we can rewrite an expression such as $\frac{3x}{x^2 - x - 2}$
    as an expression such as: $\frac{1}{x+1} + \frac{2}{x-2}$
    \bigbreak \noindent 
    The key to the method of partial fraction decomposition is being able to anticipate the form that the decomposition of a rational function will take. As we shall see, this form is both predictable and highly dependent on the factorization of the denominator of the rational function. It is also extremely important to keep in mind that partial fraction decomposition can be applied to a rational function $ \frac{P(x)}{Q(x)} $ only if $\text{deg}(P(x)) < \text{deg}(Q(x))$ In the case
    When $\text{deg}(P(x)) \geq \text{deg}(Q(x))$, we must first perform long division to rewrite the quotient $\frac{P(x)}{Q(x)}$ in the form $A(x) + \frac{R(x)}{Q(x)}$, where $\text{deg}(R(x)) < \text{deg}(Q(x))$. We then do a partial fraction decomposition on $\frac{R(x)}{Q(x)}$. 

    \bigbreak \noindent 
    \phantomsection
    \addcontentsline{toc}{subsection}{Nonrepeated Linear Factors}
    \subsection*{Nonrepeated Linear Factors}
    \bigbreak \noindent 
    If $Q(x)$ can be factored as $(a_1x + b_1)(a_2x + b_2) \dots (a_nx + b_n)$, where each linear factor is distinct, then it is possible to find constants $A_1, A_2, \dots, A_n$ satisfying
    \bigbreak \noindent 
    \[ \frac{P(x)}{Q(x)} = \frac{A_1}{a_1x + b_1} + \frac{A_2}{a_2x + b_2} + \dots + \frac{A_n}{a_nx + b_n}. \]
    \bigbreak \noindent 
    The proof that such constants exist is beyond the scope of this course.
    \bigbreak \noindent 
    In this next example, we see how to use partial fractions to integrate a rational function of this type.
    \bigbreak \noindent 
    \begin{eg}[Partial Fractions with Nonrepeated Linear Factors]
        Evaluate
        \begin{align*}
            \int \frac{3x+2}{x^{3}-x^{2}-2x}\ dx
        .\end{align*}
        \bigbreak \noindent 
    \end{eg}
    \bigbreak \noindent 
    \textbf{Solution.} Since $\text{deg}(3x+2) < \text{deg}(x^3 - x^2 - 2x)$, we begin by factoring the denominator of $\frac{3x+2}{x^3 - x^2 - 2x}$. We can see that $x^3 - x^2 - 2x = x(x-2)(x+1)$. Thus, there are constants $A$, $B$, and $C$ satisfying
    \[
    \frac{3x+2}{x(x-2)(x+1)} = \frac{A}{x} + \frac{B}{x-2} + \frac{C}{x+1}.
    \]
    \bigbreak \noindent 
    We must now find these constants. To do so, we begin by getting a common denominator on the right. Thus,
    \[ \frac{3x+2}{x(x-2)(x+1)} = \frac{A(x-2)(x+1)+ Bx(x+1)+Cx(x-2)}{x(x-2)(x+1)} \]
        \bigbreak \noindent 
        Now, we set the numerators equal to each other, obtaining
        \begin{align*}
            3x+2=A(x-2)(x+1)+Bx(x+1)+Cx(x-2)
        .\end{align*}
    \bigbreak \noindent 
    There are two different strategies for finding the coefficients  $A$, $B$, and  $C$. We refer to these as \textbf{the method of equating coefficients} and \textbf{the method of strategic substitution}.

    \pagebreak \bigbreak \noindent 
    \begin{mdframed}
    \textbf{Rule: Method of equating coefficients}
    \bigbreak \noindent 
    First, lets rewrite the equation:
    \begin{align*}
     &\frac{3x+2}{x(x-2)(x+1)} = \frac{A}{x} + \frac{B}{x-2} + \frac{C}{x+1} \\
     &=(A+B+C)x^{2}+(-A+B-2C)x+(-2A)
    .\end{align*}
    Equating coefficients produces the system of equations
    \begin{align*}
        &A+B+C=0 \\
        &-A+B-2C = 3 \\
        &-2A = 2
    .\end{align*}
    \bigbreak \noindent 
    To solve this system, we first observe that  $-2A=2\implies A=-1$. Substituting this value into the first two equations gives us the system
    \begin{align*}
        &&B+C=1 \\ 
        &B-2C = 2
    .\end{align*}
    \bigbreak \noindent 
    Multiplying the second equation by  $-1$ and adding the resulting equation to the first produces $-3C=1$,
    \bigbreak \noindent 
    which in turn implies that $C = -\frac{1}{3}$. Substituting this value into the equation $B+C=1$ yields $B=\frac{4}{3}$. Thus, solving these equations yields $A=-1$, $B=\frac{4}{3}$, and $C=-\frac{1}{3}$.
    \bigbreak \noindent 
    It is important to note that the system produced by this method is consistent if and only if we have set up the decomposition correctly. If the system is inconsistent, there is an error in our decomposition.
     \end{mdframed}
     \bigbreak \noindent 
     \begin{mdframed}
        \textbf{Rule: Method of Strategic Substitution} 
        \bigbreak \noindent 
        The method of strategic substitution is based on the assumption that we have set up the decomposition correctly. If the decomposition is set up correctly, then there must be values of $A$, $B$, and $C$ that satisfy Equation 3.8 for all values of $x$. That is, this equation must be true for any value of $x$ we care to substitute into it. Therefore, by choosing values of $x$ carefully and substituting them into the equation, we may find $A$, $B$, and $C$ easily. For example, if we substitute $x=0$, the equation reduces to $2=A(-2)(1)$. Solving for $A$ yields $A=-1$. Next, by substituting $x=2$, the equation reduces to $8=B(2)(3)$, or equivalently $B=\frac{4}{3}$. Last, we substitute $x=-1$ into the equation and obtain $-1=C(-1)(-3)$. Solving, we have $C=-\frac{1}{3}$.
        \bigbreak \noindent 
        It is important to keep in mind that if we attempt to use this method with a decomposition that has not been set up correctly, we are still able to find values for the constants, but these constants are meaningless. If we do opt to use the method of strategic substitution, then it is a good idea to check the result by recombining the terms algebraically.
     \end{mdframed}
     \pagebreak \bigbreak \noindent 
     Now that we have the values of  $A$, $B$, and  $C$, we rewrite the original integral:
     \begin{align*}
         &\int \frac{3x+2}{ x^{3}-x^{2}-2x}\ dx  \\
         &=\int \left(-\frac{1}{x}+\frac{4}{3}\cdot \frac{1}{(x-2)}-\frac{1}{3}\cdot \frac{1}{(x+1)}\right)\ dx
     .\end{align*}
     \bigbreak \noindent 
     Evaluating the integral gives us:
     \begin{align*}
        -\ln{\abs{x}} +\frac{4}{3}\ln{\abs{x-2}} -\frac{1}{3}\ln{\abs{x+1}} +C  
     .\end{align*}
     \bigbreak \noindent 



  
 



    









    


        

        


        
        

    
    
    






\end{document}
