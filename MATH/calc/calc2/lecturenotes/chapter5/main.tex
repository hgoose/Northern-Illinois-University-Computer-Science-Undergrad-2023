\documentclass{report}

\input{~/dev/latex/template/preamble.tex}
\input{~/dev/latex/template/macros.tex}

\title{\Huge{}}
\author{\huge{Nathan Warner}}
\date{\huge{}}
\pagestyle{fancy}
\fancyhf{}
\lhead{Warner \thepage}
\rhead{}
% \lhead{\leftmark}
\cfoot{\thepage}
%\setborder
% \usepackage[default]{sourcecodepro}
% \usepackage[T1]{fontenc}

\begin{document}
    % \maketitle
        \begin{titlepage}
       \begin{center}
           \vspace*{1cm}
    
           \textbf{Calculus 2} \\
           Chapter 5
    
           \vspace{0.5cm}
            
                
           \vspace{1.5cm}
    
           \textbf{Nathan Warner}
    
           \vfill
                
                
           \vspace{0.8cm}
         
           \includegraphics[width=0.4\textwidth]{~/niu/seal.png}
                
           Computer Science \\
           Northern Illinois University\\
           October 27, 2023 \\
           United States\\
           
                
       \end{center}
    \end{titlepage}
    \tableofcontents
    \pagebreak \bigbreak \noindent
    \vspace{2in} \\
    \begin{Huge}
       \textbf{Sequences and Series} 
    \end{Huge}
    \bigbreak \noindent 
    \line(1,0){490}
    \bigbreak \noindent 

    \phantomsection
    \addcontentsline{toc}{section}{5.1 Sequences}
    \section*{5.1 Sequences}
    \bigbreak \noindent 
    \phantomsection
    \addcontentsline{toc}{subsection}{Terminology of Sequences}
    \subsection*{Terminology of Sequences}
    \bigbreak \noindent 
    To work with this new topic, we need some new terms and definitions. First, an infinite sequence is an ordered list of numbers of the form
    \begin{align*}
        a_{1},\ a_{2},\ a_{3},\ ...\ a_{n},\ ....
    .\end{align*}
    \bigbreak \noindent 
    Each of the numbers in the sequence is called a term. The symbol $n$ is called the index variable for the sequence. We use the notation
    \begin{align*}
        \{a_{n}\}_{n=1}^{\infty},\ \text{or simply}\ \{a_{n}\}
    .\end{align*}
    \bigbreak \noindent 
    to denote this sequence. A similar notation is used for sets, but a sequence is an ordered list, whereas a set is not ordered. Because a particular number $a_{n}$ exists for each positive integer $n$, we can also define a sequence as a function whose domain is the set of positive integers.
     \bigbreak \noindent 
    Let’s consider the infinite, ordered list
    \begin{align*}
        2,\ 4,\ 8,\ 16,\ 32,\ ...
    .\end{align*}
    \bigbreak \noindent 
    This is a sequence in which the first, second, and third terms are given by $a_1 = 2$, $a_2 = 4$, and $a_3 = 8$. You can probably see that the terms in this sequence have the following pattern:
    \begin{align*}
        a_{1} = 2^{1},\ a_{2} = 2^{2},\ a_{3} = 2^{3},\ a_{4}= 2^{4},\ \text{and}\ a_{5} = 2^{5}
    .\end{align*}
    \bigbreak \noindent 
    Assuming this pattern continues, we can write the $n^{th}$ term in the sequence by the explicit formula $a_n = 2^{n}$. Using this notation, we can write this sequence as
    \[
    \{2n\}_{n=1}^{\infty} \quad \text{or} \quad \{2n\}.
    \]
    \bigbreak \noindent 
    Alternatively, we can describe this sequence in a different way. Since each term is twice the previous term, this sequence can be defined recursively by expressing the $n^{th}$ term $a_n$ in terms of the previous term $a_{n-1}$. In particular, we can define this sequence as the sequence $\{a_n\}$ where $a_1=2$ and for all $n \geq 2$, each term $a_n$ is defined by the \textbf{recurrence relation} $a_n = 2a_{n-1}$.

    \pagebreak \bigbreak \noindent 
    \begin{definition}[Infinite Sequence]
        An infinite sequence $\{a_n\}$ is an ordered list of numbers of the form
        \[ a_1, a_2, \ldots, a_n, \ldots \]
        The subscript $n$ is called the index variable of the sequence. Each number $a_n$ is a term of the sequence. Sometimes sequences are defined by explicit formulas, in which case $a_n = f(n)$ for some function $f(n)$ defined over the positive integers. In other cases, sequences are defined by using a recurrence relation. In a recurrence relation, one term (or more) of the sequence is given explicitly, and subsequent terms are defined in terms of earlier terms in the sequence.


    \end{definition}

    \bigbreak \noindent 
    \nt{Note that the index does not have to start at $n=1$ but could start with other integers. For example, a sequence given by the explicit formula $a_{n}=f(n)$ could start at $n=0$,in which case the sequence would be
        \begin{align*}
            a_{0},\ a_{1},\ a_{2},\ ...
        .\end{align*}
    }
    \bigbreak \noindent 
    \begin{minipage}[t]{0.52\textwidth}
        Similarly, for a sequence defined by a recurrence relation, the term $a_0$ may be given explicitly, and the terms $a_n$ for $n \geq 1$ may be defined in terms of $a_{n-1}$. Since a sequence $\{a_n\}$ has exactly one value for each positive integer $n$, it can be described as a function whose domain is the set of positive integers. As a result, it makes sense to discuss the graph of a sequence. The graph of a sequence $\{a_n\}$ consists of all points $(n, a_n)$ for all positive integers $n$. Figure 5.2 shows the graph of $\{2n\}$.
    \end{minipage}
    \begin{minipage}[t]{0.47\textwidth}
         \begin{center}
            \includegraphics[scale=0.5]{./figures/mane6.png}
        \end{center}
        \begin{center}
            \textit{Figure 5.2}
        \end{center}
    \end{minipage}
    \bigbreak \noindent 
    Two types of sequences occur often and are given special names: arithmetic sequences and geometric sequences. In an arithmetic sequence, the difference between every pair of consecutive terms is the same. For example, consider the sequence
    \begin{align*}
        3,\ 7,\ 11,\ 15,\ 19,\ \cdots
    .\end{align*}
    You can see that the difference between every consecutive pair of terms is  4. Assuming that this pattern continues, this sequence is an arithmetic sequence. It can be described by using the recurrence relation
       \begin{equation}
            \begin{cases}
                a_{1} &=  3\\
                a_{n} &= a_{n-1} + 4\ \text{for } n \geq 2
            \end{cases}
        \end{equation}
    \bigbreak \noindent 
    Note that:
    \begin{align*}
        &a_{2} = 3+4 \\
        &a_{3}=  3+4+4 = 3+2\cdot 4 \\
        &a_{4} =3+4+4+4 = 3+3 \cdot 4
    .\end{align*}
    Thus the sequence can also be described using the explicit formula
    \begin{align*}
        &a_{n} = a + (n-1)d \\
        \text{Thus}\ &a_{n} = 3 + (n-1)4
    .\end{align*}
    Where $d$ is the difference $d = a_{n} - a_{n-1}$.     In general, an arithmetic sequence is any sequence of the form  $a_{n} = a +dn$
    \bigbreak \noindent 
    In a \textbf{geometric sequence}, the ratio of every pair of consecutive terms is the same. For example, consider the sequence
    \begin{align*}
        2,\ -\frac{2}{3},\ \frac{2}{9},\ -\frac{2}{27},\ \frac{2}{81},\ \cdots 
    .\end{align*}
    \bigbreak \noindent 
    We see that the ratio of any term to the preceding term is  $-\frac{1}{3}$. Assuming this pattern continues, this sequence is a geometric sequence. It can be defined recursively as
    \begin{align*}
        &a_{1} = 2 \\
        &=a_{n} = -\frac{1}{3} \cdot a_{n-1}\ \text{for } n \geq 2
    .\end{align*}
    Alternatively, since
    \begin{align*}
        &a_{2} = -\frac{1}{3} \cdot 2 \\
        &a_{3} = \left(-\frac{1}{3}\right) \left(-\frac{1}{3}\right)(2) = \left(-\frac{1}{3}\right)^{2} \cdot 2 \\
        &a_{4} = \left(-\frac{1}{3}\right)\left(-\frac{1}{3}\right)\left(-\frac{1}{3}\right) (2) = \left(-\frac{1}{3}\right)^{3} \cdot 2
    .\end{align*}
    we see that the sequence can be described by using the explicit formula
    \begin{align*}
        a_{n} = 2\left(-\frac{1}{3}\right)^{n-1}
    .\end{align*}
    The sequence \{2n\} that we discussed earlier is a geometric sequence, where the ratio of any term to the previous term is 2. In general, a geometric sequence is any sequence of the form $a_{n} = ar^{n} $.
    \pagebreak 
    \phantomsection
    \addcontentsline{toc}{subsection}{Finding explicit formulas}
    \subsection*{Finding explicit formulas}
    \bigbreak \noindent 
    \begin{eg}
       Find an explicit formula for the sequence 
       \begin{align*}
           -\frac{1}{2},\ \frac{2}{3},\ -\frac{3}{4},\ \frac{4}{5},\ -\frac{5}{6}
       .\end{align*}
    \end{eg}
    \bigbreak \noindent 
    \textbf{Solution.}
    First, note that the sequence is alternating from negative to positive. The odd terms in the sequence are negative, and the even terms are positive. Therefore, the \( n^{th} \) term includes a factor of \( (-1)^n \). Next, consider the sequence of numerators \( \{1,2,3,\ldots\} \) and the sequence of denominators \( \{2,3,4,\ldots\} \). We can see that both of these sequences are arithmetic sequences. The \( n^{th} \) term in the sequence of numerators is \( n \), and the \( n^{th} \) term in the sequence of denominators is \( n+1 \). Therefore, the sequence can be described by the explicit formula.
    \begin{align*}
        a_{n} = \frac{(-1)^{n}n}{n+1}
    .\end{align*}

    \bigbreak \noindent 
    \begin{eg}
       Find an explicit formula for the sequence
       \begin{align*}
           \frac{3}{4},\ \frac{9}{7},\ \frac{27}{10},\ \frac{81}{13},\ \frac{243}{16}
       .\end{align*}
    \end{eg}
    \bigbreak \noindent 
    \textbf{Solution.}
    The sequence of numerators \(3,9,27,81,243,\ldots\) is a geometric sequence. The numerator of the \(n^{th}\) term is \(3^n\). The sequence of denominators \(4,7,10,13,16,\ldots\) is an arithmetic sequence. The denominator of the \(n^{th}\) term is \(4+3(n-1) = 3n+1\). Therefore, we can describe the sequence by the explicit formula \(a_n = \frac{3^n}{3n+1}\).

    \bigbreak \noindent 
    \begin{eg}[Defined by Recurrence Relations]
       For each of the following recursively defined sequences, find an explicit formula for the sequence.
       \begin{enumerate}[label=(\alph*)]
           \item $a_{1} = 2,\ a_{n} = -3a_{n-1}$ for $n \geq 2 $
            \item $a_{1} = \frac{1}{2},\ a_{n} = a_{n-1} + \left(\frac{1}{2}\right)^{n} $ for $n \geq2 $
       \end{enumerate}
    \end{eg}
    \bigbreak \noindent 
    \textbf{A: Solution.} 
    \begin{align*}
        &a = 2,\ r =-3 \\
        \text{Thus:}\ &a_{n} = 2(-3)^{n-1}
    .\end{align*}
    \bigbreak \noindent 
    \textbf{B: Solution.}
    \bigbreak \noindent 
    If we write out the first few terms we get:
    \begin{align*}
        &a_{1} = \frac{1}{2},\
        a_{2} = \frac{3}{4},\
        a_{3} = \frac{7}{8},\
        a_{4} = \frac{15}{16}
    .\end{align*}
    \bigbreak \noindent 
    We see the denominator is the geometric sequence:
    \begin{align*}
        &2,\ 4,\ 8,\ 16 \\
        \text{Thus: } &a_{n} = 2^{n}
    .\end{align*}
    \bigbreak \noindent 
    And for the numerator, examining the difference between the terms, we get $2,4,8$, thus we can see that the pattern is $a_{n}  = 2^{n}-1$
    \bigbreak \noindent 
    Thus, we have the explicit formula 
    \begin{align*}
        a_{n} = \frac{2^{n}-1}{2^{n}}
    .\end{align*}

    \pagebreak 
    \phantomsection
    \addcontentsline{toc}{subsection}{Limit of a Sequence}
    \subsection*{Limit of a Sequence}
    \bigbreak \noindent 
    A fundamental question that arises regarding infinite sequences is the behavior of the terms as \( n \) gets larger. Since a sequence is a function defined on the positive integers, it makes sense to discuss the limit of the terms as \( n \to \infty \). For example, consider the following four sequences and their different behaviors as \( n \to \infty \) (see Figure 5.3):
    \begin{enumerate}
        \item $\{1+3n\} = \{4,7,10,13,\ldots\}$. The terms \(1+3n\) become arbitrarily large as \(n \to \infty\). In this case, we say that \(1+3n \to \infty\) as \(n \to \infty\).
        
        \item $\{1-\left(\frac{1}{2}\right)^{n}\} = \{\frac{1}{2},\frac{3}{4},\frac{7}{8},\frac{15}{16},\ldots\}$. The terms \(\left(1-\frac{1}{2}\right)^{n} \to 1\) as \(n \to \infty\).
        
        \item $\{(-1)^n\} = \{-1,1,-1,1,\ldots\}$.  The terms alternate but do not approach one single value as \(n \to \infty\).
        
        \item $\{\frac{(-1)^{n}}{n}\} = \{-1,\frac{1}{2},-\frac{1}{3},\frac{1}{4},\ldots\}$ 
        The terms alternate for this sequence as well, but \(\frac{(-1)^{n}}{n}  \to 0\) as \(n \to \infty\).
    \end{enumerate}
    \bigbreak \noindent 
    \begin{center}
        \includegraphics[scale=0.4]{./figures/mane7.png}
    \end{center}
    \bigbreak \noindent 
    From these examples, we see several possibilities for the behavior of the terms of a sequence as \( n \to \infty \). 
    In two of the sequences, the terms approach a finite number as \( n \to \infty \). 
    In the other two sequences, the terms do not. If the terms of a sequence approach a finite number \( L \) as \( n \to \infty \), 
    we say that the sequence is a convergent sequence and the real number \( L \) is the limit of the sequence. We can give an informal definition here.
    \bigbreak \noindent 
    \begin{definition}
        Given a sequence \( \{a_n\} \), \\
        if the terms \( a_n \) become arbitrarily close to a finite number \( L \) as \( n \) becomes sufficiently large, we say \( \{a_n\} \) is a convergent sequence and \( L \) is the limit of the sequence. In this case, we write \\
        \[
        \lim_{{n \to \infty}} a_n = L.
        \]
        If a sequence \( \{a_n\} \) is not convergent, we say it is a divergent sequence.
    \end{definition}

    \bigbreak \noindent 
    From Figure 5.3, we see that the terms in the sequence \( \{1-\left(\frac{1}{2}\right)^{n}\} \) are becoming arbitrarily close to 1 as \( n \) becomes very large. We conclude that \( \{1-\left(\frac{1}{2}\right)^{n}\} \) is a convergent sequence and its limit is 1. In contrast, from Figure 5.3, we see that the terms in the sequence \( 1+3n \) are not approaching a finite number as \( n \) becomes larger. We say that \( \{1+3n\} \) is a divergent sequence.
    \bigbreak \noindent 
    In the informal definition for the limit of a sequence, we used the terms “arbitrarily close” and “sufficiently large.” Although these phrases help illustrate the meaning of a converging sequence, they are somewhat vague. To be more precise, we now present the more formal definition of limit for a sequence and show these ideas graphically in Figure 5.4.
    \bigbreak \noindent 
    \begin{definition}
        A sequence \( \{a_n\} \) converges to a real number \( L \) if for all \( \varepsilon > 0 \), there exists an integer \( N \) such that \( |a_n - L| < \varepsilon \) if \( n \geq N \). The number \( L \) is the limit of the sequence and we write
        \[
        \lim_{{n \to \infty}} a_n = L \quad \text{or} \quad a_n \to L.
        \]
        In this case, we say the sequence \( \{a_n\} \) is a convergent sequence. If a sequence does not converge, it is a divergent sequence, and we say the limit does not exist.
    \end{definition}
    \bigbreak \noindent 
    We remark that the convergence or divergence of a sequence \( \{a_n\} \) depends only on what happens to the terms \( a_n \) as \( n \to \infty \). Therefore, if a finite number of terms \( b_1, b_2, \ldots, b_N \) are placed before \( a_1 \) to create a new sequence
    \[ b_1, b_2, \ldots, b_N, a_1, a_2, \ldots \]
    
    \bigbreak \noindent 
    This new sequence will converge if \( \{a_n\} \) converges and diverge if \( \{a_n\} \) diverges. Further, if the sequence \( \{a_n\} \) converges to \( L \), this new sequence will also converge to \( L \).
    \bigbreak \noindent 
    \begin{center}
        \includegraphics[scale=0.5]{./figures/mane8.png}
    \end{center}
    \bigbreak \noindent 
    As \( n \) increases, the terms \( a_n \) become closer to \( L \). For values of \( n \geq N \), the distance between each point \( (n, a_n) \) and the line \( y = L \) is less than \( \varepsilon \).
    \pagebreak \bigbreak \noindent 
    As defined above, if a sequence does not converge, it is said to be a divergent sequence. For example, the sequences \( \{1+3n\} \) and \( \{(-1)^n\} \) shown in Figure 5.4 diverge. However, different sequences can diverge in different ways. The sequence \( \{(-1)^n\} \) diverges because the terms alternate between 1 and -1, but do not approach one value as \( n \to \infty \). On the other hand, the sequence \( \{1+3n\} \) diverges because the terms \( 1+3n \to \infty \) as \( n \to \infty \). We say the sequence \( \{1+3n\} \) diverges to infinity and write
    \bigbreak \noindent 
    \[
    \lim_{{n \to \infty}} (1+3n) = \infty.
    \]
    \bigbreak \noindent 
    It is important to recognize that this notation does not imply the limit of the sequence \( \{1+3n\} \) exists. The sequence is, in fact, divergent. Writing that the limit is infinity is intended only to provide more information about why the sequence is divergent. A sequence can also diverge to negative infinity. For example, the sequence \( \{-5n+2\} \) diverges to negative infinity because \( -5n+2 \to -\infty \) as \( n \to -\infty \). We write this as
    \bigbreak \noindent 
    \[
    \lim_{{n \to \infty}} (-5n+2) = -\infty.
    \]
    \bigbreak \noindent 
    Because a sequence is a function whose domain is the set of positive integers, we can use properties of limits of functions to determine whether a sequence converges. For example, consider a sequence \( \{a_n\} \) and a related function \( f \) defined on all positive real numbers such that \( f(n) = a_n \) for all integers \( n \geq 1 \). Since the domain of the sequence is a subset of the domain of \( f \), if \( \lim_{{x \to \infty}} f(x) \) exists, then the sequence converges and has the same limit. For example, consider the sequence \( \{\frac{1}{n}\} \) and the related function \( f(x) = \frac{1}{x} \). Since the function \( f \) defined on all real numbers \( x > 0 \) satisfies \( f(x) = \frac{1}{x} \to 0 \) as \( x \to \infty \), the sequence \( \{\frac{1}{n}\} \) must satisfy \( \frac{1}{n} \to 0 \) as \( n \to \infty \).

    \bigbreak \noindent 
    \begin{thrm}[Limit of a Sequence Defined by a Function]
        Consider a sequence \( \{a_n\} \) such that \( a_n = f(n) \) for all \( n \geq 1 \). If there exists a real number \( L \) such that
        \[
        \lim_{{x \to \infty}} f(x) = L,
        \]
        then \( \{a_n\} \) converges and
        \[
        \lim_{{n \to \infty}} a_n = L.
        \]

        
    \end{thrm}

    \bigbreak \noindent 
      We now consider slightly more complicated sequences. For example, consider the sequence \( \left\{\left(\frac{2}{3}\right)^n + \left(\frac{1}{4}\right)^n\right\} \). The terms in this sequence are more complicated than other sequences we have discussed, but luckily the limit of this sequence is determined by the limits of the two sequences \( \left\{\left(\frac{2}{3}\right)^n\right\} \) and \( \left\{\left(\frac{1}{4}\right)^n\right\} \). As we describe in the following algebraic limit laws, since \( \left\{\left(\frac{2}{3}\right)^n\right\} \) and \( \left\{\left(\frac{1}{4}\right)^n\right\} \) both converge to 0, the sequence \( \left\{\left(\frac{2}{3}\right)^n + \left(\frac{1}{4}\right)^n\right\} \) converges to \( 0 + 0 = 0 \). Just as we were able to evaluate a limit involving an algebraic combination of functions \( f \) and \( g \) by looking at the limits of \( f \) and \( g \) (see Introduction to Limits), we are able to evaluate the limit of a sequence whose terms are algebraic combinations of \( a_n \) and \( b_n \) by evaluating the limits of \( \{a_n\} \) and \( \{b_n\} \).

      \bigbreak \noindent 
      \begin{thrm}[Algebraic Limit Laws]
          Given sequences \( \{a_n\} \) and \( \{b_n\} \) and any real number \( c \), if there exist constants \( A \) and \( B \) such that \( \lim_{{n \to \infty}} a_n = A \) and \( \lim_{{n \to \infty}} b_n = B \), then
          \begin{itemize}
            \item \( \lim_{{n \to \infty}} c = c \)
            \item \( \lim_{{n \to \infty}} c a_n = c \lim_{{n \to \infty}} a_n = cA \)
            \item \( \lim_{{n \to \infty}} (a_n \pm b_n) = \lim_{{n \to \infty}} a_n \pm \lim_{{n \to \infty}} b_n = A \pm B \)
            \item \( \lim_{{n \to \infty}} (a_n \cdot b_n) = (\lim_{{n \to \infty}} a_n) \cdot (\lim_{{n \to \infty}} b_n) = A \cdot B \)
            \item \( \lim_{{n \to \infty}} \frac{a_n}{b_n} = \frac{\lim_{{n \to \infty}} a_n}{\lim_{{n \to \infty}} b_n} = \frac{A}{B} \), provided \( B \neq 0 \) and each \( b_n \neq 0 \).
        \end{itemize}
      \end{thrm}

      \pagebreak \bigbreak \noindent 
      Recall that if \( f \) is a continuous function at a value \( L \), then \( f(x) \to f(L) \) as \( x \to L \). This idea applies to sequences as well. Suppose a sequence \( a_n \to L \), and a function \( f \) is continuous at \( L \). Then \( f(a_n) \to f(L) \). This property often enables us to find limits for complicated sequences. For example, consider the sequence \( \sqrt{5 - \frac{3}{n^2}} \). From Example 5.3a, we know the sequence \( 5 - \frac{3}{n^2} \to 5 \). Since \( \sqrt{x} \) is a continuous function at \( x = 5 \),
      \[ \lim_{{n \to \infty}} \sqrt{5 - \frac{3}{n^2}} = \sqrt{\lim_{{n \to \infty}} 5 - \frac{3}{n^2}} = \sqrt{5} \].
      \bigbreak \noindent 
      \begin{thrm}[Continuous Functions Defined on Convergent Sequences]
          Consider a sequence \( \{a_n\} \) and suppose there exists a real number \( L \) such that the sequence \( \{a_n\} \) converges to \( L \). Suppose \( f \) is a continuous function at \( L \). Then there exists an integer \( N \) such that \( f \) is defined at all values \( a_n \) for \( n \geq N \), and the sequence \( \{f(a_n)\} \) converges to \( f(L) \).
      \end{thrm}

      \bigbreak \noindent 
      \begin{eg}[Limits Involving Continuous Functions Defined on Convergent Sequences]
          Determine whether the sequence  $\{\cos{\left(\frac{3}{n}\right)^{2}}\} $ converges. If it converges, find its limit.
      \end{eg}
      \bigbreak \noindent 
      \textbf{Solution.} Since the sequence $\left(\frac{3}{n^{2}}\right) $ converges to 0 and $\cos{x}$ is continuous at 0, we can conclude that the sequence $\left(\frac{3}{n^{2}}\right) $ converges and
      \begin{align*}
          \lim\limits_{n \to\infty }{\cos{\left(\frac{3}{n^{2}}\right)}}= \cos{0} = 1
      .\end{align*}
      

      \bigbreak \noindent 
      Another theorem involving limits of sequences is an extension of the Squeeze Theorem for limits 
      \bigbreak \noindent 
      \begin{thrm}[Squeeze Theorem for Sequences]
          Consider sequences \( \{a_n\} \), \( \{b_n\} \), and \( \{c_n\} \). Suppose there exists an integer \( N \) such that
        \[ a_n \leq b_n \leq c_n \text{ for all } n \geq N. \]
        If there exists a real number \( L \) such that
        \[ \lim_{{n \to \infty}} a_n = L = \lim_{{n \to \infty}} c_n, \]
        then \( \{b_n\} \) converges and \( \lim_{{n \to \infty}} b_n = L \)
      \end{thrm}

      \pagebreak 
      \phantomsection
      \addcontentsline{toc}{subsection}{Bounded Sequences}
      \subsection*{Bounded Sequences}
      \bigbreak \noindent 
      We now turn our attention to one of the most important theorems involving sequences: the Monotone Convergence Theorem. Before stating the theorem, we need to introduce some terminology and motivation. We begin by defining what it means for a sequence to be bounded.

      \bigbreak \noindent 
      \begin{definition}
          A sequence \( \{a_n\} \) is bounded above if there exists a real number \( M \) such that
        \[ a_n \leq M \]
        for all positive integers \( n \).
        \bigbreak \noindent 
        A sequence \( \{a_n\} \) is bounded below if there exists a real number \( M \) such that
        \[ M \leq a_n \]
        for all positive integers \( n \).
        \bigbreak \noindent 
        A sequence \( \{a_n\} \) is a bounded sequence if it is bounded above and bounded below.
        \bigbreak \noindent 
        If a sequence is not bounded, it is an unbounded sequence.
      \end{definition}

      \bigbreak \noindent 
      For example, the sequence \( \left\{\frac{1}{n}\right\} \) is bounded above because \( \frac{1}{n} \leq 1 \) for all positive integers \( n \). It is also bounded below because \( \frac{1}{n} \geq 0 \) for all positive integers \( n \). Therefore, \( \left\{\frac{1}{n}\right\} \) is a bounded sequence. On the other hand, consider the sequence \( \{2n\} \). Because \( 2n \geq 2 \) for all \( n \geq 1 \), the sequence is bounded below. However, the sequence is not bounded above. Therefore, \( \{2n\} \) is an unbounded sequence.
      \bigbreak \noindent 
    We now discuss the relationship between boundedness and convergence. Suppose a sequence \( \{a_n\} \) is unbounded. Then it is not bounded above, or not bounded below, or both. In either case, there are terms \( a_n \) that are arbitrarily large in magnitude as \( n \) gets larger. As a result, the sequence \( \{a_n\} \) cannot converge. Therefore, being bounded is a necessary condition for a sequence to converge.

    \bigbreak \noindent 
    \begin{thrm}
        If a sequence  $\{a_{n}\} $ converges, then it is bounded. 
    \end{thrm}
    \bigbreak \noindent 
    Note that a sequence being bounded is not a sufficient condition for a sequence to converge. For example, the sequence \( \{(-1)^n\} \) is bounded, but the sequence diverges because the sequence oscillates between 1 and -1 and never approaches a finite number. We now discuss a sufficient (but not necessary) condition for a bounded sequence to converge.
    \bigbreak \noindent 
    Consider a bounded sequence \( \{a_n\} \). Suppose the sequence \( \{a_n\} \) is increasing. That is, \( a_1 \leq a_2 \leq a_3 \ldots \). Since the sequence is increasing, the terms are not oscillating. Therefore, there are two possibilities. The sequence could diverge to infinity, or it could converge. However, since the sequence is bounded, it is bounded above and the sequence cannot diverge to infinity. We conclude that \( \{a_n\} \) converges. For example, consider the sequence
    \bigbreak \noindent 
    \begin{align*}
        \bigg\{\frac{1}{2}, \frac{2}{3}, \frac{3}{4}, \frac{4}{5}, \ldots\bigg\}
    .\end{align*}
    \bigbreak \noindent 
    Since this sequence is increasing and bounded above, it converges. Next, consider the sequence

    \begin{align*}
        \{2,0,3,0,4,0,1,−\frac{1}{2},−\frac{1}{3},−\frac{1}{4},\ldots\}.
    .\end{align*}

    \bigbreak \noindent 
    \begin{definition}[]
        A sequence \( \{a_n\} \) is increasing for all \( n \geq n_0 \) if
        \[ a_n \leq a_{n+1} \text{ for all } n \geq n_0. \]
        \bigbreak \noindent 
        A sequence \( \{a_n\} \) is decreasing for all \( n \geq n_0 \) if
        \[ a_n \geq a_{n+1} \text{ for all } n \geq n_0. \]
        A sequence \( \{a_n\} \) is a \textbf{monotone sequence} for all \( n \geq n_0 \) if it is increasing for all \( n \geq n_0 \) or decreasing for all \( n \geq n_0 \).
    \end{definition}

    \bigbreak \noindent \bigbreak \noindent 
    We now have the necessary definitions to state the Monotone Convergence Theorem, which gives a sufficient condition for convergence of a sequence.
    \begin{thrm}[Monotone Convergence Theorem]
        If \( \{a_n\} \) is a bounded sequence and there exists a positive integer \( n_0 \) such that \( \{a_n\} \) is monotone for all \( n \geq n_0 \), then \( \{a_n\} \) converges.
    \end{thrm}

    % \bigbreak \noindent 
    % \begin{eg}[Monotone Convergence Theorem]
    %     Use theorem 6 to show that the sequence $\bigg\{\frac{4^{n}}{n^{!}}\bigg\} $ converges and find its limit
    % \end{eg}
    % \bigbreak \noindent 
    % Writing out the first few terms, we see that
    % \begin{align*}
    %     \bigg\{\frac{4^{n}}{n!}\bigg\} = \bigg\{4, 8, \frac{32}{3}, \frac{32}{3}, \frac{128}{15} \ldots\bigg\}. 
    % .\end{align*}
    % At first, the terms increase. However, after the third term, the terms decrease. In fact, the terms decrease for all  $n \geq 3$.
    % We can show this as follows:
    % \begin{align*}
    %     a_{n+1} = \frac{4^{n+1}}{(n+1)!} = \frac{4}{n+1} \cdot \frac{4^{n}}{n!} = \frac{4}{n+1} \cdot a_n \leq a_n \text{ if } n \geq 3.
    % .\end{align*}
    %
    % 
    % 
    % 
    %         
    %
    \bigbreak \noindent 
    \begin{eg}[Monotone Convergence Theorem]
       Determine whether the sequence is convergent or divergent. If it is convergent, find its limit. 
       \begin{align*}
           a_{n}= \frac{1000^{n}}{n!}
       .\end{align*}
    \end{eg}
    \bigbreak \noindent 
    \textbf{Solution.} 
        7.f
    \bigbreak \noindent 
    \begin{remark}
        A sequence $\{a_{n}\} $ is a monotone sequence $\forall\ n \geq n_{0}$ if it is increasing $\forall\ n \geq n_{0}$ or decreasing $\forall\ n \geq n_{0}$. If $\{a_{n}\}$ is a bounded sequence  and there exists a positive integer $n_{0}$ s.t $\{a_{n}\} $ is monotone for all $n \geq n_{0}$, then $\{a_{n}\} $ converges
    \end{remark}
    \bigbreak \noindent 
    The first thing to notice about this sequence, is that it begins by increasing, but eventually must become a decreasing sequence as $n!$ grows much faster than $1000^{n}$, to find the value of $n$ for which this switch occurs...
    \bigbreak \noindent 
    \begin{align*}
        a_{n+1} = \frac{1000^{n+1}}{(n+1)!} = \frac{1000}{n+1} \cdot  \frac{1000^{n}}{n!} = \frac{1000}{n+1}\cdot a_{n}
    .\end{align*}
    \bigbreak \noindent 
    Now that we have an equation for the $n+1$ term, we can deduce for which value of $n$ the sequence will start decreasing
    \begin{align*}
        &a_{n+1} < a_{n} \\
        & \frac{1000}{n+1} \cdot a_{n} < a_{n} \\
        &\frac{1000}{n+1} < 1 \\
        &1000 < n+1 \\
        & n > 999
    .\end{align*}
    \bigbreak \noindent 
    By induction, we can show that this is true
    \bigbreak \noindent 
    \begin{prop}
       \forall $n \geq 1000$, $a_{n} > a_{n+1} $ 
    \end{prop}
    
    \pf{Proof}{
        \bigbreak \noindent 
        \bigbreak \noindent 

        Base case: $a_{1000} > a_{1001}$ 
        \begin{align*}
               &\frac{1000^{1000}}{1000!} > \frac{1000^{1001}}{1001!}  \\
               &1000^{1000} (1001)! > 1000^{1001}(1000)! \\
               & 1000^{1000}(1001)(1000)! > 1000^{1001}(1000)! \\
               &1001 > \frac{1000^{1001}}{1000^{1000}} \\
               &10001 > 1000
        .\end{align*}
        \bigbreak \noindent 

        Inductive step: $a_{n} > a_{n+1} $ if we divide $\frac{a_{n+1}}{a_{n+2}}$ ...
        \begin{align*}
            &\frac{\frac{1000^{n}}{n!}}{\frac{1000^{n+1}}{(n+1)!}} \\
            &=\frac{1000^{n}(n+1)!}{1000^{n+1}n!} \\
            &= \frac{1000^{n}(n+1)n!}{1000^{n+1}n!} \\
            &=\frac{1000^{n}(n+1)}{1000^{n+1}} \\
            &= \frac{1}{1000}(n+1)
            % &\frac{1000^{n}}{n!} > \frac{1000^{n+1}}{(n+1)!} \\
            % &1000^{n}(n+1)! > 1000^{n+1}n! \\
            % &1000^{n}(n+1)n! > 1000^{n+1}n! \\
            % &1000^{n}(n+1) > 1000^{n+1} \\
            % &n+1 > \frac{1000^{n+1}}{1000} \\
            % &n+1 > n \quad \text{(By the fact that $\frac{x^{n}}{x^{m}} = x^{n-m}$)}
        .\end{align*}
        for $n \geq 1000 $, $\frac{1}{1000}(n+1) > 1$. $\therefore \frac{a_{n}}{a_{n+1}} > 1 \implies a_{n} > a_{n+1}$
        \bigbreak \noindent 
        
        Induction: $a_{n+1} > a_{n+2}$, we can divide $\frac{a_{n+1}}{a_{n+2}}$ 
        \begin{align*}
            &\frac{\frac{1000^{n+1}}{(n+1)!}}{\frac{1000^{n+2}}{(n+2)!}}    \\
            &= \frac{1000^{n+1}(n+2)!}{1000^{n+2}(n+1)!} \\
            &=\frac{1000^{n+1}(n+2)(n+1)!}{1000^{n+2}(n+1)!} \\
            &=\frac{1000^{n+1}(n+2)}{1000^{n+2}} \\
            &(n+2)\left(\frac{1}{1000}\right)
        .\end{align*}


        For $n \geq 1000$, $(n+2)\left(\frac{1}{1000}\right)  > 1$. $\therefore \frac{a_{n+1}}{a_{n+2}} > 1 \implies a_{n+1} > a_{n+2}$
        \bigbreak \noindent 
        \blacksquare
    }
    \bigbreak \noindent 
    Thus, this sequence is decreasing for $n \geq 1000$. Furthermore, this sequence is bounded below by $0$ because $\frac{(1000)^{n}}{n!} \geq 0,\  \forall\ n \in \mathbb{Z^{+}}$. Therefore, the conditions for the monotone convergence theorem are met and this sequence must converge.
    \bigbreak \noindent 
    Using the fact that this sequence converges, and a finite number of terms does not affect the convergence of a sequence, we can propose
    \begin{align*}
        &\lim\limits_{n \to +\infty}{a_{n+1}} = \lim\limits_{n \to +\infty}{a_{n}} = L  \\
    .\end{align*}
    \bigbreak \noindent 
    Since we know...
    \begin{align*}
        a_{n+1} = \frac{1000}{n+1}\cdot a_{n} 
    .\end{align*}
    We can take the limit of both sides, 
    \begin{align*}
        &\lim\limits_{n \to +\infty}{a_{n+1}} = \lim\limits_{n \to +\infty}{\frac{1000}{n+1}a_{n}} \\
        &L = \frac{1000}{\lim\limits_{n \to +\infty}{n+1}}\cdot \lim\limits_{n \to +\infty}{a_{n}} \\
        &L = 0 \cdot \lim\limits_{n \to +\infty}{a_{n}} \\
        &L = 0
    .\end{align*}

    \pagebreak 
    \phantomsection
    \addcontentsline{toc}{section}{5.2 Infinite Series}
    \section*{5.2 Infinite Series}
    \bigbreak \noindent 

    

      


      


      

      

    





    



\end{document}
