\documentclass{report}

\input{~/dev/latex/template/preamble.tex}
\input{~/dev/latex/template/macros.tex}

\title{\Huge{}}
\author{\huge{Nathan Warner}}
\date{\huge{}}
\pagestyle{fancy}
\fancyhf{}
\lhead{Warner \thepage}
\rhead{}
% \lhead{\leftmark}
\cfoot{\thepage}
% \setborder
% \usepackage[default]{sourcecodepro}
% \usepackage[T1]{fontenc}

\begin{document}
    % \maketitle
    % \tableofcontents
    \pagebreak \bigbreak \noindent
    Nate Warner \ \quad \quad \quad \quad \quad \quad \quad \quad \quad \quad \quad \quad \quad \quad \quad \quad \quad  MATH 230 \quad  \quad \quad \quad \quad \quad \quad \quad \quad \ \ \quad \quad November 11, 2023
    \begin{center}
        \textbf{Homework/Worksheet 10 - Due: Wednesday, November 15}
    \end{center}
    \bigbreak \noindent 
    \begin{mdframed}
        1. Find the radius of convergence and interval of conference of the series.
        \begin{enumerate}[label=(\alph*)]
            \item $\summation{\infty}{n=1}\ \frac{(-1)^{n}x^{n}}{\sqrt{n}}\  $
            \item $\summation{\infty}{n=1}\ \frac{nx^{n}}{e^{n}}\  $
            \item $\summation{\infty}{n=1}\ \frac{10^{n}x^{n}}{n!}\  $
            \item $\summation{\infty}{n=1}\ \frac{(2n)!x^{n}}{n^{2n}}\  $
        \end{enumerate}
    \end{mdframed}
    \bigbreak \noindent 
    \textbf{Problem 1a.} Using the ratio test:
    \begin{align*}
        &\rho = \lim\limits_{n \to \infty}{\bigg\lvert \frac{\frac{(-1)^{n+1}x^{n+1}}{\sqrt{n+1}}}{\frac{(-1)^{n}x^{n}}{\sqrt{n}}} \bigg\rvert}  \\
        &=\lim\limits_{n \to \infty}{\bigg\lvert \frac{(-1)^{n}(-1)x^{n}x}{\sqrt{n+1}} \cdot  \frac{\sqrt{n}}{(-1)^{n}x^{n}} \bigg\rvert} \\
        &=\abs{x} \lim\limits_{n \to \infty}{\frac{\sqrt{n}}{\sqrt{n+1}}}  \\
        &=\abs{x} 
    .\end{align*}
    \bigbreak \noindent 
    Consequently, we have convergence when $\abs{x} < 1$, and divergence for $\abs{x} > 1$
    \begin{align*}
        &\implies  -1 < x < 1 \\
        &\therefore R = 1 
    .\end{align*}
    \bigbreak \noindent 
    Testing the endpoints we see
    \bigbreak \noindent 
    \begin{minipage}[t]{0.4\textwidth}
        When $x=-1$
        \begin{align*}
            &\summation{\infty}{n=1}\ \frac{(-1)^{n}(-1)^{n}}{\sqrt{n}}\  \\
            &=\summation{\infty}{n=1}\ \frac{(-1)^{2n}}{\sqrt{n}}\  \\
            &=\summation{\infty}{n=1}\ \frac{1}{\sqrt{n}}\ 
        .\end{align*}
        Since this is a p-series with $p=\frac{1}{2} $, we have divergence for $x=-1$
    \end{minipage}
    \hspace{1in}
    \begin{minipage}[t]{0.4\textwidth}
        When $x=1$
        \begin{align*}
            &\summation{\infty}{n=1}\ \frac{(-1)^{n}}{\sqrt{n}}\ 
        .\end{align*}
    Using the AST we get 
    \begin{align*}
        &\lim\limits_{n \to \infty}{\frac{1}{\sqrt{n}}} \\
        &=0
    .\end{align*}
    Thus, we have convergence when $x=1$
    \end{minipage}
    \bigbreak \noindent 
    \textbf{Conclusion.} For the power series $\summation{\infty}{n=1}\ \frac{(-1)^{n}x^{n}}{\sqrt{n}}\ $, $R=1$, $I=(-1,1]$

    \pagebreak \bigbreak \noindent 
    \textbf{Problem 1b.} By the ratio test
    \begin{align*}
        &\rho = \lim\limits_{n \to \infty}{\bigg\lvert \frac{(n+1)x^{n}x}{e^{n}e} \cdot \frac{e^{n}}{nx^{n}} \bigg\rvert} \\
        &=\abs{x}\lim\limits_{n \to \infty}{\frac{n+1}{en}} \\
        &=\frac{\abs{x}}{e} 
    .\end{align*}
    \bigbreak \noindent 
        Consequently, we have convergence when $\frac{1}{e}\abs{x} < 1$, and divergence for $\frac{1}{e}\abs{x} > 1$
        \begin{align*}
            &\implies  -e < x < e \\
            &\therefore R = e 
        .\end{align*}
    \bigbreak \noindent 
    Testing the endpoints we see
    \bigbreak \noindent 
    \begin{minipage}[t]{0.4\textwidth}
        When $x=-e$
        \begin{align*}
            &\summation{\infty}{n=1}\ \frac{n(-e)^{n}}{e^{n}}\  \\
            &=\summation{\infty}{n=1}\ n(-1)^{n}\ 
        .\end{align*}
        \bigbreak \noindent 
        Which is divergent
    \end{minipage}
    \hspace{1in}
    \begin{minipage}[t]{0.47\textwidth}
        When $x=e$
        \begin{align*}
            &\summation{\infty}{n=1}\ \frac{ne^{n}}{e^{n}}\  \\
            &=\summation{\infty}{n=1}\ n\ 
        .\end{align*}
        Which is also divergent
    \end{minipage}
    \bigbreak \noindent 
    \textbf{Conclusion.} The power series $\summation{\infty}{n=1}\ \frac{nx^{n}}{e^{n}}\  $ has $R=e$, $I=(-e,e)$
    \bigbreak \noindent 
    \textbf{Problem 1c.} Using the ratio test
    \begin{align*}
        &\rho = \lim\limits_{n \to \infty}{\bigg\lvert \frac{10^{n}10x^{n}x}{n!(n+1)} \cdot \frac{n!}{x^{n}10^{n}} \bigg\rvert} \\
        &=10\abs{x} \lim\limits_{n \to \infty}{\frac{1}{n+1}} \\
        &=0 \\
        &\therefore R = \infty 
    .\end{align*}
    \bigbreak \noindent 
    \textbf{Conclusion.} The power series $\summation{\infty}{n=1}\ \frac{10^{n}x^{n}}{n!}\ $ has $R = \infty$ and $I=\mathbb{R}$ $\implies \text{ convergence } \forall\ x\in\mathbb{R}$

    \pagebreak \bigbreak \noindent 
    \textbf{Problem 1d.} Using the ratio test
    \begin{align*}
        &\rho = \lim\limits_{n \to \infty}{\bigg\lvert \frac{(2n+2)(2n+1)(2n)!x^{n}x}{n^{2n}n^{2}} \cdot \frac{n^{2n}}{(2n)!x^{n}} \bigg\rvert} \\
        &=\abs{x}\lim\limits_{n \to \infty}{\frac{(2n+2)(2n+1)}{n^{2}}} \\
        &=4\abs{x} 
    .\end{align*} 
    \bigbreak \noindent 
    Consequently, we have convergence when $4\abs{x} < 1$, and divergence for $4\abs{x} > 1$
    \begin{align*}
        &\implies  -\frac{1}{4} < x < \frac{1}{4} \\
        &\therefore R = \frac{1}{4}
    .\end{align*}
    \bigbreak \noindent 
    Testing the endpoints we see
    \bigbreak \noindent 
    \begin{minipage}[t]{0.47\textwidth}
        When $x=-\frac{1}{4}$ 
        \begin{align*}
            &\summation{\infty}{n=1}\ \frac{(2n)!\left(-\frac{1}{4}\right)^{n}}{n^{2n}}\  \\
            &=\summation{\infty}{n=1}\ \frac{(2n)!(-1)^{n}\left(\frac{1}{4}\right)^{n}}{n^{2n}}\ 
        .\end{align*}
        \bigbreak \noindent 
        Using Sterling's approximation $n! \approx \sqrt{2\pi n}\left(\frac{n}{e}\right)^{n} \implies (2n)! \approx \sqrt{4\pi n}\left(\frac{2n}{e}\right)^{2n}$ we can rewrite the series as
        \begin{align*}
            &\summation{\infty}{n=1}\ \frac{\sqrt{4\pi n}\left(\frac{2n}{e}\right)^{2n}(-1)^{n}\left(\frac{1}{4}\right)^{n}}{n^{2n}}\  \\
        &=\summation{\infty}{n=1}\  \frac{\sqrt{4\pi n} \left(\frac{n^{2}}{e^{2}}\right)^{n}(-1)^{n}}{n^{2n}}\  \\
        &=\summation{\infty}{n=1}\  \sqrt{4\pi n} \left(\frac{1}{e^{2n}}\right)(-1)^{n}\  \\
        .\end{align*}
        Which we see is an alternating series. Thus, 
        \begin{align*}
            &\lim\limits_{n \to \infty}{\sqrt{4\pi n}\left(\frac{1}{e^{2n}}\right)} \\
            &=\lim\limits_{n \to \infty}{\frac{2 \sqrt{\pi} \sqrt{n}}{e^{2n}}} \\
            &\Heq \lim\limits_{n \to \infty}{\frac{2\sqrt{\pi}}{2e^{2n}2\sqrt{n}}} \\
            &=\lim\limits_{n \to\infty }{\frac{\sqrt{\pi}}{2e^{2n}\sqrt{n}}} \\
            &=0
        .\end{align*}
        Thus, we have convergence at $x=-\frac{1}{4} $
    \end{minipage}
    \hspace{1in}
    \begin{minipage}[t]{0.4\textwidth}
        When $x=\frac{1}{4}$, again we use Sterling's approximation to get the new series
        \begin{align*}
            &\summation{\infty}{n=1}\ \frac{\sqrt{4\pi n}\left(\frac{2n}{e}\right)^{2n}\left(\frac{1}{4}\right)^{n}}{n^{2n}}\  \\
            &=\summation{\infty}{n=1}\ \frac{1}{e^{2n}}\sqrt{4\pi n}\ 
        .\end{align*}
        Using the root test
        \begin{align*}
            &\rho = \lim\limits_{n \to \infty}{\bigg\lvert \frac{\sqrt{4\pi(n+1)}}{e^{2n}e^{2}} \cdot \frac{e^{2n}}{\sqrt{4\pi n}} \bigg\rvert} \\
            &=\lim\limits_{n \to \infty}{ \frac{\sqrt{4\pi n + 4\pi}}{e^{2}} \cdot \frac{1}{2\sqrt{\pi n}} } \\
            &=\lim\limits_{n \to \infty}{ \frac{\sqrt{4(\pi n + \pi)}}{e^{2}} \cdot \frac{1}{2\sqrt{\pi n}} } \\
            &=\lim\limits_{n \to \infty}{ \frac{2\sqrt{\pi n + \pi}}{e^{2}} \cdot \frac{1}{2\sqrt{\pi n}} } \\
            &=\frac{1}{e^{2}}\lim\limits_{n \to \infty}{ \frac{\sqrt{\pi n + \pi}}{\sqrt{\pi n}}} \\ 
            &=\frac{1}{e^{2}}
        .\end{align*}
        \bigbreak \noindent 
        Thus, we also have convergence at $x=\frac{1}{4}$
    \end{minipage}
    \bigbreak \noindent 
    \textbf{Conclusion.} The power series $\summation{\infty}{n=1}\ \frac{(2n)!x^{n}}{n^{2n}}\ $ has $R = \frac{1}{4} $ and $I=[-\frac{1}{4}, \frac{1}{4}] $

    \pagebreak \bigbreak \noindent 
    \begin{mdframed}
        2. Find the power series for each function with the given center a, and identify its interval of convergence.
        \begin{enumerate}[label=(\alph*)]
            \item $f(x) = \frac{1}{x};\ a=1 $ 
            \item $f(x) = \frac{1}{1-x^{2}};\ a=0 $
            \item $f(x) = \frac{1}{2-x};\ a=1 $
        \end{enumerate}
    \end{mdframed}
    \bigbreak \noindent 
    \begin{remark}
        A function of the form $\frac{1}{1-x}$ has the power series $\summation{\infty}{n=0}\ x^{n} = 1 + x + x^{2} + x^{3} + ...  \  $ . A function of the form $\frac{1}{1-(x-a)} $ has a power series $\summation{\infty}{n=0}\ (x-a)^{n}\  $ where $a$ is the center. Additionally, these series converge when $\abs{x} < 1 $ and $\abs{x-a} < 1  $ respectively
    \end{remark}
    
    \bigbreak \noindent 
    \textbf{Problem 2a.} We can rewrite the function as
    \begin{align*}
        f(x) = \frac{1}{1-(1-x)}
    .\end{align*}
    Which means we have the power series 
    \begin{align*}
        \summation{\infty}{n=0}\ (x-1)^{n} = 1 + (x-1) + (x-1)^{2} + (x-1)^{3} + ...\ \text{ for } \abs{x-a} < 1  
    .\end{align*}
    This implies the series will converge for 
    \begin{align*}
        &\abs{x-1} < 1 \\
        &\implies -1 < x-1 < 1 \\
        &\implies 0 < x < 2
   .\end{align*}
   \bigbreak \noindent 
   \textbf{Conclusion.} When $x=0$, the series $\summation{\infty}{n=0}\ (-1)^{n}\  $ diverges. When $x=2$ we have $\summation{\infty}{n=0}\ 1\ $, which also diverges. Therefore $I=(0,2) $

   \bigbreak \noindent 
   \textbf{Problem 2b.} For the function
   \begin{align*}
       f(x) = \frac{1}{1-x^{2}}
   .\end{align*}
   We have the power series 
   \begin{align*}
       \summation{\infty}{n=0}\ x^{2n}\  \text{ for $\abs{x^{2}} < 1$}
   .\end{align*}
   \bigbreak \noindent 
   This implies the series will converge for 
   \begin{align*}
       &\abs{x^{2}} < 1 \\
       &\implies x^{2} < 1 \\
       &\implies -1 < x < 1
   .\end{align*}
   \bigbreak \noindent 
   \textbf{Conclusion.} When $x=-1$ the series $\summation{\infty}{n=0}\ 1\ $ diverges. For $x=1$, the series will also diverge. Thus $I=(-1,1) $

   \pagebreak \bigbreak \noindent 
   \textbf{Problem 2d.} We can rewrite the function as 
   \begin{align*}
       &f(x) = \frac{1}{2-x} \\
       &=\frac{1}{2(1-\frac{x}{2})}
   .\end{align*}
   \bigbreak \noindent 
   This implies we have the power series 
   \begin{align*}
       &\summation{\infty}{n=0}\ \frac{1}{2}(\frac{x}{2})^{n}\  \text{ for $\bigg\lvert \frac{x}{2} \bigg\rvert < 1$}\\
       &=\summation{\infty}{n=0}\ \frac{1}{2^{n+1}}x^{n}\ \text{ for $\bigg\lvert \frac{x}{2} \bigg\rvert < 1$}
   .\end{align*}
   This implies the series will converge for 
   \begin{align*}
       &-2 < x < 2
   .\end{align*}
   \bigbreak \noindent 
   \textbf{Conclusion.} When $x=-2$ the series $\summation{\infty}{n=0}\ \frac{1}{2}(-1)^{n}\  $ will diverge. For $x=2 $, the series $\summation{\infty}{n=0}\ \frac{1}{2}\  $ will diverge. Thus we have $I=(-2,2) $







    





    
    
\end{document}
