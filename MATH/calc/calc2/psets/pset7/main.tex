\documentclass{report}

\input{~/dev/latex/template/preamble.tex}
\input{~/dev/latex/template/macros.tex}

\title{\Huge{}}
\author{\huge{Nathan Warner}}
\date{\huge{}}
\pagestyle{fancy}
\fancyhf{}
\lhead{Warner \thepage}
\rhead{}
% \lhead{\leftmark}
\cfoot{\thepage}
% \setborder
% \usepackage[default]{sourcecodepro}
% \usepackage[T1]{fontenc}

\begin{document}
    % \maketitle
    % \tableofcontents
    \pagebreak \bigbreak \noindent
    Nate Warner \ \quad \quad \quad \quad \quad \quad \quad \quad \quad \quad \quad \quad \quad \quad \quad \quad \quad  MATH 230 \quad  \quad \quad \quad \quad \quad \quad \quad \quad \ \ \quad \quad October 27, 2023
    \begin{center}
        \textbf{Homework/Worksheet 7 - Due: Wednesday, November 1}
    \end{center}
    \bigbreak \noindent 
    \begin{mdframed}
        1.  Use the trapezoidal rule and Simpson’s rule to approximate the integral
        \begin{align*}
            &\int_{0}^{2}\ \frac{e^{x}}{1+x^{2}}\ dx 
        .\end{align*}
        With $n=10$
    \end{mdframed}
    \bigbreak \noindent 
    \textbf{Trapezoidal approximation:}
    \begin{remark}
        $T_{n} = \frac{h}{2}\left[f(x_{0}) + 2f(x_{1}) + 2f(x_{2}) + \cdots + 2f(x_{n-1}) + f(x_{n})\right] $ with $h = \frac{b-a}{n}$
    \end{remark}
    \bigbreak \noindent 
    We have,
    \begin{align*}
        &x_{0} = 0,\ x_{1} = \frac{1}{5},\ x_{2}=\frac{2}{5},\ x_{3} = \frac{3}{5},\ x_{4} = \frac{4}{5},\ x_{5} = 1,\ x_{6} = \frac{6}{5},\ x_{7} = \frac{7}{5},\ x_{8} = \frac{8}{5},\ x_{9} = \frac{9}{5},\ x_{10} = 2 \\
        &f(x_{0}) = 1,\ f(x_{1}) = \frac{e^{\frac{1}{5}}}{1+\frac{1}{25}},\ f(x_{2}) = \frac{e^{\frac{2}{5}}}{1+\frac{4}{25}},\ f(x_{3}) = \frac{e^{\frac{3}{5}}}{1+\frac{9}{25}},\ f(x_{4}) = \frac{e^{\frac{4}{5}}}{1+\frac{16}{25}},\ f(x_{5}) = \frac{e}{2} \\
        &f(x_{6}) = \frac{e^{\frac{6}{5}}}{1+\frac{36}{25}},\ f(x_{7}) = \frac{e^{\frac{7}{5}}}{1+\frac{49}{25}},\ f(x_{8}) = \frac{e^{\frac{8}{5}}}{1+\frac{64}{25}},\ f(x_{9}) = \frac{e^{\frac{9}{5}}}{1+\frac{81}{25}},\ f(x_{10}) = \frac{e^{2}}{5}
    .\end{align*}
    \bigbreak \noindent 
    Thus:
    \begin{align*}
        T_{10} &= \frac{1}{10}\left[1 + \frac{2e^{\frac{1}{5}}}{1+\frac{1}{25}} + \frac{2e^{\frac{2}{5}}}{1+\frac{4}{25}} + \frac{2e^{\frac{3}{5}}}{1+\frac{9}{25}} + \frac{2e^{\frac{4}{5}}}{1+\frac{16}{25}} + e+ \frac{2e^{\frac{6}{5}}}{1+\frac{36}{25}} + \frac{2e^{\frac{7}{5}}}{1+\frac{49}{25}} + \frac{2e^{\frac{8}{5}}}{1+\frac{64}{25}} + \frac{2e^{\frac{9}{5}}}{1+\frac{81}{25}} + \frac{e^{2}}{5}\right] \\
             &\approx 2.6608
    .\end{align*}
    \bigbreak \noindent 
    \textbf{Simpson's rule approximation}
    \bigbreak \noindent 
    \begin{remark}
        $S_{n} = \frac{h}{3}\left[f(x_{0}) + 4f(x_{1}) + 2f(x_{2}) + \cdots + 2f(x_{n-2}) + 4f(x_{n-1}) + f(x_{n})\right] $ where $h = \frac{b-a}{n}$ and $n \in 2k$ for some integer $k $
    \end{remark}
    \bigbreak \noindent 
    Thus,
    \begin{align*}
        S_{10} &= \frac{1}{10}\left[1 + \frac{4e^{\frac{1}{5}}}{1+\frac{1}{25}} + \frac{2e^{\frac{2}{5}}}{1+\frac{4}{25}} + \frac{4e^{\frac{3}{5}}}{1+\frac{9}{25}} + \frac{2e^{\frac{4}{5}}}{1+\frac{16}{25}} + 2e+ \frac{2e^{\frac{6}{5}}}{1+\frac{36}{25}} + \frac{4e^{\frac{7}{5}}}{1+\frac{49}{25}} + \frac{2e^{\frac{8}{5}}}{1+\frac{64}{25}} + \frac{4e^{\frac{9}{5}}}{1+\frac{81}{25}} + \frac{e^{2}}{5}\right] \\
               &\approx 2.6632
    .\end{align*}

    \pagebreak \bigbreak \noindent 
    \begin{mdframed}
        2. Determine whether each integral is convergent or divergent. Evaluate those that are convergent.
        \begin{enumerate}[label=(\alph*)]
            \item $\int_{0}^{+\infty}\ \frac{1}{4+x^{2}}\ dx $  
            \item $\int_{e}^{+\infty}\ \frac{1}{x\ln^{2}{x}}$
            \item $\int_{-\infty}^{+\infty}\ \frac{e^{x}}{1+e^{2x}}\ dx $
            \item $\int_{1}^{+\infty}\ \frac{5}{x^{3}}\ dx $
        \end{enumerate}
    \end{mdframed}

    \bigbreak \noindent 
    2.a
    \begin{align*}
        &\int_{0}^{+\infty}\ \frac{1}{4+x^{2}}\ dx \\
        &=\lim\limits_{t \to +\infty}{\int_{0}^{t}\ \frac{1}{4+x^{2}}\ dx} \\
        &= \lim\limits_{t \to +\infty}{\frac{1}{2}\tan^{-1}{\frac{x}{2}}}\ \bigg|_0^{t} \\
        &= \lim\limits_{t \to +\infty}{\frac{1}{2}\tan^{-1}{\frac{t}{2}}} - \cancelto{0}{\left(\frac{1}{2}\tan^{-1}{0}\right)} \\
        &=\frac{1}{2}\lim\limits_{t \to +\infty}{\cancelto{\frac{\pi}{2}}{\tan^{-1}{\frac{t}{2}}}} \\
        &=\frac{\pi}{4}
    .\end{align*}

    \bigbreak \noindent 
    2.b
    \bigbreak \noindent 
    \begin{minipage}[]{0.47\textwidth}
   \begin{align*}
       &\text{Let $u=\ln{x}$} \\
       &du=\frac{1}{x}\ dx \\
       &\text{When } x=t,\ u=\ln{t} \\
       &\text{When } x=e,\ u=\ln{e} = 1
   .\end{align*} 
    \end{minipage}
    \begin{minipage}[]{0.47\textwidth}
    \begin{align*}
        &\int_{e}^{+\infty}\ \frac{1}{x\ln^{2}{x}}\ dx \\
        &=\lim\limits_{t \to +\infty}{\int_{e}^{t}\ \frac{1}{x\ln^{2}{x}}\ dx} \\
        &=\lim\limits_{t \to +\infty}{\int_{1}^{\ln{t}}\ \frac{1}{u^{2}}\ du} \\
        &=\lim\limits_{t \to +\infty}{-u^{-1}}\ \bigg|_1^{\ln{t}} \\
        &=\lim\limits_{t \to +\infty}{\cancelto{0}{-\frac{1}{\ln{(t)}}}  + 1} \\
        &=1
    .\end{align*}
    \end{minipage}

    \pagebreak \bigbreak \noindent
    2.c
    \bigbreak \noindent 
    \begin{minipage}[]{0.5\textwidth}
        \begin{align*} 
            &\int_{-\infty}^{+\infty}\ \frac{e^{x}}{1+(e^{x})^{2}}\ dx \\
            &\int_{-\infty}^{0}\ \frac{e^{x}}{1+(e^{x})^{2}}\ dx + \int_{0}^{+\infty}\ \frac{e^{x}}{1+(e^{x})^{2}}\ dx \\
            &\lim\limits_{t \to -\infty}{\int_{t}^{0}\ \frac{e^{x}}{1+(e^{x})^{2}}}\ dx +\lim\limits_{t \to +\infty}\int_{0}^{t}\ \frac{e^{x}}{1+(e^{x})^{2}}\ dx
        .\end{align*}
    \end{minipage}
    \begin{minipage}[]{0.47\textwidth}
        \begin{align*}
            &\text{Let $u=e^{x}$}  \\
            &du=e^{x}\ dx
        .\end{align*}
        \begin{minipage}[]{0.47\textwidth}
            \begin{align*}
                I_{1}:\ &\text{When } x=t,\ u=e^{t}\\
                        &\text{When } x=0,\ u=1
            .\end{align*}
        \end{minipage}
        \begin{minipage}[]{0.47\textwidth}
             \begin{align*}
                 I_{2}:\ &\text{When } x=0,\ u=1\\
                &\text{When } x=t,\ u=e^{t}
            .\end{align*}
        \end{minipage}
    \end{minipage}
    \bigbreak \noindent 
    $I_{1}:$
    \begin{align*}
       &\lim\limits_{t \to -\infty}{\int_{t}^{0}\ \frac{e^{x}}{1+(e^{x})^{2}}\ dx}  \\
       &=\lim\limits_{t \to -\infty}{\int_{e^{t}}^{1}\ \frac{du}{1+u^{2}}} \\
       &=\lim\limits_{t \to -\infty}{\tan^{-1}{u}}\ \bigg|_{e^{t}}^{1} \\
       &=\lim\limits_{t \to -\infty}{\tan^{-1}{1} - \tan^{-1}{e^{t}}} \\
       &= \frac{\pi}{4} - \lim\limits_{t \to -\infty}{\tan^{-1}{e^{t}}} \\
       &=\frac{\pi}{4} - \lim\limits_{t \to 0}{\tan^{-1}{t}} \quad \text{(Since $\lim\limits_{t \to -\infty}{e^{t} = 0} $)} \\
       &= \frac{\pi}{4} - 0 \\
       &=\frac{\pi}{4}
    .\end{align*}
    \bigbreak \noindent 
    Thus, $I_{1}$ converges to $\frac{\pi}{4}$
    \bigbreak \noindent 
    $I_{2}$:
    \begin{align*}
        &\lim\limits_{t \to +\infty}{\int_{0}^{t}\ \frac{e^{x}}{1+(e^{x})^{2}}\ dx} \\
        &=\lim\limits_{t \to +\infty}{\int_{1}^{e^{t}}\ \frac{1}{1+u^{2}}\ du} \\
        &=\lim\limits_{t \to +\infty}{\tan^{-1}{u}}\ \bigg|_1^{e^{t}} \\
        &=\lim\limits_{t \to +\infty}{\tan^{-1}{e^{t}}} - \tan^{-1}{0} \\
        &=\lim\limits_{t \to +\infty}{\tan^{-1}{e^{t}}} - \frac{\pi}{4} \\ 
        &=\lim\limits_{t \to +\infty}{\tan^{-1}{t}} - \frac{\pi}{4} \quad \text{(Since $\lim\limits_{t \to +\infty}{e^{t}} = +\infty $)} \\
        &=\frac{\pi}{2}-\frac{\pi}{4} \\
        &=\frac{\pi}{4}
    .\end{align*}
    \bigbreak \noindent 
    Thus, $I_{2}$ also converges to $\frac{\pi}{4}$, Which means we have:
    \begin{align*}
        I &= \frac{\pi}{4} + \frac{\pi}{4} \\
        &= \frac{\pi}{2} 
    .\end{align*}

    \pagebreak \bigbreak \noindent 
    2.d
    \begin{align*}
        &\int_{1}^{+\infty}\ \frac{5}{x^{3}}\ dx \\
        &=\lim\limits_{t \to +\infty}{\int_{1}^{t}\ \frac{5}{x^{3}}\ dx} \\
        &=\lim\limits_{t \to +\infty}{-\frac{5}{2x^{2}}}\ \bigg|_1^{t} \\
        &=\cancelto{0}{\lim\limits_{t \to +\infty}{-\frac{5}{2t^{2}}}} + \frac{5}{2} \\
        &=\frac{5}{2}
    .\end{align*}

    \bigbreak \noindent 
    \begin{mdframed}
        3. Use the Comparison Theorem to determine whether the integral
        \begin{align*}
            \int_{1}^{+\infty}\ \frac{dx}{1+\sqrt{x}}\ dx
        .\end{align*}
        is convergent or divergent
    \end{mdframed}
    \bigbreak \noindent 
    \begin{remark}
        Let $f(x)$ and $g(x)$ be continuous over $[a,+\infty)$, assume $0 \leq f(x) \leq g(x)$ \\
        if $\int_{a}^{+\infty}\ f(x)\ dx = \lim\limits_{t \to +\infty}{\int_{a}^{t}\ f(x)\ dx} = +\infty$ \\
        then $\int_{a}^{+\infty}\ g(x)\ dx  = \lim\limits_{t \to +\infty}{\int_{a}^{t}\ g(x)\ dx} = +\infty$
        \bigbreak \noindent 
        Alternatively,
        \smallbreak \noindent
        if $\int_{a}^{+\infty}\ g(x)\ dx = \lim\limits_{t \to +\infty}{\int_{a}^{t}\ g(x)\ dx} = L$ for $L \in \mathbb{R}$ \\
        then $\int_{a}^{+\infty}\ f(x)\ dx  = \lim\limits_{t \to +\infty}{\int_{a}^{t}\ f(x)\ dx} = M$ for $M \leq L $ where $M \in \mathbb{R}$
    \end{remark}
    \bigbreak \noindent 
    Let $f(x)$ be $\frac{1}{1+\sqrt{x}}$, choose $g(x) = \frac{1}{\sqrt{x}}$. If $g(x)$ diverges to $+\infty$, then $f(x)$ diverges to $+\infty$
    \begin{align*}
        &\int_{1}^{+\infty}\ \frac{1}{\sqrt{x}}\ dx      \\
        &=\lim\limits_{t \to +\infty}{\int_{1}^{t}\ \frac{1}{\sqrt{x}}\ dx} \\
        &=\lim\limits_{t \to +\infty}{2x^{\frac{1}{2}}}\ \bigg|_1^{t} \\
        &=\cancelto{+\infty}{\lim\limits_{t \to +\infty}{2t^{\frac{1}{2}}}} - 2
    .\end{align*}
    \bigbreak \noindent 
    Thus, $f(x) = \frac{1}{1+\sqrt{x}}$ diverges to $+\infty$ since $g(x)  = \frac{1}{\sqrt{x}}$ diverges to $+\infty$ 

    \pagebreak \bigbreak \noindent 
    \begin{mdframed}
        4. Find a formula for the general term of the arithmetic sequence whose first term is \( a_1 = 1 \) such that \( a_{n-1} - a_n = 17 \) for \( n \geq 1 \).
    \end{mdframed}
    \bigbreak \noindent 
    \begin{remark}
       The general form of an arithmetic sequence is of the type $a_{1}+(n-1)d$ for $n \geq 1$. S.t $d$ is the common difference defined $a_{n}- a_{n-1}$, and $a_{1}$ is the first term in the sequence
    \end{remark}

    \bigbreak \noindent 
    With this, we can deduce $a_{n-1} - a_{n} =17 \rightarrow d = -17$. With $a_{1}$ of course defined as $1$. Consequently, the general form of this sequence would be
    \begin{align*}
        &a_{n} = 1 + (n-1)(-17) \\
        &a_{n} = 1 + -17n +17 \\
        &a_{n} = -17n + 18
    .\end{align*}

    \bigbreak \noindent 
    \begin{mdframed}
        5. Find a formula for the general term of the geometric sequence whose first term is \( a_1 = 1 \) such that \(\frac{a_{n+1}}{a_n} = 10\) for \( n \geq 1 \).
    \end{mdframed}
    \bigbreak \noindent 
    \begin{remark}
       The general form of a geometric sequence is of the type $a_{n}= ar^{n-1}$ for $n \geq 1$, where $r$ is the common ratio defined $\frac{a_{n}}{a_{n-1}}$ 
    \end{remark}
    \bigbreak \noindent 
    With this, we can see that the common ratio $r$, is defined as $10$, which makes the general form:
    \begin{align*}
        a_{n} = 10^{n-1}
    .\end{align*}

    \bigbreak \noindent 
    \begin{mdframed}
        6. Find a formula for the general term of the sequence \( \{4, -1, \frac{1}{4}, -\frac{1}{16}, \frac{1}{64}, \ldots\} \).
    \end{mdframed}
    \bigbreak \noindent 
    For $n \geq 1$ this sequence has the general form
    \begin{align*}
        a_{n} = 4\left(-\frac{1}{4}\right)^{n-1}
    .\end{align*}
    \bigbreak \noindent 
    Where the common ratio $r = -\frac{1}{4}$

    \pagebreak \bigbreak \noindent 
    \begin{mdframed}
        7. Determine whether the sequence is convergent or divergent. If it is convergent, find its limit.
        \begin{enumerate}[label=(\alph*)]
            \item $a_{n} = \frac{4+5n^{2}}{1+n} $
            \item $a_{n} = \tan^{-1}{(n^{2})} $
            \item $a_{n} = \ln{\left(\frac{n+2}{n^{2}-3}\right)} $
            \item $a_{n} = n\sin{\left(\frac{1}{n}\right)} $
            \item $a_{n} = \left(1-\frac{2}{n}\right)^{n}$
            \item $a_{n}= \frac{1000^{n}}{n!}$
        \end{enumerate}

    \end{mdframed}

    \bigbreak \noindent 
    7.a
    \begin{align*}
        &\lim\limits_{n \to +\infty}{\frac{4+5n^{2}}{1+n}} \\
        &=\lim\limits_{n \to +\infty}{\frac{\frac{4}{n}+\frac{5n^{2}}{n}}{\frac{1}{n} + \frac{n}{n}}} \\
        &=\lim\limits_{n \to +\infty}{\frac{\frac{4}{n}+5n}{\frac{1}{n}+1}} \\
        &= \frac{\lim\limits_{n \to +\infty}{\frac{4}{n}}+\lim\limits_{n \to +\infty}{5n}}{\lim\limits_{n \to +\infty}{\frac{1}{n}} + \lim\limits_{n \to +\infty}{1}} \\
        &= \frac{0 + \infty}{0 + 1} \\
        &=+\infty
    .\end{align*}
    Thus, this sequence is divergent

    \bigbreak \noindent 
    7.b
    \begin{align*}
        &\lim\limits_{n \to +\infty}{\tan^{-1}{(n^{2})}} \\
        &=\frac{\pi}{2}
    .\end{align*}
    \bigbreak \noindent 
    Thus, this sequence converges

    \bigbreak \noindent 
    7.c
    \begin{align*}
        &\lim\limits_{n \to +\infty}{\ln{\left(\frac{n+2}{n^{2}-3}\right)}} 
    .\end{align*}
    \begin{minipage}[t]{0.47\textwidth}
        Considering the rational function: 
        \begin{align*}
        &\lim\limits_{n \to +\infty}{\left(\frac{n+2}{n^{2}-3}\right)} \\
        &=\lim\limits_{n \to +\infty}{\left(\frac{\frac{n}{n^{2}}+\frac{2}{n^{2}}}{\frac{n^{2}}{n^{2}}-\frac{3}{n^{2}}}\right)} \\
        &=\lim\limits_{n \to +\infty}{\left(\frac{\frac{1}{n}+\frac{2}{n^{2}}}{1-\frac{3}{n^{2}}}\right)}\\
        &= \frac{\lim\limits_{n \to +\infty}{\frac{1}{n}}+\lim\limits_{n \to +\infty}{\frac{2}{n^{2}}}}{\lim\limits_{n \to +\infty}{1}-\lim\limits_{n \to +\infty}{\frac{3}{n^{2}}}} \\
        &=\frac{0+0}{1-0} \\
        &=0
    .\end{align*}
    \end{minipage}
    \begin{minipage}[t]{0.47\textwidth}
        Consequently:
        \begin{align*}
            &\lim\limits_{n \to 0}{\ln{(n)}} \\
            &= -\infty
        .\end{align*}
    Thus, this sequence diverges
    \end{minipage}
    \bigbreak \noindent 

    \pagebreak \bigbreak \noindent 
    7.d
    \begin{align*}
        &\lim\limits_{n \to +\infty}{n\sin{\left(\frac{1}{n}\right)}} \\
        &=\lim\limits_{n \to +\infty}{\frac{\sin{\left(\frac{1}{n}\right)}}{n^{-1}}} \\
        &=\frac{\lim\limits_{n \to +\infty}{\sin{\left(\frac{1}{n}\right)}}}{\lim\limits_{n \to +\infty}{n^{-1}}} \quad \text{(Indeterminate...)}\\
        &=\lim\limits_{n \to +\infty}{\frac{\cos{\left(\frac{1}{n}\right)\cdot\left(-\frac{1}{n^{2}}\right)}}{-\frac{1}{n^{2}}}} \\
        &= \lim\limits_{n \to +\infty}{\cos{\left(\frac{1}{n}\right)}} \\
        &= \lim\limits_{n \to 0}{\cos{(n)}} \quad \text{(Since $ \lim\limits_{n \to +\infty}{\left(\frac{1}{n}\right)} = 0$)} \\
        &= 1
    .\end{align*}

    \bigbreak \noindent 
    7.e
    \bigbreak \noindent 
    \begin{minipage}[t]{0.47\textwidth}
        \begin{align*}
            &\lim\limits_{n \to +\infty}{\left(1-\frac{2}{n}\right)^{n}} \\
            &=\lim\limits_{n \to +\infty}{e^{\ln{\left(1-\frac{2}{n}\right)^{n}}}} \\
            &=\lim\limits_{n \to +\infty}{e^{n\ln{\left(1-\frac{2}{n}\right)}}} \\
            &\lim\limits_{n \to +\infty}{n\ln{\left(1-\frac{2}{n}\right)}} \\
            &=\lim\limits_{n \to +\infty}{\frac{\ln{\left(1-\frac{2}{n}\right)}}{n^{-1}}} \quad \text{(Indeterminate...)}\\
            &\implies\lim\limits_{n \to +\infty}{\frac{\frac{1}{1-\frac{2}{n}}\cdot \frac{2}{n^{2}}}{-\frac{1}{n^{2}}}} \\
            &=\lim\limits_{n \to +\infty}{\frac{\frac{2}{\left(1-\frac{2}{n}\right)n^{2}}}{-\frac{1}{n^{2}}}} \\
            &=\lim\limits_{n \to +\infty}{\frac{\frac{2}{n(n-2)}}{-\frac{1}{n^{2}}}} \\
            &=\lim\limits_{n \to +\infty}{-\frac{2n^{2}}{n(n-2)}} \\
            &=\lim\limits_{n \to +\infty}{-\frac{2n}{n-2}} \\
            &=-2
        .\end{align*}
    \end{minipage}
    \begin{minipage}[t]{0.47\textwidth}
        Thus: 
        \begin{align*}
            &\lim\limits_{n \to +\infty}{\left(1-\frac{2}{n}\right)^{n}} \\
            &=\lim\limits_{n \to +\infty}{e^{\ln{\left(1-\frac{2}{n}\right)^n}}} \\
            &=\lim\limits_{n \to +\infty}{e^{n\ln{\left(1-\frac{2}{n}\right)}}} \\
            &=\lim\limits_{n \to +\infty}{e^{-2}} \\
            &=e^{-2} \\
            &=\frac{1}{e^{2}}
        .\end{align*}
    \end{minipage}

    \pagebreak \bigbreak \noindent 
    7.f
    % \bigbreak \noindent 
    % \begin{remark}
    %     A sequence $\{a_{n}\} $ is a monotone sequence $\forall\ n \geq n_{0}$ if it is increasing $\forall\ n \geq n_{0}$ or decreasing $\forall\ n \geq n_{0}$. If $\{a_{n}\}$ is a bounded sequence  and there exists a positive integer $n_{0}$ s.t $\{a_{n}\} $ is monotone for all $n \geq n_{0}$, then $\{a_{n}\} $ converges
    % \end{remark}
    % \bigbreak \noindent 
    % The first thing to notice about this sequence, is that it begins by increasing, but eventually must become a decreasing sequence as $n!$ grows much faster than $1000^{n}$, to find the value of $n$ for which this switch occurs...
    % \bigbreak \noindent 
    % \begin{align*}
    %     a_{n+1} = \frac{1000^{n+1}}{(n+1)!} = \frac{1000}{n+1} \cdot  \frac{1000^{n}}{n!} = \frac{1000}{n+1}\cdot a_{n}
    % .\end{align*}
    % \bigbreak \noindent 
    % Now that we have an equation for the $n+1$ term, we can deduce for which value of $n$ the sequence will start decreasing
    % \begin{align*}
    %     &a_{n+1} < a_{n} \\
    %     & \frac{1000}{n+1} \cdot a_{n} < a_{n} \\
    %     &\frac{1000}{n+1} < 1 \\
    %     &1000 < n+1 \\
    %     & n > 999
    % .\end{align*}
    % \bigbreak \noindent 
    % By induction, we can show that this is true
    % \bigbreak \noindent 
    % \begin{prop}
    %    \forall $n \geq 1000$, $a_{n} > a_{n+1} $ 
    % \end{prop}
    % 
    % \pf{Proof}{
    %     \bigbreak \noindent 
    %     \bigbreak \noindent 
    %
    %     Base case: $a_{1000} > a_{1001}$ 
    %     \begin{align*}
    %            &\frac{1000^{1000}}{1000!} > \frac{1000^{1001}}{1001!}  \\
    %            &1000^{1000} (1001)! > 1000^{1001}(1000)! \\
    %            & 1000^{1000}(1001)(1000)! > 1000^{1001}(1000)! \\
    %            &1001 > \frac{1000^{1001}}{1000^{1000}} \\
    %            &1001 > 1000
    %     .\end{align*}
    %     \bigbreak \noindent 
    %
    %     Inductive step: $a_{n} > a_{n+1} $ if we divide $\frac{a_{n+1}}{a_{n+2}}$ ...
    %     \begin{align*}
    %         &\frac{\frac{1000^{n}}{n!}}{\frac{1000^{n+1}}{(n+1)!}} \\
    %         &=\frac{1000^{n}(n+1)!}{1000^{n+1}n!} \\
    %         &= \frac{1000^{n}(n+1)n!}{1000^{n+1}n!} \\
    %         &=\frac{1000^{n}(n+1)}{1000^{n+1}} \\
    %         &= \frac{1}{1000}(n+1)
    %         % &\frac{1000^{n}}{n!} > \frac{1000^{n+1}}{(n+1)!} \\
    %         % &1000^{n}(n+1)! > 1000^{n+1}n! \\
    %         % &1000^{n}(n+1)n! > 1000^{n+1}n! \\
    %         % &1000^{n}(n+1) > 1000^{n+1} \\
    %         % &n+1 > \frac{1000^{n+1}}{1000} \\
    %         % &n+1 > n \quad \text{(By the fact that $\frac{x^{n}}{x^{m}} = x^{n-m}$)}
    %     .\end{align*}
    %     for $n \geq 1000 $, $\frac{1}{1000}(n+1) > 1$. $\therefore \frac{a_{n}}{a_{n+1}} > 1 \implies a_{n} > a_{n+1}$
    %     \bigbreak \noindent 
    %     
    %     Induction: $a_{n+1} > a_{n+2}$, we can divide $\frac{a_{n+1}}{a_{n+2}}$ 
    %     \begin{align*}
    %         &\frac{\frac{1000^{n+1}}{(n+1)!}}{\frac{1000^{n+2}}{(n+2)!}}    \\
    %         &= \frac{1000^{n+1}(n+2)!}{1000^{n+2}(n+1)!} \\
    %         &=\frac{1000^{n+1}(n+2)(n+1)!}{1000^{n+2}(n+1)!} \\
    %         &=\frac{1000^{n+1}(n+2)}{1000^{n+2}} \\
    %         &(n+2)\left(\frac{1}{1000}\right)
    %     .\end{align*}
    %
    %
    %     For $n \geq 1000$, $(n+2)\left(\frac{1}{1000}\right)  > 1$. $\therefore \frac{a_{n+1}}{a_{n+2}} > 1 \implies a_{n+1} > a_{n+2}$
    %     \bigbreak \noindent 
    %     \blacksquare
    % }
    % \bigbreak \noindent 
    % Thus, this sequence is decreasing for $n \geq 1000$. Furthermore, this sequence is bounded below by $0$ because $\frac{(1000)^{n}}{n!} \geq 0,\  \forall\ n \in \mathbb{Z^{+}}$. Therefore, the conditions for the monotone convergence theorem are met and this sequence must converge.
    % \bigbreak \noindent 
    % Using the fact that this sequence converges, and a finite number of terms does not affect the convergence of a sequence, we can propose
    % \begin{align*}
    %     &\lim\limits_{n \to +\infty}{a_{n+1}} = \lim\limits_{n \to +\infty}{a_{n}} = L  \\
    % .\end{align*}
    % \bigbreak \noindent 
    % Since we know...
    % \begin{align*}
    %     a_{n+1} = \frac{1000}{n+1}\cdot a_{n} 
    % .\end{align*}
    % We can take the limit of both sides, 
    % \begin{align*}
    %     &\lim\limits_{n \to +\infty}{a_{n+1}} = \lim\limits_{n \to +\infty}{\frac{1000}{n+1}a_{n}} \\
    %     &L = \frac{1000}{\lim\limits_{n \to +\infty}{n+1}}\cdot \lim\limits_{n \to +\infty}{a_{n}} \\
    %     &L = 0 \cdot \lim\limits_{n \to +\infty}{a_{n}} \\
    %     &L = 0
    % .\end{align*}
    % 
    %
    %

    \begin{align*}
        &\lim\limits_{n \to +\infty}{\frac{1000^{n}}{n!}}    
    .\end{align*}
    \bigbreak \noindent 
    \begin{prop}
       Since factorial growth is much faster than exponential growth, as $n$ increases without bound, the denominator $n!$ will become much larger than the numerator $1000^{n}$. Thus, this limit must be zero and we can conclude that this sequence converges.
    \end{prop}
    \bigbreak \noindent 
    So we can say:
    \begin{align*}
        \lim\limits_{n \to +\infty}{\frac{1000^{n}}{n!}} = 0
    .\end{align*}
    




    
    



    
    
\end{document}
