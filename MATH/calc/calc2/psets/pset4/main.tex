\documentclass{report}

\input{~/dev/latex/template/preamble.tex}
\input{~/dev/latex/template/macros.tex}

\title{\Huge{}}
\author{\huge{Nathan Warner}}
\date{\huge{}}
\pagestyle{fancy}
\fancyhf{}
\lhead{Warner \thepage}
\rhead{}
% \lhead{\leftmark}
\cfoot{\thepage}
% \setborder
% \usepackage[default]{sourcecodepro}
% \usepackage[T1]{fontenc}

\begin{document}
    % \maketitle
    % \tableofcontents
    \pagebreak \bigbreak \noindent
    Nate Warner \ \quad \quad \quad \quad \quad \quad \quad \quad \quad \quad \quad \quad \quad \quad \quad \quad \quad  MATH 230 \quad  \quad \quad \quad \quad \quad \quad \quad \quad \ \ \quad \quad September 28, 2023
    \begin{center}
        \textbf{Homework/Worksheet 4 - Due: Wednesday, October 4}
    \end{center}
    \bigbreak \noindent 
    \begin{mdframed}
        \textbf{1.) Find the length of the functions below over the given interval. If you cannot evaluate the integral exactly, use technology to approximate it.}
        \bigbreak \noindent 
            \begin{enumerate}[label=(\alph*)]
               \item \(y = x^{\frac{3}{2}}\) from \((1, 1)\) to \((8, 4)\)
                \item \(y = \frac{1}{3}(x^2 - 2)^{\frac{3}{2}}\) from \(x = 2\) to \(x = 4\)
                \item \(y = \frac{x}{3} + \frac{1}{4x}\) from \(x = 1\) to \(x = 4\) 
            \end{enumerate}
    \end{mdframed}

    \bigbreak \noindent 
    \textbf{1.a}
    \bigbreak \noindent 
        \begin{align*}
            \frac{d}{dx}x^{\frac{3}{2}} = \frac{3}{2}x^{\frac{1}{2}}
        .\end{align*}
    Thus:
    \begin{align*}
        s &= \int_{1}^{8}\ \sqrt{1+\bigg(\frac{3}{2}x^{\frac{1}{2}}\bigg)^{2}}\ dx \\
          &= \int_{1}^{8}\ \sqrt{1+3x}\ dx \\
          &\text{Let}\ u=1+3x \\
          &du = 3dx \\
          &\frac{1}{3}du=dx \\
          &u(a) = 4 \\
          &u(b) = 25 \\
          &\frac{1}{3}\int_{4}^{25}\ \sqrt{u}\ du \\
          &=\frac{1}{3}\left[\frac{2}{3}u^{\frac{3}{2}}\right]_4^{25} \\
          &=\frac{2}{9}\left[u^{\frac{3}{2}}\right]^{25}_4 \\
          &=\frac{2}{9}\bigg[25^{\frac{3}{2}}-4^{\frac{3}{2}}\bigg] \\
          &=\frac{2}{9}\bigg[125-8\bigg] \\
          &=26
    .\end{align*}

    \pagebreak \bigbreak \noindent 
    \textbf{1.b}
    \bigbreak \noindent 
    \begin{align*}
        &\frac{1}{3}\bigg[\frac{d}{dx}(x^{2}-2)^{\frac{3}{2}}\bigg] \\
        &\frac{1}{3}\bigg[\frac{3}{2}(x^{2}-2)^{\frac{1}{2}}\bigg] \cdot 2x \\
        &=x(x^{2}-2)^{\frac{1}{2}}
    .\end{align*}
    \bigbreak \noindent 
    Thus:
    \begin{align*}
        s &= \int_{2}^{4}\ \sqrt{1+(x(x^{2}-2)^{\frac{1}{2}})^{2}}\ dx \\
          &=\int_{2}^{4}\ \sqrt{1+x^{2}(x^{2}-2)}\ dx \\
          &=\int_{2}^{4}\ \sqrt{1+x^{4}-2x^{2}}\ dx \\
          &=\int_{2}^{4}\ \sqrt{(x^{2}-1)^{2}}\ dx
    .\end{align*}
    \bigbreak \noindent 
    Since we know the domain is nonnegative, we can rewrite as:
    \begin{align*}
       &\int_{2}^{4}\ x^{2}-1\ dx  \\
       &=\frac{1}{3}x^{3}-x\ \bigg|_2^{4} \\
       &=\bigg(\frac{1}{3}(4)^{3}-4\bigg) - \bigg(\frac{1}{3}(2)^{3} -2\bigg) \\
       &=\frac{64}{3}-4 - \bigg(\frac{8}{3}-2\bigg) \\
       &=\frac{50}{3}
    .\end{align*}

    \pagebreak \bigbreak \noindent 
    \textbf{1.c}
    \bigbreak \noindent 
    \begin{align*}
        &\frac{d}{dx}\frac{1}{3}x^{3}+\frac{1}{4}x^{-1}\\
        &=x^{2}-\frac{1}{4}x^{-2}
    .\end{align*}
    \bigbreak \noindent 
    Thus:
    \begin{align*}
        s &= \int_{1}^{4}\ \sqrt{1+\bigg(x^{2}-\frac{1}{4x^{2}}\bigg)^{2}}\ dx \\
          &=\int_{1}^{4}\ \sqrt{1+x^{4}-2\bigg(\frac{1}{4}\bigg)+\frac{1}{16x^{4}}}\ dx \\
          &=\int_{1}^{4}\ \sqrt{1+x^{4}-\frac{1}{2}+\frac{1}{16x^{4}}}\ dx \\
          &=\int_{1}^{4}\ \sqrt{x^{4}+\frac{1}{16x^{4}}+\frac{1}{2}}\ dx \\
          &= \int_{1}^{4}\ \sqrt{(x^{2})^{2}+\frac{1}{(4x^{2})^{2}}+\frac{1}{2}}\ dx \\
          &=\int_{1}^{4}\ \sqrt{\bigg(x^{2}+\frac{1}{4x^{2}}\bigg)^{2}}\ dx \\
          &=\int_{1}^{4}\ \sqrt{\bigg(\frac{4x^{4}+1}{4x^{2}}\bigg)^{2}}\ dx
    .\end{align*}
    \bigbreak \noindent 
    Since the domain is \textit{nonnegative}, we can rewrite as:
    \begin{align*}
        \hspace{0.6in} &\int_{1}^{4}\ \frac{4x^{4}+1}{4x^{2}}\ dx \\
        &=\int_{1}^{4}\ x^{2}+\frac{1}{4x^{2}}\ dx \\
        &=\int_{1}^{4}\ x^{2}+\frac{1}{4}x^{-2}\ dx \\
        &=\frac{1}{3}x^{3}-\frac{1}{4}x^{-1}\ \bigg|_1^{4} \\
        &= \bigg(\frac{1}{3}(4)^{3}-\frac{1}{4}(4)^{-1}\bigg) - \bigg(\frac{1}{3}(1)^{3}-\frac{1}{4}(1)^{-1}\bigg) \\
        &= \frac{339}{16}
    .\end{align*}

    \pagebreak \bigbreak \noindent 
    \begin{mdframed}
        \textbf{2.) Find the surface area of the volume generated by revolving the curve \(y = x^3\), \(0 \leq x \leq 1\), around the \(x\)-axis.}
    \end{mdframed}
    \bigbreak \noindent 
    \begin{align*}
        \frac{d}{dx}x^{3} = 3x^{2}
    .\end{align*}
    \bigbreak \noindent 
    Thus:
    \begin{align*}
        sa &= \int_{0}^{1}\ 2\pi x^{3}\sqrt{1+(3x^{2})^{2}}\ dx \\
           &= 2\pi\int_{0}^{1}\ x^{3}\sqrt{1+9x^{4}}\ dx
    .\end{align*}
    \begin{minipage}[]{0.47\textwidth}
        \begin{align*}
            &\text{Let $u=1+9x^{4}$} \\
            &du = 36x^{3}dx \\
            &\frac{1}{36}du = x^{3}dx \\
            &u(a) = 1 \\
            &u(b) = 10
        .\end{align*}
    \end{minipage}
    \begin{minipage}[]{0.47\textwidth}
        Thus:
        \begin{align*}
            sa &= \frac{2\pi}{36}\int_{1}^{10}\ u^\frac{1}{2}\ du \\
               &=\frac{\pi}{18}\bigg[\frac{2}{3}u^{\frac{3}{2}}\bigg]_1^{10} \\
               &=\frac{\pi}{27}\bigg[\bigg((10)^{\frac{3}{2}}-1\bigg)\bigg] \\
               &=\frac{10^{\frac{3}{2}}\pi}{27} - \frac{\pi}{27} \\
               &= \frac{10^{\frac{3}{2}}\pi-\pi}{27}
        .\end{align*}
    \end{minipage}
    \bigbreak \noindent 
    \begin{mdframed}
        \textbf{3.) Find the surface area of the volume generated by revolving the curve \(y = 3x^4\), \(0 \leq x \leq 1\), around the \(y\)-axis.}
    \end{mdframed}
    \bigbreak \noindent 
    \begin{minipage}[]{0.47\textwidth}
        Derivative:
        \begin{align*}
            &\frac{d}{dx}3x^{4} = 12x^{3}
        .\end{align*}
    \end{minipage}
    \begin{minipage}[]{0.47\textwidth}
        Thus:
        \begin{align*}
            sa &= \int_{0}^{1}\ 2\pi x\sqrt{1+(12x^{3})^{2}}\ dx \\
            &=2\pi \int_{0}^{1}\ x\sqrt{1+144x^{6}}\ dx \\
            &\approx 15.8264
        .\end{align*}
    
    \end{minipage}

    \pagebreak \bigbreak \noindent 
    \begin{mdframed}
        \textbf{4.) Evaluate the following integrals}
        \bigbreak \noindent 
        \begin{enumerate}[label=(\alph*)]
            \item $\int \frac{(\ln{(x)})^{2}}{x}\ dx $
            \item $\int_{0}^{\frac{\pi}{4}}\ \tan{x}\ dx$
        \end{enumerate}
    \end{mdframed}

    \bigbreak \noindent 
    \textbf{4.a)}
    \bigbreak \noindent 
    \begin{align*}
        &\int_{}^{}\ \frac{(\ln{x})^{2}}{x}\ dx \\
        &\text{Let $u = \ln{x} $} \\
        &du = \frac{1}{x}dx \\
        &\implies \int_{}^{}\ u^{2}\ du \\
        &= \frac{1}{3}u^{3} + C \\
        &= \frac{1}{3}(\ln{x})^{3} + C \\
        &= \frac{1}{3}\ln^{3}{x} + C
    .\end{align*}

    \bigbreak \noindent 
    \textbf{4.b}
    \bigbreak \noindent 
    \begin{minipage}[t]{0.47\textwidth}
        Integrate:
    \begin{align*}
        &\int_{}^{}\ \tan{x}\ dx \\
        &=\int_{}^{}\ \frac{\sin{x}}{\cos{x}}\ dx \\
        &\text{Let $u=\cos{x}$} \\
        &du=-\sin{x}dx\\
        &-du = \sin{x}dx \\
        &\implies \int_{}^{}\ u^{-1}\ du \\
        &=-\ln{\abs{u}} + C \\
        &= -\ln{\abs{\cos{x}}} + C \\
        &= \ln{\abs{\frac{1}{\cos{x}}}}  + C\\
        &= \ln{\abs{\sec{x}}} + C
    .\end{align*}
    \end{minipage}
    \begin{minipage}[t]{0.47\textwidth}
        Thus:
        \begin{align*}
           &\ln{\abs{\sec{x}}}\ \bigg|_0^{\frac{\pi}{3}} \\
           &=\ln{\abs{\sec{\frac{\pi}{3}}}}  - \ln{\abs{\sec{0}}} \\
           &= \ln{2} - \ln{1} \\
           &= \ln{2}
        .\end{align*}
    \end{minipage}

    \pagebreak \bigbreak \noindent 
    \begin{mdframed}
        \textbf{5.) Compute the derivative,  $\frac{dy}{dx}$  of the following functions.}
            \begin{enumerate}[label=(\alph*)]
                \item $\quad y = e^{\sin x}$ 
                \item $\quad y = x e^x$ 
                \item $\quad y = \frac{x^{-1}}{\ln x}$
            \end{enumerate}
    \end{mdframed}
    \bigbreak \noindent 
    \textbf{5.a)}
    \begin{align*}
        &\frac{d}{dx}e^{\sin{x}} \\
        &=e^{\sin{x}} \cdot \frac{d}{dx} \sin{x} \\
        &= \cos{x}e^{\sin{x}}
    .\end{align*}
    \bigbreak \noindent 
    \textbf{5.b)}
    \bigbreak \noindent 
    By the product rule:
    \begin{align*}
       &\frac{d}{dx} xe^{x} \\
       &=xe^{x}+e^{x} \\
       &=e^{x}(x+1)
    .\end{align*}

    \bigbreak \noindent 
    \textbf{5.c}
    \bigbreak \noindent 
    \begin{align*}
        &\frac{d}{dx} \frac{1}{x\ln{x}} \\
        &\frac{1}{x\ln{x}} \cdot \frac{d}{dx}\ln{\left(\frac{1}{x\ln{x}}\right)} \\
        &=\frac{1}{x\ln{x}} \cdot \frac{d}{dx}\left[-\ln{\left(x\ln{x}\right)}\right] \\
        &=\frac{1}{x\ln{x}} \cdot  \frac{d}{dx} \left[-\ln{x} +\ln{(\ln{(x)})}\right] \\
        &=\frac{1}{x\ln{x}} \left[-\frac{1}{x} + \frac{1}{x\ln{x}}\right] \\
        &= \frac{1}{x\ln{x}} \left[-\frac{\ln{x} + 1}{x\ln{x}}\right] \\
        &= -\frac{\ln{x} + 1}{x^{2}\ln^{2}{x}}
    .\end{align*}



    
    

\end{document}
