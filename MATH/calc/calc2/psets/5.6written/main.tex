\documentclass{report}

\input{~/dev/latex/template/preamble.tex}
\input{~/dev/latex/template/macros.tex}

\title{\Huge{}}
\author{\huge{Nathan Warner}}
\date{\huge{}}
\pagestyle{fancy}
\fancyhf{}
\lhead{Warner \thepage}
\rhead{}
% \lhead{\leftmark}
\cfoot{\thepage}
%\setborder
% \usepackage[default]{sourcecodepro}
% \usepackage[T1]{fontenc}

\begin{document}
    % \maketitle
    %     \begin{titlepage}
    %    \begin{center}
    %        \vspace*{1cm}
    %
    %        \textbf{}
    %
    %        \vspace{0.5cm}
    %         
    %             
    %        \vspace{1.5cm}
    %
    %        \textbf{Nathan Warner}
    %
    %        \vfill
    %             
    %             
    %        \vspace{0.8cm}
    %      
    %        \includegraphics[width=0.4\textwidth]{~/niu/seal.png}
    %             
    %        Computer Science \\
    %        Northern Illinois University\\
    %        United States\\
    %        
    %             
    %    \end{center}
    % \end{titlepage}
    % \tableofcontents
    % \pagebreak \bigbreak \noindent
    \begin{remark}
       For a series $\summation{\infty}{n=1}\ a_{n}\ $ with positive terms. Then
       \begin{align*}
           &\rho = \lim\limits_{n \to \infty}{\bigg\lvert \frac{a_{n+1}}{a_{n}} \bigg\rvert} \quad \text{(Ratio test)} \\
           \text{Or:}\ &\rho = \lim\limits_{n \to \infty}{\lvert a_{n}\rvert^{\frac{1}{n}}} 
       .\end{align*}
       \begin{enumerate}[label=\roman*.]
            \item If $0 \leq \rho < 1 $, the series will converge absolutely
            \item If $\rho > 1$ or $\rho = +\infty$, the series will diverge
            \item if $\rho  = 1$, the test is inconclusive
       \end{enumerate}
    \end{remark}

    \bigbreak \noindent 
    \begin{mdframed}
        \textbf{Problem 1.} 
Test the series below for convergence using the Ratio Test.
        \begin{align*}
            \summation{\infty}{n=1}\ \frac{n^{4}}{0.5^{n}}\ 
        .\end{align*}
    \end{mdframed}
    \bigbreak \noindent 
    \begin{align*}
        &\rho = \lim\limits_{n \to \infty}{\bigg\lvert \frac{\frac{(n+1)^{4}}{0.5^{n+1}}}{\frac{n^{4}}{0.5^{n}}} \bigg\rvert}  \\
        &=\lim\limits_{n \to \infty}{\bigg\lvert \frac{(n+1)^{4}}{0.5^{n+1}}  \cdot  \frac{0.5^{n}}{n^{4}}\bigg\rvert} \\
        &=\lim\limits_{n \to \infty}{\frac{(n+1)^{4}}{0.5n^{4}}} \\
        &=2
    .\end{align*}
    \bigbreak \noindent 
    \textbf{Conclusioin. }Thus, this series will diverge

    \bigbreak \noindent 
    \begin{mdframed}
        \textbf{Problem 2.} Test the series below for convergence using the Ratio Test.
    \begin{align*}
        \summation{\infty}{n=1}\ \frac{n+2}{3^{4n+3}}     
    .\end{align*}
    \end{mdframed}
    \bigbreak \noindent 
    \begin{align*}
        &\rho = \lim\limits_{n \to \infty}{\bigg\lvert \frac{\frac{(n+1)+2}{3^{4(n+1)+3}}}{\frac{n+2}{3^{4n+3}}} \bigg\rvert} \\
        &=\lim\limits_{n \to \infty}{\bigg\lvert \frac{\frac{n+3}{3^{4n+7}}}{\frac{n+2}{3^{4n+3}}} \bigg\rvert} \\
        &=\lim\limits_{n \to \infty}{\bigg\lvert \frac{n+3}{3^{4n+7}} \cdot \frac{3^{4n+3}}{n+2} \bigg\rvert} \\
        &=\lim\limits_{n \to \infty}{\frac{n+3}{3^{4}(n+2)}} \\
        &=\lim\limits_{n \to \infty}{\frac{n+3}{81n+162}} \\
        &=\frac{1}{81}
    .\end{align*}
    \bigbreak \noindent 
    Thus, this series will converge absolutely

    \pagebreak \bigbreak \noindent 
    \begin{mdframed}
        \textbf{Problem 3.} Given the series, 
        \begin{align*}
            \summation{\infty}{n=1}\ \frac{n^{3}}{9^{n}}\ 
        .\end{align*}
        \bigbreak \noindent 
        Use the ratio test to test for convergence
    \end{mdframed}
    \bigbreak \noindent 
    \begin{align*}
        &\rho = \lim\limits_{n \to \infty}{\bigg\lvert\frac{\frac{(n+1)^{3}}{9^{n+1}}}{\frac{n^{3}}{9^{n}}}\bigg\rvert } \\
        &=\lim\limits_{n \to \infty}{\bigg\lvert \frac{(n+1)^{3}}{9^{n+1}} \cdot  \frac{9^{n}}{n^{3}}\bigg\rvert} \\
        &=\lim\limits_{n \to \infty}{\frac{(n+1)^{3}}{9n^{3}}} \\
Test the series below for convergence using the Root Test.
        &=\frac{1}{9}
    .\end{align*}
    \bigbreak \noindent 
    Thus, this series will converge absolutely

    \bigbreak \noindent 
    \begin{mdframed}
        \textbf{Problem 4.} Given the series
        \begin{align*}
            \summation{\infty}{n=1}\ \frac{8^{n}}{n!}\ 
        .\end{align*}
        \bigbreak \noindent 
        Use ratio test to test for converge
    \end{mdframed}
   \bigbreak \noindent 
   \begin{align*}
       &\rho = \lim\limits_{n \to \infty}{\bigg\lvert \frac{\frac{8^{n+1}}{(n+1)!}}{\frac{8^{n}}{n!}} \bigg\rvert} \\
       &=\lim\limits_{n \to \infty}{\bigg\lvert \frac{8^{n+1}}{(n+1)!} \cdot \frac{n!}{8^{n}} \bigg\rvert} \\
       &=\lim\limits_{n \to \infty}{\frac{8}{n+1}} \\ 
       &=0
   .\end{align*}
   \bigbreak \noindent 
   Thus, this series will converge absolutely


   \pagebreak \bigbreak \noindent 
   \begin{mdframed}
       \textbf{Problem 5.} Test the series below for convergence using the Root Test.
       \begin{align*}
           \summation{\infty}{n=1}\ \left(\frac{2n}{6n+4}\right)^{n}\ 
       .\end{align*}
   \end{mdframed}
   \bigbreak \noindent 
   \begin{align*}
       &\rho = \lim\limits_{n \to \infty}{\bigg\lvert \left(\frac{(2n)^{n}}{(6n+4)^{n}}\right)\bigg\rvert^{\frac{1}{n}}} \\
       &=\lim\limits_{n \to \infty}{\frac{2n}{6n+4}} \\
       &=\frac{1}{3}
   .\end{align*}
   \bigbreak \noindent 
   Thus, this series will converge absolutely


   \bigbreak \noindent 
   \begin{mdframed}
       \textbf{Problem 6.} Test the series below for convergence using the Root Test.
       \begin{align*}
           &\summation{\infty}{n=1}\ \left(\frac{6n}{3n+2}\right)^{n}
       .\end{align*}
   \end{mdframed}
   \bigbreak \noindent 
   \begin{align*}
       &\rho  = \lim\limits_{n \to \infty}{\bigg\lvert \left(\frac{(6n)^{n}}{(3n+2)^{n}}\right) \bigg\rvert^{\frac{1}{n}}} \\
       &=\lim\limits_{n \to \infty}{\frac{6n}{3n+2}} \\
       &=2
   .\end{align*}
   \bigbreak \noindent 
   Thus, this series will diverge

   \bigbreak \noindent 
   \begin{mdframed}
       \textbf{Problem 7.} Test the series below for convergence using the Root Test.
       \begin{align*}
           \summation{\infty}{n=1}\ \left(\frac{3n+6}{5n+3}\right)^{n}\ 
       .\end{align*}
   \end{mdframed}
   \bigbreak \noindent 
   \begin{align*}
       &\rho = \lim\limits_{n \to \infty}{\bigg\lvert \left(\frac{(3n+6)^{n}}{(5n+3)^{n}}\right) \bigg\rvert^{\frac{1}{n}}} \\
       &=\lim\limits_{n \to \infty}{\frac{3n+6}{5n+3}} \\
       &=\frac{3}{5}
   .\end{align*}
   \bigbreak \noindent 
   Thus, this series will converge absolutely

   \pagebreak \bigbreak \noindent 
   \begin{mdframed}
       \textbf{Problem 8.} Given the series
       \begin{align*}
           \summation{\infty}{n=1}\ \left(\frac{5n^{2}}{7n+3}\right)^{n}\ 
       .\end{align*}
       \bigbreak \noindent 
       Use Root Test to test for convergence.
   \end{mdframed}
   \bigbreak \noindent 
   \begin{align*}
       &\rho = \lim\limits_{n \to \infty}{\bigg\lvert \left(\frac{(7n^{2})^{n}}{(7n+3)^{n}}\right) \bigg\rvert^{\frac{1}{n}}} \\
       &=\lim\limits_{n \to \infty}{\frac{7n^{2}}{7n+3}} \\
       &=\lim\limits_{n \to \infty}{\frac{\frac{7n^{2}}{n}}{\frac{7n}{n} + \frac{3}{n}}} \\
       &=\lim\limits_{n \to \infty}{\frac{7n}{7+\frac{3}{n}}} \\
       &=\frac{\infty}{7}\\
       &=\infty
   .\end{align*}
   \bigbreak \noindent 
   Thus, this series will diverge.

   \bigbreak \noindent 
   \begin{mdframed}
       \textbf{Problem 9.} Given the series
       \begin{align*}
           \summation{\infty}{n=1}\ \frac{(\ln{n})^{3n}}{n^{n}}\ 
       .\end{align*}
       Use the Root Test to test for convergence.
   \end{mdframed}
   \bigbreak \noindent 
   \begin{align*}
       &\rho = \lim\limits_{n \to \infty}{\bigg\lvert \frac{(\ln{n})^{3n}}{n^{n}} \bigg\rvert^{\frac{1}{n}}} \\
       &=\lim\limits_{n \to \infty}{\frac{\ln^{3}{n}}{n}} \\
       &\Heq \lim\limits_{n \to \infty}{\frac{3\ln^{2}{n}}{n}} \\
       &\Heq \lim\limits_{n \to \infty}{\frac{6\ln{n}}{n}} \\
       &\Heq \lim\limits_{n \to \infty}\frac{6}{n}{} \\
       &=0
   .\end{align*}
   \bigbreak \noindent 
   Thus, this series will converge absolutely.

   \pagebreak \bigbreak \noindent 
   \begin{mdframed}
       \textbf{Problem 10.} Given the series
       \begin{align*}
           \summation{\infty}{n=1}\ \left(\frac{2}{e} + \frac{5}{n}\right)^{n}\ 
       .\end{align*}
       \bigbreak \noindent 
       Use Root Test to test for convergence.
   \end{mdframed}
   \bigbreak \noindent 
   \begin{align*}
       &\rho = \lim\limits_{n \to \infty}{\bigg\lvert \left(\frac{2}{e} + \frac{5}{n}\right)^{n} \bigg\rvert^{\frac{1}{n}}} \\
       &=\lim\limits_{n \to \infty}{\frac{2}{e}  + \frac{5}{n}} \\
       &=\frac{2}{e}
   .\end{align*}
   \bigbreak \noindent 
   Thus, this series will converge absolutely




    

\end{document}
