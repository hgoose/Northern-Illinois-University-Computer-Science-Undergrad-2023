\documentclass{report}

\input{~/dev/latex/template/preamble.tex}
\input{~/dev/latex/template/macros.tex}

\title{\Huge{}}
\author{\huge{Nathan Warner}}
\date{\huge{}}
\pagestyle{fancy}
\fancyhf{}
\lhead{Warner \thepage}
\rhead{}
% \lhead{\leftmark}
\cfoot{\thepage}
% \setborder
% \usepackage[default]{sourcecodepro}
% \usepackage[T1]{fontenc}

\begin{document}
    % \maketitle
    % \tableofcontents
    \pagebreak \bigbreak \noindent
    Nate Warner \ \quad \quad \quad \quad \quad \quad \quad \quad \quad \quad \quad \quad \quad \quad \quad \quad \quad  MATH 230 \quad  \quad \quad \quad \quad \quad \quad \quad \quad \ \ \quad \quad September 15, 2023
    \begin{center}
        \textbf{Homework/Worksheet 3 - Due: Wednesday, September 20}
    \end{center}
    \bigbreak \noindent 
    \begin{mdframed}
        \textbf{1.} Find the area between the curves $y=\cos{\theta}$ and $y = 0.5$, for $0 \leq \theta \leq \pi$
    \end{mdframed}
    \bigbreak \noindent 
    \begin{remark}
        $A = \int_{a}^{b}\ f(x) - g(x)\ dx \quad \text{For $f(x) \geq g(x)$}$ 
    \end{remark}
    
    \begin{minipage}[t]{0.47\textwidth}
        Intersection:
     \begin{align*}
         &\cos{\theta } = 0.5 \\
         &\cos{\theta} = \frac{1}{2} \\
         &\theta = \cos^{-1}{\frac{1}{2}} \\
         &\theta = \frac{\pi}{3}
     .\end{align*}
        $\abs{\cos{\theta } - 0.5} = $
        \begin{cases}
                 \cos{\theta} - 0.5 & \text{if } 0 \leq \theta \leq \frac{\pi}{3} \\
                 -\cos{\theta} + 0.5 & \text{if } \frac{\pi}{3} < \theta \leq \pi
            \end{cases}
    \end{minipage}
    \begin{minipage}[t]{0.47\textwidth}
        Thus:
     \begin{align*}
        &= \int_{0}^{\pi}\ \abs{\cos{\theta } - 0.5} \ d\theta  \\
        &= \int_{0}^{\frac{\pi}{3}}\ \cos{\theta } - 0.5\ d\theta  + \int_{\frac{\pi}{3}}^{\pi}\ -\cos{\theta } + 0.5\ d\theta  \\
        &\text{Where}\ I_{1} = \int_{0}^{\frac{\pi}{3}}\ \cos{\theta } - 0.5\ d\theta  \\
        &I_{2} = + \int_{\frac{\pi}{3}}^{\pi}\ -\cos{\theta } + 0.5\ d\theta  \\
        &A = I_{1} + I_{2} \\
        &I_{1} = \sin{\theta } -\frac{1}{2}\theta \bigg|^{\frac{\pi}{3}}_{0} \\
        &= \bigg(\sin{\bigg(\frac{\pi}{3}\bigg)} - \frac{1}{2}\bigg(\frac{\pi}{3}\bigg)\bigg)  - \bigg(\sin{0}\bigg) \\
        &= \bigg(\frac{\sqrt{3}}{2} - \frac{\pi}{6}\bigg)  \\
        &= \frac{3\sqrt{3} - \pi}{6} \\
        &I_{2} = \int_{\frac{\pi}{3}}^{\pi}\ -\cos{\theta } + 0.5\ d\theta  \\
        &= -\sin{\theta} + \frac{1}{2}\theta \bigg|_{\frac{\pi}{3}}^{\pi} \\
        &= \bigg(-\sin{\bigg(\pi\bigg)} + \frac{1}{2}\bigg(\pi\bigg)\bigg) - \bigg(-\sin{\bigg(\frac{\pi}{3}\bigg) + \frac{1}{2}\bigg(\frac{\pi}{3}\bigg)}\bigg) \\
        &= \frac{\pi}{2} - \bigg(-\frac{\sqrt{3}}{2}+\frac{\pi}{6}\bigg)  \\
        &= \frac{\pi}{2} - \bigg(-\frac{3\sqrt{3}-\pi}{6}\bigg) \\
        &= \frac{\pi}{2} + \frac{3\sqrt{3}-\pi}{6}\\
        &=\frac{3\sqrt{3}+2\pi}{6} \\
        &\therefore A = \frac{3\sqrt{3} -\pi}{6} +  \frac{3\sqrt{3}+2\pi}{6}\\
        &=\frac{6\sqrt{3}+\pi}{6} \\
        &=\sqrt{3}+\frac{\pi}{6}
    .\end{align*}
    \end{minipage}

    \pagebreak 
    \begin{mdframed}
        \textbf{2. Sketch the region enclosed by the given curves below and find its area.}
        \begin{enumerate}[label=(\alph*)]
            \item \( y = x^2, \quad y = -x^2 + 18x \)
            \item \( y = \cos x, \quad y = 2 - \cos x, \quad 0 \leq x \leq 2\pi \)
            \item \( y = x^3, \quad y = x^2 - 2x, \quad x = -1, \quad x = 1 \)
            \item \( x = y^2, \quad x = y + 2 \)
        \end{enumerate}
    \end{mdframed}

    \bigbreak \noindent 
    \textbf{2.a}
    \bigbreak \noindent 
    \begin{minipage}[]{0.47\textwidth}
        \incfig{fig1}
    \end{minipage}
    \begin{minipage}[]{0.47\textwidth}
    Intersection:
    \begin{align*}
        &x^{2} = -x^{2} + 18x \\
        &2x^{2} - 18x = 0  \\
        & 2x(x-9) = 0 \\
        &x=0,9
    .\end{align*}
    Thus:
    \begin{align*}
        &A = \int_{0}^{9}\ -x^{2}+18x-x^{2}\ dx \\
        &= \int_{0}^{9}\ -2x^{2}+18x\ dx \\
        &= -\frac{2}{3}x^{3} + 9x^{2} \bigg|_0^9 \\
        &= -\frac{2}{3}(9)^{3} + 9(9)^{2} \\
        &= -\frac{1458}{3} + 729 \\
        &=243
    .\end{align*}
    \end{minipage}
    \bigbreak \noindent 
    \textbf{2.b}
    \bigbreak \noindent 
    \begin{minipage}[]{0.47\textwidth}
        \incfig{fig4}
    \end{minipage}
    \begin{minipage}[]{0.47\textwidth}
    Intersection:
    \begin{align*}
        &\cos{x} = 2-\cos{x} \\
        & 2\cos{x} = 2 \\
        &\cos{x} = 1 \\
        &x= \cos^{-1}{1} \\
        &x=0,2\pi
    .\end{align*}
    \bigbreak \noindent 
    Thus:
    \begin{align*}
        &\int_{0}^{2\pi}\ 2-\cos{x} - \cos{x}\ dx \\
        &=\int_{0}^{2\pi}\ 2-2\cos{x}\ dx \\
        &=\int_{0}^{2\pi}\ 2(1-\cos{x})\ dx \\
        &=2 \left[x-\sin{x}\right]^{2\pi}_0 \\
        &= 2 [2\pi-\sin{(2\pi)}] \\
        &= 2 (2\pi) \\
        &= 4\pi
    .\end{align*}
    \end{minipage}

    \pagebreak \bigbreak \noindent 
    \textbf{2.c}
    \bigbreak \noindent 
    \begin{minipage}[]{0.47\textwidth}
        \incfig{fig12}
    \end{minipage}
    \begin{minipage}[]{0.47\textwidth}
    Thus:
    \begin{align*}
        &\int_{-1}^{0}\ x^{2}-2x - x^{3}\ dx + \int_{0}^{1}\ x^{3}-(x^{2}-2x)\ dx \\
        &\text{Where:}\ I_{1} = \int_{-1}^{0}\ x^{2} - 2x -x^{3}\ dx  \\
        &I_{2} = \int_{0}^{1}\ x^{3}-x^{2}+2x\ dx\\
        &I_{1} =  \frac{1}{3}x^{3}-x^{2}-\frac{1}{4}x^{4}\ \bigg|_{-1}^{0} \\
        &= -\bigg(\frac{1}{3}(-1)^{3}-(-1)^{2}-\frac{1}{4}(-1)^{4}\bigg) \\
        &= -\bigg(-\frac{1}{3}-1-\frac{1}{4}\bigg) \\
        &=\frac{19}{12} \\
        &I_{2} = \frac{1}{4}x^{4}-\frac{1}{3}x^{3}+x^{2}|\ \bigg|_{0}^{1} \\
        &= \frac{1}{4}(1)^{3}-\frac{1}{3}(1)^{3}+(1)^{2} \\
        &= \frac{1}{4}-\frac{1}{3}+1 \\
        &=\frac{11}{12} \\
        &\therefore A = I_{1} + I_{2} = \frac{19}{12} + \frac{11}{12} = \frac{30}{12}  \\
        &= \frac{5}{2} 
    .\end{align*}
    \end{minipage}

    \bigbreak \noindent 
    \textbf{2.d}
    \bigbreak \noindent 
    \begin{minipage}[]{0.47\textwidth}
        \incfig{fig10}
    \end{minipage}
    \begin{minipage}[]{0.47\textwidth}
        Intersection:
        \begin{align*}
            &x^{\frac{1}{2}} = x-2 \\
            &x = (x-2)^{2} \\
            &x = x^{2}-4x+4 \\
            &x^{2}-5x+4 = 0 \\
            &(x-1)(x-4) \\
            &\cancelto{nas}{x=1},\ x=4
        .\end{align*}
        \bigbreak \noindent 
        Thus:
        \begin{align*}
            &A = \int_{0}^{4}\ x^{\frac{1}{2}}-(x-2)\ dx \\
            &=\int_{0}^{4}\ x^{\frac{1}{2}}-x+2\ dx \\
            &= \frac{2}{3}x^{\frac{3}{2}}-\frac{1}{2}x^{2}+2x\ \bigg|_0^{4} \\
            &=\frac{16}{3}
        .\end{align*}
    \end{minipage}

    \pagebreak \bigbreak \noindent 
    \begin{mdframed}
        \textbf{3.} Find the volume of the pyramid below by using the slicing method
    \end{mdframed}
    \bigbreak \noindent 
    \begin{minipage}[]{0.47\textwidth}
        \incfig{mid}
    \end{minipage}
    \begin{minipage}[]{0.47\textwidth}
        We can see that the base of this object is a rectangle. Thus, the cross sections will also be rectangles. With the area of the cross section being $A = ab$. If we define a cross section with some length $a$, some width $b$, and some height $x$, we have:
    \end{minipage}
    \bigbreak \noindent 
    \begin{minipage}[]{0.47\textwidth}
    \incfig{fig15}
    \end{minipage}
    \begin{minipage}[]{0.47\textwidth}
        We can use proportion of similar triangles to find formulas for the lengths of $a$ and $b$:
        \begin{align*}
            &\frac{3}{4} = \frac{a}{x} \\
            &\frac{3}{4}x = a
        .\end{align*}
        \begin{align*}
            &\frac{2}{4} = \frac{b}{x} \\
            & \frac{2}{4}x = b
        .\end{align*}
    \end{minipage}
    \bigbreak \noindent 
    Thus we now have the formula for the area of a cross section $A(x)$, and we can use the volume equation $V = \int_{a}^{b}\ A(x)\ dx $ to find the volume of this shape.
    \begin{align*}
        &A(x) = \left(\frac{3}{4}x\right)\left(\frac{1}{2}x\right) \\
        &=\frac{3}{8}x^{2} \\
        &\implies V = \int_{0}^{4}\ \frac{3}{8}x^{2}\ dx \\
        &= \frac{3}{8}\int_{0}^{4}\ x^{2}\ dx \\
        &=\frac{3}{8}\left[\frac{1}{3}x^{3}\right]^{4}_0 \\
        &=\frac{3}{8}\left(\frac{1}{3}(4)^{3}\right) \\
        &=8 \\
        &\therefore V = 8\ \text{units}^{3}
    .\end{align*}

    \pagebreak \bigbreak \noindent 
    \begin{mdframed}
        \textbf{4.} Find the volume of the solid obtained by rotating the region bounded by the curves below about the specified line. Sketch the region, the solid, and a typical disk or washer.
        \begin{enumerate}
            \item \( y = 2x^2, \quad x = 0, \quad x = 4, \quad y = 0; \) about the \(x\)-axis
            \item \( y = 4 - x^2, \quad y = 2 - x; \) about the \(x\)-axis
            \item \( y = 1 + e^x, \quad x = 0, \quad x = 1, \quad y = 0; \) about the \(x\)-axis
            \item \( y = 2x^3, \quad x = 0, \quad x = 1, \quad y = 0; \) about the \(y\)-axis
            \item \( y = \sqrt{4 - x^2}, \quad y = 0, \quad x = 0; \) about the \(y\)-axis
            \item \( y = \sin x, \quad y = \cos x, \quad 0 \leq x \leq \frac{\pi}{4}; \) about \(y = -1\)
        \end{enumerate}
    \end{mdframed}
    \bigbreak \noindent 
    \textbf{4.1}
    \bigbreak \noindent 
    \begin{minipage}[]{0.47\textwidth}
        \incfig{fig21}
    \end{minipage}
    \begin{minipage}[]{0.47\textwidth}
        We can see from the figure that if we revolve this region around the $x-axis$, we will get a disk shaped cross section. Thus, the area of the cross section is given by $\pi(f(x))^{2}$, where $f(x)$ is the radius.
        \bigbreak \noindent 
        From this we can compute the volume:
        \begin{align*}
            &V = \int_{0}^{4}\ \pi[2x^{2}]^{2}\ dx \\
            &= \pi \int_{0}^{4}\ 4x^{4}\ dx  \\
            &= 4\pi \int_{0}^{4}\ x^{4}\ dx  \\
            &= 4\pi\bigg[\frac{1}{5}x^{5}\bigg]_{0}^{4} \\
            &= \frac{4\pi}{5}\bigg((4)^{5}\bigg) \\
            &= \frac{4096\pi}{5}
        .\end{align*}
    \end{minipage}

    \pagebreak \bigbreak \noindent 
    \textbf{4.2}
    \bigbreak \noindent 
    \begin{minipage}[]{0.47\textwidth}
     \incfig{fif30}
    \end{minipage}
    \begin{minipage}[]{0.47\textwidth}
        So we can from the figure that when revolved around the $x-axis$, we will end up with an annulus cross section. Where the area is given by $\pi r^{2}$, and the radius given by $f(x) - g(x)$, with $f(x)= 4-x^{2}$ and $g(x)=2-x$.
        \bigbreak \noindent 
        Intersection:
        \begin{align*}
            &4-x^{2} = 2-x \\
            &-x^{2}+x+2 = 0 \\
            &-(x^{2}-x-2) = 0\\
            &-(x+1)(x-2) =0 \\
            &x=-1,2
        .\end{align*}
        \begin{align*}
            &V = \int_{-1}^{2}\ \pi\left[(4-x^{2})^{2}-(2-x)^{2}\right]\ dx \\
            &=\int_{-1}^{2}\ \pi\left[x^{4}-8x^{2}+16-(x^{2}-4x+4)\right]\ dx \\
            &=\pi \int_{-1}^{2}\ x^{4}-9x^{2}+4x+12\ dx \\
            &=\pi\left[\frac{1}{5}x^{5}-3x^{3}+2x^{2}+12x\right]_{-1}^{2} \\
            &= \pi\bigg[\bigg(\frac{1}{5}(2)^{5}-3(2)^{3}+2(2)^{2}+12(2)\bigg) \\ 
            &-\bigg(\frac{1}{5}(-1)^{5}-3(-1)^{3}+2(-1)^{2}+12(-1)\bigg)\bigg] \\
            &=\frac{72}{5}+\frac{36}{5} \\
            &=\frac{108\pi}{5}
        .\end{align*}
    \end{minipage}
    \bigbreak \noindent 
    \textbf{4.c}
    \bigbreak \noindent 
    \begin{minipage}[]{0.47\textwidth}
        \incfig{fig22}
    \end{minipage}
    \begin{minipage}[]{0.47\textwidth}
        Thus:
        \begin{align*}
            &V = \int_{0}^{1}\ \pi[e^{x}+1]^{2}\ dx \\
            &=\pi \int_{0}^{1}\ e^{2x}+2e^{x}+1\ dx\\
            &=\pi\bigg[\frac{1}{2}e^{2x}+2e^{x}+x\bigg]_0^{1} \\
            &= \pi \bigg[\bigg(\frac{1}{2}e^2+2e^{1}+1\bigg)-\bigg(\frac{1}{2}e^{0}+2e^{0}\bigg)\bigg] \\
            &= \pi \bigg[\frac{e^{2}}{2}+2e+1- \bigg(\frac{1}{2}+2\bigg)\bigg] \\
            &= \pi\bigg[\frac{e^{2}+4e+2}{2}-\frac{5}{2}\bigg] \\
            &=\frac{\pi e^{2}+4\pi e -3\pi}{2}
        .\end{align*}
    
    \end{minipage}

    \pagebreak \bigbreak \noindent 
    \textbf{4.d}
    \bigbreak \noindent 
    \begin{minipage}[]{0.47\textwidth}
        \incfig{figher}
    \end{minipage}
    \begin{minipage}[]{0.47\textwidth}
    If $y=2x^{3}$, then:
    \begin{align*}
        &x=\bigg(\frac{y}{2}\bigg)^{\frac{1}{3}}
    .\end{align*}
    \begin{align*}
        \implies V &= \int_{0}^{2}\ \pi \bigg[\bigg(\frac{y}{2}\bigg)^{\frac{1}{3}}\bigg]^{2}\ dy \\
        &=\pi \int_{0}^{2}\ \bigg(\frac{y}{2}\bigg)^{\frac{2}{3}}\ dy \\
        &=\pi \int_{0}^{2}\ \frac{1}{2^{\frac{2}{3}}}\cdot y^{\frac{2}{3}}\ dy \\
        &= \frac{\pi}{4^{\frac{1}{3}}} \bigg[\frac{3}{5}y^{\frac{5}{3}}\bigg]_0^{2} \\
        &= \frac{3\pi}{5\cdot 4^{\frac{1}{3}}}(2)^{\frac{5}{3}} \\
        &= \frac{3\pi \cdot 32^{\frac{1}{3}}}{5 \cdot 4^{\frac{1}{3}}} \\
        &=\frac{3\pi}{5} \cdot \bigg(\frac{32}{4}\bigg)^{\frac{1}{3}} \\
        &=\frac{3\pi}{5} \cdot (8)^{\frac{1}{3}} \\
        &=\frac{3\pi}{5} \cdot (2) \\
        &\therefore V =\frac{6\pi}{5} 
    .\end{align*}
    \end{minipage}
    \bigbreak \noindent 
    \textbf{4.e}
    \bigbreak \noindent 
    \begin{minipage}[]{0.47\textwidth}
        \incfig{circle}
    \end{minipage}
    \begin{minipage}[]{0.47\textwidth}
        \begin{remark}
            Semi Circle with radius 2
        \end{remark}
        If $y=\sqrt{4-x^{2}}$, then:
        \begin{align*}
            x = \sqrt{4-y^{2}}
        .\end{align*}
        \bigbreak \noindent 
        Thus:
        \begin{align*}
           \implies  V &= \int_{0}^{2}\ \pi\left(\sqrt{4-y^{2}}\right)^{2}\ dy \\
           &=\pi\int_{0}^{2}\ 4-y^{2}\ dy \\
           &=\pi \left[4y-\frac{1}{3}y^{3}\right]_0^{2} \\
           &=\pi\left(8-\frac{8}{3}\right) \\
           &\therefore V = \frac{16\pi}{3}
        .\end{align*}
    \end{minipage}

    \pagebreak \bigbreak \noindent 
    \textbf{4.f}
    \bigbreak \noindent 
    \begin{minipage}[]{0.47\textwidth}
        \incfig{sinewave}
    \end{minipage}
    \begin{minipage}[]{0.47\textwidth}
    \begin{prop}
        If we rotate some region $R$ around a line that is not the $x$ or $y$ axis, then the radius of the disk is given by $ R =f(x) + k \iff$ A.O.R is $y=-k$ else if A.O.R $y=k \rightarrow$ $R=k-f(x)$
    \end{prop}
    \bigbreak \noindent 
    Thus:
    \begin{align*}
        \implies V = \int_{0}^{\frac{\pi}{4}}\ \pi\bigg[(\cos{(x) + 1})^{2} - (\sin{(x)+1})^{2}\bigg]\ dx 
    .\end{align*}
    \end{minipage}
    \begin{align*}
        &=\pi\int_{0}^{\frac{\pi}{4}}\ \cos^{2}{(x)}+2\cos{(x)}+1 -(\sin^{2}{(x)}+2\sin{(x)} +1)\ dx \\
        &= \pi \int_{0}^{\frac{\pi}{4}}\ \cos^{2}{(x)}+2\cos{(x)}+1 -\sin^{2}{(x)}-2\sin{(x)} -1\ dx  \\
        &=\pi \int_{0}^{\frac{\pi}{4}}\ \cos^{2}{(x)-\sin^{2}{(x)+2\cos{(x)}-2\sin{(x)}}}\ dx \\
        &=\pi \int_{0}^{\frac{\pi}{4}}\ \cos{(2x)}+2\cos{(x)}-2\sin{(x)}\ dx \\
        &=\pi \bigg[\int_{0}^{\frac{\pi}{4}}\ \cos{(2x)}\ dx + \int_{0}^{\frac{\pi}{4}}\ 2\cos{(x)}-2\sin{(x)}\ dx\bigg]
    .\end{align*}
    \bigbreak \noindent 
    \begin{interlude}
       Let $I_{1} = \int_{0}^{\frac{\pi}{4}}\ \cos{(2x)}\ dx $ and $I_{2}=\int_{0}^{\frac{\pi}{4}}\ 2\cos{(x)}-2\sin{(x)}\ dx$. Thus, $V$ will be given be $\pi(I_{1}+I_{2})$
    \end{interlude}
    \begin{minipage}[t]{0.47\textwidth}
        Regarding $I_{1}:$
        \begin{align*}
            &\text{Let $u=2x$} \\
            &\frac{1}{2}du = dx \\
            &u(a) = 0,\quad u(b) = \frac{\pi}{2}
        .\end{align*}
    \end{minipage}
    \begin{minipage}[t]{0.47\textwidth}
        Thus:
        \begin{align*}
            \implies I_{1} &= \frac{1}{2}\int_{0}^{\frac{\pi}{2}}\ \cos{(u)}\ du \\
            &= \frac{1}{2}\bigg[\sin{(u)}\bigg]_0^{\frac{\pi}{2}} \\
            &=\frac{1}{2}(\sin{\bigg(\frac{\pi}{2}\bigg)}) \\
            &\therefore I_{1}=\frac{1}{2}
        .\end{align*}
    \end{minipage}
    Regarding $I_{2}$:
    \begin{align*}
        I_{2} &= \int_{0}^{\frac{\pi}{4}}\ 2\cos{(x)}-2\sin{(x)}\ dx \\
      &=2\sin{(x)+2\cos{(x)}} \bigg|_0^{\frac{\pi}{4}} \\
      &=\bigg(2\sin{\bigg(\frac{\pi}{4}\bigg)} + 2\cos{\bigg(\frac{\pi}{4}\bigg)}\bigg) - \bigg(2\sin{(0)+2\cos{(0)}}\bigg)\\
      &=2\sqrt{2} -2
    .\end{align*}
    \bigbreak \noindent 
    Therefore:
    \begin{align*}
        &V = \pi \bigg(\frac{1}{2}+2\sqrt{2}-2) \\
        &=\frac{\pi}{2}+2\pi\sqrt{2}-2\pi \\
        &=\frac{-3\pi+4\pi\sqrt{2}}{2} \\
        &=-\frac{3\pi-4\pi\sqrt{2}}{2}
    .\end{align*}

    \bigbreak \noindent \bigbreak \noindent 
    \begin{mdframed}
        5. Use the method of cylindrical shells to find the volume generated by rotating the region bounded by the given curves about the $y$-axis.
        \begin{enumerate}[label=(\alph*)]
            \item $y = x^{3},\ y=0,\ x=1,\ x=2 $ 
            \item $y= x^{2},\ y=6x-2x^{2} $
        \end{enumerate}
    \end{mdframed}
    \bigbreak \noindent 
    \textbf{5.a}:
    \bigbreak \noindent 
    \begin{minipage}[]{0.47\textwidth}
        \incfig{figher2}
    \end{minipage}
    \begin{minipage}[]{0.47\textwidth}
        Thus:
        \begin{align*}
            V &= \int_{a}^{b}\ 2\pi xf(x)\ dx \\
              &=\int_{1}^{2}\ 2\pi x(x^{3})\ dx\\
              &=2\pi \int_{1}^{2}\ x^{4}\ dx \\
              &=2\pi \bigg[\frac{1}{5}x^{5}\bigg] \\
              &=\frac{2\pi}{5}\bigg[2^{5}-1^{5}\bigg] \\
              &=\frac{2\pi}{5}(31) \\
              &= \frac{62\pi}{5}
        .\end{align*}
    
    \end{minipage}
    \pagebreak \bigbreak \noindent 
    \textbf{5.b}
    \bigbreak \noindent 
    \begin{minipage}[]{0.47\textwidth}
        \incfig{figher3}
    \end{minipage}
    \begin{minipage}[]{0.47\textwidth}
        Intersection:
        \begin{align*}
            &x^{2} = -2x^{2} +6x \\ 
            &3x^{2}-6x = 0 \\
            &3x(x-2) =0 \\
            &x=0,2
        .\end{align*}
        Thus:
        \begin{align*}
            V &= \int_{0}^{2}\ 2\pi x\bigg[-2x^{2}+6x - x^{2}\bigg]\ dx  \\
              &=2\pi\int_{0}^{2}\ x(-3x^{2}+6x)\ dx \\
              &=2\pi\int_{0}^{2}\ -3x^{3}+6x^{2}\ dx  \\
              &=2\pi\int_{0}^{2}\ -3(x^{3}-2x^{2})\ dx \\
              &=-6\pi\int_{0}^{2}\ x^{3}-2x^{2}\ dx \\
              &=-6\pi \bigg[\frac{1}{4}x^{4}-\frac{2}{3}x^{3}\bigg]_0^{2} \\
              &=-6\pi\bigg[4-\frac{16}{3}\bigg] \\
              &=-6\pi(-\frac{4}{3}) \\
              &= \frac{24\pi}{3} \\
              &= 8\pi
        .\end{align*}
    
    \end{minipage}






    




    
    

\end{document}
