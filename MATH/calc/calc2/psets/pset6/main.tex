\documentclass{report}

\input{~/dev/latex/template/preamble.tex}
\input{~/dev/latex/template/macros.tex}

\title{\Huge{}}
\author{\huge{Nathan Warner}}
\date{\huge{}}
\pagestyle{fancy}
\fancyhf{}
\lhead{Warner \thepage}
\rhead{}
% \lhead{\leftmark}
\cfoot{\thepage}
% \setborder
% \usepackage[default]{sourcecodepro}
% \usepackage[T1]{fontenc}

\begin{document}
    % \maketitle
    % \tableofcontents
    \pagebreak \bigbreak \noindent
    Nate Warner \ \quad \quad \quad \quad \quad \quad \quad \quad \quad \quad \quad \quad \quad \quad \quad \quad \quad  MATH 230 \quad  \quad \quad \quad \quad \quad \quad \quad \quad \ \ \quad \quad October 14, 2023
    \begin{center}
        \textbf{Homework/Worksheet 6 - Due: Wednesday, October 18}
    \end{center}
    \bigbreak \noindent 
    \begin{mdframed}
        1. Evaluate the following integrals using trigonometric substitution
        \begin{enumerate}[label=(\alph*)]
            \item $\int \frac{x^{2}}{\sqrt{1-x^{2}}}\ dx$ 
            \item $\int \sqrt{x^{2}+9}\ dx $ 
            \item $\int \frac{\sqrt{x^{2}-25}}{x}\ dx $
            \item $\int \frac{1}{(x^{2}-9)^{\frac{3}{2}}}\ dx $
            \item $\int \frac{x^{2}}{\sqrt{x^{2}+4}}\ dx$
            \item $\int_{-1}^{1}\ (1-x^{2})^{\frac{3}{2}}\ dx $ 
        \end{enumerate}
    \end{mdframed}

    \bigbreak \noindent 
    1.a 
    \bigbreak \noindent 
    \begin{minipage}[t]{0.47\textwidth}
        Trig sub:
     \begin{align*}
         &x = \sin{\theta } \\        
        &dx = \cos{\theta }\ d\theta 
    .\end{align*}
    \end{minipage}
    \begin{minipage}[t]{0.47\textwidth}
        Thus we have:
        \begin{align*}
            &\int \frac{\sin^{2}{\theta}\cos{\theta}}{\sqrt{1-\sin^{2}{\theta }}}\ d\theta  \\
            &=\int \sin^{2}{\theta }\ d\theta 
        .\end{align*}
    \end{minipage}
    \bigbreak \noindent 
    \begin{interlude}
       By double angle formulas, we can solve for $\sin^{2}{\theta }$ to get an easier integrand 
       \begin{align*}
           &\cos{2\theta } = 1-2\sin^{2}{\theta } \\
            &\sin^{2}{\theta } = \frac{1}{2}-\frac{1}{2}\cos{2\theta }
       .\end{align*}
    \end{interlude}
    \bigbreak \noindent 
    Following this, we have:
    \begin{align*}
        &\int \frac{1}{2}-\frac{1}{2}\cos{2\theta }\ d\theta  \\
        &=\int \frac{1}{2}d\theta - \frac{1}{2}\int \cos{2\theta }\ d\theta \\
        &= \frac{1}{2}\theta -\frac{1}{4}\sin{2\theta } + C
    .\end{align*}
    \bigbreak \noindent 
    \begin{minipage}[t]{0.47\textwidth}
        Reference triangle:
        \bigbreak \noindent 
        \incfig{ref}
    \end{minipage}
    \begin{minipage}[t]{0.4\textwidth}
        By the reference triangle we have:
        \begin{align*}
            &\int \frac{x^{2}}{\sqrt{1-x^{2}}} = \frac{1}{2}\sin^{-1}{(x)} -\frac{1}{4}\sin{(2\sin^{-1}{(x)}}) + C
        .\end{align*}
    \end{minipage}

    \pagebreak \bigbreak \noindent 
    1.b
    \bigbreak \noindent 
    \begin{minipage}[t]{0.47\textwidth}
        \begin{align*}
            &x = 3\tan{\theta } \\
            &dx = 3\sec^{2}{\theta}\ d\theta 
        .\end{align*} 
    \end{minipage}
    \begin{minipage}[t]{0.47\textwidth}
        Thus, we have:
    \begin{align*}
        &\int \sqrt{x^{2} + 9}\ dx  \\
        &= \int \sqrt{9\tan^{2}{\theta } + 9}\ 3\sec^{2}{\theta }\ d\theta  \\
        &=9 \int \sqrt{\tan^{2}{\theta }+1}\sec^{2}{\theta }\ d\theta  \\
        &=9\int \sec^{3}{\theta }\ d\theta  \\
        &= 9\bigg[\frac{1}{2}\sec{\theta }\tan{\theta } + \frac{1}{2}\int \sec{\theta }\ d\theta \bigg] \\
        &= 9\bigg[\frac{1}{2}\sec{\theta }\tan{\theta }+\frac{1}{2}\ln{\abs{\sec{\theta }  +\tan{\theta }}}\bigg] + C \\
        &= \frac{9}{2}\sec{\theta }\tan{\theta } + \frac{9}{2} \ln{\abs{\sec{\theta } + \tan{\theta }}}  + C \\
    .\end{align*}
    \end{minipage}
    \bigbreak \noindent 
    \begin{minipage}[]{0.47\textwidth}
        Reference Triangle:
        \bigbreak \noindent 
        \incfig{ref6}
    \end{minipage}
    \begin{minipage}[]{0.47\textwidth}
        \begin{align*}
            &\therefore \int \sqrt{x^{2} + 9}\ dx = \frac{9}{2} \cdot \frac{1}{3}\sqrt{x^{2}+9}\frac{1}{3}x +\frac{9}{2}\ln{\abs{\frac{1}{3}\sqrt{x^{2}+9} + \frac{1}{3}x}} + C \\
            &=\frac{3}{2}\sqrt{x^{2}+9}\ \frac{1}{3}x +\frac{9}{2}\ln{\abs{\frac{1}{3}\sqrt{x^{2}+9} + \frac{1}{3}x}} + C \\
        .\end{align*}
    \end{minipage}

    \bigbreak \noindent 
    1.c
    \bigbreak \noindent 
    \begin{minipage}[t]{0.47\textwidth}
       \begin{align*}
            &x= 5\sec{\theta } \\     
            &dx = 5\sec{\theta }\tan{\theta }\ d\theta 
       .\end{align*} 
    \end{minipage}
    \begin{minipage}[t]{0.47\textwidth}
        Thus we have:
        \begin{align*}
            &\int \frac{\sqrt{x^{2} -25}}{x}\ dx = \int \frac{\sqrt{25\sec^{2}{\theta } - 25}}{5\sec{\theta }}\ 5\sec{\theta }\tan{\theta }\ d\theta  \\
            &=5\int \frac{\tan{\theta }}{\sec{\theta }}\sec{\theta }\tan{\theta }\ d\theta  \\
            &=5\int \tan^{2}{\theta }\ d\theta  \\
            &=5\int \sec^{2}{\theta } -1\ d\theta  \\
            &=5\tan{\theta } - 5\theta  + C
        .\end{align*}
    
    \end{minipage}

    \pagebreak \bigbreak \noindent 
    \begin{minipage}[]{0.47\textwidth}
        Reference Triangle
        \bigbreak \noindent 
        \incfig{ref7}
    \end{minipage}
    \begin{minipage}[]{0.47\textwidth}
        \begin{align*}
            &\therefore \int \frac{\sqrt{x^{2}-25}}{x}\ dx = 5\left(\frac{\sqrt{x^{2}-25}}{5}\right)-5\sec^{-1}{\left(\frac{1}{5}\right)x} + C \\
            &= \sqrt{(x-5)(x+5)}-5\sec^{-1}{\left(\frac{1}{5}x\right)} + C
        .\end{align*}
    \end{minipage}

    \bigbreak \noindent 
    1.d
    \bigbreak \noindent 
    \begin{minipage}[t]{0.47\textwidth}
        \begin{align*}
            &x = 3\sec{\theta} \\
            &dx = 3\sec{\theta}\tan{\theta}\ d\theta 
        .\end{align*}
    \end{minipage}
    \begin{minipage}[t]{0.47\textwidth}
        Thus we have:
        \begin{align*}
            &\int \frac{1}{(x^{2}-9)^{\frac{3}{2}}}\ dx = \int \frac{3\sec{\theta}\tan{\theta}}{\sqrt{(9\sec^{2}{\theta}-9)^{3}}} \\
            &=\int \frac{3\sec{\theta}\tan{\theta}}{\sqrt{729(\sec^{2}{\theta }-1)^{3}}}\ d\theta  \\
            &=\int \frac{3\sec{\theta }\tan{\theta }}{\sqrt{729\tan^{6}{\theta }}}\ d\theta  \\
            &=\int \frac{3\sec{\theta }\tan{\theta }}{27\tan^{3}{\theta }} \\
            &=\frac{1}{9}\int \frac{\sec{\theta }}{\tan^{2}{\theta }}\ d\theta \\
            &=\frac{1}{9}\int \frac{\frac{1}{\cos{\theta }}}{\frac{\sin^{2}{\theta }}{\cos^{2}{\theta }}}\ d\theta \\
            &=\frac{1}{9}\int \frac{\cos{\theta }}{\sin^{2}{\theta }}\ d\theta  \\
            &=\frac{1}{9}\int u^{-2}\ du \\
            &=-\frac{1}{9u} + C \\
            &=-\frac{1}{9\sin{\theta }} + C \\
            &=-\frac{1}{9}\csc{\theta } + C
        .\end{align*}
    \end{minipage}
    \bigbreak \noindent 
    \begin{minipage}[]{0.47\textwidth}
        Reference Triangle:
        \bigbreak \noindent 
        \incfig{ref10}
    \end{minipage}
    \begin{minipage}[]{0.47\textwidth}
        \begin{align*}
            &\therefore \int \frac{1}{(x^{2}-9)^{\frac{3}{2}}}\ dx = -\frac{x}{9\sqrt{x^{2}-9}} + C\\
            &=-\frac{x\sqrt{(x-3)(x+3)}}{9(x-3)(x+3)} + C
        .\end{align*}
    \end{minipage}

    \bigbreak \noindent 
    1.e
    \bigbreak \noindent 
    \begin{minipage}[t]{0.47\textwidth}
        \begin{align*}
            &x=2\tan{\theta } \\
            &dx = 2\sec^{2}{\theta }\ d\theta 
        .\end{align*} 
    \end{minipage}
    \begin{minipage}[t]{0.47\textwidth}
        Thus we have:
        \begin{align*}
            &\int \frac{x^{2}}{\sqrt{x^{2}+4}}\ dx = \int \frac{4\tan^{2}{\theta }}{2\sec{\theta }} \cdot 2\sec^{2}{\theta }\ d\theta  \\
            &=4\int \tan^{2}{\theta }\sec{\theta }\ d\theta  \\
            &=4\int (\sec^{2}{\theta }-1)\sec{\theta }\ d\theta  \\
            &=4\int \sec^{3}{\theta }-\sec{\theta }\ d\theta  \\
            &=4\bigg[-\int \sec{\theta }\ d\theta  + \int \sec^{3}{\theta }\ d\theta \bigg] \\
            &=4\bigg[-\ln{\abs{\sec{\theta } + \tan{\theta }}} + \frac{1}{2}\sec{\theta }\tan{\theta }+\frac{1}{2}\int \sec{\theta }\bigg] \\
            &=4\bigg[-\ln{\abs{\sec{\theta } + \tan{\theta }}} + \frac{1}{2}\sec{\theta }\tan{\theta }+\frac{1}{2}\ln{\abs{\sec{\theta } + \tan{\theta }}}\bigg] + C \\
            &=4\bigg[-\frac{1}{2}\ln{\abs{\sec{\theta } + \tan{\theta }}} + \frac{1}{2}\sec{\theta }\tan{\theta }\bigg] + C \\
            &=-2\ln{\abs{\sec{\theta} + \tan{\theta}}} + 2\sec{\theta}\tan{\theta} + C \\
        .\end{align*}
    \end{minipage}
    \bigbreak \noindent 
    \begin{minipage}[]{0.47\textwidth}
        Reference Triangle:
        \bigbreak \noindent 
        \incfig{ref13}
    \end{minipage}
    \begin{minipage}[]{0.47\textwidth}
        \begin{align*}
            &-2\ln{\abs{\frac{1}{2}\sqrt{x^{2}+4} +\frac{1}{2}x}} + \frac{1}{2}\sqrt{x^{2}+4}\ x + C \\
            &=-2\ln{\left(\frac{1}{2}\abs{\sqrt{x^{2}+4} + x}\right)} + \frac{1}{2}\sqrt{x^{2}+4}\ x + C \\
            &=\frac{1}{2}\left(-4\ln{\left(\frac{1}{2}\abs{\sqrt{x^{2}+4}+x}\right)}+\sqrt{x^{2}+4}\ x\right) + C
        .\end{align*}
    
    \end{minipage}

    \pagebreak \bigbreak \noindent 
    1.f
    \bigbreak \noindent 
    \begin{minipage}[t]{0.47\textwidth}
        \begin{align*}
            &x= \sin{\theta } \\
            &dx = \cos{\theta }\ d\theta 
        .\end{align*} 
    \end{minipage}
    \begin{minipage}[t]{0.47\textwidth}
        Thus: 
        \begin{align*}
           &\int_{-1}^{1}\ (1-x^{2})^{\frac{3}{2}}\ dx = \int_{-1}^{1}\ \sqrt{(1-\sin^{2}{\theta })^{3}}\ \cos{\theta }\ d\theta  \\
           &\int_{-1}^{1}\ \sqrt{\cos^{6}{\theta }}\ \cos{\theta }\ d\theta \\
           &=\int_{-1}^{1}\ \cos^{4}{\theta }\ d\theta  \\
           &=\frac{1}{4}\cos^{3}{\theta }\sin{\theta} + \frac{3}{4}\int \cos^{2}{\theta }\ d\theta  \\
           &= \frac{1}{4}\cos^{3}{\theta }\sin{\theta }+\frac{3}{4}\bigg[\frac{1}{2}\cos{\theta }\sin{\theta } + \frac{1}{2}\int d\theta \bigg] \\
           &= \frac{1}{4}\cos^{3}{\theta }\sin{\theta }+\frac{3}{8}\cos{\theta }\sin{\theta }+\frac{3}{8}\theta  + C
           % &= \frac{1}{4}\cos^{3}{\theta }\sin{\theta } + \frac{3}{4}\sin{\theta} + C
        .\end{align*}
    \end{minipage}
    \bigbreak \noindent 
    \begin{minipage}[t]{0.4\textwidth}
        Reference Triangle: 
        \bigbreak \noindent 
        \incfig{ref15}
    \end{minipage}
    \begin{minipage}[t]{0.47\textwidth}
        Consequently...
        \begin{align*}
           &2\int_{0}^{1}\ (1-x^{2})^{\frac{3}{2}}\ dx =  2\bigg(\frac{1}{4}(1-x^{2})^{\frac{3}{2}}x + \frac{3}{8}(1-x^{2})^{\frac{1}{2}}x +\frac{3}{8}\sin^{-1}{x}\bigg) \bigg|_0^{1}  \\
           &=2\left(\frac{3}{8}\sin^{-1}{1}\right) \\
           &=2\bigg(\frac{3\pi}{16}\bigg) \\
           &= \frac{3\pi}{8}
        .\end{align*}
    \end{minipage}

    \pagebreak 
    \begin{mdframed}
        Evaluate the following integrals using partial fractions
        \begin{enumerate}[label=(\alph*)]
            \item $\int \frac{dx}{x^{2}-5x+6}$ 
            \item $\int \frac{dx}{x(x-1)(x-2)(x-3)}$ 
            \item $\int \frac{2}{(x+2)^{2}(2-x)}\ dx$
            \item $\int \frac{x^{3}+6x^{2} +3x  +6 }{ x^{3} + 2x^{2}}\ dx$
            \item $\int \frac{2}{ (x-4)( x^{2}+2x+6)}\ dx$
            \item $\int \frac{\sin{x}}{1-\cos^{2}{x}}\ dx$
        \end{enumerate}
    \end{mdframed}

    \bigbreak \noindent 
    2.a
    \bigbreak \noindent 
    \begin{align*}
        &\int \frac{dx}{x^{2}-5x+6}  \\
        &\frac{1}{(x-2)(x-3)} = \frac{A}{(x-2)}  + \frac{B}{(x-3)} \\
        & 1= A(x-3) + B(x-2) \\
        &A = -1 \quad B = 1
    .\end{align*}
    \bigbreak \noindent 
    Thus we have:
    \begin{align*}
        &\int \frac{-1}{(x-2)} + \int \frac{1}{(x-3)} \\
        &=-\ln{\abs{x-2}} + \ln{\abs{x-3}} + C
    .\end{align*}

    \bigbreak \noindent 
    2.b
    \bigbreak \noindent 
    \begin{align*}
        &\int \frac{dx}{x(x-1)(x-2)(x-3)} \\
        &\frac{1}{x(x-1)(x-2)(x-3)} = \frac{A}{x} + \frac{B}{x-1} + \frac{C}{x-2} + \frac{D}{x-3} \\
        &1=A(x-1)(x-2)(x-3) + Bx(x-2)(x-3) + Cx(x-1)(x-3) + Dx(x-1)(x-2) \\
        &A = -\frac{1}{6},\ B=\frac{1}{2},\ C=-\frac{1}{2},\ D=\frac{1}{6}\quad \text{(By plugging in zeros)}
    .\end{align*}
    \bigbreak \noindent 
    So we have:
    \begin{align*}
       &\int -\frac{1}{6x}  +\frac{1}{2(x-1)} -\frac{1}{2(x-2)} + \frac{1}{6(x-3)}\ dx \\
       &=-\frac{1}{6}\int \frac{1}{x}\ dx + \frac{1}{2}\int \frac{1}{(x-1)}\ dx - \frac{1}{2}\int  \frac{1}{(x-2)}\ dx + \frac{1}{6}\int \frac{1}{(x-3)}\ dx \\
       &=-\frac{1}{6}\ln{\abs{x}} +\frac{1}{2}\ln{\abs{x-1}} -\frac{1}{2}\ln{\abs{x-2}} +\frac{1}{6}\ln{\abs{x-3}} + C
    .\end{align*}

    \pagebreak \bigbreak \noindent 
    2.c
    \bigbreak \noindent 
    \begin{align*}
        &\int \frac{2}{(x+2)^{2}(2-x)}\ dx \\
        &\frac{2}{(x+2)^{2}(2-x)} = \frac{A}{(x+2)} + \frac{B}{(x+2)^{2}} + \frac{C}{(2-x)} \\
        &2 = A(x+2)(2-x) + B(2-x) + C(x+2)^{2} \\
        &B = \frac{1}{2},\ C=\frac{1}{8}\quad \text{(Plugging in zeros)} \\
        &2=-Ax^{2}+4A+2B-Bx+Cx^{2}+4Cx +4C \\
        &2= (-A +C)x^{2} + (C-B)x + (4A + B + 4C) \\
        &-A+C=0 \\
        &A = \frac{1}{8} \\
        &Thus:\ A = \frac{1}{8},\ B=\frac{1}{2},\ C=\frac{1}{8}
    .\end{align*}
    \bigbreak \noindent 
    So we have the integral: 
    \begin{align*}
        &\int -\frac{1}{8(x+2)} + \frac{1}{2(x+2)^{2}} + \frac{1}{8(2-x)}\ dx \\
        &\frac{1}{8}\int \frac{1}{(x+2)}\ dx + \frac{1}{2}\int \frac{1}{(x+2)^{2}}\ dx + \frac{1}{8}\int \frac{1}{(2-x)}\ dx \\
        &=\frac{1}{8}\ln{\abs{x+2}}  -\frac{1}{2(x-2)} + \frac{1}{8}\ln{\abs{2-x}} + C
    .\end{align*}

    \bigbreak \noindent 
    1.d
    \bigbreak \noindent 
    \begin{align*}
        &\int \frac{x^{3} +6x^{2}+3x+6}{x^{3}+2x^{2}}\ dx \\
        &\int\ dx + \int \frac{4x^{2}+3x+6}{x^{3}+2x^{2}}\ dx \quad \text{(After long division)} \\
        &=x + \int \frac{4x^{2}+3x+6}{x^{3}+2x^{2}}\ dx  \\
        &=x + \int \frac{4x^{2}+3x+6}{x^{2}(x+2)}\ dx  \\ 
        &\frac{4x^{2}+3x+6}{x^{2}(x+2)} = \frac{A}{x} + \frac{B}{x^{2}} + \frac{C}{(x+2)} \quad \text{(Omitting evaluation of first integral for now...)}\\
        &4x^{2} +3x+6 = Ax(x+2) + B(x+2) + Cx^{2} \\
        &4x^{2} +3x +6 = Ax^{2}+2Ax +Bx+2B+Cx^{2} \\ 
        &(A+C)x^{2}  +(2A +B)x + 2B \\
        &A+C = 4 \quad \text{(Begin system...)} \\
        &2A+B = 3 \\
        &2B = 6 \quad \text{(End system...)} \\
        &A=0,\ B = 3,\ C=4
    .\end{align*}

    \pagebreak \bigbreak \noindent 
    Thus we have the integral:
    \begin{align*}
        &\int \frac{3}{x^{2}}\ dx + \frac{4}{(x+2)}\ dx \\
        &=-\frac{3}{x} +4\ln{\abs{x+2}}
    .\end{align*}
    \bigbreak \noindent 
    Bringing back the first integral that we evaluated we get the full solution of:
    \begin{align*}
       x-\frac{3}{x}+4\ln{\abs{x+2}}  + C
    .\end{align*}
    
    \bigbreak \noindent 
    1.e
    \bigbreak \noindent 
    \begin{align*}
        &\int \frac{2}{(x-4)(x^{2}+2x+6)}\ dx \\
        &\frac{2}{(x-4)(x^{2}+2x+6)} = \frac{A}{(x-4)} + \frac{Bx+C}{(x^{2}+2x+6)} \\
        &2 = A(x^{2}+2x+6) + (Bx+C)(x-4) \\
        &A=\frac{1}{15}\quad \text{(Plugging in 4)} \\
        &2= Ax^{2}+2Ax+6A +Bx^{2}-4Bx-4C  \\
        &2 =(A+B)x^{2}+(2A-4B+C)x+(6A-4C)  \\
        &A+B = 0 \quad \text{(Begin system...)} \\
        &2A -4B + C = 0 \\
        & 6A-4C = 2 \quad \text{(End system...)} \\
        &B = -\frac{1}{15} \\
        &C = -\frac{1}{5} \\
        &Thus:\ A = \frac{1}{15},\ B=-\frac{1}{15},\ C=-\frac{1}{5}
    .\end{align*}
    \bigbreak \noindent 
    By this  we have the integral:
    \begin{align*}
        &\int \frac{1}{15(x-4)} + \frac{\left(-\frac{1}{15}\right)x + \left(-\frac{1}{5}\right)}{x^{2}+2x+6}\ dx \\
        &=\int \frac{1}{15(x-4)}\ dx + \int \frac{-x-3}{15(x^{2}+2x+6)}\ dx \\
        &=\int \frac{1}{15(x-4)}\ dx - \int \frac{x+3}{15(x^{2}+2x+6)}\ dx \\
        &I_{1} = \frac{1}{15}\int \frac{1}{x-4}\ dx \\
        &=\frac{1}{15}\ln{\abs{x-4}} \\
        &I_{2} = -\frac{1}{15}\int \frac{x}{x^{2}+2x+6} + \frac{3}{x^{2}+2x+6}\ dx \\
        &=-\frac{1}{15}\int \frac{x}{(x+1)^{2}+5} + \frac{3}{(x+1)^{2}+5} \quad \text{(By completing the square)} \\
        &= -\frac{1}{15} \int \frac{u-1}{u^{2}+5} + \frac{3}{u^{2}+5}\ du \quad \text{(Where $u=x+1$)} \\
        &=-\frac{1}{15}\int \frac{u}{u^{2}+5} - \frac{1}{u^{2}+(\sqrt{5})^{2}} + \frac{3}{u^{2}+(\sqrt{5})^{2}}\ du \\
        &= -\frac{1}{15}\bigg[\frac{1}{2}\ln{\abs{u^{2}+5}}-\frac{1}{\sqrt{5}}\tan^{-1}{\frac{u}{\sqrt{5}}}+\frac{3}{\sqrt{5}}\tan^{-1}{\frac{u}{\sqrt{5}}}\bigg] \\
        &= -\frac{1}{15}\bigg[\frac{1}{2}\ln{\abs{(x+1)^{2}+5}}-\frac{1}{\sqrt{5}}\tan^{-1}{\frac{x+1}{\sqrt{5}}}+\frac{3}{\sqrt{5}}\tan^{-1}{\frac{x+1}{\sqrt{5}}}\bigg] + C \\
        &= -\frac{1}{15}\bigg[\frac{1}{2}\ln{\abs{(x+1)^{2}+5}}+\frac{2}{\sqrt{5}}\tan^{-1}{\frac{x+1}{\sqrt{5}}}\bigg] + C \\
        &\therefore \int \frac{2}{(x-4)(x^{2}+2x+6)}\ dx \\
        &=\frac{1}{15}\ln{\abs{x-4}}-\frac{1}{15}\bigg[\frac{1}{2}\ln{\abs{(x+1)^{2}+5}}+\frac{2}{\sqrt{5}}\tan^{-1}{\frac{x+1}{\sqrt{5}}}\bigg] + C \\
        &=\frac{1}{15}\ln{\abs{x-4}}-\frac{1}{30}\ln{\abs{(x+1)^{2}+5}}-\frac{2}{15\sqrt{5}}\tan^{-1}{\frac{x+1}{\sqrt{5}}} + C
    .\end{align*}

    \bigbreak \noindent 
    2.f
    \bigbreak \noindent 
    \begin{align*}
        &\int \frac{\sin{x}}{1-\cos^{2}{x}}\ dx \\
        &=-\int \frac{du}{1-u^{2}} \quad \text{(Let $u=\cos{x}$)} \\
        &=-\int \frac{du}{(1-u)(1+u)}\ du \\
        &\frac{1}{(1-u)(1+u)} = \frac{A}{(1-u)} + \frac{B}{(1+u)} \\
        &1 = A(1+u) + B(1-u) \\
        &A = \frac{1}{2},\ B=\frac{1}{2} \quad \text{(By zeros)}
    .\end{align*}
    \bigbreak \noindent 
    Thus:
    \begin{align*}
     &-\int \frac{1}{2(1-u)} + \frac{1}{2(1+u)}\ du \\
     &-\bigg[-\frac{1}{2}\ln{\abs{1-u}} +\frac{1}{2}\ln{\abs{1+u}}\bigg] + C \\
     &=\frac{1}{2}\ln{\abs{1-\cos{x}}} - \frac{1}{2}\ln{\abs{1+\cos{x}}} + C
    .\end{align*}



    
    

\end{document}
