\documentclass{report}

\input{~/dev/latex/template/preamble.tex}
\input{~/dev/latex/template/macros.tex}

\title{\Huge{}}
\author{\huge{Nathan Warner}}
\date{\huge{}}
\pagestyle{fancy}
\fancyhf{}
\lhead{Warner \thepage}
\rhead{}
% \lhead{\leftmark}
\cfoot{\thepage}
% \setborder
% \usepackage[default]{sourcecodepro}
% \usepackage[T1]{fontenc}

\begin{document}
    % \maketitle
    % \tableofcontents
    \pagebreak \bigbreak \noindent
    Nate Warner \ \quad \quad \quad \quad \quad \quad \quad \quad \quad \quad \quad \quad \quad \quad \quad \quad \quad  MATH 230 \quad  \quad \quad \quad \quad \quad \quad \quad \quad \ \ \quad \quad November 11, 2023
    \begin{center}
        \textbf{Homework/Worksheet 9 - Due: Wednesday, November 15}
    \end{center}
    \bigbreak \noindent 
    \begin{mdframed}
        1. Use the comparison test to determine whether the series is convergent or divergent
        \begin{enumerate}[label=(\alph*)]
            \item $\summation{\infty}{n=1}\frac{1}{2n-1}$ 
            \item $\summation{\infty}{n=1}\frac{\sin^{2}{n}}{n^{2}}$
            \item $\summation{\infty}{n=1}\frac{1}{\sqrt{n^{2}+1}}$
        \end{enumerate}
    \end{mdframed}

    \bigbreak \noindent 
    \begin{remark}
        Suppose we have two series $a_{n}, b_{n}$ and $\te[]N \in \mathbb{Z^{+}}$ s.t $0 \leq a_{n} \leq b_{n}$ $\fa n \geq N$. If $b_{n}$ converges then $a_{n}$ will also converge. Conversely, if $a_{n} \geq b_{n} \geq 0$ $\fa n \geq N $, and $b_{n}$ diverges, then $a_{n}$ will also diverge
    \end{remark}
    

    \bigbreak \noindent 
    \textbf{Problem 1.a}. If we let $b_{n}$ be the series $\summation{\infty}{n=1}\ \frac{1}{2n}\ $. We may conject that this series will diverge since it is know that the harmonic series $\summation{}{}\ \frac{1}{n}\  $ diverges, and multiplying a series by a constant factor will not affect the convergence or divergence. Furthermore, 
    \begin{align*}
        \frac{1}{2n-1} > \frac{1}{2n}
    .\end{align*}
    \bigbreak \noindent 
    \textbf{Conclusion.} Thus, since $\summation{\infty}{n=1}\ \frac{1}{2n}\  $ diverges, we can conclude that $\summation{\infty}{n=1}\ \frac{1}{2n-1}\  $ will also diverge

    \bigbreak \noindent 
    \textbf{Problem 1.b} Let $b_{n}$ be the series $\summation{\infty}{n=1}\frac{1}{n^{2}}$ Since we know the sine function produces outputs in the range [-1,1], the sine function squared will produce outputs withing the range [0,1]. However, since we are only considering integer values $[1,\infty)$, $\sin^2{n}$ will only produce outputs $(0,1)$. This is because the sine function has outputs of 1 at $\frac{\pi}{2}  + 2k\pi,\ k \in \mathbb{Z}$, and outputs of 0 at $k\pi,\ k \in \mathbb{Z}$, 
    \bigbreak \noindent 
    \textbf{Problem 1.b:} Let \( b_n = \sum_{n=1}^{\infty} \frac{1}{n^2} \). We know The function \( \sin^2(x) \) yields values in the range [0,1], as \( \sin(x) \) varies between -1 and 1. For integer values \( n \) in the range [1, \infty), \( \sin^2(n) \) will produce values in the interval (0,1). This is because \( \sin(x) \) equals 1 at \( \frac{\pi}{2} + 2k\pi \) and 0 at \( k\pi \), where \( k \) is an integer, and these points are not integers. Thus we can conclude 
    \begin{align*}
        \frac{\sin^{2}{n}}{n^{2}} < \frac{1}{n^{2}}
    .\end{align*}
    \bigbreak \noindent 
    \textbf{Conclusion.} Since we know by the p-series $\frac{1}{n^{2}}$ will converge, $\frac{\sin^{2}{n}}{n^{2}}$ will also converge

    \pagebreak \bigbreak \noindent 
    \textbf{Problem 1.c} Let  $b_{n}$ be the series $\summation{\infty}{n=1}\ \frac{1}{n+1}\ $. We know this series will diverge because it is just the harmonic series $\frac{1}{n}$ shifted over by 1. We can deduce that $\frac{1}{\sqrt{n^{2} + 1}}  > \frac{1}{n+1}$ by looking at their reciprocals
    \begin{align*}
        &\sqrt{n^{2} + 1} < n+1 \\
        &n^{2} + 1 < (n+1)^{2} \\
        &n^{2}  + 1 < n^{2} + 2n +1 \\
        &n^{2} < n^{2} + 2n
    .\end{align*}
    \bigbreak \noindent 
    \textbf{Conclusion.} Since this is clearly a true statement, then by the reciprocal identify for inequalities, which states if $0 \leq a \leq b $, then $\frac{1}{a} \geq \frac{1}{b}$ it holds that $\frac{1}{\sqrt{n^{2} + 1}} > \frac{1}{n+1}$. and since we know $\summation{\infty}{n=1}\ \frac{1}{n+1} \  $ diverges, then $\summation{\infty}{n=1}\ \frac{1}{\sqrt{n^{2}+1}}\  $ will also diverge.
    
    \bigbreak \noindent 
    \begin{mdframed}
        2. Use the Limit Comparison Test to determine whether the series is convergent or divergent.
        \begin{enumerate}[label=(\alph*)]
            \item $\summation{\infty}{n=1}\ \ln{\left(1+\frac{1}{n^{2}}\right)}\  $
            \item $\summation{\infty}{n=1}\ \frac{1}{4^{n} - 3^{n}}\  $
            \item $\summation{\infty}{n=1}\ (1-\cos{\frac{1}{n}})\  $
        \end{enumerate}
    \end{mdframed}
    \bigbreak \noindent 
    \begin{remark}
       Suppose we have two series $a_{n},\ b_{n}$ where $a_{n},\ b_{n} \geq 0 \fa n \geq 1$. Then if
       \begin{itemize}
           \item $\lim\limits_{n \to \infty}{\frac{a_{n}}{b_{n}}} = L \ne 0 \text{ or } +\infty$. Then $a_{n}$ and $b_{n}$ either both converge or both diverge
            \item  $\lim\limits_{n \to \infty}{\frac{a_{n}}{b_{n}}} = 0$, then if $b_{n}$ converges, so does $a_{n}$
            \item  $\lim\limits_{n \to \infty}{\frac{a_{n}}{b_{n}}} = +\infty$, then if $b_{n}$ diverges, so does $a_{n}$
       \end{itemize}
    \end{remark}
    \bigbreak \noindent 
    \textbf{Problem 2.a} Let $b_{n}  = \frac{1}{n^{2}}$, which by the p-series, converges
    \begin{align*}
        &\lim\limits_{n \to \infty}{\frac{a_{n}}{b_{n}}} = \frac{\ln{\left(1+\frac{1}{n^{2}}\right)}}{\frac{1}{n^{2}}} \\
        &=\lim\limits_{n \to \infty}{\frac{\ln{\left(1+\frac{1}{n^{2}}\right)}}{n^{-2}}} \\
        &\Heq\lim\limits_{n \to \infty}{\frac{\frac{1}{1+\frac{1}{n^{2}}} \cdot -\frac{2}{n^{3}}}{-\frac{2}{n^{3}}}} \\
        &=\lim\limits_{n \to \infty}{\frac{-2n^{3}}{-2n^{3}\left(1+\frac{1}{n^{2}}\right)}} \\
        &=\lim\limits_{n \to \infty}{\frac{1}{\left(1+\frac{1}{n^{2}}\right)}} \\
        &=1
    .\end{align*}
    \bigbreak \noindent 
    \textbf{Conclusion.} Since $\summation{\infty}{n=1}\ \frac{1}{n^{2}}\  $ converges, so does $\summation{\infty}{n=1}\ \ln{\left(1+\frac{1}{n^{2}}\right)}\  $

    \pagebreak \bigbreak \noindent 
    \textbf{Problem 2.b} Choose $b_{n} = \frac{1}{4^{n}} = \left(\frac{1}{4}\right)^{n}$, which is a geometric series with $\abs{r}  < 1$ and thus converges
    \begin{align*}
        &\lim\limits_{n \to \infty}{\frac{\frac{1}{4^{n} -3^{n}}}{\frac{1}{4^{n}}}}  \\
        &=\lim\limits_{n \to \infty}{\frac{4^{n}}{4^{n} -3^{n}}} \\
        &=\lim\limits_{n \to \infty}{\frac{\frac{4^{n}}{4^{n}}}{\frac{4^{n}}{4^{n}}-\frac{3^{n}}{4^{n}}}} \\
        &=\lim\limits_{n \to \infty}{\frac{1}{1-\left(\frac{3}{4}\right)^{n}}} \\
        &=1
    .\end{align*}
    \bigbreak \noindent 
    Thus, since $\summation{\infty}{n=1}\ \frac{1}{4^{n}}\  $ converges, so does $\summation{\infty}{n=1}\ \frac{1}{4^{n} -3^{n}}\  $

    \bigbreak \noindent 
    \textbf{Problem 2.c} Choose $b_{n}  =\frac{1}{n^{2}}$, which we know converges
    \begin{align*}
        &\lim\limits_{n \to \infty}{\frac{1-\cos{\left(\frac{1}{n}\right)}}{\frac{1}{n^{2}}}}\ \quad \text{(Indeterminate $\frac{0}{0}$)} \\
        &\Heq\lim\limits_{n \to \infty}{\frac{\sin{\left(\frac{1}{n}\right)}\cdot -\frac{1}{n^{2}}}{-\frac{2}{n^{3}}}} \\
        &=\lim\limits_{n \to \infty}{\frac{n^{3}\sin{\left(\frac{1}{n}\right)}}{2n^{2}}} \\
        &=\frac{1}{2}\lim\limits_{n \to \infty}{n\sin{\left(\frac{1}{n}\right)}} \\
        &=\frac{1}{2}\lim\limits_{n \to \infty}{\frac{\sin{\left(\frac{1}{n}\right)}}{n^{-1}}} \quad \text{(Indeterminate $\frac{0}{0}$)} \\
        &\Heq \frac{1}{2}\lim\limits_{n \to \infty}{\frac{\cos{\left(\frac{1}{n}\right)}\cdot -\frac{1}{n^{2}}}{-\frac{1}{n^{2}}}} \\
        &=\frac{1}{2}\lim\limits_{n \to \infty}{\cos{\left(\frac{1}{n}\right)}} \\
        &=\frac{1}{2}\cos{0} \\
        &=\frac{1}{2}
    .\end{align*}
    \bigbreak \noindent 
    \textbf{Conclusion.}  Since $\lim\limits_{n \to \infty}{\frac{a_{n}}{b_{n}}} = L \ne 0 \text{ or } +\infty$, since $b_{n}$ converges, so does $a_{n}$ 

    \pagebreak \bigbreak \noindent 
    \begin{mdframed}
        3. Use the Alternating Series Test to determine whether the series is convergent or divergent.
        \begin{enumerate}[label=(\alph*)]
            \item $\frac{2}{3} -\frac{2}{5} + \frac{2}{7} -\frac{2}{9} + \frac{2}{11} - ... $
            \item $\summation{\infty}{n=1}\ (-1)^{n+1}\ \frac{1}{\sqrt{n+3}} $
            \item $\summation{\infty}{n=1}\ (-1)^{n+1}\ \cos^{2}{n}$
            \item $\summation{\infty}{n=1}\ (-1)^{n+1}\ \frac{n^{2}}{1+n^{4}} $
            \item $\summation{\infty}{n=1}\ (-1)^{n+1}\ \frac{\cos{n\pi}}{n} $
        \end{enumerate}
    \end{mdframed}
    \bigbreak \noindent 
    \textbf{Problem 3a.} We can see that the general term for this series is given by
    \begin{align*}
        \summation{\infty}{n=1}\ \frac{2(-1)^{n+1}}{2n+1}\ 
    .\end{align*}
    \bigbreak \noindent 
    Which is decreasing by 
    \begin{align*}
        \frac{2}{2n+3} \leq \frac{2}{2n+1}
    .\end{align*}
    \bigbreak \noindent 
    And
    \begin{align*}
        &\lim\limits_{n \to \infty}{b_{n}} = \lim\limits_{n \to \infty}{\frac{2}{2n+1}} \\
        &=0
    .\end{align*}
    \bigbreak \noindent 
    Since the series is decreasing and $\lim\limits_{n \to \infty}{b_{n}} = 0$, by Leibniz's criterion, this series will converge
    \bigbreak \noindent 
    \textbf{Problem 3b.} This series is decreasing by 
    \begin{align*}
        \frac{1}{\sqrt{n+4}} \leq \frac{1}{\sqrt{n+3}}
    .\end{align*}
    And 
    \begin{align*}
        &\lim\limits_{n \to \infty}{b_{n}} = \lim\limits_{n \to \infty}{\frac{1}{\sqrt{n+3}}} \\
        &=0
    .\end{align*}
    Since the series is decreasing and $\lim\limits_{n \to \infty}{b_{n}} = 0$, by Leibniz's criterion, this series will converge

    \pagebreak \bigbreak \noindent 
    \textbf{Problem 3c.} In this case, the AST does not apply because $b_n = \cos^{2}{n}$ is not monotone decreasing $\fa n \geq 1$. In fact, because $\cos^{2}{n}$ oscillates, there will be no such $N $ s.t $\cos^{2}{n}$ is monotone decreasing $\fa n \geq N$. Furthermore. Because
    \begin{align*}
        \lim\limits_{n \to \infty}{b_{n}} = \lim\limits_{n \to \infty}{\cos^{2}{n}} \ne 0
    .\end{align*}
    \bigbreak \noindent 
    This series is likely divergent

    \bigbreak \noindent 
    \textbf{Problem 3d.} We show $b_{n} = \frac{n^{2}}{1+n^{4}} $ is decreasing by
    \begin{align*}
        \frac{(n+1)^{2}}{1+(n+1)^{4}} \leq \frac{n^{2}}{1+n^{4}}
    .\end{align*}
    \bigbreak \noindent 
    And
    \begin{align*}
        &\lim\limits_{n \to \infty}{b_{n}} = \lim\limits_{n \to \infty}{\frac{n^{2}}{1+n^{4}}} \\
        &=\lim\limits_{n \to \infty}{\frac{\frac{n^{2}}{n^{4}}}{\frac{1}{n^{4}} + \frac{n^{4}}{n^{4}}} } \\
        &=\lim\limits_{n \to \infty}{\frac{\frac{1}{n^{2}}}{\frac{1}{n^{4}} + 1}} \\
        &= \frac{0}{1} \\
        &=0
    .\end{align*}
    \bigbreak \noindent 
    Since the series is decreasing and $\lim\limits_{n \to \infty}{b_{n}} = 0$, by Leibniz's criterion, this series will converge

    \bigbreak \noindent 
    \textbf{Problem 3e.} Upon examination of $b_n = \cos{n\pi}$, we realize that this series will only ever be -1 or 1 for $n$ in the family of integers. As $\cos{x} = -1$ for $(2k+1)\pi,\ k\in \mathbb{Z}$, and $\cos{x} = 1$ for $2k\pi,\ k\in \mathbb{Z}$, which essentially boils down to a simple deduction. For $n \in 2k+1,\ k\in \mathbb{Z}$, $\cos{(n\pi)} =-1$, for $n\in 2k,\ k\in\mathbb{Z}$, $\cos{n\pi}  = 1$ and we can rewrite this series as 
    \begin{align*}
        &\summation{\infty}{n=1}\ \frac{(-1)^{n+1} (-1)^{n}}{n}\  \\
        &=\summation{\infty}{n=1}\ \frac{(-1)^{2n+1}}{n}\  \\
        &=\summation{\infty}{n=1}\ -\frac{1}{n}\  \\
        &=-\summation{\infty}{n=1}\ \frac{1}{n}\ 
    .\end{align*}
    \bigbreak \noindent 
    Which we know diverges

    \pagebreak \bigbreak \noindent 
    \begin{mdframed}
        4. Use the Root or Ratios Test to determine whether the series is convergent or divergent.
        \begin{enumerate}[label=(\alph*)]
            \item $\summation{\infty}{n=1}\ \frac{2^{3n}(n!)^{3}}{(3n!)}\  $
            \item $\summation{\infty}{n=1}\ \frac{n!}{\left(\frac{n}{e}\right)^{n}}\  $
            \item $\summation{\infty}{n=1}\ \left(\frac{n-1}{2n+3}\right)^{n}\  $
            \item $\summation{\infty}{n=1}\ \frac{1}{(1+\ln{n})^{n}}\ $
        \end{enumerate}
    \end{mdframed}
    \bigbreak \noindent 
    \textbf{Problem 4a.} By the ratio test
    \begin{align*}
        &\rho = \lim\limits_{n \to \infty}{\bigg\lvert \frac{\frac{2^{3(n+1)}((n+1)!)^{3}}{3(n+1)!}}{\frac{2^{3n}(n!)^{3}}{3(n!)}} \bigg\rvert} \\
        &=\lim\limits_{n \to \infty}{\bigg\lvert \frac{2^{3n}2^{3}(n+1)^{3}(n!)^{3}}{3n!(n+1)} \cdot \frac{3n!}{2^{3n}(n!)^{3}} \bigg\rvert}\\
        &=\lim\limits_{n \to \infty}{8(n+1)^{2}} \\
        &=+\infty
    .\end{align*}
    \bigbreak \noindent 
    Thus, by the ratio test, this series will diverge

    \bigbreak \noindent 
    \textbf{Problem 4b.} By the ratio test
    \begin{align*}
        &\rho = \lim\limits_{n \to \infty}{\bigg\lvert \frac{\frac{(n+1)!}{\left(\frac{n+1}{e}\right)^{n+1}}}{\frac{n!}{\left(\frac{n}{e}\right)^{n}}} \bigg\rvert} \\
        &=\lim\limits_{n \to \infty}{\bigg\lvert \frac{n!(n+1)}{\left(\frac{n+1}{e}\right)^{n+1}} \cdot \frac{\left(\frac{n}{e}\right)^{n}}{n!} \bigg\rvert} \\
        &=\lim\limits_{n \to \infty}{\frac{(n+1)\left(\frac{n}{e}\right)^{n}}{\left(\frac{n+1}{e}\right)^{n+1}}} \\
        &=\lim\limits_{n \to \infty}{\frac{n+1 \cdot \frac{n^{n}}{e^{n}}}{\frac{(n+1)^{n+1}}{e^{n+1}}}} \\
        &=\lim\limits_{n \to \infty}{\frac{(n+1)n^{n}e^{n}e}{e^{n}(n+1)^{n}(n+1)}} \\
        &=\lim\limits_{n \to \infty}{\frac{en^{n}}{(n+1)^{n}}} \\ 
        &=e\lim\limits_{n \to \infty}{\left(\frac{n}{n+1}\right)^{n}}
    .\end{align*}
    \bigbreak \noindent 
    By Euler's definition $e = \lim\limits_{n \to \infty}{\left(1+\frac{1}{n}\right)^{n}} \implies \frac{1}{e} = \lim\limits_{n \to \infty}{\left(\frac{n}{n+1}\right)^{n}}$. This means we have
    \begin{align*}
        e \cdot \frac{1}{e} = 1
    .\end{align*}
    \bigbreak \noindent 
    Since $\rho = 1$, the ratio test does not yield conclusive results

    \pagebreak \bigbreak \noindent 
    \textbf{Problem 4c.} By the root test
    \begin{align*}
        &\rho = \lim\limits_{n \to \infty}{\bigg\lvert \left(\frac{n-1}{2n+3}\right)^{n} \bigg\rvert^{\frac{1}{n}}} \\
        &=\lim\limits_{n \to \infty}{\frac{n-1}{2n+3}} \\
        &=\frac{1}{2}
    .\end{align*}
    \bigbreak \noindent 
    Thus, by the root test, this series will converge

    \bigbreak \noindent 
    \textbf{Problem 4d.} By the root test
    \begin{align*}
        &\rho = \lim\limits_{n \to \infty}{\bigg\lvert\left(\frac{1}{1+\ln{n}}\right)^{n}\bigg\rvert^{\frac{1}{n}}} \\
        &=\lim\limits_{n \to \infty}{\frac{1}{1+\ln{n}}} \\
        &=0
    .\end{align*}
    \bigbreak \noindent 
    Thus, by the root test, this series will converge
    
    \pagebreak \bigbreak \noindent 
    \begin{mdframed}
        5. Use an appropriate test to determine whether the series is convergent or divergent.
            \begin{enumerate}[label=(\alph*)]
                \item $ \summation{\infty}{n=1}\ \frac{n+1}{n^{3}+n^{2} +n+1}\  $
                \item $ \summation{\infty}{n=1}\ (-1)^{n+1}\frac{n+1}{n^{3} + 3n^{2}  + 3n + 1}\  $
                \item $ \summation{\infty}{n=1}\ \frac{(n-1)^{n}}{(n+1)^{n}}\  $
                \item $ \summation{\infty}{n=1}\ \frac{n^{2}}{2^{n}}\  $
            \end{enumerate}
    \end{mdframed}

    \bigbreak \noindent 
    \textbf{Problem 5a.} Choose $b_{n} = \frac{1}{n^{2}} $, which by the p-series, converges. We can show by a simple comparison test that $\summation{\infty}{n=1}\ \frac{n+1}{n^{3}+n^{2} +n+1}\ \  $ also converges. Since 
    \begin{align*}
        \frac{n+1}{n^{3}+n^{2} +n+1}\ \leq \frac{1}{n^{2}}
    .\end{align*}
    \bigbreak \noindent 
    By simple comparison test, the series converges

    \bigbreak \noindent 
    \textbf{Problem 5b.} Since $\abs{a_{n}}$ = $\summation{\infty}{n=1}\ \frac{n+1}{n^{3}+3n^{2}+3n+1}\  $, we can compare this series to $\frac{1}{n^{2}}$. Since $\frac{1}{n^{2}}$ converges by the p-series, and 
    \begin{align*}
        \frac{n+1}{n^{3}+3n^{2}+3n+1} \leq\frac{1}{n^{2}}
    .\end{align*}
    \bigbreak \noindent 
    $\frac{n+1}{n^{3}+3n^{2}+3n+1}$ also converges, since $\abs{a_{n}}$ converges, $\summation{\infty}{n=1}\ (-1)^{n+1}\frac{n+1}{n^{3} + 3n^{2}  + 3n + 1} $ converges absolutely

    \bigbreak \noindent 
    \textbf{Problem 5c.} By the divergence test
    \begin{align*}
        &\lim\limits_{n \to \infty}{\left(\frac{n-1}{n+1}\right)^{n}} \\
        &=\lim\limits_{n \to \infty}{\left(\frac{n(1-\frac{1}{n})}{n(1+\frac{1}{n})\right)^{n}}} \\
        &=\lim\limits_{n \to \infty}{\left(\frac{(1-\frac{1}{n})}{(1+\frac{1}{n})\right)^{n}}} \\
        &=\lim\limits_{n \to \infty}{\frac{(1-\frac{1}{n})^{n}}{(1+\frac{1}{n})^{n}}} \\
        &=\lim\limits_{n \to \infty}{\frac{(1+\frac{-1}{n})^{n}}{(1+\frac{1}{n})^{n}}} \\
    .\end{align*}
    \bigbreak \noindent 
    \begin{Remark}
       Knowing Euler's definition $e = \lim\limits_{n \to \infty}{\left(1+\frac{1}{n}\right)^{n}} $, which is generalized as $e^{a} = \lim\limits_{n \to \infty}{\left(1+\frac{a}{n}\right)^{n}}$, our limit becomes 
    \end{Remark}
   \begin{align*}
       &\lim\limits_{n \to \infty}{\frac{e^{-1}}{e}} \\
       &=\frac{1}{e^{2}}
   .\end{align*}
   \bigbreak \noindent 
   Since this limit is not zero, our series diverges

   \pagebreak \bigbreak \noindent 
   \textbf{Problem 5.d} Using the ratio test
   \begin{align*}
       &\rho = \lim\limits_{n \to \infty}{\bigg\lvert \frac{\frac{(n+1)^{2}}{2^{n+1}}}{\frac{n^{2}}{2^{n}}} \bigg\rvert} \\
       &=\lim\limits_{n \to \infty}{\bigg\lvert\frac{(n+1)^{2}}{2^{n+1}} \cdot \frac{2^{n}}{n^{2}}\bigg\rvert} \\
       &=\lim\limits_{n \to \infty}{\frac{n^{2}+2n+1}{2n^{2}}} \\
       &=\frac{1}{2}
   .\end{align*}
   \bigbreak \noindent 
   Since $0 \leq \rho < 1$, by the ratio test, this series will converge


    






    





    
    
\end{document}
