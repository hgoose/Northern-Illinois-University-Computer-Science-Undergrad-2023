\documentclass{report}

\input{~/dev/latex/template/preamble.tex}
\input{~/dev/latex/template/macros.tex}

\title{\Huge{}}
\author{\huge{Nathan Warner}}
\date{\huge{}}
\pagestyle{fancy}
\fancyhf{}
\lhead{Warner \thepage}
\rhead{}
% \lhead{\leftmark}
\cfoot{\thepage}
% \setborder
% \usepackage[default]{sourcecodepro}
% \usepackage[T1]{fontenc}

\begin{document}
    % \maketitle
    % \tableofcontents
    \pagebreak \bigbreak \noindent
    Nate Warner \ \quad \quad \quad \quad \quad \quad \quad \quad \quad \quad \quad \quad \quad \quad \quad \quad \quad  MATH 230 \quad  \quad \quad \quad \quad \quad \quad \quad \quad \ \ \quad \quad November 11, 2023
    \begin{center}
        \textbf{Homework/Worksheet 9 - Due: Wednesday, November 15}
    \end{center}
    \bigbreak \noindent 
    \begin{mdframed}
        1. Use the comparison test to determine whether the series is convergent or divergent
        \begin{enumerate}[label=(\alph*)]
            \item $\summation{\infty}{n=1}\frac{1}{2n-1}$ 
            \item $\summation{\infty}{n=1}\frac{\sin^{2}{n}}{n^{2}}$
            \item $\summation{\infty}{n=1}\frac{1}{\sqrt{n^{2}+1}}$
        \end{enumerate}
    \end{mdframed}

    \bigbreak \noindent 
    \begin{remark}
        Suppose we have two series $a_{n}, b_{n}$ and $\te[]N \in \mathbb{Z^{+}}$ s.t $0 \leq a_{n} \leq b_{n}$ $\fa n \geq N$. If $b_{n}$ converges then $a_{n}$ will also converge. Conversely, if $a_{n} \geq b_{n} \geq 0$ $\fa n \geq N $, and $b_{n}$ diverges, then $a_{n}$ will also diverge
    \end{remark}
    

    \bigbreak \noindent 
    \textbf{Problem 1.a}. If we let $b_{n}$ be the series $\summation{\infty}{n=1}\ \frac{1}{2n}\ $. We may conject that this series will diverge since it is know that the harmonic series $\summation{}{}\ \frac{1}{n}\  $ diverges, and multiplying a series by a constant factor will not affect the convergence or divergence. Furthermore, 
    \begin{align*}
        \frac{1}{2n-1} > \frac{1}{2n}
    .\end{align*}
    \bigbreak \noindent 
    \textbf{Conclusion.} Thus, since $\summation{\infty}{n=1}\ \frac{1}{2n}\  $ diverges, we can conclude that $\summation{\infty}{n=1}\ \frac{1}{2n-1}\  $ will also diverge

    \bigbreak \noindent 
    \textbf{Problem 1.b} Let $b_{n}$ be the series $\summation{\infty}{n=1}\frac{1}{n^{2}}$ Since we know the sine function produces outputs in the range [-1,1], the sine function squared will produce outputs withing the range [0,1]. However, since we are only considering integer values $[1,\infty)$, $\sin^2{n}$ will only produce outputs $(0,1)$. This is because the sine function has outputs of 1 at $\frac{\pi}{2}  + 2k\pi,\ k \in \mathbb{Z}$, and outputs of 0 at $k\pi,\ k \in \mathbb{Z}$, 
    \bigbreak \noindent 
    \textbf{Problem 1.b:} Let \( b_n = \sum_{n=1}^{\infty} \frac{1}{n^2} \). We know The function \( \sin^2(x) \) yields values in the range [0,1], as \( \sin(x) \) varies between -1 and 1. For integer values \( n \) in the range [1, \infty), \( \sin^2(n) \) will produce values in the interval (0,1). This is because \( \sin(x) \) equals 1 at \( \frac{\pi}{2} + 2k\pi \) and 0 at \( k\pi \), where \( k \) is an integer, and these points are not integers. Thus we can conclude 
    \begin{align*}
        \frac{\sin^{2}{n}}{n^{2}} < \frac{1}{n^{2}}
    .\end{align*}
    \bigbreak \noindent 
    \textbf{Conclusion.} Since we know by the p-series $\frac{1}{n^{2}}$ will converge, $\frac{\sin^{2}{n}}{n^{2}}$ will also converge

    \pagebreak \bigbreak \noindent 
    \textbf{Problem 1.c} Let  $b_{n}$ be the series $\summation{\infty}{n=1}\ \frac{1}{n+1}\ $. We know this series will diverge because it is just the harmonic series $\frac{1}{n}$ shifted over by 1. We can deduce that $\frac{1}{\sqrt{n^{2} + 1}}  > \frac{1}{n+1}$ by looking at their reciprocals
    \begin{align*}
        &\sqrt{n^{2} + 1} < n+1 \\
        &n^{2} + 1 < (n+1)^{2} \\
        &n^{2}  + 1 < n^{2} + 2n +1 \\
        &n^{2} < n^{2} + 2n
    .\end{align*}
    \bigbreak \noindent 
    \textbf{Conclusion.} Since this is clearly a true statement, then by the reciprocal identify for inequalities, which states if $0 \leq a \leq b $, then $\frac{1}{a} \geq \frac{1}{b}$ it holds that $\frac{1}{\sqrt{n^{2} + 1}} > \frac{1}{n+1}$. and since we know $\summation{\infty}{n=1}\ \frac{1}{n+1} \  $ diverges, then $\summation{\infty}{n=1}\ \frac{1}{\sqrt{n^{2}+1}}\  $ will also diverge.
    
    \bigbreak \noindent 
    \begin{mdframed}
        2. Use the Limit Comparison Test to determine whether the series is convergent or divergent.
        \begin{enumerate}[label=(\alph*)]
            \item $\summation{\infty}{n=1}\ \ln{\left(1+\frac{1}{n^{2}}\right)}\  $
            \item $\summation{\infty}{n=1}\ \frac{1}{4^{n} - 3^{n}}\  $
            \item $\summation{\infty}{n=1}\ (1-\cos{\frac{1}{n}})\  $
        \end{enumerate}
    \end{mdframed}
    \bigbreak \noindent 
    \begin{remark}
       Suppose we have two series $a_{n},\ b_{n}$ where $a_{n},\ b_{n} \geq 0 \fa n \geq 1$. Then if
       \begin{itemize}
           \item $\lim\limits_{n \to \infty}{\frac{a_{n}}{b_{n}}} = L \ne 0 \text{ or } +\infty$. Then $a_{n}$ and $b_{n}$ either both converge or both diverge
            \item  $\lim\limits_{n \to \infty}{\frac{a_{n}}{b_{n}}} = 0$, then if $b_{n}$ converges, so does $a_{n}$
            \item  $\lim\limits_{n \to \infty}{\frac{a_{n}}{b_{n}}} = +\infty$, then if $b_{n}$ diverges, so does $a_{n}$
       \end{itemize}
    \end{remark}
    \bigbreak \noindent 
    \textbf{Problem 2.a} Let $b_{n}  = \frac{1}{n^{2}}$, which by the p-series, converges
    \begin{align*}
        &\lim\limits_{n \to \infty}{\frac{a_{n}}{b_{n}}} = \frac{\ln{\left(1+\frac{1}{n^{2}}\right)}}{\frac{1}{n^{2}}} \\
        &=\lim\limits_{n \to \infty}{\frac{\ln{\left(1+\frac{1}{n^{2}}\right)}}{n^{-2}}} \\
        &\Heq\lim\limits_{n \to \infty}{\frac{\frac{1}{1+\frac{1}{n^{2}}} \cdot -\frac{2}{n^{3}}}{-\frac{2}{n^{3}}}} \\
        &=\lim\limits_{n \to \infty}{\frac{-2n^{3}}{-2n^{3}\left(1+\frac{1}{n^{2}}\right)}} \\
        &=\lim\limits_{n \to \infty}{\frac{1}{\left(1+\frac{1}{n^{2}}\right)}} \\
        &=1
    .\end{align*}
    \bigbreak \noindent 
    \textbf{Conclusion.} Since $\summation{\infty}{n=1}\ \frac{1}{n^{2}}\  $ converges, so does $\summation{\infty}{n=1}\ \ln{\left(1+\frac{1}{n^{2}}\right)}\  $

    \pagebreak \bigbreak \noindent 
    \textbf{Problem 2.b} Choose $b_{n} = \frac{1}{4^{n}} = \left(\frac{1}{4}\right)^{n}$, which is a geometric series with $\abs{r}  < 1$ and thus converges
    \begin{align*}
        &\lim\limits_{n \to \infty}{\frac{\frac{1}{4^{n} -3^{n}}}{\frac{1}{4^{n}}}}  \\
        &=\lim\limits_{n \to \infty}{\frac{4^{n}}{4^{n} -3^{n}}} \\
        &=\lim\limits_{n \to \infty}{\frac{\frac{4^{n}}{4^{n}}}{\frac{4^{n}}{4^{n}}-\frac{3^{n}}{4^{n}}}} \\
        &=\lim\limits_{n \to \infty}{\frac{1}{1-\left(\frac{3}{4}\right)^{n}}} \\
        &=1
    .\end{align*}
    \bigbreak \noindent 
    Thus, since $\summation{\infty}{n=1}\ \frac{1}{4^{n}}\  $ converges, so does $\summation{\infty}{n=1}\ \frac{1}{4^{n} -3^{n}}\  $

    \bigbreak \noindent 
    \textbf{Problem 2.c} Choose $b_{n}  =\frac{1}{n^{2}}$, which we know converges
    \begin{align*}
        &\lim\limits_{n \to \infty}{\frac{1-\cos{\left(\frac{1}{n}\right)}}{\frac{1}{n^{2}}}}\ \quad \text{(Indeterminate $\frac{0}{0}$)} \\
        &\Heq\lim\limits_{n \to \infty}{\frac{\sin{\left(\frac{1}{n}\right)}\cdot -\frac{1}{n^{2}}}{-\frac{2}{n^{3}}}} \\
        &=\lim\limits_{n \to \infty}{\frac{n^{3}\sin{\left(\frac{1}{n}\right)}}{2n^{2}}} \\
        &=\frac{1}{2}\lim\limits_{n \to \infty}{n\sin{\left(\frac{1}{n}\right)}} \\
        &=\frac{1}{2}\lim\limits_{n \to \infty}{\frac{\sin{\left(\frac{1}{n}\right)}}{n^{-1}}} \quad \text{(Indeterminate $\frac{0}{0}$)} \\
        &\Heq \frac{1}{2}\lim\limits_{n \to \infty}{\frac{\cos{\left(\frac{1}{n}\right)}\cdot -\frac{1}{n^{2}}}{-\frac{1}{n^{2}}}} \\
        &=\frac{1}{2}\lim\limits_{n \to \infty}{\cos{\left(\frac{1}{n}\right)}} \\
        &=\frac{1}{2}\cos{0} \\
        &=\frac{1}{2}
    .\end{align*}
    \bigbreak \noindent 
    \textbf{Conclusion.}  Since $\lim\limits_{n \to \infty}{\frac{a_{n}}{b_{n}}} = L \ne 0 \text{ or } +\infty$, since $b_{n}$ converges, so does $a_{n}$ 
    





    
    
\end{document}
