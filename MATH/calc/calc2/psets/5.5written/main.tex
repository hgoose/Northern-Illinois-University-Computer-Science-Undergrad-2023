\documentclass{report}

\input{~/dev/latex/template/preamble.tex}
\input{~/dev/latex/template/macros.tex}

\title{\Huge{}}
\author{\huge{Nathan Warner}}
\date{\huge{}}
\pagestyle{fancy}
\fancyhf{}
\lhead{Warner \thepage}
\rhead{}
% \lhead{\leftmark}
\cfoot{\thepage}
%\setborder
% \usepackage[default]{sourcecodepro}
% \usepackage[T1]{fontenc}

\begin{document}
    % \maketitle
    %     \begin{titlepage}
    %    \begin{center}
    %        \vspace*{1cm}
    % 
    %        \textbf{}
    % 
    %        \vspace{0.5cm}
    %         
    %             
    %        \vspace{1.5cm}
    % 
    %        \textbf{Nathan Warner}
    % 
    %        \vfill
    %             
    %             
    %        \vspace{0.8cm}
    %      
    %        \includegraphics[width=0.4\textwidth]{~/niu/seal.png}
    %             
    %        Computer Science \\
    %        Northern Illinois University\\
    %        United States\\
    %        
    %             
    %    \end{center}
    % \end{titlepage}
    % \tableofcontents
    % \pagebreak \bigbreak \noindent
    \begin{remark}
        For a series of the form
        \begin{align*}
            \summation{\infty}{n=1}\ (-1)^{n+1}b_{n}\  \quad \text{ or } \summation{\infty}{n=1}\ (-1)^{n}b_{n}\ 
        .\end{align*}
        If the following conditions hold
        \begin{enumerate}[label=\roman*.]
            \item $b_{n+1} \leq b_{n} \fa   n \geq 1 $
            \item $\lim\limits_{n \to \infty}{b_{n}} = 0 $
        \end{enumerate}
        Then it follows that the series is convergent. This is known as the Leibniz criterion (or alternating series test)
    \end{remark}

    \bigbreak \noindent 
    \begin{mdframed}
        \textbf{Problem 1.} We want to use the Alternating Series Test to determine if the series:
        \begin{align*}
            \summation{\infty}{k=1}\ (-1)^{k+1} \frac{k^{4}}{\sqrt{k^{3} +6}}\ 
        .\end{align*}
        converges or diverges.
    \end{mdframed}
    \bigbreak \noindent 
    First we verify that $b_{k}$ is monotone decreasing $\fa k \geq 1$. If $b_{k+1} = \frac{(k+1)^{4}}{\sqrt{(k+1)^{3} + 6}}$. Then we can quite clearly see that
    \begin{align*}
        b_{k+1} \leq b_{k}     
    .\end{align*}
    Furthermore...
    \begin{align*}
        &\lim\limits_{k \to \infty}{b_{k}} = \lim\limits_{k \to \infty}{\frac{k^{4}}{\sqrt{k^{3}  + 6}}} \\
        &=\lim\limits_{k \to \infty}{\frac{\frac{k^{4}}{k^{3}}}{\sqrt{\frac{k^{3}}{k^{3}}+\frac{6}{k^{3}}}}} \\
        &= \lim\limits_{k \to \infty}{\frac{k}{\sqrt{1+\frac{6}{k^{3}}}}} \\
        &= \lim\limits_{k \to \infty}{\frac{\cancelto{\infty}{k}}{\sqrt{1+\cancelto{0}{\frac{6}{k^{3}}}}}} \\
        &=\infty
    .\end{align*}
    \textbf{Conclusion.} The Alternating Series Test does not apply because the absolute value of the terms do not approach 0, and the series diverges for the same reason.

    \pagebreak 
    \begin{mdframed}
        \textbf{Problem 2.} We want to use the Alternating Series Test to determine if the series:
        \begin{align*}
            \summation{\infty}{k=4}\ (-1)^{k+2}\frac{k^{3}}{\sqrt{k^{8}+12}}\ 
        .\end{align*}
         converges or diverges.
   \end{mdframed}
   \bigbreak \noindent 
   First we verify that $b_{k}$ is monotone decreasing $\fa x \geq 1$. If $b_{k+1} =  \frac{(k+1)^{3}}{\sqrt{(k+1)^{8} +12}}$. Then we can see that 
   \begin{align*}
       &\frac{(k+1)^{3}}{\sqrt{(k+1)^{8} +12}}  \leq \frac{k^{3}}{\sqrt{k^{8} + 12}}\\
       &\text{Thus }b_{k+1} \leq b_{k}
   .\end{align*}
   \bigbreak \noindent 
   Now we examine
   \begin{align*}
       &\lim\limits_{k \to \infty}{b_{k}} = \lim\limits_{k \to \infty}{\frac{k^{3}}{\sqrt{k^{8} + 12}}} \\
       &=\lim\limits_{k\to \infty}{\frac{\frac{k^{3}}{k^{8}}}{\sqrt{\frac{k^{8}}{k^{8}} +  \frac{12}{k^{8}}}}} \\
       &=\lim\limits_{k \to \infty}{\frac{\frac{1}{k^{5}}}{\sqrt{1+\frac{12}{k^{8}}}}}  \\
       &=\lim\limits_{k \to \infty}{\frac{\cancelto{0}{\frac{1}{k^{5}}}}{\sqrt{1+\cancelto{0}{\frac{12}{k^{8}}}}}}  \\ 
       &=0
   .\end{align*}
   \bigbreak \noindent 
   \textbf{Conclusion.} Since $b_{k}$ is monotone decreasing $\fa k \geq 1$ and $\lim\limits_{k \to \infty}{b_{k}} = 0 $. We conclude that the series must converge.

        \pagebreak 
    \begin{mdframed}
        \textbf{Problem 3.} We want to use the Alternating Series Test to determine if the series:
        \begin{align*}
            \summation{\infty}{k=1}\ \left(\frac{\sin^{2}{\left(\frac{k\pi}{2}\right)}}{k} - \frac{\cos^{2}{\left(\frac{k\pi}{2}\right)}}{2^{k}}\right)\ 
        .\end{align*}
        Converges or diverges
    \end{mdframed}
    \bigbreak \noindent 
    This is an interesting series so let's give it a closer look. The $\frac{\pi}{2}k$ in the sine and cosine functions arguments insinuate that these functions will oscillate with a period $T = \frac{2\pi}{w} = \frac{2\pi}{\frac{\pi}{2}}  = 4$. Thus, we check cases k $\in [1,4]$
    \begin{align*}
        &k=1:\ \sin^{2}{\left(\frac{\pi}{2}\right)} = 1, \quad \cos^{2}{\left(\frac{\pi}{2}\right)} = 0\\
        &k=2:\ \sin^{2}{\left(\pi\right)} = 0 \quad \cos^{2}{\left(\frac{\pi}{2}\right)} = 1\\
        &k=3:\ \sin^{2}{\left(\frac{3\pi}{2}\right)} = 1 \quad \cos^{2}{\left(\frac{\pi}{2}\right)} = 0\\
        &k=4:\ \sin^{2}{\left(2\pi\right)} = 0 \quad \cos^{2}{\left(\frac{\pi}{2}\right)} = 1\\
    .\end{align*}
    Which means we have...
    \begin{align*}
        &k=1:\ 1 - \frac{0}{2} = 1 \\
        &k=2:\ 0 - \frac{1}{4} = -\frac{1}{4} \\
        &k=3:\ \frac{1}{3} - 0 = \frac{1}{3} \\
        &k=4:\ 0 - \frac{1}{16} = -\frac{1}{16}
    .\end{align*}
    \bigbreak \noindent 
    So we see that this series is indeed alternating, however, when we examine the absolute value of these terms
    \begin{align*}
        1, \frac{1}{4}, \frac{1}{3}, \frac{1}{16}
    .\end{align*}
    \bigbreak \noindent 
    We notice that they are \textbf{not} monotone decreasing $\fa k \geq 1 $
    \bigbreak \noindent 
    \textbf{Conclusion.} The Alternating Series Test does not apply because the absolute value of the terms are not decreasing.






    \pagebreak 
    \begin{mdframed}
        \textbf{Problem 4.} Does the series 
        \begin{align*}
            \summation{\infty}{k=1}\ \frac{k^{2}}{\sqrt{k^{10} + 4}}\ 
        .\end{align*}
     converge absolutely, converge conditionally or diverge?
     \bigbreak \noindent 
     Does the series 
     \begin{align*}
         \summation{\infty}{k=1}\ \frac{(-1)^{k}k^{2}}{\sqrt{k^{10} + 4}}\ 
     .\end{align*}
     converge absolutely, converge conditionally or diverge?
    \end{mdframed}
    \bigbreak \noindent 
          converge absolutely, converge conditionally or diverge?
      \bigbreak \noindent 
      The most efficient way to determine the end behavior of these series is to first look at the series $\summation{\infty}{k=1}\ \frac{k^{2}}{\sqrt{k^{10}+4}}\  $. We notice that this series is the absolute value of the series $\summation{\infty}{k=1}\ \frac{(-1)^{k}k^{2}}{\sqrt{k^{10}+4}}\ $. Thus, if we find $\summation{\infty}{k=1}\ \frac{k^{2}}{\sqrt{k^{10}+4}}\  $ to converge, we know that $\summation{\infty}{k=1}\ \frac{(-1)^{k}k^{2}}{\sqrt{k^{10}+4} }\  $ converges absolutely
      \bigbreak \noindent 
      For the series $\summation{\infty}{k=1}\ \frac{k^{2}}{\sqrt{k^{10} +4}}\  $, we use the comparison test. Let $b_{k} =  \frac{1}{k^{3}}$. Which, by the p-series, we know will converge. Since
      \begin{align*}
          \frac{k^{2}}{\sqrt{k^{10} + 4}} < \frac{1}{k^{3}}\
      .\end{align*}
      \bigbreak \noindent 
      Then, we can conclude that by the simple comparison test, the series will converge. 
      \bigbreak \noindent 
      \textbf{Conclusion.} Since the absolute value series converges, we can conclude that both series converge absolutely

    \pagebreak
    \begin{mdframed}
        \textbf{Problem 5.} Does the series 
        \begin{align*}
            \summation{\infty}{k=1}\ \frac{1}{\sqrt[5]{k^{8} + 7}}\ 
        .\end{align*}
        Converge absolutely, converge conditionally or diverge?

        \bigbreak \noindent 
        Does the series 
        \begin{align*}
            \summation{\infty}{k=1}\ \frac{(-1)^{k}}{\sqrt[5]{k^{8}+7}}\ 
        .\end{align*}
        Converge absolutely, converge conditionally or diverge?
    \end{mdframed}
    \bigbreak \noindent 
    We make a similar claim as we did in problem 4. Comparing the absolute value series to $b_{k} = \frac{1}{k^{\frac{8}{5}}}$. Which, by the p-series, we know will converge. We see
    \begin{align*}
        \frac{1}{\sqrt{k^{8}+7}} < \frac{1}{k^{\frac{8}{5}}}
    .\end{align*}
    \bigbreak \noindent 
    \textbf{Conclusion.} Thus, the absolute value series also converges and we conclude that both series converge absolutely
    \pagebreak
    \begin{mdframed}
        \textbf{Problem 6.} Does the series 
        \begin{align*}
            \summation{\infty}{k=2}\ (-1)^{k} \frac{\ln{(k^{6})}}{k+7}\ 
        .\end{align*}
         converge absolutely, converge conditionally or diverge?
    \end{mdframed}
    \bigbreak \noindent 
    Examining the absolute value of this series $\summation{\infty}{k=2}\ \frac{\ln{k^{6}}}{k}\  $, we can again use a comparison test to see whether it diverges or converges. If we choose $b_{n}$ to be $\frac{1}{k}$. Then we can see that 
    \begin{align*}
        \frac{\ln{k^{6}}}{k} > \frac{1}{k}
    .\end{align*}
    \bigbreak \noindent 
    And since we know $\frac{1}{k}$ diverges, we can conclude $\summation{\infty}{k=1}\ \frac{\ln{k^{6}}}{k}\  $ will also diverge. Thus, we can have no absolute convergence.
    \bigbreak \noindent 
    We know must look at the alternating series. Since
    \begin{align*}
        \frac{\ln{(k+1)^{6}}}{k+1} < \frac{\ln{k^{6}}}{k} \implies b_{k +1} \leq b_{k}
    .\end{align*}
    \bigbreak \noindent 
    We conclude that the series $b_{k}$ is monotone decreasing. Now if we look at the limit
    \begin{align*}
        &\lim\limits_{k \to \infty}{b_{k}} = \lim\limits_{k \to \infty}{\frac{\ln{k^{6}}}{k}} \\
    .\end{align*}
    \bigbreak \noindent 
    Since $\ln{k^{6}}$ grows slower than $\frac{1}{k}$, the end behavior of $\frac{1}{k}$ will determine what this limit is. Thus, the limit is zero and we can conclude that the alternating series will converge
    \bigbreak \noindent 
    \textbf{Conclusion.} Since $\abs{a_{k}}$ diverges while $a_{k}$ converges, we say that the series $\summation{\infty}{k=2}\ \frac{\ln{k^{6}}}{k}\  $ converges conditionally.




    
\end{document}
