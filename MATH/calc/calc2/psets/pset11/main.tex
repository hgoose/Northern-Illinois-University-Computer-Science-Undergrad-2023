\documentclass{report}

\input{~/dev/latex/template/preamble.tex}
\input{~/dev/latex/template/macros.tex}


\title{\Huge{}}
\author{\huge{Nathan Warner}}
\date{\huge{}}
\pagestyle{fancy}
\fancyhf{}
\lhead{Warner \thepage}
\rhead{}
% \lhead{\leftmark}
\cfoot{\thepage}
% \setborder
% \usepackage[default]{sourcecodepro}
% \usepackage[T1]{fontenc}

\begin{document}
    % \maketitle
    % \tableofcontents
    \pagebreak \bigbreak \noindent
    Nate Warner \ \quad \quad \quad \quad \quad \quad \quad \quad \quad \quad \quad \quad \quad \quad \quad \quad \quad  MATH 230 \quad  \quad \quad \quad \quad \quad \quad \quad \quad \ \ \quad \quad December 6, 2023
    \begin{center}
        \textbf{Homework/Worksheet 11 - Due: Wednesday, November 15}
    \end{center}
    \bigbreak \noindent 
    \begin{mdframed}
       1. Use term-by-term differentiation or integration to find a power series representation for each function centered at the given point. 
       \begin{enumerate}[label=(\alph*)]
           \item $f(x) = \ln{(1-x)}$ centered at $x=0$
           \item $f(x) = \frac{2x}{(1-x^{2})^{2}}$ centered at $x=0$
           \item $f(x) = \tan^{-1}{x^{2}}$ centered at $x=0$
           \item $f(x) = \ln{(1+x^{2})}$ centered at $x=0$
       \end{enumerate}
    \end{mdframed}
    \bigbreak \noindent 
    \textbf{Problem 1a.} Using the fact that $\frac{d}{dx}\ln{(1-x)} = -\frac{1}{1-x}$ and $\frac{1}{1-x} = \summation{\infty}{n=0}\ x^{n}\  $ for $\abs{x} < 1$
    \begin{equation}
        \begin{alignedat}{2}
        &-\int \frac{1}{1-x}\ dx = -\int \summation{\infty}{n=0}\ x^{n}\ dx = -\int (1 + x + x^{2} + x^{3} + ...)\ dx \quad \quad &&\text{ for } \abs{x}< 1 \\
        &\ln{(1-x)} = \summation{\infty}{n=0}\ -\frac{x^{n+1}}{n+1}\  + C = -x - \frac{1}{2}x^{2} - \frac{1}{3}x^{3} - ... \quad \quad &&\text{ for } \abs{x}< 1\
    \end{alignedat}
    \end{equation}
    \bigbreak \noindent 
    \textbf{Conclusion.} When $x=0$, we find $C=0$. Thus,
    \begin{align*}
        \ln{(1-x)}  = \summation{\infty}{n=0}\ -\frac{x^{n+1}}{n+1}\  = -x -\frac{1}{2}x^{2} - \frac{1}{3}x^{3} + ... \quad \quad \text{ for } \abs{x} < 1
    .\end{align*}

    \bigbreak \noindent 
    \textbf{Problem 1b.}
    Using the fact that $\int \frac{2x}{(1-x^{2})^{2}}\ dx = \frac{1}{1-x^{2}}$, and $\frac{1}{1-x^{2}} = \summation{\infty}{n=0}\ x^{2n}\ $ for $\abs{x} < 1 $
    \begin{equation}
    \begin{alignedat}{2}
        &\frac{d}{dx}\frac{1}{1-x^{2}} = \frac{d}{dx} \summation{\infty}{n=0}\ x^{2n}\  = \frac{d}{dx} (1 + x^{2}  +x^{4} + x^{6} +...) \quad \quad  &&\text{ for } \abs{x} <1 \\
        &\frac{2x}{(1-x^{2})^{2}} = \summation{\infty}{n=0}\ 2nx^{2n-1}\ = 0 + 2x + 4x^{3} + 6x^{5} + ... \quad \quad &&\text{ for } \abs{x} < 1
    \end{alignedat}
    \end{equation}
    \textbf{Conclusion.} Thus we have
    \begin{align*}
        \frac{2x}{(1-x^{2})^{2}} = \summation{\infty}{n=1}\ 2nx^{2n-1}\ \quad \quad \text{for } \abs{x} <1 
    .\end{align*}

    \pagebreak \bigbreak \noindent 
    \textbf{Problem 1c.} Using the fact that $\frac{d}{dx}\tan^{-1}{x^{2}} = \frac{2x}{1-(-x^{4})}$ and $\frac{2x}{1-(-x^{4})} = \summation{\infty}{n=0}\ (-1)^{n}2x^{4n+1}\ $ for $\abs{x} < 1 $
    \bigbreak \noindent 
    \textbf{Remark.} The series $\summation{\infty}{n=0}\ 2x(-x^{4})^{n}\ $ is in the form $\summation{\infty}{n=0}\ bx^{m}c_{n}x^{n}\  $. By properties of combining power series, we know that this series must converge to $bx^{m}f(x)$ on the same interval of convergence as the simpler series, Since we know that $\summation{\infty}{n=0}\ (-x^{4})^{n}\  $  converges for $\abs{x^{4}} < 1 $, or $-1 < x < 1$. We can conclude that $2x(-x^{4})^{n} $ must do the same.
    \bigbreak \noindent 
    Thus we have:
    \begin{equation}
    \begin{alignedat}{2}
        &\int \frac{2x}{1+x^{4}}\ dx = \int \summation{\infty}{n=0}\ (-1)^n2x^{4n+1}\ dx = \int (2x - 2x^{5} + 2x^{9} - 2x^{13}  + ...)\ dx \quad \quad &&\text{ for } \abs{x} <1 \\
        &\tan^{-1}{x^{2}} = \summation{\infty}{n=0}\ (-1)^{n}\frac{2x^{4n+2}}{4n+2} + C\  = C + x^{2} - \frac{1}{3}x^{6}  + \frac{1}{5}x^{10} - \frac{1}{7}x^{14} + ... \quad \quad &&\text{ for } \abs{x} < 1 \\
        &\tan^{-1}{x^{2}} = \summation{\infty}{n=0}\ (-1)^{n}\frac{x^{4n+2}}{2n+1} + C\  = C + x^{2} - \frac{1}{3}x^{6}  + \frac{1}{5}x^{10} - \frac{1}{7}x^{14} + ... \quad \quad &&\text{ for } \abs{x} < 1
     \end{alignedat}
    \end{equation}
    \bigbreak \noindent 
    \textbf{Conclusion.} When $x=0$, $C=0$. Thus the power series for $\tan^{-1}{x^{2}} = \summation{\infty}{n=0}\ (-1)^{n}\frac{x^{4n+2}}{2n+1}\ $ for $\abs{x} < 1 $

    \bigbreak \noindent
    \textbf{Problem 1d.} Using the fact that $\frac{d}{dx}\ln{(1+x^{2})} = \frac{2x}{1-(-x^{2})}$ and $\frac{2x}{1-(-x^{2})} = \summation{\infty}{n=0}\ (-1)^{n}2x^{2n+1}\ $ for $\abs{x} < 1 $. Similar to the last problem, we know that this series converges for $\abs{x} < 1 $ by properties of combining power series. Thus
    \begin{equation}
    \begin{alignedat}{2}
        &\int \frac{2x}{1+x^{2}}\ dx = \int \summation{\infty}{n=0}\ (-1)^{n}2x^{2n+1}\ dx  = \int (2x-2x^{3}+2x^{5}-2x^{7} + ...)\ dx \quad \quad &&\text{ for } \abs{x} < 1 \\
        &\ln{(1+x^{2})} = \summation{\infty}{n=0}\ (-1)^{n}\frac{x^{2n+1}}{n+1}\  + C =  C + x^{2} - \frac{1}{2}x^{4}+\frac{1}{3}x^{6} -\frac{1}{4}x^{8} + ... \quad \quad &&\text{for } \abs{x} < 1
    \end{alignedat}
    \end{equation}
    \bigbreak \noindent 
    \textbf{Conclusion.} When $x=0 $, $C=0 $. Thus, the power series for $\ln{(1+x^{2})}  = \summation{\infty}{n=0}\ (-1)^{n}\frac{x^{2n+1}}{n+1}\ $ for $\abs{x} < 1 $

    \pagebreak 
    \begin{mdframed}
        2. Find the Taylor polynomial of degree two approximating the function \( f(x) = \cos(2x) \) at \( a = \pi \).
    \end{mdframed}
    \bigbreak \noindent 
    \begin{equation}
    \begin{alignedat}{2}
        &f(x) = \cos{(2x)} \quad \quad &&f(\pi) = \cos{(2\pi)} = 1 \\
        &f^{\prime}(x) = -2\sin{(2x)} \quad \quad &&f^{\prime}(\pi) = -2\sin{(2\pi)} = 0 \\
        &f^{\prime\prime}(x) = -4\cos{(2x)} \quad \quad &&f^{\prime\prime}(\pi) = -4\cos{(2\pi)} = -4
    \end{alignedat}
    \end{equation}
    We know the taylor series for a function $f$ conforms to the form
    \begin{align*}
        f(x) \sim \frac{f^{(n)}(a)}{n!}(x-a)^{n}
    .\end{align*}
    \bigbreak \noindent 
        \textbf{Conclusion.} Thus $P_{2}(x)$ conforms to 
    \begin{align*}
       &P_{2}(x) = 1 + 0(x-\pi) + \frac{-4}{2!}(x-\pi)^{2} \\
       &=1-2(x-\pi)^{2}\ 
    .\end{align*}

    \bigbreak \noindent 
    \begin{mdframed}
        3. Find the Taylor series of the functions $f(x)$ centered at the given value of $a$
        \begin{enumerate}[label=(\alph*)]
            \item $f(x) = \sin{(x)}, \quad a=\pi $
            \item $f(x) = e^{x}, \quad  a =-1$
            \item $f(x) = \ln{(x)}, \quad  a = 1$
            \item $f(x) = \frac{1}{2x-x^{2}}, \quad  a = 1$
        \end{enumerate}
    \end{mdframed}
    \bigbreak \noindent 
    \textbf{Problem 3a.}
    \begin{equation}
    \begin{alignedat}{2}
        &f(x) = \sin{(x)} \quad \quad &&f(\pi) = \sin{(\pi)} = 0 \\
        &f^{\prime}(x) = \cos{(x)} \quad \quad &&f^{\prime}(\pi) = \cos{(\pi)} = -1 \\
        &f^{\prime\prime}(x) = -\sin{(x)} \quad \quad &&f^{\prime\prime}(\pi) = -\sin{(\pi)} = 0 \\
        &f^{\prime\prime\prime}(x) = -\cos{(x)} \quad \quad &&f^{\prime\prime\prime}(\pi) = -\cos{(\pi)} = 1 \\
        &f^{(4)}(x) = \sin{(x)} \quad \quad &&f^{(4)}(\pi) = \sin{(\pi)} = 0
    \end{alignedat}
    \end{equation}
    Again, we use the Taylor series form
    \begin{align*}
        f(x) \sim \frac{f^{(n)}(a)}{n!}(x-a)^{n} = f(a) + f^{\prime}(a)(x-a) + \frac{f^{\prime\prime}(a)}{2!}(x-a)^{2}  + ...
    .\end{align*}
    To get 
    \begin{align*}
        &\sin{(x)} \sim \summation{\infty}{n=0}\ \frac{f^{(n)}(\pi)}{n!}(x-\pi)^{n}\  = 0 + (-1)(x-\pi) + \frac{0}{2!}(x-\pi)^{2}+\frac{1}{3!}(x-\pi)^{3}  + \frac{0}{4!}(x-\pi)^{4} + \frac{-1}{5!}(x-\pi)^{5} \\ &+ \frac{0}{6!}(x-\pi)^{6} +\frac{1}{7!}(x-\pi)^{7} +...\\
        &=-(x-\pi)+ \frac{1}{3!}(x-\pi^{3})-\frac{1}{5!}(x-\pi)^{5}+\frac{1}{7!}(x-\pi)^{7} +...
    .\end{align*}
    \bigbreak \noindent 
    Thus we have
    \begin{align*}
        \sin{(x)} = \summation{\infty}{n=0}\ (-1)^{n+1}\frac{(x-\pi)^{2n+1}}{(2n+1)!}\  
    .\end{align*}
    Using the ratio test we see
    \begin{align*}
        &\lim\limits_{n \to \infty}{\bigg\lvert \frac{(x-\pi)^{2n}(x-\pi)^{3}}{(2n+3)(2n+2)(2n+1)!} \cdot \frac{(2n+1)!}{(x-\pi)^{2n}(x-\pi)} \bigg\rvert} \\
        &(x-\pi)^{2}\lim\limits_{n \to \infty}{\frac{1}{(2n+3)(2n+2)}} \\
        &=0
    .\end{align*}
    \textbf{Conclusion.} Thus the Taylor series has the form
    \begin{align*}
        \sin{(x)} = \summation{\infty}{n=0}\ (-1)^{n+1}\frac{(x-\pi)^{2n+1}}{(2n+1)!}\  \quad \forall x\in \mathbb{R}
    .\end{align*}

    \bigbreak \noindent 
    \textbf{Problem 3b.} 
    \begin{equation}
    \begin{alignedat}{2}
        &f(x) = e^{x} \quad \quad &&f(-1) = \frac{1}{e} \\
        &f^{\prime}(x) = e^{x} \quad \quad &&f(-1) = \frac{1}{e} \\
        &f^{\prime\prime}(x) = e^{x} \quad \quad &&f(-1) = \frac{1}{e}  \\
        &\vdots \quad \quad &&\vdots
    \end{alignedat}
    \end{equation}
    By the definition of a Taylor series, we have
    \begin{align*}
        &e^{x} = \frac{1}{e} + \frac{1}{e}(x+1) + \frac{1}{2!e}(x+1)^{2}+\frac{1}{3!e}(x+1)^{3} +...
    .\end{align*}
    \bigbreak \noindent 
    Thus 
    \begin{align*}
        e^{x} = \summation{\infty}{n=0}\ \frac{(x+1)^{n}}{n!e}\ 
    .\end{align*}
    Using the ratio test we see
    \begin{align*}
        &\lim\limits_{n \to \infty}{\bigg\lvert \frac{(x+1)^{n}(x+1)}{(n+1)n!e} \cdot \frac{n!e}{(x+1)^{n}} \bigg\rvert} \\
        &\abs{x+1}\lim\limits_{n \to \infty}{\frac{1}{n+1}} \\
        &=0
    .\end{align*}
    \bigbreak \noindent 
    \textbf{Conclusion.} Thus the Taylor series for $e^{x}$ at center $a=-1$ is 
    \begin{align*}
        e^{x} = \summation{\infty}{n=0}\ \frac{(x+1)^{n}}{n!e}\ \quad \forall x\in \mathbb{R}
    .\end{align*}
    \bigbreak \noindent 
    \textbf{Problem 3c.}
    \begin{equation}
    \begin{alignedat}{2}
        &f(x) =\ln{(x)}  \quad \quad &&f(1) = 0 \\
        &f^{\prime}(x) = \frac{1}{x} \quad \quad &&f^{\prime}(1) = 1 \\
        &f^{\prime\prime}(x) = -\frac{1}{x^{2}} \quad \quad &&f^{\prime\prime}(1) = -1 \\
        &f^{\prime\prime\prime}(x) = \frac{2}{x^{3}} \quad \quad &&f(1) = 2 \\
        &f^{\prime\prime\prime}(x) = -\frac{6}{x^{4}} \quad \quad &&f(1) = -6
    \end{alignedat}
    \end{equation}
    \bigbreak \noindent
    By the definition of a Taylor series, we have
    \begin{align*}
       &\ln{(x)} = 0 + 1(x-1) + -\frac{1}{2!}(x-1)^{2} + \frac{2}{3!}(x-1)^{3} + \frac{-6}{4!}(x-1)^{4} + ... \\
       &= (x-1)-\frac{1}{2}(x-1)^{2}+\frac{1}{3}(x-1)^{3}-\frac{1}{4}(x-1)^{4}
    .\end{align*}
    \bigbreak \noindent 
    Using the ratio test
    \begin{align*}
        &\lim\limits_{n \to \infty}{\bigg\lvert \frac{(x-1)^{n}(x-1)^{2}}{n+2} \cdot \frac{n+1}{(x-1)^{n}(x-1)} \bigg\rvert} \\
        &=\abs{x-1}\lim\limits_{n \to \infty}{\frac{n+1}{n+2}} \\
        &\implies \abs{x-1} < 1 \text{ or } 0 < x < 2\\
        &\therefore R=2
    .\end{align*}
    When $x=0$ we have
    \begin{align*}
        &\sum_{n=0}^{\infty} (-1)^{n}\frac{(-1)^{n+1}}{n+1} \\
        =&\sum_{n=0}^{\infty} \frac{(-1)^{2n+1}}{n+1} \\
        =&\sum_{n=0}^{\infty} \frac{-1}{n+1} \\
        =-&\sum_{n=0}^{\infty} \frac{1}{n+1} 
    \end{align*}
    Which we know is divergent by the harmonic series. When $x=2$
    \begin{align*}
        &\summation{\infty}{n=0}\ \frac{(-1)^{n}}{n+1} \
    .\end{align*}
    Which converges by the alternating series test.
    \bigbreak \noindent 
    \textbf{Conclusion.} Thus the Taylor series has the form
    \begin{align*}
        \ln{(x)} \sim \summation{\infty}{n=0}\ (-1)^{n}\frac{(x-1)^{n+1}}{n+1}\ \quad \forall x\in (0,2]
    .\end{align*}

    \pagebreak \bigbreak \noindent 
    \textbf{Problem 3d.}
    \bigbreak \noindent 
    \textbf{Remark.} By Uniqueness of Taylor series, if a function $f$ has a power series at $a$ that converges to $f$ on some open interval containing $a$, then that power series is the Taylor series for $f$ at $a$
    \bigbreak \noindent 
    We find the power series.
    \begin{align*}
        \frac{1}{2x-x^{2}} = \frac{1}{x(2-x)}
    .\end{align*}
    \bigbreak \noindent 
    By partial fraction decomposition
    \begin{align*}
        &\frac{1}{x(2-x)} = \frac{A}{x} + \frac{B}{2-x} \\
        &1=A(2-x) + Bx
    .\end{align*}
    When $x=0$, $A=\frac{1}{2} $. When $x=2$, $B=\frac{1}{2}$. Thus we have
    \begin{align*}
        &\frac{1}{2x} + \frac{1}{2(2-x)} \\
        &=\frac{\frac{1}{2}}{1-(-(x-1))} + \frac{\frac{1}{2}}{1-(x-1)} 
    .\end{align*}
    \bigbreak \noindent 
    With
    \begin{equation}
    \begin{alignedat}{2}
     \frac{\frac{1}{2}}{1-(-(x-1))} &= \summation{\infty}{n=0}\ (-1)^{n}\frac{(x-1)^{n}}{2}\  \quad \quad &&\text{for $\abs{x-1} < 1$} \\
     \frac{\frac{1}{2}}{1-(x-1)} &=  \summation{\infty}{n=0}\ \frac{(x-1)^{n}}{2} \quad \quad &&\text{for $\abs{x-1} < 1$}\ 
    \end{alignedat}
    \end{equation}
    \bigbreak \noindent 
    Combining these series we have
    \begin{align*}
          \summation{\infty}{n=0}\ (-1)^{n}\frac{(x-1)^{n}}{2}\ + \summation{\infty}{n=0}\ \frac{(x-1)^{n}}{2} \quad \text{for $\abs{x-1} < 1$}\ 
    .\end{align*}
    Writing out the first few terms we see.
    \begin{align*}
        &\left(\frac{1}{2} + \frac{1}{2}\right) + \left(-\frac{x-1}{2} + \frac{x-1}{2}\right) + \left(\frac{(x-1)^{2}}{2} + \frac{(x-1)^{2}}{2}\right) + \left(-\frac{(x-1)^{3}}{2} + \frac{(x-1)^{3}}{2}\right) + \left(\frac{(x-1)^{4}}{2} + \frac{(x-1)^{4}}{2}\right)  + ...\\
        &=1 + (x-1)^{2} + (x-1)^{4} + ...
    .\end{align*}
    Which can be represented as 
    \begin{align*}
        \summation{\infty}{n=0}\ (x-1)^{2n} \quad \text{for $(x-1)^{2} < 1$}\ 
    .\end{align*}

    \pagebreak 
    \begin{mdframed}
        4. Find the Maclaurin series for $f(x) = x\cos{(x)}$ using the definition of a Maclaurin series. Also find the associated radius of convergence.
    \end{mdframed}
    \bigbreak \noindent 
    \begin{equation}
    \begin{alignedat}{2}
        &f(x) = x\cos{(x)} \quad \quad &&f(0) = 0 \\
        &f^{\prime}(x) = -x\sin{(x)}+\cos{(x)} \quad \quad &&f^{\prime}(0) = 1 \\
        &f^{\prime\prime}(x) = -x\cos{(x)} -2\sin{(x)} \quad \quad &&f^{\prime\prime}(0) = 0 \\
        &f^{\prime\prime\prime}(x) = x\sin{(x)} -3\cos{(x)} \quad \quad &&f^{\prime\prime\prime}(0) = -3 \\
        &f^{(4)}(x) = x\cos{(x)} + 4\sin{(x)} \quad \quad &&f^{(4)}(0) = 0 \\
        &f^{(5)}(x) = -x\sin{(x)} +5\cos{(x)} \quad \quad &&f^{(5)}(0) = 5
    \end{alignedat}
    \end{equation}
    \bigbreak \noindent 
    Thus,
    \begin{align*}
        &x\cos{(x)} = 0 + 1(x-0) + \frac{0}{2!}(x-0)^{2} + \frac{-3}{3!}(x-0)^{3} + \frac{0}{4!}(x-0)^{4} + \frac{5}{5!}(x-0)^{5} \\
        &=x -\frac{1}{2!}x^{3}+\frac{1}{4!}x^{5} + ...
    .\end{align*}
    So we see
    \begin{align*}
        x\cos{(x)} = \summation{\infty}{n=0}\ (-1)^{n}\frac{x^{2n+1}}{(2n)!}\ 
    .\end{align*}
    \bigbreak \noindent 
    To find the interval of convergence we use the ratio test
    \begin{align*}
        &\rho  = \lim\limits_{n \to \infty}{\bigg\lvert \frac{x^{2n}x^{2}}{(2n+1)(2n)!} \cdot \frac{(2n)!}{x^{2n}x}\bigg\rvert} \\
        &\abs{x} \lim\limits_{n \to \infty}{\frac{1}{2n+1}}\\
        &=0
    .\end{align*}
    \bigbreak \noindent 
    Thus, this series must converge $\forall x\in\mathbb{R}$, which implies the radius of convergence is $R = \infty$

    \pagebreak 
    \begin{mdframed}
        5. Use a known Maclaurin series to obtain the Maclaurin series for the given functions.
        \begin{enumerate}[label=(\alph*)]
            \item $ f(x) = x\cos{(2x)} $
            \item $f(x) = e^{3x} -e^{2x} $
        \end{enumerate}
    \end{mdframed}
    \bigbreak \noindent 
    \textbf{Problem 5a.} 
    \begin{align*}
        &\cos{(x)} = \summation{\infty}{n=0}\ (-1)^{n}\frac{x^{2n}}{(2n)!}\  \\
        &\implies\cos{(2x)} = \summation{\infty}{n=0}\ (-1)^{n}\frac{(2x)^{2n}}{(2n)!}\  = \summation{\infty}{n=0}\ (-1)^{n}\frac{4^{n}x^{2n}}{(2n)!}\  \\
        &\therefore x\cos{(2x)} = x \cdot \summation{\infty}{n=0}\ (-1)^{n}\frac{4^{n}x^{2n}}{(2n)!}\ = \summation{\infty}{n=0}\ (-1)^{n}\frac{4^{n}x^{2n+1}}{(2n)!}\ 
    .\end{align*}
    \bigbreak \noindent 
    \textbf{Problem 5b.}
    \begin{align*}
        &e^{x} = \summation{\infty}{n=0}\ \frac{x^{n}}{n!}\  \\
        &\implies e^{3x} = \summation{\infty}{n=0}\ \frac{(3x)^{n}}{n!}\  \\
        &\implies e^{2x} = \summation{\infty}{n=0}\ \frac{(2x)^{n}}{n!}\  \\
        &\therefore e^{3x} - e^{2x} = \summation{\infty}{n=0}\ \frac{3^{n}x^{n}}{n!}\ - \summation{\infty}{n=0}\ \frac{2^{n}x^{n}}{n!}\   \\
        &=\summation{\infty}{n=0}\ \frac{x^{n}(3^{n}-2^{n})}{n!}\ 
    .\end{align*}







    
    
\end{document}
