\documentclass{report}

\input{~/dev/latex/template/preamble.tex}
\input{~/dev/latex/template/macros.tex}

\title{\Huge{}}
\author{\huge{Nathan Warner}}
\date{\huge{}}
\pagestyle{fancy}
\fancyhf{}
\lhead{Warner \thepage}
\rhead{}
% \lhead{\leftmark}
\cfoot{\thepage}
%\setborder
% \usepackage[default]{sourcecodepro}
% \usepackage[T1]{fontenc}

\begin{document}
    % \maketitle
        \begin{titlepage}
       \begin{center}
           \vspace*{1cm}
    
           \textbf{EXERCSIES 5.3 The Divergence and Integral Tests} \\
           Written assignment
    
           \vspace{0.5cm}
            
                
           \vspace{1.5cm}
    
           \textbf{Nathan Warner}
    
           \vfill
                
                
           \vspace{0.8cm}
         
           \includegraphics[width=0.4\textwidth]{~/niu/seal.png}
                
           Computer Science \\
           Northern Illinois University\\
           November 2, 2023 \\
           % United States\\
           
                
       \end{center}
    \end{titlepage}
    \tableofcontents
    \pagebreak 
    \begin{mdframed}
        3. Use the Divergene Test to determine the whether the series converges or diverges. 
        \begin{align*}
            \summation{\infty}{n=1}\ \left(1+\frac{9}{n}\right)^{n} 
        .\end{align*}
    \end{mdframed}
    \bigbreak \noindent 
    Given the fact that Euler's number has a definition of the form:
    \begin{align*}
        \lim\limits_{n \to \infty}{\left(1+\frac{1}{n}\right)^{n}} = e
    .\end{align*}
    \bigbreak \noindent 
   With a  generalization of 
   \begin{align*}
       e^{a} = \lim\limits_{n \to \infty}{\left(1+\frac{a}{n}\right)^{n}}
   .\end{align*}
   \bigbreak \noindent 
   Using the divergence test for the series $\summation{\infty}{n=1}\ \left(1+\frac{9}{n}\right)^{n}\ $, we get the $\lim\limits_{n \to \infty}{\left(1+\frac{9}{n}\right)^{n}} $. Which will trivially  yield $e^{9}$. However, this can be shown...
   \begin{align*}
       &\lim\limits_{n \to \infty}{\left(1+\frac{9}{n}\right)^{n}} \\
       &=\lim\limits_{n \to \infty}{e^{\ln{\left(1+\frac{9}{n}\right)^{n}}}} \\
       &=\lim\limits_{n \to \infty}{e^{n\ln{\left(1+\frac{9}{n}\right)}}} 
   .\end{align*}
   \bigbreak \noindent 
   Focusing on $n\ln{\left(1+\frac{9}{n}\right)} $...
   \begin{align*}
       &\lim\limits_{n \to \infty}{n\ln{\left(1+\frac{9}{n}\right)}} \quad \text{(Indeterminate $\infty\cdot 0$)} \\
       &=\lim\limits_{n \to \infty}{\frac{\ln{\left(1+\frac{9}{n}\right)}}{n^{-1}}} \quad \text{$\left(\frac{0}{0}\right)$} \\
       & \Heq\lim\limits_{n \to \infty}{\frac{\frac{1}{1+\frac{9}{n}}\cdot \left(-\frac{9}{n^{2}}\right)}{-\frac{1}{n^{2}}}} \\
       &=\lim\limits_{n \to \infty}{\frac{-\frac{9}{n^{2}+2n}}{-\frac{1}{n^{2}}}} \\
       &=\lim\limits_{n \to \infty}{\frac{9n^{2}}{n^{2}+2n}} \\
       &\lim\limits_{n \to \infty}{\frac{9n}{n+2}} \quad \text{$\left(\text{Still indeterminate... } \frac{\infty}{\infty}\right)$} \\
       &\Heq\lim\limits_{n \to \infty}{\frac{9}{1}} \\ 
       &=9
   .\end{align*}
   \bigbreak \noindent 
   Thus,
   \begin{align*}
       &\lim\limits_{n \to \infty}{\left(1+\frac{9}{n}\right)^{n}} \\
       &=\lim\limits_{n \to \infty}{e^{n\ln{\left(1+\frac{9}{n}\right)}}} \\
       &=e^{9}
   .\end{align*}

    
\end{document}
