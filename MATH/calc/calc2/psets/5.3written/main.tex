\documentclass{report}

\input{~/dev/latex/template/preamble.tex}
\input{~/dev/latex/template/macros.tex}

\title{\Huge{}}
\author{\huge{Nathan Warner}}
\date{\huge{}}
\pagestyle{fancy}
\fancyhf{}
\lhead{Warner \thepage}
\rhead{}
% \lhead{\leftmark}
\cfoot{\thepage}
%\setborder
% \usepackage[default]{sourcecodepro}
% \usepackage[T1]{fontenc}

\begin{document}
    % \maketitle
    %     \begin{titlepage}
    %    \begin{center}
    %        \vspace*{1cm}
    % 
    %        \textbf{EXERCSIES 5.3 The Divergence and Integral Tests} \\
    %        Written assignment
    % 
    %        \vspace{0.5cm}
    %         
    %             
    %        \vspace{1.5cm}
    % 
    %        \textbf{Nathan Warner}
    % 
    %        \vfill
    %             
    %             
    %        \vspace{0.8cm}
    %      
    %        \includegraphics[width=0.4\textwidth]{~/niu/seal.png}
    %             
    %        Computer Science \\
    %        Northern Illinois University\\
    %        November 2, 2023 \\
    %        % United States\\
    %        
    %             
    %    \end{center}
    % \end{titlepage}
    % \tableofcontents
    % \pagebreak \bigbreak \noindent 
    \begin{mdframed}
        \textbf{Problem 1.} Consider the series
        \begin{align*}
            \summation{\infty}{n=1}\ \frac{3n^{3}}{2n^{3}+4}\ 
        .\end{align*}
        Based on the Divergence Test, does this series Diverge?
    \end{mdframed}
    \bigbreak \noindent 
    By the divergence test
    \begin{align*}
        &\lim\limits_{n \to \infty}{\frac{3n^{3}}{2n^{3} +4}} \\
        &=\frac{3}{2}
    .\end{align*}
    \bigbreak \noindent 
    Thus, since the limit is not zero. This series will diverge

    \bigbreak \noindent 
    \begin{mdframed}
        \textbf{Problem 2.} What does the divergence test tell you about each of the series below?
        \begin{enumerate}[label=(\alph*)]
            \item $\summation{\infty}{n=1}\ 3^{n}\ $
            \item $\summation{\infty}{n=1}\ 7^{-n}\ $
            \item $\summation{\infty}{n=0}\ \left(\frac{1}{e}\right)^{n}\ $
            \item $\summation{\infty}{n=0}\ \left(\frac{7}{3}\right)^{n}\ $
        \end{enumerate}
    \end{mdframed}
    \bigbreak \noindent 
    \textbf{Part A.}
    \begin{align*}
        &\lim\limits_{n \to \infty}{3^{n}} \\
        &=+\infty
    .\end{align*}
    Thus, since the limit is not zero. The divergence test tells us that this series will diverge

    \bigbreak \noindent 
    \textbf{Part B.}
    \begin{align*}
        &\lim\limits_{n \to \infty}{7^{-n}} \\
        &=0
    .\end{align*}
    Since the limit is zero, the divergence test is inconclusive 

    \bigbreak \noindent 
    \textbf{Part C.}
    \begin{align*}
        &\lim\limits_{n \to \infty}{\left(\frac{1}{e}\right)^{n}} \\
        &=0
    .\end{align*}
    Since the limit is zero, the divergence test is inconclusive 

    \bigbreak \noindent 
    \textbf{Part D.}
    \begin{align*}
        &\lim\limits_{n \to \infty}{\left(\frac{7}{3}\right)^{n}}  \\
        &=+\infty
    .\end{align*}
    Thus, since the limit is not zero. The divergence test tells us that this series will diverge



    \pagebreak \bigbreak \noindent 
    \begin{mdframed}
        \textbf{Problem 3.} Use the Divergene Test to determine the whether the series converges or diverges. 
        \begin{align*}
            \summation{\infty}{n=1}\ \left(1+\frac{9}{n}\right)^{n} 
        .\end{align*}
    \end{mdframed}
    \bigbreak \noindent 
    Given the fact that Euler's number has a definition of the form:
    \begin{align*}
        \lim\limits_{n \to \infty}{\left(1+\frac{1}{n}\right)^{n}} = e
    .\end{align*}
    \bigbreak \noindent 
   With a  generalization of 
   \begin{align*}
       e^{a} = \lim\limits_{n \to \infty}{\left(1+\frac{a}{n}\right)^{n}}
   .\end{align*}
   \bigbreak \noindent 
   Using the divergence test for the series $\summation{\infty}{n=1}\ \left(1+\frac{9}{n}\right)^{n}\ $, we get the $\lim\limits_{n \to \infty}{\left(1+\frac{9}{n}\right)^{n}} $. Which will trivially  yield $e^{9}$. However, this can be shown...
   \begin{align*}
       &\lim\limits_{n \to \infty}{\left(1+\frac{9}{n}\right)^{n}} \\
       &=\lim\limits_{n \to \infty}{e^{\ln{\left(1+\frac{9}{n}\right)^{n}}}} \\
       &=\lim\limits_{n \to \infty}{e^{n\ln{\left(1+\frac{9}{n}\right)}}} 
   .\end{align*}
   \bigbreak \noindent 
   Focusing on $n\ln{\left(1+\frac{9}{n}\right)} $...
   \begin{align*}
       &\lim\limits_{n \to \infty}{n\ln{\left(1+\frac{9}{n}\right)}} \quad \text{(Indeterminate $\infty\cdot 0$)} \\
       &=\lim\limits_{n \to \infty}{\frac{\ln{\left(1+\frac{9}{n}\right)}}{n^{-1}}} \quad \text{$\left(\frac{0}{0}\right)$} \\
       & \Heq\lim\limits_{n \to \infty}{\frac{\frac{1}{1+\frac{9}{n}}\cdot \left(-\frac{9}{n^{2}}\right)}{-\frac{1}{n^{2}}}} \\
       &=\lim\limits_{n \to \infty}{\frac{-\frac{9}{n^{2}+2n}}{-\frac{1}{n^{2}}}} \\
       &=\lim\limits_{n \to \infty}{\frac{9n^{2}}{n^{2}+2n}} \\
       &\lim\limits_{n \to \infty}{\frac{9n}{n+2}} \quad \text{$\left(\text{Still indeterminate... } \frac{\infty}{\infty}\right)$} \\
       &\Heq\lim\limits_{n \to \infty}{\frac{9}{1}} \\ 
       &=9
   .\end{align*}
   \bigbreak \noindent 
   Thus,
   \begin{align*}
       &\lim\limits_{n \to \infty}{\left(1+\frac{9}{n}\right)^{n}} \\
       &=\lim\limits_{n \to \infty}{e^{n\ln{\left(1+\frac{9}{n}\right)}}} \\
       &=e^{9}
   .\end{align*}

   \pagebreak \bigbreak \noindent 
   \begin{mdframed}
      \textbf{Problem 4.} To test the series  $\summation{\infty}{n=1}\ e^{-3n}\ $ for convergence, you can use the Integral Test. (This is also a geometric series, so we could also investigate convergence using other methods.) What does this value tell you about the convergence of the series
      \begin{align*}
          \summation{\infty}{n=1}\ e^{-3n}\ 
      .\end{align*}
   \end{mdframed}
   \bigbreak \noindent 
   Since $a_{n}$ has positive terms, and $a_{n} = f(n)$, for $f(x) = e^{-3x}$ on $[1,+\infty)$, satisfying 
   \begin{itemize}
       \item Continuous 
        \item Positive, decreasing
   \end{itemize}
   Then by the integral test
   \begin{align*}
        &\int_{1}^{\infty}\ e^{-3x}\ dx      \\
        &=\lim\limits_{t \to \infty}{\int_{1}^{t}\ e^{-3x}\ dx} \\
        &=\lim\limits_{t \to \infty}{-\frac{1}{3}e^{-3x}}\ \bigg|_1^{t} \\
        &=\lim\limits_{t \to \infty}{-\frac{1}{3}e^{-3t}} - \left(-\frac{1}{3}e^{-3}\right) \\
        &=\frac{1}{3e^{3}}
   .\end{align*}
   Since the improper integral converges, so does the series

   \pagebreak \bigbreak \noindent 
   \begin{mdframed}
       \textbf{Problem 5.} Compute the value of the following improper integral
       \begin{align*}
          \int_{1}^{\infty}\ \frac{2\ln{(x)}}{x^{6}}\ dx 
       .\end{align*}
       What does the value of the improper integral tell use about the convergence of the series
       \begin{align*}
           \summation{\infty}{n=1}\ \frac{2\ln{(n)}}{n^{6}}\ 
       .\end{align*}
   \end{mdframed}
   \bigbreak \noindent 
   \begin{minipage}[t]{0.5\textwidth}
   \begin{align*}
       &\int_{1}^{\infty}\ \frac{2\ln{(x)}}{x^{6}}\ dx \\
       &=\lim\limits_{t \to \infty}{\int_{1}^{t}\ \frac{2\ln{(x)}}{x^{6}}\ dx} \\
      &=2\lim\limits_{t \to \infty}{\int_{1}^{\infty}\ x^{-6}\ln{(x)}\ dx} \\
      &=2\lim\limits_{t \to \infty}{-\frac{1}{5x^{5}}\ln{x}}\ \bigg|_1^{t} - \int_{1}^{t}\ -\frac{1}{5}x^{-6}\ dx \\
      &=2\lim\limits_{t \to \infty}{-\frac{1}{5t^{5}}\ln{t}} + \int_{1}^{t}\ \frac{1}{5}x^{-6}\ dx \\
      &=2\lim\limits_{t \to \infty}{-\frac{1}{5t^{5}}\ln{t}} + \left(-\frac{1}{25x^{5}}\ \bigg|_1^{t}\right)\\
        &=2\lim\limits_{t \to \infty}{-\frac{1}{5t^{5}}\ln{t}} -\frac{1}{25t^{5}} + \frac{1}{25}\\
        &=2\lim\limits_{t \to \infty}{\cancelto{0}{-\frac{1}{5t^{5}}\ln{t}}} \cancelto{0}{-\frac{1}{25t^{5}}} + \frac{1}{25}\\
        &=\frac{2}{25}
   .\end{align*}
   \end{minipage}
   \begin{minipage}[t]{0.42\textwidth}
       \begin{align*}
           &u = \ln{(x)} \quad dv = x^{-6}\ dx \\
           &du = \frac{1}{x}\ dx \quad v = -\frac{1}{5}x^{-5}
       .\end{align*}
   \end{minipage}
   \bigbreak \noindent 
   Since the improper integral converges, so does the series

   \pagebreak \bigbreak \noindent 
   \begin{mdframed}
       \textbf{Problem 6.} Compute the value of the improper integral
       \begin{align*}
           \int_{2}^{\infty}\ \frac{dx}{(2x+3)^{7}}\ dx
       .\end{align*}
       Use your answer to help determine whether the series 
       \begin{align*}
           \summation{\infty}{n=2}\ \frac{1}{(2n+3)^{7}}\ 
       .\end{align*}
       converges or diverges
   \end{mdframed}
   \begin{align*}
       &\int_{2}^{\infty}\ \frac{dx}{(2x+3)^{7}}\ dx \\
       &=\frac{1}{2}\lim\limits_{t \to \infty}{\int_{2}^{t}\ \frac{dx}{(2x+3)^{7}}\ dx} \\
       &=\frac{1}{2}\lim\limits_{t \to \infty}{-\frac{1}{6}(2x+3)^{-6}\ \bigg|_2^{t}} \\
       &=\frac{1}{2}\lim\limits_{t \to \infty}{-\frac{1}{6}\bigg[(2t+3)^{-6} - (7)^{-6}\bigg]} \\
       &=\frac{1}{2}\lim\limits_{t \to \infty}{-\frac{1}{6}\cancelto{0}{\bigg[(2t+3)^{-6}} - (7)^{-6}\bigg]} \\
        &=-\frac{1}{12}\left(-\frac{1}{7^{6}}\right) \\
        &=\frac{1}{1411788}
   .\end{align*}
   \bigbreak \noindent 
   Since the improper integral converges, so does the series


   \pagebreak \bigbreak \noindent 
   \begin{mdframed}
       \textbf{Problem 7.} Compute the value of the improper integral
       \begin{align*}
           \int_{1}^{\infty}\ \frac{3}{1+x^{2}} \ dx
       .\end{align*}
       Use the value of the improper integral to determine whether or not the series
       \begin{align*}
           \summation{\infty}{n=1}\ \frac{3}{1+n^{2}}\ 
       .\end{align*}
       converges or diverges
   \end{mdframed}
   \begin{align*}
       &\int_{1}^{\infty}\ \frac{3}{1+x^{2}}\ dx \\
       &=3\lim\limits_{t \to \infty}{\frac{1}{1+x^{2}}} \\
       &=3\lim\limits_{t \to \infty}{\tan^{-1}{x}}\ \bigg|_1^{t} \\
       &3\lim\limits_{t \to \infty}{\tan^{-1}{t} - \tan^{-1}{1}} \\
       &=3\bigg[\frac{pi}{2} - \frac{\pi}{4}\bigg] \\
       &=\frac{3\pi}{4}
   .\end{align*}
  Since the improper integral converges, so does the series


  \bigbreak \noindent 
   \begin{mdframed}
       \textbf{Problem 8.} To test the series 
       \begin{align*}
           \summation{\infty}{n=1}\ \frac{1}{k^{3}}\ 
       .\end{align*}
       For convergence, you can use the P-test. Then compute $S_{3}$, the partial sum consisting of the first 3 terms of $\summation{\infty}{k=1}\ \frac{1}{k^{3}}\  $
   \end{mdframed}
   \bigbreak \noindent 
   Since $P = 3 > 1$, this series will converge. For $S_{3}$...
   \begin{align*}
       &S_{3} = 1+ \frac{1}{8} + \frac{1}{27} \\
       &=\frac{251}{216} \\
       &\approx 1.16
   .\end{align*}

   \pagebreak \bigbreak \noindent 
   \begin{mdframed}
       \textbf{Problem 9.} To test the series 
       \begin{align*}
           \summation{\infty}{k=1}\ \frac{1}{\sqrt[5]{k^{4}}}\ 
       .\end{align*}
       for convergence, you can use the P-test. Then compute $S_{3}$, the partial sum consisting of the first 3 terms of $\summation{\infty}{k=1}\ \frac{1}{\sqrt[5]{k^{4}}}\  $
   \end{mdframed}
   \bigbreak \noindent 
   Since $P = \frac{4}{5} \leq 1$. This series will diverge. For $S_{3}$...
   \begin{align*}
       &S_{3} = 1 + \frac{1}{2^{\frac{4}{5}}} + \frac{1}{3^{\frac{4}{5}}} \\
       &\approx 1.9896
   .\end{align*}

    
\end{document}
