\documentclass{report}

\input{~/dev/latex/template/preamble.tex}
\input{~/dev/latex/template/macros.tex}

\title{\Huge{}}
\author{\huge{Nathan Warner}}
\date{\huge{}}
\pagestyle{fancy}
\fancyhf{}
\lhead{Warner \thepage}
\rhead{}
% \lhead{\leftmark}
\cfoot{\thepage}
% \setborder
% \usepackage[default]{sourcecodepro}
% \usepackage[T1]{fontenc}

\begin{document}
    % \maketitle
    % \tableofcontents
    \pagebreak \bigbreak \noindent
    Nate Warner \ \quad \quad \quad \quad \quad \quad \quad \quad \quad \quad \quad \quad \quad \quad \quad \quad \quad  MATH 230 \quad  \quad \quad \quad \quad \quad \quad \quad \quad \ \ \quad \quad November 3, 2023
    \begin{center}
        \textbf{Homework/Worksheet 8 - Due: Wednesday, November 8}
    \end{center}
    \bigbreak \noindent 
    \begin{mdframed}
        1. Determine whether the geometric series is convergent or divergent. If it is convergent, find its sum.
        \begin{enumerate}[label=(\alph*)]
            \item $1+\frac{e}{\pi} + \frac{e^{2}}{\pi^{2}} + \frac{e^{3}}{\pi^{3}} + \cdots$
            \item $a_{1} = 2$ and $an/a_{n+1} = \frac{1}{2}$ for $n \geq 1$
            \item $\summation{\infty}{n=2}\ \frac{1}{n^{2}-1}\  $
             \item $\summation{\infty}{n=1}\ (\sin{n} - \sin{(n+1)})\ $
        \end{enumerate}
    \end{mdframed}

    \bigbreak \noindent 
    \begin{remark}
        Regarding a geometric series, we know:
        \begin{align*}
            \summation{\infty}{n=1}\ ar^{n-1}\  
                    \begin{cases}
                         \frac{a}{1-r} & \text{if } \abs{r} < 1 \\
                         \text{Diverges} & \text{if }  \abs{r} \geq 1
                    \end{cases}
        .\end{align*}
    \end{remark}
    
    \bigbreak \noindent 
    \textbf{Problem 1.a}:
    We can see this series conforms to 
    \begin{align*}
       \summation{\infty}{n=1}\ \left(\frac{\pi}{e}\right)^{n-1}
    .\end{align*}
    \bigbreak \noindent 
    Thus we have $a=1$, $r=\frac{\pi}{e}$, and we can assert
    \begin{align*}
       &S = \frac{a}{1-r}  \\
       &=\frac{1}{1-\frac{e}{\pi}} \\
        &=\frac{1}{\frac{\pi-e}{\pi}} \\
        &=\frac{\pi}{\pi-e}
    .\end{align*}

    \bigbreak \noindent 
    \textbf{Problem 1.b}: We can see that $r=\frac{1}{2}$, $a=2$. Thus we have the series
    \begin{align*}
        &\summation{\infty}{n=1}\ 2\left(\frac{1}{2}\right)^{n-1}\ 
    .\end{align*}
    Where 
    \begin{align*}
        &S = \frac{2}{1-\frac{1}{2}} \\
        &= \frac{2}{\frac{1}{2}} \\
        &= 4
    .\end{align*}

    \pagebreak \bigbreak \noindent 
    \textbf{Problem 1.c}
    \begin{align*}
          &\summation{\infty}{n=2}\ \frac{1}{n^{2} - 1} \\
          =&\summation{\infty}{n=2}\ \frac{1}{(n-1)(n+1)} 
    .\end{align*}
    By a partial fraction decomposition, we have
    \begin{align*}
        &\frac{1}{(n-1)(n+1)} = \frac{A}{(n-1)} + \frac{B}{(n+1)} \\
        &1 = A(n+1) + B(n-1) \\
    .\end{align*}
    \bigbreak \noindent 
    Thus, $A=\frac{1}{2}$, $B=-\frac{1}{2}$
    \begin{align*}
        1 = \frac{1/2}{n-1} - \frac{1/2}{(n+1)}
    .\end{align*}
    Writing out the first few terms we get 
    \begin{align*}
        \left(\frac{\frac{1}{2}}{1}-\frac{\frac{1}{2}}{3}\right) + \left(\frac{\frac{1}{2}}{2} - \frac{\frac{1}{2}}{4}\right)  +\left(\frac{\frac{1}{2}}{3} - \frac{\frac{1}{2}}{5}\right) + \left(\frac{\frac{1}{2}}{4} -\frac{\frac{1}{2}}{6}\right) + \left(\frac{\frac{1}{2}}{5} - \frac{\frac{1}{2}}{7}\right) + ... + \left(\frac{\frac{1}{2}}{n-1} - \frac{\frac{1}{2}}{n+1}\right)
    .\end{align*}
    Where most of these terms cancel 
    \begin{align*}
        \left(\frac{\frac{1}{2}}{1}-\cancel{\frac{\frac{1}{2}}{3}}\right) + \left(\frac{\frac{1}{2}}{2} - \cancel{\frac{\frac{1}{2}}{4}}\right)  +\left(\cancel{\frac{\frac{1}{2}}{3}} - \cancel{\frac{\frac{1}{2}}{5}}\right) + \left(\cancel{\frac{\frac{1}{2}}{4}} -\cancel{\frac{\frac{1}{2}}{6}}\right) + \left(\cancel{\frac{\frac{1}{2}}{5}} - \cancel{\frac{\frac{1}{2}}{7}}\right) + ... + \left(\cancel{\frac{\frac{1}{2}}{n-1}} - \frac{\frac{1}{2}}{n+1}\right)
    .\end{align*}
    \bigbreak \noindent 
    (We also have the right side of the $a_{n-1}$ term not having a cancellation), leaving:
    \begin{align*}
        &S_{n} = \frac{1}{2} - \frac{1}{4}- \frac{\frac{1}{2}}{n} - \frac{\frac{1}{2}}{n+1} \\
        \therefore&\lim\limits_{n \to \infty}{S_{n}} \implies \lim\limits_{n \to \infty}{\frac{1}{2}+\frac{1}{4}-\cancelto{0}{\frac{\frac{1}{2}}{n}}- \cancelto{0}{\frac{\frac{1}{2}}{n+1}}} \\
        &=\frac{3}{4}
    .\end{align*}

    \bigbreak \noindent 
    \textbf{Problem 1.d}
    \begin{align*}
        \summation{\infty}{n=1}\ (\sin{n} - \sin{n+1})\ 
    .\end{align*}
    \bigbreak \noindent 
    Writing out the first few terms we get:
    \begin{align*}
        (\sin{1} - \sin{2}) + (\sin{(2)} - \sin{(3)}) + (\sin{(3)} - \sin{(4)}) + ... + (\sin{n} - \sin{n+1})
    .\end{align*}
    \bigbreak \noindent 
    Where all terms cancel except
    \begin{align*}
        &\sin{1} - \sin{n+1}
    .\end{align*}
    \bigbreak \noindent 
    Thus,
    \begin{align*}
        &S_{n} = \sin{1} - \sin{(n+1)} \\
        &\therefore \lim\limits_{n \to \infty}{S_{n}} \implies \lim\limits_{n \to \infty}{\sin{1} - \sin{n+1}}\\
        &\text{Diverges}
    .\end{align*}

    \pagebreak \bigbreak \noindent 
    \begin{mdframed}
        2. Determine whether the series is convergent or divergent.
        \begin{enumerate}[label=(\alph*)]
            \item $\summation{\infty}{n=1}\ \frac{1}{n\sqrt{n}}\ $
            \item $\summation{\infty}{n=1}\ \  \frac{n^{e}}{n^{\pi}}$ 
            \item $\summation{\infty}{n=1}\ \ \cos{n} $
            \item $\summation{\infty}{n=1}\ \  \frac{1}{\sqrt{n+5}}$
            \item $\summation{\infty}{n=1}\ \  \frac{2n}{1+n^{4}}$
        \end{enumerate}
    \end{mdframed}
    \bigbreak \noindent 
    \begin{remark}
        Divergence test: For a series $a_{n}$, if $\lim\limits_{n \to \infty}{a_{n}} \ne 0$ or DNE, the series is said to diverge
        \smallbreak \noindent
        Integral test: For a series $a_{n}$ with positive terms, if there exists a function $f$ and a positive integer $N$ s.t
        \begin{enumerate}
            \item $f$ is positive, continuous, and decreasing on $[N, \infty)$
            \item $a_{n}= f(n)\ \forall n \geq N, N \in \mathbb{Z^{+}}$ 
        \end{enumerate}
        Then:
        \begin{align*}
            \summation{\infty}{n=N}\ a_{n}\ \text{ and } \int_{N}^{\infty}\ f(x)\ dx 
        .\end{align*}
        Either both converge or both diverge
        \smallbreak \noindent
        We also have the p-series, which states
        \begin{align*}
            \summation{\infty}{n=1}\ \frac{1}{n^{P}}\    
                =
                    \begin{cases}
                        \text{Converges} & \text{if } p > 1  \\
                        \text{Diverges} & \text{if }  p \leq 1
                    \end{cases}
        .\end{align*}
    Which can be extended to 
    \begin{align*}
        \summation{\infty}{n=2}\ \frac{1}{n\ln^{P}{n}}\ 
    .\end{align*}
    \end{remark}
    \bigbreak \noindent 
    \textbf{Problem 2.a}
    \begin{align*}
        &\summation{\infty}{n=1}\ \frac{1}{n\sqrt{n}}\  \\
        &=\summation{\infty}{n=1}\ \frac{1}{n^{\frac{3}{2}}}\ 
    .\end{align*}
    By the p-series, this series will converge. $P > 1$

    \pagebreak \bigbreak \noindent 
    \textbf{Problem 2.b}
    \begin{align*}
        &\summation{\infty}{n=1}\ \frac{n^{e}}{n^{\pi}}\  \\
        &=\summation{\infty}{n=1}\ n^{e-\pi}\  \\
        &=\summation{\infty}{n=1}\ \frac{1}{n^{\pi-e}}\ 
    .\end{align*}
    By the p-series, this series will diverge. $P \leq 1$

    \bigbreak \noindent 
    \textbf{Problem 2.c}
    \begin{align*}
        &\summation{\infty}{n=1}\ \cos{n}\ 
    .\end{align*}
    By the divergence test, we can conclude that this series diverges, as the $\lim\limits_{n \to \infty}{\cos{n}} $ DNE
    \bigbreak \noindent 
    \textbf{Problem 2.d}
    \begin{align*}
    &\sum_{n=1}^{\infty} \frac{1}{\sqrt{n+5}}
\end{align*}

\noindent
Since \(\lim_{n \to \infty} \frac{1}{\sqrt{n+5}} = 0\), the divergence test does not yield conclusive results. Furthermore, since this series has positive terms, we can compare the series to an integral of a function \(f(x)\) where \(a_n = f(n)\).

\noindent
Let \(f(x) = \frac{1}{\sqrt{x+5}}\), which is positive, continuous, and decreasing for all \(x \geq 1\). We can then examine the improper integral \(\int_{1}^{\infty} \frac{1}{\sqrt{x+5}}\ dx\):

\begin{align*}
    &\int_{1}^{\infty} \frac{1}{\sqrt{x+5}}\ dx \\
    &= \lim_{t \to \infty} \int_{1}^{t} \frac{1}{\sqrt{x+5}}\ dx \\
    &= \lim_{t \to \infty} \left[ 2\sqrt{x+5} \right]_{1}^{t} \\
    &= \lim_{t \to \infty} \left( 2\sqrt{t+5} - 2\sqrt{6} \right) \\
    &= +\infty
\end{align*}

\noindent
Since the improper integral diverges, by the integral test, the series also diverges.

    % \begin{align*}
    %     &\summation{\infty}{n=1}\ \frac{1}{\sqrt{n+5}}
    % .\end{align*}
    % \bigbreak \noindent 
    % Since $\lim\limits_{n \to \infty}{\frac{1}{\sqrt{n+5}}} = 0$, the divergence test does not yield conclusive results. Furthermore, since this series has positive terms, we can map the series over to some $f$ where $a_{n} = f(n)$
    % \bigbreak \noindent 
    % If we call $f(x) = \frac{1}{\sqrt{x+5}}$, where $f(x)$ is positive, continuous, and decreasing $\forall n \geq N$, $n \in \mathbb{Z^{+}} $, then we can examine the improper intgeral $\int_{1}^{\infty}\ \ dx$ 
    % \begin{align*}
    %     &\int_{1}^{\infty}\ \frac{1}{\sqrt{x+5}}\ dx \\
    %     &=\lim\limits_{t \to \infty}{\int_{1}^{t}\ \frac{1}{\sqrt{x+5}}\ dx} \\
    %     &= \lim\limits_{n \to \infty}{\int_{6}^{t+5}\ u^{-\frac{1}{2}}\ du} \\
    %     &=\lim\limits_{n \to \infty}{2u^{\frac{1}{2}}}\ \bigg|_6^{t+5} \\
    %     &=\lim\limits_{n \to \infty}{2\bigg[(t+5)^{\frac{1}{2}}-(6)^{\frac{1}{2}}\bigg]} \\
    %     &=+\infty
    % .\end{align*}
    % \bigbreak \noindent 
    % Since the improper integral diverges, the series also diverges
    % 

    \bigbreak \noindent 
    \textbf{Problem 2.e}
    \begin{align*}
        \summation{\infty}{n=1}\ \frac{2n}{1+n^{4}}\ 
    .\end{align*}
    \bigbreak \noindent 
    First, we check the divergence test
    \begin{align*}
        &\lim\limits_{n \to \infty}{\frac{2n}{1+n^{4}}} \\
        &=\lim\limits_{n \to \infty}{\frac{\frac{2n}{n^{4}}}{\frac{1}{n^{4}}+\frac{n^{4}}{n^{4}}}} \\
        &=\lim\limits_{n \to \infty}{\frac{\frac{2}{n^{3}}}{\frac{1}{n^{4}} + 1}} \\
        &= 0
    .\end{align*}
    Since the limit is zero, the divergence test does not yield conclusive results. For the integral test:
    \bigbreak \noindent 
    \begin{minipage}[t]{0.47\textwidth}
    \begin{align*}
        &\int_{1}^{\infty}\ \frac{2x}{1+x^{4}}\ dx \\
        &=\lim\limits_{t \to \infty}{\int_{1}^{t}\ \frac{2x}{1+x^{4}}\ dx} \\
        &=\lim\limits_{t \to \infty}{\int_{1}^{t^{2}}\ \frac{1}{1+u^{2}}\ du} \\
        &\lim\limits_{t \to \infty}{\tan^{-1}{u}}\ \bigg|_1^{t^{2}} \\
        &=\lim\limits_{t \to \infty}{\tan^{-1}{t^{2}}} - \tan^{-1}{1} \\
        &= \frac{\pi}{2} - \frac{\pi}{4} \\
        &=\frac{\pi}{4}
    .\end{align*}
    \end{minipage}
        \begin{minipage}[t]{0.47\textwidth}
        \begin{align*}
            &\text{Let $u=x^{2}$} \\
            &du = 2x\ dx \\
            &\text{when } x = 1,\ u= 1 \\
            &\text{when } x = t,\ u=t^{2} 
        .\end{align*}
    \end{minipage}
    \bigbreak \noindent 
    Therefore, Since the improper integral converges, by the integral test, the series also converges.


    



    
    
\end{document}
