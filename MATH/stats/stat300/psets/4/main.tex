 \documentclass{report}
 
 \input{~/dev/latex/template/preamble.tex}
 \input{~/dev/latex/template/macros.tex}
 
 \title{\Huge{}}
 \author{\huge{Nathan Warner}}
 \date{\huge{}}
 \fancyhf{}
 \rhead{}
 \fancyhead[R]{\itshape Warner} % Left header: Section name
 \fancyhead[L]{\itshape\leftmark}  % Right header: Page number
 \cfoot{\thepage}
 \renewcommand{\headrulewidth}{0pt} % Optional: Removes the header line
 %\pagestyle{fancy}
 %\fancyhf{}
 %\lhead{Warner \thepage}
 %\rhead{}
 % \lhead{\leftmark}
 %\cfoot{\thepage}
 %\setborder
 % \usepackage[default]{sourcecodepro}
 % \usepackage[T1]{fontenc}
 
 % Change the title
 \hypersetup{
     pdftitle={}
 }

 \geometry{
  left=1.5in,
  right=1.5in,
  top=1in,
  bottom=1in
}
 
 \begin{document}
     % \maketitle
     %     \begin{titlepage}
     %    \begin{center}
     %        \vspace*{1cm}
     % 
     %        \textbf{}
     % 
     %        \vspace{0.5cm}
     %         
     %             
     %        \vspace{1.5cm}
     % 
     %        \textbf{Nathan Warner}
     % 
     %        \vfill
     %             
     %             
     %        \vspace{0.8cm}
     %      
     %        \includegraphics[width=0.4\textwidth]{~/niu/seal.png}
     %             
     %        Computer Science \\
     %        Northern Illinois University\\
     %        United States\\
     %        
     %             
     %    \end{center}
     % \end{titlepage}
 % \tableofcontents
 \pagebreak \bigbreak \noindent
 Nate Warner \ \quad \quad \quad \quad \quad \quad \quad \quad \quad \quad \quad \quad  STAT 300 \quad  \quad \quad \quad \quad \quad \quad \quad \quad \ \ \quad Summer 2024
 \begin{center}
     \textbf{PSET 4 - Due: Wednesday, July 3}
 \end{center}
 \bigbreak \noindent 
 \begin{mdframed}
     1. An individual who has automobile insurance from a certain company is randomly selected. Let $X$ = the number of moving violations for which the individual was cited during the last 3 years. The probability mass function of $X$ is given below.

     \begin{center}
         \begin{tabular}{|c|c|c|c|c|c|}
             \hline
             $x$ & 0 & 1 & 2 & 3 & 4 \\
             \hline
             $p(x)$ & 0.50 & 0.20 & 0.15 & 0.10 & 0.05 \\
             \hline
         \end{tabular}
     \end{center}
     \bigbreak \noindent 
     (a) Calculate the probability of each of the following events.
     \begin{itemize}
         \item[(i)] Exactly one moving violation
         \item[(ii)] At most one moving violation
         \item[(iii)] More than two moving violations
         \item[(iv)] Between 1 and 3 (inclusive of the endpoints) moving violations
     \end{itemize}
     \bigbreak \noindent 
     (b) Find the cumulative distribution function $F(x)$. Be sure to write your answer in the appropriate way.
 \end{mdframed}
 \bigbreak \noindent 
 The probability of one moving violation is
 \begin{align*}
     p(1) = 0.2
 .\end{align*}
 \bigbreak \noindent 
 The probability of at most one moving violation is the sum of $p(0)$ and $p(1)$
 \begin{align*}
     p(0) + p(1) = 0.5 + 0.2 = 0.7
 .\end{align*}
 \bigbreak \noindent 
 The probability of more than two moving violations is the sum of the following probabilities
 \begin{align*}
     p(3) + p(4) = 0.1 + 0.05 = 0.15
 .\end{align*}
 \bigbreak \noindent 
 The sum of between one and three (inclusive) moving violations is 
 \begin{align*}
     p(1) + p(2) + p(3) = 0.2 + 0.15 + 0.1 = 0.45
 .\end{align*}
 \bigbreak \noindent 
 \begin{remark}
          The \textbf{cumulative distribution function (cdf)} \( F(x) \) of a discrete rv variable \( X \) with pmf \( p(x) \) is defined for every number \( x \) by
        \[
            F(x) = P(X \leq x) = \sum_{y: y \leq x} p(y) \tag{3.3}
        \]
        For any number \( x \), \( F(x) \) is the probability that the observed value of \( X \) will be at most \( x \).
 \end{remark}
 \bigbreak \noindent 
 To find the cdf, we first find $F(x)$ for each value of $x$ in the above table.
 \begin{align*}
     F(0) &= P(X \leq 0) = p(0) = 0.5 \\
     F(1)  &= P(X \leq 1) = p(0) + p(1) = 0.5 + 0.2 = 0.7\\
     F(2)  &= P(X \leq 2) = p(0) + p(1) + p(2) = 0.5 + 0.2 + 0.15 = 0.85\\
     F(3)  &= P(X \leq 3) = p(0) + p(1) + p(2) + p(3) = 0.5 + 0.2 + 0.15 + 0.1 = 0.95\\
     F(4)  &= P(X \leq 4) = 1
 .\end{align*}
 \bigbreak \noindent 
 Thus, the cdf is given by
 \begin{align*}
     F(x) = \begin{cases}
         0 & \text{if } x < 0 \\
        0.5 & \text{if } 0 \leq x < 1 \\
        0.7 & \text{if }  1 \leq x < 2 \\
        0.85 & \text{if } 2 \leq x < 3 \\
        0.95 & \text{if } 3 \leq x < 4 \\
        1 & \text{if }  x \geq 4  
     \end{cases}
 .\end{align*}
 \bigbreak \noindent 
 A graph of the cdf is shown below
\begin{figure}[ht]
    \centering
    \incfig{cdf}
    \label{fig:cdf}
\end{figure}
 

 \pagebreak \bigbreak \noindent 
 \begin{mdframed}
     2. Suppose that two fair, six-sided dice are rolled independently of each other. For each possible roll define $X$ = the smaller of the two dice. Find the probability mass function of $X$. Write your answer using a table similar to that given in Problem 1.
 \end{mdframed}
 \bigbreak \noindent 
 First, we need to find the number of outcomes for each possible value of $x$. For this, we consider orderded pairs. For $x=1$, we need at least one of the die to show a one. We have
 \begin{align*}
     &(1, \lambda) \\ 
     &(\lambda, 1)
 .\end{align*}
 \bigbreak \noindent 
 In the first case, $\lambda$ can be any number from 1 to 6. Thus we have $6P1 = \frac{6!}{5!} = 6$ possibilities. For the second case, $\lambda$ can range from 1 to 5 (so we dont double count (1,1)). Thus we have $5P1 = 5$ possibilities. In total, we have 11 favorable outcomes for $x=1$. Similarly, when
 \bigbreak \noindent 
 \begin{align*}
     x=2 &\implies 5P1 + 4P1 = 9 \quad \text{favorable outcomes}\\
     x=3 &\implies 4P1 + 3P1 = 7 \quad \text{favorable outcomes}\\
     x=4 &\implies 3P1 + 2P1 = 5 \quad \text{favorable outcomes}\\
     x=5 &\implies 2P1 + 1P1 = 3 \quad \text{favorable outcomes}\\
     x=6 &\implies 1P1 = 1 \quad \text{favorable outcomes}
 .\end{align*}
 \bigbreak \noindent 
 If the total number of possible outcomes after rolling both dice is $n^{k} = 6^{2} = 36$, then we can find the probabilities for each value of x. For example when $x=1$ we have $p(x=1) = \frac{11}{36} = 0.3056$. With these computations we build the following pmf
 \bigbreak \noindent 
 \begin{center}
     \begin{tabular}{|c|c|c|c|c|c|c|}
         \hline 
         $x$ & 1 & 2 & 3 & 4 & 5 & 6\\ 
         \hline
         $p(x)$ & 0.3056 & 0.25 & 0.1944 & 0.1389 & 0.0833 & 0.0278 \\
         \hline
     \end{tabular}
 \end{center}

 \pagebreak \bigbreak \noindent 
 \begin{mdframed}
     3. Let $X$ be a discrete random variable having the following cumulative distribution function (cdf).
     \[
         F(x) = \begin{cases} 
             0.00 & \text{if } x < 1 \\
             0.05 & \text{if } 1 \leq x < 3 \\
             0.10 & \text{if } 3 \leq x < 5 \\
             0.25 & \text{if } 5 \leq x < 6 \\
             0.65 & \text{if } 6 \leq x < 8 \\
             0.90 & \text{if } 8 \leq x < 9 \\
             1.00 & \text{if } 9 \leq x 
         \end{cases}
     \]
     \bigbreak \noindent 
     (a) Graph the cdf. It should be neat, accurate and well-labeled.
     \bigbreak \noindent 
     (b) Using just the cdf calculate the following probabilities. (Your work should clearly show how you are using $F(x)$ to find these.)
     \begin{itemize}
         \item[(i)] $P(X \leq 3)$
         \item[(ii)] $P(X \geq 6)$
         \item[(iii)] $P(X = 5)$
         \item[(iv)] $P(3 \leq X \leq 8)$
     \end{itemize}
     \bigbreak \noindent 
     (c) Find the probability mass function. Write your answer using a table similar to that given in Problem 1.
 \end{mdframed}
 \bigbreak \noindent 
 a.) The graph of the cdf is shown below
 \bigbreak \noindent 
    \begin{figure}[ht]
        \centering
        \incfig{cdf2}
        \label{fig:cdf2}
    \end{figure}
    \bigbreak \noindent 
    The possible $x$ values are $1,3,5,6,8,9$. Using the cdf, we see
    \begin{align*}
        F(1) &= 0.05 \\
        F(3) &= 0.1 \\
        F(5) &= 0.25 \\
        F(6) &= 0.65 \\
        F(8) &= 0.9 \\
        F(9) &= 1 
    .\end{align*}
    \bigbreak \noindent 
    b.)\bigbreak \noindent 
    (i) To find $P(X \leq 3)$, we simply use $F(3) = 0.1$.
    \bigbreak \noindent 
    (ii) To find $P(X \geq 6)$. We need $p(6) + p(8) + p(9)$. To obtain this from the cdf, we use
    \begin{align*}
        F(9) - F(5) &= p(1) + p(3) + p(5) + p(6) + p(8) + p(9) - (p(1) + p(3) + p(5)) \\
                    &=p(6) + p(8) + p(9) 
    .\end{align*}
    Thus, $F(9) - F(5)  = 1- 0.25 = 0.75$. We could also use $1-F(5) = 1-0.25 = 0.75$ 
    \bigbreak \noindent 
    (iii) Similarly, to find $P(x=5)$, we use $F(5) - F(3) = 0.25 - 0.1 = 0.15$
    \bigbreak \noindent 
    (iv) Finally, to find $P(3 \leq X \leq 8)$, we use $F(8) - F(1) = 0.85$
    \bigbreak \noindent 
    c.) To obtain the pmf from the cdf, we remark
    \bigbreak \noindent 
    \begin{remark}
        For any two numbers \(a\) and \(b\) with \(a \leq b\),
        \[
            P(a \leq X \leq b) = F(b) - F(a-)
        \]
        where “\(a-\)” represents the largest possible \(X\) value that is strictly less than \(a\). In particular, if the only possible values are integers and if \(a\) and \(b\) are integers, then
        \[
            P(a \leq X \leq b) = P(X = a \text{ or } a + 1 \text{ or } \ldots \text{ or } b) = F(b) - F(a-1)
        \]
        Taking \(a = b\) yields \(P(X = a) = F(a) - F(a-1)\) in this case.
        \bigbreak \noindent 
    \end{remark}
    \bigbreak \noindent 
    Thus, to find each $p(x)$, we simply take $F(x) - F(x-)$ for each value of $x$. We find
    \begin{align*}
        p(1) &= F(1) - F(0) = 0.05 - 0 = 0.05 \\
        p(3) &= F(3) - F(1) = 0.1 - 0.05 = 0.05 \\
        p(5) &= F(5) - F(3) = 0.25 - 0.1 = 0.15 \\
        p(6) &= F(6) - F(8) = 0.65 - 0.25 = 0.4 \\
        p(8) &= F(8) - F(6) = 0.9 - 0.65 = 0.25 \\
        p(9) &= F(9) - F(8) = 1-0.9 = 0.1
    .\end{align*}
    \bigbreak \noindent 
    Thus, we have the pmf given by the following table.
    \bigbreak \noindent 
    \begin{center}
        \begin{tabular}{c|cccccc}
            x & 1 & 3 & 5 & 6 & 8 & 9 \\ 
            \hline
            p(x) & 0.05 & 0.05 & 0.15 & 0.4 & 0.25 & 0.1
        \end{tabular}
    \end{center}
    




 \end{document} % (:
