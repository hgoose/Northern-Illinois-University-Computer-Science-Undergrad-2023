 \documentclass{report}
 
 \input{~/dev/latex/template/preamble.tex}
 \input{~/dev/latex/template/macros.tex}
 
 \title{\Huge{}}
 \author{\huge{Nathan Warner}}
 \date{\huge{}}
 \fancyhf{}
 \rhead{}
 \fancyhead[R]{\itshape Warner} % Left header: Section name
 \fancyhead[L]{\itshape\leftmark}  % Right header: Page number
 \cfoot{\thepage}
 \renewcommand{\headrulewidth}{0pt} % Optional: Removes the header line
 %\pagestyle{fancy}
 %\fancyhf{}
 %\lhead{Warner \thepage}
 %\rhead{}
 % \lhead{\leftmark}
 %\cfoot{\thepage}
 %\setborder
 % \usepackage[default]{sourcecodepro}
 % \usepackage[T1]{fontenc}
 
 % Change the title
 \hypersetup{
     pdftitle={}
 }

 \geometry{
  left=1.5in,
  right=1.5in,
  top=1in,
  bottom=1in
}
 
 \begin{document}
     % \maketitle
     %     \begin{titlepage}
     %    \begin{center}
     %        \vspace*{1cm}
     % 
     %        \textbf{}
     % 
     %        \vspace{0.5cm}
     %         
     %             
     %        \vspace{1.5cm}
     % 
     %        \textbf{Nathan Warner}
     % 
     %        \vfill
     %             
     %             
     %        \vspace{0.8cm}
     %      
     %        \includegraphics[width=0.4\textwidth]{~/niu/seal.png}
     %             
     %        Computer Science \\
     %        Northern Illinois University\\
     %        United States\\
     %        
     %             
     %    \end{center}
     % \end{titlepage}
     % \tableofcontents
    \pagebreak \bigbreak \noindent
    Nate Warner \ \quad \quad \quad \quad \quad \quad \quad \quad \quad \quad \quad \quad  STAT 300 \quad  \quad \quad \quad \quad \quad \quad \quad \quad \ \ \quad Summer 2024
    \begin{center}
        \textbf{PSET 2 - Due: Wednesday, June 26}
    \end{center}
    \bigbreak \noindent 
    
    \begin{mdframed}
        Consider an experiment with the sample space \( S = \{0, 1, 2, 3, 4, 5, 6, 7, 8, 9\} \) and the events \( A = \{0, 1, 2, 3\} \), \( B = \{2, 3, 4, 5, 6\} \), \( C = \{7, 8\} \), and \( D = \{1, 3, 7\} \). Find each of the following events. 
        \begin{enumerate}[label=(\alph*)]
            \item $A^{C}$
            \item $B^{C} $
            \item $A\cup B $
            \item $(A\cup B)^{C} $
            \item $A^{C} \cap B^{C} $
            \item  $B \cap C $
            \item $C \cap D^{C} $
            \item  $A\cup B \cup C $
            \item  $A\cap B \cap D $
            \item Compare parts (d) and (e). What do you notice?
        \end{enumerate}
    \end{mdframed}
    \bigbreak \noindent 
    \begin{enumerate}[label=(\alph*)]
        \item $\{4,5,6,7,8,9\} $
        \item $\{0,1,7,8,9\} $
        \item $\{0,1,2,3,4,5,6\} $
        \item $\{7,8,9\} $
        \item $\{7,8,9\} $
        \item $\varnothing$
        \item $\{8\} $
        \item $\{0,1,2,3,4,5,6,7,8\} $
        \item $\{3\} $
        \item They are the same (De Morgan's law)
    \end{enumerate}


    \pagebreak \bigbreak \noindent 
    \begin{mdframed}
        A mutual fund company offers its customers a variety of funds. Among customers who own shares in just one fund, the percentages of customers in the different funds are given below.
        \begin{itemize}
            \item Money-market 20\%
            \item High-risk stock 18\%
            \item Short bond 15\%
            \item Moderate-risk stock 25\%
            \item Intermediate bond 10\%
            \item Balanced ??\%
            \item Long bond 5\%
        \end{itemize}
        Suppose that a customer who owns shares in just one fund is selected at random. Find each of the following probabilities. 
        \begin{enumerate}[label=(\alph*)]
            \item The individual owns shares in the balanced fund.
            \item The individual owns shares in a bond fund.
            \item The individual does not own shares in a stock fund.
        \end{enumerate}
    \end{mdframed}
    \bigbreak \noindent 
    a.) To find the missing probability (Balanced fund), we use the following axiom.
    \begin{align*}
        P(\mathcal{S}) &= \sum P(E) = 1 
    .\end{align*}
    \bigbreak \noindent 
    If we denote the proportion of customers who own shares in the balanced fund $\lambda$, we can use the above axiom to solve for it
    \begin{align*}
        1 &= 0.2 + 0.18 + 0.15 + 0.25 + 0.1 + \lambda + 0.05 \\
          \lamda &= 1 - .2 -.18 -.15 -.25 -.1 -.05 \\
          \lambda &=0.07 \\
          \implies P(\lambda) &= P(\text{balanced}) = 7\%
    .\end{align*}
    \bigbreak \noindent 
    Thus, the probability that a customer selected randomly is in the balanced fund is 7\%
    \bigbreak \noindent 
    b.) The probability the individual owns shahres in a bond fund is the summation of the probabilities  of the three bond funds. Thus,
    \begin{align*}
        &P(\text{Long bond or Short bond or Intermediate bond})  \\
        &= P(\text{Long bond}) + P(\text{Short bond}) + P(\text{Intermediate bond}) \\
        &=0.05 + 0.1 + 0.15 \\
        &=0.3 = 30\%
    .\end{align*}
    \pagebreak \bigbreak \noindent 
    c.) The probability that the individual does not own share in a stock fund is the complement of the probability that the individual does own shares in a stock fund. That is 
    \begin{align*}
        &P((\text{Stock fund})^{C})  \\
        &= 1-P(\text{High-risk stock or Moderate-risk stock)} \\
        &=1 - (P(\text{High-risk stock} + P(\text{Moderate-risk stock})  \\
        &= 1- (0.18 + 0.25) \\
        &=0.57 = 57\%
    .\end{align*}


    \pagebreak \bigbreak \noindent 
    \begin{mdframed}
        The three most popular options on a certain type of new car at a dealership are a built-in GPS (\( A \)), a sunroof (\( B \)), and an automatic transmission (\( C \)). Suppose that \( P(A) = 0.40 \), \( P(B) = 0.55 \), \( P(C) = 0.70 \), \( P(A \cup B) = 0.63 \), and \( P(B \cap C) = 0.45 \). Suppose that a customer at the dealership is selected at random. Find the probability of each of the following events.
        \begin{enumerate}[label=(\alph*)]
            \item The customer wants a built-in GPS and a sunroof. 
            \item The customer wants a sunroof or an automatic transmission. 
            \item The customer does not want a sunroof. 
            \item Consider the event “the customer wants neither a built-in GPS nor a sunroof”.
                \begin{enumerate}[label=(\roman*)]
                    \item Write the event in symbols (i.e. using \( A \), \( B \), \( C \), \( \cup \), \( \cap \), etc.). 
                    \item Find the probability of the event. 
                \end{enumerate}
            \item Consider the event “the customer does not want a sunroof but does want an automatic transmission”.
                \begin{enumerate}[label=(\roman*)]
                    \item Write the event in symbols (i.e. using \( A \), \( B \), \( C \), \( \cup \), \( \cap \), etc.). 
                    \item Find the probability of the event. 
                \end{enumerate}
        \end{enumerate}
    \end{mdframed}
    \bigbreak \noindent 
    \textbf{Note:} These events are independent, we handle $P(E_{1} \cup E_{2} \cup ...\cup E_{n})$ accordingly
    \bigbreak \noindent 
    \begin{remark}
        Given two independent events $A$ and $B$, $P(A \cup B) = P(A) + P(B) - P(A\cap B)$ 
    \end{remark}
    \bigbreak \noindent 
    a.) 
    \begin{align*}
        P(A \cap B) &= P(A) \cdot  P(B) \\
        &= 0.4(0.55) = 0.22 = 22\%
    .\end{align*}
    
    \bigbreak \noindent 
    b.) 
    \begin{align*} 
        P(B \cup C) &= P(B) + P(C) - P(B \cap C) \\
                    &= 0.55 + 0.7 -  0.45\\
                    &=0.80 = 80\%
    .\end{align*}
    \bigbreak \noindent 
    c.) 
    \begin{align*}
        P(B^{\prime}) &= 1-P(B) \\
        &=1-0.55  \\
        &=0.45 =45\%
    .\end{align*}
    \bigbreak \noindent 
    d.)
    \begin{align*}
        P(A^{\prime} \cap B^{\prime})  &= (1-P(A))(1-P(B)) \\
      &=(1-0.4)(1-0.55) \\
      &=0.27
    .\end{align*}
    \bigbreak \noindent 
    e.)
    \begin{align*}
        P(B^{\prime} \cap C) &= (1-P(B))(P(C)) \\
                             &= (1-0.55)(0.7) \\
                             &=0.315 = 31.5\%
    .\end{align*}

 \end{document} % (:
