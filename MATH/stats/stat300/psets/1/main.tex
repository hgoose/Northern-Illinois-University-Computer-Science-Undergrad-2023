 \documentclass{report}
 
 \input{~/dev/latex/template/preamble.tex}
 \input{~/dev/latex/template/macros.tex}
 
 \title{\Huge{}}
 \author{\huge{Nathan Warner}}
 \date{\huge{}}
 \fancyhf{}
 \rhead{}
 \fancyhead[R]{\itshape Warner} % Left header: Section name
 \fancyhead[L]{\itshape\leftmark}  % Right header: Page number
 \cfoot{\thepage}
 \renewcommand{\headrulewidth}{0pt} % Optional: Removes the header line
 %\pagestyle{fancy}
 %\fancyhf{}
 %\lhead{Warner \thepage}
 %\rhead{}
 % \lhead{\leftmark}
 %\cfoot{\thepage}
 %\setborder
 % \usepackage[default]{sourcecodepro}
 % \usepackage[T1]{fontenc}
 
 % Change the title
 \hypersetup{
     pdftitle={}
 }

 \geometry{
  left=1.5in,
  right=1.5in,
  top=1in,
  bottom=1in
}
 
 \begin{document}
     % \maketitle
     %     \begin{titlepage}
     %    \begin{center}
     %        \vspace*{1cm}
     % 
     %        \textbf{}
     % 
     %        \vspace{0.5cm}
     %         
     %             
     %        \vspace{1.5cm}
     % 
     %        \textbf{Nathan Warner}
     % 
     %        \vfill
     %             
     %             
     %        \vspace{0.8cm}
     %      
     %        \includegraphics[width=0.4\textwidth]{~/niu/seal.png}
     %             
     %        Computer Science \\
     %        Northern Illinois University\\
     %        United States\\
     %        
     %             
     %    \end{center}
     % \end{titlepage}
     % \tableofcontents
    \pagebreak \bigbreak \noindent
    Nate Warner \ \quad \quad \quad \quad \quad \quad \quad \quad \quad \quad \quad \quad  STAT 300 \quad  \quad \quad \quad \quad \quad \quad \quad \quad \ \ \quad Summer 2024
    \begin{center}
        \textbf{PSET 1 - Due: Sunday, June 23}
    \end{center}
    \bigbreak \noindent 
    \begin{mdframed}
        1. A small survey was conducted in which each respondent was asked how many times,
        in the previous two-week period, they had eaten at a fast food restaurant. The data appear below
        \bigbreak \noindent 
        \begin{center}
            0, 2, 1, 5, 2, 2, 3, 4, 1, 2, 7, 1, 3, 4, 1, 0, 1, 4, 2, 1, 3, 3, 2, 1, 9, 1
        \end{center}
        \begin{enumerate}[label=(\alph*)]
            \item Construct a frequency histogram. The histogram should be neat, accurate, and well labeled. 
            \item How would you describe the shape the distribution? 
            \item Find the proportion of the respondents described by each of the following. 
                \begin{enumerate}[label=(\roman*)]
                    \item Ate at a fast food restaurant at least four times
                    \item Ate at a fast food restaurant fewer than two times
                \end{enumerate}
        \end{enumerate}
    \end{mdframed}
    \bigbreak \noindent 
    Given the above data set, we can first find the frequencies of each data point. We can then construct a frequency distribution to get a better look at the data
    \bigbreak \noindent 
    \begin{center}
        \begin{tabular}{c|c}
            Number of days &  Frequency \\
            \hline
            0 & 2 \\
            1 & 8\\
            2 & 6 \\
            3 & 4\\
            4 & 3\\
            5 & 1\\
            6 & 0\\
            7 & 1\\
            8 & 0\\ 
            9 & 1
        \end{tabular}
    \end{center}
    \bigbreak \noindent 
    \tc{Number of times $n=26$ people ate a fast food restaurant in the previous two weeks organized as a frequency distribution}
    \pagebreak \bigbreak \noindent 
    From this, we create the histogram
    \bigbreak \noindent 
\begin{figure}[ht]
    \centering
    \incfig{histo22}
    \label{fig:histo22}
\end{figure}
    \fc{Number of times $n=26$ people ate a fast food restaurant in the previous two weeks organized as a histogram for discrete data}

\bigbreak \noindent 
b.) From the histogram, we can see that the shape of the distribution is skewed positively (right)
\bigbreak \noindent 
c.) Furthermore, we can find the relative frequency (proportion) of respondents who ate at a fast food restaurant at least four times by dividing the summation of the frequencies $x\in\{4,5,6,...,9\} $ by $n=26 $ (the total number of respondents). We find
\begin{align*}
    \frac{3 + 1 + 0 + 1 + 0 + 1}{26}  = 0.2308 \approx  23\%
.\end{align*}
\bigbreak \noindent 
We preform a similar calculation to find the number of respondents who ate at a fast food restaurant fewer than two times, in this case we get
\begin{align*}
    \frac{2 + 8}{26} =  0.3846 \approx 38\%
.\end{align*}






    \pagebreak 
    \begin{mdframed}
        2. The blood glucose levels (in milligrams per deciliter) for 25 patients at a medical facility appear below.
        \bigbreak \noindent 
        \begin{center}
            &63 65 66 68 69 71 73 74 75 75 76 76 77 \\
            &79 79 81 81 81 83 84 86 87 90 91 95
        \end{center}
        \bigbreak \noindent 
        \begin{enumerate}[label=(\alph*)]
            \item  Construct a relative frequency histogram. Use nine class intervals starting with 60 $\leq$ Glucose < 64. 
            \item How would you describe the shape of the distribution?
            \item What proportion of the patients had a blood glucose level described by each of the following? 
                \begin{enumerate}[label=(\roman*)]
                    \item At most 71
                    \item  Between 72 and 91 (inclusive of the endpoints)
                \end{enumerate}
        \end{enumerate}
    \end{mdframed}
    \bigbreak \noindent 
    We again begin by organizing the data via a frequency distribution
    \bigbreak \noindent 
    \begin{center}
        \begin{tabular}{c|c|c}
            Blood glucose levels $(mg/dL)$ & Frequecy & Relative frequency \\
            \hline
            60-64 & 1 &0.04\\
            64-68 & 2 &0.08\\
            68-72 & 3 &0.12\\
            72-76 & 4 &0.16\\
            76-80 & 5 &0.2\\
            80-84 & 4 &0.16\\
            84-88 & 3 &0.12\\
            88-92 & 2 &0.08\\
            92-96 & 1 & 0.04
        \end{tabular}
    \end{center}
    \tc{Blood glucose levels for $n=25$ patiens at a medical facility organized as a frequency distribution}
    \pagebreak \bigbreak \noindent 
    Using the relative frequencies found above, we construct a simple continuous data, equal class width histogram 
    \bigbreak \noindent 
\begin{figure}[ht]
    \centering
    \incfig{histo23}
    \label{fig:histo23}
\end{figure}
\bigbreak \noindent 
b.) The shape of this distribution is approximately normal (symmetric)


    \pagebreak 
    \begin{mdframed}
        3. Refer to Problem 1.
        \begin{enumerate}[label=(\alph*)]
            \item  Calculate the sample mean. 
            \item Calculate the sample median. 
            \item How do the answers to (a) and (b) compare? 
                \begin{enumerate}[label=(\roman*)]
                    \item Why should we have been able to anticipate this?  
                \end{enumerate}
        \end{enumerate}
    \end{mdframed}

    \pagebreak 
    \begin{mdframed}
        4. A new type of outdoor paint was tested by painting six homes in the same geographic
        area. The number of months the paint lasted before fading was recorded and yielded the values
        below.
        \bigbreak \noindent 
        \begin{center}
            10, 60, 50, 30, 40, 20
        \end{center}
        \begin{enumerate}[label=(\alph*)]
            \item Calculate the sample range. 
            \item Calculate the sample mean. 
            \item Calculate the sample variance using the definition formula (i.e. by using the squared
            \item deviations). 
            \item Calculate the sample variance using the short-cut formula. 
            \item Calculate the sample standard deviation. 
            \item Give the units of measurement for (b); for (c); and for (e). 
        \end{enumerate}
    \end{mdframed}

    \pagebreak 
    \begin{mdframed}
       5.  Refer to Problem 4.
       \begin{enumerate}[label=(\alph*)]
           \item Add 10 to each data value in Problem 4.
           \begin{enumerate}[label=(\roman*)]
               \item Re-calculate the sample mean. 
               \item How does the answer in (i) compare to that in 4(b)? Be specific. 
               \item Re-calculate the sample variance. Note – using the definition formula might be more illuminating. 
               \item How does the answer in (iii) compare to that in 4(c)? Be specific. 
           \end{enumerate}
       \item Divide each data value in Problem 4 by 10 (i.e. multiply each data value by 1/10).
           \begin{enumerate}[label=(\roman*)]
               \item  Re-calculate the sample mean. 
               \item How does the answer in (i) compare to that in 4(b)? Be specific. 
               \item Re-calculate the sample variance. Note – using the definition formula might be more illuminating. 
               \item How does the answer in (iii) compare to that in 4(c)? Be specific. 
           \end{enumerate}
       \end{enumerate}
    \end{mdframed}

    \pagebreak 
    \begin{mdframed}
        6. Suppose that $x_{1}, x_{2}, ..., x_{n}$ is a set of data with $\bar{x}= 40$, $\tilde{x} = 60$, $s_{x}^{2} = 25$ and $s_{x} = 25 $
        Suppose that each $x_{i}$
        is transformed to a new data value $y_{i} = 2x_{i} - 1$
        \begin{enumerate}[label=(\alph*)]
            \item Compare $\bar{x}$ to $\tilde{x}$. What do the values potentially tell us about the shape of the distribution? 
            \item Find each of the following
                \begin{enumerate}[label=(\roman*)]
                    \item $\bar{y}$ 
                    \item $\tilde{y}$
                    \item $s_{y}^{2}$
                    \item $s_{y}$
                \end{enumerate}
        \end{enumerate}
    \end{mdframed}


 \end{document} % (:
