 \documentclass{report}
 
 \input{~/dev/latex/template/preamble.tex}
 \input{~/dev/latex/template/macros.tex}
 
 \title{\Huge{}}
 \author{\huge{Nathan Warner}}
 \date{\huge{}}
 \fancyhf{}
 \rhead{}
 \fancyhead[R]{\itshape Warner} % Left header: Section name
 \fancyhead[L]{\itshape\leftmark}  % Right header: Page number
 \cfoot{\thepage}
 \renewcommand{\headrulewidth}{0pt} % Optional: Removes the header line
 %\pagestyle{fancy}
 %\fancyhf{}
 %\lhead{Warner \thepage}
 %\rhead{}
 % \lhead{\leftmark}
 %\cfoot{\thepage}
 %\setborder
 % \usepackage[default]{sourcecodepro}
 % \usepackage[T1]{fontenc}
 
 % Change the title
 \hypersetup{
     pdftitle={}
 }

 \geometry{
  left=1.5in,
  right=1.5in,
  top=1in,
  bottom=1in
}
 
 \begin{document}
     % \maketitle
     %     \begin{titlepage}
     %    \begin{center}
     %        \vspace*{1cm}
     % 
     %        \textbf{}
     % 
     %        \vspace{0.5cm}
     %         
     %             
     %        \vspace{1.5cm}
     % 
     %        \textbf{Nathan Warner}
     % 
     %        \vfill
     %             
     %             
 %        \vspace{0.8cm}
 %      
 %        \includegraphics[width=0.4\textwidth]{~/niu/seal.png}
 %             
 %        Computer Science \\
 %        Northern Illinois University\\
 %        United States\\
 %        
 %             
 %    \end{center}
 % \end{titlepage}
 % \tableofcontents
 \pagebreak \bigbreak \noindent
 Nate Warner \ \quad \quad \quad \quad \quad \quad \quad \quad \quad \quad \quad \quad  STAT 300 \quad  \quad \quad \quad \quad \quad \quad \quad \quad \ \ \quad Summer 2024
 \begin{center}
     \textbf{PSET 7 - Due: Sunday, July 17}
 \end{center}
 \bigbreak \noindent 
 \begin{mdframed}
     1. Although we all experience inflammation associated with a bruise or sprain, inflammation can also affect the cells in the body. C-reactive protein (CRP) is a measure of inflammation and can be part of a routine blood test. Let $X$ = the CRP level in healthy adults and suppose that $X$ varies according to the probability density function (pdf) given below.

     \[
         f(x) = 
         \begin{cases} 
             -\frac{2}{25}x + \frac{2}{5} & \text{if } 0 \leq x \leq 5 \\ 
             0 & \text{otherwise} 
         \end{cases}
     \]

     \begin{enumerate}
         \item[(a)] 
             \begin{enumerate}
                 \item[(i)] Sketch a graph of the pdf. It should be neat, accurate and well-labeled.
                 \item[(ii)] Verify that the total area under the pdf equals 1.
             \end{enumerate}
         \item[(b)] 
             \begin{enumerate}
                 \item[(i)] Find $P(X < 2.5)$.
                 \item[(ii)] Find $P(X \leq 2.5)$.
                 \item[(iii)] Compare your answers to (i) and (ii).
             \end{enumerate}
         \item[(c)] Find the probability that a randomly selected healthy adult will have a CRP level between 2 and 3.
         \item[(d)] If a patient has a CRP level of at least 4, then additional testing is done. Find the probability that a healthy adult will need additional testing.
     \end{enumerate}
 \end{mdframed}
 \bigbreak \noindent 
 \begin{remark}
     Let $X$ be a continuous random variable (rv). Then a \textbf{probability distribution} or \textbf{probability density function} (pdf) of $X$ is a function $f(x)$ such that for any two numbers $a$ and $b$ with $a \leq b$,
     \[
         P(a \leq X \leq b) = \int_{a}^{b} f(x) \, dx
     \]
     That is, the probability that $X$ takes on a value in the interval $[a, b]$ is the area above this interval and under the graph of the density function, as illustrated in Figure 4.2. The graph of $f(x)$ is often referred to as the \textbf{density curve}. 
     For $f(x)$ to be a legitimate pdf, it must satisfy the following two conditions:
     \begin{enumerate}
         \item $f(x) \geq 0$ for all $x$
         \item $\int_{-\infty}^{\infty} f(x) \, dx = \text{area under the entire graph of } f(x) = 1$
     \end{enumerate} \smiley{}

 \end{remark}

 \pagebreak \bigbreak \noindent 
 a.) The graph of the pdf is the following
 \bigbreak \noindent 
 \fig{.5}{./figures/1.png}
 \bigbreak \noindent 
 To verify the area under the graph is one, we can integrate the function over its domain.
 \begin{align*}
     &\int_{0}^{5} -\frac{2}{25}x+\frac{2}{5} \, dx \\
     &=-\frac{1}{25}x^{2}+\frac{2}{5}x \bigg|^{5}_{0} \\
     &= -\frac{1}{25}(25)+\frac{2}{5}(5) - 0 \\
     &= -1 + 2 = 1
 .\end{align*}
 \bigbreak \noindent 
 b.i) To find $P(X < 2.5)$, we integrate 
 \begin{align*}
     &\int_{0}^{\frac{5}{2}}-\frac{2}{25}x^{2} + \frac{2}{5}  \, dx \\
     &=-\frac{1}{25}x^{2} + \frac{2}{5}x \bigg|^{\frac{5}{2}}_{0} \\
     &= -\frac{1}{25}\left(\frac{5}{2}\right)^{2} + \frac{2}{5}\left(\frac{5}{2}\right) \\
     &=-\frac{1}{4} + 1 = \frac{3}{4} = 0.75
 .\end{align*}
 b.ii) By properties of continous random variables, $P(X < 2.5) = P(X \leq 2.5) = 0.75$ 
 \bigbreak \noindent 
 c.) To find the probability that a randomly selected healthy adult will have a CRP level between 2 and 3, we integrate the function from $x=2$ to $x=3$
 \begin{align*}
     &\int_{2}^{3} -\frac{2}{25}x + \frac{2}{5} \, dx \\
     &=-\frac{1}{25}x^{2} + \frac{2}{5}x \bigg|^{3}_{2} \\
     &=-\frac{1}{25}(3)^{2} + \frac{2}{5}(3) - (-\frac{1}{25}(2)^{2}+ \frac{2}{5}(2)) \\
     &=0.84 - 0.64 = 0.2
 .\end{align*}
 \bigbreak \noindent 
 d.) The probability that a healthy adult will need additional testing is $P(X \geq 4)$, given by the integral
 \begin{align*}
     &\int_{4}^{5} -\frac{2}{25}x + \frac{2}{5} \, dx \\
     &=-\frac{1}{25}x^{2} + \frac{2}{5} \bigg|^{5}_{4} \\
     &=-\frac{1}{25}(5)^{2} + \frac{2}{5}(5) - (-\frac{1}{25}(4)^{2} + \frac{2}{5}(4)) \\
     &=1-0.96 = 0.04
 .\end{align*}

 \pagebreak \bigbreak \noindent 
 \begin{mdframed}
     2. A certain brand of candle is designed to last nine hours. However, depending on the wind, air bubbles in the wax, the quality of the wax, and the number of times the candle is re-lit, the actual burning time (in hours) is uniformly distributed between 6.5 hours and 10.5 hours. Suppose that one of these candles is selected at random. Find the probability of each of the following events.
     \begin{enumerate}
         \item[(a)] The candle burns at most 8 hours.
         \item[(b)] The candle burns at least 7 hours.
         \item[(c)] The candle burns between 9 and 11 hours.
     \end{enumerate}
 \end{mdframed}
 \bigbreak \noindent 
 \begin{remark}
     A continuous rv $X$ is said to have a \textbf{uniform distribution} on the interval $[A, B]$ if the pdf of $X$ is
     \[
         f(x; A, B) = 
         \begin{cases} 
             \frac{1}{B - A} & A \leq x \leq B \\ 
             0 & \text{otherwise} 
         \end{cases}
     \]
 \end{remark}
 \bigbreak \noindent 
 The \textbf{cumulative distribution function} $F(x)$ for a continuous rv $X$ is defined for every number $x$ by
 \[
     F(x) = P(X \leq x) = \int_{-\infty}^{x} f(y) \, dy
 \]
 For each $x$, $F(x)$ is the area under the density curve to the left of $x$. 
 \bigbreak \noindent 
 Let $X$ be a continuous rv with pdf $f(x)$ and cdf $F(x)$. Then for any number $a$,
 \[
     P(X > a) = 1 - F(a)
 \]
 and for any two numbers $a$ and $b$ with $a < b$,
 \[
     P(a \leq X \leq b) = F(b) - F(a)
 \] \smiley{}

 \bigbreak \noindent 
 Thus, the pdf is given by 
 \begin{align*}
     f(x) = 
     \begin{cases}
        \frac{1}{10.5 - 6.5} = \frac{1}{4} & \text{if } 6.5 \leq x \leq 10.5     \\
        0 & \text{otherwise}
     \end{cases}
 .\end{align*}
 \bigbreak \noindent 
 And the cdf is given by
 \begin{align*}
     \int_{6.5}^{x}  \frac{1}{4}\, dy &=\frac{1}{4}y \bigg|^{x}_{6.5} =\frac{x-6.5}{4} \\
     \implies F(x) &= 
     \begin{cases}
         0 & \text{for } x < 6.5 \\
         \frac{x-6.5}{4} & \text{for }    6.5 \leq x \leq 10.5 \\
         1 & \text{for } x > 10.5
     \end{cases}
 .\end{align*}
 \bigbreak \noindent 
 a.) The probability of the candle burning at most 8 hours is given by 
 \begin{align*}
     P(X \leq 8) &= F(8)  \\
     &= \frac{8-6.5}{4} = \frac{\frac{3}{2}}{4}= 0.375
 .\end{align*}
 \bigbreak \noindent 
 b.) The probability of the candle burning at least 7 hours is given by
 \begin{align*}
     P(X \geq 7)  &= 1-F(7) \\
                  &= 1-\frac{7-6.5}{4} = 1-\frac{\frac{1}{2}}{4} = 0.875
 .\end{align*}
 \bigbreak \noindent 
 c.) The probabilty of the candle burning between 9 and 11 hours is given by
 \begin{align*}
     P(9 < X < 11) &= P(9 < x < 10.5) = F(10.5) - F(9) \\
                   &=1-F(9) = 1-\frac{9-6.5}{4} \\
                   &= 1-\frac{2.5}{4} = 0.375
 .\end{align*}
 

 \pagebreak \bigbreak \noindent 
 \begin{mdframed}
     3. Refer to problem 1
     \begin{enumerate}
         \item[(a)] Find the cumulative distribution function $F(X)$. Be sure to write your answer in the appropriate way.
         \item[(b)] Find the mean value of $X$, i.e. find $E(X)$.
         \item[(c)] Find the second moment, i.e. find $E(X^2)$.
         \item[(d)] Find the variance of $X$.
         \item[(e)] Find $\eta(0.50)$, i.e. find the median or the 50th percentile.
     \end{enumerate}
 \end{mdframed}
 \bigbreak \noindent 
 \begin{remark}
     The \textbf{expected} or \textbf{mean value} of a continuous rv $X$ with pdf $f(x)$ is
     \[
         \mu_X = E(X) = \int_{-\infty}^{\infty} x \cdot f(x) \, dx
     \]
     \bigbreak \noindent 
     If $X$ is a continuous rv with pdf $f(x)$ and $h(X)$ is any function of $X$, then
     \[
         E[h(X)] = \mu_{h(X)} = \int_{-\infty}^{\infty} h(x) \cdot f(x) \, dx
     \]
     \bigbreak \noindent 
     The \textbf{variance} of a continuous random variable $X$ with pdf $f(x)$ and mean value $\mu$ is
     \[
         \sigma_X^2 = V(X) = \int_{-\infty}^{\infty} (x - \mu)^2 \cdot f(x) \, dx = E[(X - \mu)^2]
     \]
     We also have
     \begin{align*}
         V(X) = E(X^{2}) - \left[E(X)\right]^{2}
     .\end{align*}
     \bigbreak \noindent 
     Let $p$ be a number between 0 and 1. The $(100p)$th percentile of the distribution of a continuous rv $X$, denoted by $\eta(p)$, is defined by
     \[
         p = F(\eta(p)) = \int_{-\infty}^{\eta(p)} f(y) \, dy
     \]
     \bigbreak \noindent 
     According to this expression , $\eta(p)$ is that value on the measurement axis such that $100p\%$ of the area under the graph of $f(x)$ lies to the left of $\eta(p)$ and $100(1 - p)\%$ lies to the right. Thus $\eta(.75)$, the 75th percentile, is such that the area under the graph of $f(x)$ to the left of $\eta(.75)$ is .75. Figure 4.10 illustrates the definition.
     \smiley{}
 \end{remark}
 \bigbreak \noindent 
 The cdf is given by
 \begin{align*}
     F(X) &= \int_{0}^{x} -\frac{2}{25}y + \frac{2}{5} \, dy \\
     &=-\frac{1}{25}y^{2} +\frac{2}{5}y \bigg|^{x}_{0} \\
     &= -\frac{1}{25}x^{2} + \frac{2}{5}x \\
     \implies F(X) &=
     \begin{cases}
         0 & \text{for } x < 0 \\
         -\frac{1}{25}x^{2} + \frac{2}{5}x & \text{for } 0 \leq x \leq 5 \\   
         1 & \text{for } x > 5
    \end{cases}
 .\end{align*}
 \bigbreak \noindent 
 b.) The expected value $E(X)$ is given by
 \begin{align*}
     &\int_{0}^{5} x\left(-\frac{2}{25}x + \frac{2}{5}\right) \, dx \\
     &=\int_{0}^{5} -\frac{2}{25}x^{2} + \frac{2}{5}x \, dx \\
     &=-\frac{2}{75}x^{3} + \frac{2}{10}x^{2} \bigg|^{5}_{0} \\
     &=-\frac{2}{75}(5)^{3} + \frac{2}{10}(5)^{2} - 0 \\
     &=1.6667
 .\end{align*}
 \bigbreak \noindent 
 c.) $E(X^{2})$ is given by
 \begin{align*}
     &\int_{0}^{5} x^{2}\left(-\frac{2}{25}x + \frac{2}{5}\right) \, dx \\
     &=\int_{0}^{5} -\frac{2}{25}x^{3} + \frac{2}{5}x^{2} \, dx \\
     &=-\frac{2}{25} \cdot \frac{1}{4} x^{4} + \frac{2}{5} \cdot \frac{1}{3} x^{3} \bigg|^{5}_{0} \\
     &=-\frac{1}{50}x^{4} + \frac{2}{15}x^{3} \bigg|^{5}_{0} \\
     &=-\frac{1}{50}(5)^{4} + \frac{2}{15}(5)^{3} - 0 \\
     &=4.1667
 .\end{align*}
 \bigbreak \noindent 
 d.) The variance $Var(X)$ is given by
 \begin{align*}
     V(X) &= E(X^{2}) - \left(E(X)\right)^{2} \\
     &=4.1667 - 1.6667^{2} = 1.3888
 .\end{align*}
 \bigbreak \noindent 
 e.) We have
 \begin{align*}
     p &= F(\eta(p)) = \int_{0}^{\eta(p)}  -\frac{2}{25}x + \frac{2}{5}\, dx \\
       &=-\frac{1}{25}x^{2} + \frac{2}{5}x \bigg|^{\eta(p)}_{0} \\
       &=-\frac{1}{25}\eta(p)^{2} + \frac{2}{5}\eta(p)
 .\end{align*}
 \bigbreak \noindent 
 Thus, to find $p=0.5$, ie the 50th percentile, we have
 \begin{align*}
     \frac{1}{2} &= -\frac{1}{25}\eta\left(\frac{1}{2}\right)^{2} + \frac{2}{5}\eta\left(\frac{1}{2}\right) \\
     12.5 &= -\eta\left(\frac{1}{2}\right)^{2} + 10 \eta\left(\frac{1}{2}\right) \\
          &\eta\left(\frac{1}{2}\right)^{2} - 10\eta\left(\frac{1}{2}\right) +12.5 = 0
 .\end{align*}
 \bigbreak \noindent 
 By the quadratic formula, we have
 \begin{align*}
     \eta\left(\frac{1}{2}\right) &= \frac{10 \pm \sqrt{(-10)^{2} - 4(1)(12.5)}}{2} \\
                                  &=\frac{10 \pm 7.0711}{2} \\
                                  \therefore \eta\left(\frac{1}{2}\right)&=8.5355,1.4645
 .\end{align*}
 \bigbreak \noindent 
 Since we must be in the range $[0,5]$, the valid solution is 1.4645. Thus, $\eta(0.5) = 1.4645$

 

 \pagebreak \bigbreak \noindent 
 \begin{mdframed}
     4. Refer to problem 2
     \begin{enumerate}
         \item[(a)] Find the mean value of $X$, i.e. find $E(X)$.
         \item[(b)] Find the standard deviation, $\sigma$.
         \item[(c)] Find a time $t$ so that 30\% of all candles burn longer than $t$ hours, i.e. so that $P(X > t) = 0.30$. (Note - the value $t$ could also be described as the 70th percentile.)
     \end{enumerate}
 \end{mdframed}
 \bigbreak \noindent 
 a.) The expected value $E(X)$ is given by
 \begin{align*}
     E(X) &= \int_{6.5}^{10.5}x \cdot \frac{1}{4} \, dx \\
     &=\frac{1}{8}x^{2} \bigg|^{10.5}_{6.5} \\
     &= \frac{1}{8}(10.5^{2} - 6.5^{2}) \\
     &=8.5
 .\end{align*}
 $E(X^{2})$ is given by
 \begin{align*}
     E(X^{2}) &= \int_{6.5}^{10.5}  \frac{1}{4}x^{2}\, dx \\
     &=\frac{1}{12}x^{3} \bigg|^{10.5}_{6.5} \\
     &= \frac{1}{12}(10.5^{3} - 6.5^{3}) \\
     &=73.5833
 .\end{align*}
 \bigbreak \noindent 
 b.) The standard deviation $\sigma$ is given by
 \begin{align*}
    \sigma = \sqrt{E(X^{2}) - \left(E(X)\right)^{2}}
 .\end{align*}
 \bigbreak \noindent 
 Thus, we have
 \begin{align*}
     \sigma &= \sqrt{73.5833 - 8.5^{2}}  \\
    &=1.1547
 .\end{align*}
 \bigbreak \noindent 
 Recall the cdf is given by
 \begin{align*}
     F(X) = 
     \begin{cases}
         0 & \text{for } x < 6.5 \\
         \frac{x-6.5}{4} & \text{for } 6.5 \leq x \leq 10.5 \\    
         1 & \text{for } x > 10.5
     \end{cases}
 .\end{align*}
 \bigbreak \noindent 
 Thus, to find the the 70th percentile, we solve
 \begin{align*}
     p &= \frac{\eta(p) - 6.5}{4} \\
     \implies 4p &= \eta(p) - 6.5
 .\end{align*}
 Letting $p=0.7$, 
 \begin{align*}
     4(0.7) &= \eta(0.7) - 6.5 \\
     \implies \eta(0.7) &= 9.3
 .\end{align*}
 \bigbreak \noindent 
 Thus, 30\% of all candles burn longer than 9.3 hours.


 \pagebreak \bigbreak \noindent 
 \begin{mdframed}
     5. Let $X$ be the amount of time a book on two-hour reserve is actually checked out, and suppose that it has the cumulative distribution function (cdf) given below.

     \[
         F(x) = 
         \begin{cases} 
             0 & \text{if } x < 0 \\ 
             \frac{x^2}{4} & \text{if } 0 \leq x < 2 \\ 
             1 & \text{if } 2 \leq x 
         \end{cases}
     \]

     \begin{enumerate}
         \item[(a)] Use the cdf to calculate each of the following.
             \begin{enumerate}
                 \item[(i)] $P(X \leq 1)$.
                 \item[(ii)] $P(0.5 \leq X \leq 1)$.
                 \item[(iii)] $P(X > 1.5)$.
             \end{enumerate}
         \item[(b)] Find the probability density function $f(x)$.
         \item[(c)] Find $\eta(0.50)$, i.e. find the median or the 50th percentile.
     \end{enumerate}
 \end{mdframed}
 \bigbreak \noindent 
 a.i) Using the given cdf, we have
 \begin{align*}
     F(1) = \frac{1}{4} = 0.25
 .\end{align*}
 \bigbreak \noindent 
 a.ii) We have
 \begin{align*}
     P(0.5 \leq X \leq 1)  &= F(1) - F(0.5) \\
    &=\frac{1}{4} - \frac{0.5^{2}}{4} \\
    &=0.1875
 .\end{align*}
 \bigbreak \noindent 
 a.iii) We have
 \begin{align*}
     P(X > 1.5) &= 1-F(1.5) \\
                &= 1-\frac{1.5^{2}}{4} \\
                &=0.4375
 .\end{align*}
 \bigbreak \noindent 
 b.) The pdf is given by $F^{\prime}(x) = f(x)$. Thus, 
 \begin{align*}
     F^{\prime}(x) &= \frac{d}{dx}\frac{x^{2}}{4} \\
     &=\frac{1}{2}x
 .\end{align*}
 \bigbreak \noindent
 Thus, the pdf is given by
 \begin{align*}
     f(x) =
     \begin{cases}
         \frac{1}{2}x & \text{if } 0 \leq x \leq 2 \\
         0 & \text{otherwise}
        \end{cases}
 .\end{align*}
 \bigbreak \noindent 
 c.) The 50th percentile $\eta(0.5)$ is given by
 \begin{align*}
     0.5 &= \frac{1}{4}\eta(0.5)^{2} \\
     2&=\eta(0.5)^{2} \\
     \eta(0.5) &= \sqrt{2} = 1.4142
 .\end{align*}
 \bigbreak \noindent 
 Thus, the 50th percentile is 1.4142


 \end{document} % (:
