 \documentclass{report}
 
 \input{~/dev/latex/template/preamble.tex}
 \input{~/dev/latex/template/macros.tex}
 
 \title{\Huge{}}
 \author{\huge{Nathan Warner}}
 \date{\huge{}}
 \fancyhf{}
 \rhead{}
 \fancyhead[R]{\itshape Warner} % Left header: Section name
 \fancyhead[L]{\itshape\leftmark}  % Right header: Page number
 \cfoot{\thepage}
 \renewcommand{\headrulewidth}{0pt} % Optional: Removes the header line
 %\pagestyle{fancy}
 %\fancyhf{}
 %\lhead{Warner \thepage}
 %\rhead{}
 % \lhead{\leftmark}
 %\cfoot{\thepage}
 %\setborder
 % \usepackage[default]{sourcecodepro}
 % \usepackage[T1]{fontenc}
 
 % Change the title
 \hypersetup{
     pdftitle={}
 }

 \geometry{
  left=1.5in,
  right=1.5in,
  top=1in,
  bottom=1in
}
 
 \begin{document}
     % \maketitle
     %     \begin{titlepage}
     %    \begin{center}
     %        \vspace*{1cm}
     % 
     %        \textbf{}
     % 
     %        \vspace{0.5cm}
     %         
     %             
     %        \vspace{1.5cm}
     % 
     %        \textbf{Nathan Warner}
     % 
     %        \vfill
     %             
     %             
 %        \vspace{0.8cm}
 %      
 %        \includegraPhics[width=0.4\textwidth]{~/niu/seal.png}
 %             
 %        Computer Science \\
 %        Northern Illinois University\\
 %        United States\\
 %        
 %             
 %    \end{center}
 % \end{titlepage}
 % \tableofcontents
 \pagebreak \bigbreak \noindent
 Nate Warner \ \quad \quad \quad \quad \quad \quad \quad \quad \quad \quad \quad \quad  STAT 300 \quad  \quad \quad \quad \quad \quad \quad \quad \quad \ \ \quad Summer 2024
 \begin{center}
     \textbf{PSET 9 - Due: Sunday, July 24}
 \end{center}
\begin{mdframed}
    1. Suppose that 10-ft lengths of a certain type of cable have breaking strengths that are normally distributed with mean \(\mu = 450\) lb and standard deviation \(\sigma = 50\).
    \begin{enumerate}[label=(\alph*)]
        \item Find the probability that one such cable will have a strength greater than 536 lb.
        \item Let \(\overline{X}\) = the mean breaking strength for a random sample of nine such cables. Clearly, the value of \(\overline{X}\) will vary from one sample to another. Describe its sampling distribution by giving the (i) shape, (ii) mean, and (iii) standard deviation of the distribution.
        \item Find the probability that the sample mean of nine cables will be between 423 and 480.
        \item Find the probability that the sample mean of 35 cables will be less than 428.
        \item Suppose that the distribution of breaking strengths for all cables in the population had been non-normal or unknown.
            \begin{enumerate}[label=(\roman*)]
            \item Could part (a) have been solved using the information given? Why or why not?
            \item Could part (c) have been solved using the information given? Why or why not?
            \item Could part (d) have been solved using the information given? Why or why not?
        \end{enumerate}
    \end{enumerate}
\end{mdframed}
\bigbreak \noindent 
a.)  The probability that one such cable will have a strength greater than 536 lbs is given by
\begin{align*}
    P(X > 536) &= P\left(Z > \frac{536 - 450}{50}\right) = P(Z > 1.72) \\
    &=1-P(Z < 1.72) = 1- 0.9573 = 0.0427
.\end{align*}
\bigbreak \noindent 
b.i) The shape of the sampling distrubution of $\bar{X}$ will be normal. By the central limit theorem, the shape of the sampling distrubution of $\bar{X}$ will be roughly normal given sufficiently large $n$. However, in this case since we know the population distribution is normal, the sampling distribution of $\bar{X}$ will be exactly normal regardless of the sample size $n$.
\bigbreak \noindent 
b.ii) The mean of the sampling distrubution $\mu_{\bar{X}}$ is given by
\begin{align*}
    \mu_{\bar{X}} = \mu = 450
.\end{align*}
b.iii) The standard deviation of the sampling distrubution is given by
\begin{align*}
    \sigma_{\bar{X}} = \frac{\sigma}{\sqrt{n}} = \frac{50}{9} = 16.6667
.\end{align*}
\bigbreak \noindent 
c.) 
\begin{align*}
    P(423 < \bar{X} < 480) &= P\left(\frac{423-450}{16.6667} < Z < \frac{480 - 450}{16.6667}\right) \\
    &= P(-1.62 < Z < 1.8) = \Phi(1.8) - \Phi(-1.62) \\
    &= \Phi(1.8) - (1-\Phi(1.62)) = 0.9641 - 0.0526  \\
    &= 0.9115
.\end{align*}
\bigbreak \noindent 
d.) With the sample size $n=35$, we have $\mu_{\bar{X}} = \mu = 450$, and $\sigma_{\bar{X}} = \frac{\sigma}{\sqrt{n}} = \frac{50}{\sqrt{35}} = 8.4515$. With this, 
\begin{align*}
    P(\bar{X} < 428) &= P\left(Z < \frac{428 - 450}{8.4515}\right)  \\
                     &=  P(Z < -2.6)  = 1-\Phi(2.6) = 1-0.9953 \\
                     &=0.0047
.\end{align*}
\bigbreak \noindent 
e.i) If the population distrubution was nonnormal, part a would not be able to be solved. Use the standard normal model requires the distrubution to be normal.
\bigbreak \noindent 
e.ii) Since the sample size $n=9$ is quite small, invoking the central limit theorem may not be wise. The CLT requires a sample size greater than 30 for the sample distrubution of the sample mean $\bar{X}$ to be approximately normal. When the sample size is small, the sample distribution would likely resemble the population distribution.
\bigbreak \noindent 
e.iii) In the case of part d, invoking the central limit theorem is perfectly acceptable since $n = 35 > 30$. With this information, we know the sample distribution will be approximately normal regardless of the population distribution.


\pagebreak \bigbreak \noindent 
\begin{mdframed}
    2. Suppose that the weights of people who work in an office building are normally distributed with a mean of \(\mu = 165\) lb and a standard deviation of \(\sigma = 25\) lb.
    \begin{enumerate}[label=(\alph*)]
        \item What is the probability that one person, selected at random from the building, weighs more than 200 lb?
        \item Suppose that three people are selected independently of each other. Find the probability that all three weigh more than 200 lb, i.e., find \(P(X_1 > 200 \cap X_2 > 200 \cap X_3 > 200)\).
        \item Find the probability that a sample of three people will have a mean weight greater than 200 lb.
        \item What is the conceptual difference between parts (b) and (c), i.e., how are the events different?
        \item Have you ever ridden in an elevator and read a sign stating its maximum load and wondered about the chances the elevator would be overloaded? What is the chance that the total weight of five people is more than 1000 lb, i.e., what is \(P(X_1 + \cdots + X_5 > 1000)\)? Hint – try to re-express this as an \(X\)-style problem.
    \end{enumerate}
\end{mdframed}
\bigbreak \noindent 
a.)
\begin{align*}
    P(X > 200) &= P\left(Z > \frac{200 - 165}{25} \right) \\
               &=1-\Phi(1.4) = 1- 0.9192 = 0.0808
.\end{align*}
\bigbreak \noindent 
b.) 
\begin{align*}
    &P(X_{1} > 200 \cap X_{2} > 200 \cap X_{3} > 200) \\
    &=P(X_{1} > 200)P(X_{2} > 200)P(X_{3}>200) \\
    &=(P(X > 200))^{3} = \left(1-P\left(Z> \frac{200-165}{25}\right)\right)^{3} \\
    &=\left(1-\Phi(1.4)\right)^{3} = 0.0808^{3} = 0.0005
.\end{align*}
\bigbreak \noindent 
c.) With a sample of $n=3$, we have $\mu_{\bar{X}} = 165 $, and $\sigma_{\bar{X}} = \frac{\sigma}{\sqrt{n}} = \frac{25}{\sqrt{3} } = 14.4338$. Thus,
\begin{align*}
    P(\bar{X} > 200) &= P\left(Z > \frac{200 - 165}{14.4338}\right) \\
    &=P(Z > 2.42) = 1-\Phi(2.42)  \\
    &= 0.0078
.\end{align*}
\bigbreak \noindent 
d.) In part b we are finding the probability in association with randomly selected individuals, specifically three independently selected individuals all having weight greater than 200. 
\bigbreak \noindent 
In part c we are finding the probability that the average weight of the 3 person sample is greater than 200.  In this case we are dealing with the sampling distribution of the sample mean $\bar{X}$, which has its own mean and standard deviation. 
\bigbreak \noindent 
e.) The sample total $T_{0}$ is given by $T_{0} = X_{1} + X_{2} + ... + X_{n}$. This quantity has mean $\mathbb{E}({T_{0}}) = \mu_{T_{0}} = n\mu$ and standard deviation $\sigma_{T_{0}} = \sqrt{n} \sigma$. In this case, with $n=5$, we have $\mathbb{E}(T_{0}) = 5(165) = 825$ and $\sigma_{T_{0}} = \sqrt{5}(25) = 55.9017$. Thus,
\begin{align*}
    P(T_{0} > 1000) &= P\left(Z > \frac{1000-825}{55.9017}\right) \\
    &=P(Z > 3.13) = 1-\Phi(3.13) \\
    &=0.0009
.\end{align*}


\pagebreak \bigbreak \noindent 
\begin{mdframed}
     3. The figure below shows graphs of the probability density functions for a population and for the sampling distribution of \( \overline{X} \) when \( n = 5 \) and for when \( n = 15 \). Which one is which? Match them up and explain how we know.
    \begin{center}
        \fig{1}{./figures/1.png}
    \end{center}
\end{mdframed}
\bigbreak \noindent 
\begin{enumerate}[label=(\alph*)]
    \item $n=15$: We know this is the largest sample because it is the most normal. 
    \item  $n=5$: We know this is the smaller sample because it isnt as normal as $a$, which means the sample must be smaller than curve $a$
    \item  \textbf{Population distribution}: Since this distribution is nonnormal, with the other two distribution being approximately normal, it must be the case that this is the population distribution
\end{enumerate}

\pagebreak \bigbreak \noindent 
\begin{mdframed}
    4. I have two errands to take care of on campus. Let \(X_1\) and \(X_2\) represent the times that it takes for the first and second errands, respectively. Let \(X_3\) = the total time in minutes that I spend walking to and from my office and between the errands. Suppose that \(X_1, X_2,\) and \(X_3\) are independent and normally distributed with \(\mu_1 = 15, \sigma_1 = 4, \mu_2 = 5, \sigma_2 = 1, \mu_3 = 12,\) and \(\sigma_3 = 3\).
    \begin{enumerate}[label=(\alph*)]
        \item Find the chance that the total time I am away from my office is less than 45 minutes, i.e., find \(P(X_1 + X_2 + X_3 < 45)\).
        \item Find the chance that the time I need for the first errand exceeds my walking-around time, i.e., find \(P(X_1 > X_3)\). Hint – this is the same as finding \(P(X_1 - X_3 > 0)\).
    \end{enumerate}
\end{mdframed}
\bigbreak \noindent 
a.) Since $X_{1}, X_{2}$, and $X_{3}$ are normally distributed, the linear combination will also be normally distributed with mean $\mathbb{E}(X_{1} + X_{2} + X_{3}) = \mathbb{E}(X_{1}) + \mathbb{E}(X_{2}) + \mathbb{E}(X_{3})  = 15 + 5 + 12 = 32$, and standard deviation $\sigma_{X_{1} + X_{2} + X_{3}} = \sqrt{\sigma_{1}^{2} + \sigma_{2}^{2} + \sigma_{3}^{2}} = \sqrt{4^{2} + 1^{2} + 3^{2}} = \sqrt{26} = 5.099$. Thus,
\begin{align*}
    P(T_{0} < 45) &= P\left(Z < \frac{45-32}{5.099}\right)  \\
                  &= \Phi(2.55) = 0.9946
.\end{align*}
\bigbreak \noindent 
b.) We have $\mathbb{E(X_{1} - X_{3})}  = \mathbb{E(X_{1})} - \mathbb{E}(X_{3}) = 15 -12 = 3$, and $\sigma_{X_{1} - X_{3}} = \sqrt{\sigma_{1}^{2} + \sigma_{3}^{2}}  =5$. Thus,
\begin{align*}
    P(X_{1} > X_{3}) &= P(X_{1} - X_{3} > 0) \\
    &=P(Z > \frac{0-3}{5}) = 1-\Phi(-0.6) \\
    &= 1-0.2743 = 0.7257
.\end{align*}




 \end{document} % (:
