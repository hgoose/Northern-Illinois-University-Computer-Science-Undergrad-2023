 \documentclass{report}
 
 \input{~/dev/latex/template/preamble.tex}
 \input{~/dev/latex/template/macros.tex}
 
 \title{\Huge{}}
 \author{\huge{Nathan Warner}}
 \date{\huge{}}
 \fancyhf{}
 \rhead{}
 \fancyhead[R]{\itshape Warner} % Left header: Section name
 \fancyhead[L]{\itshape\leftmark}  % Right header: Page number
 \cfoot{\thepage}
 \renewcommand{\headrulewidth}{0pt} % Optional: Removes the header line
 %\pagestyle{fancy}
 %\fancyhf{}
 %\lhead{Warner \thepage}
 %\rhead{}
 % \lhead{\leftmark}
 %\cfoot{\thepage}
 %\setborder
 % \usepackage[default]{sourcecodepro}
 % \usepackage[T1]{fontenc}
 
 % Change the title
 \hypersetup{
     pdftitle={}
 }

 \geometry{
  left=1.5in,
  right=1.5in,
  top=1in,
  bottom=1in
}
 
 \begin{document}
     % \maketitle
     %     \begin{titlepage}
     %    \begin{center}
     %        \vspace*{1cm}
     % 
     %        \textbf{}
     % 
     %        \vspace{0.5cm}
     %         
     %             
     %        \vspace{1.5cm}
     % 
     %        \textbf{Nathan Warner}
     % 
     %        \vfill
     %             
     %             
 %        \vspace{0.8cm}
 %      
 %        \includegraPhics[width=0.4\textwidth]{~/niu/seal.png}
 %             
 %        Computer Science \\
 %        Northern Illinois University\\
 %        United States\\
 %        
 %             
 %    \end{center}
 % \end{titlepage}
 % \tableofcontents
 \pagebreak \bigbreak \noindent
 Nate Warner \ \quad \quad \quad \quad \quad \quad \quad \quad \quad \quad \quad \quad  STAT 300 \quad  \quad \quad \quad \quad \quad \quad \quad \quad \ \ \quad Summer 2024
 \begin{center}
     \textbf{PSET 8 - Due: Sunday, July 21}
 \end{center}
 \bigbreak \noindent 
 \begin{mdframed}
     \begin{enumerate}[label=\arabic*.]
         \item Let \( Z \) be the standard normal random variable. Use the table of Standard Normal Curve Areas to obtain each of the following probabilities.
             \begin{enumerate}[label=(\alph*)]
                 \item \( P(Z < -1.25) \)
                 \item \( P(Z > 2.48) \)
                 \item \( P(-2.71 < Z < 0.58) \)
                 \item \( P(|Z| \leq 2.50) \)
             \end{enumerate}
     \end{enumerate}
 \end{mdframed}
 \bigbreak \noindent 
 a.) By symmetry, we have
 \begin{align*}
     P(Z < -1.25) = \Phi(-1.25) = 1-\Phi(1.25) = 1-0.8944 = 0.1056
 .\end{align*}
 \bigbreak \noindent 
 b.) 
 \begin{align*}
     P(Z > 2.48) = 1-\Phi(2.48) = 1-0.9934 = 0.0066
 .\end{align*}
 \bigbreak \noindent 
 c.) 
 \begin{align*}
     P(-2.71 < Z < 0.58) &= \Phi(0.58) - \Phi(-2.71) \\
                         &=\Phi(0.58) - (1-\Phi(2.71))  \\
                         &= \Phi(0.58) - 1 + \Phi(2.71) \\
                         &=0.719 - 1 + 0.9966 = 0.7156
 .\end{align*}
 \bigbreak \noindent 
 d.)
 \begin{align*}
     P(\abs{Z} \leq 2.5) &= P(-2.5 \leq Z \leq 2.5) \\
     &=\Phi(2.5) - \Phi(-2.5) \\
     &= \Phi(2.5) - (1-\Phi(2.5)) \\
     &=0.9938 - (1-0.9938) \\
     &=0.9876
 .\end{align*}

 \pagebreak \bigbreak \noindent 
 \begin{mdframed}
     \begin{enumerate}[label=\arabic*.]
         \setcounter{enumi}{1}
         \item In each case, find the value of the constant \( c \) that makes the probability statement correct.
             \begin{enumerate}[label=(\alph*)]
                 \item \( P(Z \leq c) = 0.80 \) (Note – the value \( c \) could also be described as the 80\textsuperscript{th} percentile.)
                 \item \( P(Z > c) = 0.025 \)
                 \item \( P(0 < Z < c) = 0.291 \)
                 \item \( P(-c < Z < c) = 0.668 \)
             \end{enumerate}
     \end{enumerate}
 \end{mdframed}
 \bigbreak \noindent 
 a.) By table A.3
 \begin{align*}
     P(Z \leq c ) &= 0.8 \\
     \implies c &= 0.84
 .\end{align*}
 \bigbreak \noindent 
 b.) 
 \begin{align*}
     P(Z > c) &= 0.025 \\
     \implies 1-P(Z < c) &= 0.025 \\
     \implies P(Z < c) &=0.975 \\
     \implies c &= 1.96
 .\end{align*}
 \bigbreak \noindent 
 c.) 
 \begin{align*}
     P( 0 < Z < c) &= 0.291 \\
     \implies P(Z < c) - P(Z < 0) &= 0.291 \\
     \implies \Phi(c) - \Phi(0) &= 0.291 \\
     \implies \Phi(c) - 0.5 &= 0.291 \\
     \implies \Phi(c) &= 0.791 \\
     \implies c &= 0.81
 .\end{align*}
 \bigbreak \noindent 
 d.) 
 \begin{align*}
     P(-c < Z < c) &= 0.668 \\
     \implies \Phi(c) - \Phi(-c) &=0.668 \\
     \implies \Phi(c) - (1-\Phi(c)) &= 0.668 \\
     \implies \Phi(c) -1 + \Phi(c) &= 0.668 \\
     \implies 2\Phi(c) &= 1.668 \\
     \implies \Phi(c) &= 0.834 \\
     \implies c &= 0.97
 .\end{align*}

 \pagebreak \bigbreak \noindent 
 \begin{mdframed}
     \begin{enumerate}[label=\arabic*.]
         \setcounter{enumi}{2}
         \item Suppose that the diameter at breast height (in inches) of trees of a certain type is a normally distributed random variable \( X \) with mean \( \mu = 8.5 \) and standard deviation \( \sigma = 2.5 \). Suppose that one tree of this type is selected at random.
             \begin{enumerate}[label=(\alph*)]
                 \item Find the probability that the diameter of the tree is less than 4.75 inches; i.e., find \( P(X < 4.75) \).
                 \item Find the probability that the diameter of the tree is greater than 10 inches; i.e., find \( P(X > 10) \).
                 \item Find the probability that the diameter of the tree is between 5 and 15 inches; i.e., find \( P(5 < X < 15) \).
                 \item Find the 25\textsuperscript{th} percentile of the tree diameters; i.e., find the value \( c \) so that \( P(X \leq c) = 0.25 \).
                 \item Find the tree diameter for the largest 10\% of trees; i.e., find the value \( c \) so that \( P(X > c) = 0.10 \).
                 \item Between what two values are the middle 90\% of tree diameters? That is, find the two values \( L \) and \( U \) so that \( P(L < X < U) = 0.90 \).
                 \item If two trees are selected independently of each other, what is the probability that both of them are greater than 10 inches?
                 \item If three trees are selected independently of each other, what is the probability that at least one of them has a diameter less than 10 inches?
             \end{enumerate}
     \end{enumerate}
 \end{mdframed}
 \bigbreak \noindent 
 \begin{remark}
     When \( X \sim N(\mu, \sigma^2) \), probabilities involving \( X \) are computed by “standardizing.” The \textit{standardized variable} is \( (X - \mu)/\sigma \). Subtracting \( \mu \) shifts the mean from \( \mu \) to zero, and then dividing by \( \sigma \) scales the variable so that the standard deviation is 1 rather than \( \sigma \).
     \bigbreak \noindent 
     If \( X \) has a normal distribution with mean \( \mu \) and standard deviation \( \sigma \), then
     \[
         Z = \frac{X - \mu}{\sigma}
     \]
     has a standard normal distribution. Thus
     \[
         P(a \leq X \leq b) = P\left( \frac{a - \mu}{\sigma} \leq Z \leq \frac{b - \mu}{\sigma} \right)
     \]
     \[
         = \Phi\left( \frac{b - \mu}{\sigma} \right) - \Phi\left( \frac{a - \mu}{\sigma} \right)
     \]
     \[
         P(X \leq a) = \Phi\left( \frac{a - \mu}{\sigma} \right) \quad P(X \geq b) = 1 - \Phi\left( \frac{b - \mu}{\sigma} \right)
     \]
     \bigbreak \noindent 
     The \((100p)^{\text{th}}\) percentile of a normal distribution with mean \(\mu\) and standard deviation \(\sigma\) is easily related to the \((100p)^{\text{th}}\) percentile of the standard normal distribution.
     \[
         (100p)^{\text{th}} \text{ percentile for normal } (\mu, \sigma) = \mu + \left[ (100p)^{\text{th}} \text{ for standard normal} \right] \cdot \sigma
     \]
     Another way of saying this is that if \(z\) is the desired percentile for the standard normal distribution, then the desired percentile for the normal \((\mu, \sigma)\) distribution is \(z\) standard deviations from \(\mu\).
     \smiley{}
 \end{remark}
 \pagebreak \bigbreak \noindent 
 We have $X \sim N(8.5, 2.5^{2}) $
 \bigbreak \noindent 
 a.) 
 \begin{align*}
     P(X < 4.75) &= P\left(Z < \frac{4.75-8.5}{2.5}\right) = P(Z < -1.5)  \\ 
     &= \Phi(-1.5) = 1-\Phi(1.5) \\
     &=1-0.9332 = 0.0668
 .\end{align*}
 \bigbreak \noindent 
 b.)
 \begin{align*}
     P(X > 10) &= P\left(Z > \frac{10-8.5}{2.5}\right) \\
     &= P(Z > 0.6) = 1-\Phi(0.6)  \\
     &= 1-0.7257 = 0.2743
 .\end{align*}
 \bigbreak \noindent 
 c.) 
 \begin{align*}
     P(5 < X < 15) &= P\left(\frac{5-8.5}{2.5} < Z < \frac{15-8.5}{2.5}\right) \\
                   &=P(-1.4 < Z < 2.6) = \Phi(2.6) - \Phi(-1.4) \\
                   &=\Phi(2.6) - (1-\Phi(1.4)) \\
                   &=0.9953 - 0.0808 = 0.9145
 .\end{align*}
 \bigbreak \noindent 
 d.)
 \begin{align*}
     P(X \leq c) &= 0.25 \\
     \implies P\left(Z \leq \frac{c-8.5}{2.5}\right) &= 0.25 \\
     \implies \frac{c-8.5}{2.5} &= -0.67 \\
     \implies c &= -0.67 \cdot 2.5 + 8.5 \\
     \implies c &= 6.825
 .\end{align*}
 \bigbreak \noindent 
 e.)
 \begin{align*}
     P(X > c) &= 0.1 \\
     \implies P\left(Z > \frac{c - 8.5}{2.5}\right) &= 0.1 \\
     \implies 1- P\left(Z < \frac{c - 8.5}{2.5}\right) &= 0.1 \\
     \implies P\left(Z < \frac{c - 8.5}{2.5}\right) &=0.9 \\
     \implies \frac{c-8.5}{2.5} &= 1.28 \\
     \implies c &=11.7
 .\end{align*}
 \bigbreak \noindent 
 f.) 
 \begin{align*}
     P(L < X < U) &= 0.9 \\
     \implies P(X < U) - P(X < L) &=0.9 \\
     \implies P\left(Z < \frac{U-8.5}{2.5}\right) - P\left(Z <\frac{L-8.5}{2.5}\right) &= 0.9
 .\end{align*}
 \bigbreak \noindent 
 We need to find the 5\% and 95\% percentile such that $P(X < U) = 0.95$ and $P(X < L) = 0.05$. Thus,
 \begin{align*}
     P\left(Z < \frac{U-8.5}{2.5}\right) &= 0.05 \\
     \implies \frac{U-8.5}{2.5} &= -1.64 \\
     \implies U=4.4
 .\end{align*}
 \begin{align*}
     P\left(Z < \frac{L-8.5}{2.5}\right) &= 0.95 \\
     \implies \frac{L-8.5}{2.5} &= 1.65 \\
                                \implies L&=12.625
 .\end{align*}
 \bigbreak \noindent 
 Thus,
 \begin{align*}
     P(4.4 < X < 12.625) = 0.9
 .\end{align*}
 \bigbreak \noindent 
 g.) First, we find
 \begin{align*}
     P(X > 10) &= P\left(Z > \frac{10-8.5}{2.5}\right) = 1- P\left(Z < \frac{10-8.5}{2.5}\right) \\
               &=1-\Phi(0.6) = 1-0.7257 = 0.2743
 .\end{align*}
 \bigbreak \noindent 
 That is, the probability of a selected tree having a diameter greater than 10 inches is 0.2743. The probability that two trees selected independently of each other having a diameter greater than 10 inches is 
 \begin{align*}
     0.2743^{2} = 0.0752
 .\end{align*}
 \bigbreak \noindent 
 h.) The probability that out of three independently selected trees at least one of them has a diameter less than 10 is the complement of the probability that all three have a diameter greater than 10. That is, 
 \begin{align*}
     1-(0.2743)^{3} &= 0.9794
 .\end{align*}

 \pagebreak \bigbreak \noindent 
 \begin{mdframed}
\noindent 4. Suppose that 80\% of all drivers in a certain region regularly wear a seat belt. Let \(X\) be the number of drivers out of a random sample of 500 drivers who regularly wear a seat belt. Find the (approximate) probability of each of the following events.
\begin{itemize}
    \item[(a)] \(P(X \leq 380)\)
    \item[(b)] \(P(390 \leq X \leq 410)\)
\end{itemize}
\end{mdframed}
\bigbreak \noindent 
\begin{remark}
    Let \( X \) be a binomial rv based on \( n \) trials with success probability \( p \). Then if the binomial probability histogram is not too skewed, \( X \) has approximately a normal distribution with \( \mu = np \) and \( \sigma = \sqrt{npq} \). In particular, for \( x \) = a possible value of \( X \),
    \[
        P(X \leq x) = B(x, n, p) \approx \left( \text{area under the normal curve to the left of } x + 0.5 \right)
    \]
    \[
        = \Phi \left( \frac{x + 0.5 - np}{\sqrt{npq}} \right)
    \]
    In practice, the approximation is adequate provided that both \( np \geq 10 \) and \( nq \geq 10 \), since there is then enough symmetry in the underlying binomial distribution.
    \smiley{}
\end{remark}
\bigbreak \noindent 
First, we check the conditions
\begin{enumerate}
    \item $np \geq 10 $
    \item $n(1-p) \geq 10 $
\end{enumerate}
\begin{align*}
    np &= 500(0.8) = 400 \geq 10 \\
    n(1-p) &= 500(0.2) = 100 \geq 10
.\end{align*}
\bigbreak \noindent 
Thus, this binomial experiment can be approximated by the normal distribution. We have $\mu = np = 400$ and $\sigma = \sqrt{np(1-p)} = \sqrt{80} = 8.9443$
\bigbreak \noindent 
a.)
\begin{align*}
    P(X \leq 380) &= B(380; 500, 0.8) \approx \Phi\left(\frac{380 + 0.5 - 400}{8.9443}\right) \\
    &=\Phi\left(-2.1802\right) = 1-\Phi(2.1802) \\
    &=1-0.9854 = 0.0146
.\end{align*}
\bigbreak \noindent 
b.) 
\begin{align*}
    P(390 \leq X \leq 410) &= \Phi\left(\frac{410 + 0.5 - 400}{8.9443}\right) - \Phi\left(\frac{390 - 0.5 -400}{8.9443}\right) \\
                           &=\Phi(1.17) - \Phi(-1.17) \\
                           &=\Phi(1.17) - (1-\Phi(1.17)) \\
                           &=0.8790 - 0.121 = 0.758
.\end{align*}


 
 \pagebreak \bigbreak \noindent 
\begin{mdframed}
\begin{enumerate}[label=\arabic*.]
    \setcounter{enumi}{4}
    \item Quality audit records are kept on the numbers of major and minor failures that occur to a certain type of circuit pack used during the burn-in period of large electronic switching devices. Let 
    \[ X = \text{the number of major failures} \]
    \[ Y = \text{the number of minor failures} \]
    Suppose that the random variables \( X \) and \( Y \) can be described, at least approximately, by the joint probability mass function given below.

    \[
    \begin{array}{c|ccccc}
    p(x, y) & y = 0 & y = 1 & y = 2 & y = 3 & y = 4 \\ \hline
    x = 0 & 0.15 & 0.10 & 0.10 & 0.10 & 0.05 \\
    x = 1 & 0.05 & 0.08 & 0.14 & 0.08 & 0.05 \\
    x = 2 & 0.01 & 0.01 & 0.02 & 0.03 & 0.03 \\
    \end{array}
    \]

    \begin{enumerate}[label=(\alph*)]
        \item Find the probability that a randomly selected circuit pack will have 1 major and 2 minor failures.
        \item Find \( P(X \leq 1 \text{ and } Y \leq 1) \).
        \item Find the probability that a randomly selected circuit pack will have fewer major failures than minor failures; i.e., find \( P(X < Y) \).
        \item Suppose that demerits are assigned to a circuit pack according to the formula \( D = 5X + Y \). Find the probability that a randomly selected circuit pack scores 7 or fewer demerits; i.e., find \( P(D \leq 7) \).
        \item 
            \begin{enumerate}[label=(\roman*)]
                \item Give the marginal probability mass function of \( X \).
                \item Find the mean value of \( X \); i.e., find \( E(X) \).
                \item Find the variance of \( X \).
            \end{enumerate}
        \item 
            \begin{enumerate}[label=(\roman*)]
                \item Give the marginal probability mass function of \( Y \).
                \item Find the mean value of \( Y \); i.e., find \( E(Y) \).
                \item Find the variance of \( Y \).
            \end{enumerate}
        \item 
            \begin{enumerate}[label=(\roman*)]
                \item Find \( E(XY) \).
                \item Find the expected number of demerits for a circuit pack; i.e., find \( E(D) \).
            \end{enumerate}
        \item Are \( X \) and \( Y \) independent random variables? Clearly answer yes or no and explain why or why not.
        \item Find \( \text{Cov}(X, Y) \).
        \item Find \( \text{Corr}(X, Y) \).
    \end{enumerate}
\end{enumerate}
\end{mdframed}
\bigbreak \noindent 
a.)
\begin{align*}
    P(X=1, Y=2) = 0.14
.\end{align*}
\bigbreak \noindent 
b.)
\begin{align*}
    P(X \leq 1 \text{ and } Y \leq 1) &= \sum_{x=0}^{1} \sum_{y=0}{1}\ p(x,y) \\
    &=p(0,0) + p(0,1) + p(1,0) +p(1,1) \\
    &=0.15 + 0.10 + 0.05 + 0.08 = 0.38
.\end{align*}
\bigbreak \noindent 
c.) 
\begin{align*}
    P(X < Y) &= \sum_{(x,y):} \sum_{x<y} \ p(x,y)\\ 
             &=p(0,1) + p(0,2) + p(0,3) + p(0,4) \\
             &+p(1,2) + p(1,3) + p(1,4) + p(2,3) + p(2,4) \\
             &=0.1 + 0.1 + 0.1 + 0.05 + 0.14 + 0.08 + 0.05 + 0.03 + 0.03 \\
             &=0.68
.\end{align*}
\bigbreak \noindent 
d.) First, we check all pairs to see if they satisfy $D$.
\begin{align*}
    D(0,0) &= 5(0) + 0 = 0 \leq 7 \\
    D(0,1) &= 5(0) + 1 = 1 \leq 7 \\
    D(0,2) &= 5(0) + 2 = 2 \leq 7 \\
    D(0,3) &= 5(0) + 3 = 3 \leq 7 \\
    D(0,4) &= 5(0) + 4 = 4 \leq 7  \\
    D(1,0) &= 5(1) + 0 = 5 \leq 7  \\
    D(1,1) &= 5(1) + 1 = 6 \leq 7 \\
    D(1,2) &= 5(1) + 2 = 7 \leq 7 \\
    D(1,3) &= 5(1) + 3 = 8 \nleq 7
.\end{align*}
\bigbreak \noindent 
These are the points we are interested in. Thus,
\begin{align*}
    P(D \leq 7) &= \sum_{(x,y):} \sum_{D(x,y) \leq 7} p(x,y) \\
    &=p(0,0) + p(0,1) + p(0,2) + p(0,3)  \\
    &+ p(0,4) + p(1,0) + p(1,1) + p(1,2) \\
    &= 0.15 + 3(0.10) + 2(0.05) + 0.08 + 0.14 \\
    &=0.77
.\end{align*}
\bigbreak \noindent 
e.i) 
\begin{align*}
    p_{X}(x) &= \sum_y p(x,y) \\
    p_{X}(0) &= \sum_{y} p(0,y) =  0.15 + 0.1 + 0.1 + 0.1 + 0.05 = 0.5 \\
    p_{X}(1) &= \sum_{y} p(1,y) = 0.05 + 0.08 + 0.14 + 0.08 + 0.05 = 0.4 \\
    p_{X}(2) &=\sum_{y} p(2,y) =  0.01 + 0.01 + 0.02 + 0.03 + 0.03 = 0.1
.\end{align*}
\bigbreak \noindent 
Thus,
\begin{align*}
    p_{x}(x) = 
    \begin{cases}
        0.5 & \text{if } x=0 \\    
        0.4 & \text{if } x=1 \\    
        0.1 & \text{if } x=2 \\    
        0 & \text{otherwise}
    \end{cases}
.\end{align*}
\bigbreak \noindent 
e.ii) We use the marginal probabilities for $x$ found above to compute the expected value $E(X)$
\begin{align*}
    E(X) &= \sum_{x} x \cdot p_{X}(x) \\
    &=0(0.5) + 1(0.4) + 2(0.1) \\
    &=0.6
.\end{align*}
\bigbreak \noindent 
e.iii) The variance is given by $E(X^{2}) - \left[E(X)\right]^{2} $. 
\begin{align*}
    E(X^{2}) &= \sum_{x} x^{2}p_{X}(x) = 0^{2}(0.5) + 1^{2}(0.4) + 2^{2}(0.1) = 0.8 \\
    \left[E(X)\right]^{2} &= 0.6^{2} = 0.36 \\
    \therefore V(X) &= 0.8 - 0.36 = 0.44
.\end{align*}
\bigbreak \noindent 
f.i) The marginal pmf of $Y$ is given by
\begin{align*}
    p_{Y}(Y) &= \sum_x p(x,y) \\
    p_{Y}(0) &= \sum_{x} p(x,0) = 0.15 + 0.05 + 0.01 = 0.21 \\
    p_{Y}(1) &= \sum_{x} p(x,1) = 0.10 + 0.08 + 0.01 = 0.19  \\
    p_{Y}(2) &= \sum_{x} p(x,2) = 0.10 + 0.14 + 0.02 = 0.26 \\
    p_{Y}(3) &= \sum_{x} p(x,3) =  0.10 + 0.08 + 0.03 = 0.21 \\
    p_{Y}(4) &= \sum_{x} p(x,4) = 0.05 + 0.05 + 0.03 = 0.13
.\end{align*}
\bigbreak \noindent 
Thus,
\begin{align*}
    p_{Y}(y) =
    \begin{cases}
        0.21 & \text{if } y=0,3     
        0.19 & \text{if } y=1
        0.26 & \text{if } y=2
        0.13 & \text{if } y=4
        0 & \text{otherwise}
    \end{cases}
.\end{align*}
\bigbreak \noindent 
f.ii) The expected value $E(Y)$ is given by
\begin{align*}
    E(Y) &= \sum_{y} y \cdot p_{Y}(y) \\
    &=0(0.21) + 1(0.19) + 2(0.26) + 3(0.21) + 4(0.13) = 1.86
.\end{align*}
\bigbreak \noindent 
f.iii) The variance is given by
\begin{align*}
    E(Y^{2}) &= \sum_{y} y^{2} \cdot p_{Y}(y) = 0^{2}(0.21) + 1^{2}(0.19) + 2^{2}(0.26) + 3^{2}(0.21) + 4^{2}(0.13) = 5.2 \\
    \left[E(Y)\right]^{2} &= 1.86^{2} = 3.4596 \\
    V(Y) &= E(Y^{2}) - \left[E(Y)\right]^{2} = 5.2 - 3.4596 = 1.7404
.\end{align*}
\bigbreak \noindent 
g.i) $E(XY)$ is given by
\begin{align*}
    E(XY) &= \sum_{(x,y)} xy \cdot p(x,y) = \sum_{x}\sum_{y}xy \cdot p(x,y)\\
    &= 0\cdot0\cdot \cdot 0.15 + 0 \cdot 1 \cdot 0.1 + 0 \cdot 2 \cdot 0.1 \\ 
    &\quad+ 0\cdot 3 \cdot 0.1 + 0 \cdot 4 \cdot 0.05 +1 \cdot 0 \cdot 0.05  \\
    &\quad+ 1 \cdot 1 \cdot 0.08 + 1\cdot 2\cdot 0.14 +1\cdot 3\cdot 0.08  \\
    &\quad+ 2 \cdot 2 \cdot 0.02 + 2\cdot 3 \cdot 0.03 +2\cdot 4\cdot 0.03 \\ 
    &\quad+ 1\cdot 4\cdot 0.05 + 2 \cdot 0 \cdot 0.01 + 2 \cdot 1 \cdot 0.01  \\
    &=1.32
.\end{align*}
\bigbreak \noindent 
g.ii) The expected value $E(D)$ is given by
\begin{align*}
    E(D) &= \sum_{x} \sum_{y} D(x,y) \cdot p(x,y) \\
    &= 0 \cdot 0.15 + 1 \cdot 0.1 + 2\cdot 0.1 + 3 \cdot 0.1 \cdot 4 + 0.05 \\
    &\quad+5 \cdot 0.05 + 6 \cdot 0.08 + 7 \cdot 0.14 + 8 \cdot 0.08 + 9 \cdot 0.05 \\
    &\quad+10 \cdot 0.01 + 11 \cdot 0.01 + 12 \cdot 0.02 + 13 \cdot 0.03 + 14 \cdot 0.03 \\
    &=4.86
.\end{align*}
\bigbreak \noindent 
h.) $X$ and $Y$ are independent iff $p_{X}(x)\cdot p_{Y}(y) = p(x,y)$ $\forall (x,y) $
\begin{align*}
    p(0,0) = 0.15 \ne 0.5(0.21) = 0.105
.\end{align*}
\bigbreak \noindent 
Since this does not hold, we conclude $X$ and $Y$ are not independent.
\bigbreak \noindent 
i.) The covariance is given by
\begin{align*}
    Cov(X,Y) = E\left[(x-\mu_{x})(y-\mu_{y})\right] = \sum_{x}\sum_{y}(x-\mu_{x})(y-\mu_{y})p(x,y)
.\end{align*}
Also, by the shortcut formula $Cov(X,Y) = E(XY) - \mu_{x}\mu_{y}$, we have
\begin{align*}
    Cov(X,Y) &= 1.32 - 0.6 \cdot 1.86 \\
    &=0.204
.\end{align*}
\bigbreak \noindent 
j.) The Correlation Coefficient $\rho_{X,Y}$ is given by $\frac{Cov(X,Y)}{\sigma_{X}\sigma_{Y}} $. Thus, we have
\begin{align*}
    \rho_{X,Y} &= \frac{Cov(X,Y)}{\sqrt{V(X)}\sqrt{V(Y)}} \\
    &=\frac{0.204}{\sqrt{0.44}\sqrt{1.7404}} \\
    &=0.2331
.\end{align*}



 \end{document} % (:
