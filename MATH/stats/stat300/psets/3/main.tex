 \documentclass{report}
 
 \input{~/dev/latex/template/preamble.tex}
 \input{~/dev/latex/template/macros.tex}
 
 \title{\Huge{}}
 \author{\huge{Nathan Warner}}
 \date{\huge{}}
 \fancyhf{}
 \rhead{}
 \fancyhead[R]{\itshape Warner} % Left header: Section name
 \fancyhead[L]{\itshape\leftmark}  % Right header: Page number
 \cfoot{\thepage}
 \renewcommand{\headrulewidth}{0pt} % Optional: Removes the header line
 %\pagestyle{fancy}
 %\fancyhf{}
 %\lhead{Warner \thepage}
 %\rhead{}
 % \lhead{\leftmark}
 %\cfoot{\thepage}
 %\setborder
 % \usepackage[default]{sourcecodepro}
 % \usepackage[T1]{fontenc}
 
 % Change the title
 \hypersetup{
     pdftitle={}
 }

 \geometry{
  left=1.5in,
  right=1.5in,
  top=1in,
  bottom=1in
}
 
 \begin{document}
     % \maketitle
     %     \begin{titlepage}
     %    \begin{center}
     %        \vspace*{1cm}
     % 
     %        \textbf{}
     % 
     %        \vspace{0.5cm}
     %         
     %             
     %        \vspace{1.5cm}
     % 
     %        \textbf{Nathan Warner}
     % 
     %        \vfill
     %             
     %             
     %        \vspace{0.8cm}
     %      
     %        \includegraphics[width=0.4\textwidth]{~/niu/seal.png}
     %             
     %        Computer Science \\
     %        Northern Illinois University\\
     %        United States\\
     %        
     %             
     %    \end{center}
     % \end{titlepage}
     % \tableofcontents
    \pagebreak \bigbreak \noindent
    Nate Warner \ \quad \quad \quad \quad \quad \quad \quad \quad \quad \quad \quad \quad  STAT 300 \quad  \quad \quad \quad \quad \quad \quad \quad \quad \ \ \quad Summer 2024
    \begin{center}
        \textbf{PSET 3 - Due: Sunday, June 30}
    \end{center}
    \bigbreak \noindent 

    \begin{mdframed}
        1. The three most popular options on a certain type of new car at a dealership are a built-in GPS (A), a sunroof (B), and an automatic transmission (C). Suppose that \(P(A) = 0.40\), \(P(B) = 0.55\), \(P(C) = 0.70\), \(P(A \cup B) = 0.63\), and \(P(B \cap C) = 0.45\). Suppose that a customer at the dealership is selected at random.
        \begin{itemize}
            \item[(a)] Find the probability that the customer wants a sunroof given that they want an automatic transmission.
            \item[(b)] As you walk around the dealership’s lot, you see a customer drive away in the car they just bought. You clearly see that their car has a sunroof. What is the probability that their car has a built-in GPS?
        \end{itemize}
    \end{mdframed}
    \bigbreak \noindent 
    \begin{remark}
        For any two events \( A \) and \( B \), the conditional probability of \( A \)
        given that \( B \) has occurred is defined by
        \[
            P(A \mid B) = \frac{P(A \cap B)}{P(B)}
        \]
    \end{remark}
    \bigbreak \noindent 
    a.)
    \begin{align*}
        P(B \mid C) &= \frac{P(B \cap C)}{P(C)} \\
        &= \frac{0.45}{0.70} = 0.6429 \\
        &\approx 64\%
    .\end{align*}
    \bigbreak \noindent 
    b.)  First we need to find $P(A\cap B) $
    \begin{align*}
        P(A\cup B) &= P(A) + P(B) - P(A\cap B) \\
        \implies P(A\cap B) &= P(A) + P(B) - P(A\cup B) \\
        \therefore P(A\cap B) &= 0.4 + 0.55 - 0.63 \\
                              &=0.32
    .\end{align*}
    \bigbreak \noindent 
    From here, we can compute $P(A\mid B) $
    \begin{align*}
        P(A \mid B) &= \frac{P(A\cap B)}{P(B)} \\
                    &=\frac{0.32}{0.55} =0.5818 \\
                    &\approx 58\%
    .\end{align*}


    \pagebreak \bigbreak \noindent 
    \begin{mdframed}
        2. A department store sells sport shirts in three sizes (small, medium, and large), three patterns (plaid, print, and stripe), and two sleeve lengths (long and short). The table below gives the proportions of shirts sold in the various category combinations.

        \begin{center}
            \begin{tabular}{|c|c|c|c|c|c|c|c|}
                \hline
                & \multicolumn{3}{|c|}{Short-sleeved} & \multicolumn{3}{|c|}{Long-sleeved} \\
                \hline
                Size & Plaid & Print & Stripe & Size & Plaid & Print & Stripe \\
                \hline
                S & 0.04 & 0.02 & 0.05 & S & 0.03 & 0.02 & 0.03 \\
                M & 0.08 & 0.07 & 0.12 & M & 0.10 & 0.05 & 0.07 \\
                L & 0.03 & 0.07 & 0.08 & L & 0.04 & 0.02 & 0.08 \\
                \hline
            \end{tabular}
        \end{center}
        \begin{itemize}
            \item[(a)] What is the probability that the next shirt sold is a medium, short-sleeved, striped shirt?
            \item[(b)] What is the probability that the next shirt sold is a large print shirt?
            \item[(c)] What is the probability that the size of the next shirt sold is small?
            \item[(d)] What is the probability that the next shirt sold is a long-sleeved plaid shirt?
            \item[(e)] Given that the shirt just sold was short-sleeved, what is the probability that its size was medium?
            \item[(f)] What is the probability that the shirt just sold was short-sleeved, given that it was a large print shirt?
        \end{itemize}
    \end{mdframed}
    \bigbreak \noindent 
    a.) According to the table of proportions, we see the probability of the next sold shirt being medium, short-sleeved, and striped is $0.12 = 12\% $
    \bigbreak \noindent 
    b.) The probability that the next sold shirt is large is the sum of the proportions of the large shirts
    \begin{align*}
        \therefore P(\text{large } \cap \text{ print}) &= 0.07 + 0.02 \\
        &=0.09 = 9\%
    .\end{align*}
    \bigbreak \noindent 
    c.) Similar to part b, we find
    \begin{align*}
        P(\text{small}) &= 0.04 + 0.02 + 0.05 + 0.03 + 0.02 + 0.03 \\
                        &=0.19 = 19\%
    .\end{align*}
    \bigbreak \noindent 
    d.) Looking at the given table, we find
    \begin{align*}
        P(\text{Long-sleeved } \cap \text{ Plaid}) &= 0.03 + 0.1 + 0.04 \\
        &=0.17 = 17\%
    .\end{align*}
    \bigbreak \noindent 
    e.) 
    \begin{align*}
        P(\text{Medium } \mid \text{ short-sleeved}) &= \frac{P(\text{Medium } \cap \text{ Short-sleeved})}{P(\text{Short-sleeved})} \\
                                                     &=\frac{0.08 + 0.07 + 0.12}{0.04 + 0.02 + 0.05 + 0.08 + 0.07 + 0.12 + 0.03 + 0.07 + 0.08} \\
                                                     &=0.4821 \approx  48\%
    .\end{align*}
    \bigbreak \noindent 
    f.) 
    \begin{align*}
        P(\text{Short-sleeved } \mid \text{ large } \cap \text{ print}) &= \frac{P(\text{short-sleeved } \cap \text{ large } \cap \text{ print})}{P(\text{large } \cap \text{ print})} \\
                                                                        &=\frac{0.07}{0.07 + 0.02} \\
                                                                        &=0.7778 \approx 78\%
    .\end{align*}

    \pagebreak \bigbreak \noindent  
    \begin{mdframed}
        3. Suppose an individual is randomly selected from the population of all adult males living in the United States. Let \(A\) be the event that the selected individual is over 6 feet in height, and let \(B\) be the event that the selected individual is a professional basketball player. Which do you think is larger, \(P(A|B)\) or \(P(B|A)\)? Why?
    \end{mdframed}
    \bigbreak \noindent 
    The probability that the individual is over 6 feet in height, given they are a professional basketball player $(P(A \mid B))$ would likely be larger than the probability of the individual being a professional basketball, given they are over 6 feet in height $(P(B \mid A))$. This is due to the fact that in the first case $P(B)$, (the smaller probability), would be in the denominator, which yields a greater number in contrast to $P(A)$, (the larger probability) being in the denominator.
    \bigbreak \noindent 
    It also makes sense logically. Most professional basketball players are very tall, whereas most tall people are not basketball players.



    \pagebreak \bigbreak \noindent 
    \begin{mdframed}
        4. At a certain gas station, 50\% of the customers purchase regular gas, 30\% purchase plus gas, and 20\% purchase premium gas. Of those customers purchasing regular gas, 40\% fill their tanks. Of those customers purchasing plus gas, 50\% fill their tanks, whereas of those purchasing premium gas, 60\% fill their tanks. Suppose that a customer at the gas station is selected at random.
        \begin{itemize}
            \item[(a)] Find the probability that they purchase regular gas and fill their tank.
            \item[(b)] Find the probability that they fill their tank.
            \item[(c)] If they fill their tank, what is the probability that they had purchased plus gas?
        \end{itemize}
    \end{mdframed}
    \bigbreak \noindent 
    \begin{remark}
                    Let \( A_1, \ldots, A_k \) be mutually exclusive and exhaustive events. Then for any other event \( B \),
            \[
                P(B) = P(B|A_1)P(A_1) + \cdots + P(B|A_k)P(A_k)
            \]
            \[
                = \sum_{i=1}^{k} P(B|A_i)P(A_i) 
            \]
            \bigbreak \noindent 
            Let \( A_1, A_2, \ldots, A_k \) be a collection of \( k \) mutually exclusive and exhaustive events with prior probabilities \( P(A_i) \) (\( i = 1, \ldots, k \)). Then for any other event \( B \) for which \( P(B) > 0 \), the posterior probability of \( A_j \) given that \( B \) has occurred is
            \[
                P(A_j|B) = \frac{P(A_j \cap B)}{P(B)} = \frac{P(B|A_j)P(A_j)}{\sum_{i=1}^{k} P(B|A_i) \cdot P(A_i)} \quad j = 1, \ldots, k
            \]
            \bigbreak \noindent 
             We can also relate $P(A\mid B)$ to $P(B\mid A)$ by
            \begin{align*}
                P(A \mid B) = \frac{P(B\mid A) \cdot P(A)}{P(B)}
            .\end{align*}

    \end{remark}
    
    \bigbreak \noindent 
    To solve 4, first we define some notation. Let $A_{i} = \{\text{ Gas type $i$ }\}$, for $i\in \{1,2,3\}$, $B = \{ \text{Fills tank}\}$, and $B^{\prime} = \{\text{Does not fill tank}\} $
    \bigbreak \noindent 
    From the given information, we have $P(A_{1}) = 0.5$, $P(A_{2}) = 0.3$, $P(A_{3}) = 0.2$, $P(B \mid A_{1}) = 0.4$, $P(B \mid A_{2}) = 0.5$, and $P(B \mid A_{3}) = 0.6 $
    \bigbreak \noindent 
    a.)
    \begin{align*}
        P(A_{1} \cap B) &= P(B \mid A_{1}) \cdot P(A_{1}) \\
                        &=0.4(0.5) = 0.2 = 20\%
    .\end{align*}
    \bigbreak \noindent 
    b.)  From the law of total probability, we have
    \begin{align*}
        P(B) &= \summation{k}{i=1}\ P(B\mid A_{i}) \cdot P(A_{i})\ \\
        \implies P(B) &= (0.4 \cdot 0.5) + (0.3 \cdot 0.5) + (0.6 \cdot 0.2) \\
        \therefore P(B) &= 0.47 = 47\%
    .\end{align*}
    \bigbreak \noindent 
    c.)
    \begin{align*}
        P(A_{2} \mid B) &= \frac{P(A_{2} \cap B)}{P(B)}\\
        &=\frac{P(B\mid A_{2}) \cdot P(A_{2})}{P(b)} \\
        &= \frac{0.5\cdot 0.3}{0.47} \\
        &=0.3191 \approx 32\%
    .\end{align*}




    \pagebreak \bigbreak \noindent 
    \begin{mdframed}
        5. In a particular college statistics course, twenty percent of the students had previously taken a statistics course during high school while the others had not. Among those students who had taken a high school course, twenty-five percent received an A in their college course; while only fifteen percent of those students with no prior exposure received an A in the college course. Suppose that one student from the college course is selected at random.
        \begin{itemize}
            \item[(a)] Find the probability that they had taken statistics during high school and receive an A in their college course.
            \item[(b)] Find the probability that they receive an A in the college course.
            \item[(c)] Suppose that you overhear the student boasting about receiving an A in their college statistics course. Find the probability that they had not taken a statistics course during high school.
        \end{itemize}
    \end{mdframed}
    \bigbreak \noindent 
    First, we define some notation. Let $A = \{ \text{Taken statistics in high school} \}$,  \\
    $A^{\prime} =\{ \text{No high school stats course} \} $, and $B = \{ \text{Recieved A in college course} \} $
    \bigbreak \noindent 
    We have $P(A) = 0.2 $, $P(A^{\prime})  = 0.8$, $P(B \mid A) = 0.25 $, and $P(B \mid A^{\prime}) = 0.15$
    \bigbreak \noindent 
    a.) From the information above, we find
    \begin{align*}
        P(A \cap B) &= P(B\mid A) \cdot P(A) \\
        &=0.25(0.2) = 0.05 = 5\%
    .\end{align*}
    \bigbreak \noindent 
    b.) Using the law of total probability, we see
    \begin{align*}
        P(B) &= \summation{k}{i=1}\ P(B\mid A_{i})P(A_{i})\ \\
        &= (0.25 \cdot 0.2) + (0.15 \cdot 0.8) \\
        &=0.17 = 17\%
    .\end{align*}
    \bigbreak \noindent 
    c.)  Now using Bayes' theorem, we find
    \begin{align*}
        P(A^{\prime} \mid B) &= \frac{P(B\mid A^{\prime}) \cdot P(A^{\prime})}{P(B)} \\
        &=\frac{0.15 \cdot 0.8}{0.17} \\
        &=0.7059 \approx 71\%
    .\end{align*}

    \pagebreak \bigbreak \noindent 
    \begin{mdframed}
        6. Suppose that the proportions of blood phenotypes in a particular population are as given in the table below. Suppose, further, that the phenotypes of two randomly selected individuals are independent of each another.

        \begin{center}
            \begin{tabular}{|c|c|c|c|c|}
                \hline
                Phenotype & A & B & AB & O \\
                \hline
                Proportion & 0.40 & 0.11 & 0.04 & 0.45 \\
                \hline
            \end{tabular}
        \end{center}

        \begin{itemize}
            \item[(a)] Find the probability that both phenotypes are A.
            \item[(b)] Find the probability that neither phenotype is AB.
            \item[(c)] Find the probability that the phenotypes of the two individuals match.
            \item[(d)] Suppose that three individuals are selected independently of each other. Find the probability that at least one of the three has Type O blood.
        \end{itemize}
    \end{mdframed}
    \bigbreak \noindent 
    \begin{remark}
       If two events $A$, and $B$ are independent, then 
       \begin{align*}
           P(A\cap B) = P(A)P(B)
       .\end{align*}
    \end{remark}
    \bigbreak \noindent 
    a.)
    \begin{align*}
        P(A \cap A) = P(A)^{2} = 0.4^{2} = 0.16 = 16\%
    .\end{align*}
    \bigbreak \noindent 
    b.) The probability that one selection is not AB is $1-P(AB) = 1-0.04 = 0.6 $. Thus, the probability that two selections are not AB is given by
    \begin{align*}
        0.96^{2} = 0.9216 \approx 92\%
    .\end{align*}
    \bigbreak \noindent 
    c.) The probability of having matching types is the union of the probabilities that any two selections are the same type. Thus, we have
    \begin{align*}
        &P( (A \cap A) \cup (B \cap B) \cup (AB \cap AB) \cup (O \cap O)) \\
        &=P(A)^{2} + P(B)^{2} + P(AB)^{2} + P(O)^{2} \\
        &=0.4^{2} + 0.11^{2} + 0.04^{2} + 0.45^{2} \\
        &=0.3762 \approx  38\%
    .\end{align*}
    \bigbreak \noindent 
    d.) The probability that the selection is not type O is $1-P(O) = 1-0.45  = 0.55$. From this, the probability that all three selections is not type O is given by $0.55^{3} =0.166375$. The probability that at least one selection is type O is the complement of the probability that no selection is type O. Thus, 
    \begin{align*}
        1-0.166375 = 0.8336 \approx 83\%
    .\end{align*}
    

    \pagebreak \bigbreak \noindent 
    \begin{mdframed}
        7. Consider the system of components pictured below. Components 1 and 2 are connected in series so that sub-system will work if and only if both 1 and 2 work (likewise for components 3 and 4). The top and bottom sub-systems are connected in parallel so that the overall system will work if either sub-system works. Suppose that the individual components work and fail independently of each other and that \(P(\text{component works}) = 0.95\). Find \(P(\text{overall system works})\).
        \fig{1}{./figures/system.png}
    \end{mdframed}
    \bigbreak \noindent 
    \begin{remark}
        
        For two non-disjoint events $A$ and $B$, $P(A\cup B) = P(A)+P(B) - P(A\cap B)$
    \end{remark}
    
    \bigbreak \noindent 
    \begin{align*}
        P(\text{overall system works}) &= P(\text{(1 and 2 work)} \cup \text{(3 and 4 work)}) \\
        &=P(\text{1 and 2 work}) + P(\text{3 and 4 work})  \\
        &- P(\text{1 and 2 work } \cap \text{ 3 and 4 work} ) \\
        &= 0.95^{2} + 0.95^{2} - (0.95^{2} \cdot 0.95^{2}) \\
        &=2(0.95^{2}) - 0.95^{4} \\
        &=0.9905 \approx 99\%
    .\end{align*}

    \pagebreak \bigbreak \noindent 
    \begin{mdframed}
        8. Suppose that twenty percent of all vehicles examined at a certain emissions inspection station fail the inspection. Assume that successive vehicles pass or fail independently of one another. Calculate the following probabilities.
        \begin{itemize}
            \item[(a)] \(P(\text{all of the next three vehicles inspected fail})\)
            \item[(b)] \(P(\text{exactly one of the next three inspected fails})\)
            \item[(c)] \(P(\text{at least one of the next three inspected fails})\)
            \item[(d)] \(P(\text{at most one of the next three inspected fails})\)
        \end{itemize}
    \end{mdframed}
    \bigbreak \noindent 
    a.)
    \begin{align*}
        P(\text{All three failing}) &= P(\text{one failing})^{3} \\
        &=0.2^{3} = 0.008 = .8\%
    .\end{align*}
    \bigbreak \noindent 
    b.) Without using the binomial formula, we will have to do this one manually by finding the union of the scenarios. We have three different scenarios ($3P1 = \frac{3 \cdot 2!}{2!} = 3$). 
    \begin{align*}
        \text{Scenario 1} &= P(\text{fail} \cap \text{pass} \cap \text{pass}) = 0.2 \cdot 0.8^{2}  = 0.128\\
        \text{Scenario 2} &= P(\text{pass} \cap \text{fail} \cap \text{pass}) =0.8 \cdot 0.2 \cdot 0.8 = 0.128\\
        \text{Scenario 3} &= P(\text{pass} \cap \text{pass} \cap \text{fail})  = 0.8 \cdot 0.8 \cdot 0.2 = 0.128
    .\end{align*}
    \bigbreak \noindent 
    Thus, 
    \begin{align*}
        P(\text{Exactly one fails}) &= P(\text{Scenario 1} \cup \text{Scenario 2} \cup \text{Scenario 3}) \\
        &= 3(0.128) = 0.384 \approx 38\%
    .\end{align*}
    \bigbreak \noindent 
    \textbf{Note:} All three scenarios are disjoint so we don't subtract the intersection.
    \bigbreak \noindent 
    c.) The probability that at least one fails is the complement of the probability that none fail
    \begin{align*}
        1-P(\text{None fail})  &= 1-0.8^{3} \\
       &=0.488
    .\end{align*}
    \bigbreak \noindent 
    d.) The probability that at most one fail is given by
    \begin{align*}
        P(\text{none fail} \cup \text{one fail}) &= P(\text{none fail}) + P(\text{one fail}) \\
        &=0.8^{3} + 0.384 \\
        &=0.896 \approx 90\%
    .\end{align*}


 \end{document} % (:
