 \documentclass{report}
 
 \input{~/dev/latex/template/preamble.tex}
 \input{~/dev/latex/template/macros.tex}
 
 \title{\Huge{}}
 \author{\huge{Nathan Warner}}
 \date{\huge{}}
 \fancyhf{}
 \rhead{}
 \fancyhead[R]{\itshape Warner} % Left header: Section name
 \fancyhead[L]{\itshape\leftmark}  % Right header: Page number
 \cfoot{\thepage}
 \renewcommand{\headrulewidth}{0pt} % Optional: Removes the header line
 %\pagestyle{fancy}
 %\fancyhf{}
 %\lhead{Warner \thepage}
 %\rhead{}
 % \lhead{\leftmark}
 %\cfoot{\thepage}
 %\setborder
 % \usepackage[default]{sourcecodepro}
 % \usepackage[T1]{fontenc}
 
 % Change the title
 \hypersetup{
     pdftitle={}
 }

 \geometry{
  left=1.5in,
  right=1.5in,
  top=1in,
  bottom=1in
}
 
 \begin{document}
     % \maketitle
     %     \begin{titlepage}
     %    \begin{center}
     %        \vspace*{1cm}
     % 
     %        \textbf{}
     % 
     %        \vspace{0.5cm}
     %         
     %             
     %        \vspace{1.5cm}
     % 
     %        \textbf{Nathan Warner}
     % 
     %        \vfill
     %             
     %             
     %        \vspace{0.8cm}
     %      
     %        \includegraphics[width=0.4\textwidth]{~/niu/seal.png}
     %             
     %        Computer Science \\
     %        Northern Illinois University\\
     %        United States\\
     %        
     %             
     %    \end{center}
     % \end{titlepage}
 % \tableofcontents
 \pagebreak \bigbreak \noindent
 Nate Warner \ \quad \quad \quad \quad \quad \quad \quad \quad \quad \quad \quad \quad  STAT 300 \quad  \quad \quad \quad \quad \quad \quad \quad \quad \ \ \quad Summer 2024
 \begin{center}
     \textbf{PSET 6 - Due: Sunday, July 14}
 \end{center}
 \bigbreak \noindent 

 \begin{mdframed}
     \noindent 1. Let \( X \) = the number of typos per page in the rough draft of a particular book. Suppose that \( X \) follows a Poisson distribution and that, on average, it has one typo every four pages so that \( \mu = 0.25 \) typos/page. Use the probability mass function to find each of the following.
     \begin{itemize}
         \item[(a)] Find the chance that a randomly selected page has no typos.
         \item[(b)] Find the chance that a randomly selected page has at most one typo.
         \item[(c)] Suppose that three pages are selected independently of each other. Find the chance that none of them have any typos.
     \end{itemize}
 \end{mdframed}
 \bigbreak \noindent 
 \begin{remark}
     A discrete random variable \( X \) is said to have a \textit{Poisson distribution} with parameter \( \mu \) (\( \mu > 0 \)) if the pmf of \( X \) is
     \[
         p(x; \mu) = \frac{e^{-\mu} \cdot \mu^x}{x!} \quad \text{for } x = 0, 1, 2, 3, \ldots
     \]
 \end{remark}
 \bigbreak \noindent 
 a.)
 \begin{align*}
     p(0;0.25) &= \frac{e^{-0.25} \cdot 0.25^{0}}{0!} \\
     &=\frac{1}{e^{0.25}} = 0.7788
 .\end{align*}
 \bigbreak \noindent 
 b.) 
 \begin{align*}
     P(X \leq 1) &= \summation{x}{y=0}\ p(x; \mu)\ \\
     &=p(0;0.25) + p(1;0.25) \\
     &= 0.7788 + \frac{e^{-0.25}\cdot 0.25^{1}}{1!} \\
     &=0.7788 + 0.1947 = 0.9735
 .\end{align*}
 \bigbreak \noindent 
 c.) 
 \begin{align*}
     P(X = 0)^{3} &= 0.7788^{3} = 0.4724
 .\end{align*}
 

 \pagebreak \bigbreak \noindent 
 \begin{mdframed}
     \noindent 2. Suppose that \( X \) = the number of small aircraft arriving per hour at a particular airport can be modeled by a Poisson distribution with \( \mu = 4.0 \). Use the table of cumulative Poisson probabilities to find each of the following.
     \begin{itemize}
         \item[(a)] Find the chance that fewer than six small aircraft will arrive.
         \item[(b)] Find the chance that more than two small aircraft will arrive.
         \item[(c)] Find the standard deviation \( \sigma \).
         \item[(d)] Find \( P(\mu - \sigma < X < \mu + \sigma) \).
     \end{itemize}
 \end{mdframed}
 \bigbreak \noindent 
 a.) 
\begin{align*}
     P(X < 6) = P(X \leq 5) = 0.785
 .\end{align*}
 \bigbreak \noindent 
 b.)
 \begin{align*}
     P(X > 2) = 1-P(X \leq 2) = 1-0.238  =0.762
 .\end{align*}
 \bigbreak \noindent 
 \begin{remark}
     If \( X \) has a Poisson distribution with parameter \( \mu \), then \( E(X) = V(X) = \mu \).
 \end{remark}
 \bigbreak \noindent 
 c.)
 \begin{align*}
     \sigma = \sqrt{V(X)} = \sqrt{\mu} = \sqrt{4} = 2
 .\end{align*}
 \bigbreak \noindent 
 d.)
 \begin{align*}
     P( \mu - \sigma < X < \mu + \sigma) = P(2 < x < 6) = P(X \leq 5) - P(X \leq 2) \\
     &=0.785 -  0.238 =0.547
 .\end{align*}
 




 \end{document} % (:
