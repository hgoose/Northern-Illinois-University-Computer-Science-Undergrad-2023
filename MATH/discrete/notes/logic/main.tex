\documentclass{report}

\input{~/dev/latex/template/preamble.tex}
\input{~/dev/latex/template/macros.tex}

\title{\Huge{}}
\author{\huge{Nathan Warner}}
\date{\huge{}}
\pagestyle{fancy}
\fancyhf{}
\lhead{Warner \thepage}
\rhead{}
% \lhead{\leftmark}
\cfoot{\thepage}
\setborder
% \usepackage[default]{sourcecodepro}
% \usepackage[T1]{fontenc}

\begin{document}
    % \maketitle
        \begin{titlepage}
       \begin{center}
           \vspace*{1cm}
    
           \textbf{Discrete Structures} \\
           Logic
    
           \vspace{0.5cm}
            
                
           \vspace{1.5cm}
    
           A Document By: \\
           \textbf{Nathan Warner}
    
           \vfill
                
                
           \vspace{0.8cm}
         
           \includegraphics[width=0.4\textwidth]{~/niu/seal.png}
                
           Computer Science \\
           Northern Illinois University\\
           August 11, 2023 \\
           United States\\
           
                
       \end{center}
    \end{titlepage}
    \tableofcontents
    \pagebreak \bigbreak \noindent
    \section{Statements}
    \bigbreak \noindent 
    \begin{mdframed}
        \textbf{Definition:}
        A statement (or proposition) is a sentence that is either true or false (but not both)
    \end{mdframed}
    \bigbreak \noindent 
    \begin{mdframed}
      \textbf{Example: for the following, state whether it is a statement, or not a statement}
      \bigbreak \noindent 
      \textbf{A.) "I think it will rain tomorrow" is a statement.}
      \bigbreak \noindent 
      \textbf{B.) 3 - x = 12 }
      \bigbreak \noindent 
      \textbf{C.) 2 + 2 =3 }
      \bigbreak \noindent 
      \textbf{Solutions:}
      \bigbreak \noindent 
      \textbf{A.) The sentence is not a statement because there is a chance it will rain, or not.}
      \bigbreak \noindent 
      \textbf{B.) Though the bellow sentence is a mathematical expression, however it is not a statement because it is not either true or false. Depending on what x is, the sentence is either true or false, but right now it is neither.}
      \bigbreak \noindent 
      \textbf{C.) Even though the bellow expression is false, but it is a statement because it is either true or false, but not both, and in this case it is false. Therefore "2 + 2 = 3" is a statement.}
    \end{mdframed}

    \bigbreak \noindent \bigbreak \noindent 
    \section{Compound Statements}
    \textbf{Components of a statement}
    \begin{itemize}
      \item \textbf{$p$}: Represents predicate
      \item \textbf{$q$}: Represents conclusion
    \end{itemize}
    \bigbreak \noindent 
    \textbf{Logical Connectives}
    \begin{itemize}
      \item \textbf{$\wedge$}: Represents \textbf{and}
      \item \textbf{$\land$}: Represents \textbf{or}
      \item \textbf{$\neg$ or $\sim$}: Represents \textbf{negation}
    \end{itemize}
    \bigbreak \noindent 
    By utilizing logical connectives, we can create compound statements
    \pagebreak \bigbreak \noindent 
    \begin{mdframed}
      \textbf{Example: For each sentence, choose the correct compound statement}
      \bigbreak \noindent 
      \textbf{A.) The sentence "It is not hot or it is sunny" in symbols is:}
      \bigbreak \noindent 
      \textbf{B.) The expression, "$3 \leq a$" in writing is:}
      \bigbreak \noindent 
      \textbf{C.) If p: a week has seven days, q: there are 20 hours in a day, and r: there are 60 minutes in an hour, then $\sim$p $\wedge$ $\sim$r is:}
      \bigbreak \noindent 
      \textbf{Solutions:}
      \bigbreak \noindent 
      \textbf{A.) $\sim p \lor q$}
      \bigbreak \noindent 
      \textbf{B.) $3 > a$  or $3 = a $}
      \bigbreak \noindent 
      \textbf{C.) A week doesn't have 7 days and there are not 60 minutes in an hour. }
    \end{mdframed}

    \bigbreak \noindent \bigbreak \noindent 
    \section{Truth Tables}
    \bigbreak \noindent 
    Here is a simple example of a truth table for logical and:
    \bigbreak \noindent 
    \begin{center}
        \begin{tabular}{|c|c|c|}
            \hline
            $P$ & $Q$ & $P \land Q$ \\
            \hline
            T & T & T \\
            T & F & F \\
            F & T & F \\
            F & F & F \\
            \hline
        \end{tabular}
    \end{center}
    \bigbreak \noindent 
    Here is a simple example of a truth table for logical or (inclusive):
    \bigbreak \noindent 
    \begin{center}
        \begin{tabular}{|c|c|c|}
            \hline
            $P$ & $Q$ & $P \lor Q$ \\
            \hline
            T & T & T \\
            T & F & T \\
            F & T & T \\
            F & F & F \\
            \hline
        \end{tabular}
    \end{center}
    \bigbreak \noindent 
    Here is a simple example of a truth table for logical or (exclusive):
    \bigbreak \noindent 
    \begin{center}
        \begin{tabular}{|c|c|c|}
            \hline
            $P$ & $Q$ & $P \oplus Q$ \\
            \hline
            T & T & F \\
            T & F & T \\
            F & T & T \\
            F & F & F \\
            \hline
        \end{tabular}
    \end{center}
    \bigbreak \noindent 
    Here is a simple example of a truth table for logical not (negation):
    \begin{center}
        \begin{tabular}{|c|c|}
            \hline
            $P$ & $\lnot P$ \\
            \hline
            T & F \\
            F & T \\
            \hline
        \end{tabular}
    \end{center}
    \pagebreak \bigbreak \noindent 
    \begin{mdframed}
      \textbf{Example: Construct a truth table for $(p\land q) \lor  \neg r$}
      \bigbreak \noindent 
      \textit{Figure:}
      \bigbreak \noindent 
      \begin{center}
          \begin{tabular}{|c|c|c|c|c|c|}
            \hline
            \(p\) & \(q\) & \(r\) & \((p \land q)\) & \(\neg r\) & \((p \land q) \lor \neg r\) \\
            \hline
            T & T & T & T & F & T \\
            \hline 
            T & T & F & T & T & T \\
            \hline
            T & F &T & F &F&F \\
            \hline
            T&F&F&F&T&T \\
            \hline
            F&T&T&F&F&F \\
            \hline 
            F&T&F&F&T&T \\
            \hline
            F&F&T&F&T&F \\
            \hline
            F&F&F&F&F&T \\
            \hline
        \end{tabular}
      \end{center}
    \end{mdframed}

    \bigbreak \noindent \bigbreak \noindent 
    \section{Logical Equivalence}
    \bigbreak \noindent 
    \begin{mdframed}
        \textbf{Definition}:
        Statements $p$ and $q$ are said to be logically equivalent if they have the same truth value in every model. \\
        \textbf{Notation:} the notation for logical equivalence is $\equiv $
        % in other words 
    \end{mdframed}
    \bigbreak \noindent 
    Say we have statements $p$ and $q$, and we want to show that $p \land q$, and $q \land p$ are logically equivalent, to do this, we must first construct a  truth table:
    \begin{center}
        \begin{tabular}{|c|c|c|c|}
        \hline
        $p$ & $q$ & $p\land q$ & $q \land p $\\
        \hline
        T&T&T&T  \\
        \hline
        T&F&F&F \\
        \hline
        F&T&F&F \\
        \hline   
        F&F&F&F \\
        \hline 
        \end{tabular}
    \end{center}
    \bigbreak \noindent 
    So we can see that the columns $p\land q$, $q\land p$ have the same truth values, therefore they are said to be \textbf{logically equivalent}

    \pagebreak \bigbreak \noindent 
    \section{Tautologies and Contradictions}
    \bigbreak \noindent 
    \begin{mdframed}
        \textbf{Definition:}
          A \textbf{Tautology} is a Statement that is always true, a assertion that is true in every possible interpretation \\
          A \textbf{Contradiction} is a statement that is always false.
    \end{mdframed}
    \bigbreak \noindent 
    Consider the following compound statement
    \begin{align*}
        p \lor \neg p
    .\end{align*}
    \bigbreak \noindent 
    Because this statement can never be false, we say it is a \textbf{Tautology}
    \bigbreak \noindent 
    \textbf{Contradiction}
    \bigbreak \noindent 
    Consider the statement:
    \begin{align*}
         p \land \neg p
    .\end{align*}
    \bigbreak \noindent 
    Because this statement can never be true, we say it is a \textbf{Contradiction}

    \bigbreak \noindent \bigbreak \noindent 
    \section{De Morgan's Laws}
    \bigbreak \noindent 
    De Morgan's Laws are:
    \begin{itemize}
        \item $\neg(p\land q) = \neg p \lor \neg q$
        \item $\neg(p\lor q) = \neg p \land \neg q$
    \end{itemize}
    \bigbreak \noindent 
    Consider the statement:
    \begin{align*}
         0 < x \leq 3
    .\end{align*}
    \bigbreak \noindent 
    To use De Morgan's Law, which states:
    \begin{align*}
        \neg(p \land q) = \neg p \lor \neg q
    .\end{align*}
    We can rewrite the statement as:
    \begin{align*}
        0 \geq x\ or\ x > 3
    .\end{align*}
    
    \pagebreak \bigbreak \noindent 
    \section{Logical Equivalence Laws}
    \bigbreak \noindent 
    \begin{center}
        \begin{array}{|l|l|l|}
            \hline \text { Commutative laws: } & \mathrm{p} \wedge \mathrm{q} \equiv \mathrm{q} \wedge \mathrm{p} & p \vee q \equiv q \vee p \\
            \hline \text { Associative laws: } & (\mathrm{p} \wedge \mathrm{q}) \wedge \mathrm{r} \equiv \mathrm{p} \wedge(\mathrm{q} \wedge \mathrm{r}) & (p \vee q) \vee r \equiv p \vee(q \vee r) \\
            \hline \text { Distributive laws: } & \mathrm{p} \wedge(\mathrm{q} \vee \mathrm{r}) \equiv(\mathrm{p} \wedge \mathrm{q}) \vee(\mathrm{p} \wedge \mathrm{r}) & p \vee(q \wedge r) \equiv(p \vee q) \wedge(p \vee r) \\
            \hline \text { Identity laws: } & \mathrm{p} \wedge \mathbf{t} \equiv \mathrm{p} & p \vee \mathbf{c} \equiv p \\
            \hline \text { Negation laws: } & \mathrm{p} \vee \sim \mathrm{p} \equiv \mathbf{t} & p \wedge \sim p \equiv \mathbf{c} \\
            \hline \text { Double negative law: } & \sim(\sim \mathrm{p}) \equiv \mathrm{p} & \\
            \hline \text { Idempotent laws: } & \mathrm{p} \wedge \mathrm{p} \equiv \mathrm{p} & p \vee p \equiv p \\
            \hline \text { Universal bound laws: } & \mathrm{p} \vee \mathrm{t} \equiv \mathbf{t} & p \wedge \mathbf{c} \equiv \mathbf{c} \\
            \hline \text { DeMorgan's laws: } & \sim(\mathrm{p} \wedge \mathrm{q}) \equiv \sim \mathrm{p} \vee \sim \mathrm{q} & \sim(p \vee q) \equiv \sim p \wedge \sim q \\
            \hline \text { Absorption laws: } & \mathrm{p} \vee(\mathrm{p} \wedge \mathrm{q}) \equiv \mathrm{p} & p \wedge(p \vee q) \equiv p \\
            \hline \text { Negation of $t$ and $c$ } & $\neg t = c $ & $\neg c = t$  \\
            \hline
        \end{array}
    \end{center}

    \bigbreak \noindent \bigbreak \noindent 
    \section{Conditional Statements}
    \bigbreak \noindent 
    \begin{mdframed}
        \textbf{Definition:}
        A \textbf{conditional statement} is a statement that can be written in the form “If P then Q,” where P and Q are sentences. \\
        \textbf{Syntax:} if \textit{statement} then \textit{statement}
    \end{mdframed}
    \bigbreak \noindent 
    Consider the statment
    \begin{align*}
        p \rightarrow q
    .\end{align*}
    This statement, read "if p then q", can be described with the following truth table:
    \begin{center}
        \begin{tabular}{|l|c|c|}
        \hline
        p & q & $p \rightarrow q $\\
        	\hline
        T &T  & T  \\
        	\hline
        T&F & F\\
        \hline
        F&T & T\\
        \hline
        F&F & T\\
        \hline
        \end{tabular}
    \end{center}
    \bigbreak \noindent 
    \nt{To get a truth value of "true" in $p \rightarrow q $, either $p $ and $q $ both need to be true, or both need to be false, or q needs to be true}

    \pagebreak \bigbreak \noindent 
    \section{Negation of Conditional Statements}
    \bigbreak \noindent 
    \begin{mdframed}
        \textbf{Definition:}
       The \textbf{Negation of conditional statement} is logically equivalent to a conjunction of the antecedent and the negation of the consequent.
       \begin{align*}
           Negation:\ p \rightarrow q \equiv p \land \neg q 
       .\end{align*}
       \bigbreak \noindent 
       The \textbf{Contrapositive} of a conditional statement is a combination of the converse and the inverse
       \begin{align*}
           p \rightarrow q \equiv \neg q \rightarrow \neg p
       .\end{align*}


    \end{mdframed}
    \bigbreak \noindent 
    Consider the following condition
    \begin{align*}
       p \rightarrow q 
    .\end{align*}
    \bigbreak \noindent 
    Which we know is logically equivalent to:
    \begin{align*}
        p \rightarrow q \equiv \neg p \lor q
    .\end{align*}
    \bigbreak \noindent 
    By use of De Morgan's Law, which states that:
    \begin{align*}
        \neg(p\lor q) \equiv \neg p \land \neg q
    .\end{align*}
    \bigbreak \noindent 
    We can negate $p \rightarrow q$, so:
    \begin{align*}
        \neg(p \lor q) \equiv \neg(\neg p) \land \neg q
    .\end{align*}
    \bigbreak \noindent 
    Consider the following Conditional Statement
    \begin{center}
        If my dad is at home then he cant pick me up \\
        p $\rightarrow$q
    \end{center}
    \bigbreak \noindent 
    Then the negation would be:
    \begin{align*}
        &p \land \neg q \\
        &\equiv \text{my dad is at home and he can pick me up}
    .\end{align*}


    \pagebreak \bigbreak \noindent 
    \section{Converse and Inverse}
    \bigbreak \noindent 
    \begin{mdframed}
        \textbf{Definition:}
       The \textbf{Converse} of a conditional statement is created when the hypothesis and conclusion are reversed \\
       The \textbf{Inverse} of a conditional statement is when both the hypothesis and conclusion are negated
    \end{mdframed}
    \bigbreak \noindent 
    Consider the statement
    \begin{align*}
        p \rightarrow q
    .\end{align*}
    \bigbreak \noindent 
    Then the \textbf{Converse} would be:
    \begin{align*}
        q \rightarrow p
    .\end{align*}
    \bigbreak \noindent 
    And the \textbf{Inverse} would be:
    \begin{align*}
        \neg p \rightarrow \neg q
    .\end{align*}

    \bigbreak \noindent \bigbreak \noindent 
    \section{Biconditional Statements}
    \bigbreak \noindent 
    \begin{mdframed}
        \textbf{Definition:}
       A \textbf{Biconditional Statement} is a true statement that combines a hypothesis and conclusion with the words 'if and only if' instead of the words 'if' and 'then'
    \end{mdframed}
    \bigbreak \noindent 
    Say we have the following  statement
    \begin{align*}
        q \iff p
    .\end{align*}
    Then the truth table would be:
    \begin{center}
        \begin{tabular}{|l|c|c|}
        \hline
        p & q  & q \iff p \\
        	\hline
        T&T&T   \\
        	\hline
        F&T&F \\
        \hline
        T&F&F \\ 
        \hline
        F&F&T \\
        \hline

        \end{tabular}
    \end{center}
    \bigbreak \noindent 
    \nt{Similar to conditional statements, in the truth table, $p \iff q$ is true if both p and q have the same value. So false false will be true}
    \bigbreak \noindent 
    \pagebreak \bigbreak \noindent 
    \section{Digital Logic Circuits}
    \begin{mdframed}
        \textbf{Definition:}
       A \textbf{Circuit} is a complete circular path that electricity flows through
    \end{mdframed}
    \bigbreak \noindent 
    First, let's consider the following circuit diagram.
    \bigbreak \noindent 
    \begin{figure}[ht]
    \begin{minipage}{0.47\textwidth}
        \centering
        \incfig{circuit1}
        \caption{Circuit Diagram}
        \label{fig:circuit1}
    \end{minipage}%
    \begin{minipage}{0.5\textwidth}
        \centering
        \begin{tabular}{|l|c||c|}
            \hline
            $p$ & $q$ & Light Bulb \\
            \hline
            Closed & Closed  &  On \\
            \hline
            Open & Closed  & Off\\
            \hline 
            Closed & Open  & Off\\
            \hline 
            Open & Open  & Off\\
            \hline
        \end{tabular}
        \caption{Truth Table}
        \label{fig:truth_table}
    \end{minipage}
\end{figure}
    \bigbreak \noindent 
    So assuming the electricity can only flow up from the battery, then we can clearly see that both p and q need to be closed to get power to the light bulb. Thus,
    we can see that this case is $p \land q$
    \bigbreak \noindent 
    Next, let's consider the circuit diagram:
    \bigbreak \noindent 
    \begin{figure}[ht]
    \begin{minipage}{0.47\textwidth}
        \centering
        \incfig{cirtuit4}
        \caption{Circuit Diagram}
        \label{fig2:cirtuit4}
    \end{minipage}
    \begin{minipage}{0.47\textwidth}
        \begin{center}
        \begin{tabular}{|l|c|c|}
        \hline
        $p$ & $q$  & Light Bulb \\
        	\hline
        Closed  & Closed & On  \\
        	\hline
        	Open & Closed & On\\ 
        	\hline 
        	Closed & Open & On\\
        	\hline 
        	Open & Open  & Off\\
        	\hline
        \end{tabular}
    \end{center}
    \caption{Truth Table}
    \label{fig2:truth_table}
    \end{minipage}
    \end{figure}

    \bigbreak \noindent 
    Again, asumming the electricity can only flow up from the battery, we can clearly see that either gate can be closed while the other is open to result in the light bulb getting power. 

    \pagebreak \bigbreak \noindent 
    \section{Black Boxes and Gates}
    \bigbreak \noindent 
    \begin{mdframed}
        \textbf{Definition:}
       A \textbf{Black Box} is a circuit that is observed in terms of its input, output or transfer characteristics without any knowledge of its internal workings.
    \end{mdframed}
    \begin{figure}[ht]
        \begin{minipage}{0.47\textwidth}
        \centering
        \incfig{blackbox}
        \label{fig:blackbox}
        \caption{}
        \label{fig:}
        \end{minipage}
        \begin{minipage}{0.5\textwidth}
        With this black box, we don't care about the inner workings, we are just concerned with the inputs and outputs
        % \caption{}
        % \label{fig:}
        \end{minipage}
    \end{figure}
    \bigbreak \noindent 
    Let's consider an example that's a bit more pragmatic
    \bigbreak \noindent 

\begin{figure}[ht]
    \begin{minipage}{0.47\textwidth}
        \centering
    \incfig{tryagain} 
    \label{fig:tryagain}
    \end{minipage}
    \begin{minipage}{0.5\textwidth}
    In this case, the printer is the black box, we aren't required to know the inner workings of the printer, we are just concerned with the inputs and outputs. If we send some input to the printer  through the computer, we know that we will get some output via the page.
    \label{fig:}
    \end{minipage}
\end{figure}
    \bigbreak \noindent 
    \begin{mdframed}
        \textbf{Definition:}
        A \textbf{Gate} is circuit with one or more inputs and exactly one output
    \end{mdframed}
    \textbf{Types of gates:}
    \begin{itemize}
      \item \textbf{Not (negation)}: One input and one output, negates the input
      \item \textbf{And}: Two inputs and one output, requires both inputs to be true
      \item \textbf{Or}: Two inputs and one output, requires only one input to be true
    \end{itemize}

    \pagebreak \bigbreak \noindent 
    \textit{Representation of gates:}
    \bigbreak \noindent 
\begin{figure}[ht]
    \centering
    \incfig{icry2}
    \label{fig:icry2}
\end{figure}

    \bigbreak \noindent \bigbreak \noindent 
    \section{Boolean Expressions}
    \bigbreak \noindent 
    \begin{mdframed}
        \textbf{Definition:}
        A \textbf{Boolean Variable} is a variable with only two possible values, True, or False. \\
       A \textbf{Boolean expression} is an expression that is composed of Boolean variables, connected with logic connectives
    \end{mdframed}
    \bigbreak \noindent 
\begin{figure}[ht]
    \begin{minipage}{0.5\textwidth}
        \centering
        \incfig{booleanfig}
        \caption{Boolean Expression}
        \label{fig:booleanfig}
    \end{minipage}
    \begin{minipage}{0.47\textwidth}
        \begin{align*}
            (p \lor q) \land (\neg r)
        .\end{align*}
    \caption{Boolean Expression}
    \label{fig:booleanfig}
    \end{minipage}
\end{figure}

    \pagebreak \bigbreak \noindent 

\begin{figure}[ht]

    \begin{minipage}{0.47\textwidth}
        \centering
        \incfig{booleanfig23}
        \caption{Boolean Expression}
        \label{fig:booleanfig23}
    \end{minipage}
    \begin{minipage}{0.47\textwidth}
    \begin{align*}
        (p\lor q) \land (p \land q)
    .\end{align*}
    \caption{Boolean Expression}
    \label{fig:booleanfig23}
    \end{minipage}
\end{figure}
    \bigbreak \noindent 
    Now suppose we are given the following expression, how might we construct the circuit?
    \begin{align*}
        (\neg p  \land q)  \lor q
    .\end{align*}
    \bigbreak \noindent 
\begin{figure}[ht]
    \centering
    \incfig{booleanmagic}
    \label{fig:booleanmagic}
\end{figure}
    
    \pagebreak \bigbreak \noindent 
    \section{Truth Tables and Circuits}
    \bigbreak \noindent 
    \begin{minipage}{0.47\textwidth}
    In this section we will discuss how we can turn a truth table into a circuit. Let's consider the following truth table
    \end{minipage}
    \begin{minipage}{0.47\textwidth}
            \bigbreak \noindent 
    \begin{center}
        \begin{tabular}{|l|c|c|c|}
            \hline
            p&q&r&Output\\
        	    \hline
            F&F&F&F   \\
            \hline
            F & F &T&F\\
            \hline
            F &T&F&F \\
            \hline
            F&T&T&F \\
            \hline
            T &F&F&T \\
            \hline 
            T&F&T&T \\
            \hline
            T&T&F&F \\
            \hline
            T& T&T&T\\
            \hline
        \end{tabular}
        \bigbreak \noindent 
    \caption{Truth Table}
    \label{fig:truth_table}
    \end{center}
    \end{minipage}
    \bigbreak \noindent 
    To start, we will examine the rows that have an output of 1 (True).
    Then, we want to create a expression which will lead to the given output. If we start by examining the first row in which we have an output of 1, we can see that a possible expression can be:
    \begin{align*}
         p \land \neg q  \neg r
    .\end{align*}
    \bigbreak \noindent 
    Next, if we examine the following row that has an output of 1, then we can have the following expression:
    \begin{align*}
        p \land \neg q \land r
    .\end{align*}
    \bigbreak \noindent 
    Then for the last row which has an output of 1, we can create an expression of:
    \begin{align*}
        p \land q \land r
    .\end{align*}
    \bigbreak \noindent 
    So we have 3 expressions in total:
    \begin{align*}
         p \land \neg q  \neg r \\ 
         p \land \neg q \land r \\
        p \land q \land r
    .\end{align*}
    \bigbreak \noindent 
    The next step is to connect them all with \textit{or}. So we have:
    \begin{align*}
        ( p \land \neg q  \neg r ) \lor (p \land \neg q \land r ) \lor (p \land q \land r)
    .\end{align*}
    \bigbreak \noindent 
    This is the expression for the truth table above.
    \bigbreak \noindent 
    \begin{figure}[ht]
        \centering
        \incfig{diaphram}
        \label{fig:diaphram}
    \end{figure}

    \pagebreak \bigbreak \noindent 
    \section{Equivalent Circuits}
    \bigbreak \noindent 
    \begin{mdframed}
        \textbf{Definition:}
      Two circuits are \textbf{equivalent} if they have they same input/output table

    \end{mdframed}
    \bigbreak \noindent \bigbreak \noindent 
    \section{NAND and NOR Gates}
    \bigbreak \noindent 
    \begin{mdframed}
        \textbf{Definition:}
               A \textbf{NAND Gate} is a logic gate which produces an output which is false only if all its inputs are true. NAND is denoted by:
               \begin{align*}
                   \begin{Large}\nand \end{Large}
               .\end{align*}
       A \textbf{NOR Gate} a digital logic gate that gives an output of 0 when any of its inputs are 1, otherwise 1. NOR is denoted by:
          \begin{align*}
               \begin{Large}\nor \end{Large}
           .\end{align*}

    \end{mdframed}
    \begin{figure}[ht]
        \centering
        \incfig{tnaoheuntaoheunh}
        \label{fig:tnaoheuntaoheunh}
    \end{figure}
    
    \bigbreak \noindent \bigbreak \noindent 
    \section{Quantified Statements ALL}
    \bigbreak \noindent 
    Say we have the set:
    \begin{align*}
        A = \{1,2,3,4,5\}
    .\end{align*}
    \bigbreak \noindent 
    If we want to denote that \textbf{all} elements in this set are greater than or equal to 1, we must use quantifiers, in this case, \textbf{all}. The mathematical representation is $\forall$. So we say:
    \begin{align*}
        \forall x \in A,\ x \geq 1
    .\end{align*}

   \pagebreak \bigbreak \noindent 
   \section{Quantified Statement THERE EXISTS}
   \bigbreak \noindent 
    Consider the statement
    \begin{align*}
        \exists m \in \mathbb{Z}i\ |\ m^{2} = m
    .\end{align*}
    This statement is implying that "there exists" some integer $m$, in the set of all integers such that $m^{2} = m$

    \bigbreak \noindent \bigbreak \noindent 
    \section{Negation of Quantified Statements}
    \bigbreak \noindent 
    Say we have the statement:
    \begin{align*}
        \forall x \in A,\ Q(x)
    .\end{align*}
    \bigbreak \noindent 
    And we can represent this in plain english:
    \begin{center}
        all men wear hats
    \end{center}
    \bigbreak \noindent 
    To negate this Universal quantifier, we want to show that at least one man does not wear a hat. Thus:
    \begin{align*}
        \exists x \in A,\ \neg Q(x)
    .\end{align*}
    \bigbreak \noindent 
    Now suppose we let the statement be:
    \begin{align*}
        \exists x \in A, Q(x)
    .\end{align*}
    \bigbreak \noindent 
    Then we can \textbf{negate} it using an universal statement, so:
    \begin{align*}
        \forall x \in A, \neg Q(x)
    .\end{align*}








    






    
\end{document}
