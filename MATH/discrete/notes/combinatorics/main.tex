\documentclass{report}

\input{~/dev/latex/template/preamble.tex}
\input{~/dev/latex/template/macros.tex}

\title{\Huge{}}
\author{\huge{Nathan Warner}}
\date{\huge{}}
\pagestyle{fancy}
\fancyhf{}
\lhead{Warner \thepage}
\rhead{}
% \lhead{\leftmark}
\cfoot{\thepage}
%\setborder
% \usepackage[default]{sourcecodepro}
% \usepackage[T1]{fontenc}

\begin{document}
    % \maketitle
        \begin{titlepage}
       \begin{center}
           \vspace*{1cm}
    
           \textbf{Discrete Structures} \\
           Combinatorics
    
           \vspace{0.5cm}
            
                
           \vspace{1.5cm}
    
           \textbf{Nathan Warner}
    
           \vfill
                
                
           \vspace{0.8cm}
         
           \includegraphics[width=0.4\textwidth]{}
                
           Computer Science \\
           Northern Illinois University\\
           September 21, 2023 \\
           United States\\
           
                
       \end{center}
    \end{titlepage}
    \tableofcontents
    \pagebreak \bigbreak \noindent
    \section{\LARGE Factorials}
    \bigbreak \noindent 
    \smallbreak \noindent
    \begin{definition}
        The \textbf{Factorial} for a positive integer $n$, is the product of all the positive integers that are less than or equal to $n$ 
        \begin{align*}
            n! = n \cdot (n-1) \cdots 3 \cdot 2 \cdot 1
        .\end{align*}
    \end{definition}

    \pagebreak \bigbreak \noindent 
    \section{\LARGE The fundamental counting principal (Basic counting principal)}
    \bigbreak \noindent 
    \smallbreak \noindent
    \begin{definition}
        The \textbf{fundamental counting principal} says that if there are $a$ ways of doing event 1, $b$ ways of doing event 2, and $c$ ways of doing event 3, then the total number of outcomes are (when no do events are dependent on each other)...
        \begin{align*}
            a \cdot b \cdot c
        .\end{align*}
        \bigbreak \noindent 
        \textbf{Note:} If there is dependency between the events, we must use addition
    \end{definition}

    \pagebreak \bigbreak \noindent 
    \section{\LARGE Permutations}
    \bigbreak \noindent 
    \smallbreak \noindent
    \begin{definition}
        A \textbf{permutation} of a set is an arrangement of its members into a sequence or linear order. Order matters.
    \end{definition}
    \bigbreak \noindent 
    When dealing wit permutations, there are two cases that become important. They are:
    \begin{itemize}
        \item With repetition
        \item Without repetition
    \end{itemize}
    \bigbreak \noindent 
    \begin{thrm}[Permutation with repetition]
       \begin{align*}
           n^{r}
       .\end{align*} 
         Where $n$ is the number of choices and $r$ is the repetition.
    \end{thrm}
    \bigbreak \noindent 
    \textbf{Example.} Suppose we want to know how many possible permutations can be made for a 4 digit lock using the digits 0-9. We would have:
    \begin{align*}
        10 \times 10 \times 10 \times 10 \\
        = 10,000\ \text{Possible permutations}
    .\end{align*}
    \bigbreak \noindent 
    Using the theorem:
    \begin{align*}
        10^{4} \\
        = 10,000\ \text{Possible permutations}
    .\end{align*}

    \bigbreak \noindent \bigbreak \noindent 
    \begin{thrm}[Permutations without repetition]
       \begin{align*}
           n!
       .\end{align*} 
       Where $n$ is the number of choices
    \end{thrm}
    \bigbreak \noindent 
    \textbf{Example.} Suppose we have 5 balls in a bag, and every time we pick a ball from the bag we do not replace it. Then we can compute the number of permutations by:
    \begin{align*}
        5 \times 4 \times 3 \times 2 \times 1 \\
        = 120
    .\end{align*}
    Using the theorem:
    \begin{align*}
        5! \\
        = 120
    .\end{align*}

    \bigbreak \noindent 
    But what if $r$, the number of repetitions, is less than the choices? In this case we can use the following theorem:
    \bigbreak \noindent 
    \begin{thrm}
       \begin{align*}
           \frac{n!}{(n-r)!}
       .\end{align*} 
        Denoted by:
         \begin{align*}
           P(n,k)
         .\end{align*}
    \end{thrm}

    \pagebreak \bigbreak \noindent 
    \section{\LARGE Combinations}
    \bigbreak \noindent 
    \smallbreak \noindent
    \begin{definition}
        A \textbf{combination} of a set is an arrangement of its members into  a sequence or linear order. Order does not matter
    \end{definition}
    \bigbreak \noindent 
    Like permutations, we have two possibilities:
    \bigbreak \noindent 
    \begin{itemize}
        \item When repetition is allowed
        \item When repetition is not allowed
    \end{itemize}
    \bigbreak \noindent 
    \begin{thrm}[Combinations when repetition is not allowed]
       \begin{align*}
           \frac{n!}{k!(n-k)!}
       .\end{align*} 
        Denoted by:
        \begin{align*}
            C(n,k) \quad \text{or}\quad \binom{n}{k}
        .\end{align*}
    \end{thrm}
    \bigbreak \noindent 
    \begin{thrm}[Combinations when repetition is allowed]
       \begin{align*}
           \frac{(k+n-1)!}{k!(n-1)!}
       .\end{align*} 
    \end{thrm}

    \pagebreak \bigbreak \noindent 
    \section{\LARGE Pigeonhole principle}
    \bigbreak \noindent 
    \smallbreak \noindent
    \begin{definition}
        If $n$ items are put into $m$ containers with $ n>m $, then at least one container must contain more than one item 
    \textbf{} 
    \end{definition}

    \pagebreak \bigbreak \noindent 
    \section{\LARGE Pascals Triangle}
    \bigbreak \noindent 
    \begin{center}
    \begin{tabular}{>{$n=}l<{$\hspace{12pt}}*{13}{c}}
        0 &&&&&&&1&&&&&&\\
        1 &&&&&&1&&1&&&&&\\
        2 &&&&&1&&2&&1&&&&\\
        3 &&&&1&&3&&3&&1&&&\\
        4 &&&1&&4&&6&&4&&1&&\\
        5 &&1&&5&&10&&10&&5&&1&\\
        6 &1&&6&&15&&20&&15&&6&&1
    \end{tabular}
    \end{center}

    
    
    
    

    


    
\end{document}
